%%!TEX root = the-calculus.tex

\chapter{Calculus}\label{chap:derivative}

This lecture defines several notions related to the first-order derivatives in Alexandrov spaces, including space of directions, tangent space, differential, and gradient.

\section{Space of directions} 
\label{sec:space+directions}

Let $\spc{A}$ be an Alexandrov space.
By \ref{ex:noncreasing}, the angle measure of any hinge in $\spc{A}$ is defined.
Given $p\in \spc{A}$, consider the set $\mathfrak{S}_p$ of all nontrivial geodesics starting at $p$.
By \ref{claim:angle-3angle-inq}, the triangle inequality holds for $\mangle$ on $\mathfrak{S}_p$,
that is, $(\mathfrak{S}_p,\mangle)$ 
forms a \index{semimetric}\emph{semimetric} space;
that is, $\mangle$ behaves like a metric, but might vanish for distinct geodesics.

The metric space corresponding to  $(\mathfrak{S}_p,\mangle)$ is called the \index{70@$\Sigma_p'$ (geodesic directions)}\index{space of geodesic directions}\emph{space of geodesic directions} at $p$, denoted by $\Sigma'_p$ or $\Sigma'_p\spc{A}$.
The elements of $\Sigma'_p$ are called \index{geodesic!direction}\emph{geodesic directions} at $p$.
Each geodesic direction is formed by an equivalence class of geodesics starting from $p$ 
for the equivalence relation 
\[[px]\sim[py]\quad \iff\quad \mangle\hinge pxy=0;\]
the direction of $[px]$ is denoted by $\dir px $.\index{40@$\dir{p}{q}$ (direction)}
By \ref{ex:0-angle}, 
\[[px]\sim[py]
\quad\iff\quad
[px]\subset [py]
\quad\text{or}\quad
[px]\supset[py].
\]
 
The completion of $\Sigma'_p$ is called the \index{space of directions}\emph{space of directions} at $p$ and is denoted by \index{70@$\Sigma_p$ (space of directions)}$\Sigma_p$ or $\Sigma_p\spc{A}$.
The elements of $\Sigma_p$ are called \index{direction}\emph{directions} at $p$.

\begin{thm}{Exercise}\label{ex:dir-compact}
Let $\spc{A}$ be an Alexandrov space.
Assume that a sequence of geodesics $[px_n]$ converge to a geodesic $[px_\infty]$ in the sense of Hausdorff,
and $x_\infty\ne p$.
Suppose $\Sigma_p$ is compact.
Show that $\dir p{x_n}\z\to\dir p{x_\infty}$ as $n\to\infty$.

\end{thm}


\section{Tangent space}\label{sec: tangent space}

The \index{65@$\Cone$}\index{cone}\emph{Euclidean cone} $\spc{V}=\Cone\spc{X}$
over a metric space $\spc{X}$
is defined as the metric space whose underlying set consists of
equivalence classes in
$[0,\infty)\times \spc{X}$ with the equivalence relation ``$\sim$'' given by $(0,p)\sim (0,q)$ for any points $p,q\in\spc{X}$,
and whose metric is given by the cosine rule
\[
\dist{(s,p)}{(t,q)}{\spc{V}} 
=
\sqrt{s^2+t^2-2\cdot s\cdot t\cdot \cos\theta},
\]
where $\theta= \min\{\pi, \dist{p}{q}{\spc{X}}\}$.

Note that
\[\Cone\SSS^n\iso\EE^{n+1};\]
here ``$\iso$'' stands for ``isometric to''. 

Now let us extend several notions from Euclidean space to Euclidean cones. 

The point in $\spc{V}$ that corresponds $(t,x)\z\in[0,\infty)\times \spc{X}$ will be denoted by $t\cdot x$.
The point in $\spc{V}$ formed by the equivalence class of $\{\0\}\times\spc{X}$ is called the \index{origin}\emph{origin} of the cone and is denoted by $\0$ or $\0_{\spc{V}}$.
For $v\in\spc{V}$ the distance $\dist{\0}{v}{\spc{V}}$ is called the \index{norm}\emph{norm} of $v$ and is denoted by $|v|$ or $|v|_{\spc{V}}$.
The \index{scalar product}\emph{scalar product} $\<v,w\>$
of $v=s\cdot p$ and $w=t\cdot q$
is defined by 
\[\<v,w\>
\df |v|\cdot|w|\cdot\cos\theta
\]
where $\theta= \min\{\pi, \dist{p}{q}{\spc{X}}\}$.
The value $\theta$ is undefined if $v=\0$ or $w=\0$;
in these cases we set $\<v,w\>\df0$.

\begin{thm}{Exercise}\label{ex:geodesic-cone}
Show that $\Cone\spc{X}$ is geodesic if and only if $\spc{X}$ is \index{91@$\ell$-geodesic space}\emph{$\pi$-geodesic};
that is, any two points $x,y\in \spc{X}$ such that $\dist{x}{y}{\spc{X}}<\pi$ can be joined by a geodesic in $\spc{X}$.
\end{thm}

\parbf{Tangent space.}
The Euclidean cone $\Cone\Sigma_p$ over the space of directions $\Sigma_p$ is called the \index{tangent space}\emph{tangent space} at $p$ and is denoted by \index{70@$\T_p$ (tangent space)}$\T_p$ or $\T_p\spc{A}$.
The elements of $\T_p\spc{A}$ will be called \index{tangent vector}\emph{tangent vectors} at $p$
(despite that $\T_p$ is not a vector space).
The space of directions $\Sigma_p$ can be (and will be) identified with the unit sphere in~$\T_p$;
that is, with the set $\set{v\in\T_p}{|v|=1}$.

\begin{thm}{Proposition}\label{prop:Tan-is-CBB(0)}
Any tangent space to an Alexandrov space has nonnegative curvature in the sense of Alexandrov.
\end{thm}

Halbeisen's example \cite[13.6]{alexander-kapovitch-petrunin2024} shows that the tangent space $\T_p$ at some point of Alexandrov space might fail to be geodesic;
in this case $\T_p$ is \textit{not} $\Alex0$.

\parit{Proof.}
Consider the tangent space $\T_p=\Cone \Sigma_p$ of an Alexandrov space $\spc{A}$ at a point $p$.
We need to show that the $\EE^2$-comparison holds for a given quadruple $v_0$, $v_1$, $v_2$, $v_3\in \T_p$.

Recall that the space of geodesic directions $\Sigma_p'$ is dense in $\Sigma_p$.
It follows that the subcone $\T'_p=\Cone\Sigma_p'$ is dense in $\T_p$.
Therefore, it is sufficient to consider the case $v_0$, $v_1$, $v_2$, $v_3\in \T'_p$.

For each $i$, choose a geodesic $\gamma_i$ from $p$ in the direction of $v_i$;
reparametrize each $\gamma_i$ so that it has speed $|v_i|$.
Since the angles are defined, we have
\[\dist{\gamma_i(\eps)}{\gamma_j(\eps)}{\spc{A}}=\eps\cdot\dist{v_i}{v_j}{\T_p}+o(\eps)
\eqlbl{eq:gamma-v}\]
for $\eps>0$.
The quadruple $\gamma_0(\eps)$, $\gamma_1(\eps)$, $\gamma_2(\eps)$, $\gamma_3(\eps)$ meets the $\MM^2(\kappa)$-comparison.
After rescaling all the distances by $\tfrac1\eps$, it becomes the $\MM^2(\eps^2\cdot\kappa)$-comparison.
Passing to the limit as $\eps\to 0$ and applying \ref{eq:gamma-v}, we get the $\EE^2$-comparison for $v_0$, $v_1$, $v_2$, $v_3$.
\qeds


\begin{thm}[!]{Exercise}\label{ex:GHto-tangent}
Let $p$ be a point in an Alexandrov space $\spc{A}$,
and let $\lambda_n\to\infty$.
Suppose $\Sigma_p$ is compact.
Show that there is a pointed Gromov--Hausdorff convergence $(\lambda_n\cdot \spc{A},p)\z\to (\T_p,0)$.
Moreover, the convergence can be shoosen so that for any geodesic $\gamma$ that starts at $p$, we have
\[\iota_n\circ\gamma(t/\lambda_n)\to t\cdot \gamma^+(0),\]
where $\iota_n$ sends a point in $\spc{A}$ to the corresponding point in $\lambda_n\cdot\spc{A}$.
\end{thm}

\section{Differential}\label{sec:differential}
\index{differential of a function}

Let $f$ be a semiconcave function on an Alexandrov space $\spc{A}$, and $p\z\in \Dom f$.
Choose a unit-speed geodesic $\gamma$ that starts at $p$;
let $\xi\in\Sigma_p$ be its direction.
Define 
\[(\dd_pf)(\xi)\df(f\circ\gamma)^+(0),\]
here $(f\circ\gamma)^+$ denotes the \index{right derivative}\emph{right derivative} of $(f\circ\gamma)$;
it is defined since $f$ is semiconcave.

By the following exercise, $\dd_pf$ is a Lipschitz function on $\Sigma'_p$.
It follows that the function $\dd_pf\:\Sigma_p'\to\RR$ can be uniquely extended to a Lipschitz function $\dd_pf\:\Sigma_p\to\RR$.
Further, we can extend it to the tangent space by setting 
\[(\dd_pf)(r\cdot \xi)
\df
r\cdot (\dd_pf)(\xi)\]
for any $r\ge 0$ and $\xi\in\Sigma_p$.
The obtained function $\dd_pf\:\T_p\to\RR$ is Lipschitz;
it is called the \index{differential}\emph{differential} of $f$ at $p$.

\begin{thm}{Exercise}\label{ex:df(xi)}
Let $f$ be a semiconcave function on an Alexandrov space.
Suppose $\gamma_1$ and $\gamma_2$ are geodesics that start at $p\z\in \Dom f$;
denote by $\theta$ the angle between $\gamma_1$ and $\gamma_2$ at $p$.
Show that 
\[|(f\circ\gamma_1)^+(0)-(f\circ\gamma_2)^+(0)|\le L\cdot \theta,\]
where $L$ is the Lipschitz constant of $f$ in a neighborhood of $p$.
\end{thm}

\begin{thm}[!]{Exercise (First variation formula)} \label{ex:d(distfun)}
Let $p$ and $q$ be distinct points in an Alexandrov space~$\spc{A}$.

\begin{subthm}{ex:d(distfun):<}
Show that
$\dd_p\distfun_q(v)\le -\langle\dir pq,v\rangle$
for any $v\in\T_p$.
\end{subthm}

\begin{subthm}{ex:d(distfun):=}
Suppose $\spc{A}$ is proper.
Let $\Uparrow_p^q$ be the set of all direction of  geodesics from $p$ to $q$.
Show that
\[\dd_p\distfun_q(v)=-\max_{\xi\in\Uparrow_p^q}\langle\xi,v\rangle\]
for any $v\in\T_p$.
\end{subthm}

\end{thm}

\section{Gradient}\label{sec:grad-def}

The following definition generalizes the gradient to semiconcave functions on Alexandrov space.
This generalization is not trivial even for concave functions on Euclidean space;
we suggest keeping this example in mind while reading further.

\begin{thm}{Definition}\label{def:grad} 
Let $f$ be a semiconcave function on an Alexandrov space.
A tangent vector $g\in \T_p$ is called a 
\index{gradient}\emph{gradient} of $f$ at $p$ 
(briefly,  $g\z=\nabla_p f$\index{19@$\nabla$ (gradient)}) if
\begin{subthm}{}
$(\dd_p f)(w)\le \<g,w\>$ for any $w\in \T_p$, and
\end{subthm}

\begin{subthm}{}
$(\dd_p f)(g) = \<g,g\>.$
\end{subthm}
\end{thm}

The following exercise provides a property of gradients that will play a key role in the proof of the first distance estimate (\ref{thm:dist-est}).

\begin{thm}[!]{Exercise}\label{ex:monotonicity}
Let $f$ be a $\lambda$-concave function on an Alexandrov space.
Suppose that gradients $\nabla_xf$ and $\nabla_yf$ are defined.
Show that 
\[\<\dir{x}{y},\nabla_{x}f\>
+
\<\dir{y}{x},\nabla_{y}f\>
+
\lambda\cdot\dist{x}{y}{}\ge 0.\]
\end{thm}

\begin{figure}[ht!]
\centering
\includegraphics{mppics/pic-409}
\end{figure}

\begin{thm}{Proposition}\label{prop:grad-exist}
Suppose that a semiconcave function $f$ is defined in a neighborhood of a point $p$ in an Alexandrov space.
Then the gradient $\nabla_pf$ is uniquely defined.

Moreover, if $\dd_pf\le 0$, then we have $\nabla_pf=0$;
otherwise, $\nabla_pf\z=s\cdot \overline{\xi}$, where 
$s= \dd_pf(\overline{\xi})$
and
$\overline{\xi}\in \Sigma_p$ is the direction that maximize the value $\dd_pf(\xi)$ for $\xi\in \Sigma_p$.
\end{thm}


\begin{thm}{Key lemma}\label{lem:ohta} 
Let $f$ be a semiconcave function that is defined in a neighborhood of a point $p$
in an Alexandrov space $\spc{A}$. 
Then for any $u,v\in \T_p$, we have
\[s\cdot \sqrt{|u|^2+2\cdot\<u,v\> +|v|^2}
\ge 
(\dd_p f)(u)+(\dd_p f)(v),\]
where
\[s=\sup\set{(\dd_p f)(\xi)}{\xi\in\Sigma_p}.\]

\end{thm}

If $\T_p\iso\EE^m$ and $\dd_p f$ is a concave function,
then
\[2\cdot(\dd_p f)(\tfrac{u+v}2)\ge(\dd_p f)(u)+(\dd_p f)(v).\]
The latter implies the statement since $|u+v|=\sqrt{|u|^2+2\cdot\<u,v\> +|v|^2}$.
In general, $\T_p$ is not geodesic (and not even a length space), so concavity of $\dd_p f$ does not make  sense.
The key lemma however says  that in a certain sense $\dd_p f$ behaves as a concave function.

Solving the following exercise should help to find an approach to the key lemma.

\begin{thm}[!]{Exercise}\label{ex:d(distfun):==}
Let $p$ and $q$ be distinct points in an Alexandrov space $\spc{A}$.
Suppose the geodesic $[pq]$ can be extended beyond $q$.

Show that
\[\dd_p\distfun_q(v)= -\langle\dir pq,v\rangle\]
for any $v\in\T_p$.
\end{thm}

\parit{Proof of \ref{lem:ohta}.}
We will assume that $\spc{A}$ is $\Alex0$ and $f$ is concave;
the general case requires only minor modifications.
We can assume that $v\ne 0$, $w\ne 0$, and $\alpha=\mangle(u,v)>0$; otherwise, the statement is trivial.

{

\begin{wrapfigure}{r}{34 mm}
\vskip-4mm
\centering
\includegraphics{mppics/pic-1205}
\vskip0mm
\end{wrapfigure}

Consider a model configuration of five points: $\tilde p$, $\tilde u$, $\tilde v$, $\tilde q$, $\tilde w\in\EE^2$ such that
\begin{itemize}
\item $\mangle\hinge{\tilde p}{\tilde u}{\tilde v}=\alpha$, 
\item $\dist{\tilde p}{\tilde u}{}=|u|$, 
\item $\dist{\tilde p}{\tilde v}{}=|v|$,
\end{itemize}
}
\begin{itemize}
\item $\tilde q$ lies on an extension of $[\tilde p\tilde v]$ so that $\tilde v$ is the midpoint of $[\tilde p\tilde q]$, 
\item $\tilde w$ is the midpoint between $\tilde u$ and ${\tilde v}$.
\end{itemize}
Note that 
\[\dist{\tilde p}{\tilde w}{}
=
\tfrac{1}{2}\cdot\sqrt{|u|^2+2\cdot\<u,v\>+|v|^2}.\eqlbl{eq:|p-w|=}\]

Since the geodesic space of directions $\Sigma'_p$ is dense in $\Sigma_p$,
we can assume that there are geodesics in the directions of $u$ and $v$.
Choose such geodesics $\gamma_u$ and $\gamma_v$ and assume that they are parametrized with speed $|u|$ and $|v|$ respectively.
For all small $t>0$, consider points $u_t,v_t,q_t,w_t\in \spc{A}$ such that
\begin{itemize}
\item $v_t=\gamma_v(t)$,\quad  $q_t=\gamma_v(2\cdot t)$
\item $u_t=\gamma_u(t)$.
\item $w_t$ is the midpoint of $[u_t v_t]$.
\end{itemize}
Clearly 
\[\dist{p}{u_t}{}=t\cdot |u|,\qquad \dist{p}{v_t}{}=t\cdot|v|,\qquad \dist{p}{q_t}{}=2\cdot t\cdot|v|.\] 
Since $\mangle(u,v)$ is defined, 
we have 
\[\dist{u_t}{v_t}{}=t\cdot\dist{\tilde u}{\tilde v}{}+o(t),
\qquad
\dist{u_t}{q_t}{}=t\cdot\dist{\tilde u}{\tilde q}{}+o(t).\]

From the point-on-side and hinge comparisons (\ref{point-on-side}$+$\ref{angle}), we have
\[\angk{v_t}p{w_t}
\ge
\angk{v_t}p{u_t}
\ge
\mangle\hinge{\tilde v}{\tilde p}{\tilde u}+\tfrac{o(t)}t\]
and
\[\angk{v_t}{q_t}{w_t}
\ge
\angk{v_t}{q_t}{u_t}
\ge
\mangle\hinge{\tilde v}{\tilde q}{\tilde u}+\tfrac{o(t)}t.\]
Clearly, 
$\mangle\hinge{\tilde v}{\tilde p}{\tilde u}+\mangle\hinge{\tilde v}{\tilde q}{\tilde u}=\pi$. 
From the adjacent angle comparison (\ref{2-sum}), 
$\angk{v_t}p{u_t}\z+\angk{v_t}{u_t}{q_t}\le \pi$.
Hence
$\angk{v_t}p{w_t}
\to
\mangle\hinge{\tilde v}{\tilde p}{\tilde w}$ as $t\to0+$
and thus 
\[\dist{p}{w_t}{}=t\cdot\dist{\tilde p}{\tilde w}{}+o(t).\]

Without loss of generality, we can assume that $f(p)=0$.
Since $f$ is concave, we have 
\begin{align*}
2\cdot f(w_t)&\ge f(u_t)+f(v_t)=
\\
&=t\cdot [(\dd_p f)(u)+(\dd_p f)(v)]+o(t).
\end{align*}
 
Applying concavity of $f$, we have
\begin{align*}
(\dd_p f)(\dir p{w_t})
&\ge 
\frac{f(w_t)}{\dist{p}{w_t}{}}
\ge 
\\
&\ge
\frac{t\cdot[(\dd_p f)(u)+(\dd_p f)(v)]+o(t)}{2\cdot t\cdot\dist{\tilde p}{\tilde w}{}+o(t)}.
\end{align*}
By \ref{eq:|p-w|=}, the key lemma follows.
\qeds

\parit{Proof of \ref{prop:grad-exist}; uniqueness.} 
If $g,g'\in \T_p$ are two gradients of $f$,
then 
\begin{align*}
\<g,g\>
&=(\dd_p f)(g)\le \<g,g'\>,
&
\<g',g'\>
&=(\dd_p f)(g')\le \<g,g'\>.
\end{align*}
Therefore,
\[\dist[2]{g}{g'}{}=\<g,g\>-2\cdot\<g,g'\>+\<g',g'\>\le0.\] 
It follows that $g=g'$.

\parit{Existence.} 
If $\dd_p f\le 0$, then one can take $\nabla_p f=\0$.

Suppose $s=\sup\set{(\dd_p f)(\xi)}{\xi\in\Sigma_p}>0$, 
it is sufficient to show that there is  $\overline{\xi}\in \Sigma_p$ such that 
\[
(\dd_p f)\left(\overline{\xi}\right)=s.
\eqlbl{overlinexi}
\]
Indeed, suppose $\overline{\xi}$ exists.
Applying \ref{lem:ohta} for $u=\overline{\xi}$, $v=\eps\cdot w$ with $\eps\to0+$, 
we get
\[(\dd_p f)(w)\le \<w,s\cdot\overline{\xi}\>\] 
for any $w\in\T_p$;
that is, $s\cdot\overline{\xi}$ is the gradient at $p$.

Take a sequence of directions $\xi_n\in \Sigma_p$, such that $(\dd_p f)(\xi_n)\to s$.
Applying \ref{lem:ohta} for $u=\xi_n$ and $v=\xi_m$, we get
\[s
\ge
\frac{(\dd_p f)(\xi_n)+(\dd_p f)(\xi_m)}{\sqrt{2+2\cdot\cos\mangle(\xi_n,\xi_m)}}.\]
Therefore $\mangle(\xi_n,\xi_m)\to0$ as $n,m\to\infty$;
that is, $\xi_1,\xi_2,\dots$ is a Cauchy sequence.
Clearly, $\overline{\xi}=\lim_n\xi_n$ meets \ref{overlinexi}.
\qeds

\begin{thm}[!]{Exercise}\label{ex:convergence-grad}
Let $f$ and $g$ be locally Lipschitz semiconcave functions defined in a neighborhood of a point $p$ in an Alexandrov space.
Show that 
\[\dist[2]{\nabla_p f}{\nabla_p g}{\T_p}
\le 
s\cdot(|\nabla_p f|+|\nabla_p g|),\]
where
\[s=\sup\set{|(\dd_p f)(\xi)-(\dd_p g)(\xi)|}{\xi\in\Sigma_p}.\]

Conclude that if the sequence of restrictions $\dd_p f_n|_{\Sigma_p}$ converges uniformly, then $\nabla_pf_n$ converges as $n\to\infty$.
Here we assume that all functions $f_1$, $f_2,\dots$ are semiconcave and locally Lipschitz. 
\end{thm}

\begin{thm}[!]{Exercise}\label{ex:semicontinuous-grad}
Let $f$ be a locally Lipschitz $\lambda$-concave function on an Alexandrov space $\spc{A}$.

\begin{subthm}{ex:semicontinuous-grad:>s}
Suppose $s\ge 0$.
Show that $|\nabla_xf|> s$ if and only if for some point $y$ we have
\[f(y)-f(x)>s\cdot \ell+\lambda\cdot \tfrac{\ell^2}2,\]
were $\ell=\dist{x}{y}{}$.
\end{subthm}

\begin{subthm}{ex:semicontinuous-grad:lim} Show that $x\mapsto|\nabla_xf|$ is lower semicontinuous;
that is,
\[|\nabla_{x_\infty}f|\le \liminf_{x_n\to x_\infty} |\nabla_{x_n}f|.\]

\end{subthm}

\end{thm}
