\chapter{Convex polyhedra}\label{chap:alex-embedding}

In this lecture we discuss intrisic geometry of surfaces of convex polyhedra.
We will prove the Cauchy theorem, and then modify it to prove the Alexandrov uniqueness theorem.

\section{Surface of convex polyhedra}

Let us define a \index{convex body}\emph{convex body} as a compact convex subset in $\EE^3$ with non-empty interior.

The \index{surface}\emph{surface} of a convex body is defined as its boundary equipped with the induced length metric.

\begin{thm}{Exercise}\label{ex:surf-S2}
Show that the surface of a convex body is homeomorphic to $\SSS^2$.
\end{thm}

A \index{convex polyhedron}\emph{convex polyhedron} is a convex body with a finite number of extremal points, called its \index{vertex}\emph{vertices}.

The surface, say $\Sigma$, of a convex polyhedron $P$ admits a finite triangulation such that each triangle is isometric to a plane triangle.
In other words, $\Sigma$ is a closed \index{polyhedral surface}\emph{polyhedral surface};
that is, it is a 2-dimensional manifold with a length metric that admits a finite triangulation such that each triangle is isometric to a solid plane triangle.
A \index{triangulation}\emph{triangulation} of a polyhedral surface will always be assumed to satisfy this condition.

The total angle around a vertex $v$ in $\Sigma$ is defined as the sum of angles at $v$ of all triangles in the triangulation that contain $v$.

If a point $p\in \Sigma$ is not a vertex of $P$,
then
\begin{itemize}
\item $p$ lies in the interior of a face of $P$, and its neighborhood in $\Sigma$ is a piece of plane, or
\item $p$ lies on an edge, and its neighborhood is two half-planes glued along the boundary.
\end{itemize}
In both cases, a neighborhood of $p$ in $\Sigma$ (with the induced length metric) is isometric to an open domain of the plane.
In this case, the total angle around $p$ will be defined to be $2\cdot\pi$.

\begin{thm}{Claim}\label{clm:total-angle}
Let $\Sigma$ be the surface of a convex polyhedron $P$.
Then, the total angle around any point in $\Sigma$ cannot exceed $2\cdot\pi$.
\end{thm}

The proof relies on the triangle inequality for angles (or the spherical triangle inequality).
It follows from \ref{claim:angle-3angle-inq}, but our proof of this statement is a straightforward generalization of the argument in the classical geometry textbook \cite[§ 47]{kiselev-stereo-en} that proves following staement.

\begin{thm}{Spherical triangle inequality}\label{ex:angle-triangle}
Let $w_1,w_2,w_3$ be unit vectors in $\EE^3$.
Denote by $\alpha_{i,j}$ the angle between the vectors $v_i$ and $v_j$.
Then
$$\alpha_{1,3}\le \alpha_{1,2}+\alpha_{2,3}.$$
Moreover, in the case of equality, the three solid angles spanned by $w_1$, $w_2$, and $w_3$ form a plane.
\end{thm}

\parit{Proof of \ref{clm:total-angle}.}
Consider the intersection of $P$ with a small sphere centered at~$p$;
it is a convex spherical polygon, say $F$.
Applying rescaling we may assume that the sphere has unit radius.
Then we need to show that the perimeter of $F$ does not exceed $2\cdot\pi$.

\begin{wrapfigure}{o}{22mm}
\vskip-4mm
\centering
\includegraphics{mppics/pic-1103}
\end{wrapfigure}

Note that $F$ lies in a hemisphere, say $H$.
Moreover, there is a decreasing sequence of convex spherical polygons
\[H=H_0\supset\dots\supset H_n=F,\]
such that $H_{i+1}$ is obtained from $H_{i}$ by cutting along a chord.

By the spherical triangle inequality (\ref{ex:angle-triangle}), we have
\[
2\cdot\pi=\perim H=\perim H_0\ge\dots\ge\perim H_n=\perim F
\]
--- hence the result.
\qedsf

A vertex of a triangulation of a polyhedral surface is called \index{essential vertex}\emph{essential} if the total angle around it is not $2\cdot\pi$.

\begin{thm}{Exercise}\label{ex:vertex-essential-vertex}
Let $v$ be a point on the surface $\Sigma$ of a convex polyhedron $P$.
Show that $v$ is a vertex of $P$ if and only if
$v$ is an essential vertex of $\Sigma$.
\end{thm}


\begin{thm}{Exercise}\label{ex:geodesic-vertex}
Show that geodesics on the surface of a convex polyhedron may have essential vertices only at their ends.
\end{thm}

\section{Curvature}

Let $p$ be a point on the surface of a (possibly nonconvex) polyhedron, and $\theta_p$ is the total angle around $p$.
The value $2\cdot \pi -\theta_p$ is called the \index{curvature}\emph{curvature} of the polyhedral surface at $p$.
If $p$ is not a vertex, then its curvature is defined to be zero.

\begin{thm}{Exercise}\label{pr:tetrahedron}
Assume that the surface of a nondegenerate tetrahedron $T$ has curvature $\pi$ at each of its vertices.
Show that

\begin{subthm}{pr:tetrahedron:=}
all faces of $T$ are congruent;
\end{subthm}

\begin{subthm}{pr:tetrahedron:perp} the line containing midpoints of opposite edges of $T$ intersects these edges at right angles.
\end{subthm}

\end{thm}

Claim~\ref{clm:total-angle} says that \textit{surfaces of convex polyhedra have nonnegative curvature} in the sense of the above definition.
Now we show that this definition agrees with the 4-point comparison.

\begin{thm}{Proposition}\label{prop:poly-CBB}
A polyhedral surface with nonnegative curvature at each vertex is $\Alex0$.
\end{thm}

\parit{Proof.}
Denote the surface by $\Sigma$.
By \ref{comp-kappa}, it is sufficient to check that
$\distfun_p^2\circ\gamma$ is 1-concave for any geodesic $\gamma$ and a point $p$ in $\Sigma$.

We can assume that $p$ is not a vertex;
the vertex case can be done by approximation.
Applying the same argument as in \ref{ex:geodesic-vertex}, we may assume that $\gamma$ does not contain vertices.

Given a point $x=\gamma(t_0)$, choose a geodesic $[px]$.
Again, by \ref{ex:geodesic-vertex}, $[px]$ does not contain vertices.
Therefore a small neighborhood of $U\supset [px]$ can be unfolded on a plane;
that is, there is an injective length-preserving map $z\mapsto \tilde z$
of $U$ into the Euclidean plane.
This way we map part of $\gamma$ in $U$ to a line segment $\tilde\gamma$.
Let
\[\tilde f(t)\df\tfrac12\cdot\distfun_{\tilde p}^2\circ\tilde \gamma(t).\]
Since the geodesic $[px]$ maps to a line segment, we have $\tilde f(t_0)= f(t_0)$.
Furthermore, since the unfolding $z\mapsto \tilde z$ preserves lengths of curves, we get
$\tilde f(t)\ge f(t)$ if $t$ is sufficiently close to $t_0$.
That is, $\tilde f$ is a local upper barrier of $f$ at $t_0$.
Evidently, $\tilde f''\equiv 1$; therefore $f''\le 1$.
It remains to apply \ref{comp-kappa}.
\qeds

\begin{thm}{Exercise}\label{ex:poly-CBB}
Prove the converse to the proposition;
that is, show that if a poyhedral surface is $\Alex0$, then it has nonnegative curvature in the sense defined in this section.
\end{thm}


\section{Cauchy theorem}

Recall that \textit{surfaces} of convex polyhedrons are considered with the induced length metric.
 
\begin{thm}{Theorem}\label{thm:cauchy} Let $K$ and $K'$ be two non-degenerate convex polyhedrons in $\EE^3$;
denote their surfaces 
by $P$ and $P'$.
Suppose there is an isometry $P\to P'$ that sends each face of $K$ to a face of $K'$.
Then $K$ is congruent to $K'$; moreover the isometry $P\to P'$ can be extended to a motion of $\EE^3$ that maps $K$  to $K'$.
\end{thm}

\parit{Proof modulo two lemmas.}
Consider the graph $\Gamma$ formed by the edges of $K$;
the edges of $K'$ form an isomorphic graph.
 
For an edge $e$ in $\Gamma$, denote by $\alpha_e$ and $\alpha'_e$ the corresponding dihedral angles in $K$ and $K'$, respectively.
Mark $e$ by plus if $\alpha_e < \alpha'_e$ and by minus if $\alpha_e > \alpha'_e$.

Let us remove from $\Gamma$ everything that is not marked;
that is, leave only the edges marked by $(+)$ or $(-)$ and their endpoints.
If $\Gamma$ is an empty graph, then the theorem follows.
Now assume the contrary.

The graph $\Gamma$ is embedded into $P$, which is homeomorphic to the sphere.
In particular, the edges coming from one vertex have a natural cyclic order. 
Given a vertex $v$ of $\Gamma$, count the \textit{number of sign changes} around $v$;
that is, the number of consequent pairs edges with different signs. 

\begin{thm}{Local lemma}\label{lem:local}
For any vertex of $\Gamma$ the number of sign changes is at least $4$.
\end{thm}

In other words, at each vertex of $\Gamma$, one can choose 4 edges marked by $(+)$, $(-)$, $(+)$, $(-)$ in the same cyclical order.
Note that the local lemma contradicts the following.

\begin{thm}{Global lemma}\label{lem:global}
Let $\Gamma$ be a nonempty planar graph.
Then it is impossible to mark all of the edges of $\Gamma$ by $(+)$ or $(-)$
such that the number of sign changes around each vertex of $\Gamma$ is at least $4$.
\end{thm}

It remains to prove these two lemmas.
\qeds


\section{Local lemma}

The following lemma is the main ingredient in our proof of the local lemma.

\begin{thm}{Arm lemma}\label{lem:arm}
Assume that $A=[a_0 a_1\dots a_n]$ is a convex polygon in $\EE^2$
and $A'=[a'_0 a'_1\dots a'_n]$ be a polygonal line in $\EE^3$
such that 
$|a_i-a_{i+1}|=|a'_i-a'_{i+1}|$ for any $i\in\{0,\dots,n-1\}$
and 
$\measuredangle a_i\le \measuredangle a'_i$
for each $i\in\{1,\dots,n-1\}$.
Then 
$$|a_0-a_n|\le |a'_0-a'_n|$$
and equality holds if and only if $A$ is congruent to $A'$.
\end{thm}

One may view the polygonal lines $[a_0a_1\dots a_n]$ and $[a'_0a'_1\dots a'_n]$ as a robot's arm in two positions.
Informally speaking, the arm lemma says that when the arm opens,
the distance between the shoulder and tip of a finger increases. 

\begin{thm}{Exercise}\label{ex:arm-nonconvex}
Show that the arm lemma does not hold if 
instead of the convexity,
one only the local convexity;
that is, if you go along the polygonal line $a_0 a_1\dots a_n$, then you only turn left.
\end{thm}

\begin{thm}{Exercise}\label{ex:cauchy}
Suppose $A=[a_1\dots a_n]$ and $A'=[a'_1\dots a'_n]$ be noncongruent convex plane polygons with equal corresponding sides.
Mark each vertex $a_i$ with plus (minus) if the interior angle of $A$ at $a_i$ is smaller (respectively bigger) than the interior angle of $A'$ at $a_i'$.
Show that there are at least 4 sign changes around $A$. %+PIC

Give an example showing the statement does not hold without assuming convexity.

\end{thm}

\parit{Proof.}
We will view $\EE^2$ as the $xy$-plane in~$\EE^3$; 
so both $A$ and $A'$ lie in~$\EE^3$.

Let $a_m$ be the vertex of $A$ that lies on the maximal distance to the line $(a_0a_n)$.
Let us shift indexes of $a_i$ and $a'_i$ down by $m$,
so that 
\begin{align*}
a_{-m}&:=a_0,
&&\dots
&
a_{0}&:=a_m,
&&\dots
&
a_k&:=a_n,
\\
a'_{-m}&:=a'_0,
&&\dots
&
a'_{0}&:=a'_m,
&&\dots
&
a'_k&:=a'_n,
\end{align*}
where $k=n-m$.
(Here the symbol ``$:=$'' means an assignment as in programming.)

Without loss of generality, we may assume that
\begin{itemize}
\item $a_0=a'_0$ and they both coincide with the origin in $\EE^3$;
\item all $a_i$ lie in the $xy$-plane and the $x$-axis is parallel to the line $(a_{-m}a_k)$;
\item the angle $\measuredangle a'_0$ lies in $xy$-plane and contains the angle $\measuredangle a_0$ inside so that the directions to $a'_{-1}$,$a_{-1}$, $a_{1}$ and $a'_{1}$ from $a_0$ appear in the same cyclic order.
\end{itemize}

Denote by $x_i$ and $x'_i$ the projections of $a_i$ and $a'_i$ to the $x$-axis.
We can assume in addition that $x_k\ge x_{-m}$.
In this case,
$$|a_k-a_{-m}|=x_k-x_{-m}.$$
Since the projection is a distance non-expanding, we also have
$$|a'_k-a'_{-m}|\ge x'_k-x'_{-m}.$$ 

\begin{wrapfigure}{r}{60mm}
\vskip-5mm
\centering
\includegraphics{mppics/pic-30}
\vskip3mm
\end{wrapfigure}

Therefore it is sufficient to show
that 
$$x'_k-x'_{-m}\ge x_k-x_{-m}.$$
The latter holds if
$$x'_i-x'_{i-1}\ge x_i-x_{i-1}.\eqlbl{eq:|bb|=<|aa|}$$
for each $i$.
It remains to prove \ref{eq:|bb|=<|aa|}.

Let us assume that $i>0$; 
the case $i\le 0$ is similar.
Denote by $\sigma_i$ ($\sigma'_i$) the angle between the vector $w_i=a_{i}-a_{i-1}$ (respectively $w_i'=a'_{i}-a'_{i-1}$) and the $x$-axis.
Note that
$$\begin{aligned}
x_i-x_{i-1}&=|a_i-a_{i-1}|\cdot\cos\sigma_i,
\\
x'_i-x'_{i-1}&=|a_i-a_{i-1}|\cdot\cos\sigma'_i
\end{aligned}
\eqlbl{eq:proj}$$
for each $i>0$.
By construction $\sigma_1\ge \sigma'_1$.
Note that $\measuredangle (w_{i-1},w_i)\z=\pi -\measuredangle a_i$.
From convexity of $[a_1 a_1\dots a_i]$, we have
$$\sigma_i=\sigma_1+(\pi-\measuredangle a_1)+\dots+(\pi-\measuredangle a_i)$$
 for any $i>0$.
Since $\measuredangle (w'_{i-1},w'_i)=\pi -\measuredangle a'_i$,
applying \ref{ex:angle-triangle} several times,
we get
$$\sigma'_i\le\sigma'_1+(\pi-\measuredangle a'_1)+\dots+(\pi-\measuredangle a'_i).$$
Since $\measuredangle a'_j\ge \measuredangle a_j$ for each $j$, we get
$\sigma'_i\le \sigma_i$, and therefore
\[\cos \sigma'_i\ge \cos\sigma_i\]
Applying \ref{eq:proj}, we get \ref{eq:|bb|=<|aa|}.

In the case of equality, we have $\sigma_i=\sigma'_i$,
which implies $\measuredangle a_i=\measuredangle a'_i$ for each $i$.
This also implies that all $a'_i$ lie in $xy$-plane.
The latter easily follows from the equality case in \ref{ex:angle-triangle}.
\qeds
 
\parit{Proof of the local lemma (\ref{lem:local}).}
Assume that the local lemma does not hold at the vertex $v$ of $\Gamma$.
Cut from $P$ a small pyramid $\Delta$ with the vertex~$v$.
One can choose two points $a$ and $b$ on the base of $\Delta$
so that on one side of the segments $[va]$ and $[vb]$ we have only pluses
and on the other side only minuses.

The base of $\Delta$ has two polygonal lines with ends at $a$ and $b$.
Choose the one that has only pluses;
denote it by $a_0 a_1 \dots a_n$;
so $a=a_0$ and $b=a_n$.
Denote by $a'_0 a'_1 \dots a'_n$
the corresponding line in $P'$;
let $a'=a'_0$ and $b'=a'_n$.

{

\begin{wrapfigure}{r}{40mm}
\vskip-0mm
\centering
\includegraphics{mppics/pic-40}
\vskip-0mm
\end{wrapfigure}

Since each marked edge passing thru $a_i$ has a $(+)$ on it or nothing, 
we have 
$$\measuredangle a_i\le\measuredangle a'_i$$
for each $i$.

}

\begin{thm}{Exercise}\label{ex:a<a}
Prove the last statement. 
\end{thm}

By the construction we have $|a_i-a_{i-1}|=|a'_i-a'_{i-1}|$ for all $i$.
By the arm lemma (\ref{lem:arm}), 
we get 
\[|a-b|\le |a'-b'|.
\eqlbl{clm:ab<ab}\]

Swap $K$ and $K'$ and repeat the same construction for a plane passing thru $a'$ and $b'$.
We get
\[|a-b|\ge |a'-b'|.
\eqlbl{clm:ab>ab}\]

The inequalities
\ref{clm:ab<ab} and \ref{clm:ab>ab} 
together imply $|a-b|=|a'-b'|$.
The equality case in the arm lemma implies that no edge at $v$ is marked;
that is, $v$ is not a vertex of $\Gamma$
--- a contradiction.
\qeds

From the proof, it follows that the local lemma is indeed local --- it works for two nonconguent convex polyhedral angles with equal corresponding faces.
Use this observation to solve the following exercise.

\begin{wrapfigure}{r}{20mm}
\vskip-0mm
\centering
\includegraphics{mppics/pic-10}
\bigskip
\includegraphics{mppics/pic-20}
\vskip-0mm
\end{wrapfigure}

\begin{thm}{Exercise}\label{ex:disc-bend}
Consider two polyhedral discs in $\EE^3$ glued from regular polygons by the rule on the diagrams.
Assume that each disc is part of a surface of a convex polyhedron.

\begin{subthm}{}
The first configuration is rigid; that is, one can not fix the position of the pentagon and continuously move the remaining 5 vertices in a new position so that each triangle moves by a one-parameter family of isometries of $\EE^3$.
\end{subthm}

\begin{subthm}{}
Show that the second configuration has a rotational symmetry with the axis passing thru the midpoint of the marked edge.
\end{subthm}

\end{thm}

\section{Global lemma}

The proof of the global lemma is based on counting the sign changes 
in two ways;
first while moving around each vertex of $\Gamma$ 
and second while moving around each of the regions separated by $\Gamma$
on the surface~$P$. 
If two edges are adjacent at a vertex,
then they are also adjacent in a region. 
The converse is true as well. 
Therefore, both countings give the same number.

It is instructive to do the next exercise before diving into the proof.

\begin{thm}{Exercise}\label{ex:octahedron}
Try to mark the edges of an octahedron
by pluses and minuses
such that there would be 4 sign changes at each vertex.

Show that this is impossible.
\end{thm}

\parit{Proof of \ref{lem:global}.}
We can assume that $\Gamma$ is connected;
that is, one can get from any vertex to any other vertex by walking along edges.
(If not, pass to a connected component of $\Gamma$.)

Denote by $k$ and $l$ the number of vertices and edges in $\Gamma$.
Denote by $m$ the number of \textit{regions} that $\Gamma$ cuts from $P$.
Since $\Gamma$ is connected, each region is homeomorphic to an open disc.

\begin{thm}{Exercise}\label{ex:disc}
Prove the last statement.
\end{thm}

Now we can apply Euler's formula
$$k-l+m=2.
\eqlbl{eq:cauchy:euler}$$

Denote by $s$ the total number of sign changes in $\Gamma$ for all vertices. 
By the local lemma (\ref{lem:local}), we have 
$$ 4\cdot k\le s.\eqlbl{eq:S>=4k}$$

Let us get an upper bound on $s$ by counting the number of sign changes when one travels around
each region. 
Denote by $m_n$ the number of regions bounded by $n$ edges;
if an edge appears twice when it is counted twice.
Note that each region is bounded by at least $3$ edges;
therefore
$$m=m_3+m_4+m_5+\dots\eqlbl{eq:3-4-5}$$
Counting edges and using the fact that each edge belongs to exactly two regions, we get
$$2\cdot l=3\cdot m_3+ 4\cdot m_4+5\cdot m_5+\dots$$
Combining this with Euler's formula (\ref{eq:cauchy:euler}), we get
$$4\cdot k=8+2\cdot m_3+4\cdot m_4+6\cdot m_5+8\cdot m_6+\dots
\eqlbl{eq:k=2+}$$
Observe that the number of sign changes in $n$-gon regions has to be even and $\le n$.
Therefore
$$s \le 2\cdot m_3 + 4\cdot m_4 + 4\cdot m_5 + 6\cdot m_6+\dots
\eqlbl{eq:23-44-45}$$
Clearly, \ref{eq:S>=4k} and \ref{eq:23-44-45} contradict \ref{eq:k=2+}.
\qeds





\section{Comments}

This lecture contains selected material from Alexandrov's book~\cite{alexandrov}.

In Euclid's Elements, 
solids were called equal if the same holds for their faces, but no proof was given.
Adrien-Marie Legendre became interested in this problem towards the end of the 18th century.
He discussed it with his colleague Joseph-Louis Lagrange, who suggested this problem to Augustin-Louis Cauchy in 1813; soon he proved it \cite{cauchy}.
This theorem is included in many popular books \cite{aigner-zigler,dolbilin,tabacnikov-fuks}.

The observation that the face-to-face condition can be removed was made by 
Alexandr Alexandrov \cite{alexandrov-1941}.

\parit{Arm lemma.}
Original Cauchy's proof \cite{cauchy}
also used a version of the arm lemma, but its proof contained a small mistake (corrected in one century).

Several proofs of the arm lemma can be found in the letters between him and Isaac Schoenberg \cite{schoenberg-zaremba}.

The following variation of the arm lemma makes sense for nonconvex spherical polygons.
It is due to Viktor Zalgaller \cite{zalgaller}.
It can be used instead of the standard arm lemma.

\begin{thm}{Another arm lemma}
Let $A=[a_1\dots a_n]$ and $A'\z=[a'_1\z\dots a'_n]$ be two spherical $n$-gons (not necessarily convex).
Assume that $A$ lies in a half-sphere,
the corresponding sides of $A$ and $A'$ are equal
and each angle of $A$ is at least the corresponding angle in $A'$.
Then $A$ is congruent to~$A'$. 
\end{thm}

Another close relative of the arm lemma is Reshetnyak's majorization theorem \cite{alexander-kapovitch-petrunin-2019}.

\parit{Global lemma.}
A more visual proof of the global lemma is given in \cite[II \S 1.3]{alexandrov}.
This argument reused by Anton Klyachko \cite{klyachko} in his \index{car-crash lemma}\emph{car-crash lemma}.
