\chapter{Polyhedral surfaces}\label{chap:alex-embedding}

In this lecture we discuss intrisic geometry of surfaces of convex polyhedra and convex bodies.
Furhter, we prove the Cauchy theorem, and then modify the proof to get the Alexandrov uniqueness theorem.

\section{Surface of convex polyhedron}

Let us define a \index{convex body}\emph{convex body} as a compact convex subset in $\EE^3$ with non-empty interior.
The \index{surface}\emph{surface} of a convex body is defined as its boundary equipped with the induced length metric.

\begin{thm}{Exercise}\label{ex:surf-S2}
Show that the surface of a convex body is homeomorphic to the 2-dimensional sphere.
\end{thm}

A \index{convex polyhedron}\emph{convex polyhedron} is a convex body with a finite number of extremal points, called its \index{vertex}\emph{vertices}.

The surface, say $P$, of a convex polyhedron $K$ admits a finite triangulation such that each triangle is isometric to a plane triangle.
In other words, $P$ is a closed \index{polyhedral surface}\emph{polyhedral surface};
that is, it is a 2-dimensional manifold with a length metric that admits a finite triangulation such that each triangle is isometric to a solid plane triangle.
A \index{triangulation}\emph{triangulation} of a polyhedral surface will always be assumed to satisfy this condition.

The total angle around a vertex $v$ in $P$ is defined as the sum of angles at $v$ of all triangles in the triangulation that contain $v$.

If a point $p\in P$ is not a vertex of $K$,
then
\begin{itemize}
\item $p$ lies in the interior of a face of $K$, and its neighborhood in $P$ is a piece of plane, or
\item $p$ lies on an edge, and its neighborhood is two half-planes glued along the boundary.
\end{itemize}
In both cases, a neighborhood of $p$ in $P$ (with the induced length metric) is isometric to an open domain of the plane.
In this case, the total angle around $p$ will be defined to be $2\cdot\pi$.

\begin{thm}{Claim}\label{clm:total-angle}
Let $P$ be the surface of a convex polyhedron $K$.
Then, the total angle around any point $p\in P$ cannot exceed $2\cdot\pi$.
\end{thm}

The proof relies on the triangle inequality for angles (or the spherical triangle inequality).
It follows from \ref{claim:angle-3angle-inq}, but our proof of this statement is a straightforward generalization of the argument in the classical geometry textbook \cite[§ 47]{kiselev-stereo-en} that proves following staement.

\begin{thm}{Spherical triangle inequality}\label{ex:angle-triangle}
Let $w_1,w_2,w_3$ be unit vectors in $\EE^3$.
Denote by $\alpha_{i,j}$ the angle between the vectors $v_i$ and $v_j$.
Then
$$\alpha_{1,3}\le \alpha_{1,2}+\alpha_{2,3}.$$
Moreover, in the case of equality, the three solid angles spanned by $w_1$, $w_2$, and $w_3$ form a plane.
\end{thm}

\parit{Proof of \ref{clm:total-angle}.}
Consider the intersection of $K$ with a small sphere centered at~$p$;
it is a convex spherical polygon, say $F$.
Applying rescaling we may assume that the sphere has unit radius.
Then we need to show that the perimeter of $F$ does not exceed $2\cdot\pi$.

\begin{wrapfigure}{o}{22mm}
\vskip-4mm
\centering
\includegraphics{mppics/pic-1103}
\end{wrapfigure}

Note that $F$ lies in a hemisphere, say $H$.
Moreover, there is a decreasing sequence of convex spherical polygons
\[H=H_0\supset\dots\supset H_n=F,\]
such that $H_{i+1}$ is obtained from $H_{i}$ by cutting along a chord.

By the spherical triangle inequality (\ref{ex:angle-triangle}), we have
\[
2\cdot\pi=\perim H=\perim H_0\ge\dots\ge\perim H_n=\perim F
\]
--- hence the result.
\qeds

\section{Curvature}

Let $p$ be a point on a polyhedral surface, and $\theta_p$ is the total angle around $p$.
The value $2\cdot \pi -\theta_p$ is called the \index{curvature}\emph{curvature} of the polyhedral surface at $p$.

Note that if $p$ is not a vertex in a triangulation of $P$, then its curvature is zero.
A vertex of a triangulation of a polyhedral surface is called \index{essential vertex}\emph{essential} if its curvature is not $0$.

\begin{thm}{Exercise}\label{ex:vertex-essential-vertex}
Let $v$ be a point on the surface $P$ of a convex polyhedron $K$.
Show that $v$ is a vertex of $K$ if and only if
$v$ is an essential vertex of $P$.
\end{thm}


\begin{thm}{Exercise}\label{ex:geodesic-vertex}
Show that geodesics on a closed polyhedral surface with nonnegative curvature may have essential vertices only at their ends.
\end{thm}

\begin{thm}{Exercise}\label{pr:tetrahedron}
Assume that the surface of a nondegenerate tetrahedron $T$ has curvature $\pi$ at each of its vertices.
Show that

\begin{subthm}{pr:tetrahedron:=}
all faces of $T$ are congruent;
\end{subthm}

\begin{subthm}{pr:tetrahedron:perp} the line containing midpoints of opposite edges of $T$ intersects these edges at right angles.
\end{subthm}

\end{thm}

\begin{thm}{Exercise}\label{ex:gauss-bonnet}
Show that sum of curvatures of a closed polyhedral surface $P$ equals to $2\cdot\pi\cdot\chi(P)$,
where $\chi(P)$ denotes the Euler characteristic of $P$.
\end{thm}


Claim~\ref{clm:total-angle} says that \textit{surfaces of convex polyhedra have nonnegative curvature} in the sense of the above definition.
Now we show that this definition agrees with the 4-point comparison.

\begin{thm}{Proposition}\label{prop:poly-CBB}
A polyhedral surface with nonnegative curvature at each vertex is $\Alex0$.
\end{thm}

\parit{Proof.}
Denote the surface by $P$.
By \ref{comp-kappa}, it is sufficient to check that
$\distfun_p^2\circ\gamma$ is 1-concave for any geodesic $\gamma$ and any point $p$ in $P$.

We can assume that $p$ is not a vertex;
the vertex case can be done by approximation.
By \ref{ex:geodesic-vertex}, $\gamma$ does not contain vertices.

Given a point $x=\gamma(t_0)$, choose a geodesic $[px]$.
Again, by \ref{ex:geodesic-vertex}, $[px]$ does not contain vertices.
Therefore a small neighborhood of $U\supset [px]$ can be unfolded on a plane;
that is, there is an injective length-preserving map $z\mapsto \tilde z$
of $U$ into the Euclidean plane.
This way we map part of $\gamma$ in $U$ to a line segment $\tilde\gamma$.
Let
\[\tilde f(t)\df\tfrac12\cdot\distfun_{\tilde p}^2\circ\tilde \gamma(t).\]
Since the geodesic $[px]$ maps to a line segment, we have $\tilde f(t_0)= f(t_0)$.
Furthermore, since the unfolding $z\mapsto \tilde z$ preserves lengths of curves, we get
$\tilde f(t)\ge f(t)$ if $t$ if the left-hand side is defined.
That is, $\tilde f$ is a local upper barrier of $f$ at $t_0$.
Evidently, $\tilde f''\equiv 1$; therefore $f''\le 1$.
It remains to apply \ref{comp-kappa}.
\qeds

\begin{thm}{Exercise}\label{ex:poly-CBB}
Prove the converse to the proposition;
that is, show that if a poyhedral surface is $\Alex0$, then it has nonnegative curvature in the sense defined in this section.
\end{thm}

\section{Surface of convex body}

\begin{thm}{Advanced exercise}\label{ex:surface-covergence}
Let $K_1,K_2,\dots,$ and $K_\infty$ be convex bodies in $\EE^m$.
Denote by $P_n$ the surface of $K_n$ with induced length metric.
Suppose $K_n\z\to K_\infty$ in the sense of Hausdorff.
Show that $P_n\to P_\infty$ in the sense of Gromov--Hausdorff.
\end{thm}

Any convex body is a Hausdorff limit of a sequence of convex polyhedra.
Therefore, the next proposition follows from \ref{prop:poly-CBB}, \ref{ex:surface-covergence}, and \ref{thm:CBB-closed}.

\begin{thm}{Proposition}\label{prop:conv-surf-CBB(0)}
The surface of a convex body in $\EE^3$ is $\Alex0$.
\end{thm}

\begin{thm}{Very advanced exercise}\label{ex:liberman+milka}
Let $P$ be the surface of a nondegenerate convex body $K\subset\EE^3$;
we assume that $P$ is equipped with the induced length metric.

\begin{subthm}{ex:liberman+milka:liberman}
Show that any geodesic $\gamma$ in $P$ is one-sided differentiable as a curve in $\EE^3$.
\end{subthm}

\begin{subthm}{ex:liberman+milka:convex}
Suppose a plane $\Pi$ cuts from $P$ a disc $\Delta$, and the reflection of $\Delta$ across $\Pi$ lies in $K$.
Show that $\Delta$ is a convex subset of $P$;
that is, if a geodesic has ends in $\Delta$, then it completely lies in $\Delta$.
\end{subthm}


\begin{subthm}{ex:liberman+milka:milka}
Let $\gamma_1$ and $\gamma_2$ be geodesic paths in $P$ that start at one point $p\z=\gamma_1(0)\z=\gamma_2(0)$.
Suppose $x_i=\gamma_i(1)$, and $y_i\z=p+\gamma_i^+(0)$.
Show that
\[\dist{x_1}{x_2}{P}\le \dist{y_1}{y_2}{W},\]
where $W$ is the complement to the interior of $K$.
\end{subthm}

\end{thm}


\section{Cauchy theorem}

Recall that \textit{surfaces} of convex polyhedrons are considered with the induced length metric.
 
\begin{thm}{Theorem}\label{thm:cauchy}
Let $K$ and $K'$ be convex polyhedrons in $\EE^3$;
denote their surfaces 
by $P$ and $P'$.
Suppose there is an isometry $P\to P'$ that sends each face of $K$ to a face of $K'$.
Then $K$ is congruent to $K'$; moreover the isometry $P\to P'$ can be extended to a motion of $\EE^3$ that maps $K$  to $K'$.
\end{thm}

\parit{Proof modulo two lemmas.}
Consider the graph $\Gamma$ formed by the edges of $K$;
the edges of $K'$ form a naturally isomorphic graph.
 
For an edge $e$ in $\Gamma$, denote by $\alpha_e$ and $\alpha'_e$ the dihedral angles in $K$ and $K'$, respectively.
Mark $e$ by plus if $\alpha_e < \alpha'_e$ and by minus if $\alpha_e > \alpha'_e$.

Let us remove from $\Gamma$ everything that is not marked;
that is, leave only the edges marked by $(+)$ or $(-)$ and their endpoints.
If $\Gamma$ is an empty graph, then the theorem follows.
Let us assume the contrary.

The graph $\Gamma$ is embedded into $P$, which is homeomorphic to the sphere.
In particular, the edges coming from one vertex have a natural cyclic order. 
Given a vertex $v$ of $\Gamma$, count the \textit{number of sign changes} around $v$;
that is, the number of consequent pairs edges with different signs. 

\begin{thm}{Local lemma}\label{lem:local}
For any vertex of $\Gamma$ the number of sign changes is at least $4$.
\end{thm}

In other words, at each vertex of $\Gamma$, one can choose 4 edges marked by $(+)$, $(-)$, $(+)$, $(-)$ in the same cyclical order.
Note that the local lemma contradicts the following.

\begin{thm}{Global lemma}\label{lem:global}
Let $\Gamma$ be a nonempty planar graph.
Then it is impossible to mark all of the edges of $\Gamma$ by $(+)$ or $(-)$
such that the number of sign changes around each vertex of $\Gamma$ is at least $4$.
\end{thm}

It remains to prove these two lemmas.
\qeds


\section{Arm lemma}

\begin{thm}{Lemma}\label{lem:arm}
Assume that $A=[a_0 a_1\dots a_n]$ is a convex polygon in $\EE^2$
and $A'=[a'_0 a'_1\dots a'_n]$ be a polygonal line in $\EE^3$
such that 
$|a_i-a_{i+1}|=|a'_i-a'_{i+1}|$ for any $i\in\{0,\dots,n-1\}$
and 
$\measuredangle a_i\le \measuredangle a'_i$
for each $i\in\{1,\dots,n-1\}$.
Then 
$$|a_0-a_n|\le |a'_0-a'_n|$$
and equality holds if and only if $A$ is congruent to $A'$.
\end{thm}

One may view the polygonal lines $[a_0a_1\dots a_n]$ and $[a'_0a'_1\dots a'_n]$ as a robot's arm in two positions.
Informally speaking, the arm lemma says that when the arm opens,
the distance between the shoulder and tip of a finger increases;
assuming that starting position a convex plane polygon.

\begin{thm}{Exercise}\label{ex:arm-nonconvex}
Show that the arm lemma does not hold if 
instead of the convexity,
one only the local convexity;
that is, if we assume only that the polygonal line $a_0 a_1\dots a_n$ turns only left.
\end{thm}

\begin{thm}{Exercise}\label{ex:cauchy}
Suppose $A=[a_1\dots a_n]$ and $A'=[a'_1\dots a'_n]$ be noncongruent convex plane polygons with equal corresponding sides.
Mark each vertex $a_i$ with plus (minus) if the interior angle of $A$ at $a_i$ is smaller (respectively bigger) than the interior angle of $A'$ at $a_i'$.
Show that there are at least 4 sign changes around $A$. %+PIC

Give an example showing the statement does not hold without assuming convexity.

\end{thm}

\parit{Proof.}
We will view $\EE^2$ as the $xy$-plane in~$\EE^3$; 
so both $A$ and $A'$ lie in~$\EE^3$.

Let $a_m$ be the vertex of $A$ that lies on the maximal distance to the line $(a_0a_n)$.
Let us shift indexes of $a_i$ and $a'_i$ down by $m$,
so that 
\begin{align*}
a_{-m}&:=a_0,
&&\dots
&
a_{0}&:=a_m,
&&\dots
&
a_k&:=a_n,
\\
a'_{-m}&:=a'_0,
&&\dots
&
a'_{0}&:=a'_m,
&&\dots
&
a'_k&:=a'_n,
\end{align*}
where $k=n-m$.
(Here the symbol ``$:=$'' means an assignment as in programming.)

Without loss of generality, we may assume that
\begin{itemize}
\item $a_0=a'_0$ and they both coincide with the origin in $\EE^3$;
\item all $a_i$ lie in the $xy$-plane and the $x$-axis is parallel to the line $a_{-m}a_k$;
\item the angle $\measuredangle a'_0$ lies in $xy$-plane and contains the angle $\measuredangle a_0$ inside so that the directions to $a'_{-1}$,$a_{-1}$, $a_{1}$ and $a'_{1}$ from $a_0$ appear in the same cyclic order.
\end{itemize}

Denote by $x_i$ and $x'_i$ the projections of $a_i$ and $a'_i$ to the $x$-axis.
We can assume in addition that $x_k\ge x_{-m}$.
In this case,
$$|a_k-a_{-m}|=x_k-x_{-m}.$$
Since the projection is a distance non-expanding, we also have
$$|a'_k-a'_{-m}|\ge x'_k-x'_{-m}.$$ 

\begin{wrapfigure}{r}{60mm}
\vskip-5mm
\centering
\includegraphics{mppics/pic-30}
\vskip3mm
\end{wrapfigure}

Therefore it is sufficient to show
that 
$$x'_k-x'_{-m}\ge x_k-x_{-m}.$$
The latter holds if
$$x'_i-x'_{i-1}\ge x_i-x_{i-1}.\eqlbl{eq:|bb|=<|aa|}$$
for each $i$.
It remains to prove \ref{eq:|bb|=<|aa|}.

Let us assume that $i>0$; 
the case $i\le 0$ is similar.
Denote by $\sigma_i$ ($\sigma'_i$) the angle between the vector $w_i=a_{i}-a_{i-1}$ (respectively $w_i'=a'_{i}-a'_{i-1}$) and the $x$-axis.
Note that
$$\begin{aligned}
x_i-x_{i-1}&=|a_i-a_{i-1}|\cdot\cos\sigma_i,
\\
x'_i-x'_{i-1}&=|a_i-a_{i-1}|\cdot\cos\sigma'_i
\end{aligned}
\eqlbl{eq:proj}$$
for each $i>0$.
By construction $\sigma_1\ge \sigma'_1$.
Note that $\measuredangle (w_{i-1},w_i)\z=\pi -\measuredangle a_i$.
From convexity of $[a_1 a_1\dots a_i]$, we have
$$\sigma_i=\sigma_1+(\pi-\measuredangle a_1)+\dots+(\pi-\measuredangle a_i)$$
 for any $i>0$.
Since $\measuredangle (w'_{i-1},w'_i)=\pi -\measuredangle a'_i$,
applying the triangle inequality for angles (\ref{ex:angle-triangle}) several times,
we get
$$\sigma'_i\le\sigma'_1+(\pi-\measuredangle a'_1)+\dots+(\pi-\measuredangle a'_i).$$
Since $\measuredangle a'_j\ge \measuredangle a_j$ for each $j$, we get
$\sigma'_i\le \sigma_i$, and therefore
\[\cos \sigma'_i\ge \cos\sigma_i\]
Applying \ref{eq:proj}, we get \ref{eq:|bb|=<|aa|}.

In the case of equality, we have $\sigma_i=\sigma'_i$,
which implies $\measuredangle a_i=\measuredangle a'_i$ for each $i$.
This also implies that all $A'$ is a convex polygonal lien in the $xy$-plane.
The latter easily follows from the equality case in \ref{ex:angle-triangle}.
\qeds

\begin{thm}{Advanced exercise}\label{ex:arm'}
Let $A$ and $A'$ be as in the arm lemma (\ref{lem:arm}).

\begin{subthm}{ex:bow'+}
Suppose that $\measuredangle \hinge{a_n}{a_{n-1}}{a_0}\le\tfrac\pi2$.
Show that $\measuredangle \hinge{a_0}{a_1}{a_n}\ge \measuredangle \hinge{a_0'}{a_1'}{a_n'}$.
\end{subthm}

\begin{subthm}{ex:bow'-} Show that the inequality $\measuredangle \hinge{a_0}{a_1}{a_n}\ge \measuredangle \hinge{a_0'}{a_1'}{a_n'}$ does not hold in general.
\end{subthm}

\end{thm}

\section{Proof of local lemma}
 
\parit{Proof of the local lemma (\ref{lem:local}).}
Assume that the local lemma does not hold at the vertex $v$ of $\Gamma$.
Choose a plane that cuts from $P$ a small pyramid $\Delta$ with the vertex~$v$.
One can choose two points $a$ and $b$ on the base of $\Delta$
so that on one side of the segments $[va]$ and $[vb]$ we have only pluses
and on the other side only minuses.

The base of $\Delta$ has two polygonal lines with ends at $a$ and $b$.
Choose the one that has only pluses;
denote it by $a_0 a_1 \dots a_n$;
so $a=a_0$ and $b=a_n$.
Denote by $a'_0 a'_1 \dots a'_n$
the corresponding line in $P'$;
let $a'=a'_0$ and $b'=a'_n$.

{

\begin{wrapfigure}{r}{40mm}
\vskip-0mm
\centering
\includegraphics{mppics/pic-40}
\vskip-0mm
\end{wrapfigure}

Since each marked edge passing thru $a_i$ has a $(+)$ on it or nothing, 
we have 
$$\measuredangle a_i\le\measuredangle a'_i$$
for each $i$.

}

\begin{thm}{Exercise}\label{ex:a<a}
Prove the last statement. 
\end{thm}

By the construction we have $|a_i-a_{i-1}|=|a'_i-a'_{i-1}|$ for all $i$.
By the arm lemma (\ref{lem:arm}), 
we get 
\[|a-b|\le |a'-b'|.
\eqlbl{clm:ab<ab}\]

Swap $K$ and $K'$ and repeat the same construction for a plane passing thru $a'$ and $b'$.
We get
\[|a-b|\ge |a'-b'|.
\eqlbl{clm:ab>ab}\]

The inequalities
\ref{clm:ab<ab} and \ref{clm:ab>ab} 
together imply $|a-b|=|a'-b'|$.
The equality case in the arm lemma implies that no edge at $v$ is marked;
that is, $v$ is not a vertex of $\Gamma$
--- a contradiction.
\qeds

From the proof, it follows that the local lemma is indeed local --- it works for two nonconguent convex polyhedral angles with equal corresponding faces.
Use this observation in the following exercise.

\begin{wrapfigure}{r}{20mm}
\vskip-0mm
\centering
\includegraphics{mppics/pic-10}
\bigskip
\includegraphics{mppics/pic-20}
\vskip-0mm
\end{wrapfigure}

\begin{thm}{Exercise}\label{ex:disc-bend}
Consider two polyhedral discs in $\EE^3$ glued from regular polygons by the rule on the diagrams.
Assume that each disc is part of a surface of a convex polyhedron.

\begin{subthm}{}
The first configuration is rigid; that is, one can not fix the position of the pentagon and continuously move the remaining 5 vertices in a new position so that each triangle moves by a one-parameter family of isometries of $\EE^3$.
\end{subthm}

\begin{subthm}{}
Show that the second configuration has a rotational symmetry with the axis passing thru the midpoint of the marked edge.
\end{subthm}

\end{thm}

\section{Proof of global lemma}

It is instructive to do the next exercise before diving into the proof.

\begin{thm}{Exercise}\label{ex:octahedron}
Try to mark the edges of an octahedron
by pluses and minuses
such that there would be 4 sign changes at each vertex.

Show that this is impossible.
\end{thm}

The proof of the global lemma is based on counting the sign changes
in two ways;
first while walking around each vertex of $\Gamma$
and second while moving around each of the regions separated by $\Gamma$
on the surface~$P$.
If two edges are adjacent at a vertex,
then they are also adjacent in a region.
The converse is true as well.
Therefore, both countings give the same number.

\parit{Proof of \ref{lem:global}.}
We can assume that $\Gamma$ is connected;
that is, one can get from any vertex to any other vertex by walking along edges.
(If not, pass to a connected component of $\Gamma$.)

We can assume that $\Gamma$ is embedded in the sphere.
Denote by $k$ and $l$ the number of vertices and edges in $\Gamma$.
Denote by $m$ the number of \textit{regions} that $\Gamma$ cuts from $P$.
Since $\Gamma$ is connected, each region is homeomorphic to an open disc.

\begin{thm}{Exercise}\label{ex:disc}
Prove the last statement.
\end{thm}

Now we can apply Euler's formula
$$k-l+m=2.
\eqlbl{eq:cauchy:euler}$$

Denote by $s$ the total number of sign changes in $\Gamma$ for all vertices. 
By the local lemma (\ref{lem:local}), we have 
$$ 4\cdot k\le s.\eqlbl{eq:S>=4k}$$

Let us get an upper bound on $s$ by counting the number of sign changes when one walks around
each region. 
Denote by $m_n$ the number of regions bounded by $n$ edges;
if an edge appears twice when it is counted twice.
Note that each region is bounded by at least $3$ edges;
therefore
$$m=m_3+m_4+m_5+\dots\eqlbl{eq:3-4-5}$$
Since edge belongs to exactly two regions, we get
$$2\cdot l=3\cdot m_3+ 4\cdot m_4+5\cdot m_5+\dots$$
Combining this with Euler's formula \ref{eq:cauchy:euler}, we get
$$4\cdot k=8+2\cdot m_3+4\cdot m_4+6\cdot m_5+8\cdot m_6+\dots
\eqlbl{eq:k=2+}$$
Observe that the number of sign changes in $n$-gon regions has to be even and $\le n$.
Therefore
$$s \le 2\cdot m_3 + 4\cdot m_4 + 4\cdot m_5 + 6\cdot m_6+\dots
\eqlbl{eq:23-44-45}$$
Clearly, \ref{eq:S>=4k} and \ref{eq:23-44-45} contradict \ref{eq:k=2+}.
\qeds


\section{Alexandrov uniqueness theorem}

Alexandrov's uniqueness theorem states that the conclusion of the Cauchy theorem (\ref{thm:cauchy}) still holds without the face-to-face assumption.

\begin{thm}{Theorem}\label{thm:alexandrov-uni'}
Any two convex polyhedrons in $\EE^3$ with isometric surfaces are congruent.

Moreover, any isometry between surfaces of convex polyhedrons can be extended to an isometry of the whole $\EE^3$.
\end{thm}

Instead of proof we list the modifications needed in the proof of Cauchy's theorem.

\parit{List of modifications in the proof of \ref{thm:cauchy}.}
Suppose $\iota\:P\z\to P'$ be an isometry between surfaces of $K$ and $K'$.
Mark in $P$ all the edges of $K$ and all the inverse images of edges in $K'$.
It might happen that an edge in $K'$ does not correspond to an edge in $K$;
it this case its inverse image in $K$ will be called a \index{fake edge}\emph{fake edge} of $K$.

The marked lines divide $P$ into convex polygons, and the restriction of $\iota$ to each polygon is a rigid motion.
These polygons should be used instead of faces in the proof of \ref{thm:cauchy}.

A vertex of the obtained graph can be a vertex of $K$, or it can be a fake vertex;
that is, it might be an intersection of an edge and a fake edge.

\begin{figure}[ht!]
\vskip-0mm
\centering
\includegraphics{mppics/pic-50}
\vskip-0mm
\end{figure}

For the first type of vertex, the local lemma can be proved the same way.
For a fake vertex $v$, it is easy to see that both parts of the edge coming thru $v$ are marked with minus
while both of the fake edges at $v$ are marked with plus.
Therefore, the local lemma holds for the fake vertices as well.

The remaining parts of the proof need no modifications.
\qeds

\begin{thm}{Exercise}\label{pr:K-P-simmetry}
Let $K$ be a convex polyhedron in $\EE^3$;
denote by $P$ its surface.
Show that each isometry $\iota\:P\z\to P$,
can be extended to an isometry of $\EE^3$.
\end{thm}


\section{Remarks}

This lecture contains selected material from Alexandrov's book~\cite{alexandrov}.

In Euclid's Elements, 
solids were claimed to be equal if the same holds for their faces, but no proof was given.
Adrien-Marie Legendre became interested in this problem towards the end of the 18th century.
He discussed it with his colleague Joseph-Louis Lagrange, who suggested this problem to Augustin-Louis Cauchy in 1813; soon he proved it \cite{cauchy}.
This theorem is included in many popular books \cite{aigner-zigler,dolbilin,tabacnikov-fuks}.
The key observation that the face-to-face condition can be removed was made by
Alexandr Alexandrov \cite{alexandrov-1941}.

\parit{Arm lemma.}
Cauchy's proof \cite{cauchy}
also used a version of the arm lemma, but its proof contained a small mistake that was corrected in a century \cite{sabitov}.

Several proofs of the arm lemma can be found in the letters between Isaac Schoenberg and Stanisław Zaremba \cite{schoenberg-zaremba}.

The following variation of the arm lemma makes sense for nonconvex spherical polygons.
It is due to Viktor Zalgaller \cite{zalgaller}.
It can be used instead of the standard arm lemma.

\begin{thm}{Another arm lemma}
Let $A=[a_1\dots a_n]$ and $A'\z=[a'_1\z\dots a'_n]$ be two spherical $n$-gons (not necessarily convex).
Assume that $A$ lies in a half-sphere,
the corresponding sides of $A$ and $A'$ are equal
and each angle of $A$ is at least the corresponding angle in $A'$.
Then $A$ is congruent to~$A'$. 
\end{thm}

Another close relative of the arm lemma is Reshetnyak's majorization theorem \cite{reshetnyak}.

\parit{Global lemma.}
A more visual proof of the global lemma is given in \cite[II \S 1.3]{alexandrov}.
This argument reused by Anton Klyachko \cite{klyachko} in his \index{car-crash lemma}\emph{car-crash lemma}.

\parit{Approximation.}
Proposition \ref{prop:conv-surf-CBB(0)} generalizes to boundaries of convex bodies  in $\EE^m$ for any $m\ge 2$.
This could be considered as a partial case of the conjecture about boundary of Alexandrov space; see \ref{conj:bry}.
Another partial case is proved by the authors and Stephanie Alexander \cite{alexander-kapovitch-petrunin-2008}.

\parit{Existance theorem.}
\ref{ex:surf-S2} and \ref{prop:poly-CBB} imply that the surface of a convex body is a sphere with nonnegative curvature in the sense of Alexandrov.
The celebrated theorem of Alexandrov states that the converse also holds if we allow degeneration of convex bodies to plane figures;
the surface of a plane figure is defined as its doubling across the boundary.
In other words, any $\Alex0$ metric on the two-sphere is isometric to a surface of a (possibly degenerate) convex body.
Moreover this convex body is unique up to congruence.
The last part is due to Alexei Pogorelov \cite{pogorelov}.

Originally, Alexandrov proved the statement for polyhedral metrics on the sphere; this proof is sketched in the appendix.
Then he used approximation to extend the result to  arbitrary $\Alex0$ metrics on the sphere.


