\chapter{Alexandrov embedding theorem}\label{chap:alex-embedding}

We will prove the Cauchy theorem, and then modify it to prove the Alexandrov uniqueness theorem.
Further, we sketch a proof of the Alexandrov embedding theorem.


\section{Cauchy theorem}

Recall that \textit{surfaces} of convex polyhedrons are considered with the induced length metric..
 
\begin{thm}{Theorem}\label{thm:cauchy} Let $K$ and $K'$ be two non-degenerate convex polyhedrons in $\EE^3$;
denote their surfaces 
by $P$ and $P'$.
Suppose there is an isometry $P\to P'$ that sends each face of $K$ to a face of $K'$.
Then $K$ is congruent to $K'$; moreover the isometry $P\to P'$ can be extended to a motion of $\EE^3$ that maps $K$  to $K'$.
\end{thm}

\parit{Proof.} 
Consider the graph $\Gamma$ formed by the edges of $K$;
the edges of $K'$ form the same graph.
 
For an edge $e$ in $\Gamma$, denote by $\alpha_e$ and $\alpha'_e$ the corresponding dihedral angles in $K$ and $K'$ respectively.
Mark $e$ by plus if $\alpha_e < \alpha'_e$ and by minus if $\alpha_e > \alpha'_e$.

Now remove from $\Gamma$ everything that was not marked;
that is, leave only the edges marked by $(+)$ or $(-)$ and their endpoints.

Note that the theorem follows if $\Gamma$ is an empty graph;
assume the contrary.

The graph $\Gamma$ is embedded into $P$, which is homeomorphic to the sphere.
In particular, the edges coming from one vertex have a natural cyclic order. 
Given a vertex $v$ of $\Gamma$, count the \textit{number of sign changes} around $v$;
that is, the number of consequent pairs edges with different signs. 

\begin{thm}{Local lemma}\label{lem:local}
For any vertex of $\Gamma$ the number of sign changes is at least $4$.
\end{thm}

In other words, at each vertex of $\Gamma$, one can choose 4 edges marked by $(+)$, $(-)$, $(+)$, $(-)$ in the same cyclical order.
Note that the local lemma contradicts the following.

\begin{thm}{Global lemma}\label{lem:global}
Let $\Gamma$ be a nonempty subgraph of the graph formed by the edges of a convex polyhedron. Then it is impossible to mark all of the edges of $\Gamma$ by $(+)$ or $(-)$ 
such that the number of sign changes around each vertex of $\Gamma$ is at least $4$.
\end{thm}

It remains to prove these two lemmas.
\qeds


\section{Local lemma}

Next lemma is the main ingredient in our proof of the local lemma.

\begin{thm}{Arm lemma}\label{lem:arm}
Assume that $A=[a_0 a_1\dots a_n]$ is a convex polygon in $\EE^2$
and $A'=[a'_0 a'_1\dots a'_n]$ be a polygonal line in $\EE^3$
such that 
$$|a_i-a_{i+1}|=|a'_i-a'_{i+1}|$$ for any $i\in\{0,\dots,n-1\}$
and 
$$\measuredangle a_i\le \measuredangle a'_i$$ 
for each $i\in\{1,\dots,n-1\}$.
Then 
$$|a_0-a_n|\le |a'_0-a'_n|$$
and equality holds if and only if $A$ is congruent to $A'$.
\end{thm}

One may view the polygonal lines $[a_0a_1\dots a_n]$ and $[a'_0a'_1\dots a'_n]$ as a robot's arm in two positions.
The arm lemma states that when the arm opens, 
the distance between the shoulder and tip of a finger increases. 

\begin{thm}{Exercise}\label{ex:arm-nonconvex}
Show that the arm lemma does not hold if 
instead of the convexity,
one only the local convexity;
that is, if you go along the polygonal line $a_0 a_1\dots a_n$, then you only turn left.
\end{thm}

\begin{thm}{Exercise}\label{ex:cauchy}
Suppose $A=[a_1\dots a_n]$ and $A'=[a'_1\dots a'_n]$ be noncongruent convex plane polygons with equal corresponding sides.
Mark each vertex $a_i$ with plus (minus) if the interior angle of $A$ at $a_i$ is smaller (respectively bigger) than the interior angle of $A'$ at $a_i'$.
Show that there are at least 4 sign changes around $A$. %+PIC

Give an example showing the statement does not hold without assuming convexity.

\end{thm}

\parit{Proof.}
We will view $\EE^2$ as the $xy$-plane in~$\EE^3$; 
so both $A$ and $A'$ lie in~$\EE^3$.
Let $a_m$ be the vertex of $A$ that lies on the maximal distance to the line $(a_0a_n)$.

Let us shift indexes of $a_i$ and $a'_i$ down by $m$,
so that 
\begin{align*}
a_{-m}&:=a_0,
&&\dots
&
a_{0}&:=a_m,
&&\dots
&
a_k&:=a_n,
\\
a'_{-m}&:=a'_0,
&&\dots
&
a'_{0}&:=a'_m,
&&\dots
&
a'_k&:=a'_n,
\end{align*}
where $k=n-m$.
(Here the symbol ``$:=$'' means an assignment as in programming.)

Without loss of generality, we may assume that
\begin{itemize}
\item $a_0=a'_0$ and they both coincide with the origin $(0,0,0)\in\EE^3$;
\item all $a_i$ lie in the $xy$-plane and the $x$-axis is parallel to the line $(a_{-m}a_k)$;
\item the angle $\measuredangle a'_0$ lies in $xy$-plane and contains the angle $\measuredangle a_0$ inside
and the directions to $a'_{-1}$,$a_{-1}$, $a_{1}$ and $a'_{1}$ from $a_0$ appear in the same cyclic order.
\end{itemize}

Denote by $x_i$ and $x'_i$ the projections of $a_i$ and $a'_i$ to the $x$-axis.
We can assume in addition that $x_k\ge x_{-m}$.
In this case,
$$|a_k-a_{-m}|=x_k-x_{-m}.$$
Since the projection is a distance non-expanding, we also have
$$|a'_k-a'_{-m}|\ge x'_k-x'_{-m}.$$ 
{

\begin{wrapfigure}{r}{60mm}
\vskip-7mm
\centering
\includegraphics{mppics/pic-30}
\vskip-0mm
\end{wrapfigure}

Therefore it is sufficient to show
that 
$$x'_k-x'_{-m}\ge x_k-x_{-m}.$$
The latter holds if
$$x'_i-x'_{i-1}\ge x_i-x_{i-1}.\eqlbl{eq:|bb|=<|aa|}$$
for each $i$.
It remains to prove \ref{eq:|bb|=<|aa|}.

}

Let us assume that $i>0$; 
the case $i\le 0$ is similar.
Denote by $\sigma_i$ ($\sigma'_i$) the angle between the vector $w_i=a_{i}-a_{i-1}$ (respectively $w_i'=a'_{i}-a'_{i-1}$) and the $x$-axis.
Note that
$$\begin{aligned}
x_i-x_{i-1}&=|a_i-a_{i-1}|\cdot\cos\sigma_i,
\\
x'_i-x'_{i-1}&=|a_i-a_{i-1}|\cdot\cos\sigma'_i
\end{aligned}
\eqlbl{eq:proj}$$
for each $i>0$.
By construction $\sigma_1\ge \sigma'_1$.
Note that $\measuredangle (w_{i-1},w_i)\z=\pi -\measuredangle a_i$.
From convexity of $[a_1 a_1\dots a_i]$, we have
$$\sigma_i=\sigma_1+(\pi-\measuredangle a_1)+\dots+(\pi-\measuredangle a_i)$$
 for any $i>0$.
Since $\measuredangle (w'_{i-1},w'_i)=\pi -\measuredangle a'_i$,
applying \ref{ex:angle-triangle} several times,
we get
$$\sigma'_i\le\sigma'_1+(\pi-\measuredangle a'_1)+\dots+(\pi-\measuredangle a'_i).$$
Since $\measuredangle a'_j\ge \measuredangle a_j$ for each $j$, we get
$\sigma'_i\le \sigma_i$, and therefore
\[\cos \sigma'_i\ge \cos\sigma_i\]
Applying \ref{eq:proj}, we get \ref{eq:|bb|=<|aa|}.

In the case of equality, we have $\sigma_i=\sigma'_i$,
which implies $\measuredangle a_i=\measuredangle a'_i$ for each $i$.
This also implies that all $a'_i$ lie in $xy$-plane.
The latter easily follows from the equality case in \ref{ex:angle-triangle}.
\qeds
 
\parit{Proof of the local lemma (\ref{lem:local}).}
Assume that the local lemma does not hold at the vertex $v$ of $\Gamma$.
Cut from $P$ a small pyramid $\Delta$ with the vertex~$v$.
One can choose two points $a$ and $b$ on the base of $\Delta$
so that on one side of the segments $[va]$ and $[vb]$ we have only pluses
and on the other side only minuses.

The base of $\Delta$ has two polygonal lines with ends at $a$ and $b$.
Choose the one that has only pluses;
denote it by $a_0 a_1 \dots a_n$;
so $a=a_0$ and $b=a_n$.
Denote by $a'_0 a'_1 \dots a'_n$
the corresponding line in $P'$;
let $a'=a'_0$ and $b'=a'_n$.

\begin{wrapfigure}{r}{40mm}
\vskip-0mm
\centering
\includegraphics{mppics/pic-40}
\vskip-0mm
\end{wrapfigure}

Since each marked edge passing thru $a_i$ has a $(+)$ on it or nothing, 
we have 
$$\measuredangle a_i\le\measuredangle a'_i$$
for each $i$.

\begin{thm}{Exercise}\label{ex:a<a}
Prove the last statement. 
\end{thm}

By the construction we have $|a_i-a_{i-1}|=|a'_i-a'_{i-1}|$ for all $i$.
By the arm lemma (\ref{lem:arm}), 
we get 
\begin{clm}{}\label{clm:ab<ab}
$|a-b|\le |a'-b'|$.
\end{clm}
Swap $K$ and $K'$ and repeat the same construction for a plane passing thru $a'$ and $b'$.
We get
\begin{clm}{}\label{clm:ab>ab}
$|a-b|\ge |a'-b'|$.
\end{clm}

The claims 
\ref{clm:ab<ab} and \ref{clm:ab>ab} 
together imply $|a-b|=|a'-b'|$.
The equality case in the arm lemma implies that no edge at $v$ is marked;
that is, $v$ is not a vertex of $\Gamma$
--- a contradiction.
\qeds

From the proof, it follows that the local lemma is indeed local --- it works for two nonconguent convex polyhedral angles with equal corresponding faces.
Use this observation to solve the following exercise.

\begin{wrapfigure}{r}{20mm}
\vskip-0mm
\centering
\includegraphics{mppics/pic-10}
\bigskip
\includegraphics{mppics/pic-20}
\vskip-0mm
\end{wrapfigure}

\begin{thm}{Exercise}\label{ex:disc-bend}
Consider two polyhedral discs in $\EE^3$ glued from regular polygons by the rule on the diagrams.
Assume that each disc is part of a surface of a convex polyhedron.

\begin{subthm}{}
The first configuration is rigid; that is, one can not fix the position of the pentagon and continuously move the remaining 5 vertices in a new position so that each triangle moves by a one-parameter family of isometries of $\EE^3$.
\end{subthm}

\begin{subthm}{}
Show that the second configuration has a rotational symmetry with the axis passing thru the midpoint of the marked edge.
\end{subthm}

\end{thm}

\section{Global lemma}

The proof of the global lemma is based on counting the sign changes 
in two ways;
first while moving around each vertex of $\Gamma$ 
and second while moving around each of the regions separated by $\Gamma$
on the surface~$P$. 
If two edges are adjacent at a vertex,
then they are also adjacent in a region. 
The converse is true as well. 
Therefore, both countings give the same number.

It is instructive to do the next exercise before diving into the proof.

\begin{thm}{Exercise}\label{ex:octahedron}
Try to mark the edges of an octahedron
by pluses and minuses
such that there would be 4 sign changes at each vertex.

Show that this is impossible.
\end{thm}

\parit{Proof of \ref{lem:global}.}
We can assume that $\Gamma$ is connected;
that is, one can get from any vertex to any other vertex by walking along edges.
(If not, pass to a connected component of $\Gamma$.)

Denote by $k$ and $l$ the number of vertices and edges in $\Gamma$.
Denote by $m$ the number of \textit{regions} that $\Gamma$ cuts from $P$.
Since $\Gamma$ is connected, each region is homeomorphic to an open disc.

\begin{thm}{Exercise}\label{ex:disc}
Prove the last statement.
\end{thm}

Now we can apply Euler's formula
$$k-l+m=2.
\eqlbl{eq:cauchy:euler}$$

Denote by $s$ the total number of sign changes in $\Gamma$ for all vertices. 
By the local lemma (\ref{lem:local}), we have 
$$ 4\cdot k\le s.\eqlbl{eq:S>=4k}$$

Let us get an upper bound on $s$ by counting the number of sign changes when you go around
each region. 
Denote by $m_n$ the number of regions bounded by $n$ edges;
if an edge appears twice when it is counted twice.
Note that each region is bounded by at least $3$ edges;
therefore
$$m=m_3+m_4+m_5+\dots\eqlbl{eq:3-4-5}$$
Counting edges and using the fact that each edge belongs to exactly two regions, we get
$$2\cdot l=3\cdot m_3+ 4\cdot m_4+5\cdot m_5+\dots$$
Combining this with Euler's formula (\ref{eq:cauchy:euler}), we get
$$4\cdot k=8+2\cdot m_3+4\cdot m_4+6\cdot m_5+8\cdot m_6+\dots
\eqlbl{eq:k=2+}$$
Observe that the number of sign changes in $n$-gon regions has to be even and $\le n$.
Therefore
$$s \le 2\cdot m_3 + 4\cdot m_4 + 4\cdot m_5 + 6\cdot m_6+\dots
\eqlbl{eq:23-44-45}$$
Clearly, \ref{eq:S>=4k} and \ref{eq:23-44-45} contradict \ref{eq:k=2+}.
\qeds


\section{Uniqueness}

Alexandrov's uniqueness theorem states that the conclusion of the Cauchy theorem (\ref{thm:cauchy}) still holds without the face-to-face assumption.

\begin{thm}{Theorem}\label{thm:alexandrov-uni'}
Any two convex polyhedrons in $\EE^3$ with isometric surfaces are congruent.

Moreover, any isometry between surfaces of convex polyhedrons can be extended to an isometry of the whole $\EE^3$. 
\end{thm}

\parit{Needed modifications in the proof of \ref{thm:cauchy}.}
Suppose $\iota\:P\to P'$ be an isometry between surfaces of $K$ and $K'$.
Mark in $P$ all the edges of $K$ and all the inverse images of edges in $K'$; further, these will be called \index{fake edge}\emph{fake} edges.
The marked lines divide $P$ into convex polygons, and the restriction of $\iota$ to each polygon is a rigid motion.
These polygons play the role of faces in the proof above.

A vertex of the obtained graph can be a vertex of $K$, or it can be a fake vertex;
that is, it might be an intersection of an edge and a fake edge.

\begin{figure}[ht!]
\vskip-0mm
\centering
\includegraphics{mppics/pic-50}
\vskip-0mm
\end{figure}

For the first type of vertex, the local lemma can be proved the same way. 
For a fake vertex $v$, it is easy to see that both parts of the edge coming thru $v$ are marked with minus
while both of the fake edges at $v$ are marked with plus.
Therefore, the local lemma holds for the fake vertices as well.

What remains in the proof needs no modifications.
\qeds

\begin{thm}{Exercise}\label{pr:K-P-simmetry}
Let $K$ be a convex polyhedron in $\EE^3$;
denote by $P$ its surface.
Show that each isometry $\iota\:P\z\to P$,
can be extended to an isometry of $\EE^3$. 
\end{thm}

\section{Existence}\label{sec:Alexandrov-existence}

\begin{thm}{Theorem}\label{thm:exist}
A polyhedral metric on the sphere is isometric to the surface of a convex polyhedron (possibly degenerate to a flat polygon) if and only if it has nonnegative curvature at each point.
\end{thm}

\begin{wrapfigure}{r}{30mm}
\vskip-5mm
\centering
\includegraphics{mppics/pic-1010}
\vskip-0mm
\end{wrapfigure}

By \ref{thm:alexandrov-uni'}, a convex polyhedron is completely defined by the intrinsic metric of its surface.
By \ref{thm:exist}, it follows that knowing the metric we could find the position of the edges.
However, in practice, it is not easy to do.

For example, the surface glued from a rectangle as shown on the diagram defines a tetrahedron.
Some of the glued lines appear inside facets of the tetrahedron and some edges (dashed lines) do not follow the sides of the rectangle.

\paragraph{Space of polyhedrons.}
Let us denote by $\mathbf{K}$ the space of all convex polyhedrons in the Euclidean space,
including polyhedrons that degenerate to a plane polygon.
Polyhedra in $\mathbf{K}$ will be considered up to a motion of the space, 
and the whole space $\mathbf{K}$ will be considered with Hausdorff distance up to a motion of the space;
that is, the distance between $K$ and $K'$ is the exact lower bound on Hausdorff distance from $\iota(K)$ to $K'$, where $\iota$ is arbitrary motion of $\EE^3$.

Further, denote by $\mathbf{K}_n$ the polyhedrons in $\mathbf{K}$ with exactly $n$ vertices.
Since any polyhedron has at least 3 vertices, the space $\mathbf{K}$ admits a subdivision into a countable number of subsets $\mathbf{K}_3,\mathbf{K}_4,\dots$

\paragraph{Space of polyhedral metrics.}
The space of polyhedral metrics on the sphere with nonnegative curvature will be denoted by $\mathbf{P}$.
The metrics in $\mathbf{P}$ will be considered up to an isometry, and the whole space $\mathbf{P}$ will be equipped with the topology induced by the Gromov--Hausdorff metric.

The subset of $\mathbf{P}$ of all metrics with exactly $n$ essential vertices will be denoted by $\mathbf{P}_n$.
It is easy to see that any metric in $\mathbf{P}$ has at least 3 essential vertices.
Therefore $\mathbf{P}$ is subdivided into countably many subsets
 $\mathbf{P}_3,\mathbf{P}_4,\dots$

\paragraph{From a polyhedron to its surface.}

By \ref{prop:poly-CBB}, passing from a polyhedron to its surface defines a map
\[\iota\:\mathbf{K}\to \mathbf{P}.\]

By \ref{ex:vertex-essential-vertex}, the number of vertices of a polyhedron is equal to the number of essential vertices on its surface.
In other words, $\iota(\mathbf{K}_n)\subset \mathbf{P}_n$ for any $n\ge 3$.

Using the introduced notation, we can unite \ref{thm:alexandrov-uni'} and \ref{thm:exist} in the following more exact statement.

\begin{thm}{Reformulation}
For any integer $n\ge 3$,
the map $\iota$ induces a bijection between $\mathbf{K}_n$ and~$\mathbf{P}_n$.
\end{thm}

The proof is based on a construction of a one-parameter family of polyhedrons that starts at an arbitrary polyhedron
and ends at a polyhedron with its surface isometric to the given one.
This type of argument is called the \textit{continuity method}; it is often used in the theory of differential equations.


\parit{Sketch.}
By \ref{thm:alexandrov-uni'}, the map $\iota\:\mathbf{K}_n\to\mathbf{P}_n$ is injective.
Let us prove that it is surjective.

\begin{thm}{Lemma}
For any integer $n\ge 3$, the space $\mathbf{P}_n$ is connected.
\end{thm}

The proof of this lemma is not complicated, but it requires ingenuity;
it can be done by the direct construction of a one-parameter family of metrics in $\mathbf{P}_n$ that connects two given metrics.
Such a family can be obtained by а sequential application of the following construction and its inverse.

Let $P\in\mathbf{P}_n$.
Suppose $v$ and $w$ are essential vertices in $P$.
Let us cut $P$ along a geodesic from $v$ to $w$.
Note that the geodesic cannot pass thru an essential vertex of $P$.
Further, note that there is a three-parameter family of patches that can be used to patch the cut so that the obtained metric remains in $\mathbf{P}_n$;
in particular, the obtained metric has exactly $n$ essential vertices (after the patching, the vertices $v$ and $w$ may become inessential).


\begin{thm}{Lemma}
The map $\iota\:\mathbf{K}_n\to\mathbf{P}_n$ is open, 
that is, it maps any open set in $\mathbf{K}_n$ to an open set in $\mathbf{P}_n$.

In particular, for any $n\ge 3$, the image $\iota(\mathbf{K}_n)$ is open in~$\mathbf{P}_n$.
\end{thm}

This statement is very close to the so-called \textit{invariance of domain theorem};
the latter states that a continuous injective map between manifolds of the same dimension is open.

Recall that $\iota$ is injective.
The proof of the invariance of domain theorem can be adapted to our case since both spaces $\mathbf{K}_n$ and $\mathbf{P}_n$ are $(3\cdot n-6)$-dimensional and both look like manifolds, altho, formally speaking, they are \textit{not} manifolds.
In a more technical language, $\mathbf{K}_n$ and $\mathbf{P}_n$ have the natural structure of $(3\cdot n-6)$-dimensional \textit{orbifolds},
and the map $\iota$ respects the \textit{orbifold structure}.

We will only show that both spaces $\mathbf{K}_n$ and $\mathbf{P}_n$ are $(3\cdot n-6)$-dimensional.

Choose  $K\in\mathbf{K}_n$.
Note that $K$ is uniquely determined by the $3\cdot n$ coordinates of its $n$ vertices.
We can assume that the first vertex is the origin, the second has two vanishing coordinates and the third has one vanishing coordinate; therefore, all polyhedrons in $\mathbf{K}_n$ that lie sufficiently close to $K$ can be described by $3\cdot n-6$ parameters.
If $K$ has no symmetries, then this description can be made one-to-one;
in this case, a neighborhood of $K$ in $\mathbf{K}_n$ is a $(3\cdot n-6)$-dimensional manifold.
If $K$ has a nontrivial symmetry group, then this description is not one-to-one but it does not have an impact on the dimension of~$\mathbf{K}_n$.

The case of polyhedral metrics is analogous.
We need to construct a subdivision of the sphere into plane triangles using only essential vertices.
By Euler's formula, there are exactly $3\cdot n-6$ edges in this subdivision.
Note that the lengths of edges completely describe the metric, and slight changes in these lengths produce a metric with the same property.
Again, if $P$ has no symmetries, then this description is one-to-one.

\begin{thm}{Lemma}
The map $\iota\:\mathbf{K}_n\to\mathbf{P}_n$ is closed;
that is, the image of a closed set in $\mathbf{K}_n$ is closed in $\mathbf{P}_n$.

In particular, for any $n\ge 3$, the set $\iota(\mathbf{K}_n)$ is closed in~$\mathbf{P}_n$.
\end{thm}

Choose a closed set $Z$ in $\mathbf{K}_n$.
Denote by $\bar Z$ the closure of $Z$ in $\mathbf{K}$; note that $Z=\mathbf{K}_n\cap \bar Z$.
Assume $K_1,K_2,\dots\in Z$ is a sequence of polyhedrons that converges to a polyhedron $K_\infty\in\bar Z$.
By \ref{lem:H>GH}, $\iota(K_n)$ converges to $\iota(K_\infty)$ in $\mathbf{P}$.
In particular, $\iota(\bar Z)$ is closed in $\mathbf{P}$.

Since $\iota(\mathbf{K}_n)\subset \mathbf{P}_n$ for any $n\ge 3$, we have $\iota (Z)=\iota(\bar Z)\cap \mathbf{P}_n$;
that is, $\iota (Z)$ is closed in $\mathbf{P}_n$. 

\medskip

Summarizing, $\iota(\mathbf{K}_n)$ is a nonempty closed and open set in $\mathbf{P}_n$, and $\mathbf{P}_n$ is connected for any $n\ge 3$.
Therefore, $\iota(\mathbf{K}_n)=\mathbf{P}_n$; that is, $\iota\:\mathbf{K}_n\z\to\mathbf{P}_n$ is surjective.
\qeds

\section{Approximation}

By now, the embedding theorem is proved for polyhedral metrics on the sphere.
The general case is done by approximation, using the following statement.

\begin{thm}{Proposition}\label{prop:H>GH}
Let $K_1,K_2,\dots$ be a sequence of convex bodies that converge to $K_\infty$ in the sense of Hausdorff.
Then the surface of $K_n$ converges to the surface of $K_\infty$ in the sense of Gromov--Hausdorff.
\end{thm}

If $K_\infty$ is nondegenerate, then the statement follows from \ref{lem:H>GH}.
The degenerate case is left as an exercise.

Let $\spc{X}_\infty$ be an $\Alex0$ space that is homeomorphic to $\SSS^2$.
Suppose that $\spc{X}_\infty$ is a Gromov--Hausdorff limit of a sequence of spheres with polyhedral metrics $\spc{X}_1,\spc{X}_2,\dots$
By \ref{thm:exist}, there is a sequence of convex polyhedra $K_1,K_2,\dots$ with surfaces isometric to $\spc{X}_1,\spc{X}_2,\dots$, respectively.
Note that  $\diam K_n\le \diam \spc{X}_n$ for any $n$.
Therefore we can assume that all polyhedra $K_1,K_2,\dots$ lie in a closed ball of sufficiently large radius.

Applying Blaschke selection theorem, we can pass to a subsequence of $K_1,K_2,\dots$ that converges in the sense of Hausdorff; denote its limit by $K_\infty$.
By \ref{prop:H>GH} the surface of $K_\infty$ is isometric to $\spc{X}_\infty$.

Therefore it remains to prove the following lemma.

\begin{thm}{Lemma}\label{lem:GH-approximation}
Let $\spc{X}$ be an $\Alex0$ space that is homeomorphic to $\SSS^2$.
Then there is a sphere with polyhedral metrics $\spc{X}'$
that is arbitrarily close to $\spc{X}$ in the sense of Gromov--Hausdorff.
\end{thm}

\parit{Idea behind the proof.}
Suppose we can triangulate $\spc{X}_\infty$ by small geodesic triangles;
that is, we can choose a finite set of points $p_1,\dots,p_n\z\in \spc{X}_\infty$ and some geodesics $[p_ip_j]$ that cut $\spc{X}_\infty$ into regions of small diameter bounded by geodesic triangles $[p_ip_jp_k]$.
(The actual proof constructs a triangulation with a weaker property.)

Observe that total angle around each $p_i$ cannot exceed $2\cdot \pi$.
That is, suppose $p_{j_1},\dots,p_{j_k}$ are points connected to $p_i$ by geodesics.
Assume that they are ordered in the natural cyclic order.
Then 
\[\mangle\hinge{p_i}{p_{j_1}}{p_{j_2}}+\dots+\mangle\hinge{p_i}{p_{j_{k-1}}}{p_{j_k}}+\mangle\hinge{p_i}{p_{j_{k}}}{p_{j_1}}\le 2\cdot\pi.\]
By comparison, we get
\[\angk{p_i}{p_{j_1}}{p_{j_2}}+\dots+
\angk{p_i}{p_{j_{k-1}}}{p_{j_k}}+\angk{p_i}{p_{j_{k}}}{p_{j_1}}\le 2\cdot\pi.\eqlbl{eq:sum<=<2pi}\]

Now let us exchange each triangle by its model triangle.
That is, consider a model triangle for each region in the subdivision of $\spc{X}$ and glue them together by the same rule.
By \ref{eq:sum<=<2pi}, the obtained polyhedral surface $\spc{X}'$ has nonnegative curvature.
It remains to show that this way we can produce $\spc{X}'$ arbitrarily close to $\spc{X}$.

Denote by $p_i\to p_i'$ the natural map; it takes $p_i$ in $\spc{X}$ and returns the corresponding point in $\spc{X}'$.
Observe that 
\[\dist{p_i'}{p_j'}{\spc{X}'}
\le
\dist{p_i}{p_j}{\spc{X}}.\eqlbl{eq:|pp|}\]
Indeed, choose a geodesic $\gamma$ from $p_i$ to $p_j$.
Let $p_i=x_0,x_1,\dots,x_n=p_j$ be the points of intersections of $\gamma$ with the edges of the triangulation listed as they appear on $\gamma$.
For each $i$, denote by $x_i'$ the corresponding point in $\spc{X}'$.
By comparison, we get 
\[\dist{x_k'}{x_{k-1}'}{\spc{X}'}
\le
\dist{x_k}{x_{k-1}}{\spc{X}}.\]
for each $k$.
Therefore, \ref{eq:|pp|} follows.

Suppose $\eps>0$ is small, the points $p_1,\dots,p_n$ form an $\eps$-net in $\spc{X}$, all edges of the triangulation are smaller than $\eps$ and
\[\dist{p_i'}{p_j'}{\spc{X}'}
\ge
\dist{p_i}{p_j}{\spc{X}} -100\cdot \eps.\eqlbl{eq:|pp|>=}\]
Then, together with the inequality above it proves that the lemma.

Note that the sides of the model triangles are local geodesics in $\spc{X}'$,
but not necessarily geodesic; that is they do not have to be length-minimizing.
Now, let us make another unjustified assumption:
\textit{Suppose that the sides of model triangles in $\spc{X}'$ are geodesics.}
(The actual proof does not use this assumption.)

Choose a geodesic $\gamma'$ from $p_i'$ to $p_j'$ in $\spc{X}'$.
Note that $\gamma'$ visits each triangle in the triangulation of $\spc{X}'$ at most once.

Let $p_i'=x_0',x_1',\dots,x_n'\z=p_j'$ be the points of intersections of $\gamma'$ with the edges of the triangulation listed from $p_i'$ to $p_j'$.
For each $i$, denote by $x_i$ the corresponding point in $\spc{X}$.
Let $\Delta_k'$ be the triangle that contains arc $[x'_{k-1}x'_k]$ of $\gamma'$ and $\Delta_k$ the corresponding triangle in $\spc{X}$.
Note that
\[\dist{x_k'}{x_{k-1}'}{\spc{X}'}
\ge
\dist{x_k}{x_{k-1}}{\spc{X}} -\eps\cdot K(\Delta_k),
\eqlbl{eq:|xx|<}\]
where $K(\Delta_k)$ denotes the access of $\Delta_k$;
that is, the sum of its internal angles minus $\pi$.

Euler's formula and \ref{eq:sum<=<2pi} imply that the sum of all accesses is at most $4\cdot\pi$.
Therefore, summing up \ref{eq:|xx|<}, we get 
\[\dist{p_i'}{p_j'}{\spc{X}'}
\ge
\dist{p_i}{p_j}{\spc{X}}-4\cdot \pi\cdot \eps.\]
Whence \ref{eq:|pp|>=} follows.
\qeds


\section{Comments}

This lecture contains selected material from Alexandrov's book~\cite{alexandrov}.

In Euclid's Elements, 
solids were called equal if the same holds for their faces, but no proof was given.
Adrien-Marie Legendre became interested in this problem towards the end of the 18th century.
He discussed it with his colleague Joseph-Louis Lagrange, who suggested this problem to Augustin-Louis Cauchy in 1813; soon he proved it \cite{cauchy}.
This theorem is included in many popular books \cite{aigner-zigler,dolbilin,tabacnikov-fuks}.

The observation that the face-to-face condition can be removed was made by 
Alexandr Alexandrov \cite{alexandrov-1941}.

\parit{Arm lemma.}
Original Cauchy's proof \cite{cauchy}
also used a version of the arm lemma, but its proof contained a small mistake (corrected in one century).

Our proof of the arm lemma is due to Stanisław Zaremba.
This and a couple of other proofs can be found in the letters between him and Isaac Schoenberg \cite{schoenberg-zaremba}.

The following variation of the arm lemma makes sense for nonconvex spherical polygons.
It is due to Viktor Zalgaller \cite{zalgaller}.
It can be used instead of the standard arm lemma.

\begin{thm}{Another arm lemma}
Let $A=[a_1\dots a_n]$ and $A'\z=[a'_1\z\dots a'_n]$ be two spherical $n$-gons (not necessarily convex).
Assume that $A$ lies in a half-sphere,
the corresponding sides of $A$ and $A'$ are equal
and each angle of $A$ is at least the corresponding angle in $A'$.
Then $A$ is congruent to~$A'$. 
\end{thm}

\parit{Global lemma.}
A more visual proof of the global lemma is given in \cite[II \S 1.3]{alexandrov}.
This argument was reused by Anton Klyachko \cite{klyachko} in his \index{car-crash lemma}\emph{car-crash lemma}.

\parit{Existence theorem.}
This theorem was proved by Alexandr Alexandrov~\cite{alexandrov-1941}.
Our sketch is taken from \cite{lebedeva-petrunin};
a complete proof is nicely written in~\cite{alexandrov}.
In the original proof, the spaces $\mathbf{K}_n$ and $\mathbf{P}_n$ were modified so the they become $(3\cdot n-6)$-dimensional manifolds.
It was done by introducing extra structure (for $\mathbf{K}_n$ it is orientation + a marked vertex and an edge) that \textit{brakes symmetries} of the spaces.
After that one could apply the domain invariance theorem directly.
Alternatively, one may first remove from $\mathbf{K}_n$ and $\mathbf{P}_n$ elements (polyhedron or surface)with nontrivial symmetries (after that the spaces become manifolds) and show that any symmetric polyhedron (or surface) can be approximated by a non-symmetric polyhedron (or surface).

A very different proof was found by Yuri Volkov in his thesis \cite{volkov};
it uses a deformation of three-dimensional polyhedral space.

%P and \Sigma???
