%%!TEX root = the-volume.tex

\chapter{Limit spaces}\label{chap:lim}


\section{Survival of curvature bounds}

\begin{thm}{Theorem}\label{thm:CBB-closed}
Let $\spc{X}_n\z\to \spc{X}_\infty$ be a convergence in the sense of Gromov--Hausdorff.
Suppose that each for each $n$, the space $\spc{X}_n$ has curvature $\ge\kappa$ in the sense of Alexandrov.
Then the same holds for~$\spc{X}_\infty$.
\end{thm}

\parit{Proof}.
Choose a quadruple of points $p_\infty, x_\infty,y_\infty,z_\infty\in \spc{X}_\infty$.

By the definition of Gromov--Hausdorff convergence, we can choose a points $p_n$,  $x_n$, $y_n$, $z_n\in \spc{X}_n$ for each $n$ that converge to $p_\infty$, $x_\infty$, $y_\infty$, $z_\infty\in \spc{X}_\infty$, respectively.
In particular, each of 6 distances between pairs of $p_n$, $x_n$, $y_n$, $z_n$ converge to the distance between the corresponding pairs of $p_\infty, x_\infty,y_\infty,z_\infty$.

Since $\MM^2(\kappa)$-comparison holds for $p_n$, $x_n$, $y_n$, $z_n\z\in \spc{X}_n$,
passing to the limit, we get the $\MM^2(\kappa)$-comparison for $p_\infty$,  $x_\infty$, $y_\infty$, $z_\infty$.
\qeds

\begin{thm}{Exercise}\label{ex:dim-lim}
Suppose that a sequence $\spc{L}_1,\spc{L}_2,\dots$ of $\Alex\kappa$ spaces that converges to $\spc{L}_\infty$ in the sense of Gromov--Hausdorff.
Show that $\spc{L}_\infty$ is $\Alex\kappa$ and
\[\dim \spc{L}_\infty\le \liminf_{n\to\infty} \dim \spc{L}_n.\]
\end{thm}


\section{Volume}

Fix a positive integer $m$.
The $m$-dimensional Hausdorff measure of a Borel set $B$ in a metric space will be called its \index{volume}\emph{$m$-volume}; it will be denoted by $\vol_m B$.
We assume that the Hausdorff measure is calibrated so that the unit cube in $\EE^m$ has unit volume.

This definition will be applied mostly to subsets in $m$-dimensional Alexandrov spaces.
In this case, we may write $\vol B$ instead of $\vol_m B$.


\begin{thm}{Bishop--Gromov inequality}\label{inq:BG}
Let $\spc{L}$ be an $\Alex0$ space.
Suppose $\dim \spc{L}=m<\infty$.
Then 
\[\vol \oBall(p,r)\le \omega_m\cdot r^m,\]
where $\omega_m$ denotes the volume of the unit ball in $\EE^m$.
Moreover, the function 
\[r\mapsto \frac{\vol B(p,r)}{r^m}\]
is nonincreasing.
\end{thm}

\parit{Proof.}
Given $x\in\spc{L}$ choose a geodesic path $\gamma_x$ from $p$ to $x$.
Recall that $\log_p\:\spc{L}\to \T_p$ can be defined by $\log_p\:x\mapsto \gamma_x^+(0)$.
By comparison, $\log_p$ is distance-noncontracting.
Note that $\log_p$ maps $\oBall(p,r)_{\spc{L}}$ to $\oBall(0,r)_{\T_p}$.

\begin{wrapfigure}{r}{44 mm}
\vskip-0mm
\centering
\includegraphics{mppics/pic-803}
\vskip1mm
\end{wrapfigure}

If $\T_p\iso \EE^m$, then $\vol\oBall(0,r)_{\T_p}\z=\omega_m\cdot r^m$,
and the first statement follows.

If $\T_p$ is not isometric to $\EE^m$, then by \ref{ex:tangent=Em} we can find a point $p'$ arbitrarily close to $p$ such that $\T_{p'}\iso \EE^m$.
If $\eps>\dist{p}{p'}{}$, then $\oBall(p,r)\subset \oBall(p',r+\eps)$.
Therefore,
\[\vol \oBall(p,r)\le \omega_m\cdot (r+\eps)^m\]
for any $\eps>0$.
Hence the first statement follows.

Now, suppose $0<r_1<r_2$.
Consider the map $w\: \spc{L}\to \spc{L}$ defined by $w\:x\mapsto \gamma_x(\tfrac {r_1}{r_2})$.
(The map $w$ mimics the dilation with center at $p$ and coefficient $\tfrac {r_1}{r_2}$.)
By comparison,
\[\dist{w(x)}{w(y)}{}\ge \tfrac {r_1}{r_2}\cdot \dist{x}{y}{}.\]
Observe that $\oBall(p,r_1) \supset w[\oBall(p,r_2)]$.
Therefore, 
\[\vol \oBall(p,r_1)\ge (\tfrac {r_1}{r_2})^m\cdot\vol \oBall(p,r_2).\]
\qedsf

The following exercise generalizes the Bishop--Gromov inequality to $\Alex{-1}$ case. 
This statement is sufficient for most applications, but a more exact statement is given in \ref{inq:BG+} which also includes the case of  $\Alex{1}$ spaces.

\begin{thm}{Exersice}\label{ex:BG}
Let $\spc{L}$ be an $\Alex{-1}$ space.
Suppose $\spc{L}=m<\infty$.
Show that
\[\vol \oBall(p,r)\le \omega_m\cdot(\sinh r)^m,\]
where $\omega_m$ denotes the volume of the unit ball in $\EE^m$.
Moreover, the function 
\[r\mapsto \frac{\vol B(p,r)}{(\sinh r)^m}\]
is nonincreasing.
\end{thm}

\section{Gromov's selection theorem}

\begin{thm}{Theorem}\label{thm:gromov-compactness}
Let $D,\kappa\in\RR$, and $m$ be a positive integer. 
Then any sequence of $m$-dimensional $\Alex\kappa$ spaces with diameters at most $D$
has a converging subsequence in the sense of Gromov--Hausdorff.
\end{thm}

Let $X$ be a subset of a metric space $\spc{W}$.
Recall that a set $Z\subset \spc{W}$ is called \index{$\eps$-net}\emph{$\eps$-net} of $X$ if for any point $x\in X$, there is a point $z\in Z$ such that $\dist{x}{z}{}<\eps$.

We will use the following characterization of compact sets.

\begin{thm}{Exercise}\label{ex:net}
A closed subset $X$ of a complete metric space.

\begin{subthm}{ex:net:finite}
Show that $X$ is compact if and only if it admits a finite $\eps$-net for any $\eps>0$.
\end{subthm}

\begin{subthm}{ex:net:compact}
Show that $X$ is compact if and only if it admits a compact $\eps$-net for any $\eps>0$.
\end{subthm}

\end{thm}

Recall that $\pack_\eps\spc{X}$ is the exact upper bound on the number of points $x_1,\z\dots,x_n\in \spc{X}$ such that $\dist{x_i}{x_j}{}\ge\eps$ if $i\ne j$.

If $n=\pack_\eps\spc{X}<\infty$, then
the collection of points $x_1,\dots,x_n$ is called a \index{maximal packing}\emph{maximal $\eps$-packing}.

\begin{thm}{Exercise}\label{ex:pack-net}
Show that any maximal $\eps$-packing $x_1,\dots,x_n$ is an $\eps$-net.
Conclude that a complete metric space $\spc{X}$ is compact if and only if $\pack_\eps\spc{X}\z<\infty$ for any $\eps>0$.
\end{thm}


\parit{Proof of \ref{thm:gromov-compactness}.}
Denote by $\bm{K}$ the set of isometry classes of $\Alex0$ spaces with dimension $\le m$ and diameter $\le D$.
By \ref{ex:dim-lim}, $\bm{K}$ is a closed subset of $\GH$.

Choose a space $\spc{L}\in \bm{K}$;
suppose $x_1,\dots,x_n\in \spc{L}$ is a collection of points such that $\dist{x_i}{x_j}{}\ge \eps$ for all $i\ne j$.
Note that the balls $B_i=\oBall(x_i,\tfrac\eps2)$ do not overlap.

By \ref{thm:right-inverse}, $\vol \spc{L}>0$.
By Bishop--Gromov inequality, $\vol \spc{L}<\infty$,
and if $\eps<D$, then 
\[\vol B_i\ge (\tfrac\eps{2\cdot D})^m\cdot\vol \spc{L}\]
for any $i$.
It follows that $n\le (\tfrac{2\cdot D}\eps)^m$;
that is, 
\[\pack_\eps\spc{L}\le  N(\eps)\df(\tfrac{2\cdot D}\eps)^m\]
for all small $\eps>0$.

Choose a maximal $\eps$-packing $x_1,\z\dots,x_n\in \spc{L}$.
By \ref{ex:pack-net}, $\spc{F}_\eps\z\df\{x_1,\z\dots,x_n\}$ is an $\eps$-net of $\spc{L}$.
Observe that $\dist{\spc{F}_\eps}{\spc{L}}{\GH}\le \eps$.
Further, note that the set $\bm{F}_\eps$ of finite metric spaces with diameter $\le D$ and at most $N(\eps)$ points forms a compact subset in $\GH$.

Summarizing, for any $\eps>0$ we can find a compact $\eps$-net $\bm{F}_\eps\subset \GH$ of $\bm{K}$.
Since $\GH$ is complete (\ref{prop:complete}), it remains to apply \ref{ex:net:compact}.

Note that rescaling reduces $\Alex\kappa$ case to the $\Alex{-1}$ case.
The latter can be proved the same way, using \ref{ex:BG} instead of \ref{inq:BG}.
\qeds

\begin{thm}{Exercise}\label{ex:pack-vol}
Let $\spc{L}$ be an $m$-dimensional $\Alex0$ space with diameter $\le D$.
Suppose $\vol\spc{L}\ge v_0>0$.
Show that 
\[\pack_\eps\spc{L}\ge \frac\Const{\eps^m}\]
for some constant $\Const=\Const(m,D,v_0)>0$.

Conclude that if $\spc{L}_n$ is a sequence of $m$-dimensional $\Alex0$ spaces with diameter $\le D$, and volume $\ge v_0$, then its Gromov--Hausdorff limit $\spc{L}_\infty$ (if it exists) has dimension~$m$.%
\end{thm}

\begin{thm}{Exercise}\label{ex:diam-compact}
Show that any finite-dimensional Alexandrov space is proper.

Let $(\spc{L}_1,p_1),(\spc{L}_2,p_2),\dots$ be a sequence of $m$-dimensional $\Alex\kappa$ spcaes with marked points.
Show that it contains a subsequence pointed-converging in the sense of Gromov--Hausdorff. 
\end{thm}

\section{Comments}

Let us state a version of Bishop--Gromov inequality for $\Alex\kappa$ spaces.
Its proof requires additionally the so-called \textit{coarea formula} for Alexandrov spaces. 
The weaker inequality from \ref{ex:BG} is sufficient for the sequel.

\begin{thm}{Bishop--Gromov inequality}\label{inq:BG+}
Let $p$ be a point in an $m$-dimensional $\Alex\kappa$ space.
Consider the function $v(r)\z=\vol_m\oBall(p,r)$;
denote by $\tilde v(r)$ the volume of $r$ ball in $\MM(\kappa)$.
Then 
\[v(r)\le \tilde v(r)\]
for $r>0$ and the function 
\[r\mapsto \frac{v(r)}{\tilde v(r)}\] is nonincreasing.
If $\kappa>0$, then one has to assume that $r<\tfrac\pi{\sqrt\kappa}$.
\end{thm}

This inequality was originally proved for Riemannian manifolds with lower Ricci curvature.
The first part is also called \emph{Bishop's inequality}.
It is due to Richard Bishop; see \cite{bishop1964} and \cite[Corollary 4, p. 256]{bishop-crittenden}.
The second part is due to Michael Gromov \cite{gromov1981}.

Gromov's selection theorem provides the main source of applications of Alexandrov spaces to Riemannian geometry.
The homotopy-type finiteness theorem (\ref{thm:h-finiteness}) in the next lecture illustrates this technique.

A version Gromov's selection theorem holds for $m$-dimensional Riemannian manifolds with a lower bound on Ricci curvature.
It motivates the study of the so-called $\mathrm{CD}(K,m)$ spaces; $\mathrm{CD}$ stands for curvature-dimension condition.
This theory has serious applications in Alexandrov geometry;
in particular, it provides a version of Liouville theorem about phase-space volume of geodesic flow in Alexandrov space \cite{brue-mondino-semola}.
