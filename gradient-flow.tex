%%!TEX root = the-gradient-flow.tex

\chapter{Gradient flow}\label{chap:GF}

Here we define the gradient flow and prove the distance estimates.



\section{Velocity of curve}\label{Velocity of curve}

Let $\alpha$ be a curve in an Alexandrov space $\spc{A}$.
If for any choice of 
geodesics $[p\,\alpha(t_0+\eps)]$ the vectors 
\[\tfrac{1}{\eps}\cdot\dist{p}{\alpha(t_0+\eps)}{}\cdot\dir p{\alpha(t_0+\eps)}\]
converge as $\eps\to 0+$, then their limit in $\T_p$ is called the \index{right derivative}\emph{right derivative} of $\alpha$ at $t_0$; it will be denoted by $\alpha^+(t_0)$.
In addition, we assume that $\alpha^+(t_0)\df0$
if $\tfrac{1}{\eps}\cdot\dist{p}{\alpha(t_0+\eps)}{}\to 0$ as $\eps\to 0+$.

The tangent vector $v=\dist px{}\cdot\dir px$ will be called the \index{logarithm}\emph{logarithm} of $x$ at $p$ (briefly, \index{40@$\log_p x$ (logarithm)}$v=\log_p x$).
The logarithm is a multivalued function from $\spc{A}$ to $\T_p$; so, $v=\log_p x$ and $w=\log_p x$ does \textit{not} mean $v=w$.
Note that $\gamma^+(0)=\log_px$ for any geodesic path $\gamma$ from $p$ to $x$.\label{page:log}

\begin{thm}{Claim}\label{clm:fa'=dfa'}
Let $\alpha$ be a curve in an Alexandrov space $\spc{A}$.
Suppose $f$ is a semiconcave Lipschitz function
defined in a neighborhood of $p\z=\alpha(0)$,
and $\alpha^+(0)$ is defined.
Then $(f\circ\alpha)^+(0)$ exists and 
\[(f\circ\alpha)^+(0)
=
(\dd_pf)(\alpha^+(0)).\]

\end{thm}

\parit{Proof.}
Without loss of generality, we can assume that $f(p)=0$.
Suppose $f$ and therefore $\dd_pf$ are $L$-Lipschitz.

Let $\gamma$ be a geodesic with a constant-speed reparametrization that starts from $p$, and
such that the distance
$s=\dist{\alpha^+(0)}{\gamma^+(0)}{\T_p}$
is small.
By the definition of differential,
\[(f\circ\gamma)^+(0)=\dd_pf(\gamma^+(0)).\]

By comparison and the definition of $\alpha^+$,
\[\dist{\alpha(\eps)}{\gamma(\eps)}{\spc{A}}\le s\cdot\eps+o(\eps)\]
for $\eps>0$.
Therefore,
\[|f\circ\alpha(\eps)-f\circ\gamma(\eps)|\le L\cdot s\cdot\eps+o(\eps).\]

Suppose $(f\circ\alpha)^+(0)$ is defined.
Then
\[|(f\circ\alpha)^+(0)-(f\circ\gamma)^+(0)|\le L\cdot s.\]
Since $\dd_pf$ is $L$-Lipschitz, we also get 
\[|\dd_pf(\alpha^+(0))-\dd_pf(\gamma^+(0))|\le L\cdot s.\]
It follows that the needed identity holds up to error $2\cdot L\cdot s$.
The statement follows since $s>0$ can be chosen arbitrarily.

The same argument is applicable if in place of $(f\circ\alpha)^+(0)$
we use any limit of $\tfrac1{\eps_n}\cdot [f\circ\alpha(\eps_n)-f(p)]$ for a sequence $\eps_n\to 0+$.
It proves that all such limits coincide; in particular, $(f\circ\alpha)^+(0)$ is defined and equals to $(\dd_pf)(\alpha^+(0))$.
\qeds


\section{Gradient curves}

\begin{thm}{Definition}\label{def:grad-curve}
Let $f$ be a semiconcave function on an Alexandrov space $\spc{A}$.

A locally Lipschitz curve $\alpha\:[t_{\min},t_{\max})\to\Dom f$ will be called an \index{gradient!curve}\emph{$f$-gradient curve} if
\[\alpha^+=\nabla_{\alpha} f;\]
that is, for any $t\in[t_{\min},t_{\max})$, $\alpha^+(t)$ is defined and 
$\alpha^+(t)=\nabla_{\alpha(t)} f$.
\end{thm}

A complete proof of the following theorem is given in our book \cite[16.15]{alexander-kapovitch-petrunin2024};
it mimics the proof of the standard Picard theorem on the existence  and uniqueness of solutions of ordinary differential equations.
The uniqueness will follow from the first distance estimate (\ref{thm:dist-est}) proved in the next section.
We omit the proof of existence as it is rather lengthy.

\begin{thm}{Picard theorem}\label{thm:glob-exist-grad-curv}
Let $f\:\spc{A}\subto \RR$ be a locally Lipschitz and $\lambda$-concave function on an Alexandrov space $\spc{A}$.
Then for any $p\in \Dom f$, there are unique $t_{\max}\in(0,\infty]$ and $f$-gradient curve $\alpha\:[0,t_{\max})\to \spc{A}$ with $\alpha(0)=p$ such that for any sequence $t_n\to t_{\max}-$, the sequence $\alpha(t_n)$ does not have a limit point in $\Dom f$.
\end{thm}

According to the theorem, the future of a gradient curve is determined by its present.
Let us show that its past is not determined by the present.

Consider the function $f\:x\mapsto -|x|$ on the real line $\RR$.
The tangent space $\T_x\RR$ can be identified with $\RR$.
Note that $\nabla_xf=-\mathrm{sgn}\, x$; that is,
\[\nabla_xf=
\begin{cases}
1&\text{if}\quad x<0,
\\
0&\text{if}\quad x=0,
\\
-1&\text{if}\quad x>0.
\end{cases}
\]
So, the $f$-gradient curves go to the origin with unit speed and then stand there forever.
In particular, if $\alpha$ is an $f$-gradient curve that starts at $x$,
then $\alpha(t)=0$ for any $t\ge |x|$.

Here is a slightly more interesting example;
it shows that gradient curves can merge even in the region where $|\nabla f|\z\ne 0$. 


\begin{wrapfigure}[8]{r}{34 mm}
\vskip-0mm
\centering
\includegraphics{mppics/pic-1215}
\vskip0mm
\end{wrapfigure}

\begin{thm}{Example}
Consider the function $f\:(x,y)\mapsto-|x|-|y|$ on the $(x,y)$-plane.
It is concave, and its gradient field is sketched on the figure.

Let $\alpha$ be an $f$-gradient curve that starts at $(x,y)$ for $x>y>0$.
Then 
\[\alpha(t)=
\begin{cases}
(x-t,y-t) &\text{for}\quad 0\le t\le  x-y,
\\
(x-t,0) &\text{for}\quad x-y\le t\le  x,
\\
(0,0) &\text{for}\quad x\le t.
\end{cases}
\]

\end{thm}


\section{Distance estimates}

\begin{thm}{Lemma}\label{eq:fist-var-inq+}
Let $\alpha$ be a gradient curve of a $\lambda$-concave function $f$
defined on an Alexandrov space.
Choose a point $p$; let $\ell(t)\df\distfun_p\z\circ\alpha(t)$ and $q=\alpha(t_0)$.
Then 
\[
\ell^+(t_0)\le -\left({f(p)}-{f(q)}-\tfrac\lambda2\cdot\ell^2(t_0)\right)/\ell(t_0)
\]
\end{thm}

\parit{Proof.}
Let $\gamma$ be the unit-speed parametrization of $[qp]$ from $q$ to $p$, so $q=\gamma(0)$.
Then 
\begin{align*}
\ell^+(t_0)&=(\dd_q\distfun_p)(\nabla_qf)\le\tag{by \ref{clm:fa'=dfa'}}
\\
&\le -\langle\dir qp,\nabla_qf\rangle \le \tag{by \ref{ex:d(distfun):<}}
\\
&\le -\dd_qf(\dir qp)=\tag{by \ref{def:grad}}
\\
&=-(f\circ\gamma)^+(0)\le 
\\
&\le -\left({f(p)}-{f(q)}-\tfrac\lambda2\cdot\ell^2(t_0)\right)/\ell(t_0)
\end{align*}
The last two lines follow by
the definition of differential,
and the concavity of $t\z\mapsto f\circ\gamma(t)-\tfrac \lambda2\cdot {t^2}$.
\qeds

The following estimate implies uniqueness in the Picard theorem (\ref{thm:glob-exist-grad-curv}).

\begin{thm}{First distance estimate}\label{thm:dist-est}
Let $f$ be a $\lambda$-concave locally Lipschitz function defined on an Alexandrov space $\spc{A}$.
Then
\[\dist{\alpha(t)}{\beta(t)}{}
\le 
e^{\lambda\cdot t}\cdot\dist[{{}}]{\alpha(0)}{\beta(0)}{}\]
for any $t\ge 0$ and any two $f$-gradient curves $\alpha$ and $\beta$.

Moreover, the statement holds for a locally Lipschitz $\lambda$-concave function defined in an open domain if there is a geodesic $[\alpha(t)\,\beta(t)]$ in $\Dom f$ for any~$t$.
\end{thm}

\parit{Proof.} 
Fix a choice of geodesic $[\alpha(t)\,\beta(t)]$ for each $t$.
Let $\ell(t)=\dist{\alpha(t)}{\beta(t)}{}$. 
Note that
\[\ell^+(t)
\le-
\<\dir{\alpha(t)}{\beta(t)},\nabla_{\alpha(t)}f\>-\<\dir{\beta(t)}{\alpha(t)},\nabla_{\beta(t)}f\>
\le
\lambda\cdot\ell(t).\]
Here one has to apply \ref{eq:fist-var-inq+} for distance to the midpoint $m$ of $[\alpha(t)\,\beta(t)]$, then apply the triangle inequality and \ref{ex:monotonicity}.

Integrating this inequality, we get the result.
\qeds



The following exercise describes a global geometric property of a gradient curve without direct reference to its function.
It is based on the notion of \index{self-contracting curves}\emph{self-contracting curves} introduced by Aris Daniilidis, Olivier Ley, and Stéphane Sabourau \cite{daniilidis-ley-sabourau}.

\begin{thm}{Exercise}\label{ex:elf-contracting}
Let $\alpha$ be a gradient curve of a concave function on an Alexandrov space.
Show that
\[\dist{\alpha(t_1)}{\alpha(t_3)}{\spc{A}}\ge \dist{\alpha(t_2)}{\alpha(t_3)}{\spc{A}}.\]
if $t_1\le t_2\le t_3$.
\end{thm}

\begin{thm}{Exercise}\label{ex:mayer}
Let $f$ be a locally Lipschitz concave function defined on an Alexandrov space $\spc{A}$.
Suppose $\hat\alpha\:[0,\ell]\to\spc{A}$ is an arc-length reparametrization of an $f$-gradient curve.
Show that $f\circ\hat\alpha$ is concave.
\end{thm}




The following exercise implies that gradient curves for a uniformly converging sequence of $\lambda$-concave functions converge to the gradient curves of the limit function.

\begin{thm}{Exercise}\label{lem:fg-dist-est}
Let $f$ and $g$ be $\lambda$-concave locally Lipschitz functions on an Alexandrov space $\spc{A}$.
Suppose
$\alpha,\beta\:[0,t_{\max})\to \spc{A}$ are respectively $f$- and $g$-gradient curves.
Assume $|f-g|<\eps$; let $\ell\:t\mapsto\dist{\alpha(t)}{\beta(t)}{}$.
Show that
\[\ell^+\le \lambda\cdot\ell+\tfrac{2\cdot\eps}{\ell}.\]

Conclude that if $\alpha(0)=\beta(0)$ and $t_{\max}<\infty$, then
\[\dist{\alpha(t)}{\beta(t)}{}
\le
\Const\cdot\sqrt{\eps\cdot t}\]
for some constant $\Const=\Const(t_{\max},\lambda)$.
\end{thm}

\section{Gradient flow}

Let $f$ be a locally Lipschitz semiconcave function defined on an open subset of an Alexandrov space $\spc{A}$.
If there is an $f$-gradient curve $\alpha$ such that $\alpha(0)=x$ and $\alpha(t)=y$,
then we will write 
\[\GF^t_f(x)=y.\]
The partially defined map $\GF^t_f$ from $\spc{A}$ to itself is called the \index{gradient!flow}\emph{$f$-gradient flow} for time $t$.
Note that
\[\GF^{t_1+t_2}_f=\GF_f^{t_1}\circ\GF_f^{t_2}.\]
In other words, the gradient flow is a partial action of the \textit{semigroup} $([0,\infty),+)$ on the space.
 
From the first distance estimate \ref{thm:dist-est}, 
it follows that for any $t\z\ge 0$, the domain of definition of $\GF^t_f$ is an open subset of $\spc{A}$.
For sufficiently nice functions, the gradient flow is globally defined.
For example, if $f$ is a $\lambda$-concave function and it is defined on the whole space $\spc{A}$, then $\GF^t_f(x)$ is defined for all $x\in \spc{A}$ and $t\ge0$;
see \cite[16.19]{alexander-kapovitch-petrunin2024}.

Using this new terminology, we can reformulate several statements about gradient curves.
From the first distance estimate, we have the following.

\begin{thm}{Proposition}\label{prop:GF-is-lip}
Let $f$ be a semiconcave function defined on an Alexandrov space $\spc{A}$.
Then the map $x\mapsto\GF^t_f(x)$ is locally Lipschitz.

Moreover, if $f$ is $\lambda$-concave, then $\GF^t_f$ is $e^{\lambda\cdot t}$-Lipschitz.
\end{thm}

The next proposition follows from \ref{lem:fg-dist-est}.

\begin{thm}{Proposition}\label{grad-curve-conv}
Let $\spc{A}$ be an Alexandrov space.
Suppose $f_n\:\spc{A}\z\to\RR$ is a sequence of
$\lambda$-concave functions 
that uniformly converges to $f_\infty\:\spc{A}\z\to \RR$. 
Then for any $x\in \spc{A}$ and $t\ge 0$, we have
\[\GF_{f_n}^t(x)\to \GF_{f_\infty}^t(x)\]
as $n\to \infty$.
\end{thm}

This proposition can be generalized to a converging sequence $\spc{A}_n\z\to \spc{A}_\infty$ of spaces and a converging sequence of functions $f_n\:\spc{A}_n\z\to\RR$; see \cite[16.21]{alexander-kapovitch-petrunin2024}.

\section{Gradient exponent}\label{gexp}

One of the technical difficulties in Alexandrov geometry comes from
nonextendability of geodesics. 
In particular, the exponential map, $\exp_p\:\T_p\to \spc{A}$, if defined in the usual way, can
be undefined in an arbitrarily small neighborhood of the origin. 

Now we will construct the \index{gradient!exponential map}\emph{gradient exponential map}
\[\gexp_p\:\T_p\to\spc{A},\]
which essentially solves this problem. 
It shares many properties with the ordinary exponential map and is even better in certain respects,
even in the Riemannian universe.

Let $p$ be a point in an $\Alex0$ space $\spc{A}$.
Consider the function $f\z=\distfun_p^2/2$.
Recall that $\GF^t_{f}$ denotes the gradient flow.
Let us define the \textit{gradient exponential map} as the limit
\[\gexp_p(v)=\lim_{n\to\infty}\GF^{t_n}_{f}(x_n),\]
where the sequences $x_n\in \spc{A}$ and $t_n\ge 0$ are chosen so that $t_n\to\infty$
and $e^{t_n}\cdot\log_px_n\to v$ as $n\to\infty$.

More intuitively, suppose $i_{\lambda}\:\lambda\cdot \spc{A}\to \spc{A}$ sends a point in the rescaled copy $\lambda\cdot\spc{A}$ to the corresponding point in $\spc{A}$.
By the first distance estimate (\ref{thm:dist-est}), the map
$$\GF^t_{f}\circ i_{e^t}\:e^t\cdot \spc{A}\to \spc{A}\eqlbl{eq:gexp}$$
is short for any $t\ge 0$.
If we have a pointed Gromov--Hausdorff convergence $(e^{t_n}\cdot \spc{A},p)\to (\T_p,o_p)$,
then $\gexp_p\:\T_p\to \spc{A}$ is the limit of $\GF^{t_n}_{f}\circ i_{e^{t_n}}$.
This way we get that $\gexp_p$ is short as a limit of short maps.
This observation is generalized in the following proposition. 


\begin{thm}{Proposition}\label{prop:gexp}
Let $\spc{A}$ be a proper $\Alex0$ space.
Then for any $p\in \spc{A}$ the gradient exponent $\gexp_p\:\T_p\to\spc{A}$ is uniquely defined.
Moreover, $\gexp_p$ is a short map and 
\[\gexp_p(\gamma^+(0))=\gamma(1)\]
for any geodesic path $\gamma$ that starts at $p$.
\end{thm}

The last statement implies that 
\[\gexp_p\circ\log_p=\id,\]
so it is appropriate to use term \textit{exponent} for $\gexp$.


\parit{Proof.} 
Note that $f''\le 1$.
Since the space is proper we can choose a limit in \ref{eq:gexp}.

Let $\gamma$ be a geodesic that stats at $p$.
Observe that $t\mapsto \gamma\circ\ln(t)$ is an $f$-gradient curve.
By the first distance estimate, we have that $\GF^t_{f}$ is an $e^t$-Lipschitz.
This implies that any limit in \ref{eq:gexp} has the same value;
that is, $\gexp_p$ is uniquely defined.

Again, since $\GF^t_{f}$ is an $e^t$-Lipschitz, we get that $\gexp_p$ is short.
\qeds

\section{Remarks}

The idea to use gradient flows in Alexandrov geometry was inspired by the success of \index{Sharafutdinov's retraction}\emph{Sharafutdinov's retraction} in comparison geometry \cite{sharafutdinov}.
The gradient flow was introduced by the second author to construct quasigeodesics with given initial data \cite{perelman-petrunin:qg,petrunin:qg, petrunin:survey}.
It turned out that gradient flow and gradient exponent are better tools than quiasigeodesics.
These tools quickly found applications in other types of singular spaces \cite{jost,mayer,lytchak:open-map,ohta,sevare,ambrosio-gigli-savare}.


For a general lower curvature bound $\kappa$, the construction of gradient exponent has to be modified;
it is denoted by $\gexp_p^\kappa$ \cite[16.36]{alexander-kapovitch-petrunin2024}. It is done by taking limits of appropriately reparameterized gradient curves of  the modified distance function.

For $\kappa=-1$ we have 
that $\gexp_p(\gamma^+(0))=\gamma(1)$
for any geodesic path $\gamma$ that starts at $p$
and 
\[\dist{\gexp_p^{-1}v}{\gexp_p^{-1}w}{\spc{A}}\le \side\hinge0vw_{\HH^2}.\]
In other words $\gexp_p$ is short if we equip $\T_p$ with the hyperbolic cone metric.

Similarly, for $\kappa=1$ we have $\gexp_p^1(\gamma^+(0))=\gamma(1)$
for any geodesic path $\gamma$ that starts at $p$ and 
\[\dist{\gexp_p^{1}v}{\gexp_p^{1}w}{\spc{A}}\le \side\hinge0vw_{\SSS^2},\]
but this time all this holds only if $|v|,|w|\le\tfrac\pi2$ and $\length\gamma\le\tfrac\pi2$.

The gradient exponential map in a Riemannian manifold $(M,g)$ coincides with the Riemannian exponential map before the cut locus, but \textit{is different} from the  Riemannian exponential after that.
The following exercise is ment to show that this technique can prove something nontrivial even for Riemannian manifolds.

\begin{thm}{Exercise}\label{ex:short-onto}
Let $(M,g)$ be a complete $m$-dimensional Riemannian with sectional curvature at least $1$.
Assume $M$ is not homeomorphic to $\SSS^m$.
Show that there is a short onto map $\SSS^m\to (M,g)$.
\end{thm}
