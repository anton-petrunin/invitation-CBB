\chapter{Gradient flow}

\section{Velocity of curve}

Let $\alpha$ be a curve in a geodesic $\CBB(\kappa)$ space $\spc{L}$;
choose $p=\alpha(t_0)$.
If for any choice of geodesics $[p\,\alpha(t_0+\eps)]$ the vectors 
\[\tfrac{1}{\eps}\cdot\dist{p}{\alpha(t_0+\eps)}{}\cdot\dir p{\alpha(t_0+\eps)}\]
converge as $\eps\to 0+$, then their limit in $\T_p$ is called the \index{right derivative}\emph{right derivative} of $\alpha$ at $t_0$; it will be denoted by $\alpha^+(t_0)$.
In addition, $\alpha^+(t_0)\df0$
if $\tfrac{1}{\eps}\cdot\dist{p}{\alpha(t_0+\eps)}{}\to 0$ as $\eps\to 0+$.

\begin{thm}{Claim}\label{clm:fa'=dfa'}
Let $\alpha$ be a curve in a geodesic $\CBB(\kappa)$ space $\spc{L}$.
Suppose $f$ a semiconcave Lipschitz function
defined in a neighborhood of $p\z=\alpha(0)$,
and $\alpha^+(0)$ is defined.
Then 
\[(f\circ\alpha)^+(0)
=
(\dd_pf)(\alpha^+(0)).\]
\end{thm}

\parit{Proof.}
Without loss of generality, we can assume that $f(p)=0$.
Suppose $f$ and therefore $\dd_pf$ are $L$-Lipschitz.

Choose a constant-speed geodesic $\gamma$ that starts from $p$,
such that the distance
$s=\dist{\alpha^+(0)}{\gamma^+(0)}{\T_p}$
is small.
Observe that by the definition of differential,
\[(f\circ\gamma)^+(0)=\dd_pf(\gamma^+(0)).\]

By comparison and the definition of $\alpha^+$,
\[\dist{\alpha(\eps)}{\gamma(\eps)}{\spc{L}}\le s\cdot\eps+o(\eps)\]
for $\eps>0$.
Therefore,
\[|f\circ\alpha(\eps)-f\circ\gamma(\eps)|\le L\cdot s\cdot\eps+o(\eps).\]

Suppose $(f\circ\alpha)^+(0)$ is defined.
Then
\[|(f\circ\alpha)^+(0)-(f\circ\gamma)^+(0)|\le L\cdot s.\]
Since $\dd_pf$ is $L$-Lipschitz, we also get 
\[|\dd_pf(\alpha^+(0))-\dd_pf(\gamma^+(0))|\le L\cdot s.\]
It follows that the needed identity holds up to error $2\cdot L\cdot s$.
The statement follows since $s>0$ can be chosen arbitrarily.

The same argument is applicable if in the place of $(f\circ\alpha)^+(0)$
we use any limit of $\tfrac1{\eps_n}\cdot [f\circ\alpha(\eps_n)-f(p)]$ as $\eps_n\to 0+$.
It proves that all such limits coincide; in particular, $(f\circ\alpha)^+(0)$ is defined and equals to $(\dd_pf)(\alpha^+(0))$.
\qeds


\section{Gradient curves}

\begin{thm}{Definition}\label{def:grad-curve}
Let $f\:\spc{L}\subto\RR$ be a locally Lipschitz and semiconcave function on a complete geodesic $\CBB(\kappa)$ space
$\spc{L}$.

A locally Lipschitz curve $\alpha\:[t_{\min},t_{\max})\to\Dom f$ will be called an \index{gradient curve}\emph{$f$-gradient curve} if
\[\alpha^+=\nabla_{\alpha} f;\]
that is, for any $t\in[t_{\min},t_{\max})$, $\alpha^+(t)$ is defined and 
$\alpha^+(t)=\nabla_{\alpha(t)} f$.
\end{thm}

A complete proof of the following theorem takes about 5 pages \cite{alexander-kapovitch-petrunin2024}; 
it mimics the proof of the standard Picard theorem on the existence and uniqueness of solutions of ordinary differential equations.
We omit the proof of existence;
the uniqueness will be proved in the next section.


\begin{thm}{Picard theorem}\label{thm:glob-exist-grad-curv}
Let $f\:\spc{L}\subto \RR$ be a locally Lipschitz and $\lambda$-concave function on a complete geodesic $\CBB(\kappa)$ space $\spc{L}$.
Then for any $p\in \Dom f$, there are unique $t_{\max}\in(0,\infty]$ and $f$-gradient curve $\alpha\:[0,t_{\max})\to \spc{L}$ with $\alpha(0)=p$ such that any sequence $t_n\to t_{\max}-$, the sequence $\alpha(t_n)$ does not have a limit point in $\Dom f$.
\end{thm}

Note that the theorem says that the future of a gradient curve is determined by its present, but it says nothing about its past.

Here is an example showing that the past is not determined by the present.
Consider the function $f\:x\mapsto -|x|$ on the real line $\RR$.
The tangent space $\T_x\RR$ can be identified with $\RR$.
Note that 
\[\nabla_xf=
\begin{cases}
1&\text{if}\quad x<0,
\\
0&\text{if}\quad x=0,
\\
-1&\text{if}\quad x>0.
\end{cases}
\]
So, the $f$-gradient curves go to the origin with unit speed and then stand there forever.
In particular, if $\alpha$ is an $f$-gradient curve that starts at $x$,
then $\alpha(t)=0$ for any $t\ge |x|$.

Here is a slightly more interesting example;
it shows that gradient curves can merge even in the region where $|\nabla f|\z\ne 0$. 


\begin{wrapfigure}[8]{r}{34 mm}
\vskip-0mm
\centering
\includegraphics{mppics/pic-1215}
\vskip0mm
\end{wrapfigure}

\begin{thm}{Example}
Consider the function $f\:(x,y)\mapsto-|x|-|y|$ on the $(x,y)$-plane.
Note that $f$ is concave;
its gradient field is sketched on the figure.

Let $\alpha$ be an $f$-gradient curve that starts at $(x,y)$ for $x>y>0$.
Then 
\[\alpha(t)=
\begin{cases}
(x-t,y-t) &\text{for}\quad 0\le t\le  x-y,
\\
(x-t,0) &\text{for}\quad x-y\le t\le  x,
\\
(0,0) &\text{for}\quad x\le t.
\end{cases}
\]

\end{thm}


\section{Distance estimates}

\begin{thm}{Observation}\label{eq:fist-var-inq+}
Let $\alpha$ be a gradient curve of a $\lambda$-concave function $f$ 
defined on a complete geodesic $\CBB$ space.
Choose a point $p$; let $\ell(t)\df\distfun_p\circ\alpha(t)$ and $q=\alpha(t_0)$.
Then 
\[
\ell^+(t_0)\le -\left({f(p)}-{f(q)}-\tfrac\lambda2\cdot\ell^2(t_0)\right)/\ell(t_0)
\]
\end{thm}

\parit{Proof.}
Let $\gamma$ be the unit-speed parametrization of $[qp]$ from $q$ to $p$, so $q=\gamma(0)$.
Then 
\begin{align*}
\ell^+(t_0)&=(\dd_q\distfun_p)(\nabla_qf)\le
\\
&\le -\langle\dir qp,\nabla_qf\rangle \le
\\
&\le -\dd_qf(\dir qp)=
\\
&=-(f\circ\gamma)^+(0)\le
\\
&\le -\left({f(p)}-{f(q)}-\tfrac\lambda2\cdot\ell^2(t_0)\right)/\ell(t_0)
\end{align*}
In the above calculations we consequently applied
\ref{clm:fa'=dfa'},
\ref{ex:d(distfun)},
the definition of gradient,
the definition of differential,
and concavity of $t\z\mapsto f\circ\gamma(t)-\tfrac \lambda2\cdot {t^2}$.
\qeds

Note that the following estimate implies uniqueness in the Picard theorem (\ref{thm:glob-exist-grad-curv}).

\begin{thm}{First distance estimate}\label{thm:dist-est}
Let $f$ be a $\lambda$-concave locally Lipschitz function on a complete geodesic $\CBB(\kappa)$ space $\spc{L}$.
Then
\[\dist{\alpha(t)}{\beta(t)}{}
\le 
e^{\lambda\cdot t}\cdot\dist[{{}}]{\alpha(0)}{\beta(0)}{}\]
for any $t\ge 0$ and any two $f$-gradient curves $\alpha$ and $\beta$.

Moreover, the statement holds for a locally Lipschitz $\lambda$-concave function defined in an open domain if there is a geodesic $[\alpha(t)\,\beta(t)]$ in $\Dom f$ for any~$t$.
\end{thm}

\parit{Proof.} 
Fix a choice of geodesic $[\alpha(t)\,\beta(t)]$ for each $t$.
Let $\ell(t)=\dist{\alpha(t)}{\beta(t)}{}$. 
Note that
\[\ell^+(t)
\le-
\<\dir{\alpha(t)}{\beta(t)},\nabla_{\alpha(t)}f\>-\<\dir{\beta(t)}{\alpha(t)},\nabla_{\beta(t)}f\>
\le
\lambda\cdot\ell(t).\]
Here one has to apply \ref{eq:fist-var-inq+} for distance to the midpoint $m$ of $[\alpha(t)\,\beta(t)]$, and then apply the triangle inequality.
Hence the result. 
\qeds



The following exercise describes a global geometric property of a gradient curve without direct reference to its function.
It uses the notion of \textit{self-contracting curves} introduced by Aris Daniilidis, Olivier Ley, and St\'ephane Sabourau \cite{daniilidis-ley-sabourau}.

\begin{thm}{Exercise}\label{ex:elf-contracting}
Let $f\:\spc{L}\subto\RR$ be a locally Lipschitz and concave function on a complete geodesic $\CBB(\kappa)$ space
$\spc{L}$.
Then 
\[\dist{\alpha(t_1)}{\alpha(t_3)}{\spc{L}}\ge \dist{\alpha(t_2)}{\alpha(t_3)}{\spc{L}}.\]
for any $f$-gradient curve $\alpha$ and $t_1\le t_2\le t_3$.
\end{thm}

\begin{thm}{Exercise}\label{ex:mayer}
Let $f$ be a locally Lipschitz concave function defined on a $\CBB(\kappa)$ space.
Suppose $\hat\alpha\:[0,\ell]$ is an arc-length reparametrization of an $f$-gradient curve.
Show that $(f\circ\hat\alpha)$ is concave.
\end{thm}




The following exercise implies that gradient curves for a uniformly converging sequence of $\lambda$-concave functions converge to the gradient curves of the limit function.

\begin{thm}{Exercise}\label{lem:fg-dist-est}
Let $f$ and $g$ be $\lambda$-concave locally Lipschitz functions on a complete geodesic $\CBB(\kappa)$ space $\spc{L}$.
Suppose
$\alpha,\beta\:[0,t_{\max})\to \spc{L}$ are respectively $f$- and $g$-gradient curves.
Assume $|f-g|<\eps$; let $\ell\:t\mapsto\dist{\alpha(t)}{\beta(t)}{}$.
Show that
\[\ell^+\le \lambda\cdot\ell+\tfrac{2\cdot\eps}{\ell}.\]

Conclude that if $\alpha(0)=\beta(0)$ and $t_{\max}<\infty$, then
\[\dist{\alpha(t)}{\beta(t)}{}
\le
\Const\cdot\sqrt{\eps\cdot t}\]
for some constant $\Const=\Const(t_{\max},\lambda)$.
\end{thm}

\section{Gradient flow}

Let $\spc{L}$ be a complete geodesic $\CBB(\kappa)$ space 
and $f$ be a locally Lipschitz semiconcave function defined on an open set of $\spc{L}$.
If there is an $f$-gradient curve $\alpha$ such that $\alpha(0)=x$ and $\alpha(t)=y$,
then we will write 
\[\GF^t_f(x)=y.\]
The partially defined map $\GF^t_f$ from $\spc{L}$ to itself is called the \index{gradient flow}\emph{$f$-gradient flow} for time $t$.
Note that
\[\GF^{t_1+t_2}_f=\GF_f^{t_1}\circ\GF_f^{t_2}.\]
In other words, one may think that gradient flow is an action of the \textit{semigroup} $(\RR_{\ge0},+)$ on the space.
 
From the first distance estimate \ref{thm:dist-est}, 
it follows that for any $t\ge 0$, the domain of definition of $\GF^t_f$ is an open subset of $\spc{L}$.
In some cases, it is globally defined.
For example, if $f$ is a $\lambda$-concave function defined on the whole space $\spc{L}$, then $\GF^t_f(x)$ is defined for all $x\in \spc{L}$ and $t\ge0$;
see \cite[16.19]{alexander-kapovitch-petrunin2024}.

Now let us reformulate statements obtained earlier using this new terminology.
From the first distance estimate, we have the following.

\begin{thm}{Proposition}\label{prop:GF-is-lip}
Let $\spc{L}$ be a complete geodesic $\CBB(\kappa)$ space 
and $f\:\spc{L}\to \RR$ be a semiconcave function.
Then the map $x\mapsto\GF^t_f(x)$ is locally Lipschitz.

Moreover, if $f$ is $\lambda$-concave, then $\GF^t_f$ is $e^{\lambda\cdot t}$-Lipschitz.
\end{thm}

The next proposition follows from \ref{lem:fg-dist-est}.

\begin{thm}{Proposition}\label{grad-curve-conv}
Let $\spc{L}$ be complete geodesic $\CBB(\kappa)$ space.
Suppose $f_n\:\spc{L}\to\RR$ is a sequence of
$\lambda$-concave functions 
that converges to $f_\infty\:\spc{L}\to \RR$. 
Then for any $x\in \spc{L}$ and $t\ge 0$, we have
\[\GF_{f_n}^t(x)\to \GF_{f_\infty}^t(x)\]
as $n\to \infty$.
\end{thm}

%??? do we need GH-limit version???

\section{Gradient exponent}\label{gexp}

One of the technical difficulties in Alexandrov's geometry comes from
nonextendability of geodesics. 
In particular, the exponential map, $\exp_p\:\T_p\to \spc{L}$, if defined the usual way, can
be undefined in an arbitrary small neighborhood of the origin. 

We construct its analog, the \emph{gradient exponential map} 
\[\gexp_p\:\T_p\to\spc{L},\]
which practically solves this problem. 
It shares many properties with the ordinary exponential map, and better in certain respects,
even in the Riemannian universe.

Let $\spc{L}$ be Alexandrov's space and $p\in \spc{L}$, consider the function $f\z=\distfun_p^2/2$.
Suppose $i_{\lambda}\:\lambda\cdot \spc{L}\to \spc{L}$ sends a point in the rescaled copy $\lambda\cdot\spc{L}$ to the corresponding point in $\spc{L}$.
Consider the one parameter family of maps
$$\Phi^t_{f}\circ i_{e^t}\:e^t{\cdot} \spc{L}\to \spc{L}$$
where $\Phi^t_{f}$ denotes gradient flow. 
Note that $(e^t{\cdot} \spc{L},p)\GHto (\T_p,o_p)$ as $t\to\infty$.
Let us define the \textit{gradient exponential map} as the limit
\[\gexp_p=\lim_{t\to\infty}\Phi^t_{f}\circ i_{e^t}.\]

\begin{thm}{Proposition}
Let $\spc{L}$ be a proper geodesic $\CBB(0)$ space.
Then for any $p\in \spc{L}$ the gradient exponent $\gexp_p\:\T_p\to\spc{L}$ is defined.
Moreover, $\gexp_p$ is a short map and 
\[\gexp_p(\gamma^+(0))=\gamma(1)\]
for any geodesic path $\gamma$ that starts at $p$.
\end{thm}

The last statement in the proposition says that it is appropriate to use term \textit{exponent} for $\gexp$.


\parit{Proof.} 
Note that $f''\le 1$.
By the first distance estimate, we have that $\Phi^t_{f}$ is an $e^t$-Lipschitz.
Therefore, the compositions $\Phi^t_{f}\circ i_{e^t}\:e^t{\cdot} \spc{L}\to \spc{L}$ are short. 
Hence a partial limit $\gexp_p\:\T_p
\spc{L}\to \spc{L}$ exists, and it is a short map.

Clearly for any partial limit we have
\[\Phi^t_f\circ\gexp_p(v)=\gexp_p(e^t\cdot v).\]
Since $\Phi^t$ is $e^t$-Lipschitz, it follows that $\gexp_p$ is uniquely
defined.
\qeds

\section{Remarks}

??? gradient exponent for $\kappa\ne 0$
and for nonproper.
