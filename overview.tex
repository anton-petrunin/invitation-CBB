\chapter{An overview of surface theory}\label{chap:surfaces}

This lecture aims to provide guidance on the key ideas and theorems in the geometry of convex surfaces, which is the main source for modern Alexandrov geometry.
For a deeper dive into this theory, we recommend turning to the classic and brilliantly written books by Alexandr Alexandrov \cite{alexandrov,alexandrov-1948}.
Also, the book by Alexey Pogorelov \cite{pogorelov1969} is very recommened, despite being a challenge to read.

\section{Polyhedral surfaces}

A \index{polyhedral surface}\emph{polyhedral surface} is defined as a 2-dimensional manifold with a length metric that admits a finite triangulation such that each triangle is isometric to a solid plane triangle.
A \index{triangulation}\emph{triangulation} of a polyhedral surface will always be assumed to satisfy this condition.

Note that according to our definition, any polyhedral surface is compact.

Choose a point $p$ on a polyhedral surface $\spc{P}$.
We can assume that $p$ is a vertex of a triangulation $\spc{P}$;
if not, subdivide the triangulation.
Denote by $\theta_p$ the total angle around $p$;
that is, the sum of all angles at $p$ of all the triangles with vertex at $p$.

Note that $\theta_p$ does not depend on the choice of triangulation.
If $p$ is an interior point, then the value $2\cdot\pi-\theta_p$ is called \emph{curvature} at $p$.
If $p$ lies on the boundary of $\spc{P}$, then the value $\pi-\theta_p$ is called \emph{inner turn} at $p$.

A point with nonvanishing curvature will be called an \emph{essential vertex} of the surface.
Observe that an essential vertex is a vertex in any triangulation.

\begin{thm}{Exercise}\label{ex:geodesic-vertex}
Show that geodesics on a closed polyhedral surface with nonnegative curvature may have essential vertices only at their endpoints.
\end{thm}

The following statement is an analog of the Gauss--Bonnet formula.

\begin{thm}{Exercise}\label{ex:gauss-bonnet}
Let $K(\spc{P})$ and $T(\spc{P})$ denote the sum of curvatures of all interior points
and the sum of all inner turns of the boundary points a polyhedral surface $\spc{P}$.
Show that
\[K(\spc{P})+T(\spc{P})=2\cdot\pi\cdot\chi(\spc{P}),\]
where $\chi(\spc{P})$ denotes the Euler characteristic of $\spc{P}$.
\end{thm}

Let us show that this new definition of curvature agrees with the $\Alex0$ comparison.

\begin{thm}{Proposition}\label{prop:poly-CBB}
A polyhedral surface is $\Alex0$ if and only if it has nonnegative curvature at every point.
\end{thm}

\parit{Proof.}
By \ref{comp-kappa}, it is sufficient to check that
$f=\tfrac12\cdot\distfun_p^2\circ\gamma$ is 1-concave for any geodesic $\gamma$ and any point $p$.

We can assume that $p$ is not a vertex and the endpoints of $\gamma$ are not vertices;
the vertex case can be done by approximation.
By \ref{ex:geodesic-vertex}, $\gamma$ does not contain vertices.

Given a point $x=\gamma(t_0)$, choose a geodesic $[px]$.
Again, by \ref{ex:geodesic-vertex}, $[px]$ does not contain vertices.
Therefore, a neighborhood $U\supset [px]$ can be unfolded on a plane;
that is, there is an injective length-preserving map $z\mapsto \tilde z$
of $U$ into the Euclidean plane.
This way we map the part of $\gamma$ in $U$ to a line segment $\tilde\gamma$.
Let
\[\tilde f(t)\df\tfrac12\cdot\distfun_{\tilde p}^2\circ\tilde \gamma(t).\]
Since the geodesic $[px]$ maps to a line segment, we have $\tilde f(t_0)= f(t_0)$.
Furthermore, since the unfolding $z\mapsto \tilde z$ preserves lengths of curves, we get
$\tilde f(t)\ge f(t)$ if $t$ is close to $t_0$.
That is, $\tilde f$ is a local upper barrier of $f$ at $t_0$.
Evidently, $\tilde f''(t)\equiv 1$.
Therefore, $f$ is 1-concave.

\begin{thm}{Exercise}\label{ex:poly-CBB}
The converse is left to the reader.\qeds
\end{thm}

\section{Approximation}

The following theorem is the main extra tool available in Alexandrov geometry of surfaces.
It can be used to reduce questions about $\Alex0$ surfaces to polyhedral surfaces with nonnegative curveature; for example, see~\ref{cor:Alex0-convex}.

\begin{thm}{Theorem}\label{thm:approximation}
Any closed $\Alex0$ surface can be approximated by a polyhedral $\Alex0$ surfaces in the sense of Gromov--Hausdorff.
\end{thm}

The proof relies on a sequence of statements from the following exercise.

Suppose a convex solid triangle $\Delta$ in a  $\Alex0$ surface has angles $\alpha$, $\beta$ and $\gamma$.
Then its excess( briefly $\excess\Delta$) is defined as $\alpha+\beta+\gamma-\pi$.
Since the angles of a model triangle sum up to $\pi$,
the excess of any triangle in a $\Alex0$ surface is nonnegative.

\begin{thm}{Exercise}\label{ex:approximation}
Let $\spc{P}$ be an $\Alex0$ surface.

\begin{subthm}{ex:approximation:nbhd}
Show that any point $p$ admits a closed convex polygonal neighborhood $N$;
that is, $N$ is convex and bounded by a broken geodesic.

\end{subthm}

\begin{subthm}{ex:approximation:triangulation}
Given $\delta>0$, show that $\spc{P}$ admits a triangulation $\tau$ by convex triangles
with excess and diameter smaller than $\delta$.
\end{subthm}

\begin{subthm}{ex:approximation:poly}
Suppose $\tau$ is a triangulation of $\spc{P}$ by convex triangles.
Let us prepare a solid model triangle $\tilde\Delta$ for each triangle $\Delta$ in $\tau$ and glue them as prescribed by $\tau$;
as a result, we get a surface $\tilde{\spc{P}}$ that is homeomorphic to $\spc{P}$.
Show that $\tilde{\spc{P}}$ is a polyhedral surface with nonnegative curvature.
\end{subthm}

\begin{subthm}{ex:approximation:diangle}
Let $\Upsilon$ be a disc in $\tilde{\spc{P}}$ bounded by a geodesic $\gamma_0$ and local geodesic $\gamma_1$ with common endpoints.
Let $\omega$ be the sum of the curvatures of points in $\Upsilon$.
Show that
\[\cos \omega\cdot \length \gamma_1\le \length \gamma_0\]
if $\omega<\pi$.
\end{subthm}


\begin{subthm}{ex:approximation:excess}
Let $\tau$ be a triangulation of $\spc{P}$ by convex triangles $\Delta_1,\dots,\Delta_n$
Show that
\[\excess\Delta_1+\dots+\excess\Delta_n \le 2\cdot\pi\cdot \chi(\spc{P}),\]
where $\chi(\spc{P})$ denotes the Euler characteristic of $\spc{P}$.

Conclude that $\chi(\spc{P})\ge 0$; so $\spc{P}$ is homeomorphic to a sphere, projctive plane, torus, or Klein bottle.
\end{subthm}


\begin{subthm}{ex:approximation:length}
Let $x$ and $y$ be points on the sides of a triangle $\Delta$ of $\tau$, and let $\tilde x$ and $\tilde y$ be the corresponding points in the corresponding triangle $\tilde \Delta$ in $\tilde{\spc{P}}$.
Show that
\[\dist{\tilde x}{\tilde y}{\tilde \Delta}\le\dist{x}{y}{\Delta}\le \dist{\tilde x}{\tilde y}{\tilde \Delta}+\excess\Delta\cdot \diam \Delta.\]
Conclude that
\[\dist{\tilde v}{\tilde w}{\tilde {\spc{P}}}\le \dist{v}{w}{\spc{P}}\]
for any vertices $v$ and $w$ of $\tau$ in $\spc{P}$ and corresponding vertices $\tilde v$ and $\tilde w$ in $\tilde{\spc{P}}$.
\end{subthm}

%\begin{subthm}{ex:approximation:area}
%Let $\Delta$ be a solid triangle in the triangulation $\tau$ of $\spc{P}$, and $\tilde \Delta$ --- the corresponding triangle in $\tilde{\spc{P}}$.
%Show that
%\[\area \tilde \Delta\le \area \Delta\le \area \tilde \Delta+\tfrac12\cdot\excess\Delta\cdot (\diam \Delta)^2.\]
%Conclude that if $\tau$ is as in \ref{SHORT.ex:approximation:triangulation}, then
%\[\area \tilde{\spc{P}}\le \area \spc{P}\le \area \tilde{\spc{P}}+2\cdot \pi\cdot \delta^2.\]
%\end{subthm}

\end{thm}

To prove the approximation theorem, we need to show that $\tilde {\spc{P}}\to \spc{P}$ as $\delta\to0$ in the sense of Gromov--Hausdorff.
The latter follows from the next claim.

\begin{thm}{Claim}
There is $\eps\z=\eps(\area\spc{P}, \diam\spc{P}, \delta)>0$ such that $\eps\to0$ as $\delta\to 0$ and
\[\dist{v}{w}{\spc{P}}\le \dist{\tilde v}{\tilde w}{\tilde{\spc{P}}}+\eps\]
for any vertices $v$ and $w$ of $\tau$ in $\spc{P}$ and corresponding vertices $\tilde v$ and $\tilde w$ in $\tilde{\spc{P}}$.
\end{thm}

If each triangle in the triangulation of $\tilde{\spc{P}}$ were convex, then the claim would follow from \ref{SHORT.ex:approximation:excess} and \ref{SHORT.ex:approximation:length}.
The same argument would work if any such triangle could not be visited by a geodesic more than 1000 times,
but it seems that there is no univeral bound on the number of such visits.

The rest of the statements in the exercise help to deal with this issue.
The reader might try to complete the proof on their own or read it in \cite[VII § 6]{alexandrov-1948}.
%Part \ref{SHORT.ex:approximation:area} was not used in Alexandrov's proof, but it could be used.

\section{Surface of polyhedrons and bodies}

Let us define a \index{convex body}\emph{convex body} as a compact convex subset in $\EE^3$ with a non-empty interior.
The \index{surface}\emph{surface} of a convex body is defined as its boundary equipped with the induced length metric.

\begin{thm}{Exercise}\label{ex:surf-S2}
Show that the surface of a convex body is homeomorphic to the 2-dimensional sphere.
\end{thm}

A \index{convex polyhedron}\emph{convex polyhedron} is a convex body with a finite number of extremal points, called its \index{vertex}\emph{vertices}.

Note that the surface, say $\spc{P}$, of a convex polyhedron $K$ is a polyhedral surface;
that is, it admits a finite triangulation such that each triangle is isometric to a plane triangle.

\begin{thm}{Exercise}\label{pr:tetrahedron}
Assume that the surface of a nondegenerate tetrahedron $T$ has curvature $\pi$ at each of its vertices.
Show that

\begin{subthm}{pr:tetrahedron:=}
all faces of $T$ are congruent;
\end{subthm}

\begin{subthm}{pr:tetrahedron:perp} the line containing the midpoints of opposite edges of $T$ intersects these edges at right angles.
\end{subthm}

\end{thm}

\begin{thm}{Claim}\label{clm:total-angle}
The surface $\spc{P}$ of any convex polyhedron $K$ has nonnegative curvature.
Moreover, a point $v$ is a vertex of $K$ if and only if
$v$ is an essential vertex of $\spc{P}$.
\end{thm}

A proof is given in Kiselyov's school textbook \cite[§ 48]{kiselev-stereo-en};
one can also deduce it from \ref{claim:angle-3angle-inq}.

\begin{thm}{Exercise}\label{ex:surface-covergence}
Let $K_1,K_2,\dots,$ and $K_\infty$ be convex bodies in $\EE^m$.
Denote by $\spc{P}_n$ the surface of $K_n$.
Suppose $K_n\z\to K_\infty$ in the sense of Hausdorff.
Show that $\spc{P}_n\to \spc{P}_\infty$ in the sense of Gromov--Hausdorff.
\end{thm}

Since any convex body is a Hausdorff limit of a sequence of convex polyhedrons, the next proposition follows from \ref{prop:poly-CBB}, \ref{ex:surface-covergence}, and \ref{thm:CBB-closed}.

\begin{thm}{Proposition}\label{prop:conv-surf-CBB(0)}
The surface of a convex body in $\EE^3$ is $\Alex0$.
\end{thm}

\section{Uniqueness theorem}

\begin{thm}{Theorem}\label{thm:alexandrov-uni'}
Any two convex polyhedrons in $\EE^3$ with isometric surfaces are congruent.

Moreover, any isometry between the surfaces of convex polyhedrons can be extended to an isometry of the whole $\EE^3$.
\end{thm}

If one assumes that the isometry between the surfaces is face-to-face,
then we get an equivalent reformulation of Cauchy's theorem.
Cauchy's argument, with a small addition, proves \ref{thm:alexandrov-uni'}.

First, let us remind the framework of Cauchy's proof, assuming the reader knows it.
If not, then read it in one of the classical texts \cite{aigner-zigler,dolbilin,tabacnikov-fuks}.

\parit{Sketch of Cauchy's proof.}
Suppose $K$ and $K'$ are convex polyhedrons in $\EE^3$;
denote their surfaces
by $\spc{P}$ and $\spc{P}'$.
Suppose there is an isometry $\iota\:\spc{P}\to \spc{P}'$ that sends each face of $K$ to a face of $K'$.

Let us mark an edge of $K$ with ``$+$'' (or ``$-$'') if the dihedral angle at this edge in $K$ is smaller (respectively, bigger) than the corresponding angle in $K'$.
Further, we consider the  graph $\Gamma$ that is formed by all marked edges.
If $\Gamma$ is empty, then Cauchy's theorem follows; assume the contrary.

The graph $\Gamma$ is embedded into $\spc{P}$, which is homeomorphic to the sphere.
In particular, the edges coming from one vertex have a natural cyclic order.
Given a vertex $v$ of $\Gamma$, we can count the \textit{number of sign changes} around $v$;
that is, the number of consequent pairs of edges with different signs.

Now we need to show two statements:

\begin{thm}{Local lemma}
At any vertex of $\Gamma$, the number of sign changes is at least $4$.
\end{thm}

\begin{thm}{Global lemma}
No (nonempty) planar graph meets the condition of the local lemma.
\end{thm}

Once the lemmas are proved, Cauchy's theorem follows.
\qeds

Once more, the argument above is  written only to make sure we are on the same page;
it will not work without reading the actual proof.

\parit{Alexandrov's addition.}
We need to remove the assumption that the isometry $\iota\:\spc{P}\z\to \spc{P}'$ is face-to-face.
Mark in $\spc{P}$ all the edges of $K$ as we did above.
In addition, if an edge in $K'$ does not correspond to an edge of $K$, then mark its inverse image in $K$   with ``$-$''; these lines on $K$ will be referred to as \index{fake edges and vertices}\emph{fake edges}.

The marked lines divide $\spc{P}$ into convex polygons, and the restriction of $\iota$ to each polygon is a rigid motion.
These polygons should be used instead of faces in the Cauchy's argument.

A vertex of the obtained graph can be a vertex of $K$, or it can be a {}\emph{fake vertex};
that is, it might be an intersection of an edge and a fake edge.

\begin{figure}[ht!]
\vskip-0mm
\centering
\includegraphics{mppics/pic-50}
\vskip-0mm
\end{figure}

For a usual vertex, the local lemma can be proved the same way.
For a fake vertex $v$, it is easy to see that both parts of the edge coming thru $v$ are marked with minus
while both of the fake edges at $v$ are marked with plus.
Therefore, we still have at least four sign changes at $v$.
The remaining argument works as before.
\qeds

Let us also state the following result of Alexey Pogorelov \cite[chapter III]{pogorelov}.

\begin{thm}{Theorem}
Any two convex bodies in $\EE^3$ with isometric surfaces are congruent.

Moreover, any isometry between surfaces of convex bodies can be extended to an isometry of the whole $\EE^3$.
\end{thm}

At first glance, this theorem might look like a small improvement of Alexandrov's uniqueness,
but this improvement is huge.
The proof is quite hard.
Let us just mention that it would follow if any two polyhedra $K$ and $K'$  with close surfaces in the sense of Gromov--Hausdorff would be almost congruent;
that is, there is a motion $\mu$ of $\EE^3$ such that the Hausdorff distance from $K$ to $\mu(K')$ is small.


\section{Existence theorem}

By \ref{prop:poly-CBB}, \ref{clm:total-angle}, and \ref{ex:surf-S2}, the surface of a convex polyhedron is an $\Alex0$ and homeomorphic to the sphere.
Alexandrov's theorem states that the converse holds if one includes in the consideration \textit{twice covered polygons}.
In other words, we have to consider a plane polygon as a degenerate polyhedron;
in this case, its surface is defined as the doubling of the polygon across its boundary.

Further, we assume that a polyhedron can degenerate to a plane polygon.

\begin{thm}{Theorem}\label{thm:alexandrov-first}
A polyhedral metric on the two-sphere is isometric to the surface of a convex polyhedron if and only if it has nonnegative curvature.

\end{thm}

Applying the approximation theorem (\ref{thm:approximation}) and \ref{ex:surface-covergence}, we get the following statement.
Here we again assume that a convex body can degenerate to a convex plane figure,
and, in this case, its surface is defined as the doubling of the figure across its boundary.

\begin{thm}{Corollary}\label{cor:Alex0-convex}
A metric on the two-sphere is $\Alex0$ if and only if it is isometric to the surface of a convex body (possibly degenerate).

\end{thm}

The proof of the existence theorem will be discussed in the following two sections.
It is instructive to solve the following exercise before going further.

\begin{thm}{Exercise}\label{ex:alexandrov=<4}
Let $\spc{P}$ be the 2-sphere equipped with a polyhedral metric with nonnegative curvature.

\begin{subthm}{ex:alexandrov=<4:>=3}
Prove that $\spc{P}$ has at least 3 essential vertices.
\end{subthm}

\begin{subthm}{ex:alexandrov=<4:=3}
If $\spc{P}$ has exactly 3 essential vertices $u$, $v$, and $w$, then it is isometric to the doubling of the solid model triangle $\modtrig(uvw)$.
\end{subthm}

\begin{subthm}{ex:alexandrov=<4:4}
If $\spc{P}$ has exactly 4 essential vertices, then it is isometric to the surface of a tetrahedron (possibly degenerate to a quadrangle).
\end{subthm}

\end{thm}

\section{Reformulation}

In this section, we introduce several notions and use them to reformulate the existence theorem (\ref{thm:reformulation}).

\paragraph{Space of polyhedrons.}
Let us denote by $\bm{K}$ the space of all convex polyhedrons in the Euclidean space,
including polyhedrons that degenerate to a plane polygon.
Polyhedrons in $\bm{K}$ will be considered up to a motion of the space; we will not distinguish between a convex polyhedron and its congruence class.

The space $\bm{K}$ will be considered with the topology induced by the {}\emph{Hausdorff metric up to a motion};
that is, the distance between (equivalence classes of) polyhedrons $K$ and $L$ is defined by
\[\dist{K}{L}{}\df \inf_\mu \{\dist{K}{\mu(L)}{\Haus}\},\]
where $\mu$ runs among all motions of $\EE^3$.

We say that a polyhedron $K$ in $\bm{K}$ has \emph{no symmetries} if  $K\z\ne \mu(K)$ for any nontrivial motion $\mu$ of $\EE^3$.
The set of all polyhedrons without symmetry in $\bm{K}$ will be denoted by $\bm{K}^\circ$.
Observe that $\bm{K}^\circ$ is an open set in $\bm{K}$.

Further, denote by $\bm{K}_n$ the polyhedrons in $\bm{K}$ with exactly $n$ vertices, and let $\bm{K}_n^\circ=\bm{K}_n\cap \bm{K}^\circ$.
Since any polyhedron has at least 3 vertices, the space $\bm{K}$ admits a subdivision into a countable number of subsets $\bm{K}_3,\bm{K}_4,\dots$

\paragraph{Space of surfaces.}
The space of polyhedral surfaces with nonnegative curvature that are homeomorphic to the 2-sphere will be denoted by $\bm{P}$.
The surfaces in $\bm{P}$ will be considered up to an isometry, and the whole space $\bm{P}$ will be equipped with the natural topology induced by the Gromov--Hausdorff metric.

We say that a surface $\spc{P}$ in $\bm{P}$ has \emph{no symmetries} if there is no nontrivial isometry
$\mu\:\spc{P}\to \spc{P}$.
The set of all surfaces without symmetry in $\bm{P}$ will be denoted by $\bm{P}^\circ$.
Observe that $\bm{P}^\circ$ is an open set in $\bm{P}$.

The subset of $\bm{P}$ of all surfaces with exactly $n$ essential vertices will be denoted by $\bm{P}_n$; let $\bm{P}_n^\circ=\bm{P}_n\cap \bm{P}^\circ$.
By \ref{ex:alexandrov=<4:>=3}, any surface in $\bm{P}$ has at least 3 essential vertices.
Therefore $\bm{P}$ is subdivided into countably many subsets
 $\bm{P}_3,\bm{P}_4,\dots$

\paragraph{From a polyhedron to its surface.}
Recall that the surface of a convex polyhedron is a sphere with nonnegative curvature.
Therefore, passing from a polyhedron to its surface defines a map
\[\iota\:\bm{K}\to \bm{P}.\]

Note that the existence theorem (\ref{thm:alexandrov-first}) follows from the next statement.

\begin{thm}{Theorem}\label{thm:reformulation}
For any integer $n\ge 3$,
the map $\iota$ is a bijection from $\bm{K}_n$ to~$\bm{P}_n$.
\end{thm}

\section{About the proof of existence}

By \ref{ex:surface-covergence}, the map $\iota\:\bm{K}\to\bm{P}$ is continuous.
Combining \ref{clm:total-angle} with the uniqueness theorem (\ref{thm:alexandrov-uni'}), we get that $\iota(\bm{K}_n)\subset \bm{P}_n$ and the map $\iota\:\bm{K}_n\to\bm{P}_n$ is injective.
It remains to prove the following.

\begin{thm}{Claim}\label{clm:surjective}
For any $n\ge 3$, the map $\iota\:\bm{K}_n\to\bm{P}_n$ is surjective.
\end{thm}

The proof is based on the construction of a one-parameter family of polyhedrons that starts at an arbitrary polyhedron
and ends at a polyhedron with its surface isometric to the given surface $\spc{P}$.
This type of argument is called the \index{continuity method}\emph{continuity method}; it is often used in the theory of differential equations.

\medskip

Now let us get into details.
First, observe that the second part of the uniqueness theorem (\ref{thm:alexandrov-uni'}) implies that $\iota(\bm{K}_n^\circ)\subset \bm{P}_n^\circ$.

\begin{thm}{Lemma}\label{lem:connected}
For any integer $n\ge 4$, the space $\bm{P}_n^\circ$ is connected and dense in $\bm{P}_n$.
\end{thm}

Note that $\bm{P}_3^\circ=\emptyset$;
indeed the surface of a triangle admits a reflection symmetry.
The case $n=4$ can be deduced from \ref{ex:alexandrov=<4:4}; thus, we can assume that $n\ge 5$.

The second statement is proved by a general-position-type argument.

The proof of the first statement is not complicated, but it requires ingenuity;
it can be done by the direct construction of a one-parameter family of surfaces in $\bm{P}_n^\circ$ that connects two given surfaces.
Such a family can be obtained as a sequence of the following deformations (direct or reversed).

Start with a surface $\spc{P}$ from $\bm{P}_n^\circ$.
Suppose $v$ and $w$ are essential vertices in $\spc{P}$.
Let us cut $\spc{P}$ along a shortest path from $v$ to~$w$.
This way we obtain a sphere with a hole.
The hole can be patched by a disc so that the obtained surface remains in $\bm{P}_n$.
In particular, the obtained surface has exactly $n$ essential vertices;
note that after the patching, the vertices $v$ and $w$ may become inessential.
(There is a three-parameter family of such patches, so we have something to choose from.)
Choosing a one-parameter family of such patches, we can get a deformation of~$\spc{P}$.

Again, applying a general-position-type argument to the above construction, we get a path in $\bm{P}_n^\circ$, assuming that the starting and ending surfaces are in $\bm{P}_n^\circ$.

\begin{thm}{Lemma}\label{lem:open}
The map $\iota\:\bm{K}_n^\circ\to\bm{P}_n^\circ$ is open,
that is, it maps any open set in $\bm{K}_n^\circ$ to an open set in $\bm{P}_n^\circ$.

In particular, for any $n\ge 3$, the image $\iota(\bm{K}_n^\circ)$ is open in~$\bm{P}_n^\circ$.
\end{thm}

This statement follows from the so-called \index{invariance of domain}\emph{invariance of domain theorem},
which states that a \textit{continuous injective map between manifolds of the same dimension is open}.

Recall that $\iota$ defines a continuous and injective $\bm{K}_n^\circ\to\bm{P}_n^\circ$.
It remains to check that both spaces $\bm{K}_n^\circ$ and $\bm{P}_n^\circ$ are $(3\cdot n-6)$-dimensional manifolds.

Choose a polyhedron $K$ in $\bm{K}_n$.
It is uniquely determined by the $3\cdot n$ coordinates of its $n$ vertices.
We can assume that the first vertex is at the origin,
the second has a positive $x$-coordinate
and the remaining two coordinates vanish,
and the third has a vanishing $z$-coordinate and a positive $y$-coordinate.
Therefore, all polyhedrons in $\bm{K}_n$ that lie sufficiently close to $K$ can be described by $3\cdot n-6$ parameters.
If $K$ has no symmetries, then this description is one-to-one;
in this case, a neighborhood of $K$ in $\bm{K}_n$ admits a parametrization by an open set in $\RR^{3\cdot n-6}$.

The case of surfaces is analogous.
We need to construct a subdivision of the sphere into plane triangles using only essential vertices.
By Euler's formula, there are exactly $3\cdot n-6$ edges in this subdivision.
The lengths of the edges completely describe the surface $\spc{P}$ and any surface near by.
If the surface has no symmetries, then this description is one-to-one, and a neighborhood of $\spc{P}$ in $\bm{P}_n$ admits a parametrization by an open set in  $\RR^{3\cdot n-6}$.

\begin{thm}{Lemma}\label{lem:closed}
The map $\iota\:\bm{K}_n\to\bm{P}_n$ is closed;
that is, the image of a closed set in $\bm{K}_n$ is closed in $\bm{P}_n$.

In particular, for any $n\ge 3$, the set $\iota(\bm{K}_n)$ is closed in~$\bm{P}_n$.
\end{thm}

Choose a sequence of polyhedrons $K_1,K_2,\ldots$ in $\bm{K}_n$.
Assume that the sequence $\spc{P}_i=\iota(K_n)$ converges in $\bm{P}_n$ as $i\to \infty$;
denote its limit by $\spc{P}_\infty$.
We need to construct a polyhedron $K_\infty\in \bm{K}_n$ such that $\iota(K_\infty)=\spc{P}_\infty$;
let us do it.

Passing to a subsequence, we can assume that $K_i$ converges in $\bm{K}$;
denote the limit polyhedron by $K_\infty$.
Since $\iota$ is continuous, $\iota(K_i)$ converges to $\iota(K_\infty)$ in~$\bm{P}$; so, $\iota(K_\infty)=\spc{P}_\infty$.
Recall that $\iota(\bm{K}_m)\subset\bm{P}_m$ for each $m$; therefore, $K_\infty\in \bm{K}_n$.


\parit{Proof of \ref{clm:surjective}.}
The case $n\le 4$ is already solved in \ref{ex:alexandrov=<4}; so we assume that $n\ge 5$.
By \ref{lem:closed} and \ref{lem:open},
$\iota(\bm{K}_n^\circ)$ is a non-empty closed and open set in $\bm{P}_n^\circ$, and $\bm{P}_n^\circ$ is connected.
Therefore, $\iota(\bm{K}_n^\circ)=\bm{P}_n^\circ$.

By \ref{lem:closed}, $\iota(\bm{K}_n)$ is closed in $\bm{P}_n$.
By \ref{lem:connected}, $\bm{P}_n^\circ$ is dense in $\bm{P}_n$.
Since $\iota(\bm{K}_n^\circ)=\bm{P}_n^\circ$, we have $\bm{P}_n^\circ\subset \iota(\bm{K}_n)$;
therefore, $\iota(\bm{K}_n)=\bm{P}_n$;
that is, $\iota\:\bm{K}_n\z\to\bm{P}_n$ is surjective.
\qeds

\section{Ambient space}

On one hand the Alexandrov surface theory is simpler since it has extra tools,
but it is also more complicated since these tools produce more questions.
The following result of Joseph Liberman \cite{liberman} gives an example.

\begin{thm}{Theorem}
Any geodesic in the surface of a convex body is one-sided differentiable as a curve in $\EE^3$.
\end{thm}

\parit{Proof.}
Let $\gamma$ be a geodesic on the surface of a convex body $K$.
Choose $p\in K$.
By \ref{ex:liberman}, the function $f_p\:t\mapsto \distfun_p\circ\gamma(t)$ is semiconcave for any $p\in K$.
In particular, one-sided derivatives $f_p^+(t)$ are defined for every $t$.

Given $x=\gamma(t)$, choose three points $p_1,p_2,p_3\in K$ in general position;
that is, the four points $x,p_1,p_2,p_3$ do not lie in one plane.
Observe that the distance functions $\distfun_{p_i}$ give smooth coordinates in a neighborhood of $x$.
From above the functions $f_{p_i}$ have one-sided derivatives at $t$.
Since the coordinates are smooth, we get that $\gamma^+(t)$ is defined as well.
\qeds




\begin{thm}{Exercise}\label{ex:convex}
Suppose a plane $\Pi$ cuts from the surface of a convex body $K$ a disc $\Delta$, and the reflection of $\Delta$ across $\Pi$ lies in $K$.
Show that $\Delta$ is a convex subset of the surface;
that is, if a geodesic has endpoints in $\Delta$, then it completely lies in $\Delta$.
\end{thm}

The following exercise gives a more exact version of comparison for convex surfaces;
it is due to Anatolii Milka \cite[Theorem 2]{milka1982}.

\begin{thm}{Very advanced exercise}\label{ex:milka}
Let $\spc{P}$ be the surface of a nondegenerate convex body $K\subset\EE^3$,
and let $\gamma_1$ and $\gamma_2$ be geodesic paths in $\spc{P}$ that start at one point $p\z=\gamma_1(0)\z=\gamma_2(0)$.
Suppose $x_i=\gamma_i(1)$, and $y_i\z=p+\gamma_i^+(0)$.
Show that
\[\dist{x_1}{x_2}{\spc{P}}\le \dist{y_1}{y_2}{W},\]
where $W$ is the complement to the interior of $K$.

\end{thm}



\section{Remarks}


The statement of Cauchy's theorem was conjectured by Adrien-Marie Legendre at the end of the 18$^\text{th}$ century;
a formulation was given in the first edition of his geometry textbook \cite{legendre}.
It was motivated by a vague definition in Euclid's Elements, which could be interpreted as
\textit{polyhedrons are equal if the same holds for their faces}.

The local lemma was already known to Legendre.
Legendre discussed this question with his colleague Joseph-Louis Lagrange, who suggested this problem to Augustin-Louis Cauchy in 1813; soon he proved it \cite{cauchy}.

The key observation that the face-to-face condition can be removed was made by
Alexandr Alexandrov in 1941; in the same paper he proved the uniqueness theorem \cite{alexandrov-1941}.
A quite different proof was found by Yurii Volkov in his thesis \cite{volkov}; it uses a deformation of three-dimensional polyhedral space.
(Be aware that the proof of this theorem given in the book by Igor Pak contains an essential mistake \cite{petrunin-2023}.)

In Cauchy's proof \cite{cauchy}, it was deducted from an analog of the following lemma.
Cauchy made a small mistake in its proof that was fixed in a century \cite{sabitov}.
Several proofs of the arm lemma can be found in the letters between Isaac Schoenberg and Stanisław Zaremba \cite{schoenberg-zaremba}.

\begin{thm}{Arm lemma}\label{lem:arm}
Assume that $A=[a_0 a_1\dots a_n]$ is a convex polygon in $\EE^2$
and $A'=[a'_0 a'_1\dots a'_n]$ is a polygonal line in $\EE^3$
such that
$|a_i-a_{i+1}|=|a'_i-a'_{i+1}|$ for any $i\in\{0,\dots,n-1\}$
and
$\measuredangle a_i\le \measuredangle a'_i$
for each $i\in\{1,\dots,n-1\}$.
Then
$$|a_0-a_n|\le |a'_0-a'_n|$$
and equality holds if and only if $A$ is congruent to $A'$.
\end{thm}

The following variation of the arm lemma makes sense for nonconvex spherical polygons.
It is due to Viktor Zalgaller \cite{zalgaller}.
It can be used instead of the standard arm lemma.

\begin{thm}{Another arm lemma}
Let $A=[a_1\dots a_n]$ and $A'\z=[a'_1\z\dots a'_n]$ be two spherical $n$-gons (not necessarily convex).
Assume that $A$ lies in a half-sphere,
the corresponding sides of $A$ and $A'$ are equal,
and each angle of $A$ is at least the corresponding angle in $A'$.
Then $A$ is congruent to~$A'$.
\end{thm}

Another close relative of the arm lemma is Reshetnyak's majorization theorem \cite{reshetnyak}.

Alexandrov gave two proofs of the global lemma \cite[2.1.2 and 2.1.3]{alexandrov}.
The first is combinatorial, and the second is more visual.
The argument in the second proof was reused by Anton Klyachko \cite{klyachko} in his \index{car-crash lemma}\emph{car-crash lemma}.

Proposition \ref{prop:conv-surf-CBB(0)} generalizes to the boundaries of convex bodies  in $\EE^m$ for any $m\ge 2$.
It could be considered as a partial case of the conjecture about the boundary of Alexandrov space; see \ref{conj:bry}.
Another partial case, for Riemannian manifolds with boundary, is proved by the authors and Stephanie Alexander \cite{alexander-kapovitch-petrunin-2008}.


\begin{wrapfigure}{r}{30mm}
\vskip-3mm
\centering
\includegraphics{mppics/pic-15}
\vskip-0mm
\end{wrapfigure}

According to the uniqueness theorem, a convex polyhedron is completely defined by the intrinsic metric of its surface.
In particular, knowing the metric, we could find the position of the edges.
However, in practice, it is not easy to do.
For example, the surface glued from a rectangle, as shown in the picture, defines a tetrahedron.
Some of the glued lines appear inside the facets of the tetrahedron, and some edges (dashed lines) do not follow the sides of the rectangle.

