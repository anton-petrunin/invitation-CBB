
%%!TEX root = the-prelim.tex
\mainmatter

\chapter{Preliminaries}\label{chap:prelim}

\section{Prerequisite}

We assume that the reader is familiar with the following topics in metric geometry:
\begin{itemize}
\item Compactness and proper metric spaces;
recall that a metric space is \index{proper space}\emph{proper} if all its closed balls (with finite radius) are compact.
\item Complete metric spaces and completion.
\item Curves, semicontinuity of length and rectifiability.
\item Hausdorff and Gromov--Hausdorff convergence.
These are discussed briefly in \ref{sec:Hausdorff convergence}--\ref{sec:Gromov--Hausdorff-metric}, but it is better to have prior acquaintance with these convergences.
\end{itemize}
These topics are treated in \cite{burago-burago-ivanov} and \cite{petrunin2023pure}.
Occasionally we use Baire category theorem and Rademacher's theorem; these could be used as black boxes.

We use some topology. 
Most of the time any introductory text in algebraic topology should be sufficient.
For some examples, we use more advanced results, but these could be used as black boxes.

Most of applications come from Riemannian geometry.
It is better to be familiar with the Toponogov comparison theorem.
The classical book of Jeff Cheeger and David Ebin \cite{cheeger-ebin} contains more than you will need.

\section{Notations}

The distance between two points $x$ and $y$ in a metric space $\spc{X}$ will be denoted by \index{$\dist{x}{y}{}$ and $\dist{x}{y}{\spc{X}}$ (distance)}$\dist{x}{y}{}$ or $\dist{x}{y}{\spc{X}}$.\label{page:|x-y|X}
The latter notation is used if we need to emphasize 
that the distance is taken in the space~${\spc{X}}$.

Given radius $r\in[0,\infty]$ and center $x\in \spc{X}$, the sets
\begin{align*}
\oBall(x,r)&=\set{y\in \spc{X}}{\dist{x}{y}{}<r},
\\
\cBall[x,r]&=\set{y\in \spc{X}}{\dist{x}{y}{}\le r}
\end{align*}
are called, respectively, the \index{open ball}\emph{open} and  the \index{closed ball}\emph{closed  balls}.
The notations $\oBall(x,r)_{\spc{X}}$ and $\cBall[x,r]_{\spc{X}}$
might be used if we need to emphasize that these balls are taken in the metric space $\spc{X}$.

We will denote by $\SSS^n$, $\EE^n$, and $\HH^n$ the $n$-dimensional sphere (with angle metric), 
Euclidean space, and Lobachevsky space respectively.
More generally, $\MM^n(\kappa)$ will denote the \emph{model $n$-space} of curvature $\kappa$;
that is,
\begin{itemize}
\item if $\kappa>0$, then $\MM^n(\kappa)$ is the $n$-sphere of radius $\tfrac{1}{\sqrt{\kappa}}$, so $\SSS^n=\MM^n(1)$
\item $\MM^n(0)=\EE^n$,
\item if $\kappa<0$, then $\MM^n(\kappa)$ is the Lobachevsky $n$-space $\HH^n$ rescaled by factor $\tfrac{1}{\sqrt{-\kappa}}$;
in particular $\MM^n(-1)=\HH^n$.
\end{itemize}

\section{Length spaces}\label{sec:length}

Let $\spc{X}$ be a metric space.
If for any $\eps>0$ and any pair of points $x,y\in\spc{X}$, there is a path $\alpha$ connecting $x$ to $y$ such that
\[\length\alpha< \dist{x}{y}{}+\eps,\]
then $\spc{X}$ is called a \index{length space}\emph{length space} and the metric on $\spc{X}$ is called a \index{length metric}\emph{length metric}.\label{page:length metric}

\begin{thm}{Exercise}\label{ex:compact+connceted}
Let $\spc{X}$ be a complete length space.
Show that for any compact subset $K\subset\spc{X}$
there is a compact path-connected subset $K'\subset\spc{X}$ that contains $K$.  
\end{thm}

\parbf{Induced length metric.}
Directly from the definition, it follows that if $\alpha\:[0,1]\to\spc{X}$ is a path from $x$ to $y$ 
(that is, $\alpha(0)=x$ and $\alpha(1)=y$), then 
\[\length\alpha\ge \dist{x}{y}{}.\]
Set 
\[\yetdist{x}{y}{}=\inf\{\,\length\alpha\,\}\]
where the greatest lower bound is taken for all paths from $x$ to~$y$.
It is straightforward to check that $(x,y)\mapsto \yetdist{x}{y}{}$ is an \emph{$\infty$-metric};
that is, $(x,y)\mapsto \yetdist{x}{y}{}$ is a metric in the extended positive reals $[0,\infty]$. 
The metric $\yetdist{*}{*}{}$ is called the \index{induced length metric}\emph{induced length metric}.

\begin{thm}{Exercise}\label{ex:compact=>complete}
Suppose $(\spc{X},\dist{*}{*}{})$ is a complete metric space.
Show that $(\spc{X},\yetdist{*}{*}{})$ is complete;
that is, any Cauchy sequence of points in $(\spc{X},\yetdist{*}{*}{})$ converges in $(\spc{X},\yetdist{*}{*}{})$.
\end{thm}

Let $A$ be a subset of a metric space $\spc{X}$.
Given two points $x,y\in A$,
consider the value
\[\dist{x}{y}{A}=\inf_{\alpha}\{\,\length\alpha\,\},\]
where the greatest lower bound is taken for all paths $\alpha$ from $x$ to $y$ in~$A$.
In other words, $\dist{*}{*}{A}$ denotes the induced length metric on the subspace $A$.
(The notation $\dist{*}{*}{A}$ conflicts with the previously defined notation for distance $\dist{x}{y}{\spc{X}}$ in a metric space $\spc{X}$.
However, most of the time we will work with ambient length spaces where the meaning will be unambiguous.)

\section{Geodesics}

Let $\spc{X}$ be a metric space 
and $\II$\index{$\II$} a real interval. 
A distance-preserving map $\gamma\:\II\to \spc{X}$ is called a \index{geodesic}\emph{geodesic}%
\footnote{Others call it differently: \textit{shortest path}, \textit{minimizing geodesic}.
Also, note that the meaning of the term \textit{geodesic} is different from what is used in Riemannian geometry, altho they are closely related.}; 
in other words, $\gamma\:\II\z\to \spc{X}$ is a geodesic if 
\[\dist{\gamma(s)}{\gamma(t)}{}=|s-t|\]
for any pair $s,t\in \II$.

If $\gamma\:[a,b]\to \spc{X}$ is a geodesic such that $p=\gamma(a)$, $q=\gamma(b)$, then we say that $\gamma$ is a geodesic from $p$ to $q$.
In this case, the image of $\gamma$ is denoted by $[p q]$\index{$[{*}{*}]$}, and, with abuse of notations, we also call it a \index{geodesic}\emph{geodesic}.
We may write $[p q]_{\spc{X}}$ 
to emphasize that the geodesic $[p q]$ is in the space  ${\spc{X}}$.

In general, a geodesic from $p$ to $q$ need not exist and if it exists, it need not  be unique;
for example, any meridian is a geodesic between poles on the sphere.
However, once we write $[p q]$ we assume that we have chosen such a geodesic.

A \index{geodesic path}\emph{geodesic path} is a geodesic with constant-speed parameterization by the unit interval $[0,1]$.

A metric space is called \index{geodesic space}\emph{geodesic} if any pair of its points can be joined by a geodesic.

Evidently, any geodesic space is a length space.

\begin{thm}{Exercise}\label{ex:compact-length}
Show that any proper length space is geodesic.
\end{thm}

\section{Menger's lemma}

\begin{thm}{Lemma}\label{lem:mid>geod}
Let $\spc{X}$ be a complete metric space.
Assume that for any pair of points $x,y\in \spc{X}$, 
there is a midpoint~$z$.
Then $\spc{X}$ is a geodesic space.

\end{thm}

This lemma is due to Karl Menger \cite[Section 6]{menger}.

%???+PIC!!!

\parit{Proof.}
Choose $x,y\in \spc{X}$;
set $\gamma(0)=x$, and $\gamma(1)=y$.

\begin{figure}[ht!]
\vskip-0mm
\centering
\includegraphics{mppics/pic-104}
\end{figure}

Let $\gamma(\tfrac12)$ be a midpoint between $\gamma(0)$ and $\gamma(1)$.
Further, let $\gamma(\frac14)$ 
and $\gamma(\frac34)$ be midpoints between the pairs $(\gamma(0),\gamma(\tfrac12))$ 
and $(\gamma(\tfrac12),\gamma(1))$ respectively.
Applying the above procedure recursively,
on the $n$-th step we define $\gamma(\tfrac{k}{2^n})$,
for every odd integer $k$ such that $0<\tfrac k{2^n}<1$, 
as a midpoint of the already defined
$\gamma(\tfrac{k-1}{2^n})$ and $\gamma(\tfrac{k+1}{2^n})$.

This way we define $\gamma(t)$ for all dyadic rationals $t$ in $[0,1]$.
Moreover, $\gamma$ has Lipschitz constant $\dist{x}{y}{}$.
Since $\spc{X}$ is complete, the map $\gamma$ can be extended continuously to $[0,1]$.
Moreover,
\[
\length\gamma\le \dist{x}{y}{}.
\]
Therefore $\gamma$ is a geodesic path from $x$ to $y$.
\qedsf

\begin{thm}{Exercise}\label{ex:menger}
Let $\spc{X}$ be a complete metric space.
Assume that for any pair of points $x,y\in \spc{X}$, 
there is an \index{almost midpoint}\emph{almost midpoint};
that is, given $\eps>0$, there is a point $z$ such that 
\[\dist{x}{z}{}<\tfrac12\cdot\dist{x}{y}{}+\eps 
\quad\text{and}\quad
\dist{y}{z}{}<\tfrac12\cdot\dist{x}{y}{}+\eps.\]
Show that $\spc{X}$ is a length space.
\end{thm}


\section{Triangles and model tangles}

\parbf{Triangles.}
Given a triple of distinct points $p,q,r$ in a metric space $\spc{X}$, a choice of geodesics $([q r], [r p], [p q])$ will be called a \index{triangle}\emph{triangle}; we will use the short notation 
$\trig p q r=\trig p q r_{\spc{X}}=([q r], [r p], [p q])$\index{$\trig {{*}}{{*}}{{*}}$}.

Given a triple $p,q,r\in \spc{X}$ there may be no triangle 
$\trig p q r$ simply because one of the pairs of these points cannot be joined by a geodesic.
Also, many different triangles with these vertices may exist, any of which can be denoted by $\trig p q r$.
If we write $\trig p q r$, it means that we have chosen such a triangle.


\parbf{Model triangles.}
Given three points $p,q,r$ in a metric space $\spc{X}$,
let us define its \index{model triangle}\emph{model triangle} $\trig{\tilde p}{\tilde q}{\tilde r}$ 
(briefly, 
$\trig{\tilde p}{\tilde q}{\tilde r}=\modtrig(p q r)_{\EE^2}$%
\index{$\modtrig$}) to be a triangle in the Euclidean plane $\EE^2$ such that
\begin{align*}\dist{\tilde p}{\tilde q}{\EE^2}&=\dist{p}{q}{\spc{X}},
&
\quad\dist{\tilde q}{\tilde r}{\EE^2}&=\dist{q}{r}{\spc{X}},
&
\quad\dist{\tilde r}{\tilde p}{\EE^2}&=\dist{r}{p}{\spc{X}}.
\end{align*}

In the same way, we can define the \index{hyperbolic model triangle}\emph{hyperbolic} and the \index{spherical model triangles}\emph{spherical model triangles} $\modtrig(p q r)_{\HH^2}$, $\modtrig(p q r)_{\SSS^2}$
in the Lobachevsky plane $\HH^2$ and the unit sphere~$\SSS^2$.
In the latter case, the model triangle is said to be defined if in addition
\[\dist{p}{q}{}+\dist{q}{r}{}+\dist{r}{p}{}< 2\cdot\pi.\]
In this case, the model triangle again exists and is unique up to an isometry of~$\SSS^2$.

\parbf{Model angles.}
If 
$\trig{\tilde p}{\tilde q}{\tilde r}=\modtrig(p q r)_{\EE^2}$ 
and $\dist{p}{q}{},\dist{p}{r}{}>0$, 
the angle measure of 
$\trig{\tilde p}{\tilde q}{\tilde r}$ at $\tilde p$ 
will be called the \index{model angle}\emph{model angle} of the triple $p$, $q$, $r$ and will be denoted by
$\angk p q r_{\EE^2}$%
\index{$\angk{p}{q}{r}$ (model angle)}.\label{page:model-angle}

For example, if $\dist{p}{q}{}=\dist{q}{r}{}=\dist{r}{p}{}$, then $\angk p q r_{\EE^2}=\tfrac\pi3$ regardless of existence and relative position of geodesics $[pq]$ and $[pr]$.

The same way we define $\angk p q r_{\MM^2(\kappa)}$;
in particular, $\angk p q r_{\HH^2}$ and $\angk p q r_{\SSS^2}$.
We may use the notation $\angk p q r$ if it is evident which of the model spaces is meant.

\begin{thm}{Exercise}\label{ex:k-><mono}
Show that for any triple of point $p$, $q$, and $r$,
the function
\[\kappa\mapsto \angk p q r_{\MM^2(\kappa)}\]
is nondecreasing in its domain of definition.
\end{thm}


\section{Hinges and their angle measure}\label{sec:angles}

\parbf{Hinges.} Let $p,x,y\in \spc{X}$ be a triple of points such that $p$ is distinct from $x$ and~$y$.
A pair of geodesics $([p x],[p y])$ will be called  a \index{hinge}\emph{hinge} and will be denoted by 
$\hinge p x y=([p x],[p y])$\index{$\hinge{{*}}{{*}}{{*}}$}.

\parbf{Angles.}
The angle measure of a hinge $\hinge p x y$ is defined as the following limit
\[\mangle\hinge p x y=\lim_{\bar x,\bar y\to p} \angk p{\bar x}{\bar y},\]
where $\bar x\in\left]p x\right]$ and $\bar y\in\left]p y\right]$.

Note that if $\mangle\hinge p x y$ is defined, then
\[0\le \mangle\hinge p x y\le \pi.\]

\begin{thm}{Exercise}\label{ex:angkK}
Suppose that in the above definition, one uses spherical or hyperbolic model angles instead of Euclidean.
Show that it does not change the value $\mangle\hinge p x y$.
\end{thm}


\begin{thm}{Exercise}\label{ex:undefined-angle}
Give an example of a hinge $\hinge p x y$ in a metric space with an undefined angle measure $\mangle\hinge p x y$.
\end{thm}

\section{Triangle inequality for angles}

\begin{thm}{Proposition}\label{claim:angle-3angle-inq}
Let  $[px_1]$, $[px_2]$, and $[px_3]$ be three geodesics in a metric space.
Suppose all the angle measures $\alpha_{i j}=\mangle\hinge p {x_i}{x_j}$ are defined.
Then 
\[\alpha_{13}\le \alpha_{12}+\alpha_{23}.\]

\end{thm}



\parit{Proof.}
Since $\alpha_{13}\le\pi$, we can assume that $\alpha_{12}+\alpha_{23}< \pi$.
Denote by $\gamma_i$ the unit-speed parametrization of $[px_i]$ from $p$ to $x_i$.
Given any $\eps>0$, for all sufficiently small $t,\tau,s\in\RR_{\ge0}$ we have
\begin{align*}
\dist{\gamma_1(t)}{\gamma_3(\tau)}{}
&\le 
\dist{\gamma_1(t)}{\gamma_2(s)}{}+\dist{\gamma_2(s)}{\gamma_3(\tau)}{}<\\
&<
\sqrt{t^2+s^2-2\cdot t\cdot  s\cdot \cos(\alpha_{12}+\eps)} +
\\
&\quad+\sqrt{s^2+\tau^2-2\cdot s\cdot \tau\cdot \cos(\alpha_{23}+\eps)}\le
\end{align*}

\begin{wrapfigure}{o}{30 mm}
\vskip-6mm
\centering
\includegraphics{mppics/pic-615}
\vskip6mm
\end{wrapfigure}

Below we define 
$s(t,\tau)$ so that for 
$s=s(t,\tau)$, this chain of inequalities can be continued as follows:
\[\le
\sqrt{t^2+\tau^2-2\cdot t\cdot \tau\cdot \cos(\alpha_{12}+\alpha_{23}+2\cdot \eps)}.
\]

Thus for any $\eps>0$, 
\[\alpha_{13}\le \alpha_{12}+\alpha_{23}+2\cdot \eps.\]
Hence the result follows.

To define $s(t,\tau)$, consider three half-lines $\tilde \gamma_1$, $\tilde \gamma_2$, $\tilde \gamma_3$ on a Euclidean plane starting at one point, such that
$\mangle(\tilde \gamma_1,\tilde \gamma_2)\z=\alpha_{12}+\eps$,
$\mangle(\tilde \gamma_2,\tilde \gamma_3)\z=\alpha_{23}+\eps$,
and $\mangle(\tilde \gamma_1,\tilde \gamma_3)\z=\alpha_{12}\z+\alpha_{23}\z+2\cdot \eps$.
We parametrize each half-line by the distance from the starting point.
Given two positive numbers $t,\tau\in\RR_{\ge0}$, let $s=s(t,\tau)$ be 
the number such that 
$\tilde \gamma_2(s)\in[\tilde \gamma_1(t)\ \tilde \gamma_3(\tau)]$. 
Clearly, $s\le\max\{t,\tau\}$, so $t,\tau,s$ may be taken sufficiently small.
\qeds 

\begin{thm}{Exercise}\label{ex:adjacent-angles}
Prove that the sum of adjacent angles is at least $\pi$.

More precisely: suppose two hinges $\hinge pxz$ and $\hinge pyz$ are \index{adjacent hinges}\emph{adjacent};
that is, they share side $[pz]$, and the union of two sides $[px]$ and $[py]$ form a geodesic $[xy]$.
Show that
\[\mangle\hinge pxz+\mangle\hinge pyz\ge \pi\]
whenever  each angle on the left-hand side is defined.

Give an example showing that the inequality can be strict.
\end{thm}

\begin{thm}{Exercise}\label{ex:first-var}
Assume that the angle measure of $\hinge q p x$ is defined.
Let $\gamma$ be the unit speed parametrization of $[qx]$ from $q$ to $x$.
Show that
\[\dist{p}{\gamma(t)}{}
\le
\dist{q}{p}{}-t\cdot \cos(\mangle\hinge q p x)+o(t).\]

\end{thm}

\section{Hausdorff convergence}\label{sec:Hausdorff convergence}

\begin{thm}{Definition}\label{def:gen-Haus-conv}
Let $A_1,A_2,\dots$ be a sequence of closed sets in a metric space $\spc{X}$.
We say that the sequence $A_n$ \index{Hausdorff limit}\emph{converges} to a closed set $A_\infty$ in the {}\emph{sense of Hausdorff} if, for any $x\in\spc{X}$, we have
$\distfun_{A_n}(x)\z\to \distfun_{A_\infty}(x)$ as $n\to\infty$.
\end{thm}

For example, suppose $\spc{X}$ is the Euclidean plane and $A_n$ is the circle with radius $n$ and center at the point $(0,n)$; it converges to the $x$-axis.

\begin{figure}[ht!]
\vskip-0mm
\centering
\includegraphics{mppics/pic-415}
\end{figure}

Further, consider the sequence of one-point sets $B_n=\{(n,0)\}$ in the Euclidean plane.
It converges to the empty set;
indeed, for any point $x$ we have $\distfun_{B_n}(x)\to\infty$ as $n\to \infty$ and $\distfun_{\emptyset}(x)= \infty$ for any~$x$.

The following exercise is an extension of the so-called Blaschke selection theorem to our version of Hausdorff convergence.

\begin{thm}{Exercise}\label{ex:generalized-selection}
Show that any sequence of closed sets in a proper metric space has a convergent subsequence in the sense of Hausdorff.
\end{thm}

\section{Hausdorff metric}

\begin{thm}{Definition}\label{def:hausdorff-convergence}
Let $A$ and $B$ be two nonempty compact subsets of a metric space $\spc{X}$.
Then the \index{Hausdorff distance}\emph{Hausdorff distance} between $A$ and $B$ is defined as 
$$|A-B|_{\Haus\spc{X}}
\df
\sup_{x\in \spc{X}}\{\,|\distfun_A(x)-\distfun_B(x)|\,\}.
$$

\end{thm}

The following observation gives a useful reformulation of the definition:

\begin{thm}{Observation}\label{obs:Haus-nbhds}
Suppose $A$ and $B$ be two compact subsets of a metric space $\spc{X}$.
Then $|A-B|_{\Haus\spc{X}}< R$ if and only if and only if 
$B$ lies in an $R$-neighborhood of $A$, 
and 
$A$ lies in an $R$-neighborhood of~$B$.
\end{thm}

The following exersice implies that for compact subsets the Hausdorff convergence is the convergence in Hausdorff metric.

\begin{thm}{Exercise}\label{ex:Haus-conv}
Let $A_1,A_2,\dots$ and $A_\infty$ be compact nonempty sets in a metric space $\spc{X}$.
Show that $\dist{A_n}{A_\infty}{\Haus\spc{X}}\to 0$ as $n\to\infty$
if and only if $A_n\to A_\infty$ in the sense of Hausdorff.
\end{thm}

\section{Gromov--Hausdorff convergence}\label{sec:Gromov--Hausdorff}

Let $\spc{X}_1,\spc{X}_2,\dots$ and $\spc{X}_\infty$ be a sequence of complete metric spaces.
Suppose that there is a metric on the disjoint union 
\[\bm{X}=\bigsqcup_{n\in \NN\cup\{\infty\}} \spc{X}_n\] 
that satisfies the following property:

\begin{thm}{Property}\label{propery:GH}
The restriction of metric on each $\spc{X}_n$ and $\spc{X}_\infty$ coincides with its original metric, 
and $\spc{X}_n\to \spc{X}_\infty$ as subsets in $\bm{X}$ in the sense of Hausdorff.
\end{thm}

In this case we say that the metric on $\bm{X}$ \textit{defines} a \index{Gromov--Hausdorff limit}\emph{convergence} $\spc{X}_n\z\to \spc{X}_\infty$ in the {}\emph{sense of Gromov--Hausdorff}.
The metric on  $\bigsqcup \spc{X}_n$ makes it possible to talk about limits of sequences $x_n\in \spc{X}_n$ as $n\to\infty$, as well as weak limits of a sequence of Borel measures $\mu_n$ on $\spc{X}_n$ and so on.

The limit space is not uniquely defined by the sequence.
For example, if each space $\spc{X}_n$ in the sequence is isometric to the half-line, then its limit might be isometric to the half-line or the whole line.
The first convergence is evident and the second could be guessed from the diagram.

\begin{figure}[ht!]
\vskip-0mm
\centering
\includegraphics{mppics/pic-500}
\end{figure}

Note that for any sequence of spaces has an empty space as its limit for a Gromov--Hausdorff convergence.
Exercise \ref{ex:compact-GH} states that if limit nonempty and compact, then it is unique up to isometry. 

\begin{thm}{Exercise}\label{ex:geod-closed}
Let $\spc{X}_1,\spc{X}_2,\dots$ be a sequence of geodesic metric spaces.
Suppose $\spc{X}_n\to \spc{X}_\infty$ is a convergence in the sense of Gromov--Hausdorff.
Assume $\spc{X}_\infty$ is proper, show that it is geodesic.
\end{thm}

\parbf{Pointed convergence.}
Often the isometry class of the limit can be fixed by marking a point $p_n$ in each space $\spc{X}_n$, it is called \index{pointed convergence}\emph{pointed Gromov--Hausdorff convergence} --- we say that $(\spc{X}_n,p_n)$ converges to $(\spc{X}_\infty,p_\infty)$ if there is a metric on $\bm{X}$ as in \ref{propery:GH} such that $p_n\to p_\infty$.
For example, the sequence $(\spc{X}_n,p_n)=(\RR_+,0)$ converges to $(\RR_+,0)$, while $(\spc{X}_n,p_n)=(\RR_+,n)$ converges to $(\RR,0)$.

\section{Gromov--Hausdorff metric}\label{sec:Gromov--Hausdorff-metric}

In this section we cook up a metric space out of all compact nonempty metric spaces
that defines the Gromov--Hausdorff convergence.
We want to define the metric on the set of \textit{isometry classes} of compact metric spaces.
Further, term \textit{metric space} might also stand for its \textit{isometry class}.

The obtained metric is called Gromov--Hausdorff metric;
the corresponding metric space will be denoted by $\GH$.
This distance is defined as the maximal metric such that \textit{the distance between subspaces in a metric space is not greater than the Hausdorff distance between them}.
Here is a formal definition.

\begin{thm}{Definition}\label{def:GH}
The \index{Gromov--Hausdorff distance}\emph{Gromov--Hausdorff distance} $|\spc{X}-\spc{Y}|_{\GH}$ between compact metric spaces $\spc{X}$ and $\spc{Y}$
is defined by the following
relation.
 
Given  $r > 0$, we have that $|\spc{X}-\spc{Y}|_{\GH} < r$ if and only if there exists a metric
space $\spc{W}$ and subspaces $\spc{X}'$ and $\spc{Y}'$ in $\spc{W}$ that are isometric to $\spc{X}$ and $\spc{Y}$,
respectively, such that $|\spc{X}'-\spc{Y}'|_{\Haus\spc{W}} < r$. 
(Here $|\spc{X}'-\spc{Y}'|_{\Haus\spc{W}}$ denotes the Hausdorff distance between sets $\spc{X}'$ and $\spc{Y}'$ in $\spc{W}$.)
\end{thm}

For the proof of the following statement we refer to \cite{burago-burago-ivanov} and \cite{petrunin2023pure}.

\begin{thm}{Proposition}\label{prop:complete}
$\GH$ is a complete metric space.
\end{thm}

Note that this means in particular that if $X,Y$ are compact and $|\spc{X}-\spc{Y}|_{\GH}=0$ then $X$ and $Y$ are isometric.

Gromov--Hausdorff convergence of compact spaces has particularly nice properties.
From the technical point of view they follow from the next statement that we formulate as an exercise.

\begin{thm}{Exercise}\label{ex:non-contracting-map}
Let $f$ be a distance noncontracting map from 
a compact metric space $\spc{K}$ to itself.
Show that $f$ is an isometry; that is, it is a distance-preserving bijection.
\end{thm}

For two metric spaces $\spc{X}$ and $\spc{Y}$,
we write $\spc{X}\le \spc{Y}+\eps$ if
there is a map $f\:\spc{X}\to \spc{Y}$ such that 
\[\dist{x}{x'}{\spc{X}}\le \dist{f(x)}{f(x')}{\spc{Y}}+\eps\]
for any $x,x'\in \spc{X}$.

\begin{thm}{Exercise}\label{ex:GH-po}
Let $\spc{X}_1,\spc{X}_2,\dots$ and $\spc{X}_\infty$ are compact metric spaces.
Show that there is a Gromov--Hausdorff convergence $\spc{X}_n\to\spc{X}_\infty$ if and only if for some sequence $\eps_n\to 0$,
we have 
\[\spc{X}_\infty\le \spc{X}_n+\eps_n\quad\text{and}\quad \spc{X}_n\le \spc{X}_\infty+\eps_n.\]
\end{thm}

\begin{thm}{Exercise}\label{ex:compact-GH}
Let $\spc{X}_1,\spc{X}_2,\dots$ be a sequence of metric spaces.
Suppose $\spc{X}_\infty$ and $\spc{X}_\infty'$ are nonempty limit spaces for some Gromov--Hausdorff convergences of $\spc{X}_n$.
Assume $\spc{X}_\infty$ is compact, show that it is isometric to~$\spc{X}_\infty'$.
\end{thm}

\section{Comments}

In principle, our definition of Gromov--Hausdorff distance works for complete metric spaces which are not necessarily compact.
However, according to the following exercise, it only defines semimetric;
by that reason it is not in use.

\begin{thm}{Exercise}\label{ex:GH-noncompact}
Construct two nonisometric proper geodesic spaces $\spc{X}$ and $\spc{Y}$ with vanishing Gromov--Hausdorff distance.
\end{thm}
