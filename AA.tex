

\parit{\ref{SHORT.thm:doubling:concave}.}
Let us apply induction on $m=\LinDim \spc{L}$.

\begin{wrapfigure}{r}{30mm}
\vskip-2mm
\centering
\includegraphics{mppics/pic-1305}
\end{wrapfigure}

Choose a geodesic $[pz]$; let $\gamma(0)=p$.
Suppose $p\notin\partial \spc{L}$.
Let $q\in \partial\spc{L}$ be a closest point to $p$ and $\alpha\df\mangle\hinge pzq$.

By the definition of boundary points, 
\[\partial \Sigma_q\z\ne\emptyset.\]
Let $\xi=\dir qp$.
Theorem~\ref{thm:partial-Sigma} implies that 
\[\dist{\xi}{\zeta}{\Sigma_q}\ge \tfrac\pi2
\eqlbl{eq:<>pi/2}\]
for any $\zeta\in\partial\Sigma_q$.

By \ref{thm:finite-space-of-directions}, $\Sigma_q$ is an $(m-1)$-dimensional geodesic $\CBB(1)$ space.
Applying the induction hypothesis, we get that the doubling $\hat\Sigma_q$ of $\Sigma_q$ across $\partial \Sigma_q$ is an $(m-1)$-dimensional geodesic $\CBB(1)$ space.
Denote by $\xi_1$ and $\xi_2$ the two directions in $\hat\Sigma_q$ that correspond to $\xi$.
Note that \ref{eq:<>pi/2} implies that $\dist{\xi_1}{\xi_2}{\hat\Sigma_q}\ge \pi$.
Applying the line splitting theorem (\ref{thm:splitting}), we can identify 
$\Cone\hat\Sigma_q$ with $\RR\times \partial\Sigma_q$.
It follows that 
\[\T_q=[0,\infty)\oplus \partial\T_q;\]
in particular, there is a natural projection $\proj\:\T_q\to \partial\T_q$.

Given $x\in [pz]$, choose a geodesic path $\gamma_x$ from $q$ to $x$.
Let 
\[y
\df
\gexp_q\circ\proj(\gamma^+_x(0)).\]
By \ref{thm:gexp}, $y\in \partial\spc{L}$ and 
\[\dist{x}{y}{}\le \dist{p}{q}{}+\dist{p}{x}{}\cdot \cos\alpha.\eqlbl{eq:|x-y|}\]
The latter inequality uses in addition the $\CBB$ comparison for $[pqx]$ and it requires some work.

Note that \ref{eq:|x-y|} implies that $f\circ\gamma$ is concave for any geodesic that lies in $\spc{L}\setminus \partial \spc{L}$.
If $\gamma(t)\in \partial \spc{L}$ for some $t$, then it is easy to see that $(f\circ\gamma)'(t)=0$.
These two statements imply that $f\circ\gamma$ is concave for any geodesic that lies in $\spc{L}$.
















\section{Morse theory}

Let $f$ be a semiconcave function.
A point $p\in \Dom f$ is called \emph{critical} point of $f$ if $\dd_pf\le 0$; 
otherwise it is called \emph{regular}.

The proof of the following statement is quite technical, we omit its proof.

\begin{thm}{Theorem}
Let $f$ be a semiconcave function on a finite-dimensional geodesic $\CBB(\kappa)$ space.
Suppose $K$ is a compact set of regular points of $f$ in its level set $f=a$.
Then an open neighborhood $\Omega$ of $K$ admits homeomorphism $x\mapsto (h(x),f(x))$ to a product space $\Lambda\times (a-\eps,a+\eps)$.

\end{thm}

Note that distance function $\distfun_p$ has no critical points in a neighborhood of $p$ and the level set $\distfun_p=\eps$ is compact for small $\eps>0$.
Combining this observation with 

Applying the theorem we get the following.

\begin{thm}{Corollary}
A small spherical neighborhood of any point $p$ in a finite-dimensional geodesic $\CBB(\kappa)$ space $\spc{L}$ is homeomorphic to an open cone over small sphere around $p$.
\end{thm}
