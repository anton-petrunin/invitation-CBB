\parbf{\ref{ex:arm'}};
\ref{SHORT.ex:bow'+}.
We can assume that $a_0=a'_0$ is the origin, $a_1=a'_1$ lies on the positive part of the $x$-axis,
and the points $a_n$ and $a'_n$ lie in the upper half-plane of the $(x,y)$-plane.

Suppose $\mangle a_0<\mangle a_0'$.
Find a point $x$ such that
$\dist{x}{a_n}{}<\dist{x}{a_n'}{}$, and $[x a_0\dots a_n]$ is a convex polygon in the $(x,y)$-plane.
(Here you have to use that $\mangle a_n\le\tfrac\pi2$.)

{

\begin{wrapfigure}{r}{47 mm}
\vskip-7mm
\centering
\includegraphics{mppics/pic-1118}
\vskip0mm
\end{wrapfigure}

Apply the arm lemma (\ref{lem:arm}) to $[x a_0\dots a_n]$ and $[x a_0'\dots a_n']$, and arrive at a contradiction.

\parit{\ref{SHORT.ex:bow'-}.} Look at the picture and think.

}

\parbf{\ref{ex:a<a}.}
Apply the angle monotonicity (\ref{angle-monotonicity}) couple of times.

\parbf{\ref{ex:disc-bend}.}

\parit{Source:} The second part of the exercise was added by Viktor Zalgaller to the translation of Alexandrov's book \cite{alexandrov}.
More statements of this type can be found in Zalgaller's classification of convex polyhdra with regular faces \cite{zalgaller}.

\parbf{\ref{ex:octahedron}.}

\parbf{\ref{ex:disc}.}
The exercise can be deduces from the Riemann mapping theorem.
Let us sketch the proof suggested by Fedor Petrov \cite{petrov};
it only used Jordan's theorem.

\medskip

It is sufficient to prove that an open simply connected domain $\Omega$ of the plane is homeomorphic to an open disc.
Let us represent $\Omega$ as a union $P_1\cup P_2\cup\ldots$, where all $P_i$ are (closed) polygons and $P_i\Subset P_{i+1}$ for all $i$; that is $P_i$ lies in the interior of $P_{i+1}$.
Apply the Jordan's theorem couple of times to show that there is a sequence of homeomorphisms $f_i$ form $P_i$ to the closed disc of radius $1-\tfrac1{2^i}$ such that $f_{i+1}$ extends $f_i$.
Given $x\in \Omega$, set $f(x)=f_i(x)$ for all sufficiently large $i$.
Observe that $f$ is a homeomorphism from $\Omega$ to the open unit disc.

Now, let us recursively construct our system of polygons.
Note that $\Omega$ can be presented as a union of a countable collection of coordinate squares, say $S_1,S_2,\dots$
Set $P_1=S_1$.
Suppose $P_1,\dots,P_{i-1}$ are constructed,
Add to $P_{i-1}$ several squares that cover its boundary together with $S_i$.
If necessary, add to it a finite number of squares so that the union becomes connected.
The obtained set may contain holes, but since $\Omega$ is simply connected everything in holes belongs to $\Omega$, add all this to the set.
Denote the obtained set by $P_i$.


\parbf{\ref{pr:K-P-simmetry}.}




















\parbf{\ref{ex:native}.}
Choose a geodesic $\gamma$ in $\spc{W}$.
Arguing as in the proof of \ref{thm:doubling:doubling}, we get 
that $\gamma$ can cross the common boundary of two halves $\spc{A}_0$ $\spc{A}_1$ of $\spc{W}$ at most once, or it lies in the common boundary.

In the later case $\lambda$-convexity of $f\circ\proj$ follows from $\lambda$-convexity of $f$.
In the former case the convexity has to be checked only at the point of crossing;
we may assume that it happens at $x=\gamma(0)$.
Since $\nabla_x f\in \T_x\partial\spc{A}$, we have 
\[\dd_x(f\circ\proj)(v)\le\langle\nabla_x f,v\rangle\]
Argue as in the proof of \ref{thm:doubling:doubling}, show that 
$\langle\gamma^+(0),v\rangle+\langle\gamma^-(0),v\rangle \ge 0$
















\parit{\ref{SHORT.thm:doubling:concave}.}
Let us apply induction on $m=\LinDim \spc{A}$.

\begin{wrapfigure}{r}{30mm}
\vskip-2mm
\centering
\includegraphics{mppics/pic-1305}
\end{wrapfigure}

Choose a geodesic $[pz]$; let $\gamma(0)=p$.
Suppose $p\notin\partial \spc{A}$.
Let $q\in \partial\spc{A}$ be a closest point to $p$ and $\alpha\df\mangle\hinge pzq$.

By the definition of boundary points, 
\[\partial \Sigma_q\z\ne\emptyset.\]
Let $\xi=\dir qp$.
Theorem~\ref{thm:partial-Sigma} implies that 
\[\dist{\xi}{\zeta}{\Sigma_q}\ge \tfrac\pi2
\eqlbl{eq:<>pi/2}\]
for any $\zeta\in\partial\Sigma_q$.

By \ref{thm:finite-space-of-directions}, $\Sigma_q$ is an $(m-1)$-dimensional $\Alex1$ space.
Applying the induction hypothesis, we get that the doubling $\hat\Sigma_q$ of $\Sigma_q$ across $\partial \Sigma_q$ is an $(m-1)$-dimensional $\Alex1$ space.
Denote by $\xi_1$ and $\xi_2$ the two directions in $\hat\Sigma_q$ that correspond to $\xi$.
Note that \ref{eq:<>pi/2} implies that $\dist{\xi_1}{\xi_2}{\hat\Sigma_q}\ge \pi$.
Applying the line splitting theorem (\ref{thm:splitting}), we can identify 
$\Cone\hat\Sigma_q$ with $\RR\times \partial\Sigma_q$.
It follows that 
\[\T_q=[0,\infty)\oplus \partial\T_q;\]
in particular, there is a natural projection $\proj\:\T_q\to \partial\T_q$.

Given $x\in [pz]$, choose a geodesic path $\gamma_x$ from $q$ to $x$.
Let 
\[y
\df
\gexp_q\circ\proj(\gamma^+_x(0)).\]
By \ref{thm:gexp}, $y\in \partial\spc{A}$ and 
\[\dist{x}{y}{}\le \dist{p}{q}{}+\dist{p}{x}{}\cdot \cos\alpha.\eqlbl{eq:|x-y|}\]
The latter inequality uses in addition the comparison for $[pqx]$ and it requires some work.

Note that \ref{eq:|x-y|} implies that $f\circ\gamma$ is concave for any geodesic that lies in $\spc{A}\setminus \partial \spc{A}$.
If $\gamma(t)\in \partial \spc{A}$ for some $t$, then it is easy to see that $(f\circ\gamma)'(t)=0$.
These two statements imply that $f\circ\gamma$ is concave for any geodesic that lies in $\spc{A}$.
















\section{Morse theory}

Let $f$ be a semiconcave function.
A point $p\in \Dom f$ is called \index{critical point}\emph{critical} point of $f$ if $\dd_pf\le 0$; 
otherwise it is called \index{regular point}\emph{regular}.

The proof of the following statement is quite technical, we omit its proof.

\begin{thm}{Theorem}
Let $f$ be a semiconcave function on a finite-dimensional $\Alex\kappa$ space.
Suppose $K$ is a compact set of regular points of $f$ in its level set $f=a$.
Then an open neighborhood $\Omega$ of $K$ admits homeomorphism $x\mapsto (h(x),f(x))$ to a product space $\Lambda\times (a-\eps,a+\eps)$.

\end{thm}

Note that distance function $\distfun_p$ has no critical points in a neighborhood of $p$ and the level set $\distfun_p=\eps$ is compact for small $\eps>0$.
Combining this observation with 

Applying the theorem we get the following.

\begin{thm}{Corollary}
A small spherical neighborhood of any point $p$ in a finite-dimensional $\Alex\kappa$ space $\spc{A}$ is homeomorphic to an open cone over small sphere around $p$.
\end{thm}
