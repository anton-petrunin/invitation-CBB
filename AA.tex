\section{Rectangles}

\begin{thm}{Theorem}\label{thm:area-rect}
The area of a rectangle is the product of its adjacent sides.
\end{thm}

\parit{Proof.}
Let us denote by $\mathcal{R}_{a,b}$ the solid rectangle with adjacent sides $a$ and~$b$,
and let $s(a,b)=\area \mathcal{R}_{a,b}$.
We need to show that
\[s(a,b)=a\cdot b\eqlbl{eq:a(a,b)=ab}\]
for any $a>0$ and $b>0$.

Note that $\mathcal{R}_{1,1}$ is a solid unit square.
Therefore
\[s(1,1)=1.\]

Given a positive interger $n$,
we can subdivide a solid unit square into $n^2$ of squares with side lenght $\tfrac1n$.
By Proposition~\ref{prop:subdivision} and invariance of area, we get
\[n^2\cdot s(\tfrac1n,\tfrac1n)=1
\quad\text{and}\quad
s(\tfrac1n,\tfrac1n)=\tfrac1n\cdot\tfrac1n.\]

Further, for positive intergers $k$, $m$, and $n$, the rectangle $\mathcal{R}_{\frac kn,\frac mn}$ can be sudivided into $k\cdot m$ squares with side lenght $\tfrac1n$.
Therefore
\[s(\tfrac kn,\tfrac mn)=\tfrac kn\cdot \tfrac mn,\eqlbl{eq:a(a,b)=ab-}\]
which is a partial case of \ref{eq:a(a,b)=ab}.

Suppose $a\le a'$ and $b\le b'$.
Then we can assume that $\mathcal{R}_{a,b}\subset \mathcal{R}_{a',b'}$ and therefore
\[s(a,b)\le s(a',b').\eqlbl{s(a,b)>s(a',b')}\]

Now let us argue by contradiction.
Assume that $s(a,b)\ne a\cdot b$ for some $a>0$ and $b>0$.
Then $s(a,b)> a\cdot b$ or $s(a,b) <a\cdot b$.

\raggedcolumns\setlength{\multicolsep}{.5mm}
\setlength{\columnseprule}{1pt}
\begin{multicols}{2}
If $s(a,b)> a\cdot b$,
we can choose a positive integer $n$ such that
\[s(a,b)> (a+\tfrac1n)\cdot (b+\tfrac1n).\eqlbl{s(a,b)>}\]
Choose integers $k$ and $m$ such that
\begin{align*}
a< \tfrac kn&\le a+\tfrac1n,
\\
b<\tfrac mn&\le b+\tfrac1n.
\end{align*}
By \ref{s(a,b)>s(a',b')} and \ref{eq:a(a,b)=ab-}, we get that
\begin{align*}
s(a,b)&\le s(\tfrac kn,\tfrac mn)=
\\
&=\tfrac kn\cdot\tfrac mn\le
\\
&\le (a+\tfrac1n)\cdot(b+\tfrac1n),
\end{align*}
which contradicts \ref{s(a,b)>}.

\columnbreak

If $s(a,b)< a\cdot b$, choose a positive integer $n$ such that $a>\tfrac1n$, $b>\tfrac1n$, and
\[s(a,b)< (a-\tfrac1n)\cdot (b-\tfrac1n).\eqlbl{s(a,b)<}\]
Choose integers $k$ and $m$ such that
\begin{align*}
a> \tfrac kn&\ge a-\tfrac1n,
\\
b>\tfrac mn&\ge b-\tfrac1n.
\end{align*}
By \ref{s(a,b)>s(a',b')}, we get that
\begin{align*}
s(a,b)&\ge s(\tfrac kn,\tfrac mn)=
\\
&=\tfrac kn\cdot\tfrac mn\ge
\\
&\ge (a-\tfrac1n)\cdot(b-\tfrac1n),
\end{align*}
which contradicts \ref{s(a,b)<}.\qeds
\end{multicols}
\setlength{\columnseprule}{0pt}

















\section{Rectangles}

\begin{thm}{Theorem}\label{thm:area-rect}
The area of a rectangle is the product of its adjacent sides.
\end{thm}

\begin{thm}{Algebraic lemma}\label{lem:alg-area}
Assume that a function $s$
returns a nonnegative real number $s(a,b)$
for any pair of positive real numbers $(a,b)$
and it satisfies the following identities:
\[\begin{aligned}
s(1,1)&=1,
\\
s(a,b+c)&=s(a,b)+s(a,c),
\\
s(a+b,c)&=s(a,c)+s(b,c)
\end{aligned}
\eqlbl{3s}
\]
for any $a,b,c>0$.
Then
\[s(a,b)=a\cdot b\]
for any $a,b>0$.
\end{thm}

The proof is similar to the proof of Lemma~\ref{lem:R-auto}.

\parit{Proof.}
If $a>a'$ and $b>b'$ then
\[s(a,b)\ge s(a',b').\eqlbl{s(a,b)>s(a',b')}\]
Indeed, since $s$ returns nonnegative numbers, we get that
\begin{align*}
s(a,b)&=s(a',b)+s(a-a',b)\ge s(a',b)=
\\
&= s(a',b')+s(a',b-b')\ge s(a',b').
\end{align*}

Applying the second and third identity in \ref{3s} a few times we get that
\begin{align*}
m\cdot s(a,b)&=s(a,m\cdot b)=
\\&=s(m\cdot a,b)
\end{align*}
for any positive integer $m$. Therefore
\begin{align*}
s(\tfrac kl,\tfrac mn)
&=k \cdot s(\tfrac 1l,\tfrac mn)=
\\
&=k\cdot m \cdot s(\tfrac 1l,\tfrac 1n)=
\\
&=k\cdot m\cdot \tfrac 1l\cdot s(1, \tfrac 1n)=
\\
&=k\cdot m\cdot \tfrac 1l\cdot \tfrac 1n\cdot s(1,1)=
\\
&=\tfrac kl\cdot\tfrac mn
\end{align*}
for any positive integers $k$, $l$, $m$, and $n$.
That is, the needed identity holds for any pair of rational numbers $a=\tfrac kl$ and $b=\tfrac mn$.

Arguing by contradiction, assume $s(a,b)\ne a\cdot b$ for a pair of positive real numbers $(a,b)$;
so, $s(a,b)> a\cdot b$ or $s(a,b)\z< a\cdot b$.

Recall that \index{63@$\lfloor x \rfloor$, $\lceil x \rceil$}$\lfloor x \rfloor$ and $\lceil x \rceil$  denote \index{floor and ceiling}\emph{floor} and \emph{ceiling} of $x$;
that is, $\lfloor x \rfloor$ is the greatest integer such that $\lfloor x \rfloor\le x$,
and $\lceil x \rceil$ is the least integer such that $\lceil x \rceil\ge x$.

\raggedcolumns\setlength{\multicolsep}{.5mm}
\setlength{\columnseprule}{1pt}
\begin{multicols}{2}
If $s(a,b)> a\cdot b$,
we can choose a positive integer $n$ such that
\[s(a,b)> (a+\tfrac1n)\cdot (b+\tfrac1n).\eqlbl{s(a,b)>}\]
Set $k=\lfloor a\cdot n \rfloor+1$ and $m=\lfloor b\cdot n \rfloor+1$;
equivalently, $k$ and $m$ are positive integers such that
\begin{align*}
a< \tfrac kn&\le a+\tfrac1n,
\\
b<\tfrac mn&\le b+\tfrac1n.
\end{align*}
By \ref{s(a,b)>s(a',b')}, we get that
\begin{align*}
s(a,b)&\le s(\tfrac kn,\tfrac mn)=
\\
&=\tfrac kn\cdot\tfrac mn\le
\\
&\le (a+\tfrac1n)\cdot(b+\tfrac1n),
\end{align*}
which contradicts \ref{s(a,b)>}.

\columnbreak

If $s(a,b)< a\cdot b$, choose a positive integer $n$ such that $a>\tfrac1n$, $b>\tfrac1n$, and
\[s(a,b)< (a-\tfrac1n)\cdot (b-\tfrac1n).\eqlbl{s(a,b)<}\]
Set $k=\lceil a\cdot n \rceil-1$ and $m=\lceil b\cdot n \rceil-1$; that is,
\begin{align*}
a> \tfrac kn&\ge a-\tfrac1n,
\\
b>\tfrac mn&\ge b-\tfrac1n.
\end{align*}
By \ref{s(a,b)>s(a',b')}, we get that
\begin{align*}
s(a,b)&\ge s(\tfrac kn,\tfrac mn)=
\\
&=\tfrac kn\cdot\tfrac mn\ge
\\
&\ge (a-\tfrac1n)\cdot(b-\tfrac1n),
\end{align*}
which contradicts \ref{s(a,b)<}.\qeds
\end{multicols}
\setlength{\columnseprule}{0pt}








\parit{Proof of Theorem~\ref{thm:area-rect}.}
Suppose that $\mathcal{R}_{a,b}$ denotes the solid rectangle with sides $a$ and~$b$.
Let
\[s(a,b)=\area \mathcal{R}_{a,b}.\]

By the definition of area,
$s(1,1)=\area(\mathcal{K}_1)=1$.
That is, the first identity in the algebraic lemma holds.


\begin{wrapfigure}{o}{30 mm}
\vskip-0mm
\centering
\includegraphics{mppics/pic-304}
\end{wrapfigure}

The rectangle $\mathcal{R}_{a+b,c}$
can be subdivided into two rectangles $\mathcal{R}_{a,c}$
and~$\mathcal{R}_{b,c}$.
By Proposition~\ref{prop:subdivision},
\[
\area \mathcal{R}_{a+b,c}=\area \mathcal{R}_{a,c}+\area \mathcal{R}_{b,c}.
\]
That is, the second identity in the algebraic lemma holds.
The proof of the third identity is similar.

It remains to apply the algebraic lemma.
\qeds














\section{The construction}

Let $\spc{P}$ be an $\Alex0$ surface that is homeomorphic to the 2-sphere.
Choose a triangulation $\tau_0$ of $\spc{P}$ by nondegenerate convex triangles; it exists by \ref{ex:approximation:triangulation}.
Note that we can construct a sequence of triangulations $\tau_1,\tau_2,\dots$ such that each $\tau_i$ is a subdivision of $\tau_{i-1}$ and the diameters of triangles in $\tau_i$ are smaller $\delta_i$, for some sequence $\delta_i\to 0$.

Consider the 1-skeleton of $\tau_i$ and glue in it a model solid triangle instead of each triangle in $\tau_i$.
This way we obtain a polyhedral surface $\tilde{\spc{P}}_i$ for each $i$.
By construction, $\tilde{\spc{P}}_i$ is homeomorphic to the sphere.
Moreover, there is a homeomorphism $f_i\:\spc{P}\to\tilde{\spc{P}}_i$ that each triangle of $\tau_i$ to the corresponding model triangle.


\begin{thm}{Claim}
The polyhedral surfaces $\tilde{\spc{P}}_i$ converge to $\spc{P}$ as $i\to\infty$ in the sense of Gromov--Hausdorff.
Moreover,
\[\dist{f_i(x)}{f_i(y)}{\spc{P}_i}\lg\dist{x}{y}{\spc{P}}\pm\eps_i\]
for some fixed sequence $\eps_i$ and any points $x,y\in \spc{P}$.
\end{thm}

Note that the claim implies the approximation theorem (\ref{thm:approximation}).

Given a vertex $v$ of $\tau_i$, let us denote by $\tilde v_j$ the corresponding vertex in $\tilde{\spc{P}}_j$ for $j\ge i$.
By \ref{ex:approximation:length},
\[\dist{v}{w}{\spc{P}}\ge \dist{\tilde v_j}{\tilde w_j}{\tilde{\spc{P}}_j}\]
for any two vertices $v,w$ of $\tau_i$.

Observe that the set of vertices of $\tau_i$ forms a $\delta_i$-net in $\spc{P}$.
Morever, the corresponding vertces form a $\delta_i$-net in $\tilde{\spc{P}}_i$.
By Gromov's selection theorem, we can pass to a converging subsequence of $\tilde{\spc{P}}_i$.
Denote its Gromov--Hausdorff limit by $\spc{P}'$.
From above we have that there is a short onto map $\spc{P}\to \spc{P}'$;
in particular, $\spc{P}'\le \spc{P}$.

Suppose $K$ is a topological disc in $\spc{P}$ that is a union of several triangles of $\tau_i$.
Let $\tilde K_j$ be the corresponding subset in $\tilde{\spc{P}}_j$.




Recall that $T(K)$ denotes the sum of inner turns of $K$.
According to \ref{ex:gauss-bonnet}, $T(K)\le 2\cdot\pi$.

\begin{thm}{Claim}
There is a constant $C$ such that if $T(K)>\pi$, then
\[\dist{\tilde v_j}{\tilde w_j}{\tilde K_j}\le C\cdot\delta_j\]
where $v$ and $w$ be vertices of $\tau_j$ in $K$.
\end{thm}

\parit{Proof.}
Choose a geodsic path $\gamma$ from $\tilde v_j$ to $\tilde w_j$ in $\tilde K_j$.

Let $t_0=0$, and let $t_1\in[0,1]$ be the maximal value such that $v_0=\gamma(t_0)$ and $v_1=\gamma(t_1)$ lie in the same triangle of $\tau_j$.
Further, let $t_2\in[0,1]$ be the maximal value such that $v_1=\gamma(t_1)$ and $v_2=\gamma(t_2)$ lie in the same triangle.
Continuing this process, we get a partition $t_0<t_1<\dots<t_n$ of $[0,1]$ and points $v=v_0,v_1\dots,v_n=w$.
Note that all these $n$ points lie on different edges of the triangulation $\tau_i$.
Consider the broken local geodesic $\bar\gamma$ with vertices $v_0,v_1\dots,v_n$ and edges formed by line segments from $v_{i-1}$ to $v_i$ in their common triangle.

We can assume that $\bar\gamma(t_i)=\gamma(t_i)$ for each $i$.
The arc $\bar\gamma|_{[t_{i-1},t_i]}$ might be different from $\gamma|_{[t_{i-1},t_i]}$,
in this case, $\gamma|_{[t_{i-1},t_i]}$ has to leave the common triangle of $v_{i-1}$ and $v_i$ and come back to it.

We claim that $\gamma$ and $\bar\gamma$ intersect in a sousage-like way;
that is, $\gamma$ and $\bar\gamma$ cut from $\tilde{\spc{P}}_j$ one \textit{big} component containing the complement $\tilde{\spc{P}}_j\setminus K_j$ and several diangles.
In particular, the common points of $\gamma$ and $\bar\gamma$ appear in the same orger on $\gamma$ and $\bar\gamma$.

Assume the contrary.
Let us follow $\gamma$ form $0$ to $1$ untill the time moment, say $c$, when the condition fails.
Note that the point $s=\gamma(c)$ lies a boundary of already created diangle, say between $p=\gamma(a)$ and $q=\gamma(b)$.
In this case the points $p$ and $q$ lie in one triaggle, denote it by $\triangle$, and the diangle is formed by arc $\gamma|_{[a,b]}$ and a line sement $[pq]_\triangle$.
The geodesic $\gamma$ has no self-interections.
Therefore, $\gamma$ must enter the diangle thru the side $[pq]_\triangle$;
that is, $s\in [pq]_\triangle$.
It means that the arc $\gamma|_{[a,c]}$ and the line segment $[ps]_\triangle$ bound a diangle;
moreover it has negative inner tun at $s$.
By \ref{ex:gauss-bonnet}, the total curvature of the diangle exceeds $\pi$, which contradicts the assumption.

Now, denote by $D_1,\dots, D_k$ the diangles between $\gamma$ and $\bar\gamma$.
We already proved that they do not overlap.
Each $D_k$ is bounded by arc of $\gamma$ and $\bar\gamma$;
denote their length by $\ell_k$, $\bar \ell_k$;
since $\gamma$ is a geodesic, we have that
\[\ell_k\le \bar\ell_k.\]

Denote by $\omega_k$ the total curvature of $D_k$.
By \ref{ex:approximation:diangle},
\[\cos\tfrac{\omega_k}2\cdot\bar\ell_k\le \ell_k.
\eqlbl{eq:l<l}\]
Summing up, and taking into account that $\ell_k<\delta_i$ for each $k$, we get
\[\length\bar\gamma-\length\gamma=\sum_k(\bar\ell_k-\ell_k)\le \frac{\delta_i}{\cos\tfrac{\omega}2}.\]

Now consider the points $v=v_0,v_1\dots,v_n=w$ in $\spc{P}$ that correspond to $v=v_0,v_1\dots,v_n=w$.
???
\qeds



Furthemore, the vertces of $\tau_i$ form a $\delta_i$-net in $\tilde{\spc{P}}_i$.???

It remains to show that there is a  converging-to-zero sequence $\eps_0,\eps_1,\dots$ such that
\[\dist{v}{w}{\spc{P}}\le \dist{\tilde v_j}{\tilde w_j}{\tilde{\spc{P}}_j}+\eps_i\]
for any two veritces $v,w$ of $\tau _i$ and all sufficiently large $j\ge i$.



Consider a geodesic $\tilde\gamma\:[0,\ell]\to\tilde{\spc{P}}_i$.
Perturbing the endpoints a bit, we may assume that connects a point in the interior of one triangle $T_0$ with the interior of another triangle of $\tilde\tau_i$ and it does not pass thru a vertex of $\tilde\tau_i$???.
Let $t_0=0$.
Let $t_1$ be the maximal value such that $\tilde\gamma(t_1) \in T_1$.
After that $\tilde\gamma$ has to enter another triangle, denote it by $T_2$.
Denote by $t_2$ be the maximal value such that $\tilde\gamma(t_2) \in T_2$.
Continuing this way, we get a sequence $0=t_0<t_1<\dots<t_n=\ell$.
Note that the arc $\tilde\gamma|_{[t_{i-1},t_i]}$ starts and ends in $T_i$, but since $T_i$ is not convex it might travel outside of $T_i$.
Consider the broken local geodesic $\bar\gamma$ with edges formed by line segments from $\tilde\gamma(t_{i-1})$ to $\tilde\gamma(t_i)$ in $T_i$.

\begin{thm}{Claim}
Let $K$ be a geodesic polygon in a $\Alex0$ surface, and $\omega(K)<\pi$.
Suppose that a geodesic $\tilde\gamma$ in $K$ and $\alpha$ be local geodesic with ends $\tilde\gamma(t_0)$ and $\tilde\gamma(t_1)$ that does not intersects arc $\tilde\gamma|_{[t_0,t_1]}$.
Then $\tilde\gamma$ does intersects $\alpha$ only at these two points.
\end{thm}




















\parbf{\ref{ex:arm'}};
\ref{SHORT.ex:bow'+}.
We can assume that $a_0=a'_0$ is the origin, $a_1=a'_1$ lies on the positive part of the $x$-axis,
and the points $a_n$ and $a'_n$ lie in the upper half-plane of the $(x,y)$-plane.

Suppose $\mangle a_0<\mangle a_0'$.
Find a point $x$ such that
$\dist{x}{a_n}{}<\dist{x}{a_n'}{}$, and $[x a_0\dots a_n]$ is a convex polygon in the $(x,y)$-plane.
(Here you have to use that $\mangle a_n\le\tfrac\pi2$.)

{

\begin{wrapfigure}{r}{47 mm}
\vskip-7mm
\centering
\includegraphics{mppics/pic-1118}
\vskip0mm
\end{wrapfigure}

Apply the arm lemma (\ref{lem:arm}) to $[x a_0\dots a_n]$ and $[x a_0'\dots a_n']$, and arrive at a contradiction.

\parit{\ref{SHORT.ex:bow'-}.} Look at the picture and think.

}

\parbf{\ref{ex:a<a}.}
Apply the angle monotonicity (\ref{angle-monotonicity}) couple of times.

\parbf{\ref{ex:disc-bend}.}

\parit{Source:} The second part of the exercise was added by Viktor Zalgaller to the translation of Alexandrov's book \cite{alexandrov}.
More statements of this type can be found in Zalgaller's classification of convex polyhdra with regular faces \cite{zalgaller}.

\parbf{\ref{ex:octahedron}.}

\parbf{\ref{ex:disc}.}
The exercise can be deduces from the Riemann mapping theorem.
Let us sketch the proof suggested by Fedor Petrov \cite{petrov};
it only used Jordan's theorem.

\medskip

It is sufficient to prove that an open simply connected domain $\Omega$ of the plane is homeomorphic to an open disc.
Let us represent $\Omega$ as a union $P_1\cup P_2\cup\ldots$, where all $P_i$ are (closed) polygons and $P_i\Subset P_{i+1}$ for all $i$; that is $P_i$ lies in the interior of $P_{i+1}$.
Apply the Jordan's theorem couple of times to show that there is a sequence of homeomorphisms $f_i$ form $P_i$ to the closed disc of radius $1-\tfrac1{2^i}$ such that $f_{i+1}$ extends $f_i$.
Given $x\in \Omega$, set $f(x)=f_i(x)$ for all sufficiently large $i$.
Observe that $f$ is a homeomorphism from $\Omega$ to the open unit disc.

Now, let us recursively construct our system of polygons.
Note that $\Omega$ can be presented as a union of a countable collection of coordinate squares, say $S_1,S_2,\dots$
Set $P_1=S_1$.
Suppose $P_1,\dots,P_{i-1}$ are constructed,
Add to $P_{i-1}$ several squares that cover its boundary together with $S_i$.
If necessary, add to it a finite number of squares so that the union becomes connected.
The obtained set may contain holes, but since $\Omega$ is simply connected everything in holes belongs to $\Omega$, add all this to the set.
Denote the obtained set by $P_i$.


\parbf{\ref{pr:K-P-simmetry}.}




















\parbf{\ref{ex:native}.}
Choose a geodesic $\gamma$ in $\spc{W}$.
Arguing as in the proof of \ref{thm:doubling:doubling}, we get 
that $\gamma$ can cross the common boundary of two halves $\spc{A}_0$ $\spc{A}_1$ of $\spc{W}$ at most once, or it lies in the common boundary.

In the later case $\lambda$-convexity of $f\circ\proj$ follows from $\lambda$-convexity of $f$.
In the former case the convexity has to be checked only at the point of crossing;
we may assume that it happens at $x=\gamma(0)$.
Since $\nabla_x f\in \T_x\partial\spc{A}$, we have 
\[\dd_x(f\circ\proj)(v)\le\langle\nabla_x f,v\rangle\]
Argue as in the proof of \ref{thm:doubling:doubling}, show that 
$\langle\gamma^+(0),v\rangle+\langle\gamma^-(0),v\rangle \ge 0$
















\parit{\ref{SHORT.thm:doubling:concave}.}
Let us apply induction on $m=\LinDim \spc{A}$.

\begin{wrapfigure}{r}{30mm}
\vskip-2mm
\centering
\includegraphics{mppics/pic-1305}
\end{wrapfigure}

Choose a geodesic $[pz]$; let $\gamma(0)=p$.
Suppose $p\notin\partial \spc{A}$.
Let $q\in \partial\spc{A}$ be a closest point to $p$ and $\alpha\df\mangle\hinge pzq$.

By the definition of boundary points, 
\[\partial \Sigma_q\z\ne\emptyset.\]
Let $\xi=\dir qp$.
Theorem~\ref{thm:partial-Sigma} implies that 
\[\dist{\xi}{\zeta}{\Sigma_q}\ge \tfrac\pi2
\eqlbl{eq:<>pi/2}\]
for any $\zeta\in\partial\Sigma_q$.

By \ref{thm:finite-space-of-directions}, $\Sigma_q$ is an $(m-1)$-dimensional $\Alex1$ space.
Applying the induction hypothesis, we get that the doubling $\hat\Sigma_q$ of $\Sigma_q$ across $\partial \Sigma_q$ is an $(m-1)$-dimensional $\Alex1$ space.
Denote by $\xi_1$ and $\xi_2$ the two directions in $\hat\Sigma_q$ that correspond to $\xi$.
Note that \ref{eq:<>pi/2} implies that $\dist{\xi_1}{\xi_2}{\hat\Sigma_q}\ge \pi$.
Applying the line splitting theorem (\ref{thm:splitting}), we can identify 
$\Cone\hat\Sigma_q$ with $\RR\times \partial\Sigma_q$.
It follows that 
\[\T_q=[0,\infty)\oplus \partial\T_q;\]
in particular, there is a natural projection $\proj\:\T_q\to \partial\T_q$.

Given $x\in [pz]$, choose a geodesic path $\gamma_x$ from $q$ to $x$.
Let 
\[y
\df
\gexp_q\circ\proj(\gamma^+_x(0)).\]
By \ref{thm:gexp}, $y\in \partial\spc{A}$ and 
\[\dist{x}{y}{}\le \dist{p}{q}{}+\dist{p}{x}{}\cdot \cos\alpha.\eqlbl{eq:|x-y|}\]
The latter inequality uses in addition the comparison for $[pqx]$ and it requires some work.

Note that \ref{eq:|x-y|} implies that $f\circ\gamma$ is concave for any geodesic that lies in $\spc{A}\setminus \partial \spc{A}$.
If $\gamma(t)\in \partial \spc{A}$ for some $t$, then it is easy to see that $(f\circ\gamma)'(t)=0$.
These two statements imply that $f\circ\gamma$ is concave for any geodesic that lies in $\spc{A}$.
















\section{Morse theory}

Let $f$ be a semiconcave function.
A point $p\in \Dom f$ is called \index{critical point}\emph{critical} point of $f$ if $\dd_pf\le 0$; 
otherwise it is called \index{regular point}\emph{regular}.

The proof of the following statement is quite technical, we omit its proof.

\begin{thm}{Theorem}
Let $f$ be a semiconcave function on a finite-dimensional $\Alex\kappa$ space.
Suppose $K$ is a compact set of regular points of $f$ in its level set $f=a$.
Then an open neighborhood $\Omega$ of $K$ admits homeomorphism $x\mapsto (h(x),f(x))$ to a product space $\Lambda\times (a-\eps,a+\eps)$.

\end{thm}

Note that distance function $\distfun_p$ has no critical points in a neighborhood of $p$ and the level set $\distfun_p=\eps$ is compact for small $\eps>0$.
Combining this observation with 

Applying the theorem we get the following.

\begin{thm}{Corollary}
A small spherical neighborhood of any point $p$ in a finite-dimensional $\Alex\kappa$ space $\spc{A}$ is homeomorphic to an open cone over small sphere around $p$.
\end{thm}
