\chapter{Semisolutions}

\parbf{\ref{ex:compact-length}.}
Given a pair of points $p$ and $q$, choose a sequence of paths $\gamma_n$ from $p$ to $q$ such that
\[\length\gamma_n\to \dist pq{}
\quad\text{as}\quad
n\to\infty;\]
it exists since we are in a length space.
Note that we can assume that each $\gamma_n$ is parametrized proportional to the arc length;
in particular, $\gamma_n$ are equicontinuous.
Show that paths $\gamma_n$ lie in a closed ball, say $\cBall[p,r]$ for some $r<\infty$.
Since the space is proper, $\cBall[p,r]$ is compact.
By Arzelà--Ascoli theorem, we can pass to a converging subsequence of $\gamma_n$.
Show that its limit is a geodesic path from $p$ to $q$.

\parbf{\ref{ex:compact=>complete}.}
Choose a Cauchy sequence $x_n$ in $(\spc{X},\|*\z-*\|)$; it is sufficient to show that a subsequence of $x_n$ converges.

Observe that the sequence $x_n$ is Cauchy in $(\spc{X},|*-*|)$;
denote its limit by $x_\infty$.

Passing to a subsequence, we can assume that $\|x_n-x_{n+1}\|\z<\tfrac1{2^n}$.
It follows that there is a 1-Lipschitz path $\gamma$ in $(\spc{X},\|*-*\|)$ such that $x_n=\gamma(\tfrac1{2^n})$ for each $n$ and $x_\infty=\gamma(0)$.
Therefore,
\begin{align*}
\|x_\infty-x_n\|&\le \length\gamma|_{[0,\frac1{2^n}]}\le \tfrac1{2^n}.
\end{align*}
In particular, $x_n$ converges to $x_\infty$ in $(\spc{X},\|*\z-*\|)$.

\parit{Source:} \cite[Corollary]{hu-kirk}; see also \cite[Lemma 2.3]{petrunin-stadler}.

\parbf{\ref{ex:compact+connceted}.}
Choose a sequence of positive numbers $\varepsilon_n\to 0$ and a finite $\varepsilon_n$-net $N_n$ of $K$ for each $n$.
We can assume that $\eps_0>\diam K$, and $N_0$ is a one-point set.
If $\dist{x}{y}{}<\eps_k$ for some $x\in N_{k+1}$ and $y\in N_{k}$, then connect them by a curve of length at most $\eps_k$.

Let $K'$ be the union of all these curves and $K$.
Show that $K'$ is compact and path-connected.

\parit{Source:} This problem is due to Eugene Bilokopytov \cite{bilokopytov}.

\parbf{\ref{ex:menger}.}
Choose a sequence $\eps_n>0$ that converges to zero very fast, say such that $\sum_n10^n\cdot \eps_n$ is small.
Follow the argument in the proof of Menger's lemma, taking $\eps_n$-midpoints at the $n^{\text{th}}$ stage.

\parbf{\ref{ex:k-><mono}.}

\parbf{\ref{ex:angkK}.}

\parbf{\ref{ex:undefined-angle}.}
Consider a hinge in the plane $\RR^2$ with a metric defined by norm, say by $\ell^\infty$-norm.

\parbf{\ref{ex:adjacent-angles}.}
Assume $\mangle\hinge pxz+\mangle\hinge pyz<\pi$.
By \ref{claim:angle-3angle-inq}, $\mangle\hinge pxy<\pi$.
Therefore,
$\angk p{\bar x}{\bar y}<\pi$
for some $\bar x\in \left]px\right]$ and $\bar y\in \left]py\right]$.
Hence 
\[\dist p{\bar x}{}+\dist {\bar y}p{}<\dist {\bar x}{\bar y}{}\]
--- a contradiction.

\parbf{\ref{ex:first-var}.}
Denote by $\alpha$ the arc-length parametrization of $[qp]$ from $q$ to $p$.
Choose $\eps>0$.
Observe that 
\[\dist[2]{\gamma(t)}{\alpha(\tfrac1\eps\cdot t)}{}\le t^2\cdot(1-\tfrac2\eps\cdot\cos\phi+\tfrac1{\eps^2})+o(t^2),\]
where $\phi=\mangle\hinge q p x$.
By the triangle  inequality
\[\dist{p}{\gamma(t)}{}\le \dist{\gamma(t)}{\alpha(\tfrac1\eps\cdot t)}{}+\dist{q}{p}{}-\tfrac1\eps\cdot t.\]
Conclude that
\[\dist{p}{\gamma(t)}{}
\le
\dist{q}{p}{}-t\cdot \cos \phi+\delta(\eps)\cdot t+o(t),\]
where $\delta(\eps)\to 0$ as $\eps\to0$.
The statement follows since $\eps>0$ is arbitrary.


\parbf{\ref{ex:Euclid-is-CBB}.}
The definition of $\CBB$ spaces (\ref{def:CBB}) reduces our question to the following.
\textit{Any spherical triangle has perimeter at most $2\cdot\pi$.}
Choose a spherical triangle $[xyz]$.
Let $x'$ be the antipode of $x$; that is $x'=-x$.
The spherical triangle inequality implies that
\[\dist{x}{z}{\mathbb{S}^2}\le \dist{y}{x'}{\mathbb{S}^2}+\dist{x'}{z}{\mathbb{S}^2}.\]
Observe that 
\[
\dist{x}{y}{\mathbb{S}^2}+\dist{y}{x'}{\mathbb{S}^2}=\pi,
\quad\text{and}\quad
\dist{x}{z}{\mathbb{S}^2}+\dist{z}{x'}{\mathbb{S}^2}=\pi.
\]
Hence
\[\dist{x}{y}{\mathbb{S}^2}+\dist{x}{z}{\mathbb{S}^2}+\dist{y}{z}{\mathbb{S}^2}\le2\cdot \pi.\]

\parbf{\ref{ex:(3+1)-expanding}.} For the only-if part consider the following two cases.

If $\angk p{x_1}{x_2}+\angk p{x_2}{x_3}\ge \pi$, then choose two model triangles $[qy_1y_2]\z=\modtrig(px_1x_2)$ and $[qy_2y_3]=\modtrig(px_2x_y)$ that lie on the opposite sides of $[qy_2]$.
By comparison, $\dist{y_1}{y_3}{}\ge \dist{x_1}{x_3}{}$.
Therefore the obtained configuration meets all the conditions.

If $\angk p{x_1}{x_2}+\angk p{x_2}{x_3}\ge \pi$, then choose two model triangles $[qy_1y_2]=\modtrig(px_1x_2)$
and take $y_3$ on the extension of $[y_1q]$ behind $q$ such that $\dist{q}{y_3}{}=\dist{p}{x_3}{}$.
Then $\mangle \hinge q{y_2}{y_3}\ge \angk p{x_2}{x_3}$, therefore $\dist{y_2}{y_3}{}\ge \dist{x_2}{x_3}{}$.
Further, $\dist{y_2}{y_3}{}=\dist{x_2}{p}{}+\dist{p}{x_3}{} \ge \dist{x_2}{x_3}{}$,
and again, the obtained configuration meets all the conditions.

To prove the if part, choose a configuration $q,y_1,y_2,y_3$ that meets all the conditions and maximize the sum
\[\dist{y_1}{y_2}{}+\dist{y_2}{y_3}{}+\dist{y_3}{y_1}{}.\]
Show that that $q$ lies in the solid triangle $y_1y_2y_3$;
in particular 
\[\mangle \hinge q{y_1}{y_2}+\mangle \hinge q{y_2}{y_3}+ \mangle \hinge q{y_3}{y_1}=2\cdot\pi.\]
Moreover, $\dist{q}{y_i}{}=\dist{p}{x_i}{}$ for each $i$.
Applying that increasing the opposite side in a plane triangle increases the corresponding angle, we get 
\[\angk  p{x_1}{x_2}+\angk p{x_2}{x_3}+\angk p{x_3}{x_1}
\le 
2\cdot\pi.
\]

\parbf{\ref{ex:alex-lemma-cat}.}
Consider model triangles $[\tilde p\tilde x\tilde z]=\modtrig(pxz)$ and $[\tilde p\tilde y\tilde z]=\modtrig(pyz)$
that share side $[\tilde p\tilde z]$ and lie on its opposite sides.
Note that 
\begin{align*}
\dist{\tilde x}{\tilde y}{\EE^2}
&\ge \dist{\tilde x}{\tilde y}{\EE^2}+\dist{\tilde x}{\tilde y}{\EE^2}=
\\
&=\dist{x}{z}{\spc{X}}+\dist{z}{y}{\spc{X}}=
\\
&=\dist{x}{y}{\spc{X}},
\end{align*}
where $\spc{X}$ is our metric space.
It remains to apply the monotonicity of angle in a triangle with respect to its opposite side. 


\parbf{\ref{ex:noncreasing}.}
Apply \ref{clm:angle-mono}.

\parbf{\ref{ex:0-angle}.}
Without loss of generality, we can assume that $\dist{p}{x}{}\le \dist{p}{y}{}$.
Choose $\bar x\in [px]$;
let $\bar y\in [px]$ be such that $\dist{p}{\bar x}{}=\dist{p}{\bar y}{}$.
Apply \ref{clm:angle-mono} to show that $\bar x=\bar y$.
Conclude that $[px]\subset [py]$.

\parbf{\ref{ex:pi-angle}.}
Assume that there are two distinct geodesics from $z$ to $x$.
Then we can choose distinct points $p$ and $q$ one these geodesics such that $\dist{z}{p}{}=\dist{z}{q}{}$.
Observe that $\angk zpq>0$.
By triangle inequality, we get 
\[\dist{x}{p}{}+\dist{p}{y}{}\le \dist{x}{p}{}+\dist{p}{z}{}+\dist{z}{y}{}=\dist{x}{z}{}+\dist{z}{y}{}\]
Observe that $\angk zxy=\pi$.
Therefore $\mangle\hinge zxy=\pi$ for any geodesic $[zx]$.

\parbf{\ref{ex:adjacent-CBB}.}
By \ref{ex:adjacent-angles}, we have
\[\mangle\hinge pxz+\mangle\hinge pyz\ge \pi.\]
Since $z\in \left]xy\right[$ we have 
\[\angk z{\bar x}{\bar y}=\pi\]
for any $\bar x\in \left[xz\right[$ and $\bar y\in \left]zy\right]$.
By comparison, we have that 
\[\angk z{\bar x}{\bar p}+\angk z{\bar p}{\bar y}\le\pi\]
for any $\bar p\in \left]zp\right]$.
Passing to the limit as
$\dist{z}{\bar x}{}\to 0$,
$\dist{z}{\bar y}{}\to 0$, and
$\dist{z}{\bar p}{}\to 0$,
we get the statement.

\parbf{\ref{ex:pxyvw}.} 
Without loss of generality, we can assume that $x$, $v$, $w$, and $y$ appear on 
$[xy]$ in this order.
By \ref{clm:angle-mono},
\[
\angk xyp\ge \angk xwp \ge\angk xvp.
\]
Hence, $\Rightarrow$ follows.

By Alexandrov's lemma,
\begin{align*}
\angk xyp=\angk xvp
\quad&\Longleftrightarrow\quad
\angk yxp=\angk yvp,
\\
\angk xyp=\angk xwp
\quad&\Longleftrightarrow\quad
\angk yxp=\angk ywp.
\end{align*}
Whence, $\Leftarrow$ follows.

\parbf{\ref{ex:angle-lim}.}

\parbf{\ref{ex:urysohn}.}
The Urysohn space provides an example;
see for example \cite[Lecture 2]{petrunin2023pure}.

\parbf{\ref{ex:fat}.}
The natural map will be denoted by $\tilde p\mapsto p$.
Apply \ref{clm:angle-defined} twice 
to show that 
$\dist{\tilde p}{\tilde q}{}\le \dist pq{}$
for any
$\tilde p\in[\tilde x\tilde y]$ 
and
$\tilde q\in[\tilde x\tilde z]$.
Conclude that $[xyz]$ is fat.

\parbf{\ref{ex:normCBB}}; \ref{SHORT.ex:normCBB:thin}.
Choose a triangle $[0vw]$.
Note that $m=\tfrac12(v+w)$ is the midpoint of $[vw]$.

Since $[0vw]$ is thin, we get
\[2\cdot |\tfrac12(v+w)|^2+2\cdot |\tfrac12(v-w)|^2\le |v|^2+|w|^2.\]

Note this inequality implies the opposite one;
it follows if we rewrite it via $x=\tfrac12(v+w)$ and $y=\tfrac12(v-w)$.
Hence we have 
\[2\cdot |\tfrac12(v+w)|^2+2\cdot |\tfrac12(v-w)|^2= |v|^2+|w|^2\]
for any $v,w$.
That is the norm is quadratic and the statement follows.

\parit{\ref{SHORT.ex:normCBB:fat}.}
Apply the same argument changing the signs of inequalities. 

\parbf{\ref{ex:mono-mod-angle}.}

\parbf{\ref{ex:CBB(1)notitCBB(0)}.}
Note that $\spc{X}$ has no defined sphericlal model angles;
therefore it is $\CBB(1)$.

However, $\spc{X}$ is not $\CBB(0)$ since
\[\angk  p{x_1}{x_2}_{\EE^2}=\angk  p{x_2}{x_3}_{\EE^2}=\angk  p{x_1}{x_3}_{\EE^2}=\pi.\]

\parbf{\ref{ex:RisCBB(1)}.}
Suppose $\mangle\hinge mxp\ne 0$ and $\mangle\hinge mxp\ne\pi$, or equivalently $\mangle\hinge mxq\ne0$.

We can assume that $\dist pq{}$ only slightly exceeds $\pi$,
so $\dist pm{}<\pi$ and $\dist qm{}<\pi$.
We can also assume that $\dist xm{}<\pi$.
Use the comparison to show that 
\[\dist px{}+\dist qx{} < \dist pq,\]
and arrive at a contradiction with the triangle inequality.

Extend $[pq]$ to a maximal local geodesic $\gamma$.
It might be a closed or a line segment.
Argue as above to show that any point lies on $\gamma$ and make a conclusion.

\parbf{\ref{ex:perim-k>0}.}
Arguing by contradiction, suppose 
\[\dist{p}{q}{}+\dist{q}{r}{}+\dist{r}{p}{}> 2\cdot\pi\eqlbl{eq:perimeter-of-triange<2pi}\] 
for $p,q,r\in \spc{L}$. 
Rescaling the space slightly, we can assume that $\diam\spc{L}<\pi$,
but the inequality \ref{eq:perimeter-of-triange<2pi} still holds.
By \ref{clm:K>k},
after rescaling $\spc{L}$ is still $\Alex1$.

Take $z_0\in [q r]$ on maximal distance from $p$.
Consider the following model configuration:
two geodesics $[\tilde p\tilde z_0]$, $[\tilde q\tilde r]$ in $\mathbb{S}^2$ such that 
\begin{align*}
\dist{\tilde p}{\tilde z_0}{}&=\dist{p}{z_0}{},
&  
\dist{\tilde q}{\tilde r}{}&=\dist{q}{r}{},
\\ 
\dist{\tilde z_0}{\tilde q}{}&=\dist{z_0}{q}{},
&  
\dist{\tilde z_0}{\tilde r}{}&=\dist{z_0}{q}{},
\end{align*}
and 
\[\mangle\hinge{\tilde z_0}{\tilde q}{\tilde p}
=\mangle\hinge{\tilde z_0}{\tilde r}{\tilde p}
=\tfrac\pi2.\]

Let $\tilde z\in [\tilde q\tilde r]$,
and let $z\in [q r]$ be the corresponding point.
By comparison, $\dist pz{}\le\dist {\tilde p}{\tilde z}{}$ for points $z$ near $z_0$.
Moreover, this inequality holds as far as 
\[\dist{\tilde p}{\tilde z_0}{}+\dist{\tilde z_0}{\tilde z}{}+\dist{\tilde p}{\tilde z}{}<2\cdot\pi.\]
But this inequality holds for all $\tilde z$ since  $\dist{\tilde p}{\tilde z_0}{}<\pi$, $\dist{\tilde z_0}{\tilde q}{}<\pi$, and $\dist{\tilde z_0}{\tilde r}{}<\pi$.
Hence we get $\dist pq{}\le\dist {\tilde p}{\tilde q}{}$ and $\dist pr{}\le\dist {\tilde p}{\tilde r}{}$.
The latter contradicts \ref{eq:perimeter-of-triange<2pi}.

\parbf{\ref{ex:alm-min}.}
Suppose such point does not exists;
that is, for any $p\in \spc{X}$ there is a point $p'$ such that $r(p')\le  (1-\eps)\cdot r(p)$ and $\dist p{p'}{}<\tfrac{1}{\eps}\cdot r(p)$.
Construct a sequence of points $p_0,p_1,\dots$ such that $p_n=p_{n-1}'$ for any~$n$.
Show that this sequence is Cauchy; denote its limit by $p_\infty$.
Arrive at a contradiction by showing that $r(p_\infty)\le0$.

\parbf{\ref{ex:dir-compact}}; \ref{SHORT.ex:dir-compact:compact}.
Suppose $\dir p{x_n}\not\to\dir p{x_\infty}$.
Since $\Sigma_p$ is compact we may pass to a converging subsequence of $\dir p{x_n}$;
denote by $\xi$ its limit.
We may assume that $\mangle (\dir p{x_\infty},\xi)>0$.

Denote by $\gamma_n$ and $\gamma_\infty$ the arc-length parametrization of $[px_n]$ and $[px_\infty]$ from $p$.
Choose a geodesic $\alpha$ that starts from $p$ and goes in a direction sufficiently close to $\xi$.
By comparison we can choose $\alpha$ so that
\[\dist{\alpha(t)}{\gamma_n(t)}{}<\eps\cdot t\]
for all large $n$ and all sufficiently small $t$.
Moreover, we can assume that
\[\dist{\alpha(t)}{\gamma_\infty(t)}{}>a\cdot t\]
for some fixed $a>0$ and all small $t$.
These two inequalities imply 
that 
\[\dist{\gamma_n(t)}{\gamma_\infty(t)}{}>\tfrac a2\cdot t\]
for all small $t$ and all large $n$.
On the other hand, by assumption, $\dist{\gamma_n(t)}{\gamma_\infty(t)}{}\to0$ as $n\to\infty$ --- a contradiction.

\parit{\ref{SHORT.ex:dir-compact:}} ???

\parbf{\ref{ex:geodesic-cone}.}
Note that any point of $\Cone \spc{X}$ can be connected to the origin by a geodesic.
???

\parbf{\ref{ex:distfun-semiconcave}.}
Since angles are defined, it follows that 
\[\dist{\gamma_1(t)}{\gamma_2(t)}{}\le \theta\cdot t\]
for all small $t>0$.
Since $f$ is $L$-Lipschitz, we get 
\[|f(\gamma_1(t))-f(\gamma_2(t))|\le L\theta\cdot t,\]
hence the statement.

\parbf{\ref{ex:df(xi)}}; \ref{SHORT.ex:d(distfun):<}
Note that we can assume that there is a geodesic in the direction of $v$, and apply \ref{ex:first-var}.

\parit{\ref{SHORT.ex:d(distfun):=}.}
By \ref{SHORT.ex:d(distfun):<} we have an $\le$ inequality.
Suppose this inequalities is strict for some $v$.
We can assume that $|v|=1$ and there is a geodesic, say $\gamma$ in the direction of $v$.
Suppose ...
Let $\alpha=???$.

The function $f=\distfun_q\circ\gamma$ is Lipschitz;
By Rademacher's theorem it is differentaible almost everywhere;
moreover, 
\[f(t)-f(0)=\int_0^t f'(t)\cdot dt.\]
Suppose $f'(t)$ is defined.
Use \ref{SHORT.ex:d(distfun):<} to show that 
$f'(t)=-\cos\alpha(t)$, where $\alpha(t)$ is the angle between $\gamma$ and a geodesic from $\gamma(t)$ to $q$.
Note that we can choose a sequence $t_n\to 0$ such that 
\[\lim_{n\to\infty}\alpha(t_n) \le \alpha.\]
Consider a sequence of geodsics $[p\gamma(t_n)]$.
Since the space is proper, we can pass to its convergent subsequence.
Its limit is a geodesic from $p$ to $q$, denote it by $[pq]$.
Observe that $[pq]$ makes angle at most $\alpha$ with $\gamma$ --- a contradiction.






\parbf{\ref{ex:d(distfun)}.}

\parbf{\ref{ex:monotonicity}.}

\parbf{\ref{ex:first-var-CBB}.}

\parbf{\ref{ex:convergence-grad}.}

\parbf{\ref{ex:semicontinuous-grad}.}

\parbf{\ref{ex:elf-contracting}.}
Note that $f\circ\alpha$ is a nondecreasing function.
Apply \ref{ex:d(distfun):<} and the definition of gradient to show that
\[
-\dd_{\alpha(t)}\distfun_{\alpha(t_3)}(\nabla_{\alpha(t)}f)
\ge
\langle \nabla_{\alpha(t)},\dir{\alpha(t)}{\alpha(t_3)}\rangle
\ge
\dd_{\alpha(t)}(\dir{\alpha(t)}{\alpha(t_3)})
\ge0
\]
for any $t<t_3$.
Conclude that the function 
$t\mapsto \distfun_{\alpha(t_3)}\circ\alpha(t)$ is noncreasing for $t\le t_3$.


\parbf{\ref{ex:mayer}.}

\parbf{\ref{lem:fg-dist-est}.}

\parbf{\ref{ex:busemann-CBB}.} Apply \ref{ex:distfun-semiconcave}.

\parbf{\ref{ex:bus+bus}.} By the triangle inequality, 
\[\dist{\gamma(-t)}{x}{}+\dist{\gamma(t)}{x}{}-2\cdot t\ge 0\]
for any $t\ge 0$.
Passing to the limit as $t\to\infty$, we get the result.

\parbf{\ref{ex:cone-CBB}.}

\parbf{\ref{ex:|antisum|}.}
Observe that
\begin{align*}
\langle u,u\rangle+\langle v,u\rangle+\langle w,u\rangle &\ge 0,
\\
\langle u,v\rangle+\langle v,v\rangle+\langle w,v\rangle &\ge 0,
\\
\langle u,w\rangle+\langle v,w\rangle+\langle w,w\rangle &= 0.
\end{align*}
Add the first two inequalities and subtract the last identity.

\parbf{\ref{prop:two-opp}.}

\parbf{\ref{ex:3<,>=0}.} Show and use that
\[\langle u,x\rangle +\langle v,x\rangle +\langle w,x\rangle \ge 0\]
and
\[\langle u,-x\rangle +\langle v,-x\rangle +\langle w,-x\rangle \ge 0.\]

\parbf{\ref{ex:-u}.} Part $\Rightarrow$ is evident.
To prove part $\Leftarrow$, observe that 
\[\langle u^*,u^*\rangle =-\langle u,u^*\rangle\le \langle u,u\rangle\]
and since $|u|=|u^*|$, we have equality.

\parbf{\ref{ex:grad-dist}};
\ref{SHORT.ex:grad-dist:G-delta}.
Let $S_n\subset \spc{L}$ be defined by inequality $|\nabla_xf|>1-\tfrac1n$.
Apply \ref{ex:semicontinuous-grad:>s} to show that $S_n$ is open.
Choose a point $q\ne p$, observe that $|\nabla_xf|=1$ for any point $x\in\left]pq\right[$.
Conclude that $S_n$ is dense in $\spc{L}$.
Observe and use that $S=\bigcap_nS_n$.

\parit{\ref{SHORT.ex:grad-dist:lin}+\ref{SHORT.ex:grad-dist:|grad|=1}.}
Apply \ref{ex:-u}.

\parbf{\ref{ex:tangent=Em}.}
Apply \ref{ex:grad-dist:lin} to show that for any finite set of points $p_1,\dots,p_n$ there is a G-delta dense set of points $x$ such that $\Lin_x\ni \dir x{p_i}$ for every $i$.
???



\parbf{\ref{ex:dim=1}.}

\parbf{\ref{ex:resporka}.}

\parbf{\ref{ex:finite-tan}.}

\parbf{\ref{ex:GHto-tangent}.}

\parbf{\ref{ex:geod-closed}.}

\parbf{\ref{ex:dim-lim}.}

\parbf{\ref{ex:diam-compact}.}

\parbf{\ref{ex:pack-net}.}

\parbf{\ref{ex:pack-vol}.}

\parbf{\ref{ex:bry-convex}.}

\parbf{\ref{ex:bry-product}.}

\parbf{\ref{ex:conic}.}

\parbf{\ref{ex:conic-tangent}.}
\ref{SHORT.ex:conic-tangen:tangent}
\ref{SHORT.ex:conic-tangen:dir}
\ref{SHORT.ex:conic-tangen:example}

\parbf{\ref{ex:nonstability}.}

\parbf{\ref{ex:bry2bry}.}

\parbf{\ref{ex:bry-closed}.}

\parbf{\ref{ex:bry-connected}.}

\parbf{\ref{ex:dist-to-bry}.}

\parbf{\ref{ex:liberman}.}

\parbf{\ref{ex:Hilbert/G}.}

\parbf{\ref{ex:sumbetries(S^2)}.}

\parbf{\ref{ex:sumbetries(E^2)}.}

\parbf{\ref{ex:S^3/S^1}.}

\parbf{\ref{ex:number(m-1)}.}
\ref{SHORT.ex:number(m-1):2}
\ref{SHORT.ex:number(m-1):1}

\parbf{\ref{ex:surf-S2}.}

\parbf{\ref{ex:angle-triangle}.}

\parbf{\ref{ex:vertex-essential-vertex}.}

\parbf{\ref{ex:geodesic-vertex}.}

\parbf{\ref{pr:tetrahedron}.}

\parbf{\ref{ex:poly-CBB}.}

\parbf{\ref{ex:surface-covergence}.}
We will use that the closest-point projection form the Euclidean space to a convex body is \index{short map}\emph{short};
that is, distance-nonexpanding \cite[13.3]{petrunin-zamora}.

Assume $K_\infty$ is nondegenerate.
Without loss of generality, we may assume that 
\[\cBall(0,r)\subset K_\infty\subset\cBall(0,1)\]
for some $r>0$.
Note that there is a sequence $\eps_n\to 0$ such that 
\[ K_n\subset(1+\eps_n)\cdot K_\infty
\quad\text{and}\quad
K_\infty\subset(1+\eps_n)\cdot K_n\]
for each large $n$.

Given $x\in K_n$, denote by $g_n(x)$ the closest-point projection of $(1+\eps_n)\cdot x$ to $K_\infty$.
Similarly, given $x\in K_\infty$, denote by $h_n(x)$ the closest point projection of $(1+\eps_n)\cdot x$ to $K_n$.
Note that 
\begin{align*}
\dist{g_n(x)}{g_n(y)}{}&\le (1+\eps_n)\cdot\dist{x}{y}{}
\intertext{and}
\dist{h_n(x)}{h_n(y)}{}&\le (1+\eps_n)\cdot\dist{x}{y}{}.
\end{align*}

Denote by $\Sigma_\infty$ and $\Sigma_n$ the surface of $K_\infty$ and $K_n$ respectively. 
The above inequlities imply 
\begin{align*}
\dist{g_n(x)}{g_n(y)}{\Sigma_\infty}&\le (1+\eps_n)\cdot\dist{x}{y}{\Sigma_n}
\intertext{for any $x,y\in \Sigma_n$, and}
\dist{h_n(x)}{h_n(y)}{\Sigma_n}&\le (1+\eps_n)\cdot\dist{x}{y}{\Sigma_\infty}.
\end{align*}
for any $x,y\in \Sigma_\infty$.

Since the closest-point projection cannot increase the length of curve, we also get
\begin{align*}
\dist{x}{h_n\circ g_n(x)}{\Sigma_\infty}&\le 10\cdot \eps_n
\\
\dist{y}{g_n\circ h_n(y)}{\Sigma_n}&\le 10\cdot \eps_n.
\end{align*}
for all large $n$.
Therefore, $g_n$ is a $\delta_n$-isometry $\Sigma_n\to\Sigma_\infty$ for a sequence $\delta_n\to 0$.

??? degenerate case???

\parit{Comments.}
More generally, if a sequence of geodesic $\CBB(\kappa)$ spaces $\spc{L}_1,\spc{L}_2,\dots$ converges to $\spc{L}_\infty$ and $\dim \spc{L}_n=\dim \spc{L}_infty<\infty$ for any $n$,
then $\partial \spc{L}_n$ equipped with induced length metric converge to  $\partial \spc{L}_\infty$.
This statement is a partial case of theorem about extremal subsets \cite[1.2]{petrunin1997}.

\parbf{\ref{ex:GH-doubling}.}

\parbf{\ref{ex:arm-nonconvex}.}

\parbf{\ref{ex:cauchy}.}

\parbf{\ref{ex:a<a}.}

\parbf{\ref{ex:disc-bend}.}

\parbf{\ref{ex:octahedron}.}

\parbf{\ref{ex:disc}.}

\parbf{\ref{pr:K-P-simmetry}.}

%\parbf{\ref{}.}
