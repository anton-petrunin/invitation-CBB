%%!TEX root = the-sols.tex

\chapter{Semisolutions}

\parbf{\ref{ex:net}};
\ref{SHORT.ex:net:finite}.
Suppose $X$ is compact.
Then for any $\eps>0$ any cover of $X$ by open $\eps$-balls have a finite subcover.
Note that the centers of these balls form an $\eps$-net of $X$.

Now suppose $X$ has a finite $\eps$-net.
Show that any sequence $x_n$ of points in $X$ has a subsequence such that all of its points lie in one $\eps$-ball.
Apply this statement for $\eps=\tfrac1n$ together with the diagonal procedure.

\parit{\ref{SHORT.ex:net:compact}.}
Let $Z$ be a compact $\eps$-net of $X$.
By \ref{SHORT.ex:net:finite}, $Z$ has a finite $\eps$-net, say~$F$.
Note that $F$ is a $2\cdot\eps$-net of $X$.
Since $\eps>0$ is arbitrary, we get the result.

\parbf{\ref{ex:compact+connceted}.}
Choose a sequence of positive numbers $\varepsilon_n\to 0$ and a finite $\varepsilon_n$-net $N_n$ of $K$ for each $n$.
%???eps-net are not defined!!!
We can assume that $\eps_0>\diam K$, and $N_0$ is a one-point set.
If $\dist{x}{y}{}<\eps_k$ for some $x\in N_{k+1}$ and $y\in N_{k}$, then connect them by a curve of length at most $\eps_k$.

Let $K'$ be the union of all these curves and $K$.
Show that $K'$ is compact and path-connected.

\parit{Source:} This problem is due to Eugene Bilokopytov \cite{bilokopytov}.

\parbf{\ref{ex:compact=>complete}.}
Choose a Cauchy sequence $x_n$ in $(\spc{X},\|*\z-*\|)$; it is sufficient to show that a subsequence of $x_n$ converges.

Observe that the sequence $x_n$ is Cauchy in $(\spc{X},|*-*|)$;
denote its limit by $x_\infty$.

Passing to a subsequence, we can assume that $\|x_n-x_{n+1}\|\z<\tfrac1{2^n}$ for each $n$.
It follows that there is a 1-Lipschitz path $\gamma$ in $(\spc{X},\|*-*\|)$ such that $x_n=\gamma(\tfrac1{2^n})$ for each $n$ and $x_\infty=\gamma(0)$.
Therefore,
\begin{align*}
\|x_\infty-x_n\|&\le \length\gamma|_{[0,\frac1{2^n}]}\le \tfrac1{2^n}.
\end{align*}
In particular, $x_n$ converges to $x_\infty$ in $(\spc{X},\|*\z-*\|)$.

\parit{Source:} \cite[Corollary]{hu-kirk}; see also \cite[Lemma 2.3]{petrunin-stadler}.

\parbf{\ref{ex:compact-length}.}
Given a pair of points $p$ and $q$, choose a sequence of paths $\gamma_n$ from $p$ to $q$ such that
\[\length\gamma_n\to \dist pq{}
\quad\text{as}\quad
n\to\infty;\]
these paths exist since we are in a length space.
We can assume that each $\gamma_n$ is parametrized proportionally to the arc length;
in particular, $\gamma_n$ are equicontinuous.
Show that paths $\gamma_n$ lie in a closed ball, say $\cBall[p,r]$ of some radius $r<\infty$.
Since the space is proper, $\cBall[p,r]$ is compact.
By the Arzelà--Ascoli theorem, we can pass to a converging subsequence of $\gamma_n$.
Show that its limit is a geodesic path from $p$ to $q$.

\parbf{\ref{ex:menger}.}
Choose a sequence $\eps_n>0$ that converges to zero very fast, say such that $\sum_n10^n\cdot \eps_n$ is small.
Follow the argument in the proof of Menger's lemma, taking $\eps_n$-midpoints at the $n^{\text{th}}$ stage.

\parbf{\ref{ex:k-><mono}.}
Let us write the Riemannian metric on $\MM^2(\kappa)$ in polar coordinates $(\theta,r)$;
it has the form 
$(\begin{smallmatrix}
h^2&0
\\
0&1
\end{smallmatrix})$, where $h=h(\kappa,r)\ge 0$.
Calculate $h(\kappa,r)$.
Show that for fixed $r$, the function $r\mapsto h(\kappa,r)$ is nonincreasing in the domain of definition.
Suppose $\kappa<\Kappa$, consider the partially defined map $\MM^2(\kappa)\to\MM^2(\Kappa)$ that sends a point to the point with the same polar coordinates.
Show that this map is short in its domain of definition.
Use it to prove the statement in the exercise.


\parbf{\ref{ex:angkK}.} Show and use that 
$\angk p{x}{y}_{\SSS^2}-\angk p{x}{y}_{\EE^2}=O(\dist[2]{p}{x}{}+\dist[2]{p}{y}{})$
and 
$\angk p{x}{y}_{\EE^2}-\angk p{x}{y}_{\HH^2}=O(\dist[2]{p}{x}{}+\dist[2]{p}{y}{})$.

\parbf{\ref{ex:undefined-angle}.}
Consider a hinge in the plane $\RR^2$ with a metric defined by norm, say by the $\ell^\infty$-norm.

\parbf{\ref{ex:adjacent-angles}.}
Assume $\mangle\hinge pxz+\mangle\hinge pyz<\pi$.
By \ref{claim:angle-3angle-inq}, $\mangle\hinge pxy<\pi$.
Therefore,
$\angk p{\bar x}{\bar y}<\pi$
for some $\bar x\in \left]px\right]$ and $\bar y\in \left]py\right]$.
Hence 
\[\dist p{\bar x}{}+\dist {\bar y}p{}<\dist {\bar x}{\bar y}{}\]
--- a contradiction.

\parbf{\ref{ex:first-var}.}
Denote by $\alpha$ the arc-length parametrization of $[qp]$ from $q$ to $p$.
Choose $\eps>0$.
Observe that 
\[\dist[2]{\gamma(t)}{\alpha(\tfrac1\eps\cdot t)}{}\le t^2\cdot(1-\tfrac2\eps\cdot\cos\phi+\tfrac1{\eps^2})+o(t^2).\]
By the triangle  inequality
\[\dist{p}{\gamma(t)}{}\le \dist{\gamma(t)}{\alpha(\tfrac1\eps\cdot t)}{}+\dist{q}{p}{}-\tfrac1\eps\cdot t.\]
Conclude that
\[\dist{p}{\gamma(t)}{}
\le
\dist{q}{p}{}-t\cdot \cos \phi+\delta(\eps)\cdot t+o(t),\]
where $\delta(\eps)\to 0$ as $\eps\to0$.
The statement follows since $\eps>0$ is arbitrary.

\parbf{\ref{ex:generalized-selection}.}
Since the space is proper, it is separable; 
that is, we can choose an countable everywhere dense set $\{x_1,x_2,\dots\}$.

Let $A_1,A_2,\dots$ be a sequence of closed sets.
Applying the diagonal procedure, we can pass to a subsequence such that for each $i$ the sequence $\distfun_{A_n}x_i$ converges as $n\to\infty$;
denote its limit by $f(x_i)$.

Since $\distfun_{A_n}$ is $1$-Lipschitz for any $n$, we have 
\[|f(x_i)-f(x_j)|\le \dist{x_i}{x_j}{}\]
for all $i$ and $j$.
Suppose $f(x_i)<\infty$ for some $i$, then the same holds for any $i$.
Therefore, the function $f$ can be extended to a continuous function defined on the ambient space.
Show that $A_\infty=f^{-1}\{0\}$ is the limit of $A_n$ in the sense of Hausdorff.

If $f(x_i)=\infty$ for some $i$, then the same holds for any $i$.
Show that in this case $A_n\to\emptyset$ in the sense of Hausdorff.

\parbf{\ref{ex:Haus-conv}.}
Apply the definition of Hausdorff distance (\ref{def:hausdorff-convergence}).

\parbf{\ref{ex:geod-closed}.}
Given $x_\infty,y_\infty\in\spc{X}_\infty$, choose $x_n,y_n\in \spc{X}_n$ such that $x_n\to x_\infty$ and $y_n\to y_\infty$.
Let $z_n$ be the midpoint of $[x_ny_n]$.
Since $\spc{X}_\infty$ is proper, we can choose a subsequence of $z_m$ that converges to a point, say $z_\infty\in \spc{X}_\infty$.
Note that $z_\infty$ is a midpoint of $x_\infty$ and $y_\infty$, then apply Menger's lemma (\ref{lem:mid>geod}).

\parbf{\ref{ex:non-contracting-map}.}
Given a pair of points $x_0,y_0\in \spc{K}$, 
consider two sequences $x_0,x_1,\dots$ and $y_0,y_1,\dots$
such that $x_{n+1}=f(x_n)$ and $y_{n+1}\z=f(y_n)$ for each $n$.

Since $\spc{K}$ is compact, 
we can choose an increasing sequence of integers $n_k$
such that both sequences $(x_{n_i})_{i=1}^\infty$ and $(y_{n_i})_{i=1}^\infty$
converge.
In particular, both are Cauchy;
that is,
\[
|x_{n_i}-x_{n_j}|_{\spc{K}}\to 0 
\quad\text{and}\quad
|y_{n_i}-y_{n_j}|_{\spc{K}}\to 0
\]
as $\min\{i,j\}\to\infty$.

Since $f$ is distance-noncontracting, 
\[
|x_0-x_{|n_i-n_j|}|
\le 
|x_{n_i}-x_{n_j}|
\]
for any $i$ and $j$.
Therefore, there is a sequence $m_i\to\infty$ such that
\[
x_{m_i}\to x_0\quad\text{and}\quad y_{m_i}\to y_0
\leqno({*})\]
as $i\to\infty$.

Since $f$ is distance-noncontracting, the sequence $\ell_n=|x_n-y_n|_{\spc{K}}$ is nondecreasing.
By $({*})$,  $\ell_{m_i}\to\ell_0$ as $m_i\to\infty$.
It follows that 
\[\ell_0=\ell_1=\dots\]
In particular, 
\[|x_0-y_0|_{\spc{K}}=\ell_0=\ell_1=|f(x_0)-f(y_0)|_{\spc{K}}\]
for any pair of points $(x_0,y_0)$ in $\spc{K}$.
That is, the map $f$ is distance-preserving and hence injective.
From $({*})$, we also get that $f(\spc{K})$ is everywhere dense.
Since $\spc{K}$ is compact, $f$ is surjective --- hence the result.

\parit{Remarks.}
This is a basic lemma in the introduction to Gromov--Hausdorff distance \cite[see 7.3.30 in][]{burago-burago-ivanov}.
The presented proof was suggested by Travis Morrison.

\parbf{\ref{ex:non-expanding-map}.}
Apply \ref{ex:non-contracting-map} to a left inverse of the map.


\parbf{\ref{ex:compact-GH}.} Show and use that $\dist{\spc{X}_\infty}{\spc{X}_\infty'}{\GH}<\eps$ for any $\eps>0$.

\parbf{\ref{ex:GH-po}.}
The only-if part is trivial.
Let us prove the if part.

Let $f_n\:\spc{X}_n\to \spc{X}_\infty$ and $h_n\:\spc{X}_\infty\to \spc{X}_n$ be the maps in the definition of the inequalities $\spc{X}_n\le \spc{X}_\infty+\eps_n$ and $\spc{X}_\infty\le \spc{X}_n+\eps_n$, respectively.
Apply \ref{ex:non-contracting-map}, to show that any partial limit of $f_n\circ h_n$ is an isometry of $\spc{X}_\infty$.
Conclude that $f_n$ is an $\eps_n'$-isometry for some converging-to-zero sequence $\eps_n'$ and apply \ref{lem:almost-isom}.

\parbf{\ref{ex:GH-noncompact}}; \ref{SHORT.ex:GH-noncompact:proper}
 Consider the graphs of the following functions with the induced metric from $\RR^2$.
\[
x\mapsto \cos x+\cos \tfrac x\pi
\quad\text{and}\quad
x\mapsto \cos x+\sin \tfrac x\pi.
\]


\parit{\ref{SHORT.ex:GH-noncompact:bounded}}
For every rational number  $q\in[1,2]$ consider an interval of length~$q$.
Let $\spc{X}$ be obtained by identifying all endpoints of the intervals.

Let $\spc{Y}$ be constructed in the same way but skipping the interval of length $1.5$.

\parbf{\ref{ex:CBB+-}.} Note that we can assume that we have equality in all inequalities except one.
In other words, we can asuume that equality holds in all inequalities except $\dist{y_1}{y_2}{}\le\dist{x_1}{x_2}{}$ or $\dist{q}{y_2}{}\ge\dist{p}{x_2}{}$.

In the first case,
\[
\angk  q{y_1}{y_2}\le \angk  p{x_1}{x_2},
\quad
\angk  q{y_2}{y_3}= \angk  p{x_2}{x_3},
\quad
\text{and}
\quad
\angk  q{y_3}{y_1}= \angk  p{x_3}{x_1}.
\]
Thefore the comparison for $p,x_1,x_2,x_3$ implies the comparsion for $q,y_1,y_2,y_3$.

In the second case, let us argue by contradiction.
Assume that the comparsion does not hold for $q,y_1,y_2,y_3$.
Show that
\[\angk  q{y_1}{y_2}+\angk  q{y_2}{y_3}> \pi.\]
Then show and use that
\[\angk  p{x_1}{x_2}+\angk  p{x_2}{x_3}\ge\angk  q{y_1}{y_2}+\angk  q{y_2}{y_3}.\]


\parbf{\ref{ex:Euclid-is-CBB}.}
The 4-point comparison (\ref{def:CBB}) reduces our question to the following.
\textit{Any spherical triangle has perimeter at most $2\cdot\pi$.}
Choose a spherical triangle $[xyz]$.
Let $x'$ be the antipode of $x$; that is $x'=-x$.
The spherical triangle inequality (\ref{claim:angle-3angle-inq}) implies that
\[\dist{x}{z}{\mathbb{S}^2}\le \dist{y}{x'}{\mathbb{S}^2}+\dist{x'}{z}{\mathbb{S}^2}.\]
Observe that 
\[
\dist{x}{y}{\mathbb{S}^2}+\dist{y}{x'}{\mathbb{S}^2}=\pi,
\quad\text{and}\quad
\dist{x}{z}{\mathbb{S}^2}+\dist{z}{x'}{\mathbb{S}^2}=\pi.
\]
Hence
\[\dist{x}{y}{\mathbb{S}^2}+\dist{x}{z}{\mathbb{S}^2}+\dist{y}{z}{\mathbb{S}^2}\le2\cdot \pi.\]

\parbf{\ref{ex:(3+1)-expanding}.} For the only-if part consider the following two cases.

If $\angk p{x_1}{x_2}+\angk p{x_2}{x_3}\le \pi$, then choose two model triangles $[qy_1y_2]\z=\modtrig(px_1x_2)$ and $[qy_2y_3]=\modtrig(px_2x_3)$ that lie on the opposite sides of $[qy_2]$.
By the comparison, $\dist{y_1}{y_3}{}\ge \dist{x_1}{x_3}{}$.
Therefore the obtained configuration meets all the conditions.

If $\angk p{x_1}{x_2}+\angk p{x_2}{x_3}\ge \pi$, then choose a model triangle $[qy_1y_2]\z=\modtrig(px_1x_2)$
and take $y_3$ on the extension of $[y_1q]$ behind $q$ such that $\dist{q}{y_3}{}=\dist{p}{x_3}{}$.
Then $\mangle \hinge q{y_2}{y_3}\ge \angk p{x_2}{x_3}$, therefore $\dist{y_2}{y_3}{}\ge \dist{x_2}{x_3}{}$.
Further, $\dist{y_2}{y_3}{}=\dist{x_2}{p}{}+\dist{p}{x_3}{} \ge \dist{x_2}{x_3}{}$,
and again, the obtained configuration meets all the conditions.

To prove the if part, choose a configuration $q,y_1,y_2,y_3$ that meets all the conditions and maximize the sum
\[\dist{y_1}{y_2}{}+\dist{y_2}{y_3}{}+\dist{y_3}{y_1}{}.\]
Show that $\dist{q}{y_i}{}=\dist{p}{x_i}{}$ for each $i$ and $q$ lies in the solid triangle $y_1y_2y_3$;
in particular 
\[\mangle \hinge q{y_1}{y_2}+\mangle \hinge q{y_2}{y_3}+ \mangle \hinge q{y_3}{y_1}=2\cdot\pi.\]
Applying the angle monotonicity (\ref{angle-monotonicity}), we get
\[\angk  p{x_1}{x_2}+\angk p{x_2}{x_3}+\angk p{x_3}{x_1}
\le 
2\cdot\pi.
\]

\parbf{\ref{ex:alex-lemma-cat}.}
Consider model triangles $[\tilde p\tilde x\tilde z]=\modtrig(pxz)$ and $[\tilde p\tilde y\tilde z]=\modtrig(pyz)$
that share side $[\tilde p\tilde z]$ and lie on its opposite sides.
Note that 
\begin{align*}
\dist{\tilde x}{\tilde y}{\EE^2}
&\ge \dist{\tilde x}{\tilde y}{\EE^2}+\dist{\tilde x}{\tilde y}{\EE^2}=
\\
&=\dist{x}{z}{\spc{X}}+\dist{z}{y}{\spc{X}}=
\\
&=\dist{x}{y}{\spc{X}},
\end{align*}
where $\spc{X}$ is our metric space.
It remains to apply the angle monotonicity (\ref{angle-monotonicity}).


\parbf{\ref{ex:noncreasing}.}
Apply \ref{clm:angle-mono}.

\parbf{\ref{ex:0-angle}.}
Without loss of generality, we can assume that $\dist{p}{x}{}\le \dist{p}{y}{}$.
Choose $\bar x\in [px]$;
let $\bar y\in [px]$ be such that $\dist{p}{\bar x}{}=\dist{p}{\bar y}{}$.
Apply \ref{clm:angle-mono} to show that $\bar x=\bar y$.
Conclude that $[px]\subset [py]$.

\parbf{\ref{ex:pi-angle}.}
Assume that there are two distinct geodesics from $z$ to $x$.
Then we can choose distinct points $p$ and $q$ on these geodesics such that $\dist{z}{p}{}=\dist{z}{q}{}$.
Observe that $\angk zpq>0$.
By comparison,
\[\angk zpq+\angk zpy+\angk zqy\le 2\cdot\pi.\]
Therefore, one of the angles $\angk zpy$ or $\angk zqy$ is strictly less than $\pi$.
The latter contradicts the triangle inequality.

\parbf{\ref{ex:adjacent-CBB}.}
By \ref{ex:adjacent-angles}, we have
\[\mangle\hinge pxz+\mangle\hinge pyz\ge \pi.\]
Since $z\in \left]xy\right[$ we have 
\[\angk z{\bar x}{\bar y}=\pi\]
for any $\bar x\in \left[xz\right[$ and $\bar y\in \left]zy\right]$.
By comparison, we have that 
\[\angk z{\bar x}{\bar p}+\angk z{\bar p}{\bar y}\le\pi\]
for any $\bar p\in \left]zp\right]$.
Passing to the limit as
$\dist{z}{\bar x}{}\to 0$,
$\dist{z}{\bar y}{}\to 0$, and
$\dist{z}{\bar p}{}\to 0$,
we get \[\mangle\hinge pxz+\mangle\hinge pyz\le \pi.\]

\parbf{\ref{ex:pxyvw}.} 
Without loss of generality, we can assume that $x$, $v$, $w$, and $y$ appear on 
$[xy]$ in this order.
By \ref{clm:angle-mono},
\[
\angk xyp\ge \angk xwp \ge\angk xvp.
\]
Hence, $\Rightarrow$ follows.

By Alexandrov's lemma,
\begin{align*}
\angk xyp=\angk xvp
\quad&\Longleftrightarrow\quad
\angk yxp=\angk yvp,
\\
\angk xyp=\angk xwp
\quad&\Longleftrightarrow\quad
\angk yxp=\angk ywp.
\end{align*}
Whence, $\Leftarrow$ follows.

\parbf{\ref{ex:angle-lim}.} Suppose $\mangle \hinge {x_\infty}{y_\infty}{z_\infty}>\alpha$.
Then we can choose $\bar y_\infty\in\left]x_\infty y_\infty\right]$
and $\bar z_\infty\in\left]x_\infty z_\infty\right]$ such that 
$\angk{x_\infty}{\bar y_\infty}{\bar z_\infty}>\alpha$.
Now choose $\bar y_n\in\left]x_n y_n\right]$ and $\bar y_n\in\left]x_n z_n\right]$ such that $\bar y_n\to \bar y_\infty$ and $\bar z_n\to \bar z_\infty$.
Observe that 
\[\liminf_{n\to\infty}\mangle \hinge {x_n}{y_n}{z_n}\ge\liminf_{n\to\infty}\angk{x_n}{\bar y_n}{\bar z_n} = \angk{x_\infty}{\bar y_\infty}{\bar z_\infty},\]
hence the result.

\parbf{\ref{ex:urysohn}.}
The Urysohn space provides an example;
see, for example, \cite[Lecture 2]{petrunin2023pure}.

\parbf{\ref{ex:normCBB}.}
Choose a triangle $[0vw]$.
Note that $m=\tfrac12(v+w)$ is the midpoint of $[vw]$.

Use comparison, to show that
\[2\cdot |\tfrac12(v+w)|^2+2\cdot |\tfrac12(v-w)|^2\ge |v|^2+|w|^2.\]

This inequality implies the opposite one;
it follows if we rewrite it via $x=\tfrac12(v+w)$ and $y=\tfrac12(v-w)$.
Hence we have 
\[2\cdot |\tfrac12(v+w)|^2+2\cdot |\tfrac12(v-w)|^2= |v|^2+|w|^2\]
for any $v,w$.
That is, the norm meets the parallelogram identity.
It is well known that all such norms are quadratic, and the statement follows.

\parbf{\ref{ex:concave'}.}
Choose a $\lambda$-concave function $f\:\II\to\RR$.
Note that
\[h\:t\mapsto f(t)-\lambda\cdot\tfrac{t^2}2\]
is concve,
and it is sufficient to show that $h$ is one-sided differentiable.

Choose $t_0\in \II$.
Show and use that the function
$t\mapsto \tfrac {h(t)-h(t_0)}{t-t_0}$ is nonincreasing (nondecreasing) for $t>t_0$ (respectively, for $t<t_0$).



\parbf{\ref{ex:concave-open}.}
We can assume that $\lambda=0$, otherwise pass to the function $t\mapsto f(t)-\lambda\cdot\tfrac{t^2}2$.

Suppose that $f$ is defined at $t_0$.
Since $f$ is defined on an open inteval, we can choose $a$ and $b$ such that $a<t_0<b$ and $f$ is defined on $[a,b]$.
Suppose that $t\in [a,b]\setminus\{t_0\}$.
Show and use that
\begin{align*}
t&>t_0
&&\Longrightarrow
&\frac{f(b)-f(t_0)}{b-t_0}&\le \frac{f(t)-f(t_0)}{t-t_0}\le \frac{f(a)-f(t_0)}{a-t_0},
\\
t&<t_0
&&\Longrightarrow
&\frac{f(b)-f(t_0)}{b-t_0}&\ge \frac{f(t)-f(t_0)}{t-t_0}\ge \frac{f(a)-f(t_0)}{a-t_0}.
\end{align*}



\parbf{\ref{ex:distfun-semiconcave}.}
From \ref{comp-kappa}, this inequality follows in the sense of distributions, and hence in any other sense.

\parbf{\ref{ex:alm-min}.}
Suppose such a point does not exist;
that is, for any $p\in \spc{X}$ there is a point $p'$ such that $r(p')\le  (1-\eps)\cdot r(p)$ and $\dist p{p'}{}<\tfrac{1}{\eps}\cdot r(p)$.
Construct a sequence of points $p_0,p_1,\dots$ such that $p_n=p_{n-1}'$ for any~$n$.
Show that this sequence is Cauchy; denote its limit by $p_\infty$.
Arrive at a contradiction by showing that $r(p_\infty)\le0$.

\parbf{\ref{ex:RisCBB(1)}.}
Suppose $\mangle\hinge mxp\ne 0$ and $\mangle\hinge mxp\ne\pi$;
equivalently, $\mangle\hinge mxp\ne 0$ and $\mangle\hinge mxq\ne0$.

We can assume that $\dist pq{}$ only slightly exceeds $\pi$,
so $\dist pm{}<\pi$ and $\dist qm{}<\pi$.
We can also assume that $\dist xm{}<\pi$.
Use the comparison to show that 
\[\dist px{}+\dist qx{} < \dist pq,\]
and arrive at a contradiction with the triangle inequality.

Extend $[pq]$ to a maximal local geodesic $\gamma$.
Argue as above to show that any point lies on $\gamma$.
If $\gamma$ is closed, then the space is isometric to a circle;
otherwise, it is isometric to a line segment.

\parbf{\ref{ex:perim-k>0}.}
Arguing by contradiction, suppose 
\[\dist{p}{q}{}+\dist{q}{r}{}+\dist{r}{p}{}> 2\cdot\pi\eqlbl{eq:perimeter-of-triange<2pi}\] 
for $p,q,r\in \spc{A}$. 
Rescaling the space slightly, we can assume that $\diam\spc{A}<\pi$,
but the inequality \ref{eq:perimeter-of-triange<2pi} still holds.
By \ref{clm:K>k},
after rescaling $\spc{A}$ is still $\Alex1$.

\begin{wrapfigure}{o}{25mm}
\vskip-6mm
\centering
\includegraphics{mppics/pic-380}
\vskip-0mm
\end{wrapfigure}

Take $z_0\in [q r]$ on maximal distance from $p$.
Consider the following model configuration:
two geodesics $[\tilde p\tilde z_0]$, $[\tilde q\tilde r]$ in $\mathbb{S}^2$ such that 
\begin{align*}
\dist{\tilde p}{\tilde z_0}{}&=\dist{p}{z_0}{},
&  
\dist{\tilde q}{\tilde r}{}&=\dist{q}{r}{},
\\ 
\dist{\tilde z_0}{\tilde q}{}&=\dist{z_0}{q}{},
&  
\dist{\tilde z_0}{\tilde r}{}&=\dist{z_0}{q}{},
\end{align*}
and 
\[\mangle\hinge{\tilde z_0}{\tilde q}{\tilde p}
=\mangle\hinge{\tilde z_0}{\tilde r}{\tilde p}
=\tfrac\pi2.\]

Choose $\tilde z\in [\tilde q\tilde r]$,
and let $z\in [q r]$ be the corresponding point.
By comparison, $\dist pz{}\le\dist {\tilde p}{\tilde z}{}$ if $z$ lies near $z_0$.
Moreover, this inequality holds as far as 
\[\dist{\tilde p}{\tilde z_0}{}+\dist{\tilde z_0}{\tilde z}{}+\dist{\tilde p}{\tilde z}{}<2\cdot\pi.\]
But this inequality holds for all $\tilde z$ since  $\dist{\tilde p}{\tilde z_0}{}<\pi$, $\dist{\tilde z_0}{\tilde q}{}<\pi$, and $\dist{\tilde z_0}{\tilde r}{}<\pi$.
Hence $\dist pq{}\le\dist {\tilde p}{\tilde q}{}$ and $\dist pr{}\le\dist {\tilde p}{\tilde r}{}$.
The latter contradicts \ref{eq:perimeter-of-triange<2pi}.

\parbf{\ref{ex:dir-compact}.}
Suppose $\dir p{x_n}\not\to\dir p{x_\infty}$.
Since $\Sigma_p$ is compact, we may pass to a converging subsequence of $\dir p{x_n}$;
denote its limit by $\xi$.
We may assume that $\mangle (\dir p{x_\infty},\xi)>0$.

Denote by $\gamma_n$ and $\gamma_\infty$ the arc-length parametrization of $[px_n]$ and $[px_\infty]$ from $p$.
For a  geodesic $\alpha$ that starts from $p$ and goes in a direction sufficiently close to $\xi$, we have
\[\dist{\alpha(t)}{\gamma_n(t)}{}<\eps\cdot t
\quad\text{and}\quad
\dist{\alpha(t)}{\gamma_\infty(t)}{}>a\cdot t
\]
for some $\eps \ll a$, all large $n$ and all sufficiently small $t$.
These two inequalities imply 
that 
\[\dist{\gamma_n(t)}{\gamma_\infty(t)}{}>\tfrac a2\cdot t\]
for all small $t$ and all large $n$.
On the other hand, by assumption, $\dist{\gamma_n(t)}{\gamma_\infty(t)}{}\to0$ as $n\to\infty$ --- a contradiction.

\parit{Comments.}
The compactness of $\Sigma_p$ is necessary.
A counterexample with noncompact  $\Sigma_p$ can be built using iterated warped product of line segments and applying \cite[Theorem 1.2]{alexander-bishop2004}.
The space $\spc{A}$ can be assumed to be compact.


\parbf{\ref{ex:geodesic-cone}.}
Any point of $\Cone \spc{X}$ can be connected to the origin by a geodesic.
Given a nonzero element $v\in\Cone \spc{X}$, denote by $v'$ its projection in~$\spc{X}$; so, $v=|v|\cdot v'$.

Suppose $\spc{X}$ is $\pi$-geodesic.
Choose two nonzero elements $v,w\z\in\Cone \spc{X}$; let $\alpha=\mangle(v,w)=\dist{v'}{w'}{\spc{X}}$.
If $\alpha\ge \pi$, then the product of geodesics $[v0]$ and $[0w]$ is a geodesic $[vw]$.
If $\alpha<\pi$, there is a geodesic $\gamma\:[0,\alpha]\to \spc{X}$ from $v'$ to $w'$.
Consider hinge $\hinge {\tilde o}{\tilde v}{\tilde w}$ in the plane 
such that $\mangle\hinge {\tilde o}{\tilde v}{\tilde w}=\alpha$, $\dist{\tilde o}{\tilde v}{}=|v|$, and $\dist{\tilde o}{\tilde w}{}=|w|$.
Let $t\mapsto (\phi(t),r(t))$ be geodesic $[\tilde v\tilde w]$ written in polar coordinates with origin at $\tilde o$, so that $\phi(0)=0$.
Show that $t\mapsto r(t)\cdot\gamma\circ\phi(t)$ is a geodesic from $v$ to $w$;
here we identify $\spc{X}$ with the unit sphere in $\Cone \spc{X}$.

To prove the converse, reverse the steps in the argument.

\parbf{\ref{ex:GHto-tangent}.}
Let  $\spc{A}_n=\lambda_n\cdot\spc{A}$.
Note that for any $n$ the space $\Sigma_p\spc{A}$ is identical to $\Sigma_{\iota_n(p)} \spc{A}_n$.
In particular, we can identify isometrically $\T_p\spc{A}$ with $\T_{\iota_n(p)}(\lambda\cdot \spc{A})$.
So for any geodesic $\gamma$ that starts at $p$, the vector $\gamma^+(0)$ corresponds to $\frac{1}{\lambda_n}\cdot(\iota_n\circ\gamma)^+(0)$.

Consider the logarithm maps $f_n=\log_{\iota_n(p)}\:\spc{A}_n\to T_p\spc{A}$.
We claim that this sequence of maps satisfies the assumptions in \ref{lem:almost-isom-pointed};
the condition in \ref{SHORT.lem:almost-isom-pointed-basepoint} is evident.  

It is sufficient to check the conditions in \ref{SHORT.lem:almost-isom-pointed-b} and \ref{SHORT.lem:almost-isom-pointed-c} only for $R=1$. 

Choose $\eps>0$.
By compactness of $\Sigma_p$ we can find a finite $\eps$-net $\xi_1,\dots,\xi_N$ in $\Sigma_p$. Moreover, without loss of generality we can assume that these directions are geodesic;
that is, there exist geodesics $\gamma_1,\ldots, \gamma_N$ starting at $p$ such that $\xi_i=\gamma_i^+(0)$ for each $i$.

Choose $T>0$ such that all $\gamma_i$ are defined on $[0,T]$.
Show that for any $\lambda_n>\frac{1}{T}$ the image under $f_n$ of the union $\bigcup_N\gamma_i([0,T])$ is an $\eps$-net in $\oBall(0,1)_{T_p}$.
This proves \ref{SHORT.lem:almost-isom-pointed-c}.

By comparison, we have that
\[\dist{\xi_i}{\xi_j}{\Sigma_p}\ge \angk p{\gamma_i(t_i)}{\gamma_j(t_j)}\]
for all $i\ne j$ and any $t_i,t_j\in (0,T]$.
By the definition of angle, we can assume that $T$ has been chosen so that in addition
\[\dist{\xi_i}{\xi_j}{\Sigma_p}\le \angk p{\gamma_i(t)}{\gamma_j(t)}+\eps\]
for all $i\ne j$ and any $t\in (0,T]$.

By construction of the map $f_n$ this implies that 
\[\left|\dist{x}{x'}{\spc{A}_n}-\dist{f_n(x)}{f_n(x')}{\T_p}\right|<\eps\]
for all $\lambda_n>\frac{1}{T}$ and all points $x,x'$ in $\bigcup_N\gamma_i([0,\frac{1}{\lambda_n}])\subset \oBall(p,1)_{\spc{A}_n}$.
  
Now hinge comparison and the triangle inequality imply that the same  holds for arbitrary points $x,x'$  in  $\oBall(p,1)_{\spc{A}_n}$ with $\eps$ replaced by $3\cdot\eps$.
This verifies \ref{SHORT.lem:almost-isom-pointed-b}.


\parbf{\ref{ex:df(xi)}.}
Since angles are defined, it follows that 
\[\dist{\gamma_1(t)}{\gamma_2(t)}{}\le \theta\cdot t\]
for all small $t>0$.     
Since $f$ is $L$-Lipschitz, we get 
\[|f(\gamma_1(t))-f(\gamma_2(t))|\le L\cdot \theta\cdot t,\]
and the statement follows.

\parbf{\ref{ex:d(distfun)}}; \ref{SHORT.ex:d(distfun):<}
Show that we can assume there is a geodesic in the direction of $v$, and apply \ref{ex:first-var}.

\parit{\ref{SHORT.ex:d(distfun):=}.}
By \ref{SHORT.ex:d(distfun):<}, $\dd_p\distfun_q(v)\le-\max_{\xi\in\Uparrow_p^q}\langle\xi,v\rangle$.
Suppose this inequality is strict for some $v$.
We can assume that $|v|=1$ and there is a geodesic, say $\gamma$ in the direction of $v$.
Let $\dd_p\distfun_q(v)=-\cos\alpha_0$ for some $\alpha_0\in [0,\pi]$.
Note that any geodesic from $p$ to $q$ makes angle bigger than $\alpha_0$ with $\gamma$.

The function $f=\distfun_q\circ\gamma$ is Lipschitz.
By Rademacher's theorem it is differentiable almost everywhere;
moreover, 
\[f(t)-f(0)=\int_0^t f'(t)\cdot dt.\]
Suppose $f'(t)$ is defined.
Use \ref{SHORT.ex:d(distfun):<} to show that 
$f'(t)=-\cos\alpha(t)$, where $\alpha(t)$ is the angle between $\gamma$ and any geodesic from $\gamma(t)$ to $q$.
We can choose a sequence $t_n\to 0$ such that 
\[\lim_{n\to\infty}\alpha(t_n) \le \alpha_0.\]
Consider a sequence of geodsics $[p\,\gamma(t_n)]$.
Since the space is proper, we can pass to its convergent subsequence.
Its limit is a geodesic from $p$ to $q$, denote it by $[pq]$.

Use \ref{ex:angle-lim} to show that $[pq]$ makes an angle at most $\alpha_0$ with $\gamma$ --- a contradiction.
 
\parbf{\ref{ex:monotonicity}.}
Let $\gamma\:[0,\ell]\to \spc{A}$ be the geodesic $[xy]$ parametrized from $x$ to $y$,
and let $\phi=f\circ\gamma$.
Observe that 
\[\phi'(0)=\dd_xf(\dir xy)\le \<\dir{x}{y},\nabla_{x}f\>.\]
The same way we get $-\phi'(\ell)\le \<\dir{y}{x},\nabla_{y}f\>$.
Since $f$ is $\lambda$-concave, we have
\begin{align*}
f(y)&\le f(x)+\phi'(0)\cdot \ell+\tfrac\lambda2\cdot\ell^2,
\\
f(x)&\le f(y)-\phi'(\ell)\cdot \ell+\tfrac\lambda2\cdot\ell^2.
\end{align*}
Hence the statement follows.

\parbf{\ref{ex:d(distfun):==}.}
If the space is proper, then the statement follows from \ref{SHORT.ex:d(distfun):=} and \ref{ex:pi-angle}.

To do the general case, let us argue by contradiction.
By the assumption, we can shoose be a point $z$ on the extension of $[pq]$ behind $q$.
We can assume that $|v|=1$ and it is a direction of a geodesic, say $[px]$.

Show that there is a sequence $x_n\in \left]px\right]$ such that $\dist{p}{x_n}{}\to0$ and
$\mangle \hinge q{x_n}p>\eps$ for each $n$ and some fixed $\eps>0$.
Observe that $\mangle\hinge q{x_n}z\z<\pi-\eps$; therefore
\[\dist{z}{x_n}{}<\dist{x_n}{q}{}+\dist{q}z{}-\delta\]
for each $n$ and some fixed $\delta>0$.
Pass to the limit as $x_n\to p$ and arrive at a contradiction to the triangle inequality.

\parbf{\ref{ex:convergence-grad}.}
Note that
$|(\dd_p f)(v)-(\dd_p g)(v)|\le s\cdot|v|$
for any $v\in \T_p$.
From the definition of gradient (\ref{def:grad}) we have:
\begin{align*}
&(\dd_p f)(\nabla_p g)\le\<\nabla_p f,\nabla_p g\>,
&&(\dd_p g)(\nabla_p f)\le\<\nabla_p f,\nabla_p g\>,
\\
&(\dd_p f)(\nabla_p f)=\<\nabla_p f,\nabla_p f\>,
&&(\dd_p g)(\nabla_p g)=\<\nabla_p g,\nabla_p g\>.
\end{align*}
Therefore,
\begin{align*}
&\dist[2]{\nabla_pf}{\nabla_pg}{}
=\<\nabla_p f,\nabla_p f\>+\<\nabla_p g,\nabla_p g\>-2\cdot\<\nabla_p f,\nabla_p g\>
\le
\\
&\qquad\le (\dd_p f)(\nabla_p f)+(\dd_p g)(\nabla_p g)-
(\dd_p f)(\nabla_p g)-(\dd_p g)(\nabla_p f)
\le
\\
&\qquad\le s\cdot(|\nabla_p f|+|\nabla_p g|).
\end{align*}

\parbf{\ref{ex:semicontinuous-grad}.}
Suppose $|\nabla_xf|> s$.
Then we can choose a geodesic $\gamma$ that starts at $x$ such that 
$(f\circ\gamma)^+(0)>s$.
In particular, there is $\eps>0$ such that
\[f\circ\gamma(t)>(s+\eps)\cdot t+o(t),\]
and the only-if part follows.

Now suppose $f(y)-f(x)>s\cdot \ell+\lambda\cdot \tfrac{\ell^2}2$,
were $\ell=\dist{x}{y}{}$.
Let $\gamma\:[0,\ell]\to \spc{A}$ be a geodesic from $x$ to $y$.
Since $f\circ\gamma$ is $\lambda$-concave, we have
\[f\circ\gamma(\ell)\le f\circ\gamma(0)+(f\circ\gamma)^+(0)\cdot\ell+\lambda\cdot \tfrac{\ell^2}2.\]
It follows that 
\[\dd_x(\dir xy)=(f\circ\gamma)^+(0)>s,\]
and by \ref{prop:grad-exist}, $|\nabla_x f|>s$.

\parbf{\ref{ex:elf-contracting}.}
Note that $f\circ\alpha$ is a nondecreasing function.
Apply \ref{ex:d(distfun):<} and the definition of gradient to show that
\[
-\dd_{\alpha(t)}\distfun_{\alpha(t_3)}(\nabla_{\alpha(t)}f)
\ge
\langle \nabla_{\alpha(t)},\dir{\alpha(t)}{\alpha(t_3)}\rangle
\ge
\dd_{\alpha(t)}(\dir{\alpha(t)}{\alpha(t_3)})
\ge0
\]
for any $t<t_3$.
Conclude that the function 
$t\mapsto \distfun_{\alpha(t_3)}\circ\alpha(t)$ is noncreasing for $t\le t_3$.

\parbf{\ref{ex:mayer}.}
Suppose $s>s_0$, then
\begin{align*}
(f\circ\hat\alpha)^+(s_0)&=|\nabla_{\hat\alpha(s_0)}f|
\ge
\\
&\ge
(d_{\hat\alpha(s_0)}f)(\dir{\hat\alpha(s_0)}{\hat\alpha(s)})
\ge
\\
&\ge
\frac{f\circ\hat\alpha(s)-f\circ\hat\alpha(s_0)}{\dist{\hat\alpha(s)}{\hat\alpha(s_0)}{}}.
\end{align*} 
Since $s-s_0\ge\dist{\hat\alpha(s)}{\hat\alpha(s_0)}{}$,
\[(f\circ\hat\alpha)^+(s_0)\ge
\frac{f\circ\hat\alpha(s)-f\circ\hat\alpha(s_0)}{s-s_0},\]
which implies the statement.

\parbf{\ref{lem:fg-dist-est}.}
Fix $t$, and let $p=\alpha(t)$ and $q=\beta(t)$.
Apply \ref{eq:fist-var-inq+} to get
\begin{align*}
 \ell^+
&\le -\<\dir{p}{q},\nabla_{p}f\>
-\<\dir{q}{p},\nabla_{q}g\>
\le
\\
&\le -{\left({f(q)}-{f(p)}-\lambda\cdot\tfrac{\ell^2}2\right)}/{\ell}
-{\left({g(p)}-{g(q)}-\lambda\cdot\tfrac{\ell^2}2\right)}/{\ell}\le
\\
&\le \lambda\cdot\ell+\tfrac{2\cdot\eps}{\ell}.
\end{align*}
Integrating this inequality, we get the second statement.

\parbf{\ref{ex:short-onto}.}
Choose a point $p\in M$.
Observe that $\gexp_p^1$ provides a short map from the unit hemisphere $\SSS_+^m$ to $(M,g)$.
Composing this map with the quotient map $\SSS^m\to \SSS_+^m$ brings us a short map $s\:\SSS^m\z\to(M,g)$.

Since $M$ is not homeomorphic to the sphere,
the diameter sphere theorem, $\diam (M,g)\le \tfrac\pi2$.
Hence, the map $s$ is onto.

\parbf{\ref{ex:busemann-CBB}.} Apply \ref{ex:distfun-semiconcave}.

\parbf{\ref{ex:bus+bus}.} By the triangle inequality, 
\[\dist{\gamma(-t)}{x}{}+\dist{\gamma(t)}{x}{}-2\cdot t\ge 0\]
for any $t\ge 0$.
Passing to the limit as $t\to\infty$, we get the result.

\parbf{\ref{ex:oplus}}; \ref{SHORT.ex:oplus:a}.
Show and use that $\proj_AB=\proj_BA$ is a one-point set.

\parit{\ref{SHORT.ex:oplus:b}.} Suppose $\alpha$ is a rectifiable curve with the endpoints in $A$.
Show and use that the curve $\beta=\proj_A\circ\alpha$ has the same endpoints and
\[\length\beta\le \length\alpha;\]
moreover, in case of equality, $\beta=\alpha$.

\parbf{\ref{ex:cone-CBB}.}
Suppose $\Cone\spc{X}$ is $\Alex0$.
Observe that two half-lines in $\Cone\spc{X}$ that start from the origin and go into directions $x$ and $y\in\spc{X}$ form a line if and only if $\dist{x}{y}{\spc{X}}\ge \pi$.
Apply the splitting theorem to show that for any $x\in \spc{X}$ there is at most one point $y$ such that $\dist{x}{y}{\spc{X}}\ge \pi$ and in this case we have equality.
Conclude that $\diam \spc{X}\z\le \pi$.

Now choose a quadruple of points $p,x_1,x_2,x_3\in \spc{X}$;
we will identify $\spc{X}$ with the unit sphere in $\Cone\spc{X}$.
Suppose $\dist{p}{x_i}{}<\tfrac\pi2$ for any $i$.
Consider the following points in the cone: $y_i=\tfrac1{\cos \dist{p}{x_i}{\spc{X}}}\cdot x_i$, and $q=p$.
Show that $\EE^2$-comparison for $q,y_1,y_2,y_3$ in $\Cone\spc{X}$ implies $\SSS^2$-comparsion for $p,x_1,x_2,x_3$ in $\spc{X}$.
Conclude that $\spc{X}$ is locally $\Alex1$. 
Apply the globalization theorem (\ref{thm:globalization+}).

Now assume $\spc{X}$ is $\Alex1$ and $\diam\spc{X}\le \pi$.
By \ref{ex:perim-k>0}, the perimeter of any triangle in $\spc{X}$ is at most $2\cdot\pi$.
We need to check $\EE^2$-comparison for a given quadruple of points $q,y_1,y_2,y_3$ in $\Cone\spc{X}$.
We can assume that none of these points is the origin; otherwise perturb them a bit.

Set $x_i=y_i/|y_i|$ for each $i$ and $p=q/|q|$; we can assume that $p,x_1,x_2,x_3$ are distinct in $\spc{X}$, which is the unit sphere in $\Cone\spc{X}$.

Assume the model triangles $\modtrig(px_1x_2)$, $\modtrig(px_2x_3)$, and $\modtrig(px_3x_1)$ are defined;
that is, perimeters triangles $[px_1x_2]$, $[px_2x_3]$, and $[px_3x_1]$ are strictly less than $2\cdot\pi$. 
Note that $\EE^3\iso\Cone\SSS^2$.
Use this together with the $\SSS^2$-comparison for $p,x_1,x_2,x_3$ in $\spc{X}$ to show that $\EE^2$-comparison holds for $q,y_1,y_2,y_3$ in $\Cone\spc{X}$.

Finally, if one of the model triangles is undefined, consider rescaling of $\spc{X}$ with a coefficient $\lambda$ slightly smaller than 1.
Apply the argument above to show that the comparison holds for the corresponding points in $\Cone(\lambda\cdot\spc{X})$ and pass to the limit as $\lambda\to 1$.

\parit{Comment.}
The last part of the proof is close to the argument in \ref{thm:CBB-closed}.

\parbf{\ref{ex:|antisum|}.}
Observe that
\begin{align*}
\langle u,u\rangle+\langle v,u\rangle+\langle w,u\rangle &\ge 0,
\\
\langle u,v\rangle+\langle v,v\rangle+\langle w,v\rangle &\ge 0,
\\
\langle u,w\rangle+\langle v,w\rangle+\langle w,w\rangle &= 0.
\end{align*}
Add the first two inequalities and subtract the last identity.

\parbf{\ref{prop:two-opp}.}
Apply \ref{prop:opposite} to show that 
$\langle v,v\rangle =\langle v,w\rangle=\langle w,w\rangle$,
and use it.

\parbf{\ref{ex:3<,>=0}.} Show and use that
\[\langle u,x\rangle +\langle v,x\rangle +\langle w,x\rangle \ge 0\]
and
\[\langle u,-x\rangle +\langle v,-x\rangle +\langle w,-x\rangle \ge 0.\]

\parbf{\ref{ex:-u}.} Part $\Rightarrow$ is evident.
To prove part $\Leftarrow$, observe that 
\[\langle u^*,u^*\rangle =-\langle u,u^*\rangle\le \langle u,u\rangle\]
and since $|u|=|u^*|$, we have equality.

\parbf{\ref{ex:grad-dist}.}
Apply \ref{ex:-u}.

\parbf{\ref{ex:tangent=Em}.}
By \ref{ex:diam-compact:proper}, $\spc{A}$ is \emph{separable}; that is, it contains a countable dense set of points.
Apply \ref{cor:euclid-subcone} to this set.

\parbf{\ref{ex:dim=1}.} Argue as in \ref{ex:RisCBB(1)}.

\parbf{\ref{ex:resporka}.} The only-if part is trivial.
Suppose the configuration $p$, $a_0,\z\dots, a_{m}\in \spc{A}$ meets the condition.
By \ref{ex:grad-dist} the directions $\dir q{a_0},\z\dots,\dir q{a_m}\in \Lin_q$ for G-delta dense set of points $q\in \spc{A}$.
If $q$ is sufficiently close to $p$, then $\angk q{a_i}{a_j}>\tfrac\pi2$,
and therefore, $\mangle\hinge q{a_i}{a_j}>\tfrac\pi2$ for $i\ne j$.
Conclude that $\dim\Lin_q\ge m$ in this case. To see this show and use that given $m+1$ vectors in $\RR^k$ making pairwise obtuse angles with each other any $m$ of them must be linearly independent.

\parbf{\ref{ex:finite-tan}}; 
\ref{SHORT.ex:finite-tan:tan}. Apply \ref{ex:geodesic-cone}, \ref{prop:Tan-is-CBB(0)}, and \ref{thm:finite-space-of-directions}.

\parit{\ref{SHORT.ex:finite-space-of-directions-dim}.}
Apply \ref{ex:resporka} to show that $\LinDim\T_p=\LinDim\spc{A}$ (argue as in \ref{prop:Tan-is-CBB(0)}).

\parit{\ref{SHORT.ex:finite-tan:sigma}.}
By \ref{thm:finite-space-of-directions} for any two points $\xi,\zeta\in\Sigma_p$ such that $\dist{\xi}{\zeta}{\Sigma_p}<\pi$ there is a geodesic $[\xi\zeta]_{\Sigma_p}$.
Suppose $\dist{\xi}{\zeta}{\Sigma_p}\ge\pi$, then $\T_p$ contains a line thru the origin in the directions $\xi$ and $\zeta$.
By \ref{SHORT.ex:finite-tan:tan} we can apply the splitting theorem (\ref{thm:splitting}) to $\T_p$.
We get that $\Sigma_p$ is a spherical suspension with poles $\xi$ and $\zeta$.
Hence, $\dist\xi\zeta{}=\pi$ and there is a geodesic $[\xi\zeta]$.


\parbf{\ref{ex:proof-right-inverse}}; \ref{SHORT.ex:proof-right-inverse:grad}.
By \ref{ex:distfun-semiconcave}, each function $\distfun_{a_i}$ is semiconcave in a small neighborhood of $p$.
Therefore we can choose $\lambda$ and $r>0$ so that $f_{\bm{y}}$ is $\lambda$-concave in $\oBall(p,r)$; further we will assume that $r$ is sufficiently small.
Choose $\alpha>0$ such that $\angk{x}{a_i}{a_j}>\tfrac\pi2+\alpha$ for all $i\ne j$;
we may assume that $\alpha<\tfrac{1}{10}$;
in particular,
\[(\dd_x\distfun_{a_j}{}{})(\dir{x}{a_i})
\ge
-\cos\angk{x}{a_i}{a_j}
>
\tfrac\alpha2\eqlbl{inq-a_j}\]
for $j\ne i$.

By the definition of gradient and \ref{ex:d(distfun):<}, we have
\begin{align*}
-(\dd_x\distfun_{a_i}{}{})(\nabla_x f_{\bm{y}})
&\ge
\<\dir x{a_i},\nabla_x f_{\bm{y}}\>
\ge
\\
&\ge
(\dd_xf_{\bm{y}})(\dir x{a_i}).
\end{align*}
If $\dist{a_i}{x}{}>y_i$, then 
\[\dd_xf_{\bm{y}}=\sigma+\eps\cdot \dd_x\distfun_{a_0},\]
where $\sigma$ is a minimum of a subset of the following functions:
$0$, and $\dd_x\distfun_{a_j}$ for $0\ne j\ne i$.
By \ref{inq-a_j}, 
\[(\dd_x\distfun_{a_i}{}{})(\nabla_x f_{\bm{y}})< -\tfrac\alpha2\cdot\eps.\]
Hence (\ref{111}) holds for all sufficiently small $\eps>0$.

Now assume that $\dist{a_i}{x}{}-y_i=\min_j\{\dist{a_j}{x}{}\z-y_j\}<0$.
Then
\begin{align*}
\dd_x f_{\bm{y}}
&=
\min_{i\in S} \{\,\dd_x\distfun_{a_j}\,\}+\eps\cdot \dd_x\distfun_{a_0}
\le
\\
&\le
\dd_x \distfun_{a_i}{}{}+\eps\cdot(\dd_p\distfun_{a_0}{}{}),
\end{align*}
where $j\in S$ if and only if $\dist{a_i}{x}{}-y_i=\dist{a_j}{x}{}-y_j$.
Applying \ref{inq-a_j}, we get
\begin{align*}
(\dd_x \distfun_{a_i}{}{})(\nabla_x f_{\bm{y}})
&\ge 
\dd_xf_{\bm{y}}(\nabla_x f_{\bm{y}}) -\eps\cdot(\dd_x \distfun_{a_0}{}{})(\nabla_x f_{\bm{y}}) 
\ge 
\\
&\ge
\left[(\dd_xf_{\bm{y}})(\dir x{a_0})\right]^2-2\cdot \eps
\ge
\\
&\ge
\left[\tfrac\alpha2-\eps\right]^2-2\cdot \eps.
\end{align*}
Thus, (\ref{222}) holds for all sufficiently small $\eps>0$. 

\parit{\ref{SHORT.ex:proof-right-inverse:alpha}}
Consider the following real-to-real functions:
\[\begin{aligned}
\phi(t)
&\df
\max_{i}\{\dist{a_i}{\alpha_{\bm{y}}(t)}{}-y_i\},
\\
\psi(t)
&\df
\min_{i}\{\dist{a_i}{\alpha_{\bm{y}}(t)}{}-y_i\}.
\end{aligned}\eqlbl{eq:xy-def}\]
Use \ref{SHORT.ex:proof-right-inverse:grad}, to show that for $t\in[0,t_0]$, we have $\phi^+(t)<-\tfrac{1}{10}\cdot\eps^2$ if $\phi(t)>0$
and $\psi^+(t)>\tfrac{1}{10}\cdot\eps^2$ if $\psi(t)<0$.
Conclude that $\phi(t_0)=\psi(t_0)=0$; hence the result.

\parit{\ref{SHORT.ex:proof-right-inverse:end}}
A straightforward application of \ref{lem:fg-dist-est} and a reformulation of \ref{SHORT.ex:proof-right-inverse:alpha}.

\parit{Remarks.}
By \ref{lem:fg-dist-est}, that the constructed map $\map$ is bi-Hölder with the exponent $\tfrac12$.
In particular, if an infinite-dimensional Alexandrov space $\spc{A}$ contains a bi-Hölder copy of Euclidean ball of arbitrary dimension.
It seems plausible that $\spc{A}$ should contain a bi-Lipschitz copy of Euclidean ball of arbitrary dimension,
but this question is open.


\parbf{\ref{ex:proof-dist-chart}.}
Apply the (\textit{n}+1)-comparison (\ref{thm:n+1}) to show that at least one of the inequalities
\[
\mangle\hinge xy{a_0}<\tfrac\pi2-\eps,\ \dots,\  \mangle\hinge xy{a_m}<\tfrac\pi2-\eps,
\]
holds.
Similarty, we get that at least one of the inequalities
\[
\mangle\hinge yx{a_0}<\tfrac\pi2-\eps,\ \dots,\  \mangle\hinge yx{a_m}<\tfrac\pi2-\eps,
\]
holds.

Suppose our statement does not hold for $x$ and $y$ in a sufficiently small neighborhood of $p$.
It follows that 
\[\mangle\hinge yx{a_0}<\tfrac\pi2-\eps
\quad\text{and}\quad
\mangle\hinge yx{a_0}<\tfrac\pi2-\eps.
\eqlbl{eq:a0}
\]
Note that $\dist{x}{y}{}$ is small compared to $\dist{a_0}{x}{}$ and $\dist{a_0}{y}{}$.
Therefore, the comparison contradicts \ref{eq:a0}. 

By the construction, $f$ is Lipschitz.
From above, we can choose $i>0$ so that $\mangle\hinge xy{a_i}<\tfrac\pi2-\eps$ (if $\mangle\hinge yx{a_i}<\tfrac\pi2-\eps$, then swap $x$ and $y$).
By comparison, there is $c>0$ such that $\dist{a_i}{y}{}\le \dist{a_i}{x}{}+c\cdot \dist{x}{y}{}$.
Hence $f$ is bi-Lipschitz, and now \ref{thm:right-inverse} implies \ref{thm:dist-chart}.

\parbf{\ref{ex:BG}.} 
Follow the proof Bishop--Gromov inequality, plus prove the following two inequalities
\begin{align*}
\sinh r_2\cdot \dist{\log_p x}{\log_p y}{\T_p} &\ge\dist{x}{y}{\spc{A}}
\\
\sinh r_2\cdot\dist{w(x)}{w(y)}{\spc{A}} &\ge \sinh r_1\cdot\dist{x}{y}{\spc{A}}
\end{align*}
for any $x,y\in\oBall(p,r)$.

\parbf{\ref{ex:diam-compact:proper}.}
Reuse the argument from  the first part of the proof of Bishop--Gromov inequality.

\parbf{\ref{ex:dim=dim}.} 
Suppose $K$ is a compact set in $\spc{A}$ such that $\HausDim K\ge m$.
Use the map $w$ from the proof of the Bishop--Gromov inequality (\ref{inq:BG} and \ref{ex:BG}) to show that any open ball in $\spc{A}$ contains a compact set $K'$ such that $\HausDim K'\ge m$.

Use this in addition to the arguments in \ref{thm:dim=dim}. 

\parbf{\ref{ex:dim-lim}.}
Apply \ref{ex:resporka}.




\parbf{\ref{ex:pack-net}.} If $x_1,\dots,x_n$ is not an $\eps$-net, then there is a point $y$ such that $\dist{x_i}{y}{}\ge\eps$ for any $i$.
Therefore $x_1,\dots,x_n$ is not a maximal packing --- a contradiction.

\parbf{\ref{ex:pack-vol}}; \ref{SHORT.ex:pack-vol:pack}
Apply the Bishop--Gromov inequality (\ref{inq:BG}).

\parit{\ref{SHORT.ex:pack-vol:dim}}
By \ref{ex:dim-lim}, $\dim\spc{A}_\infty\le m$.
To show that $\dim\spc{A}_\infty\ge m$,
apply \ref{cor:euclid-subcone} to a maximal packing and use the estimate in \ref{SHORT.ex:pack-vol:pack}.

\parit{Comment.}
The following stronger statement holds
\[\vol_m\spc{A}_\infty=\lim_{n\to\infty} \vol_m\spc{A}_n.\]
In other words, if $\bm{K}_m\subset \GH$ denotes the set of isometry classes of all compact $\Alex\kappa$ spaces with dimension $\le m$, then the function
$\vol_m\:\bm{K}_m\to \RR$ is continuous.


\parbf{\ref{ex:diam-compact:GH}.}
Argue as in \ref{thm:gromov-compactness} to construct a Gromov--Hausdorff convergence of $\cBall(p_n,R)_{\spc{A}_n}$ for given $R>0$, then apply the diagonal procedure to construct the needed convergence.

\parbf{\ref{ex:no-conc}.}
Consider the infinite product $\SSS^1\times ({\tfrac 12}\cdot \SSS^1)\times ({\tfrac 14}\cdot \SSS^1)\times\dots$

\parbf{\ref{ex:no-cont-lifting}.}
Consider the canonical metric $g$ on the round unit sphere $\SSS^3$ and the Hopf bundle $\SSS^3\to \SSS^2$.
Let $g_n$ be obtained from $g$ by rescaling the Hopf fibers by $\frac{1}{n}$.
Then all  $(\SSS^3,g_n)$ have nonnegative curvature and this sequence converges to the round metric on $\SSS^2$ of radius $\frac{1}{2}$.
Verify that this convergence provides the desired counterexample.

\parit{Comments.}


\parbf{\ref{ex:conic}.}
Let $V$ and $W$ be two conic neighborhoods of a point~$p$.
Without loss of generality, we may assume that $V\Subset W$;
that is, the closure of $V$ lies in $W$.

Construct a sequence of embeddings $f_n\:V\to W$
such that 
\begin{itemize}
\item 
For any compact set $K\subset V$ 
there is a positive integer $n=n_K$ such that 
$f_n(k)=f_m(k)$ for any $k\in K$ and $m, n \ge n_K$.
\item For any point $w\in W$ there is a point $v\in V$ such that $f_n(v)=w$ for all large $n$.
\end{itemize}

Once such a sequence is constructed, $f\:V\to W$ can be defined by $f(v)=f_n(v)$ for all large values of $n$ gives the needed homeomorphism.

The sequence $f_n$ can be constructed recursively
\[f_{n+1}=\Psi_n\circ f_n\circ \Phi_n,\]
where $\Phi_n\:V\to V$ 
and $\Psi_n\:W\to W$ 
are homeomorphisms
of the form 
\[\Phi_n(x)=\phi_n(x)\ast x\quad \text{and}\quad \Phi_n(x)=\psi_n(x)\star x,\]
where $\phi_n\:V\to \RR_{\ge 0}$, $\psi_n\:W\to \RR_{\ge 0}$ are suitable continuous functions;
``$\ast$'' and ``$\star$'' denote the multiplications in the cone structures of $V$ and $W$ respectively.

\parit{Comment.} One may also read the original proof by Kyung Whan Kwun \cite{kwun1964}.

\parbf{\ref{ex:conic-tangent}}; \ref{SHORT.ex:conic-tangen:tangent}. Apply \ref{thm:spherical-nbhd} and \ref{lem:kwun}.

\parit{\ref{SHORT.ex:conic-tangen:dir}.} Apply \ref{SHORT.ex:conic-tangen:tangent}.

\parit{\ref{SHORT.ex:conic-tangen:example}.} Recall that the Poincaré homology sphere can be obtained as a quotient space $\Sigma=\SSS^3/\Gamma$ by an isometric action of a finite group $\Gamma$  --- the so-called binary icosahedral group.
By the double suspension theorem,  $\Susp^2\Sigma\cong\SSS^5$.
Note that $\Susp^2\Sigma$ is an Alexandrov space and it has a point with space of directions isometric to $\Susp\Sigma$.
Observe that $\Susp\Sigma$ is not a manifold; in particular $\Susp\Sigma\ncong\SSS^4$.
Therefore the pair $\Susp^2\Sigma$ and $\SSS^5$ provides the needed example.

\parbf{\ref{ex:bry2bry}.} Apply \ref{thm:spherical-nbhd}, \ref{lem:kwun}, and \ref{thm:top-bry}.

\parbf{\ref{ex:bry-closed}.}
Let $\spc{A}$ be a finite-dimensional Alexandrov space.
Choose $x\in\spc{A}$.
By \ref{thm:spherical-nbhd}, a neighborhood $U\ni x$ is homeomorphic to $\T_x$.
Therefore \ref{ex:bry2bry}, implies that $U\cap\partial\spc{A}=\emptyset\  \Leftrightarrow\ x\notin \partial\spc{A}$;
that is, the complement $\spc{A}\setminus\partial\spc{A}$ is open, and therefore, $\spc{A}$ is closed.

\parbf{\ref{ex:pz<ypz}.}
Consider the model triangle $[\tilde x\tilde y\tilde z']=\modtrig(xyz)$.
\begin{figure}[ht!]
\vskip-0mm
\centering
\includegraphics{mppics/pic-1015}
\end{figure}

Show that 
\[\dist{\tilde p}{\tilde z}{}\le \dist{\tilde p}{\tilde z'}{}\le\side\hinge yp{z}.\]


\parbf{\ref{ex:bry-connected}.}
Assume that $\partial\spc{A}$ has at least two connected components, say $A$ and $B$.
Let $\gamma$ be a geodesic that minimizes the distance from $A$ to $B$.

Consider two-sided infinite sequence of copies of $\spc{A}$
\[\dots,\spc{A}_{-1},\spc{A}_{0},\spc{A}_{1},\dots\]
Let us glue $\spc{A}_{i}$ to $\spc{A}_{i+1}$ along $A$ if $i$ is even and along $B$ if $i$ is odd.

By the doubling theorem, every point in the obtained space $\spc{N}$ has a neighborhood that is isometric to a neighborhood of the corresponding point in $\spc{A}$ or its doubling.
By the globalization theorem, $\spc{N}$ is $\Alex1$.

The copies of $\gamma$ in $\spc{A}_{i}$ form a line in $\spc{N}$.
By the splitting theorem, $\spc{N}$ is isometric to a product $\spc{N}'\oplus \RR$.
Since $\dim\spc{N}>1$, Exercise~\ref{ex:dim=1} implies that $\diam\spc{N}\le \pi$ --- a contradiction.

\parbf{\ref{ex:dist-to-bry}}; \ref{SHORT.ex:dist-to-bry:geod}
Apply the gradient flow as in the proof of \ref{thm:doubling} (part \ref{SHORT.thm:partial-grad:flow}$_{m}\z+$\ref{SHORT.thm:doubling:doubling}$_{m-1}\z\Rightarrow$\ref{SHORT.thm:doubling:doubling}$_m$).

\parit{\ref{SHORT.ex:dist-to-bry:dist}.}
We can assume that $\dim \spc{A}\ge 2$; otherwise the statement is tivial.

Choose an interior point $x$ on $\gamma$;
we can assume that $x=\gamma(0)$.
Let $y\in \partial \spc{A}$ be a closest point to $x$,
and let $\alpha=\mangle(\dir xy,\gamma^+(0)$.

By \ref{SHORT.ex:dist-to-bry:geod}, we can assume that $x\notin \partial \spc{A}$.
Show that $\T_y=[0,\infty)\times\T_y\partial \spc{A}$
and $\dir yx\perp \T_y\partial \spc{A}$.

Given a vector $v\in \T_y$, denote by $\bar v$ its projection to $\T_y\partial \spc{A}$.
Apply the comparison and \ref{prop:gexp} to show that 
\[\dist{\gamma(t)}{\gexp_y(\overline{\log_x\gamma(t)})}{}\le \dist{x}{y}{}+t\cdot\cos\alpha.\]
Conclude that $(\distfun_{\partial \spc{A}}\circ\gamma)''(0)\le 0$ in the barrier sense.


\parbf{\ref{ex:liberman}.}
Suppose $\gamma$ is defined on the interval $[0,\ell]$.
Assume that the function $\rho\:t\mapsto \tfrac12\cdot\distfun_p^2\circ\gamma(t)$ is not $1$-concave.
Let $\bar\rho\:[0,\ell]\to\RR$ be the minimal $1$-concave function such that $\bar\rho\ge \rho$.
Note that $\bar\rho=\rho$ at the ends of $[0,\ell]$.

Consider the curve $\bar\gamma(t)\df \GF_f^{s(t)}\gamma(t)$;
where $f=\tfrac12\cdot\distfun_p^2$ and $s(t)\z=\ln\circ\bar\rho(t)-\ln\circ\rho(t)$.
Use the first distance estimate to show that $\length\bar\gamma<\length\gamma$ and arrive at a contradiction.

\parit{Comment.}
This is the so-called \index{Liberman's lemma}\emph{Liberman's lemma};
it was proved by Grigory Perelman and the second author \cite{perelman-petrunin} and generalizes a theorem of Joseph Liberman \cite{liberman} about geodesics on convex surfaces.

\parbf{\ref{ex:native}.}
Choose a geodesic $\gamma$ in $\spc{W}$.
Arguing as in the proof of \ref{thm:doubling:doubling}, we get 
that $\gamma$ can cross the common boundary of two halves $\spc{A}_0$ and $\spc{A}_1$ of $\spc{W}$ at most once, or it lies in the common boundary.

In the later case $\lambda$-concavity of $f\circ\proj\circ\gamma$ follows from $\lambda$-concavity of $f$.
In the former case the convexity has to be checked only at the point of crossing;
we may assume that it happens at $x=\gamma(0)$.
Since $\nabla_xf\in\partial\T_x$ for any $x\in\partial\spc{A}$ the $f$-gradient flows on $\spc{A}_0$ and $\spc{A}_1$ agree on the common boundary; so they induce a continuous flow on $\spc{W}$.

Assume $f\circ\proj\circ\gamma$ is not $\lambda$-concavity at $0$.
Apply the constructed flow on $\spc{W}$ to shorten $\gamma$ keeping its ends as in the proof of \ref{ex:liberman},
and arrive at a contradiction.

\parbf{\ref{ex:Hilbert/G}.} Read \cite[Section 4]{terng-thorbergsson} and/or the solution for ``Quotient of the Hilbert space'' in \cite{petrunin2020}.

\parbf{\ref{ex:sumbetries(S^2)}}; \ref{SHORT.ex:sumbetries(S^2):1}.
Choose an isometric $\SSS^1$-action on $\SSS^2$ that fixes the poles of the sphere.
Consider the projection to the quotient space $\sigma_1\:\SSS^2\z\to \SSS^2/\SSS^1=[0,\pi]$.

\parit{\ref{SHORT.ex:sumbetries(S^2):2}.}
Take a half-circle $\gamma$ on $\SSS^2$ and define 
$\sigma_2(x)\df\distfun_\gamma(x)_{\SSS^2}$.

\parit{\ref{SHORT.ex:sumbetries(S^2):n}.}
Consider the subdivision of $\SSS^2$ into $\SSS^1$-orbits of the action from~\ref{SHORT.ex:sumbetries(S^2):1}.
Cut $\SSS^2$ into two hemispheres by meridians rotate one hemisphere by an angle $\alpha=\pi/n$ and glue it back.
Observe that there is a submetry $\sigma_n$ such that the inverse image $\sigma_n^{-1}\{y\}$ is a union of the arcs from the original $\SSS^1$-orbits.

Note that for $n=2$ we get the solution in \ref{SHORT.ex:sumbetries(S^2):2}.

\parbf{\ref{ex:sumbetries(E^2)}.}
Show that for any $x\in\EE^2$ there is a half-line $H\ni x$ such that 
the restriction $\sigma|_H$ is an isometry.
Suppose such a half-line $H$ starts at $p$ and passes thru $q$.
Show that $\langle x-p,q-p \rangle\le 0$ for any $x\in \sigma^{-1}\{0\}$.
Conclude that $\sigma^{-1}\{0\}$ is a convex closed set.
Use the definition of submetry to show that  $\sigma^{-1}\{0\}$ has no interior points.
Make a conclusion.

\parbf{\ref{ex:S^3/S^1}};
\ref{SHORT.ex:S^3/S^1:pq}.
Our $\SSS^1$ is a commutative subgroup of $\SO(3)$.
Therefore it is a subgroup of a maximal torus in $\SO(3)$.
Show that the described torus action is induced by a maximal torus in $\SO(3)$.
Use that maximal tori in $\SO(3)$ are conjugate.

\parit{\ref{SHORT.ex:S^3/S^1:sphere}.}
Cut $\SSS^3$ into by the Clifford torus $\tfrac1{\sqrt2}\cdot (\SSS^1\times \SSS^1)$.
Observe that the quotient of each half is a disc;
conclude that $\Sigma_{p,q}$ is a sphere.
The torus action on $\SSS^3$ induce the needed $\SSS^1$-action on $\Sigma_{p,q}$.

\parit{\ref{SHORT.ex:S^3/S^1:a}+\ref{SHORT.ex:S^3/S^1:b}+\ref{SHORT.ex:S^3/S^1:c}.} Calculations.

\parit{\ref{SHORT.ex:S^3/S^1:cc}.}
Consider the map $\Sigma_{p,q}\to\Sigma_{1,1}$ that sends poles to poles,
preserve the distance to the poles and respects the $\SSS^1$-actions.

\parbf{\ref{ex:number(m-1)}};
\ref{SHORT.ex:number(m-1):2}.
Suppose $\mathfrak{M}_{m-1}(\Gamma)\ge 3$;
that is, $\spc{A}=\EE^m/\Gamma$ has at least 3 boundary components.
Follow Case~3 in the proof \ref{thm:hsiang-kleiner} to glue a train-space from copies of $\spc{A}$ using two of these components.
Show that the obtained space splits and arrive at a contradiction.

(Alternatively, apply a similar construction to all components of the boundary.
Show that the obtained space has {}\emph{exponential volume growth};
that is, there is $a>1$ such that $\vol \oBall(p,r)>a^r$ for all large~$r$.
Arrive at a contradiction with the Bishop--Gromov inequality.)

\parit{\ref{SHORT.ex:number(m-1):1}.}
Apply the doubling theorem as in Case~2 in the proof \ref{thm:hsiang-kleiner}.

\parbf{\ref{ex:S1actsS3}.}
Show that the quotient space $\Delta=\spc{A}/\mathbb{S}^1$ is an $\Alex1$ disc and $\gamma$ projects isometrically to itsboundary $\partial\Delta$.
It remains to show that the perimeter of $\Delta$ cannot exceed $2\cdot\pi$.
The latter follows from the Lytchak's problem \cite[3.3.5]{petrunin:survey};
it states that if $\Delta$ as an $m$-dimensional $\Alex1$ space, then $\vol_{m-1}\partial \Delta\z\le \vol_{m-1}\partial \mathbb{S}^{m-1}$.

\parbf{\ref{ex:geodesic-vertex}.}
Suppose a geodesic $\gamma$ passes thru a vertex $v$.
Denote by $\alpha$ and $\beta$ the angles that $\gamma$ cuts at $v$.
Since $v$ is essential, $\alpha+\beta<2\cdot\pi$.
Therefore $\alpha<\pi$ or $\beta<\pi$.
Arrive at a contradiction by showing that $\gamma$ is not length-minimizing.

\parbf{\ref{ex:gauss-bonnet}.}
Assume $\spc{P}$ has no boundary.
Denote by $k$, $l$, and $m$ the number of vertices, edges, and triangles, respectively in a chosen triangulation of $\spc{P}$.
Note that
\[2\cdot l=3\cdot m
\qquad\text{and}\qquad
k-l+m=\chi(\spc{P}).\]
The first identity follows since each edge appears in two triangles and each triangle has 3 edges;
the second identity is the Euler's formula.

Since each triangle contributes $\pi$ to the total sum of angles, we get that total curvature is $2\cdot\pi\cdot k-\pi\cdot m$.
It remains to apply straightforward algebraic manipulations.

If $\spc{P}$ has nonempty boundary, then pass to its doubling, apply the formula and rewrite the result using inner turns.

\parbf{\ref{ex:poly-CBB}.}
We need to show that if a polyhedral surface is $\Alex0$, then the total angle $\theta$ at every vertex $p$ it at most $2\cdot\pi$.

Assume that $\theta>2\cdot\pi$,
let $\phi=\max\{\,\pi,\tfrac13\cdot\theta\,\}$.
We can choose three points $x_1$, $x_2$, and $x_3$ close to $p$ such that
$\mangle \hinge p{x_i}{x_j}=\phi$ for $i\ne j$.
Since the points $x_i$ are close to $p$, we have $\mangle \hinge p{x_i}{x_j}=\angk p{x_i}{x_j}$.
The latter contradicts $\EE^2$-comparison.

\parbf{\ref{ex:construction}.}
\ref{SHORT.ex:approximation:nbhd}.
Apply \ref{cor:convex-nbhd}.

\parit{\ref{SHORT.ex:approximation:triangulation}.}
Observe that any chord divides a convex figure on $\spc{P}$ into two convex figure.
Use it to show that union of two convex poygons can triangulated by convex triangles.
Apply the last stattement recurcevely to a finite cove of $\spc{P}$ by convex polygons.

\parit{\ref{SHORT.ex:approximation:poly}.}
By comparison, $\tilde{\spc{P}}$ has nonnegative curvature.
It remains to apply \ref{prop:poly-CBB}.

%\parit{\ref{SHORT.ex:approximation:diangle}.}
%Suppose $\ell=\length\gamma_1$, and $\gamma_1$ is parametrized by $[0,\ell]$.
%Consider the function $f\:t\mapsto \dist{\gamma_1(0)}{\gamma_1(t)}{\Upsilon}$.
%Apply the Gauss--Bonnet formula (\ref{ex:gauss-bonnet})
%to show that $f'(t)\ge \cos\omega$ is the left-hand side is defined.
%Observe that $f$ is Lipschitz and make a conclusion.\footnote{With a bit more work, one can improve the inequality to
%\[\cos \tfrac\omega2\cdot \length \gamma_1\le \length \gamma_0.\]}

\parbf{\ref{ex:approximation}}; \ref{SHORT.ex:approximation:excess}.
Show that the total angle around each vertex of the triangulation is at most $2\cdot\pi$ and argue as in \ref{ex:gauss-bonnet}.

\parit{\ref{SHORT.ex:approximation:length}.}
Apply the hinge comparison (\ref{angle}).

\parit{\ref{SHORT.ex:approximation:area}.}
The first inequality follows form the generalized vesion of Kiszbraun's theorem proved by Urs Lang and Viktor Schroeder \cite{lang-schroeder:kirszbraun, alexander-kapovitch-kirszbraun,alexander-kapovitch-petrunin2024};
we will not discuss its proof since this inequalty is not used in the proof.
In the proof of the second inequality, we use Alexandrov's idea \cite[X §~1]{alexandrov-1948}.

Suppose that a convex triangle $\Delta$ has vertices $x$, $y$, and $z$, with opposite side lengths $a$, $b$ and $c$ and angles $\alpha$, $\beta$, and $\gamma$, respectively.
Let $\tilde\Delta$ be its solid model triangle and let $\tilde \alpha$, $\tilde \beta$, and $\tilde \gamma$ the corresponding angles.

Use that $\log_x$ is a noncontracting map to show that
\[\area\Delta-\area\tilde\Delta\le \tfrac12\cdot(\alpha-\tilde\alpha)\cdot (\diam \Delta)^2+o(a).\]

Now let us subdivide $\Delta$ into triangles $\Delta_1,\dots,\Delta_n$ with common vertex $x$ and small opposite sides $a_1,\dots,a_n$; so, $a=a_1+\dots+a_n$.
The model triangles $\tilde\Delta_1,\dots,\tilde\Delta_n$ can be arranged on the plane with common corresponding sides.
This way we get a convex triangle $\tilde\Lambda$ with two straight sides $b$ amd $c$ and a polygonal side of total length $a$.
Summing up the above inequalities for $\Delta_1,\dots,\Delta_n$, we get
\[\area\Delta-\area\tilde\Lambda\le \tfrac12\cdot(\alpha-\tilde\alpha)\cdot (\diam\Delta)^2+o(a_1)+\dots +o(a_n).\]
It follows that given $\eps>0$, we can choose the subdivision so that
\[\area\Delta-\area\tilde\Lambda\le \tfrac12\cdot(\alpha-\tilde\alpha)\cdot (\diam\Delta)^2+\eps.\]

By the hinge comparison (\ref{angle}), the angles adjacent to polygonal sides of $\tilde\Lambda$ do not exceed $\beta$ and $\gamma$.
It follows that
\[\area\tilde\Lambda-\area\tilde\Delta\le \tfrac12\cdot(\beta+\gamma-\tilde\beta-\tilde\gamma)\cdot (\diam \Delta)^2.\]
Therefore
\[\area\Delta-\area\tilde\Delta\le \tfrac12\cdot\excess\Delta\cdot (\diam \Delta)^2+\eps\]
for arbitrary $\eps>0$, hence the result.



\parbf{\ref{ex:surf-S2}.}
We can assume that the origin lies in the interior of the convex body.
Consider the central projection from its surface, say $\Sigma$, to the sphere $\SSS^2$ centered at the origin.
Show that this projection $\Sigma\to \SSS^2$ is a homeomorphism.

\begin{wrapfigure}{r}{25mm}
\vskip-6mm
\centering
\includegraphics{mppics/pic-1160}
\vskip-0mm
\end{wrapfigure}

\parbf{\ref{pr:tetrahedron}}; \ref{SHORT.pr:tetrahedron:=}.
Cut the surface of $T$ along three edges coming from one vertex $v_1$ and unfold the obtained surface onto the plane.
Show that this way we get a triangle, the three vertices correspond to $v_1$ and the midpoints of sides correspond to the remaining three vertices.
Make a conclusion.

\parit{\ref{SHORT.pr:tetrahedron:perp}}.
Let $v_1,v_2,v_3,v_4\in\RR^3$ be the vertices of $T$.
From \ref{SHORT.pr:tetrahedron:=}, we have that 
\[|v_1-v_2|=|v_3-v_4|,\quad |v_1-v_3|=|v_2-v_4|,\quad|v_1-v_4|=|v_2-v_3|.\]
Use it to show that $\langle v_1-v_2,v_1+v_2-v_3-v_4\rangle=0$.
Make a conclusion.

\parbf{\ref{ex:surface-covergence}.}
We will use that the closest-point projection from the Euclidean space to a convex body is short;
that is, distance-nonexpanding \cite[13.3]{petrunin-zamora}.

Assume $K_\infty$ is nondegenerate.
Without loss of generality, we may assume that 
\[\cBall(0,r)\subset K_\infty\subset\cBall(0,1)\]
for some $r>0$.
Then there is a sequence $\eps_n\to 0$ such that 
\[ K_n\subset(1+\eps_n)\cdot K_\infty
\quad\text{and}\quad
K_\infty\subset(1+\eps_n)\cdot K_n\]
for each large $n$.

Given $x\in K_n$, denote by $g_n(x)$ the closest-point projection of $(1+\eps_n)\cdot x$ to $K_\infty$.
Similarly, given $x\in K_\infty$, denote by $h_n(x)$ the closest point projection of $(1+\eps_n)\cdot x$ to $K_n$.
Note that 
\begin{align*}
\dist{g_n(x)}{g_n(y)}{}&\le (1+\eps_n)\cdot\dist{x}{y}{}
\intertext{and}
\dist{h_n(x)}{h_n(y)}{}&\le (1+\eps_n)\cdot\dist{x}{y}{}.
\end{align*}

Denote by $\Sigma_\infty$ and $\Sigma_n$ the surface of $K_\infty$ and $K_n$ respectively. 
The above inequalities imply 
\begin{align*}
\dist{g_n(x)}{g_n(y)}{\Sigma_\infty}&\le (1+\eps_n)\cdot\dist{x}{y}{\Sigma_n}
\intertext{for any $x,y\in \Sigma_n$, and}
\dist{h_n(x)}{h_n(y)}{\Sigma_n}&\le (1+\eps_n)\cdot\dist{x}{y}{\Sigma_\infty}.
\end{align*}
for any $x,y\in \Sigma_\infty$.

By a degree argument the maps $g_n$ and $h_n$ are onto.
Apply \ref{ex:GH-po} and finish the proof.

Alternatively, since the closest-point projection cannot increase the length of curve, we also get
\begin{align*}
\dist{x}{h_n\circ g_n(x)}{\Sigma_\infty}&\le 10\cdot \eps_n
\\
\dist{y}{g_n\circ h_n(y)}{\Sigma_n}&\le 10\cdot \eps_n.
\end{align*}
for all large $n$.
Therefore, $g_n$ is a $\delta_n$-isometry $\Sigma_n\to\Sigma_\infty$ for a sequence $\delta_n\to 0$.

\parit{Comments.}
More generally, if a sequence of $m$-dimensional $\Alex\kappa$ spaces $\spc{A}_1,\spc{A}_2,\dots$ converges to $\spc{A}_\infty$ and $\dim \spc{A}_\infty=m<\infty$,
then $\partial \spc{A}_n$ equipped with the induced length metrics converge to  $\partial \spc{A}_\infty$.
This statement is a partial case of the theorem about extremal subsets proved by the second author \cite[1.2]{petrunin1997}.

\parbf{\ref{ex:alexandrov=<4}.}
\ref{SHORT.ex:alexandrov=<4:>=3}. Observe that curvature of any essential vertes is less than $2\cdot\pi$ and apply the Gauss--Bonnet formula \ref{ex:gauss-bonnet}.

\parit{\ref{SHORT.ex:alexandrov=<4:=3}.} Show that geodeiscs between essential vertices divide the surface into two flat triangles, which have to be isometric since their sides are equal.
Make a conclusion.

\parit{\ref{SHORT.ex:alexandrov=<4:4}.}
Show that geodeiscs between the essential vertices can be shoosen so that they do not cross each other;
that is, every pair of geodesics intersect only at the common vertex.
In this case they divide the surfase into plane triangles.

Since the curvature is nonnegative, the sum of three angles of the triangles at each vertex is at most $2\cdot\pi$.
Show that these triangles form a faces of tetrhedron (possibly degenerate to a quadrangle)
if the three angles at one (and therefore any) vertex satisfy the triangle inequality.
In the latter case, the sum of angles at each vertex is less than $\pi$.
Therefore the sum of all 12 angles of these 4 triangles has to be less than $4\cdot \pi$.

On the other hand, the angles of each triangle sum up to $\pi$.
Therefore the sum of all 12 angles has to be $4\cdot \pi$ --- a contradiction.

\parbf{\ref{ex:convex}.} Assume the contrary,
then there is a minimizing geodesic $\gamma\not\subset\Delta$ with ends $p$ and $q$ in $\Delta$.

Without loss of generality, we may assume that only one arc of $\gamma$ lies outside of $\Delta$.
Reflection of this arc  with respect to $\Pi$ together with the remaining part of $\gamma$ forms another curve $\hat\gamma$ from $p$ to $q$;
it runs partly along surface
and partly outside $K$,
but does not get in the interior of $K$.
Note that
\[\length\hat\gamma=\length\gamma.\]

Denote by $\bar\gamma$ the closest point projection of $\hat\gamma$ to $K$.
Since $K$ is convex, the projection decreases the length.
Therefore
the curve $\bar\gamma$ lies on the surface of $K$,
it has the same ends as $\gamma$,
and
\[\length\bar\gamma<\length\gamma.\]
This means that $\gamma$ is not length-minimizing
--- a contradiction.

\parbf{\ref{ex:milka}.}
If the plane $py_1y_2$ supports $K$, then 
$\mangle\hinge p{y_1}{y_2}_{\EE^3}=\mangle\hinge p{x_1}{x_2}_{\spc{P}}$.
In this case, the statement follows from \ref{prop:conv-surf-CBB(0)}.

Now suppose that the line segment $[y_1y_2]_{\EE^3}$ intersects $K$.
Choose a geodesic $[y_1y_2]_W$;
note that it contains a point of $K$, say $z$.
Now consider the one-parameter family of points 
\[y_i(t)\df \gamma_i(t)+\gamma_i^+(t)\z\cdot (1-t).\]
This family is not necessarily continuous; note that $y_i(0)=y_i$ and $y_i(1)=x_i$.

Show that for any point $q\in K$, the function $t\mapsto \dist{q}{y_i(t)}{\EE^3}$ is nonincreasing.
Conclude that the function $t\mapsto \dist{q}{y_i(t)}{W}$ is nonincreasing for any $q\in \spc{P}$.
Therefore, 
\begin{align*}
\dist{y_1}{y_2}{W}
&=\dist{y_1(0)}{y_2(0)}{W}=
\\
&=\dist{y_1(0)}{z}{W}+\dist{y_2(0)}{z}{W}\ge
\\
&\ge\dist{y_1(1)}{z}{W}+\dist{y_2(1)}{z}{W}\ge
\\
&\ge\dist{x_1}{x_2}{\spc{P}}.
\end{align*}
The last inequality follows since the closest point projection $W\to \spc{P}$ is short.

\begin{wrapfigure}{o}{28mm}
\vskip-3mm
\centering
\includegraphics{mppics/pic-1111}
\vskip-0mm
\end{wrapfigure}

It remains to consider the case when the plane $py_1y_2$ does not support $K$,
and $[y_1y_2]_{\EE^3}$ does not intersect $K$.
In this case the plane $py_1y_2$ intersects $K$ along a convex figure $F$ that lies in the solid triangle 
$py_1y_2$ and contains its vertex $p$.

Choose points $y_1'\in [py_1]_{\EE^3}$ and $y_2'\in [py_2]_{\EE^3}$ such that $[y_1'y_2']$ touches~$F$.
Denote by $x_1'\in [px_1]_{\spc{P}}$ and $x_2'\in [px_2]_{\spc{P}}$ the corresponding points;
that is, $\dist{p}{y_1'}{\EE^3}=\dist{p}{x_1'}{\spc{P}}$ and $\dist{p}{y_2'}{\EE^3}=\dist{p}{x_2'}{\spc{P}}$.
From  above, we have that $\dist{y_1'}{y_2'}{\EE^3}\ge\dist{x_1'}{x_2'}{\spc{P}}$;
in other words, 
\[\angk p{y_1'}{y_2'}\ge \angk p{x_1'}{x_2'};\]
here we think of  $[p{y_1'}{y_2'}]$ as a triangle in $\EE^3$, but $[p{x_1'}{x_2'}]$ as a triangle in $\spc{P}$.
Note that 
\[\angk p{y_1'}{y_2'}=\angk p{y_1}{y_2}
\quad\text{and}\quad
\angk p{x_1}{x_2}\le \angk p{x_1'}{x_2'};
\]
the second inequality follows from \ref{ex:noncreasing}.
Hence the remaining case follows.
