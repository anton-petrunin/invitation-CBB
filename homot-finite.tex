%%!TEX root = the-homot-finite.tex
\section{Controlled concavity}

Alexandrov spaces have plenty of semiconcave functions;
for instance, squared distance functions. 
The following theorem provides a source of strictly concave functions  defined on small open sets of finite-dimensional Alexandrov spaces. 

\begin{thm}{Theorem}
\label{thm:strictly-concave}
Let $\spc{A}$ be a complete finite-dimensional Alexandrov  space.
Then for any point $p\in \spc{A}$, there is  a strictly concave function $f$ defined in an
open neighborhood of $p$.

Moreover, given $0\ne v\in T_p$, the differential, $\dd_p f$, can be chosen
arbitrarily close to $x\mapsto -\<v,x\>$.
\end{thm}

\begin{wrapfigure}{o}{44 mm}
\vskip-0mm
\centering
\includegraphics{mppics/pic-901}
\vskip1mm
\end{wrapfigure}

\parit{Proof.} 
Fix small $r>0$ and large $c$;
consider the real-to-real function 
$$\phi_{r,c}(x)=(x-r)- c\cdot(x-r)^2/r,$$
so we have 
$\phi_{r,c}(r)=0$,
$\phi_{r,c}'(r)=1$,
and $\phi_{r,c}''(r)=- {2c}/{r}$. 

Let $\gamma$ be a unit-speed geodesic, fix a point $q$ and let 
$$\alpha(t)=\mangle(\gamma^+(t),\dir{\gamma(t)}{q}).$$
Recall that $r$ is small.
If $\dist q{\gamma(t)}{}$ is sufficiently close to
$r$, then direct calculations show that
$$(\phi_{r,c}\circ\distfun_q\circ\gamma)''(t)
\le 
\frac{3-c\cdot \cos^2[\alpha(t)]}{r}.$$
(Since $c$ is large, this inequality implies that $\phi_{r,c}\circ\distfun_q\circ\gamma$ is strictly concave at $t$ unless $\alpha(t)\approx\tfrac\pi2$.) 

Now, assume $\{q_1,\dots, q_N\}$ is a finite set of points such that $\dist p{q_i}{}=r$ for any $i$. 
For a geodesic $\gamma$, set $\alpha_i(t)=\mangle(\gamma^+(t),\dir {\gamma(t)}{q_i})$. 
Assume that
\[\max_i\{|\alpha_i(t)-\tfrac\pi2|\}\ge\eps>0\]
for any geodesic $\gamma$ in $\oBall(p,\eps)$. 
We can assume that $c>3N/\cos^2\eps$;
then the inequality above implies that the function
$$f=\sum_i \phi_{r,c}\circ\distfun_{q_i}$$
is strictly concave in $\oBall(p,\eps')$ for some positive $\eps'<\eps$.

The same argument as in \ref{ex:pack-vol} shows that for small $r>0$, one can
choose $N\ge \Const/\delta^{m-1}$ points $\{q_i\}$ such that $\dist{p}{q_i}{}=r$
and $\angk p{q_j}{q_i}>\delta$ (here $\Const=\Const(\Sigma_p)>0$).
On the other hand, suppose $\gamma$ runs from $x$ to~$y$.
If $|\alpha_i(t)- \tfrac\pi2|<\eps\ll\delta$, then applying the ($n$+1)-point comparison to $\gamma(t)$, $x$, $y$ and all $\{q_i\}$ we get that
$N\le \Const(m)/\delta^{m-2}$. 
Therefore, for small $\delta>0$ and yet smaller $\eps>0$, the set $\{q_i\}$ forms the needed collection.

If $r$ is small, then points $q_i$ can be chosen so that all directions
$\dir p {q_i}$ will be $\eps$-close to a given direction $\xi$ and
therefore the second property follows.
\qeds

The function $f$ in \ref{thm:strictly-concave} can be chosen to have maximum value $0$ at $p$,
$f(p)=0$ and with $\dd_p f(x)\approx-|x|$.
It can be constructed by taking the minimum of the functions in the theorem.
Then the set $K=\set{x\in\spc{A}}{f(x)\ge -\eps}$ forms a closed convex neighborhood of $p$ for any small $\eps>0$, so we get the following.


\begin{thm}{Corollary}\label{cor:convex-nbhd}
Any point $p$ of a finite-dimensional Alexandrov space admits an arbitrarily small convex closed neighborhood $K$ and a strictly concave function $f$ defined in a neighborhood of $K$ such that $p$ is the maximum point of $f$
and $f|_{\partial K}=0$.
\end{thm}

\begin{thm}{Exercise}\label{ex:no-conc}
Construct an Alexandrov space $\spc{A}$ such that
there is no open set $\Omega\subset A$ with strictly concave function $f\:\Omega\to\RR$.
\end{thm}


\section{Liftings}

Suppose that the Gromov--Haudorff distance $\dist{\spc{A}}{\spc{A}'}{\GH}$ is sufficiently small, so we may think that both spaces $\spc{A}$ and $\spc{A}'$ lie at a small Hausdorff distance in an ambient metric space $\spc{W}$.
In particular, we may choose a small $\eps>0$, so that for any point $p\in \spc{A}$, there is a point $p'\in \spc{A}'$ such that $\dist{p}{p'}{\spc{W}}<\eps$;
the point $p'$ will be called a \index{lifting}\emph{lifting} (or \emph{$\eps$-lifting}) of $p$ in $\spc{A}'$.
We may choose a lifting $p'\in\spc{A}'$ for every point $p\in\spc{A}$, 
in this case the map $p\mapsto p'$ is called a {}\emph{($\eps$-)lifting map}.

Let us emphasise that liftings are not uniquely defined, and the lifting map is not assumed to be continuous.
Also, to talk about liftings, we have to choose that $\eps>0$, the inclusions $\spc{A},\spc{A}'\hookrightarrow\spc{W}$.

Let $\spc{A}$ be  a compact $m$-dimensional Alexandrov space.
Suppose $\spc{A}'$ is another compact $m$-dimensional Alexandrov space such that $\dist{\spc{A}}{\spc{A}'}{\GH}$ is sufficiently small --- smaller than some $\eps=\eps(\spc{A})>0$.
Then the construction in $\spc{A}$ from the previous section  
can be repeated in $\spc{A}'$ for the liftings of all points and the same function $\phi$.
It produces a strictly concave function defined in a controlled neighborhood of the lifting $p'$ of $p$.

The results of this and related constructions will be also called \index{lifting}\emph{liftings},
say we can talk about a lifting from $\spc{A}$ to $\spc{A}'$ of a function provided by \ref{thm:strictly-concave} (if the Gromov--Hausdorff distance $\dist{\spc{A}}{\spc{A}'}{\GH}$ is small, then these liftings are strictly concave)
and a lifting of a convex neighborhood from \ref{cor:convex-nbhd}.
Here one cannot use \ref{thm:strictly-concave} and \ref{cor:convex-nbhd} as black boxes --- one has to understand the construction, but it is straightforward.

\begin{thm}{Exercise}\label{ex:no-cont-lifting}
Give an example of  Gromov--Hausdorff convergence of spaces $\spc{A}_n\z\to \spc{A}_\infty$ such that all $\spc{A}_n$ and $\spc{A}_\infty$ are compact finite-dimensional $\Alex0$ and for any small $\eps>0$  there is no continuous $\eps$-lifting map $\spc{A}_\infty\to \spc{A}_n$ for any large $n$.
\end{thm}

\section{Nerves}

Let $\{\Omega_1,\dots,\Omega_k\}$ be a finite open cover of a compact metric space $\spc{X}$.
Consider an abstract simplicial complex $\spc{N}$, with one vertex $v_i$ for each set $\Omega_i$ such that a simplex with vertices $v_{i_1},\dots, v_{i_m}$ is included in $\spc{N}$ if 
the intersection $\Omega_{i_1}\cap\dots\cap \Omega_{i_m}$ is non-empty.
\begin{figure}[ht!]
\vskip-0mm
\centering
\includegraphics{mppics/pic-1402}
\end{figure}
The obtained simplicial complex $\spc{N}$ is called the \index{nerve}\emph{nerve} of the covering $\{\Omega_i\}$.
Evidently, $\spc{N}$ is a finite simplicial complex ---
it is a subcomplex of a simplex with the vertices $\{v_1,\dots,v_k\}$.
Recall that $\Star_{v_i}$ denotes the union of all simplices in $\spc{N}$ that share the vertex $v_i$.

The next statement follows from \cite[4G.3]{hatcher}.


\begin{thm}{Nerve theorem}\label{thm:nerve}
Let $\{\Omega_1,\dots,\Omega_k\}$ be an open cover of a compact metric space $\spc{X}$
and let $\spc{N}$ be the corresponding nerve with vertices $\{v_1,\dots,v_k\}$.
Suppose that every non-empty finite intersection $\Omega_{\alpha_1}\cap\z\dots\cap\Omega_{\alpha_k}$ is contractible.
Then $\spc{X}$ is homotopy equivalent to the nerve $\spc{N}$ of the cover.

Moreover homotopy equivalences  $a\:\spc{X}\to \spc{N}$ and $b\:\spc{N}\to\spc{X}$ can be chosen so that 
if $x\in \Omega_i$, then $a(x)\in \Star_{v_i}$,
and if $y\in\spc{N}$ lies in the simplex with vertices $v_{i_1},\dots, v_{i_m}$, then $b(y)\in \Omega_{i_1}\cup\dots\cup \Omega_{i_m}$.
\end{thm}


%???Вить, посмотри на это утверждение --- оно мне не сильно нравится.


\section{Homotopy stability}

\begin{thm}{Theorem}\label{thm:h-stability}
Let $\spc{A}_1,\spc{A}_2,\dots$, and $\spc{A}_\infty$ be compact $m$-dimensional $\Alex\kappa$ spaces, and $m<\infty$.
Suppose $\spc{A}_n\z\to \spc{A}_\infty$ as $n\to \infty$ in the sense of Gromov--Hausdorff.
Then $\spc{A}_\infty$ is homotopy equivalent to $\spc{A}_n$ for all large $n$.

Moreover, given $\eps>0$ there are maps $h_n\:\spc{A}_\infty\to \spc{A}_n$ that are homotopy equivalences and $\eps$-liftings for all large $n$.
\end{thm}

Applying this theorem with Gromov's selection theorem (\ref{thm:gromov-compactness}) and Exercise \ref{ex:pack-vol}, we get the following.


\begin{thm}{Theorem}\label{thm:h-finiteness}
Given $\kappa\in \RR, D>0, v_0>0$ and a positive integer $m$,
there are only finitely many homotopy types of $m$-dimensional $\Alex\kappa$ spaces with diameter $\le D$, and volume $\ge v_0$.
\end{thm}

\parit{Proof of \ref{thm:h-finiteness} modulo \ref{thm:h-stability}.}
Assume the contrary, then we can choose a sequence of $m$-dimensional $\Alex\kappa$ spaces $\spc{A}_1,\spc{A}_2,\dots$   that have different homotopy types and satisfy the assumptions of the theorem.
By Gromov's selection theorem, we can assume that $\spc{A}_n$ converges to some spaces $\spc{A}_\infty$ in the sense of Gromov--Hausdorff.

By \ref{ex:pack-vol}, $\dim \spc{A}_\infty=m$.
It remains to apply \ref{thm:h-stability}.
\qeds

\parit{Proof of \ref{thm:h-stability}.}
Since $\spc{A}_\infty$ is compact, applying \ref{cor:convex-nbhd}, we can find a finite open cover of $\spc{A}_\infty$ by convex open sets $\Omega_1,\dots, \Omega_k$ such that 
for each $\Omega_i$ there is a strictly concave function $f_i$ that is defined in a neighborhood of the closure $\bar \Omega_i$ and such that $f_i|_{\partial \Omega_i}=0$.

Subtracting from functions $f_i$ some small value $\eps>0$,
we can ensure that $\bigcap_{i\in S}\Omega_{i}\ne \emptyset$ if and only if $\bigcap_{i\in S}\bar\Omega_{i}\ne \emptyset$.

Suppose that $W=\bigcap_{i\in S}\Omega_{i}\ne \emptyset$.
Then $W$ is contractible.
Indeed, the function 
\[f_S\df\min_{i\in S} f_i\]
is strictly concave and it vanishes on the boundary of $W$.
The $f_S$-gradient flow $(t,x)\mapsto \GF_{f_S}^t(x)$ defines a homotopy
$[0,\infty)\times W\z\to W$.
By the first distance estimate (\ref{thm:dist-est}), $\GF_{f_S}^t(x)$ converges to the (necessarily unique) maximum point of $f_S$ as $t\to\infty$.
Therefore, in the obtained homotopy we can parametrize $[0,\infty)$ by $[0,1)$ and extend the homotopy continuously to $[0,1]$;
thus we get that $W$ is contractible.
In other words, the cover $\{\Omega_1,\dots, \Omega_k\}$ meets the assumptions of the nerve theorem (\ref{thm:nerve}).

The functions $f_i$ and sets $\Omega_i$ can be lifted to $\spc{A}_n$ keeping their properties for all large $n$. 
More precisely, there are liftings $f_{i,n}$ of all $f_i$ to $\spc{A}_n$ which are strictly concave for all large $n$ and such that $\bar\Omega_{i,n}=\set{x\in \spc{A}_n}{f_{i,n}(x)\ge 0}$ is a compact convex set and $\Omega_{i,n}\z=\set{x\in \spc{A}_n}{f_{i,n}(x)> 0}$ is an open convex set for each $i$.

Notice that $\{\Omega_{1,n},\dots,\Omega_{k,n}\}$ is an open cover of $\spc{A}_n$ for all large~$n$.
Indeed suppose we have $p_n\in \spc{A}_n\setminus(\Omega_{1,n}\cup\dots\cup\Omega_{k,n})$ for arbitrary large $n$.
Since $\spc{A}_\infty$ is compact, there is a limit point $p_\infty\in \spc{A}_\infty$ for a subsequnce of $p_n$.
But $p_\infty\in\Omega_i$ for some $i$ and therefore $p_n\in \Omega_{i,n}$ for arbitrary large $n$ --- a contradiction.

In a similar fashion, we can show that if $n$ is large, then any collection $\{\Omega_{i,n}\}_{i\in S}$ has a common point in $\spc{A}_n$ 
if and only if $\{\Omega_{i}\}_{i\in S}$ has a common point in $\spc{A}_\infty$.
Here we have to use that $\bigcap_{i\in S}\Omega_{i}\ne \emptyset$ if and only if $\bigcap_{i\in S}\bar\Omega_{i}\ne \emptyset$.

It follows that for any large $n$ the covers 
\begin{itemize}
\item $\{\Omega_{1},\dots,\Omega_{k}\}$ of $\spc{A}_\infty$ and 
\item $\{\Omega_{1,n},\dots,\Omega_{k,n}\}$ of $\spc{A}_n$.
\end{itemize}
have the same nerve.
By the nerve theorem (\ref{thm:nerve}), $\spc{A}_n$ and $\spc{A}_\infty$ are homotopy equivalent for all large $n$ --- a contradiction.
\qeds

The proof above implies the following.

\begin{thm}{Theorem}\label{thm:finite-dim-hom-simplicial}
Any compact finite-dimensional Alexandrov space is homotopy equivalent to a finite simplicial complex.
\end{thm}

\section{Remarks}

Originally, Gromov's selection theorem was proved for Riemannian manifolds with a lower bound on Ricci curvature \cite{gromov1981}.
It motivates the study of limits of manifolds with lower Ricci curvature bounds and their synthetic generalizations, the so-called $\mathrm{CD}(K,m)$ spaces; $\mathrm{CD}$ stands for curvature-dimension condition.
This theory has significant applications in Alexandrov geometry;
in particular, it provides a version of the Liouville theorem about phase-space volume of geodesic flow in Alexandrov space \cite{brue-mondino-semola}.

The construction of a strictly concave function (\ref{thm:strictly-concave}) is due to Grigory Perelman \cite{perelman1993,perelman-petrunin}.

The homotopy-type finiteness theorem (\ref{thm:h-finiteness}) illustrates the main source of applications of Alexandrov spaces: to prove a statement about 
 $m$-dimensional manifolds with lower sectional curvature bound we argue by contradiction and assume that there is a sequence of such manifolds  that violates our assumption, 
then pass to the limit --- the limit must be an Alexandrov space and this can be used to arrive at a contradiction.

In principle, the same strategy might work for a sequence of Riemannian manifolds with dimension approaching infinity, but no applications of this kind were found.
The following question can be attacked by this type of argument.

\begin{thm}{Question}
Is it true that no simply connected closed manifold (of arbitrary large dimension) admits a Riemannian metric with sectional curvature and diameter bounded by a fixed positive sufficiently small value?
\end{thm}

If the dimension is bounded, then Gromov's theorem \cite{gromov1978} implies that such a manifold can be covered by a compact nil-manifold; in particular, the manifold cannot be simply connected.
However, if  the dimension is unbounded, then Riemannian manifolds with these conditions may not be covered by compact nil-manifolds; such examples were found by Galina Guzhvina \cite{guzhvina2008}.

Let us finish with a list of results that can be proved by applying Gromov's selection theorem
in the same fashion as in the proof of homotopy-type finiteness theorem (\ref{thm:h-finiteness}).

\begin{thm}{Betti-number theorem}\label{thm:betti}
For any   $\kappa\in \RR, D>0$ and a positive integer $m$,
there is a constant $\Const=\Const(m,D,\kappa)$ such that 
\[\beta_0(M)+\dots+\beta_m(M)\le \Const\]
for any closed $m$-dimensional Riemannian manifold $M$ with sectional curvature $\ge \kappa$ and diameter $\le D$.
Here $\beta_i(M)$ denotes $i^\text{th}$ Betti number of $M$.
\end{thm}

Gromov's original proof \cite{gromov-1981} of the Betti-number theorem did not use Alexandrov geometry directly;
but it is quite natural to prove it via Gromov's selection theorem.
The following result was proved by the second author \cite{petrunin2008}, and it uses the same technique.

\begin{thm}{Scalar curvature bound}
Given $\kappa\in \RR, D>0$ and a positive integer $m$, there is a constant $\Const=\Const(m,D,\kappa)$ such that
\[\int_M\Sc\le \Const\]
for any closed $m$-dimensional Riemannian manifold $M$ with sectional curvature $\ge \kappa$ and diameter $\le D$.
Here $\Sc$ denotes the scalar curvature.
\end{thm}

The following theorem is a more exact version of \ref{thm:h-stability}.
Its close relative (\ref{thm:spherical-nbhd}) will play an important role in the following lecture.

\begin{thm}{Stability theorem}\label{thm:stability}
Let $\spc{A}_1,\spc{A}_2,\dots$, and $\spc{A}_\infty$ be compact $m$-dimensional $\Alex\kappa$ spaces, and $m<\infty$.
Suppose $\spc{A}_n\to \spc{A}_\infty$ as $n\to \infty$ in the sense of Gromov--Hausdorff.
Then $\spc{A}_\infty$ is homeomorphic to $\spc{A}_n$ for all large $n$.

Moreover, given $\eps>0$ there are maps $h_n\:\spc{A}_\infty\to \spc{A}_n$ that are homeomorphisms and $\eps$-liftings for all large $n$.
\end{thm}

This theorem was proved by Grigory Perelman \cite{perelman1991};
the proof was rewritten with more details by the first author \cite{kapovitch}.
In private conversations, Perelman claimed that the homeomorphisms in the theorem can be assumed to be bi-Lipschitz with constants that depend on $\spc{A}_\infty$.
Since no proof has been written, this statement should be considered as a conjecture;
partial results in this direction were obtained by Mohammad Alattar \cite{alattar}.

Theorem \ref{thm:h-finiteness} was originally proved by Karsten Grove and Peter Petersen \cite{grove-petersen1988}.
Perelman's stability theorem (\ref{thm:stability}) implies the following stronger statement.

\begin{thm}{Homeomorphism-type finiteness}
For any   $\kappa\in \RR, D>0, v>0$ and a positive integer $m$, there are only finitely many homeomorphism types of closed $m$-dimensional manifolds that admit a Riemannian metric with sectional curvature $\ge \kappa$, volume $\ge v$, and diameter $\le D$.
\end{thm}

This statement can be improved to diffeomorphism-type finiteness in all dimensions $m\ne 4$.
Indeed, for $m=4$ a closed topological $m$-manifold admits only finitely many smooth structures; see \cite{kirby-siebenmann} and \cite{moise,thurston} for cases $m\ge 5$ and $m\le 3$, respectively.


