%%!TEX root = the-homot-finite.tex
\section{Controlled concavity}

Alexandrov spaces have plenty of semiconcave functions, including squared distance functions and their convex combinations.
The following theorem provides a source of strictly concave functions  defined on small open sets of finite-dimensional Alexandrov spaces. 

\begin{thm}{Theorem}
\label{thm:strictly-concave}
Let $\spc{A}$ be a complete finite-dimensional Alexandrov  space.
Then for any point $p\in \spc{A}$, there is  a strictly concave function $f$ defined in an
open neighborhood of $p$.

Moreover, given $0\ne v\in \T_p$, the differential, $\dd_p f$, can be chosen
arbitrarily close to $x\mapsto -\<v,x\>$.
\end{thm}

\parit{Idea of the proof.}
Suppose $f$ is a semiconcave function.
One can improve concavity of $f$ by passing to a composition $\phi\circ f$ with a vary concave real-to-real function $\phi$ such that $\phi'\approx 1$.
Namely, if $|(f\circ\gamma)^\pm|$ is bounded away from zero for a geodesic $\gamma$, then $\phi\circ f\circ\gamma$ can be made as concave as you wish.
This trick does not help if $|(f\circ\gamma)^\pm|\approx 0$, but it does not make cocavity worse;
so $\phi\circ f$ is very concave in most of directions and at least as concave as $f$ in all directios.
In the proof we consider sum $\phi\circ f_1+\ldots+\phi\circ f_N$ for sutably chousen sequence of functions $f_1,\ldots,f_N$, so that for any geodesic $\gamma$ in a smalle neighborghood of $p$ the value $|(f_i\circ\gamma)^\pm|$ is bounded away from zero for some $i$.

\medskip

Before getting into the proof, we need to refresh material from \ref{Function comparison} and \ref{sec:packing-numbers}.

\begin{thm}{Lemma}\label{lem:barrier}
Let $f$ be a locally Lipschitz function defined on a real interval $\II$.
Then $f$ is $\lambda$-concave if and only if for any $t_0\in\II$ there is smooth (local) upper barrier
$\bar f$ of $f$ at $t_0$ such that $\bar f''(t_0)\le \lambda$.

Moreover, in the only-if part, we can assume in addition, that $\bar f'(t_0)$ takes a given value in the interval $[f^+(t_0),f^-(t_0)]$ (by \ref{ex:concave'}, this interval is not empty).
\end{thm}

\parit{Proof.}
Passing to function $t\mapsto f(t)-\lambda\cdot\tfrac{t^2}2$, we can assume that $\lambda=0$.

The only-if part follows since concave function is supported by the linear function $t\mapsto f(t_0)+ \Const\cdot (t-t_0)$ for any $\Const\in [f^+(t_0),f^-(t_0)]$.

On the other hand, suppose $f$ is not concave.
Then adding a linear function to $f$ we can assume that there is a subinterval $[a,b]\subset\II$ such that
\[
f(a)>0,
\quad
f(b)>0,
\quad\text{but}\quad
f(t)<0
\quad\text{for some}\quad
t\in [a,b].
\]
Let $s$ be the minimal value such that $f(t)\ge \tfrac s2\cdot (t-a)\cdot(t-b)$ for any $t\in [a,b]$.
Note that $s>0$ and $f(t_0)=\tfrac s2\cdot (t_0-a)\cdot(t_0-b)$ for some $t_0\in (a,b)$;
that is, $t\mapsto \tfrac s2\cdot (t-a)\cdot(t-b)$ is a local lower barrier of $f$ at $t_0$.
Therefore, $\bar f''(t_0)\ge s>0$ for any upper barrier $\bar f$ of $f$ at $t_0$,
and the if part follows.
\qeds

\begin{thm}[!]{Exercise}\label{ex:barrier}
Let $f$ be Lipschitz real-to-real function, and let $\phi$ be a smooth function defined in a neighborhood of $x_0=f(t_0)$.
Suppose that $f$ is $\lambda_1$-concave for some $\lambda_1\ge 0$ and there are some constants $c_1,c_2,c_3\in\RR$ such that (1) $|f^+(t_0)|>c_1>0$ (or $|f^-(t_0)|>c_1>0$), (2) $c_2>\phi'(x_0)\ge 0$, and (3) $0>c_3\ge \phi''(x_0)$.
Let $\lambda_2=c_2\cdot\lambda_1+c_1^2\cdot c_3$.
Show that $\phi\circ f$ is $\lambda_2$-concave in a neighborhood of~$t_0$.
\end{thm}

\begin{thm}[!]{Exercise}\label{ex:pack-sphere}
Let $\Omega$ be a nonempty open set in an $m$-dimensional Alexandrov space $\spc{A}$; assume $m<\infty$.
Given a point $p$ there are $r>0$ and $\Const>0$ such that for any sufficiently small $\eps>0$ there are at least $\Const/\eps^{m-1}$ points $q_1,\dots,q_N\in \Omega$ such that $\dist{p}{q_i}{}=r$ for any $i$ and $\mangle\hinge p{q_i}{q_j}>\eps$ if $i\ne j$.
\end{thm}


\parit{Proof of \ref{thm:strictly-concave}.}
Suppose $r>0$ small and $c$ is large (these values will be specified a bit latter).
Consider the real-to-real function
$$\phi_{r,c}(x)=(x-r)- c\cdot(x-r)^2/r;$$
so,
\begin{align*}
\phi_{r,c}(r)&=0,
&
\phi_{r,c}'(r)&=1,
&
\phi_{r,c}''(r)&=-\frac{2\cdot c}{r}.
\end{align*}

{

\begin{wrapfigure}{o}{44 mm}
\vskip-4mm
\centering
\includegraphics{mppics/pic-901}
\vskip1mm
\end{wrapfigure}

Let $\gamma$ be a geodesic.
Given points $q_1,\dots, q_N$, set
\begin{align*}
\alpha_i(t)&=\mangle(\gamma^+(t),\dir{\gamma(t)}{q_i}),
\\
\lambda_i(t)&=\frac{3-c\cdot \cos^2[\alpha_i(t)]}{r},
\\
h_i&=\phi_{r,c}\circ\distfun_{q_i}\circ\gamma.
\end{align*}

}

Suppose that $\dist{q_i}{\gamma(t_0)}{}$ is sufficiently close to
$r$.
Then by \ref{ex:barrier} we have that
$h_i$
is $\lambda_i(t_0)$-concave in a neighborhood of $t_0$.

Assume that $\dist{p}{q_i}{}=r$ for each $i$, and $\eps_1>0$ is small.
If the geodesic $\gamma$ lies in $\oBall(p,\eps_1)$, then from above we have that $\sum h_i$ is $\sum\lambda_i(t_0)$-concave in a neighborhood of any $t_0$.

Suppose that there is $\eps_2>0$ such that
\[\max_i\{|\alpha_i(t)-\tfrac\pi2|\}\ge\eps_2>0\eqlbl{eq:angle-ne-pi/2}\]
for any geodesic $\gamma$ in $\oBall(p,\eps_1)$.
If $c>3\cdot N/\sin^2\eps_2$, then $\sum\lambda_i(t_0)<0$ and
$\sum h_i$ is strictly concave in a neighborhood of $t_0$ for any choice of geodesic $\gamma$ in $\oBall(p,\eps_1)$.
It follows that
\[f=\sum_i \phi_{r,c}\circ\distfun_{q_i}\]
is strictly concave in $\oBall(p,\eps_1)$.

It remains to choose $r,\eps_1,\eps_2>0$ and points $q_1,\dots, q_N$ such that $\dist{p}{q_i}{}=r$ for each $i$, and \ref{eq:angle-ne-pi/2} holds.

Let us assume that $\spc{A}$ is $\Alex0$.
Applying \ref{ex:pack-sphere} to a small open ball around $p$ and small $\eps_3>0$,
we can choose $N\ge \Const_1/\eps_3^{m-1}$ points $q_1,\dots,q_N$ such that $\dist{p}{q_i}{}=r$ for some small $r>0$
and $\angk p{q_j}{q_i}>\eps_3$ (here $\Const_1>0$).
On the other hand, suppose $\gamma$ runs from $x$ to~$y$.
Let us apply the ($n$+1)-point comparison (\ref{thm:n+1}) to $\gamma(t)$, $x$, $y$ and $q_1,\dots,q_N$;
denote the obtained points by $\tilde x, \tilde y,\tilde q_1,\dots,\tilde q_N$.
If $\eps_1\ll \eps_3$, $\eps_2\ll \eps_3$, and
$|\alpha_i(t)- \tfrac\pi2|<\eps_2$ for each $i$, then the points $\tilde q_1,\dots,\tilde q_N$ lie very close to an $r$-sphere in hyperplane in $\EE^m$ that is perpendicular to $[\tilde x\tilde y]$;
in particular, $N\le \Const_2(m)/\eps_3^{m-2}$.
Therefore, for small $\eps_3>0$ and yet smaller $\eps_1$ and $\eps_2$, points $q_1,\dots,q_N$ meet \ref{eq:angle-ne-pi/2}.


To prove the last statement, choose a geodesic $[pz]$ in the direction sufficiently close to $v$ and apply \ref{ex:pack-sphere} to a small ball centered on a point $\bar z\in [pz]$ close to $p$.

The $\Alex\kappa$-case can be reduced to $\Alex{-1}$-case by rescaling, and it is done the same way.
The only difference is that the points $\tilde q_1,\dots,\tilde q_N$ lie very close to an $r$-sphere in hyperplane in $\HH^m$.
\qeds

The function $f$ in \ref{thm:strictly-concave} can be chosen to have maximum value $0$ at $p$,
$f(p)=0$ and with $\dd_p f(x)\approx-|x|$.
It can be constructed by taking the minimum of the functions in the theorem.
Then the set $K=\set{x\in\spc{A}}{f(x)\ge -\eps}$ forms a closed convex neighborhood of $p$ for any small $\eps>0$, so we get the following.


\begin{thm}{Corollary}\label{cor:convex-nbhd}
Any point $p$ of a finite-dimensional Alexandrov space admits an arbitrarily small convex closed neighborhood $K$ and a strictly concave function $f$ defined in a neighborhood of $K$ such that $p$ is the maximum point of $f$
and $f|_{\partial K}=0$.
\end{thm}

\begin{thm}{Exercise}\label{ex:no-conc}
Construct an Alexandrov space $\spc{A}$ such that
there is no open set $\Omega\subset A$ with strictly concave function $f\:\Omega\to\RR$.
\end{thm}

\begin{thm}{Exercise}\label{ex:dist-chart+strictly-concave}
Suppose $p,a_0,\dots,a_m$ are points in an $m$-dimensional Alexandrov space $\spc{A}$ such that $\dist{p}{a_i}{}=r$ forany $i$ and $\angk p{a_i}{a_j}>\tfrac\pi2$
for any $i\ne j$.
Let $\phi_{c,r}$ be as in the proof of \ref{thm:strictly-concave} and $c$ is large.
Show that the function $f=\phi_{c,r}\circ\distfun_{a_1}+\dots+\phi_{c,r}\circ\distfun_{a_m}$
is a strictly function in a neighborhood of $p$.
\end{thm}


\section{Liftings}\label{sec:Liftings}

Suppose that the Gromov--Hausdorff distance $\dist{\spc{A}}{\spc{A}'}{\GH}$ is sufficiently small, so we may think that both spaces $\spc{A}$ and $\spc{A}'$ lie at a small Hausdorff distance in an ambient metric space $\spc{W}$.
In particular, we may choose a small $\eps>0$, so that for any point $p\in \spc{A}$, there is a point $p'\in \spc{A}'$ such that $\dist{p}{p'}{\spc{W}}<\eps$;
the point $p'$ will be called a \index{lifting}\emph{lifting} (or \emph{$\eps$-lifting}) of $p$ in $\spc{A}'$.
We may choose a lifting $p'\in\spc{A}'$ for every point $p\in\spc{A}$, 
in this case the map $p\mapsto p'$ is called a {}\emph{($\eps$-)lifting map}.

Let us emphasize that liftings are not uniquely defined, and the lifting map is not assumed to be continuous.
Also, to talk about liftings, we have to choose $\eps>0$, as well as the inclusions $\spc{A},\spc{A}'\hookrightarrow\spc{W}$.

Let $\spc{A}$ be  a compact $m$-dimensional Alexandrov space.
Suppose $\spc{A}'$ is another compact $m$-dimensional Alexandrov space such that $\dist{\spc{A}}{\spc{A}'}{\GH}$ is sufficiently small --- smaller than some $\eps=\eps(\spc{A})>0$.
If the construction of the previous section is performed in $\spc{A}$,
then it can be repeated in $\spc{A}'$ for the liftings of all points and the same function $\phi$.
It produces a strictly concave function defined in a controlled neighborhood of the lifting $p'$ of $p$.

The results of this and related constructions will also be called \index{lifting}\emph{liftings};
say we can talk about a lifting from $\spc{A}$ to $\spc{A}'$ of a function provided by \ref{thm:strictly-concave} (if the Gromov--Hausdorff distance $\dist{\spc{A}}{\spc{A}'}{\GH}$ is small, then these liftings are strictly concave)
and a lifting of a convex neighborhood from \ref{cor:convex-nbhd}.
Here one cannot use \ref{thm:strictly-concave} and \ref{cor:convex-nbhd} as black boxes --- one has to understand the construction, but it is straightforward.

\begin{thm}{Exercise}\label{ex:no-cont-lifting}
Give an example of  Gromov--Hausdorff convergence of spaces $\spc{A}_n\z\to \spc{A}_\infty$ such that all $\spc{A}_n$ and $\spc{A}_\infty$ are compact finite-dimensional $\Alex0$ and for any small $\eps>0$  there is no continuous $\eps$-lifting map $\spc{A}_\infty\to \spc{A}_n$ for any large $n$.
\end{thm}

\section{Nerves}

Let $\{\Omega_1,\dots,\Omega_k\}$ be a finite open cover of a compact metric space $\spc{X}$.
Consider an abstract simplicial complex $\spc{N}$, with one vertex $v_i$ for each set $\Omega_i$ such that a simplex with vertices $v_{i_1},\dots, v_{i_m}$ is included in $\spc{N}$ if 
the intersection $\Omega_{i_1}\cap\dots\cap \Omega_{i_m}$ is nonempty.
\begin{figure}[ht!]
\vskip-0mm
\centering
\includegraphics{mppics/pic-1402}
\end{figure}
The obtained simplicial complex $\spc{N}$ is called the \index{nerve}\emph{nerve} of the covering $\{\Omega_i\}$.
Evidently, $\spc{N}$ is a finite simplicial complex ---
it is a subcomplex of a simplex with the vertices $\{v_1,\dots,v_k\}$.
Recall that $\Star_{v_i}$ denotes the union of all simplices in $\spc{N}$ that share the vertex $v_i$.

The next statement follows from \cite[4G.3]{hatcher}.


\begin{thm}{Nerve theorem}\label{thm:nerve}
Let $\{\Omega_1,\dots,\Omega_k\}$ be an open cover of a compact metric space $\spc{X}$
and let $\spc{N}$ be the corresponding nerve with vertices $\{v_1,\dots,v_k\}$.
Suppose that every nonempty finite intersection $\Omega_{\alpha_1}\cap\z\dots\cap\Omega_{\alpha_m}$ is contractible.
Then $\spc{X}$ is homotopy equivalent to the nerve $\spc{N}$ of the cover.

Moreover, homotopy equivalences  $a\:\spc{X}\to \spc{N}$ and $b\:\spc{N}\to\spc{X}$ can be chosen so that
if $x\in \Omega_i$, then $a(x)\in \Star_{v_i}$,
and if $y\in\spc{N}$ lies in the simplex with vertices $v_{i_1},\dots, v_{i_m}$, then $b(y)\in \Omega_{i_1}\cup\dots\cup \Omega_{i_m}$.
\end{thm}

\section{Homotopy stability}

\begin{thm}{Theorem}\label{thm:h-stability}
Let $\spc{A}_1,\spc{A}_2,\dots$, and $\spc{A}_\infty$ be compact $m$-dimensional $\Alex\kappa$ spaces, and $m<\infty$.
Suppose $\spc{A}_n\z\to \spc{A}_\infty$ as $n\to \infty$ in the sense of Gromov--Hausdorff.
Then $\spc{A}_\infty$ is homotopy equivalent to $\spc{A}_n$ for all large $n$.

Moreover, given $\eps>0$ there are maps $h_n\:\spc{A}_\infty\to \spc{A}_n$ that are homotopy equivalences and $\eps$-liftings for all large $n$.
\end{thm}

Applying this theorem with Gromov's selection theorem (\ref{thm:gromov-compactness}) and Exercise \ref{ex:diam-compact:GH}, we get the following.


\begin{thm}{Theorem}\label{thm:h-finiteness}
Given $\kappa\in \RR, D>0, v_0>0$ and a positive integer $m$,
there are only finitely many homotopy types of $m$-dimensional $\Alex\kappa$ spaces with diameter $\le D$, and volume $\ge v_0$.
\end{thm}

\parit{Proof of \ref{thm:h-finiteness} modulo \ref{thm:h-stability}.}
Assume the contrary, then we can choose a sequence of $m$-dimensional $\Alex\kappa$ spaces $\spc{A}_1,\spc{A}_2,\dots$   that have different homotopy types and satisfy the assumptions of the theorem.
By Gromov's selection theorem, we can assume that $\spc{A}_n$ converges to some spaces $\spc{A}_\infty$ in the sense of Gromov--Hausdorff.

By \ref{ex:diam-compact:GH}, $\dim \spc{A}_\infty=m$.
It remains to apply \ref{thm:h-stability}.
\qeds

\parit{Proof of \ref{thm:h-stability}.}
Since $\spc{A}_\infty$ is compact, applying \ref{cor:convex-nbhd}, we can find a finite open cover of $\spc{A}_\infty$ by convex open sets $\Omega_1,\dots, \Omega_k$ such that 
for each $\Omega_i$ there is a strictly concave function $f_i$ that is defined in a neighborhood of the closure $\bar \Omega_i$ and such that $f_i|_{\partial \Omega_i}=0$.

Subtracting from functions $f_i$ some small value $\eps>0$,
we can ensure that $\bigcap_{i\in S}\Omega_{i}\ne \emptyset$ if and only if $\bigcap_{i\in S}\bar\Omega_{i}\ne \emptyset$.

Suppose that $W=\bigcap_{i\in S}\Omega_{i}\ne \emptyset$.
Then $W$ is contractible.
Indeed, the function 
\[f_S\df\min_{i\in S} f_i\]
is strictly concave and it vanishes on the boundary of $W$.
The $f_S$-gradient flow $(t,x)\mapsto \GF_{f_S}^t(x)$ defines a homotopy
$[0,\infty)\times W\z\to W$.
By the first distance estimate (\ref{thm:dist-est}), $\GF_{f_S}^t(x)$ converges to the (necessarily unique) maximum point of $f_S$.
Therefore, in the obtained homotopy we can reparametrize $[0,\infty)$ by $[0,1)$ and extend it continuously to $[0,1]$;
thus we get that $W$ is contractible.
In other words, the cover $\{\Omega_1,\dots, \Omega_k\}$ meets the assumptions of the nerve theorem (\ref{thm:nerve}).

The functions $f_i$ and sets $\Omega_i$ can be lifted to $\spc{A}_n$ while keeping their properties for all large $n$.
More precisely, there are liftings $f_{i,n}$ of all $f_i$ to $\spc{A}_n$ which are strictly concave for all large $n$ and such that $\bar\Omega_{i,n}=\set{x\in \spc{A}_n}{f_{i,n}(x)\ge 0}$ is a compact convex set and $\Omega_{i,n}\z=\set{x\in \spc{A}_n}{f_{i,n}(x)> 0}$ is an open convex set for each $i$.

Notice that $\{\Omega_{1,n},\dots,\Omega_{k,n}\}$ is an open cover of $\spc{A}_n$ for all large~$n$.
Indeed suppose we have $p_n\in \spc{A}_n\setminus(\Omega_{1,n}\cup\dots\cup\Omega_{k,n})$ for arbitrarily large $n$.
Since $\spc{A}_\infty$ is compact, there is a limit point $p_\infty\in \spc{A}_\infty$ for a subsequence of $p_n$.
But $p_\infty\in\Omega_i$ for some $i$ and therefore $p_n\in \Omega_{i,n}$ for arbitrarily large $n$ --- a contradiction.

In a similar fashion, we can show that if $n$ is large, then any collection $\{\Omega_{i,n}\}_{i\in S}$ has a common point in $\spc{A}_n$ 
if and only if $\{\Omega_{i}\}_{i\in S}$ has a common point in $\spc{A}_\infty$.
Here we have to use that $\bigcap_{i\in S}\Omega_{i}\ne \emptyset$ if and only if $\bigcap_{i\in S}\bar\Omega_{i}\ne \emptyset$.

It follows that for any large $n$ the covers 
\begin{itemize}
\item $\{\Omega_{1},\dots,\Omega_{k}\}$ of $\spc{A}_\infty$ and 
\item $\{\Omega_{1,n},\dots,\Omega_{k,n}\}$ of $\spc{A}_n$.
\end{itemize}
have the same nerve.
By the nerve theorem (\ref{thm:nerve}), $\spc{A}_n$ and $\spc{A}_\infty$ are homotopy equivalent for all large $n$ --- a contradiction.
\qeds

Note that the proof above implies the following.

\begin{thm}{Theorem}\label{thm:finite-dim-hom-simplicial}
Any compact finite-dimensional Alexandrov space is homotopy equivalent to a finite simplicial complex.
\end{thm}

\section{Remarks}

Originally, Gromov's selection theorem was proved for Riemannian manifolds with a lower bound on Ricci curvature \cite{gromov1981}.
It motivates the study of limits of manifolds with lower Ricci curvature bounds and their synthetic generalizations, the so-called $\mathrm{CD}(K,m)$ spaces; $\mathrm{CD}$ stands for curvature-dimension condition.
This theory has significant applications in Alexandrov geometry;
in particular, it provides a version of the Liouville theorem about phase-space volume of geodesic flow in Alexandrov space \cite{brue-mondino-semola}.

The construction of a strictly concave function (\ref{thm:strictly-concave}) is due to Grigory Perelman \cite{perelman1993,perelman-petrunin}.
More exact vesions of this construction is given in the survey of the third author \cite[7.2]{petrunin:survey}.
Let us mention the following closely related result of Artem Nepechiy \cite{nepechiy2019}:
\textit{given a point $p$ in Alexandrov space there is a $2$-convex function that approximates $\distfun_p^2$ up to second order at $p$}.

The fact that the controlled concavity survives after lifting (see Section \ref{sec:Liftings}) was observed and used by the first author \cite[4.2]{kapovitch2002}.
Functions with controlled concavity can be lifted to a collapsing sequence, in this case the resulting function is not concave but it has many proprties.
In particular, if the limit space has dimension $m$, then the Morse index of of the lifing is is at least $m$; this property implies a partial answer to an old question of \textit{which finite-dimensional Alexandrov spaces can be approximated by Riemannian manifolds with a fixed lower curvature bound}; see \cite{kapovitch2005}.
Furthermore, these liftings have the following peculior property (let us call it \emph{gap-convexity}) which, as far as we know, did not find its application:
\textit{If $f$ is controllably concave and $f_n$ are its lifing to a collapsing sequecce of Alexandrov spaces with uniform lower curvature bound, then for any $\eps>0$, and all large $n$, the restriction of $f_n$ to any geodesic of length at least $\eps$ is concave.}
(Recall that our geodesic are always length-minimizing.)


The homotopy-type finiteness theorem (\ref{thm:h-finiteness}) illustrates the main source of applications of Alexandrov spaces: to prove a statement about $m$-dimensional manifolds with lower sectional curvature bound we follow the following steps:
\begin{itemize}
\item Arguing by contradiction, we assume existence of a sequence of manifolds that violates our assumption.
\item Use Gromov's selection theorem to choose a converging subsequence.
\item Use curvature survival to conclude that the limit space is Alexandrov.
\item Try to arrive at a contradiction using Alexandrov geometry.
\end{itemize}

In principle, the same strategy might work for a sequence of Riemannian manifolds with dimension approaching infinity, but no applications of this kind were found.
The following question gives an example where such strategy might be successful in principle.

\begin{thm}{Question}
Is it true that no simply connected closed manifold (of arbitrarily large dimension) admits a Riemannian metric with sectional curvature and diameter bounded by a fixed positive sufficiently small value?
\end{thm}

If the dimension is bounded, then Gromov's theorem \cite{gromov1978} implies that such a manifold can be covered by a compact nil-manifold; in particular, the manifold cannot be simply connected.
However, if  the dimension is unbounded, then Riemannian manifolds with these conditions may not be covered by compact nil-manifolds; such examples were found by Galina Guzhvina \cite{guzhvina2008}.

Let us finish with a list of results that can be proved by applying Gromov's selection theorem
in the same fashion as in the proof of homotopy-type finiteness theorem (\ref{thm:h-finiteness}).

\begin{thm}{Betti-number theorem}\label{thm:betti}
For any   $\kappa\in \RR, D>0$ and a positive integer $m$,
there is a constant $\Const=\Const(m,D,\kappa)$ such that 
\[\beta_0(M)+\dots+\beta_m(M)\le \Const\]
for any closed $m$-dimensional Riemannian manifold $M$ with sectional curvature $\ge \kappa$ and diameter $\le D$.
Here $\beta_i(M)$ denotes $i^\text{th}$ Betti number of $M$.
\end{thm}

Gromov's original proof \cite{gromov-1981} of the Betti-number theorem did not use Alexandrov geometry directly.
However, it is easy to produce a proof following the steps above (no one bothered to write it so far).
The following result was proved by the third author \cite{petrunin2008}, and it uses the same technique.

\begin{thm}{Scalar curvature bound}
Given $\kappa\in \RR, D>0$ and a positive integer $m$, there is a constant $\Const=\Const(m,D,\kappa)$ such that
\[\int_M\Sc\le \Const\]
for any closed $m$-dimensional Riemannian manifold $M$ with sectional curvature $\ge \kappa$ and diameter $\le D$.
Here $\Sc$ denotes the scalar curvature.
\end{thm}

The following theorem is a more exact version of \ref{thm:h-stability}.
Its close relative (\ref{thm:spherical-nbhd}) will play an important role in the following lecture.

\begin{thm}{Stability theorem}\label{thm:stability}
Let $\spc{A}_1,\spc{A}_2,\dots$, and $\spc{A}_\infty$ be compact $m$-dimensional $\Alex\kappa$ spaces, and $m<\infty$.
Suppose $\spc{A}_n\to \spc{A}_\infty$ as $n\to \infty$ in the sense of Gromov--Hausdorff.
Then $\spc{A}_\infty$ is homeomorphic to $\spc{A}_n$ for all large $n$.

Moreover, given $\eps>0$ there are maps $h_n\:\spc{A}_\infty\to \spc{A}_n$ that are homeomorphisms and $\eps$-liftings for all large $n$.
\end{thm}

This theorem was proved by Grigory Perelman \cite{perelman1991}.
The proof was rewritten with more details by the first author \cite{kapovitch};
for more exact statements, see \cite{kapovitch2002,kapovitch2005}.
In private conversations, Perelman claimed that the homeomorphisms in the theorem can be assumed to be bi-Lipschitz with constants that depend on $\spc{A}_\infty$.
Since no proof has been written, this statement should be considered as a conjecture;
partial results in this direction were obtained by Mohammad Alattar \cite{alattar}.

Theorem \ref{thm:h-finiteness} was originally proved by Karsten Grove and Peter Petersen \cite{grove-petersen1988}.
Perelman's stability theorem (\ref{thm:stability}) implies the following stronger statement.

\begin{thm}{Homeomorphism-type finiteness}
For any real numbers $\kappa$, $D\z>0$, $v>0$ and a positive integer $m$, there are only finitely many homeomorphism types of closed $m$-dimensional manifolds that admit a Riemannian metric with sectional curvature $\ge \kappa$, volume $\ge v$, and diameter $\le D$.
\end{thm}

This statement can be improved to diffeomorphism-type finiteness in all dimensions $m\ne 4$.
Indeed, for $m\ne4$ a closed topological $m$-manifold admits only finitely many smooth structures; see \cite{kirby-siebenmann} and \cite{moise,thurston} for cases $m\ge 5$ and $m\le 3$, respectively.


