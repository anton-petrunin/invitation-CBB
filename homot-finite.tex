\chapter{Homotopy finiteness theorem}\label{chap:stability}

\section{Controlled concavity}

While Alexandrov spaces have plenty of semiconcave functions (for instance, square of distance function), 
it is not at all easy to construct a strictly concave one. 

\begin{thm}{Theorem}
\label{thm:strictly-concave}
Let $\spc{L}$ be a complete finite-dimensional Alexandrov  space.
Then for any point $p\in \spc{L}$, there is  a strictly concave function $f$ defined in an
open neighborhood of $p$.

Moreover, given $0\ne v\in T_p$, the differential, $\dd_p f$, can be chosen
arbitrarily close to $x\mapsto -\<v,x\>$.
\end{thm}

\parit{Proof.} 
Fix small $r>0$ and large $c$;
consider the real-to-real function 
$$\phi_{r,c}(x)=(x-r)- c{(x-r)^2}/r,$$
so we have 
$\phi_{r,c}(r)=0$,
$\phi_{r,c}'(r)=1$,
and $\phi_{r,c}''(r)=- {2c}/{r}$. 

Let $\gamma$ be a unit-speed geodesic, fix a point $q$ and let 
$$\alpha(t)=\mangle(\gamma^+(t),\dir{\gamma(t)}{q}).$$
If $\dist q{\gamma(t)}{}$ is sufficiently close to
$r$, then direct calculations show that
$$(\phi_{r,c}\circ\distfun_q\circ\gamma)''(t)
\le 
\frac{3-c\cdot \cos^2[\alpha(t)]}{r}.$$
(Since $c$ is large, this inequality implies that $\phi_{r,c}\circ\distfun_q\circ\gamma$ is strictly concave at $t$ unless $\gamma$ runs nearly perpendicular to the direction to $q$.) 

Now, assume $\{q_1,\dots, q_N\}$ is a finite set of points such that $\dist p{q_i}{}=r$ for any $i$. 
For a geodesic $\gamma$, set $\alpha_i(t)=\mangle(\gamma^+(t),\dir {\gamma(t)}{q_i})$. 
Assume we have a collection $\{q_i\}$ such
that for any geodesic $\gamma$ in $\oBall(p,\eps)$
we have $\max_i\{|\alpha_i(t)-\tfrac\pi2|\}\ge\eps>0$. 
We can assume that $c>3N/\cos^2\eps$;
then the inequality above implies that the function
$$f=\sum_i \phi_{r,c}\circ\distfun_{q_i}$$
is strictly concave in $\oBall(p,\eps')$ for some positive $\eps'<\eps$.

To construct the needed collection $\{q_i\}$, note that for small $r>0$, one can
choose $N\ge \Const/\delta^{(m-1)}$ points $\{q_i\}$ such that $\dist{p}{q_i}{}=r$
and $\angk p{q_j}{q_i}>\delta$ (here $\Const=\Const(\Sigma_p)>0$).
%???why!!!
On the other hand, suppose $\gamma$ runs from $x$ to $y$.
If $|\alpha_i(t)- \tfrac\pi2|<\eps\ll\delta$, then applying the ($n$+1)-point comparison to $\gamma(t)$, $x$, $y$ and all $\{q_i\}$ we get that
$N\le \Const(m)/\delta^{(m-2)}$. 
Therefore, for small $\delta>0$ and yet smaller $\eps>0$, the set $\{q_i\}$ forms the needed collection.

If $r$ is small, then points $q_i$ can be chosen so that all directions
$\dir p {q_i}$ will be $\eps$-close to a given direction $\xi$ and
therefore the second property follows.
\qeds

Note that in \ref{thm:strictly-concave} the function $f$ can be chosen to have maximum value $0$ at $p$,
$f(p)=0$ and with $\dd_p f(x)\approx-|x|$.
It can be constructed by taking the minimum of the functions in the theorem.
Then the set $\Omega=\set{x\in\spc{L}}{f(x)\ge -\eps}$ forms an open convex neighborhood of $p$ for any small $\eps>0$, so we get the following.


\begin{thm}{Corollary}\label{cor:convex-nbhd}
Any point $p$ of a finite-dimensional Alexandrov space admits an arbitrary small convex neighborhood $\Omega$ and a strictly concave function $f$ defined in a neighborhood of the closure $\bar\Omega$ such that $p$ is the maximum point of $f$
and $f|_{\partial\Omega}=0$.
\end{thm}

\section{Liftings}

Suppose that the Gromov--Haudorff distance $\dist{\spc{L}}{\spc{L}'}{\GH}$ is sufficienlty small, so we may think that both spaces $\spc{L}$ and $\spc{L}'$ lie on small Haudorff distance in an ambient metric space $\spc{W}$.
In particular, we may choose small $\eps>0$, so that for any point $p\in \spc{L}$, we can find a point $p'\in \spc{L}'$ such that $\dist{p}{p'}{\spc{W}}<\eps$;
the point $p'$ will be called an \index{lifting}\emph{lifting} (or \emph{$\eps$-lifting}) of $p$ in $\spc{L}'$.
We may choose a lifting $p'\in\spc{L}'$ for every point $p\in\spc{L}$, 
in this case the map $p\mapsto p'$ is called a {}\emph{($\eps$-)lifting map}.

Note that the lifting is not uniquely defined.
The lifting maps is not assumed to be continuous.
When we talk about liftings, we assume that $\eps>0$, the inclusions $\spc{L},\spc{L}'\hookrightarrow\spc{W}$,
as well as $\spc{W}$ are chosen.


Choose a compact $m$-dimensional Alexandrov space $\spc{L}$.
Suppose $\spc{L}'$ is another compact $m$-dimensional Alexandrov space such that $\dist{\spc{L}}{\spc{L}'}{\GH}$ is sufficiently small --- smaller than some $\eps=\eps(\spc{L})>0$.
Then the construction in $\spc{L}$ from the previous section  
can be repeated in $\spc{L}'$ for the liftings of all points and the same function $\phi$.
It produces a strictly concave function defined in a controlled neighborhood of the lifting $p'$ of $p$.

The result of this and related constructions will be called \index{lifting}\emph{liftings},
say we can talk about a lifting from $\spc{L}$ to $\spc{L}'$ of a function provided by \ref{thm:strictly-concave} (if the Gromov--Hausdorff distance $\dist{\spc{L}}{\spc{L}'}{\GH}$ is small, then these liftings are stricly concave)
and a lifting of a convex neighborhood from \ref{cor:convex-nbhd}.
Here one cannot use \ref{thm:strictly-concave} and \ref{cor:convex-nbhd} as black boxes --- one has to understand the construction, but it is straightforward.

\section{Nerves}

Let $\{\Omega_1,\dots,\Omega_k\}$ be a finite open cover of a compact metric space $\spc{X}$.
Consider an abstract simplicial complex $\spc{N}$, with one vertex $v_i$ for each set $\Omega_i$ such that a simplex with vertices $v_{i_1},\dots, v_{i_m}$ is included in $\spc{N}$ if 
the intersection $\Omega_{i_1}\cap\dots\cap \Omega_{i_m}$ is nonempty.
\begin{figure}[ht!]
\vskip-0mm
\centering
\includegraphics{mppics/pic-1402}
\end{figure}
The obtained simplicial complex $\spc{N}$ is called the \index{nerve}\emph{nerve} of the covering $\{\Omega_i\}$.
Evidently $\spc{N}$ is a finite simplicial complex ---
it is a subcomplex of a simplex with the vertices $\{v_1,\dots,v_k\}$.
Recall that $\Star_{v_i}$ denotes the union of all simplexes in $\spc{N}$ that shares vertex $v_i$.

The next statement follows from \cite[4G.3]{hatcher}.


\begin{thm}{Nerve theorem}
Let $\{\Omega_1,\dots,\Omega_k\}$ be an open cover of a compact metric space $\spc{X}$
and let $\spc{N}$ be the corresponging nerve with vertices $\{v_1,\dots,v_k\}$.
Suppose that every nonempty finite intersection $\Omega_{\alpha_1}\cap\z\dots\cap\Omega_{\alpha_k}$ is contractible.
Then $\spc{X}$ is homotopy equivalent to the nerve $\spc{N}$ of the cover.

Moreover homotopy equivalences  $a\:\spc{X}\to \spc{N}$ and $b\:\spc{N}\to\spc{X}$ can be chosen so that 
if $x\in \Omega_i$, then $a(x)\in \Star_{v_i}$,
and if $y\in\spc{N}$ lies in the simplex with vertices $v_{i_1},\dots, v_{i_m}$, then $b(y)\in \Omega_{i_1}\cup\dots\cup \Omega_{i_m}$.
\end{thm}

%???Вить, посмотри на это утверждение --- оно мне не сильно нравится.


\section{Homotopy stability}

\begin{thm}{Theorem}\label{thm:h-stability}
Let $\spc{L}_1,\spc{L}_2,\dots,\spc{L}_\infty$ be a sequence of $m$-dimensional $\Alex\kappa$ spaces, and $m<\infty$.
Suppose $\spc{L}_n\z\GHto \spc{L}_\infty$ as $n\to \infty$.
Then there is a $\spc{L}_\infty$ is homotopically equivalent to $\spc{L}_n$ for all large $n$.

Moreover, given $\eps>0$ there are maps $h_n\:\spc{L}_\infty\to \spc{L}_n$ that are homotopy equivalences and $\eps$-liftings for all large $n$.
\end{thm}

Applying this theorem with the Gromov's selection theorem (\ref{thm:gromov-compactness}) and Exercise \ref{ex:pack-vol}, we get the following.


\begin{thm}{Theorem}\label{thm:h-finiteness}
There are only finitely many homotopy types of $m$-dimensional $\Alex\kappa$ spaces with diameter $\le D$, and volume $\ge v_0$;
here we assume that an integer $m$, and $v_0,D\in\RR$ such that $v_0>0$ are given.
\end{thm}

\parit{Proof of \ref{thm:h-finiteness} modulo \ref{thm:h-stability}.}
Assume the contrary, then we can choose a sequence of spaces $\spc{L}_1,\spc{L}_2,\dots$ that have different homotopy types and satisfy the assumptions of the theorem.
By Gromov's compactness theorem, we can assume that $\spc{L}_n$ converges to say $\spc{L}_\infty$ in the sense of Gromov--Hausdorff.

By \ref{ex:pack-vol}, $\LinDim \spc{L}_\infty=m$.
It remains to apply \ref{thm:h-stability}.
\qeds

\parit{Proof of \ref{thm:h-stability}.}
Since $\spc{L}_\infty$ is compact, applying \ref{cor:convex-nbhd}, we can find a finite open cover of $\spc{L}_\infty$ by convex open sets $\Omega_1,\dots, \Omega_k$ such that 
for each $\Omega_i$ there is a strictly concave function $f_i$ that is defined in a neighborhood of $\bar \Omega_i$ and such that $f_i|_{\partial \Omega_i}=0$.

Subtracting from functions $f_i$ some small value $\eps>0$,
we can ensure that $\bigcap_{i\in S}\Omega_{i}\ne \emptyset$ if and only if $\bigcap_{i\in S}\bar\Omega_{i}\ne \emptyset$.

Suppose that $W=\bigcap_{i\in S}\Omega_{i}\ne \emptyset$.
Then $W$ is contractible.
Indeed the function 
\[f_S\df\min_{i\in S} f_i\]
is strictly concave and it vanished on the boundary of $W$.
The $f_S$-gradient flow $(t,x)\mapsto \GF_{f_S}^t(x)$ defines a homotopy
$[0,\infty)\times W\to W$.
Note that $\GF_{f_S}^t(x)$ converges to the (necessarily unique) maximum point of $f_S$ as $t\to\infty$.
Therefore, in the obtained homotoly we can parametrize $[0,\infty)$ by $[0,1)$ and extend the homotopy by continiously to $[0,1]$;
this way we get that $W$ is contractible.
In other words, the cover $\{\Omega_1,\dots, \Omega_k\}$ meets the assumptions of the nerve theorem.
Therefore $\spc{L}_\infty$ is homotopy equivalent to the nerve $\spc{N}$ of the cover.

The functions $f_i$ and sets $\Omega_i$ can be lifted to $\spc{L}_n$ keeping its properties for all large $n$. 
More precisely, there are liftings $f_{i,n}$ of all $f_i$ to $\spc{L}_n$ which are strictly concave for all large $n$ and such that $\bar\Omega_{i,n}=\set{x\in \spc{L}_n}{f_{i,n}(x)\ge 0}$ is a compact convex set and $\Omega_{i,n}\z=\set{x\in \spc{L}_n}{f_{i,n}(x)> 0}$ is an open convex set for each $i$.

Notice that $\{\Omega_{1,n},\dots,\Omega_{k,n}\}$ is an open cover of $\spc{L}_n$ for all large $n$.
Indeed suppose we have $p_n\in \spc{L}_n\setminus(\Omega_{1,n}\cup\dots\cup\Omega_{k,n})$ for arbitrary large $n$.
Since $\spc{L}_\infty$ is compact, there is a limit point $p_\infty\in \spc{L}_\infty$ for a subsequnce of $p_n$.
But $p_\infty\in\Omega_i$ for some $i$ and therefore $p_n\in \Omega_{i,n}$ for arbitrary large $n$ --- a contradiction.

In a similar fashion, we can show that if $n$ is large, then any collection $\{\Omega_{i,n}\}_{i\in S}$ has a common point in $\spc{L}_n$ 
if and only if $\{\Omega_{i}\}_{i\in S}$ has a common point in $\spc{L}_\infty$.
Here we have to use that $\bigcap_{i\in S}\Omega_{i}\ne \emptyset$ if and only if $\bigcap_{i\in S}\bar\Omega_{i}\ne \emptyset$.

It follows that for any large $n$ the following two covers the same nerve
\begin{itemize}
\item $\{\Omega_{1},\dots,\Omega_{k}\}$ of $\spc{L}_\infty$ and 
\item $\{\Omega_{1,n},\dots,\Omega_{k,n}\}$ of $\spc{L}_n$.
\end{itemize}
Therefore, $\spc{L}_n$ is homotopy equivalent to $\spc{N}$ for all large $n$ --- a contradiction.
\qeds


\section{Comments}

The construction of strictly concave function is due to Grigori Perelman \cite{perelman1993,perelman-petrunin}.

Let us list some results that can be proved by applying Gromov's selection theorem
in the same fashion as in the proof of homotopy-type finiteness theorem (\ref{thm:h-finiteness}).
The following theorem can be proved using this technique, altho Gromov's original proof \cite{gromov-1981} did not use Alexandrov geometry directly.

\begin{thm}{Betti-number theorem}
There is a constant $\Const=\Const(m,D,\kappa)$ such that 
\[\beta_0(M)+\beta_1(M)+\dots+\beta_m(M)\le \Const\]
for any closed $m$-dimensional Riemannian manifold $M$ with sectional curvature $\ge \kappa$ and diameter $\le D$.
Here $\beta_i(M)$ denotes $i^\text{th}$ Betti number of $M$.
\end{thm}

The following result of the second author \cite{petrunin2008}, and it uses the same technique.

\begin{thm}{Scalar curvature bound}
There is a constant $\Const=\Const(m,D,\kappa)$ such that 
\[\int_M\Sc\le \Const\]
for any closed $m$-dimensional Riemannian manifold $M$ with sectional curvature $\ge \kappa$ and diameter $\le D$.
Here $\Sc$ denotes the scalar curvature.
\end{thm}

The following theorem is a more exact version of \ref{thm:h-stability}.
It will play an important role in the following lecture.

\begin{thm}{Stability theorem}\label{thm:stability}
Let $\spc{L}_1,\spc{L}_2,\dots,\spc{L}_\infty$ be a sequence of $m$-dimensional $\Alex\kappa$, and $m<\infty$.
Suppose $\spc{L}_n\GHto \spc{L}_\infty$ as $n\to \infty$.
Then there is a $\spc{L}_\infty$ is homotopically equivalent to $\spc{L}_n$ for all large $n$.

Moreover, given $\eps>0$ there are maps $h_n\:\spc{L}_\infty\to \spc{L}_n$ that are homeomorphisms and $\eps$-liftings for all large $n$.
\end{thm}

This theorem was proved by Grigori Perelman \cite{perelman1991};
the proof was rewritten with more details by the first author \cite{kapovitch}.
Around the same time, he made informal annoncemnt that the homeomorphisms in the theorem can be made bi-Lipschitz with constants that depend on $\spc{L}_\infty$;
but the proof was not written and it save to consider it as a conjecture.

The last statement in the theorem implies the following finiteness result.

\begin{thm}{Homeorphism-type finiteness}
There are only finitely many homeomorphism types of closed $m$-dimensional manifolds that admit a Riemannian metric with sectional curvature $\ge \kappa$, and diameter $\le D$.
\end{thm}

In fact, this theorem implies diffeomorphism-type finiteness in all dimensions except 4.


