\documentclass[twoside]{book}

%\newcommand{\spell}[2]{#1} %spell
\newcommand{\spell}[2]{#2} %notes


\def\thetitle{A journey into Alexandrov geometry:\\
curvature bounded below}
\def\theauthors{Vitali Kapovitch and Anton Petrunin}

\usepackage{lectures}
\usepackage[colorlinks=true,
citecolor=black,
linkcolor=black,
anchorcolor=black,
filecolor=black,
menucolor=black,
urlcolor=black,
pdftitle={\thetitle},
pdfsubject={Geometry},
pdfauthor={\theauthors}
]{hyperref}
\makeindex

%\usepackage[x-1a]{pdfx}

%\overfullrule=100mm
\def\red{\color{red}}
\begin{document}

\spell{\pagestyle{empty}\renewcommand\includegraphics[2][{}]{}\def\emph{\textit}\renewcommand\footnote[1]{\ (#1)}\renewcommand\z{}\renewcommand\section[1]{SECTION. {#1} SECTION.}}{}

\frontmatter
\title{\thetitle}
\author{\theauthors\\
with an appendix by\\
Nina Lebedeva and Anton Petrunin}
\date{}
\maketitle
\thispagestyle{empty}

\mainmatter
\newpage
\tableofcontents

\chapter*{Preface}

As in our previous invitation \cite{alexander-kapovitch-petrunin-2019},
we try to demonstrate the beauty and power of Alexandrov geometry by reaching interesting applications and theorems with a minimum of preparation.
This time we do spaces with curvature bounded below in the sense of Alexandrov.

This subject is more technical, this time we jumped over proofs of couple of technical results,
namely existence part in generalized Picard's theorem (\ref{thm:glob-exist-grad-curv})
and Perelman's theorem about conic neighborhoods (\ref{thm:spherical-nbhd}).
The rest of our presentation is nearly rigorous.

\medskip 

In Lecture~\ref{chap:prelim}, we discuss necessary preliminaries and fix notations.

Lecture~\ref{chap:defs} introduces the main object of our study --- spaces with curvature bounded below in the sense of Alexandrov.

In Lecture~\ref{chap:globalization} we formulate and prove the globalization theorem --- local Alexandrov condition implies global.
To simplify the presentation we consider only compact case, but this case is leading.

In Lecture~\ref{chap:derivative} we do beginning of calculus --- tangent space and space of directions, differential, and gradient.

Lecture~\ref{chap:GF} introduces gradient flow --- this is the main technical tool in the theory.

Lecture~\ref{chap:splitting} proves the line splitting theorem.
It provides the first application of gradient flow.

In Lecture~\ref{chap:dim} we introduce and discuss dimension of Alexandrov spaces,
introduce volume,
and prove the Bishop--Gromov inequality.

Lecture~\ref{chap:lim} shows that lower curvature bound survives in the Gromov--Hausdorff limit and proves Gromov's selection theorem.
Further we do Perelman's construction of strictly concave functions and apply it with Gromov's selection theorem to prove the homotopy finiteness theorem.
This proof illustrates the main source of applications of Alexandrov geometry.

In Lecture~\ref{chap:bry} we introduce boundary of finite-dimensioanal Alexandrov space and prove the doubling theorem.

Lecture~\ref{chap:L/G} we show that quotient Alexandrov space by isometric group action is an Alexandrov space and give several applications of this statement.
These proofs illustrate another source of applications of Alexandrov geometry.

Lecture~\ref{chap:convex-body} brings us back to the original object of study of Alexandrov.
We show that surface of a convex body in Euclidean space is an Alexandrov space.
This is historically the first serious application of Alexandrov geometry.

Finally, Appendix~\ref{chap:embedding} sketches Alexandrov embedding theorem of convex polyhedra.
Historically, this theorem is the first remarkable result in Alexandrov geometry that dates back to 1941.
The proof is very well written by Alexandrov, but we decided to include its sketch here due to its beauty and importance.
This appendix was written by Nina Lebedeva and the second author for a book about .

Let us give a list of available texts on Alexandrov spaces with curvature bounded below: 
\begin{itemize}
\item The first introduction to Alexandrov geometry is given in the original paper of Yuriy Burago, Michael Gromov, and Grigory Perelman \cite{burago-gromov-perelman} 
and its extension \cite{perelman1991} written by Perelman.
\item A brief and reader-friendly introduction was written by Katsuhiro Shiohama \cite[Sections 1--8]{shiohama}.
\item Another reader-friendly introduction, written by Dmiti Burago, Yuriy
Burago, and Sergei Ivanov \cite[Chapter 10]{burago-burago-ivanov}.
\item Survey by Conrad Plaut \cite{plaut:survey}.
\item Survey by the second author \cite{petrunin:survey}.
\end{itemize}

\parbf{Acknowledgments.}
Our notes were shaped in a number of lectures given by the authors
at different occasions in Penn State, including the MASS program,
at the Summer School ``Algebra and Geometry'' in Yaroslavl,
at SPbSU,
and University of Toronto.
We want to thank these institution for hospitality and support.

We were partially supported by the following grants:
Vitali Kapovitch ---   NSERC Discovery grants;
Anton Petrunin --- 
NSF grant DMS-2005279. %??? check!!!




%%!TEX root = the-prelim.tex

\chapter{Preliminaries}\label{chap:prelim}

\section{Prerequisites}

We assume that the reader is familiar with the following topics in metric geometry:
\begin{itemize}
\item Compactness and proper metric spaces;
recall that a metric space is \index{proper space}\emph{proper} if all its closed balls (with finite radius) are compact.
\item Complete metric spaces and completion.
\item Curves, semicontinuity of length and rectifiability.
\item Hausdorff and Gromov--Hausdorff convergence.
These are discussed briefly in \ref{sec:Hausdorff convergence}--\ref{sec:Gromov--Hausdorff-metric}.
The definitions are there, but it would be hard to follow without prior experience.
\end{itemize}
These topics are treated in \cite{burago-burago-ivanov} and \cite{petrunin2023pure}.
Occasionally, we use the Baire category theorem and Rademacher's theorem, but these could be used as black boxes.

We use some topology. 
Most of the time, any introductory text in algebraic topology should be sufficient.
For some examples, we use more advanced results, but these could also be used as black boxes.

Since most of the applications come from Riemannian geometry, it is better to be familiar with the Toponogov comparison theorem and related topics.
The classical book by Jeff Cheeger and David Ebin \cite{cheeger-ebin} contains more than you will need.

\section{Notations}

The distance between two points $x$ and $y$ in a metric space $\spc{X}$ will be denoted by \index{$\dist{x}{y}{}=\dist{x}{y}{\spc{X}}$ (distance)}$\dist{x}{y}{}$ or $\dist{x}{y}{\spc{X}}$.\label{page:|x-y|X}
The latter notation is used if we need to emphasize 
that the distance is taken in the space~${\spc{X}}$.

Given radius $r\in[0,\infty]$ and center $x\in \spc{X}$, the sets
\begin{align*}
\oBall(x,r)&=\set{y\in \spc{X}}{\dist{x}{y}{}<r},
\\
\cBall[x,r]&=\set{y\in \spc{X}}{\dist{x}{y}{}\le r}
\end{align*}
are called, respectively, the \index{open ball}\emph{open} and  the \index{closed ball}\emph{closed  balls}.
The notations $\oBall(x,r)_{\spc{X}}$ and $\cBall[x,r]_{\spc{X}}$
might be used if we need to emphasize that these balls are taken in the metric space $\spc{X}$.

We will denote by \index{$\SSS^n$, $\EE^n$, $\HH^n$, and $\MM^n(\kappa)$}$\SSS^n$, $\EE^n$, and $\HH^n$ the $n$-dimensional sphere (with angle metric), 
Euclidean space, and Lobachevsky space respectively.
More generally, $\MM^n(\kappa)$ will denote the \index{model!space}\emph{model $n$-space} of curvature $\kappa$;
that is,
\begin{itemize}
\item if $\kappa>0$, then $\MM^n(\kappa)$ is the $n$-sphere of radius $\tfrac{1}{\sqrt{\kappa}}$, so $\SSS^n=\MM^n(1)$
\item $\MM^n(0)=\EE^n$,
\item if $\kappa<0$, then $\MM^n(\kappa)$ is the Lobachevsky $n$-space $\HH^n$ rescaled by factor $\tfrac{1}{\sqrt{-\kappa}}$;
in particular $\MM^n(-1)=\HH^n$.
\end{itemize}

\section{Length spaces}\label{sec:length}

Let $\spc{X}$ be a metric space.
If for any $\eps>0$ and any pair of points $x,y\in\spc{X}$, there is a path $\alpha$ connecting $x$ to $y$ such that
\[\length\alpha< \dist{x}{y}{}+\eps,\]
then $\spc{X}$ is called a \index{length space}\emph{length space} and the metric on $\spc{X}$ is called a \index{length metric}\emph{length metric}.\label{page:length metric}

\begin{thm}{Exercise}\label{ex:compact+connceted}
Let $\spc{X}$ be a complete length space.
Show that for any compact subset $K\subset\spc{X}$
there is a compact path-connected subset $K'\subset\spc{X}$ that contains $K$.  
\end{thm}

\parbf{Induced length metric.}
Directly from the definition, it follows that if $\alpha\:[0,1]\to\spc{X}$ is a path from $x$ to $y$ 
(that is, $\alpha(0)=x$ and $\alpha(1)=y$), then 
\[\length\alpha\ge \dist{x}{y}{}.\]
Set 
\[\yetdist{x}{y}{}=\inf\{\,\length\alpha\,\}\]
where the greatest lower bound is taken for all paths from $x$ to~$y$.
It is straightforward to check that $(x,y)\mapsto \yetdist{x}{y}{}$ is an \emph{$\infty$-metric};
that is, $(x,y)\mapsto \yetdist{x}{y}{}$ is a metric in the extended positive reals $[0,\infty]$. 
The metric $\yetdist{*}{*}{}$ is called the \index{induced length metric}\emph{induced length metric}.

\begin{thm}{Exercise}\label{ex:compact=>complete}
Suppose $(\spc{X},\dist{*}{*}{})$ is a complete metric space.
Show that $(\spc{X},\yetdist{*}{*}{})$ is complete;
that is, any Cauchy sequence of points in $(\spc{X},\yetdist{*}{*}{})$ converges in $(\spc{X},\yetdist{*}{*}{})$.
\end{thm}

Let $A$ be a subset of a metric space $\spc{X}$.
Given two points $x,y\in A$,
consider the value
\[\dist{x}{y}{A}=\inf_{\alpha}\{\,\length\alpha\,\},\]
where the greatest lower bound is taken for all paths $\alpha$ from $x$ to $y$ in~$A$.
In other words, $\dist{*}{*}{A}$ denotes the induced length metric on the subspace $A$.
(The notation $\dist{*}{*}{A}$ conflicts with the previously defined notation for distance $\dist{x}{y}{\spc{X}}$ in a metric space $\spc{X}$.
However, most of the time we will work with ambient length spaces where the meaning will be unambiguous.)

\section{Geodesics}

Let $\spc{X}$ be a metric space 
and $\II$\index{$\II$ (real interval)} a real interval. 
A distance-preserving map $\gamma\:\II\to \spc{X}$ is called a \index{geodesic}\emph{geodesic}%
\footnote{Others call it differently: \textit{shortest path}, \textit{minimizing geodesic}.
Also, note that the meaning of the term \textit{geodesic} is different from what is used in Riemannian geometry, altho they are closely related.}; 
in other words, $\gamma\:\II\z\to \spc{X}$ is a geodesic if 
\[\dist{\gamma(s)}{\gamma(t)}{}=|s-t|\]
for any pair $s,t\in \II$.

If $\gamma\:[a,b]\to \spc{X}$ is a geodesic such that $p=\gamma(a)$, $q=\gamma(b)$, then we say that $\gamma$ is a geodesic from $p$ to $q$.
In this case, the image of $\gamma$ is denoted by $[p q]$\index{$[pq]$ (geodesic)}, and, with abuse of notations, we also call it a \index{geodesic}\emph{geodesic}.
We may write $[p q]_{\spc{X}}$ 
to emphasize that the geodesic $[p q]$ is in the space  ${\spc{X}}$.

In general, a geodesic from $p$ to $q$ need not exist and if it exists, it need not  be unique;
for example, any meridian is a geodesic between poles on the sphere.
However, once we write $[p q]$ we assume that we have chosen such a geodesic.

A \index{geodesic!path}\emph{geodesic path} is a geodesic with constant-speed parameterization by the unit interval $[0,1]$.

A metric space is called \index{geodesic!space}\emph{geodesic} if any pair of its points can be joined by a geodesic.

Evidently, any geodesic space is a length space.

\begin{thm}{Exercise}\label{ex:compact-length}
Show that any proper length space is geodesic.
\end{thm}

\section{Menger's lemma}

\begin{thm}{Lemma}\label{lem:mid>geod}
Let $\spc{X}$ be a complete metric space.
Assume that for any pair of points $x,y\in \spc{X}$, 
there is a midpoint~$z$.
Then $\spc{X}$ is a geodesic space.

\end{thm}

This lemma is due to Karl Menger \cite[Section 6]{menger}.

%???+PIC!!!

\parit{Proof.}
Choose $x,y\in \spc{X}$;
set $\gamma(0)=x$, and $\gamma(1)=y$.

\begin{figure}[ht!]
\vskip-0mm
\centering
\includegraphics{mppics/pic-104}
\end{figure}

Let $\gamma(\tfrac12)$ be a midpoint between $\gamma(0)$ and $\gamma(1)$.
Further, let $\gamma(\frac14)$ 
and $\gamma(\frac34)$ be midpoints between the pairs $(\gamma(0),\gamma(\tfrac12))$ 
and $(\gamma(\tfrac12),\gamma(1))$ respectively.
Applying the above procedure recursively,
on the $n$-th step we define $\gamma(\tfrac{k}{2^n})$,
for every odd integer $k$ such that $0<\tfrac k{2^n}<1$, 
as a midpoint of the already defined
$\gamma(\tfrac{k-1}{2^n})$ and $\gamma(\tfrac{k+1}{2^n})$.

This way we define $\gamma(t)$ for all dyadic rationals $t$ in $[0,1]$.
Moreover, $\gamma$ has Lipschitz constant $\dist{x}{y}{}$.
Since $\spc{X}$ is complete, the map $\gamma$ can be extended continuously to $[0,1]$.
Moreover,
\[
\length\gamma\le \dist{x}{y}{}.
\]
Therefore $\gamma$ is a geodesic path from $x$ to $y$.
\qedsf

\begin{thm}{Exercise}\label{ex:menger}
Let $\spc{X}$ be a complete metric space.
Assume that for any pair of points $x,y\in \spc{X}$, 
there is an \index{almost midpoint}\emph{almost midpoint};
that is, given $\eps>0$, there is a point $z$ such that 
\[\dist{x}{z}{}<\tfrac12\cdot\dist{x}{y}{}+\eps 
\quad\text{and}\quad
\dist{y}{z}{}<\tfrac12\cdot\dist{x}{y}{}+\eps.\]
Show that $\spc{X}$ is a length space.
\end{thm}


\section{Triangles and model tangles}

\parbf{Triangles.}
Given a triple of distinct points $p,q,r$ in a metric space $\spc{X}$, a choice of geodesics $([q r], [r p], [p q])$ will be called a \index{triangle}\emph{triangle}; we will use the short notation 
$\trig p q r=\trig p q r_{\spc{X}}=([q r], [r p], [p q])$\index{$\trig p q r=\trig p q r_{\spc{X}}$ (triangle)}.

Given a triple $p,q,r\in \spc{X}$ there may be no triangle 
$\trig p q r$ simply because one of the pairs of these points cannot be joined by a geodesic.
Also, many different triangles with these vertices may exist, any of which can be denoted by $\trig p q r$.
If we write $\trig p q r$, it means that we have chosen such a triangle.


\parbf{Model triangles.}
Given three points $p,q,r$ in a metric space $\spc{X}$,
let us define its \index{model!triangle}\emph{model triangle} $\trig{\tilde p}{\tilde q}{\tilde r}$ 
(briefly, 
$\trig{\tilde p}{\tilde q}{\tilde r}=\modtrig(p q r)_{\EE^2}$%
\index{$\modtrig$ (model triangle)}) to be a triangle in the Euclidean plane $\EE^2$ such that
\begin{align*}\dist{\tilde p}{\tilde q}{\EE^2}&=\dist{p}{q}{\spc{X}},
&
\quad\dist{\tilde q}{\tilde r}{\EE^2}&=\dist{q}{r}{\spc{X}},
&
\quad\dist{\tilde r}{\tilde p}{\EE^2}&=\dist{r}{p}{\spc{X}}.
\end{align*}

In the same way, we can define the \index{hyperbolic model triangle}\emph{hyperbolic} and the \index{spherical model triangles}\emph{spherical model triangles} $\modtrig(p q r)_{\HH^2}$, $\modtrig(p q r)_{\SSS^2}$
in the Lobachevsky plane $\HH^2$ and the unit sphere~$\SSS^2$.
In the latter case, the model triangle is said to be defined if in addition
\[\dist{p}{q}{}+\dist{q}{r}{}+\dist{r}{p}{}< 2\cdot\pi.\]
In this case, the model triangle again exists and is unique up to an isometry of~$\SSS^2$.

\parbf{Model angles.}
If 
$\trig{\tilde p}{\tilde q}{\tilde r}=\modtrig(p q r)_{\EE^2}$ 
and $\dist{p}{q}{},\dist{p}{r}{}>0$, 
the angle measure of 
$\trig{\tilde p}{\tilde q}{\tilde r}$ at $\tilde p$ 
will be called the \index{model!angle}\emph{model angle} of the triple $p$, $q$, $r$ and will be denoted by
$\angk p q r_{\EE^2}$%
\index{$\angk{p}{q}{r}$ (model angle)}.\label{page:model-angle}

For example, if $\dist{p}{q}{}=\dist{q}{r}{}=\dist{r}{p}{}$, then $\angk p q r_{\EE^2}=\tfrac\pi3$ regardless of existence and relative position of geodesics $[pq]$ and $[pr]$.

The same way we define $\angk p q r_{\MM^2(\kappa)}$;
in particular, $\angk p q r_{\HH^2}$ and $\angk p q r_{\SSS^2}$.
We may use the notation $\angk p q r$ if it is evident which of the model spaces is meant.

\begin{thm}{Exercise}\label{ex:k-><mono}
Show that for any triple of point $p$, $q$, and $r$,
the function
\[\kappa\mapsto \angk p q r_{\MM^2(\kappa)}\]
is nondecreasing in its domain of definition.
\end{thm}


\section{Hinges and their angle measure}\label{sec:angles}

\parbf{Hinges.} Let $p,x,y\in \spc{X}$ be a triple of points such that $p$ is distinct from $x$ and~$y$.
A pair of geodesics $([p x],[p y])$ will be called  a \index{hinge}\emph{hinge} and will be denoted by 
$\hinge p x y=([p x],[p y])$\index{$\hinge p x y$ (hinge)}.

\parbf{Angles.}
The angle measure of a hinge $\hinge p x y$ is defined as the following limit
\[\mangle\hinge p x y=\lim_{\bar x,\bar y\to p} \angk p{\bar x}{\bar y},\]
where $\bar x\in\left]p x\right]$ and $\bar y\in\left]p y\right]$.

Note that if $\mangle\hinge p x y$ is defined, then
\[0\le \mangle\hinge p x y\le \pi.\]

\begin{thm}{Exercise}\label{ex:angkK}
Suppose that in the above definition, one uses spherical or hyperbolic model angles instead of Euclidean.
Show that it does not change the value $\mangle\hinge p x y$.
\end{thm}


\begin{thm}{Exercise}\label{ex:undefined-angle}
Give an example of a hinge $\hinge p x y$ in a metric space with an undefined angle measure $\mangle\hinge p x y$.
\end{thm}

\section{Triangle inequality for angles}

\begin{thm}{Proposition}\label{claim:angle-3angle-inq}
Let  $[px_1]$, $[px_2]$, and $[px_3]$ be three geodesics in a metric space.
Suppose all the angle measures $\alpha_{i j}=\mangle\hinge p {x_i}{x_j}$ are defined.
Then 
\[\alpha_{13}\le \alpha_{12}+\alpha_{23}.\]

\end{thm}



\parit{Proof.}
Since $\alpha_{13}\le\pi$, we can assume that $\alpha_{12}+\alpha_{23}< \pi$.
Denote by $\gamma_i$ the unit-speed parametrization of $[px_i]$ from $p$ to $x_i$.
Given any $\eps>0$, for all sufficiently small $t,\tau,s\in\RR_{\ge0}$ we have
\begin{align*}
\dist{\gamma_1(t)}{\gamma_3(\tau)}{}
&\le 
\dist{\gamma_1(t)}{\gamma_2(s)}{}+\dist{\gamma_2(s)}{\gamma_3(\tau)}{}<\\
&<
\sqrt{t^2+s^2-2\cdot t\cdot  s\cdot \cos(\alpha_{12}+\eps)} +
\\
&\quad+\sqrt{s^2+\tau^2-2\cdot s\cdot \tau\cdot \cos(\alpha_{23}+\eps)}\le
\end{align*}

\begin{wrapfigure}{o}{30 mm}
\vskip-6mm
\centering
\includegraphics{mppics/pic-615}
\vskip6mm
\end{wrapfigure}

Below we define 
$s(t,\tau)$ so that for 
$s=s(t,\tau)$, this chain of inequalities can be continued as follows:
\[\le
\sqrt{t^2+\tau^2-2\cdot t\cdot \tau\cdot \cos(\alpha_{12}+\alpha_{23}+2\cdot \eps)}.
\]

Thus for any $\eps>0$, 
\[\alpha_{13}\le \alpha_{12}+\alpha_{23}+2\cdot \eps.\]
Hence the result follows.

To define $s(t,\tau)$, consider three half-lines $\tilde \gamma_1$, $\tilde \gamma_2$, $\tilde \gamma_3$ on a Euclidean plane starting at one point, such that
$\mangle(\tilde \gamma_1,\tilde \gamma_2)\z=\alpha_{12}+\eps$,
$\mangle(\tilde \gamma_2,\tilde \gamma_3)\z=\alpha_{23}+\eps$,
and $\mangle(\tilde \gamma_1,\tilde \gamma_3)\z=\alpha_{12}\z+\alpha_{23}\z+2\cdot \eps$.
We parametrize each half-line by the distance from the starting point.
Given two positive numbers $t,\tau\in\RR_{\ge0}$, let $s=s(t,\tau)$ be 
the number such that 
$\tilde \gamma_2(s)\in[\tilde \gamma_1(t)\ \tilde \gamma_3(\tau)]$. 
Clearly, $s\le\max\{t,\tau\}$, so $t,\tau,s$ may be taken sufficiently small.
\qeds 

\begin{thm}{Exercise}\label{ex:adjacent-angles}
Prove that the sum of adjacent angles is at least $\pi$.

More precisely: suppose two hinges $\hinge pxz$ and $\hinge pyz$ are \index{adjacent hinges}\emph{adjacent};
that is, they share side $[pz]$, and the union of two sides $[px]$ and $[py]$ form a geodesic $[xy]$.
Show that
\[\mangle\hinge pxz+\mangle\hinge pyz\ge \pi\]
whenever  each angle on the left-hand side is defined.

Give an example showing that the inequality can be strict.
\end{thm}

\begin{thm}{Exercise}\label{ex:first-var}
Assume that the angle measure of $\hinge q p x$ is defined.
Let $\gamma$ be the unit speed parametrization of $[qx]$ from $q$ to $x$.
Show that
\[\dist{p}{\gamma(t)}{}
\le
\dist{q}{p}{}-t\cdot \cos(\mangle\hinge q p x)+o(t).\]

\end{thm}

\section{Hausdorff convergence}\label{sec:Hausdorff convergence}

\begin{thm}{Definition}\label{def:gen-Haus-conv}
Let $A_1,A_2,\dots$ be a sequence of closed sets in a metric space $\spc{X}$.
We say that the sequence $A_n$ \index{Hausdorff!limit}\emph{converges} to a closed set $A_\infty$ in the {}\emph{sense of Hausdorff} if, for any $x\in\spc{X}$, we have
$\distfun_{A_n}(x)\z\to \distfun_{A_\infty}(x)$ as $n\to\infty$.
\end{thm}

For example, suppose $\spc{X}$ is the Euclidean plane and $A_n$ is the circle with radius $n$ and center at the point $(0,n)$; it converges to the $x$-axis.

\begin{figure}[ht!]
\vskip-0mm
\centering
\includegraphics{mppics/pic-415}
\end{figure}

Further, consider the sequence of one-point sets $B_n=\{(n,0)\}$ in the Euclidean plane.
It converges to the empty set;
indeed, for any point $x$ we have $\distfun_{B_n}(x)\to\infty$ as $n\to \infty$ and $\distfun_{\emptyset}(x)= \infty$ for any~$x$.

The following exercise is an extension of the so-called Blaschke selection theorem to our version of Hausdorff convergence.

\begin{thm}{Exercise}\label{ex:generalized-selection}
Show that any sequence of closed sets in a proper metric space has a convergent subsequence in the sense of Hausdorff.
\end{thm}

\section{Hausdorff metric}

\begin{thm}{Definition}\label{def:hausdorff-convergence}
Let $A$ and $B$ be two non-empty compact subsets of a metric space $\spc{X}$.
Then the \index{Hausdorff!distance}\emph{Hausdorff distance} between $A$ and $B$ is defined as 
$$|A-B|_{\Haus\spc{X}}
\df
\sup_{x\in \spc{X}}\{\,|\distfun_A(x)-\distfun_B(x)|\,\}.
$$

\end{thm}

The following observation gives a useful reformulation of the definition:

\begin{thm}{Observation}\label{obs:Haus-nbhds}
Suppose $A$ and $B$ be two compact subsets of a metric space $\spc{X}$.
Then $|A-B|_{\Haus\spc{X}}< R$ if and only if and only if 
$B$ lies in an $R$-neighborhood of $A$, 
and 
$A$ lies in an $R$-neighborhood of~$B$.
\end{thm}

The following exercise implies that Hausdorff convergence of compact subsets is the convergence in Hausdorff metric.

\begin{thm}{Exercise}\label{ex:Haus-conv}
Let $A_1,A_2,\dots,$ and $A_\infty$ be compact non-empty sets in a metric space $\spc{X}$.
Show that $\dist{A_n}{A_\infty}{\Haus\spc{X}}\to 0$ as $n\to\infty$
if and only if $A_n\to A_\infty$ in the sense of Hausdorff.
\end{thm}

\section{Gromov--Hausdorff convergence}\label{sec:Gromov--Hausdorff}

Let $\spc{X}_1,\spc{X}_2,\dots,$ and $\spc{X}_\infty$ be a sequence of complete metric spaces.
Suppose that there is a metric on the disjoint union 
\[\bm{X}=\bigsqcup_{n\in \NN\cup\{\infty\}} \spc{X}_n\] 
that satisfies the following property:

\begin{thm}{Property}\label{propery:GH}
The restriction of the metric on each $\spc{X}_n$ and $\spc{X}_\infty$ coincides with its original metric, 
and $\spc{X}_n\to \spc{X}_\infty$ as subsets in $\bm{X}$ in the sense of Hausdorff.
\end{thm}

In this case we say that the metric on $\bm{X}$ \textit{defines} a \index{Gromov--Hausdorff limit}\emph{convergence} $\spc{X}_n\z\to \spc{X}_\infty$ in the {}\emph{sense of Gromov--Hausdorff}.
The metric on  $\bigsqcup \spc{X}_n$ makes it possible to talk about limits of sequences $x_n\in \spc{X}_n$ as $n\to\infty$, as well as weak limits of a sequence of Borel measures $\mu_n$ on $\spc{X}_n$ and so on.

The limit space is not uniquely defined by the sequence.
For example, if each space $\spc{X}_n$ in the sequence is isometric to the half-line, then its limit might be isometric to the half-line or the whole line.
The first convergence is evident and the second could be guessed from the diagram.

\begin{figure}[ht!]
\vskip-0mm
\centering
\includegraphics{mppics/pic-500}
\end{figure}

Note that any sequence of spaces has an empty space as its limit in some  Gromov--Hausdorff convergence.
Exercise \ref{ex:compact-GH} states that if the limit is non-empty and compact, then it is unique up to isometry. 

\begin{thm}{Exercise}\label{ex:geod-closed}
Let $\spc{X}_1,\spc{X}_2,\dots$ be a sequence of geodesic metric spaces.
Suppose $\spc{X}_n\to \spc{X}_\infty$ is a convergence in the sense of Gromov--Hausdorff.
Assume $\spc{X}_\infty$ is proper, show that it is geodesic.
\end{thm}

\parbf{Pointed convergence.}
Often the isometry class of the limit can be fixed by marking a point $p_n$ in each space $\spc{X}_n$.
We say that $(\spc{X}_n,p_n)$ converges to $(\spc{X}_\infty,p_\infty)$ if there is a metric on $\bm{X}$ as in \ref{propery:GH} such that $p_n\to p_\infty$.
This is called \index{pointed convergence}\emph{pointed Gromov--Hausdorff convergence}.
For example, the sequence $(\spc{X}_n,p_n)=(\RR_{\ge0},0)$ converges to $(\RR_{\ge0},0)$, while $(\spc{X}_n,p_n)=(\RR_{\ge0},n)$ converges to $(\RR,0)$ as $n\to \infty$.

\section{Gromov--Hausdorff metric}\label{sec:Gromov--Hausdorff-metric}

In this section we cook up a metric space out of all compact non-empty metric spaces
that defines Gromov--Hausdorff convergence.
We want to define the metric on the set of \textit{isometry classes} of compact metric spaces.
Further, the term \textit{metric space} might also stand for its \textit{isometry class}.

The obtained metric is called the Gromov--Hausdorff metric;
the corresponding metric space will be denoted by $\GH$.
This distance is defined as the maximal metric such that \textit{the distance between subspaces in a metric space is not greater than the Hausdorff distance between them}.
Here is a formal definition.

\begin{thm}{Definition}\label{def:GH}
The \index{Gromov--Hausdorff distance}\emph{Gromov--Hausdorff distance} $|\spc{X}-\spc{Y}|_{\GH}$ between compact metric spaces $\spc{X}$ and $\spc{Y}$
is defined by the following
relation.
 
Given  $r > 0$, we have $|\spc{X}-\spc{Y}|_{\GH} < r$ if and only if there exists a metric
space $\spc{W}$ and subspaces $\spc{X}'$ and $\spc{Y}'$ in $\spc{W}$ that are isometric to $\spc{X}$ and $\spc{Y}$,
respectively, such that $|\spc{X}'-\spc{Y}'|_{\Haus\spc{W}} < r$. 
(Here $|\spc{X}'-\spc{Y}'|_{\Haus\spc{W}}$ denotes the Hausdorff distance between sets $\spc{X}'$ and $\spc{Y}'$ in $\spc{W}$.)
\end{thm}

For the proof of the following statement we refer to \cite{burago-burago-ivanov} and \cite{petrunin2023pure}.

\begin{thm}{Proposition}\label{prop:complete}
$\GH$ is a complete metric space.
\end{thm}

Note that this means in particular that if $X,Y$ are compact and $|\spc{X}-\spc{Y}|_{\GH}=0$ then $X$ and $Y$ are isometric.

Gromov--Hausdorff convergence of compact spaces has particularly nice properties.
From the technical point of view, they follow from the next statement, which we formulate as an exercise.

\begin{thm}{Exercise}\label{ex:non-contracting-map}
Let $f$ be a distance noncontracting map from 
a compact metric space $\spc{K}$ to itself.
Show that $f$ is an isometry; that is, it is a distance-preserving bijection.
\end{thm}

For two metric spaces $\spc{X}$ and $\spc{Y}$,
we write $\spc{X}\le \spc{Y}+\eps$ if
there is a map $f\:\spc{X}\to \spc{Y}$ such that 
\[\dist{x}{x'}{\spc{X}}\le \dist{f(x)}{f(x')}{\spc{Y}}+\eps\]
for any $x,x'\in \spc{X}$.

\begin{thm}{Exercise}\label{ex:GH-po}
Let $\spc{X}_1,\spc{X}_2,\dots,$ and $\spc{X}_\infty$ are compact metric spaces.
Show that there is a Gromov--Hausdorff convergence $\spc{X}_n\to\spc{X}_\infty$ if and only if for some sequence $\eps_n\to 0$,
we have 
\[\spc{X}_\infty\le \spc{X}_n+\eps_n\quad\text{and}\quad \spc{X}_n\le \spc{X}_\infty+\eps_n.\]
\end{thm}

\begin{thm}{Exercise}\label{ex:compact-GH}
Let $\spc{X}_1,\spc{X}_2,\dots$ be a sequence of metric spaces.
Suppose $\spc{X}_\infty$ and $\spc{X}_\infty'$ are non-empty limit spaces for some Gromov--Hausdorff convergences of $\spc{X}_n$.
Assume $\spc{X}_\infty$ is compact, show that it is isometric to~$\spc{X}_\infty'$.
\end{thm}

\section{Almost isometries}

\begin{thm}{Definition}
Let $\spc{X}$ and $\spc{Y}$ be metric spaces.
A map $f\:\spc{X}\to\spc{Y}$
is called an \index{isometry!$\eps$-isometry}\emph{$\eps$-isometry}
if the following two conditions hold:

\begin{subthm}{}
$f(\spc{X})$ is an \index{$\eps$-net}\emph{$\eps$-net} in $\spc{Y}$; that is, for any $y\in \spc{Y}$ there is $x\in \spc{X}$ such that $\dist{f(x)}{y}{\spc{Y}}<\eps$.
\end{subthm}

\begin{subthm}{}
$\bigl|\dist{f(x)}{f(x')}{\spc{Y}}-\dist{x}{x'}{\spc{X}}\bigr|\le \eps$ for any $x,x'\in\spc{X}$.
\end{subthm}

\end{thm}

When dealing with Gromov--Hausdorff convergence the following lemma is often useful as it allows to bypass constructing explicit metrics on the disjoint unions of $\spc{X}_1,\spc{X}_2,\dots$, and $\spc{X}_\infty$

\begin{thm}{Lemma}\label{lem:almost-isom}
Let $\spc{X}_1,\spc{X}_2,\dots$, and $\spc{X}_\infty$ be complete metric spaces,
and let $\eps_n\to\0+$ as $n\to\infty$.
Suppose that either 
\begin{subthm}{lem:almost-isom-a}
for each $n$ there is an $\eps_n$-isometry $f_n\:\spc{X}_n\to\spc{X}_\infty$, or
\end{subthm}
\begin{subthm}{lem:almost-isom-b}
for each $n$ there is an $\eps_n$-isometry $h_n\:\spc{X}_\infty\to\spc{X}_n$.
\end{subthm}
Then there is a Gromov--Hausdorff convergence $\spc{X}_n\to \spc{X}_\infty$.

Furthermore, a partial converse also holds.

\begin{subthm}{lem:almost-isom-c}
Suppose we have a Gromov--Hausdorff convergence $\spc{X}_n\to \spc{X}_\infty$ and $\spc{X}_\infty$ is compact. Then there exist $\eps_n\to\0+$ as $n\to\infty$ and  $\eps_n$-isometris $f_n\:\spc{X}_n\to\spc{X}_\infty$ (and $h_n\:\spc{X}_\infty\to\spc{X}_n$)
such that $x_n\in \spc{X}_n$ converges to $x_\infty \in  \spc{X}_\infty$ with respect to the  convergence $\spc{X}_n\z\to \spc{X}_\infty$ if and only if $f_n(x_n)\to x_\infty$ (respectively, $\dist{h_n(x_\infty) }{x_n}{\spc{X}_n}\to 0$) as $n\to\infty$.
\end{subthm}
\end{thm}


\parit{Proof.}
To prove part \ref{SHORT.lem:almost-isom-a} let us construct a common space $\bm{X}$ for the spaces $\spc{X}_1,\spc{X}_2,\dots$, and $\spc{X}_\infty$
by taking the metric $\rho$ on the disjoint union $\spc{X}_\infty\sqcup\spc{X}_1\sqcup\spc{X}_2\sqcup\dots$ that is defined the following way:
\begin{align*}
\dist{x_n}{y_n}{\bm{X}}&=\dist{x_n}{y_n}{\spc{X}_n},
\\
\dist{x_\infty}{y_\infty}{\bm{X}}&=\dist{x_\infty}{y_\infty}{\spc{X}_\infty},
\\
\dist{x_n}{x_\infty}{\bm{X}}&=\inf\set{\dist{x_n}{y_n}{\spc{X}_n}+\eps_n+\dist{x_\infty}{f(y_n)}{\spc{X}_\infty}}{{y_n}\in \spc{X}_n},
\\
\dist{x_n}{x_m}{\bm{X}}&=\inf\set{\dist{x_n}{y_\infty}{\bm{X}}+\dist{x_m}{y_\infty}{\bm{X}}}{y_\infty\in\spc{X}_\infty},
\end{align*}
where we assume that $x_\infty,y_\infty\in \spc{X}_\infty$, and $x_n,y_n\in \spc{X}_n$ for each $n$. 
It remains to observe that this indeed defines a metric on $\bm{X}$, and $\spc{X}_n\to \spc{X}_\infty$ in the sense of Hausdorff.

The proof of the second part is analogous; one only needs to change one line in the definition of the metric to the following:
\[\dist{x_n}{x_\infty}{\bm{X}}=\inf\set{\dist{x_n}{h(y_\infty)}{\spc{X}_n}+\eps_n+\dist{x_\infty}{y_\infty}{\spc{X}_\infty}}{{y_\infty}\in \spc{X}_\infty}.\]

We leave part \ref{SHORT.lem:almost-isom-c} as an exercise.
\qedsf

Lemma~\ref{lem:almost-isom} has a natural analogue for pointed convergence.
For simplicity we only state part \ref{SHORT.lem:almost-isom-a} of the lemma.
Parts \ref{SHORT.lem:almost-isom-b} and \ref{SHORT.lem:almost-isom-c} can be rephrased similarly.



\begin{thm}{Lemma}\label{lem:almost-isom-pointed}
Let $(\spc{X}_1, p_1),(\spc{X}_2,p_2) ,\dots$, let $(\spc{X}_\infty, p_\infty)$ be pointed metric spaces, and let $\eps(n,R)\to\0+$ as $n\to\infty$ for any fixed $R>0$.
Suppose that for each $n$ there is a map $f_n\:\spc{X}_n\to\spc{X}_\infty$ such that


\begin{subthm}{}
$f_n(p_n)\to p_\infty$
\end{subthm}

\begin{subthm}{}
$\bigl|\dist{f_n(x)}{f_n(x')}{\spc{X}_\infty }-\dist{x}{x'}{\spc{X}_n}\bigr|\le \eps(n,R)$ for any $x,x'\z\in \oBall(p_n,R)$.
\end{subthm}

\begin{subthm}{}
For any $x \in \oBall(p_\infty,R)$ there is $x_n\in \oBall(p_n,R)$ such that $\dist{x}{f_n(x_n)}{}\le \eps(n,R)$
\end{subthm}

Then there is a pointed  Gromov--Hausdorff convergence $(\spc{X}_n,p_n)\z\to (\spc{X}_\infty,p_\infty)$.
\end{thm}

The proofs of \ref{lem:almost-isom-pointed} and \ref{lem:almost-isom} are analogous;
we leave the former to the reader.



\section{Comments}

In principle, our definition of Gromov--Hausdorff distance works for complete metric spaces that are not necessarily compact.
However, according to the following exercise, it only defines a \emph{semimetric}; that is, zero Gromov--Hausdorff distance does not imply that the spaces are isometric.
For that reason it is not in use.

\begin{thm}{Exercise}\label{ex:GH-noncompact}
Construct two nonisometric proper (noncompact) metric spaces with vanishing Gromov--Hausdorff distance.
\end{thm}


%%%%%%%%%%%%%%%%%%%%%%%%%%%%
%\chapter{Definitions}

The first synthetic description of curvature is due to Abraham Wald \cite{wald} published in 1936;
it was his student work, written under the supervision of Karl Menger. 
This publication was not noticed for about 50 years \cite{berestovskii}.
In 1941, similar definitions were rediscovered by Alexandr Alexandrov \cite{alexandrov:def}.



\section{Wald's approach}

Abraham Wald noticed that given a \textit{typical} metric on the quadruple of points $\spc{X}\z=\{x_1,x_2,x_3,x_4\}$ there is a closed interval,
say 
\[[\kappa_{\min}(x_1,x_2,x_3,x_4),\kappa_{\max}(x_1,x_2,x_3,x_4)]\subset \RR\]
such that there is a \textit{model configuration} in $\MM^3(\kappa)$;
that is, $\tilde x_1$, $\tilde x_2$, $\tilde x_3$, $\tilde x_4\in\MM^3(\kappa)$ such that
\[\dist{\tilde x_i}{\tilde x_j}{\MM^3(\kappa)}=\dist{x_i}{x_j}{\spc{X}}\]
for all $i$ and $j$.


\begin{wrapfigure}{r}{33mm}
\vskip-2mm
\centering
\includegraphics{mppics/pic-710}
\end{wrapfigure}

In $\MM^3(\kappa_{\min})$ and $\MM^3(\kappa_{\max})$, the points $\tilde x_1,\tilde x_2,\tilde x_3,\tilde x_4$ form degenerate tetrahedrons shown on the diagram (for $\kappa_{\min}$ it is a convex quadrangle and for $\kappa_{\max}$ --- a triangle with a point inside).
In the interior of the interval, the tetrahedron is nondegenerate.

Moreover, one can use $[-\infty,\infty)$ instead of $\RR$ 
and let
\[\kappa_{\min}(x_1,x_2,x_3,x_4)=-\infty\]
if there is \textit{almost} model quadruple in
$\MM^3(\kappa)$ for $\kappa\to -\infty$;
that is, for any $\eps>0$ there is a quadruple
$\tilde x_1,\tilde x_2,\tilde x_3,\tilde x_4\in\MM^3(\kappa)$
such that $\kappa\le -\tfrac1\eps$, and
\[\dist{\tilde x_i}{\tilde x_j}{\MM^3(\kappa)}\lege\dist{x_i}{x_j}{\spc{X}}\pm\eps\]
for all $i$ and $j$.
In this case the interval 
\[[\kappa_{\min}(x_1,x_2,x_3,x_4),\kappa_{\max}(x_1,x_2,x_3,x_4)]\subset [-\infty,\infty)\]
is defined for \textit{any} quadruple.

\begin{thm}{Exercise}
Let $x_1,x_2,x_3,x_4$ be a quadruple in a metric space such that $\kappa_{\min}(x_1,x_2,x_3,x_4)=-\infty$.
Show that two maximal numbers from the following three are equal to each other.
\begin{align*}
a&=\dist{x_1}{x_2}{}+\dist{x_3}{x_4}{},
\\
b&=\dist{x_1}{x_3}{}+\dist{x_2}{x_4}{},
\\
c&=\dist{x_1}{x_4}{}+\dist{x_2}{x_3}{}.
\end{align*}


\end{thm}


\begin{thm}{Exercise}
Suppose that $x_1,x_2,x_3,x_4$ in a metric space
such that
\begin{align*}
\dist{x_1}{x_2}{}=\dist{x_1}{x_3}{}=\dist{x_1}{x_4}{}&=1,
\\
\dist{x_2}{x_3}{}=\dist{x_3}{x_4}{}=\dist{x_4}{x_1}{}&=2.
\end{align*}
Show that 
\[\kappa_{\min}(x_1,x_2,x_3,x_4)=\kappa_{\max}(x_1,x_2,x_3,x_4)=-\infty.\]
\end{thm}

\begin{thm}{Exercise}
Let $x_1,x_2,x_3,x_4$ be a quadruple in $\EE^2$.
Suppose that $x_3$ lie on the line thru $x_1$ and $x_2$,
but $x_4$ does not.
Show that 
\[\kappa_{\min}(x_1,x_2,x_3,x_4)=\kappa_{\max}(x_1,x_2,x_3,x_4)=0.\]
\end{thm}

\begin{thm}{Wald-style definition}
Let $\kappa\in \RR$.
A metric space $\spc{X}$ has curvature $\ge\kappa$ (or $\le\kappa$) 
if for any quadruple $x_1,x_2,x_3,x_4\in \spc{X}$ we have 
$\kappa_{\max}(x_1,x_2,x_3,x_4)\ge \kappa$ (or $\kappa_{\min}(x_1,x_2,x_3,x_4)\le \kappa$ respectively). 
\end{thm}

This definition is given for its historical value.
It will not be used further in the sequel.
We will use another definition that is very close, but not equivalent.

\section{Substance}\label{sec:manifesto}

Consider the space $\mathcal{M}_4$ of all isometry classes of 4-point metric spaces.
Each element in $\mathcal{M}_4$ can be described by 6 numbers 
 --- the distances between all 6 pairs of its points, say $\ell_{i,j}$ for $1\le i< j\le 4$ modulo permutations of the index set $(1,2,3,4)$.
These 6 numbers are subject to 12 triangle inequalities; that is,
\[\ell_{i,j}+\ell_{j,k}\ge \ell_{i,k}\]
holds for all $i$, $j$ and $k$, where we assume that $\ell_{j,i}=\ell_{i,j}$, and $\ell_{i,i}=0$.

{

\begin{wrapfigure}{o}{33mm}
\vskip-3mm
\centering
\includegraphics{mppics/pic-700}
\end{wrapfigure}

The space $\mathcal{M}_4$ comes with topology.
It can be defined as a quotient topology of the cone in $\RR^6$ by permutations of the 4 points of the space.

Consider the subset $\mathcal{E}_4\subset \mathcal{M}_4$ of all isometry classes of 4-point metric spaces that admit isometric embeddings into Euclidean space.

}

\begin{thm}{Claim}\label{clm:two-components-of-M4}
The complement $\mathcal{M}_4\setminus \mathcal{E}_4$ has two connected components.
\end{thm}

\begin{thm}{Exercise}
Spend 10 minutes trying to prove the claim.
\end{thm}


The definition of Alexandrov spaces is based on the claim above.
Let us denote one of the components by $\mathcal{P}_4$ and the other by~$\mathcal{N}_4$.
Here $\mathcal{P}$ and $\mathcal{N}$ stand for {}\emph{positive} 
and {}\emph{negative curvature} because spheres have no quadruples of type $\mathcal{N}_4$ and 
hyperbolic space
has no quadruples of type~$\mathcal{P}_4$.

A metric space that has no quadruples of points of type $\mathcal{P}_4$ or $\mathcal{N}_4$
respectively 
is called an Alexandrov space with non-positive or non-negative curvature (briefly.

\begin{wrapfigure}{r}{33mm}
\vskip-0mm
\centering
\includegraphics{mppics/pic-710}
\end{wrapfigure}

Let us describe the subdivision into  $\mathcal{P}_4$, $\mathcal{E}_4$, and $\mathcal{N}_4$ intuitively.
Imagine that you move out of $\mathcal{E}_4$ --- your path is a one-parameter family of 4-point metric spaces.
The last thing you see in $\mathcal{E}_4$ is one of the two plane configurations shown on the diagram.
If you see the right configuration then you move into $\mathcal{N}_4$;
if it is the one on the left, then you move into $\mathcal{P}_4$.
More degenerate pictures can be avoided; for example, a triangle with a point on a side.
From such a configuration one may move in $\mathcal{N}_4$ and $\mathcal{P}_4$ (as well as come back to $\mathcal{E}_4$).

Here is an exercise, solving which would force you to rebuild a considerable part of Alexandrov geometry.
It is wise to spend some time thinking about this it before proceeding.

\begin{thm}{Advanced exercise}\label{ex:convex-set}
Assume $\spc{X}$ is a complete metric space with length metric (see Section~\ref{sec:length}), 
containing only quadruples of type~$\mathcal{E}_4$.
Show that $\spc{X}$ is isometric to a convex set in a Hilbert space.
\end{thm}

If in the definition above, we take $\MM^3(\kappa)$ instead of $\EE^3$.
Then we will arrive at Wald's definition of curvature bounded below and above by $\kappa$.
The parameter $\kappa$ has three interesting choices $-1$, $0$, and $1$;
the rest can be obtained from these three applying rescaling.

Again, the definition that we are going to use is not equivalent.


\section{Embedding theorem}

The following theorem is historically the first remarkable result in Alexandrov geometry.
The main part of the following theorem is due to Alexandr Alexandrov~\cite{alexandrov-1948}.
The last part is very difficult; it was proved by Aleksei Pogorelov~\cite{pogorelov}.

\begin{thm}{Theorem}\label{thm:alexandrov+pogorelov}
A metric space $\spc{X}$ is isometric to the surface of a convex body in the Euclidean space if and only if $\spc{X}$ is an $\Alex0$ space that is homeomorphic to $\SSS^2$.

Moreover, $\spc{X}$ determines the convex body up to congruence.
\end{thm}

The convex body above is a compact convex subset in $\EE^3$;
we assume that it does not lie in a line but might degenerate to a plane figure, say $F$.
In the latter case, its surface is defined as two copies of $F$ glued along the boundary.
For nondegenerate convex body $B$, its surface is its boundary $\partial B$ equipped with the induced length metric. 

The only-if part of the theorem is the simplest; we will give a complete proof of it eventually.
The if part will be sketched.
We will not touch the last part.

%%!TEX root = the-definitions.tex
\chapter{Definitions}\label{chap:defs}

In this lecture we prove equivalence of several definitions of Alexandrov space.


\section{Four-point comparison}\label{sec:4-point}

Recall that $\angk  pxy$ denotes the model angle; see page \pageref{page:model-angle}.

Let $p,x,y,z$ be a quadruple of points in a metric space.
If the inequality 
\[\angk  pxy_{\EE^2}+\angk pyz_{\EE^2}+\angk pzx_{\EE^2}
\le 
2\cdot\pi
\eqlbl{eq:CBB-comparison}\]
holds, then we say that the quadruple meets \index{comparison}\emph{$\EE^2$-comparison}.
If the left-hand side is undefined, then we assume that the comparison holds.

\begin{thm}{Exercise}\label{ex:CBB+-}
Suppose $\EE^2$-comparison holds for quadruple $p,x_1,x_2,x_3$.
Show that $\EE^2$-comparison holds for quadruple $q,y_1,y_2,y_3$ if
\[\dist{q}{y_i}{}\ge\dist{p}{x_i}{}\qquad\text{and}\qquad\dist{y_i}{y_j}{}\le\dist{x_i}{x_j}{}\]
for all $i$ and $j$.
\end{thm}

Instead of $\EE^2$, we can use $\SSS^2$ or $\HH^2$.
This way we get the definition of $\SSS^2$- or $\HH^2$-comparisons.
Recall that $\angk  pxy_{\EE^2}$ and $\angk  pxy_{\HH^2}$ are defined if $p\ne x$, $p\ne y$,
but for $\angk  pxy_{\SSS^2}$ we require in addition that
\[\dist{p}{x}{}+\dist{p}{y}{}+\dist{x}{y}{}<2\cdot\pi;\]
if this does not hold for one of the angles, then we assume that $\SSS^2$-comparison holds for this quadruple.

More generally, one may apply this definition to $\MM^2(\kappa)$ and  define $\MM^2(\kappa)$-comparison for any real $\kappa$.
However, if you see $\MM^2(\kappa)$-comparison, it is safe to assume that $\kappa=-1$, $0$, or $1$;
applying rescaling, the $\MM^2(\kappa)$-comparison can be reduced to these three cases.

\begin{thm}{Definition}\label{def:CBB}
A metric space $\spc{X}$ has {}\emph{curvature $\ge\kappa$} in the sense of Alexandrov
if $\MM^2(\kappa)$-comparison
holds for any quadruple of points in $\spc{X}$.
\end{thm}

While this definition can be applied to any metric space,
we will use it mostly for geodesic spaces that are complete (and often compact or proper). 
If a complete geodesic space has curvature $\ge\kappa$ in the sense of Alexandrov, 
then it will be called an $\Alex\kappa$ space; here $\Alex\kappa$ is an adjective.
An $\spc{X}$ is $\Alex\kappa$ for some $\kappa$, then we say that $\spc{X}$ is an \index{Alexandrov space}\emph{Alexandrov space}.

It is common practice in Alexandrov geometry to write proofs for nonnegative curvature and 
leave the general curvature bound as an exercise. These generalizations are usually straightforward. We will add notes when they are not.
We will also often formulate statements just for $\kappa=0$ even when they admit straightforward generalizations to arbitrary curvature bounds;
see \cite{alexander-kapovitch-petrunin2024} for a more formal tratment.


\begin{thm}{Exercise}\label{ex:Euclid-is-CBB}
Show that $\EE^n$ is $\Alex0$.
\end{thm}

\begin{thm}{Exercise}\label{ex:(3+1)-expanding}
Show that a metric space $\spc{X}$ has nonnegative curvature in the sense of Alexandrov
if and only if for any quadruple of points $p,x_1,x_2,x_3\in \spc{X}$ 
there is a quadruple of points $q,y_1,y_2,y_3\in\EE^3$
such that 
\[\dist{p}{x_i}{\spc{X}}\ge\dist{q}{y_i}{\EE^2} 
\quad \text{and}\quad
\dist{x_i}{x_j}{\spc{X}}\le\dist{y_i}{y_j}{\EE^2}\] 
for all $i$ and $j$.
\end{thm}

\section{Alexandrov's lemma}

Recall that $[xy]$ denotes a geodesic from $x$ to $y$;
set  
\index{10@$\left]x y\right]$, $\left[x y\right[$, $\left]x y\right[$}
\[
\left]x y\right]=[xy]\setminus\{x\},
\quad
\left[x y\right[=[xy]\setminus\{y\},
\quad
\left]x y\right[=[xy]\setminus\{x,y\}.\]

\begin{thm}{Lemma}
\index{Alexandrov's lemma}
\label{lem:alex}  
Let $p,x,y,z$ be distinct points in a metric space such that $z\in \left]x y\right[$.
Then 
the following expressions have the same sign:

\begin{subthm}{lem-alex-difference}
$\angk x p y
-\angk x p z$,
\end{subthm} 

\begin{subthm}{lem-alex-angle}
$\angk z p x
+\angk z p y -\pi$.
\end{subthm}

\begin{wrapfigure}{r}{25mm}
\vskip-6mm
\centering
\includegraphics{mppics/pic-730}
\end{wrapfigure}

The same holds for the hyperbolic and spherical model angles, 
but in the latter case, one has to assume in addition that
\[\dist{p}{x}{}+\dist{p}{y}{}+\dist{x}{y}{}< 2\cdot\pi.\]

\end{thm}

In the proof we will apply the following statement from elementary geometry.

\begin{thm}{Angle monotonicity}\label{angle-monotonicity}
Increasing the opposite side in a plane triangle increases the corresponding angle, and the other way around.

Moreover, the same statement holds for spherical and hyperbolic triangles.
\end{thm}


\parit{Proof.} 
Consider the model triangle $\trig{\tilde x}{\tilde p}{\tilde z}=\modtrig(x p z)$.
Take 
a point $\tilde y$ on the extension of 
$[\tilde x \tilde z]$ beyond $\tilde z$ so that $\dist{\tilde x}{\tilde y}{}=\dist{x}{y}{}$ (and therefore $\dist{\tilde x}{\tilde z}{}=\dist{x}{z}{}$). 

\begin{wrapfigure}{r}{33mm}
\vskip-0mm
\centering
\includegraphics{mppics/pic-740}
\end{wrapfigure}

By the angle monotonicity (\ref{angle-monotonicity}),
the following expressions have the same sign:
\begin{enumerate}[(i)]
\item $\mangle\hinge{\tilde x}{\tilde p}{\tilde y}-\angk{x}{p}{y}$,
\item $\dist{\tilde p}{\tilde y}{}-\dist{p}{y}{}$,
\item $\mangle\hinge{\tilde z}{\tilde p}{\tilde y}-\angk{z}{p}{y}$.
\end{enumerate}
Since 
\[\mangle\hinge{\tilde x}{\tilde p}{\tilde y}=\mangle\hinge{\tilde x}{\tilde p}{\tilde z}=\angk{x}{p}{z}\]
and
\[ \mangle\hinge{\tilde z}{\tilde p}{\tilde y}
=\pi-\mangle\hinge{\tilde z}{\tilde x}{\tilde p}
=\pi-\angk{z}{x}{p},\]
the statement follows.


The spherical and hyperbolic cases can be proved along the same lines.
\qeds

\begin{thm}{Exercise}\label{ex:alex-lemma-cat}
Assume $p,x,y,z$ are as in Alexandrov's lemma (\ref{lem:alex}).
Show that
\[\angk p x y
\ge
\angk p x z + \angk p z y,\]
with equality if and only if the expressions in \ref{SHORT.lem-alex-difference} and \ref{SHORT.lem-alex-angle} in Alexandrov's lemma vanish.
\end{thm}

Note that 
\[p\in\left]x y\right[
\quad\Longrightarrow\quad
\angk pxy=\pi.
\]
Applying it with Alexandrov's lemma and $\EE^2$-comparison, we get the following.

\begin{thm}{Claim}\label{clm:angle-mono}
If $p,x,y,z$ are points in an $\Alex0$ space.
Suppose $p\in\left]x y\right[$, then 
\[\angk xyz\le \angk xpz.\]
\end{thm}

\begin{wrapfigure}{r}{25mm}
\vskip-0mm
\centering
\includegraphics{mppics/pic-750}
\end{wrapfigure}

\begin{thm}{Exercise}\label{ex:noncreasing}
Let $\hinge p x y$ be a hinge in an $\Alex0$ space.
Consider the function
\[f\:(\dist{p}{\bar x}{},\dist{p}{\bar y}{})\mapsto \angk p{\bar x}{\bar y},\]
where $\bar x\in\left]p x\right]$ and $\bar y\in\left]p y\right]$.
Show that $f$ is nonincreasing in each argument.
\end{thm}

This exercise implies the following.

\begin{thm}{Claim}\label{clm:angle-defined}
The angle measure of any hinge in an $\Alex0$ 
space is defined and  is at least as large as the corresponding model angle;
that is,
\[\mangle\hinge p x y\ge \angk p x y\]
for any hinge $\hinge p x y$ in an $\Alex0$.

\end{thm}

\begin{thm}{Exercise}\label{ex:0-angle}
Let $\hinge p x y$ be a hinge in an $\Alex0$ space.
Suppose $\mangle\hinge p x y=0$; show that $[px]\subset [py]$ or $[py]\subset [px]$.

Conclude that geodesics in $\Alex0$ space cannot \emph{bifurcate};
that is, if two geodesics $[px]$ and $[py]$ share a nontrivial arc with an end at $p$, then $[px]\subset [py]$ or $[py]\subset [px]$.
\end{thm}

\begin{thm}{Exercise}\label{ex:pi-angle}
Let $[xy]$ be a geodesic in an $\Alex0$ space.
Suppose $z\in \left]xy\right[$. Show that there is a unique geodesic $[xz]$ and $[xz]\subset [xy]$.
\end{thm}

Recall that adjacent hinges are defined in \ref{ex:adjacent-angles}.

\begin{thm}{Exercise}\label{ex:adjacent-CBB}
Let $\hinge pxz$ and $\hinge pyz$ be adjacent hinges in an $\Alex0$ 
space.
Show that
\[\mangle\hinge pxz+\mangle\hinge pyz= \pi.\]
\end{thm}


\begin{thm}{Exercise}\label{ex:pxyvw}
Let $\spc{A}$ be an $\Alex0$ 
space.
Show that  
\[
\angk xyp=\angk xvp
\quad\Longleftrightarrow\quad
\angk xyp=\angk xwp
\]
for any points
$p,x,y,v,w$ in $\spc{A}$ such that $v,w\in \left]xy\right[$.
\end{thm}

\begin{thm}{Exercise}\label{ex:angle-lim}
Let $\spc{A}$ be an $\Alex0$ space.
Suppose hinges $\hinge {x_n}{y_n}{z_n}$ in $\spc{A}$ converge to a hinge $\hinge {x_\infty}{y_\infty}{z_\infty}$;
that is, geodesics $[x_ny_n]$ and $[x_nz_n]$ converge to the geodesics $[x_\infty y_\infty]$ and $[x_\infty z_\infty]$ in the sense of Hausdorff.
Show that 
\[\liminf_{n\to\infty}\mangle \hinge {x_n}{y_n}{z_n}\ge \mangle \hinge {x_\infty}{y_\infty}{z_\infty}.\]
\end{thm}

The last inequality might be strict;
for example, on the surface of convex polyhedron, which is a $\Alex0$ space by \ref{prop:conv-surf-CBB(0)}.

\section{Hinge comparison}

Let $\hinge pxy$ be a hinge in an $\Alex0$ space $\spc{A}$.
By \ref{ex:noncreasing}, the angle measure $\mangle\hinge pxy$ is defined and
\[\mangle\hinge pxy\ge \angk pxy.\]
Further, according to \ref{ex:adjacent-CBB}, we have 
\[\mangle\hinge pxz+\mangle\hinge pyz=\pi\]
for adjacent hinges $\hinge pxz$ and $\hinge pyz$ in $\spc{A}$.

The following theorem provides a converse.

\begin{thm}{Theorem}\label{thm:angle-cbb}
A complete geodesic space $\spc{A}$ is $\Alex0$ if the following conditions hold.

\begin{subthm}{angle-a}
For any hinge $\hinge x p y$ in $\spc{A}$, the angle 
$\mangle\hinge x p y$ is defined and 
\[\mangle\hinge x p y\ge\angk x p y.\]
\end{subthm}

\begin{subthm}{angle-b}
For any two adjacent hinges $\hinge pxz$ and $\hinge pyz$ in $\spc{A}$, we have
\[\mangle\hinge pxz+\mangle\hinge pyz\le\pi.\]
\end{subthm}

\end{thm}

\parit{Proof.}
Consider a point  $w\in \mathopen{]} p z \mathclose{[}$ close to $p$.
From \ref{SHORT.angle-b}, it follows that 
\[\mangle\hinge w x z+ \mangle\hinge w x{p}\le\pi\quad \text{and}\quad \mangle\hinge w y z + \mangle\hinge w y{p}\le\pi.\]

\begin{wrapfigure}{o}{30 mm}
\vskip-0mm
\centering
\includegraphics{mppics/pic-805}
\vskip4mm
\end{wrapfigure}

Since $\mangle\hinge w x y\le \mangle\hinge w x p +\mangle\hinge w y{p}$ (see \ref{claim:angle-3angle-inq}), we get 
\[\mangle\hinge w x z+ \mangle\hinge w y z +\mangle\hinge w x y
\le
2\cdot\pi.\]
Applying \ref{SHORT.angle-a}, 
\[\angk w x z
+ \angk w y z 
+\angk w x y
\le
2\cdot\pi.\]
Passing to the limits as $w\to p$, we have
\[\angk p x z 
+ \angk p y z 
+\angk p x y
\le
2\cdot\pi.\]
\qedsf

\section{Equivalent conditions}

The following theorem summarizes \ref{clm:angle-mono}, \ref{clm:angle-defined}, \ref{ex:adjacent-CBB}, and \ref{thm:angle-cbb}.

\begin{thm}{Theorem}\label{thm:defs_of_alex} 
Let $\spc{A}$ be a complete geodesic space.
Then the following conditions are equivalent.

\begin{subthm}{cbb}
$\spc{A}$ is $\Alex0$.
\end{subthm}
 

\begin{subthm}{2-sum} 
(adjacent angle comparison\index{comparison!adjacent angle comparison})
\[\angk z p x
+\angk z p y\le \pi\]
for any geodesic $[x y]$ and point $z\in \mathopen{]}x y\mathclose{[}$, $z\ne p$ in $\spc{A}$.
\end{subthm}

\begin{subthm}{point-on-side}
(\index{comparison!point-on-side comparison}point-on-side comparison)
\[\angk x p y\le\angk x p z\]
for any geodesic $[x y]$ and $z\in \mathopen{]}x y\mathclose{[}$ in $\spc{A}$.
\end{subthm}

\begin{subthm}{angle}(hinge comparison\index{comparison!hinge comparison})
\index{hinge!comparison}
the angle $\mangle\hinge x p y$ is defined for any hinge $\hinge x p y$ in $\spc{A}$.
Moreover, 
\[\mangle\hinge x p y\ge\angk x p y\]
for any hinge $\hinge x p y$, and
\[\mangle\hinge z p y + \mangle\hinge z p x\le\pi\]
for any adjacent hinges $\hinge z p y$ and $\hinge z p x$.
\end{subthm}

Moreover, the implications \ref{SHORT.cbb}$\Rightarrow$\ref{SHORT.2-sum}$\Rightarrow$\ref{SHORT.point-on-side}$\Rightarrow$\ref{SHORT.angle} hold in any space, not necessarily a geodesic one.
\end{thm}

\begin{thm}{Advanced Exercise}\label{ex:urysohn}
Construct a complete geodesic space $\spc{X}$ that is not $\Alex0$, but satisfies the following weaker version of the adjacent angle comparison \ref{2-sum}.

For any three points $p,x,y\in \spc{X}$ there is a geodesic $[x y]$ such that for any $z\in \left]x y\right[$
\[\angk{z}{p}{x}+\angk{z}{p}{y}
\le
\pi.\]
\end{thm}

\begin{thm}{Exercise}\label{ex:normCBB}
Let $\spc{W}$ be $\RR^n$ with the metric induced by a norm.
Show that if $\spc{W}$ is $\Alex0$, then $\spc{W}$ is isometric to the Euclidean space~$\EE^n$.
\end{thm}

\section{Function comparison}\label{Function comparison}

\parbf{Real-to-real functions.}
Choose $\lambda\in \RR$.
Let $s\:\II\to\RR$ be a locally Lipschitz function defined on an interval $\II$.
The following statement are equivalent;
if one (and therefore any) of them holds for $s$, then we say that $s$ is \index{91@$\lambda$-concave function}\emph{$\lambda$-concave}.
\begin{itemize}
\item We have inequality $s''\le \lambda$, where the second derivative $s''$ is understood in the sense of distributions.
\item The function $t\mapsto s(t)-\lambda\cdot\tfrac{t^2}2$ is concave.
\item The \index{Jensen inequality}\emph{Jensen inequality}
\[s(a\cdot t_0+(1-a)\cdot t_1)\ge a\cdot s(t_0)+(1-a)\cdot s(t_1)+\tfrac\lambda2\cdot a\cdot(1-a)\cdot(t_1-t_0)^2 \]
holds for any $t_0,t_1\in \II$ and $a\in[0,1]$.
\item for any $t_0\in \II$ there is a quadratic polynomial $\ell=\tfrac\lambda2\cdot t^2+a\cdot t+b$ (it is called a \index{barrier}\emph{barrier}) that supports (locally) $s$ at $t_0$ from above;
that is, $\ell(t_0)\z= s(t_0)$ and $\ell(t)\ge s(t)$ for any $t$ (in a neighborhood of $t_0$)
\end{itemize}

To prove equivalence, approximate $f$ by smooth functions taking a convolutions $f_n=f*k_n$ for a suitable sequence of kernels $k_n$.
Note that all the conditions are equivalent for $f_n$;
passing to the limit we get the same for $f$.

\begin{thm}{Exercise}\label{ex:concave'}
Show that $\lambda$-concave functions are one-sided differentiable.
\end{thm}

The following exercise implies that if the function defined on an open interval, then the Lipschitz condition can be dropped from the definition of $\lambda$-concavity.

\begin{thm}{Exercise}\label{ex:concave-open}
Suppose a real-to-real function $f$ is defined on an open inerval and satisfies one the Jensen inequality stated above.
Show that $f$ is locally Lipscitz.
\end{thm}

\parbf{Functions on metric spaces.}
A function on a metric space $\spc{A}$ will usually mean a \textit{locally Lipschitz real-valued function defined on an open subset of $\spc{A}$}.
The domain of a function $f$ will be denoted by $\Dom f$.

We say that $f$ is \index{91@$\lambda$-concave function}\emph{$\lambda$-concave} (briefly $f''\le \lambda$) if
for any unit-speed geodesic $\gamma\:\II\z\to \Dom f$
the real-to-real function $t\mapsto f\circ\gamma(t)$ is $\lambda$-concave.

The following proposition is simple but conceptual ---
it reduces a global comparison to an infinitesimal condition on distance functions.

\begin{thm}{Proposition}\label{comp-kappa}
A complete geodesic space $\spc{A}$ is $\Alex0$ if and only if $f''\le 1$ for any function $f$ of the form
\[f\:x\mapsto \tfrac12\cdot\dist[2]{p}{x}{}.\] 
\end{thm} 

\parit{Proof.}
Choose a unit-speed geodesic $\gamma$ in $\spc{A}$ and two points $x=\gamma(t_0)$, $y=\gamma(t_1)$ for some $t_0<t_1$.
Consider the model triangle $\trig{\tilde p}{\tilde x}{\tilde y}\z=\modtrig(p x y)$.
Let $\tilde \gamma\:[t_0,t_1]\to\EE^2$ be the unit-speed parametrization of $[\tilde x \tilde y]$ from $\tilde x$ to $\tilde y$.

Set
\begin{align*} 
\tilde r(t)&\df\dist{\tilde p}{\tilde\gamma(t)}{},
& 
r(t)&\df\dist{p}{\gamma(t)}{}.
\end{align*}
Clearly, $\tilde r(t_0)=r(t_0)$ and $\tilde r(t_1)=r(t_1)$.
Note that the point-on-side comparison (\ref{point-on-side}) says that the implication
\[t_0\le t\le t_1
\qquad\Longrightarrow\qquad
\tilde r(t)\le r(t)
\eqlbl{eq:r=<r}\]
holds for any $\gamma$ and $t_0<t_1$.

Jensen's inequality for the function $h$ is equivalent to \ref{eq:r=<r}.
Hence the proposition follows.
\qeds

\section{Semiconcave functions}\label{sec:Semiconcave functions}

Recall that $\lambda$-concave functions were defined in Section \ref{Function comparison},
and when we say \textit{function} we usually mean a \textit{locally Lipschitz function defined on an open domain}.

Let $f$ be a locally Lipschitz real-valued function defined in an open subset $\Dom f$ of an Alexandrov space $\spc{A}$.
Suppose $\phi$ is a continuous function defined in $\Dom f$.
We will write $f''\le \phi$ if for any point $x\in \Dom f$ and any $\eps>0$ there is a neighborhood $U\ni x$ such that
the restriction $f|_U$ is $(\phi(x)+\eps)$-concave.

If $f''\le \phi$ for some continuous function $\phi$, then $f$ is called  \index{semiconcave function}\emph{semiconcave}.

\begin{thm}{Exercise}\label{ex:distfun-semiconcave}
Let $f$ be a \emph{distance function} on an $\Alex0$ space $\spc{A}$;
that is, $f(x)\equiv\dist{p}{x}{}$ for some $p\in \spc{A}$.
Show that $f''\le \tfrac1f$.
In particular, $f$ is semiconcave in $\spc{A}\setminus\{p\}$.
\end{thm}

Proposition~\ref{comp-kappa} admits the following generalization.
The is nearly the same, but the formulas are getting more complicated.

\begin{thm}{Proposition}
A complete geodesic space $\spc{A}$ is $\Alex{\mp1}$
if $f''\z\le \pm f$ for any function of the type $f=\cosh\circ\distfun_p$ (respectively, $f=-\cos\circ\distfun_p|_{\oBall(p,\pi)}$).
\end{thm}

The geometric meaning of these inequalities remains the same:
\textit{distance functions are more concave than distance functions in $\MM^2(\kappa)$}.

\section{Remarks}

Note that Alexandrov's lemma is a result in neutral geometry;
it has the following useful variation; see \cite[10.2]{alexander-kapovitch-petrunin2024} or \cite[3.3]{alexander-kapovitch-kirszbraun}.

\begin{thm}{Overlap lemma}\label{lem:extend-overlap}
Let $\tilde x^1$, $\tilde x^2$, $\tilde x^3$, $\tilde p^1$, $\tilde p^2$, ans $\tilde p^3$ be points in $\EE^2$, $\SSS^2$, or $\HH^2$.
Assume that, for any permutation $\{i,j,k\}$ of $\{1,2,3\}$, we have
\begin{enumerate}[(i)]

\item
\label{no-overlap:px=px}
$\dist{\tilde p^i}{\tilde x^\kay}{}=\dist{\tilde p^j}{\tilde x^\kay}{}$,
%$\dist{\tilde p^i}{\tilde x^\kay}{}=\dist{\tilde p^j}{\tilde x^\kay}{}$,

\item
\label{no-overlap:orient-1}
$\tilde p^i$ and $\tilde x^i$ lie in the same closed half-space determined by $[\tilde x^j\tilde x^\kay]$,
\end{enumerate}

If no pair of triangles $\trig{\tilde p^i}{\tilde x^j}{\tilde x^\kay}$ overlap,
then
\[\mangle{\tilde p^1} +\mangle {\tilde p^2}+\mangle{\tilde p^3}> 2\cdot\pi,\]
where $\mangle\tilde p^i\df\mangle\hinge{\tilde p^i}{\tilde x^\kay}{\tilde x^j}$
for a permutation $\{i,j,k\}$ of $\{1,2,3\}$.
\end{thm}

The condition \ref{SHORT.angle-b} in \ref{thm:angle-cbb} might be superfluous.
This is a long-standing open problem possibly dating back to Alexandrov \cite[footnote in 4.1.5]{burago-burago-ivanov}.
Let us state it formally.

\begin{thm}{Open question}\label{open:hinge-}
Let $\spc{A}$ be a complete geodesic space (you can also assume that $\spc{A}$ is homeomorphic to $\mathbb{S}^2$ or $\RR^2$)
such that for any hinge $\hinge x p y$ in $\spc{A}$,
the angle $\mangle\hinge x p y$ is defined and
\[\mangle\hinge x p y\ge\angk x p y.\]
Is it true that $\spc{A}$ is an Alexandrov space?
\end{thm}

Our 4-point comparison in Section~\ref{sec:4-point} is closely related to the so-called $\CAT$ comparison, which defines an \textit{upper} curvature bound in the sense of Alexandrov;
this is the subject of our previous  book  \cite{alexander-kapovitch-petrunin-2019}.

In both comparisons we check certain conditions on the 6 distances between pairs of points in a 4-point set.
Michael Gromov \cite[Section 1.19$_+$]{gromov1999} suggested considering other conditions of that type for $n$-point subsets;
see \cite{toyoda,lebedeva-petrunin-zolotov,lebedeva2019,petrunin2017,lebedeva-petrunin2024,lebedeva-petrunin2023,lebedeva-petrunin2021,lebedeva-petrunin2025,eskenazis-mendel-naor,gromov2001} for the development of this idea.

One coul define Alxandrov space as a complete \textit{length} space with curvature $\ge \kappa$.
This condition is more natural and general, but many statements can be reduced to the geodesic case.
In particular, suppose $\spc{A}$ is a complete length space with curvature $\ge \kappa$,
then 
\textit{$\spc{A}$ can be isometrically embedded into an $\Alex\kappa$ space} --- the ultrapower of $\spc{A}$; see \cite[4.11+8.4]{alexander-kapovitch-petrunin2024}.
Also, by Plaut's theorem, any point $p$ in $\spc{A}$ can be connected by geodesics to \textit{most} of points in $\spc{A}$
\cite[8.11]{alexander-kapovitch-petrunin2024}; compare to \ref{ex:grad-dist:geod}.

%%!TEX root = the-globalization.tex
\chapter{Globalization}\label{chap:globalization}

The globalization theorem states that a locally Alexandrov space is globally Alexandrov.
We start with the simplest meaningful case of this theorem and indicate a way to extend.

\section{Globalization}

A complete geodesic metric space $\spc{A}$ is \index{locally $\Alex0$}\emph{locally $\Alex0$} if any point $p\in\spc{A}$ admits a neighborhood $U\ni p$ such that the $\EE^2$-comparison holds for any quadruple of points in $U$.

\begin{thm}{Globalization theorem}\label{thm:glob} 
Any compact locally $\Alex0$ space is $\Alex0$.
\end{thm}

\parit{Proof modulo the key lemma.}
Note that condition \ref{angle-b} holds in $\spc{A}$ (the proof is the same).
It remains to check \ref{angle-a};
that is,
\[\mangle\hinge x p y\ge\angk x p y
\eqlbl{eq:mod-angle-CBB-comp-glob}\]
for any hinge $\hinge x p y$ in $\spc{A}$.

First note that \ref{eq:mod-angle-CBB-comp-glob} holds for hinges in a small neighborhood of any point;
this can be proved the same way as \ref{clm:angle-defined} and \ref{ex:adjacent-CBB}, applying the local version of the $\EE^2$-comparison.
Since $\spc{A}$ is compact, there is $\eps>0$ such that \ref{eq:mod-angle-CBB-comp-glob} holds if $\dist{x}{p}{}+\dist{p}{y}{}<\eps$.
Applying the key lemma several times we get that \ref{eq:mod-angle-CBB-comp-glob} holds for any given hinge.
\qeds

\begin{thm}{Key lemma}\label{key-lem:globalization} 
Let $\spc{A}$ be locally $\Alex0$. 
Assume that the comparison
\[\mangle\hinge x p q
\ge\angk x p q\]
holds for any hinge $\hinge x p q$ with 
$\dist{x}{y}{}+\dist{x}{q}{}
<
\frac{2}{3}\cdot\ell$.
Then the comparison
\[\mangle\hinge x p q
\ge\angk x p q\] 
holds for any hinge $\hinge x p q$ with $\dist{x}{ p}{}+\dist{x}{q}{}<\ell$.
\end{thm}

Let $\hinge x p q$ be a hinge in $\spc{A}$.
Denote by $\side \hinge x p q$ its \index{$\side \hinge x p q$ (model side)}\index{model!side}\emph{model side};
this is the opposite side in a flat triangle with the same angle and two adjacent sides as in $\hinge x p q$.

\begin{wrapfigure}{r}{44mm}
\centering
\includegraphics{mppics/pic-105}
\end{wrapfigure}

More precisely,
consider the model hinge $\hinge {\tilde x} {\tilde p} {\tilde q}$ in $\EE^2$ that is defined by 
\begin{align*}
\mangle\hinge {\tilde x} {\tilde p} {\tilde q}_{\EE^2}&=\mangle\hinge x p q_{\spc{A}},
\\
\dist{\tilde x} {\tilde p}{\EE^2}&=\dist{x} {p}{\spc{A}},
\\
\dist{\tilde x} {\tilde q}{\EE^2}&=\dist{x} {q}{\spc{A}};
\intertext{then}
\side \hinge x p q_{\spc{A}}
&\df
\dist{\tilde p}{\tilde q}{\EE^2}.
\end{align*}

Note that 
\[\side \hinge x p q \ge\dist{p}{q}{}
\quad\Longleftrightarrow\quad
\mangle\hinge x p q\ge \angk x p q.
\]
We will use it in the following proof.

\parit{Proof.} 
It is sufficient to prove the inequality
\[\side \hinge x p q
\ge\dist{p}{q}{}\eqlbl{eq:thm:=def-loc*}\] 
for any hinge $\hinge x p q$ with $\dist{x}{p}{}+\dist{x}{q}{}<\ell$.

Consider a hinge $\hinge x p q$ such that 
\[\tfrac{2}{3}\cdot\ell \le\dist{p}{x}{}\z+\dist{x}{q}{}< \ell.\]
First, let us construct a new hinge $\hinge{x'}p q$ with
\[
\dist{p}{x}{}+\dist{x}{q}{}\ge\dist{p}{x'}{}+\dist{x'}{q}{},
\eqlbl{eq:thm:=def-loc-fourstar}\]
such that 
\[\side \hinge x p q
\ge\side \hinge{x'}p q.
\eqlbl{eq:thm:=def-loc-fivestar}\]

\parit{Construction.}
Assume $\dist{x}{q}{}\ge\dist{x}{p}{}$; otherwise, switch the roles of $p$ and $q$.
Take $x'\in [x q]$ such that 
\[\dist{p}{x}{}+3\cdot\dist[{{}}]{x}{x'}{}
=\tfrac{2}{3}\cdot\ell. \eqlbl{3|xx'|}\]
Choose a geodesic $[x' p]$ and consider the  hinge $\hinge{x'}p q$ formed by $[x'p]$ and $[x' q]\subset [x q]$.
The triangle inequality implies \ref{eq:thm:=def-loc-fourstar}.
Further, note that 
\begin{align*}
\dist{p}{x}{}\z+\dist{x}{x'}{}&<\tfrac{2}{3}\cdot\ell,
&
\dist{p}{x'}{}\z+\dist{x'}{x}{}&<\tfrac{2}{3}\cdot\ell.
\end{align*}
In particular, 
\[\mangle\hinge x p{x'}
\ge\angk x p{x'}
\quad \text{and}\quad 
\mangle\hinge {x'}p x
\ge\angk {x'}p x.
\eqlbl{eq:thm:=def-loc-threestar}\]

{

\begin{wrapfigure}{r}{30 mm}
\vskip-0mm
\centering
\includegraphics{mppics/pic-820}
\vskip-4mm
\end{wrapfigure}

Now, let 
$\trig{\tilde x}{\tilde x'}{\tilde p}=\modtrig(x x' p)$.
Take $\tilde  q$ on the extension of $[\tilde  x\tilde  x']$ beyond $x'$ such that $\dist{\tilde x}{\tilde q}{}\z=\dist{x}{q}{}$ (and therefore $\dist{\tilde x'}{\tilde q}{}=\dist{x'}{q}{}$).
By~\ref{eq:thm:=def-loc-threestar},
\[\mangle\hinge x p q
=\mangle\hinge  x p{x'}\ge\angk x p{x'}\quad \Rightarrow\quad 
\side \hinge x q p\ge\dist{\tilde p}{\tilde q}{}.\]
Hence
\begin{align*}
\mangle\hinge{\tilde x'}{\tilde p}{\tilde q}&= 
\pi
-\angk{x'}p x
\ge
\\
&\ge
\pi-\mangle\hinge{x'}p x
=
\\
&=
\mangle\hinge{x'}p q,
\end{align*}
and \ref{eq:thm:=def-loc-fivestar} follows.

}

\medskip

Let us continue the proof.
Set $x_0=x$.
Let us apply inductively the above construction to get a sequence of hinges  $\hinge{x_n}p q$ with $x_{n+1}=x_n'$.
From \ref{eq:thm:=def-loc-fivestar}, we have that the sequence  $s_n\z=\side \hinge{x_n}p q$ is nonincreasing.
\begin{figure}[ht!]
\centering
\includegraphics{mppics/pic-825}
\end{figure}

The sequence might terminate at some $n$ only if $\dist{p}{x_n}{}+\dist{x_n}{q}{}\z< \tfrac{2}{3}\cdot\ell $.
In this case, by the assumptions of the lemma, $\side \hinge{x_n}p q\ge\dist{p}{q}{}$.
Since the sequence $s_n$ is nonincreasing, inequality \ref{eq:thm:=def-loc*} follows.

Otherwise, the sequence $r_n=\dist{p}{x_n}{}+\dist{x_n}{q}{}$ is nonincreasing, and $r_n\ge\tfrac{2}{3}\cdot\ell$ for all $n$.
Note that by construction, the distances
$\dist{x_n}{x_{n+1}}{}$, $\dist{x_n}{p}{}$, and $\dist{x_n}{q}{}$ are bounded away from zero for all large $n$.
Indeed, since on each step, we move $x_n$ toward to the point $p$ or $q$ that is further away, the distances $\dist{x_n}{p}{}$ and $\dist{x_n}{q}{}$ become about the same.
Namely, by \ref{3|xx'|}, we have that $\dist{p}{x_n}{}-\dist{x_n}{q}{}\le \tfrac29\cdot\ell$ for all large $n$.
Since $\dist{p}{x_n}{}+\dist{x_n}{q}{}\ge \tfrac23\cdot\ell$, we have $\dist{x_n}{p}{}\ge \tfrac\ell{100}$ and $\dist{x_n}{q}{}\ge \tfrac\ell{100}$.
Further, since $r_n\ge\tfrac{2}{3}\cdot\ell$, \ref{3|xx'|} implies that $\dist{x_n}{x_{n+1}}{}>\tfrac\ell{100}$.


Since the sequence $r_n$ is nonincreasing, it converges.
In particular, $r_n-r_{n+1}\to 0$ as $n\to\infty$.
It follows that $\angk{x_n}{p_n}{x_{n+1}}\to \pi$,
where $p_n=p$ if $x_{n+1}\in [x_nq]$, and otherwise $p_n=q$.
Since $\mangle\hinge{x_n}{p_n}{x_{n+1}}\ge\angk{x_n}{p_n}{x_{n+1}}$, we have
$\mangle\hinge{x_n}{p_n}{x_{n+1}}\to \pi$  as $n\to\infty$.

It follows that
\[r_n-s_n=\dist{p}{x_n}{}+\dist{x_n}{q}{}-\side \hinge{x_n}p q\to 0.\] 
Together with the triangle inequality
\[
\dist{p}{x_n}{}+\dist{x_n}{q}{}\ge\dist{p}{q}{}
\]
this yields
\[\lim_{n\to\infty}\side \hinge{x_n}p q\ge \dist{p}{q}{}.\]
Finally, the monotonicity of the sequence $s_n=\side \hinge{x_n}p q$ implies \ref{eq:thm:=def-loc*}.
\qeds

\section{General case}

The globalization theorem  can be generalized to any curvature bound $\kappa$.
The case $\kappa\le 0$ is proved in the same way, but the case $\kappa>0$ requires modifications.

The compactness condition in our version of the theorem can be traded for completeness.
The proof uses the following statement where $r(x)$ measures the size of a neighborhood of $x$ where the comparison holds.

\begin{thm}{Exercise}\label{ex:alm-min}
Let $\spc{X}$ be a complete metric space.
Suppose $r\:\spc{X}\to \RR$ is a positive continuous function.
Show that for any $\eps>0$ there is a point $p\in \spc{X}$ such that 
\[r(x)> (1-\eps)\cdot r(p)\] 
for any $x\in \cBall[p,\tfrac{1}{\eps}\cdot r(p)]$.

\end{thm}

This implies the following general version of the globalization theorem.

\begin{thm}{Theorem}\label{thm:globalization+}
Any locally $\Alex\kappa$ length space is $\Alex\kappa$.
\end{thm}

By \ref{ex:k-><mono}, we have
\[\angk x y z_{\MM^2(\kappa)}\le \angk x y z_{\MM^2(\Kappa)}\]
if $\kappa\le \Kappa$ and the right-hand side is defined.
It follows that a $\Alex\Kappa$ space is \textit{locally} $\Alex\kappa$.
Therefore, the globalization theorem implies the following.

\begin{thm}{Claim}\label{clm:K>k}
If $\Kappa>\kappa$, then any $\Alex\Kappa$ space is $\Alex\kappa$.
\end{thm}

In other words the expression \textit{curvature bounded below by $\kappa$} makes sense for geodesic spaces.
However, by the following exercise, it does not make much sense in general.

\begin{thm}{Exercise}\label{ex:CBB(1)notitCBB(0)}
Let $\spc{X}$ be the set $\{p,x_1,x_2,x_3\}$ with the metric defined by
\[\dist{p}{x_i}{}=\pi,\quad\dist{x_i}{x_j}{}=2\cdot\pi\]
for all $i\ne j$.
Show that $\spc{X}$ has curvature $\ge 1$, but does not have curvature $\ge 0$.
\end{thm}

\begin{thm}{Exercise}\label{ex:RisCBB(1)}
Let $p$ and $q$ be points in an $\Alex1$ space $\spc{A}$.
Suppose $\dist{p}{q}{}>\pi$.
Denote by $m$ the midpoint of $[pq]$.
Show that for any hinge $\hinge mxp$ we have
either $\mangle\hinge mxp=0$ or $\mangle\hinge mxp=\pi$.

Conclude that $\spc{A}$ is isometric to a line interval or a circle.

\end{thm}

\begin{thm}{Exercise}\label{ex:perim-k>0}
Suppose  
$\spc{A}$ is an $\Alex1$
and $\diam \spc{A}\le \pi$.
Show that 
\[\dist{x}{y}{}+\dist{y}{z}{}+\dist{z}{x}{}\le 2\cdot\pi\]
for any triple of points $x,y,z\in \spc{A}$.
\end{thm}


\section{Remarks}

The following question about \ref{angle-a} was stated in \cite[footnote in 4.1.5]{burago-burago-ivanov} but this is a long-standing open problem (possibly dating back to Alexandrov).

\begin{thm}{Open question}\label{open:hinge-}
Let $\spc{A}$ be a complete geodesic space (you can also assume that $\spc{A}$ is homeomorphic to $\mathbb{S}^2$ or $\RR^2$) 
such that for any hinge $\hinge x p y$ in $\spc{A}$, 
the angle $\mangle\hinge x p y$ is defined and 
\[\mangle\hinge x p y\ge\angk x p y.\]
Is it true that $\spc{A}$ is an Alexandrov space?
\end{thm}

The globalization theorem is also known as the \textit{generalized Toponogov theorem}.
Its two-dimensional case was proved by Paolo Pizzetti \cite{pizzetti};
later it was reproved independently by Alexandr Alexandrov \cite{alexandrov:devel}. %is it right ref?? 
Victor Toponogov \cite{toponogov-globalization+splitting} proved it for Riemannian manifolds of all dimensions.
For Alexandrov spaces of all dimensions, the theorem first appears in the paper of Michael Gromov, Yuriy Burago, and Grigory Perelman \cite{burago-gromov-perelman}.
Their statement is slightly more general than \ref{thm:globalization+}; it is for complete length spaces.
Another version for noncomplete, but geodesic spaces was proved by the second author \cite{petrunin:globalization}.


We took the proof from our book \cite{alexander-kapovitch-petrunin2024}, but reduced generality to compact nonnegatively curved spaces.
This proof is based on simplifications obtained by Conrad Plaut \cite{plaut:dimension} and Dmitry Burago, Yuriy Burago, and Sergei Ivanov \cite{burago-burago-ivanov}.
The same proof was rediscovered independently by Urs Lang and Viktor Schroeder \cite{lang-schroeder:globalization}.
Another simplified argument was found by Katsuhiro Shiohama \cite{shiohama}.





%%!TEX root = the-calculus.tex

\chapter{Calculus}\label{chap:derivative}

This lecture defines several notions related to the first-order derivatives in Alexandrov spaces;
this includes space of directions, tangent space, differential, and gradient.

\section{Space of directions} 
\label{sec:space+directions}

Let $\spc{A}$ be an Alexandrov space.
By \ref{ex:noncreasing}, the angle measure of any hinge in is defined.
Given $p\in \spc{A}$, consider the set $\mathfrak{S}_p$ of all nontrivial geodesics starting at $p$.
By \ref{claim:angle-3angle-inq}, the triangle inequality holds for $\mangle$ on $\mathfrak{S}_p$,
that is, $(\mathfrak{S}_p,\mangle)$ 
forms a \index{semimetric}\emph{semimetric} space;
that is, $\mangle$ behaves like a metric, but might vanish for distinct directions. 

The metric space corresponding to  $(\mathfrak{S}_p,\mangle)$ is called the \index{70@$\Sigma_p'$ (geodesic directions)}\index{space of geodesic directions}\emph{space of geodesic directions} at $p$, denoted by $\Sigma'_p$ or $\Sigma'_p\spc{A}$.
The elements of $\Sigma'_p$ are called \index{geodesic!direction}\emph{geodesic directions} at $p$.
Each geodesic direction is formed by an equivalence class of geodesics starting from $p$ 
for the equivalence relation 
\[[px]\sim[py]\quad \iff\quad \mangle\hinge pxy=0;\]
the direction of $[px]$ is denoted by $\dir px $.\index{40@$\dir{p}{q}$ (direction)}
By \ref{ex:0-angle}, 
\[[px]\sim[py]
\quad\iff\quad
[px]\subset [py]
\quad\text{or}\quad
[px]\supset[py].
\]
 
The completion of $\Sigma'_p$ is called the \index{space of directions}\emph{space of directions} at $p$ and is denoted by \index{70@$\Sigma_p$ (space of directions)}$\Sigma_p$ or $\Sigma_p\spc{A}$.
The elements of $\Sigma_p$ are called \index{direction}\emph{directions} at $p$.

\begin{thm}{Exercise}\label{ex:dir-compact}
Let $\spc{A}$ be an Alexandrov space.
Assume that a sequence of geodesics $[px_n]$ converge to a geodesic $[px_\infty]$ in the sense of Hausdorff,
and $x_\infty\ne p$.
Suppose $\Sigma_p$ is compact.
Show that $\dir p{x_n}\z\to\dir p{x_\infty}$ as $n\to\infty$.

\end{thm}


\section{Tangent space}\label{sec: tangent space}

The \index{65@$\Cone$}\index{cone}\emph{Euclidean cone} $\spc{V}=\Cone\spc{X}$
over a metric space $\spc{X}$
is defined as the metric space whose underlying set consists of
equivalence classes in
$[0,\infty)\times \spc{X}$ with the equivalence relation ``$\sim$'' given by $(0,p)\sim (0,q)$ for any points $p,q\in\spc{X}$,
and whose metric is given by the cosine rule
\[
\dist{(s,p)}{(t,q)}{\spc{V}} 
=
\sqrt{s^2+t^2-2\cdot s\cdot t\cdot \cos\theta},
\]
where $\theta= \min\{\pi, \dist{p}{q}{\spc{X}}\}$.

The leading example is
\[\Cone\SSS^n\iso\EE^{n+1};\]
here ``$\iso$'' stands for ``isometric to''. 
Now let us extend several notions from Euclidean space to Euclidean cones. 

The point in $\spc{V}$ that corresponds $(t,x)\z\in[0,\infty)\times \spc{X}$ will be denoted by $t\cdot x$.
The point in $\spc{V}$ formed by the equivalence class of $\{\0\}\times\spc{X}$ is called the \index{origin}\emph{origin} of the cone and is denoted by $\0$ or $\0_{\spc{V}}$.
For $v\in\spc{V}$ the distance $\dist{\0}{v}{\spc{V}}$ is called the \index{norm}\emph{norm} of $v$ and is denoted by $|v|$ or $|v|_{\spc{V}}$.
The \index{scalar product}\emph{scalar product} $\<v,w\>$
of $v=s\cdot p$ and $w=t\cdot q$
is defined by 
\[\<v,w\>
\df |v|\cdot|w|\cdot\cos\theta
\]
where $\theta= \min\{\pi, \dist{p}{q}{\spc{X}}\}$.
The value $\theta$ is undefined if $v=\0$ or $w=\0$;
in these cases we set $\<v,w\>\df0$.

\begin{thm}{Exercise}\label{ex:geodesic-cone}
Show that $\Cone\spc{X}$ is geodesic if and only if $\spc{X}$ is \index{91@$\ell$-geodesic space}\emph{$\pi$-geodesic};
that is, any two points $x,y\in \spc{X}$ such that $\dist{x}{y}{\spc{X}}<\pi$ can be joined by a geodesic in $\spc{X}$.
\end{thm}

\parbf{Tangent space.}
The Euclidean cone $\Cone\Sigma_p$ over the space of directions $\Sigma_p$ is called the \index{tangent space}\emph{tangent space} at $p$ and is denoted by \index{70@$\T_p$ (tangent space)}$\T_p$ or $\T_p\spc{A}$.
The elements of $\T_p\spc{A}$ will be called \index{tangent vector}\emph{tangent vectors} at $p$
(despite that $\T_p$ is only a cone --- not a vector space).
The space of directions $\Sigma_p$ can be (and will be) identified with the unit sphere in~$\T_p$;
that is, with the set $\set{v\in\T_p}{|v|=1}$.

\begin{thm}{Proposition}\label{prop:Tan-is-CBB(0)}
Any tangent space to an Alexandrov space has nonnegative curvature in the sense of Alexandrov.
\end{thm}

Halbeisen's example \cite{alexander-kapovitch-petrunin2024} shows that the tangent space $\T_p$ at some point of Alexandrov space might fail to be geodesic;
in this case $\T_p$ is \textit{not} $\Alex0$.

\parit{Proof.}
Consider the tangent space $\T_p=\Cone \Sigma_p$ of an Alexandrov space $\spc{A}$ at a point $p$.
We need to show that the $\EE^2$-comparison holds for a given quadruple $v_0$, $v_1$, $v_2$, $v_3\in \T_p$.

Recall that the space of geodesic directions $\Sigma_p'$ is dense in $\Sigma_p$.
It follows that the subcone $\T'_p=\Cone\Sigma_p'$ is dense in $\T_p$.
Therefore, it is sufficient to consider the case $v_0$, $v_1$, $v_2$, $v_3\in \T'_p$.

For each $i$, choose a geodesic $\gamma_i$ from $p$ in the direction of $v_i$;
reparametrize each $\gamma_i$ so that it has speed $|v_i|$.
Since the angles are defined, we have
\[\dist{\gamma_i(\eps)}{\gamma_j(\eps)}{\spc{A}}=\eps\cdot\dist{v_i}{v_j}{\T_p}+o(\eps)
\eqlbl{eq:gamma-v}\]
for $\eps>0$.
The quadruple $\gamma_0(\eps)$, $\gamma_1(\eps)$, $\gamma_2(\eps)$, $\gamma_3(\eps)$ meets the $\MM^2(\kappa)$-comparison.
After rescaling all the distances by $\tfrac1\eps$, it becomes the $\MM^2(\eps^2\cdot\kappa)$-comparison.
Passing to the limit as $\eps\to 0$ and applying \ref{eq:gamma-v}, we get the $\EE^2$-comparison for $v_0$, $v_1$, $v_2$, $v_3$.
\qeds


\begin{thm}{Exercise}\label{ex:GHto-tangent}
Let $p$ be a point in an Alexandrov space $\spc{A}$,
and let $\lambda_n\to\infty$.
Suppose $\Sigma_p$ is compact.
Show that there is a pointed Gromov--Hausdorff convergence $(\lambda_n\cdot \spc{A},p)\z\to (\T_p,0)$.
Moreover, for any geodesic $\gamma$ that starts at $p$, we have
\[\iota_n\circ\gamma(t/\lambda_n)\to t\cdot \gamma^+(0),\]
where $\iota_n$ sends a point in $\spc{A}$ to the corresponding point in $\lambda_n\cdot\spc{A}$.
\end{thm}

\section{Semiconcave functions}\label{sec:Semiconcave functions}

Recall that $\lambda$-concave functions were defined in Section \ref{Function comparison},
and when we say \textit{function} we usually mean a \textit{locally Lipschitz function defined on an open domain}.

Let $f$ be a locally Lipschitz real-valued function defined in an open subset $\Dom f$ of an Alexandrov space $\spc{A}$.
Suppose $\phi$ is a continuous function defined in $\Dom f$.
We will write $f''\le \phi$ if for any point $x\in \Dom f$ and any $\eps>0$ there is a neighborhood $U\ni x$ such that 
the restriction $f|_U$ is $(\phi(x)+\eps)$-concave.


If $f''\le \phi$ for some continuous function $\phi$, then $f$ is called  \index{semiconcave function}\emph{semiconcave}.


\begin{thm}{Exercise}\label{ex:distfun-semiconcave}
Let $f$ be a \emph{distance function} on an $\Alex0$ space $\spc{A}$;
that is, $f(x)\equiv\dist{p}{x}{}$ for some $p\in \spc{A}$.
Show that $f''\le \tfrac1f$.
In particular, $f$ is semiconcave in $\spc{A}\setminus\{p\}$.
\end{thm}


\section{Differential}\label{sec:differential}
\index{differential of a function}

Let $f$ be a semiconcave function on an Alexandrov space $\spc{A}$, and $p\z\in \Dom f$.
Choose a unit-speed geodesic $\gamma$ that starts at $p$;
let $\xi\in\Sigma_p$ be its direction.
Define 
\[(\dd_pf)(\xi)\df(f\circ\gamma)^+(0),\]
here $(f\circ\gamma)^+$ denotes the \index{right derivative}\emph{right derivative} of $(f\circ\gamma)$;
it is defined since $f$ is semiconcave.

By the following exercise, the value $(\dd_pf)(\xi)$ is defined; that is, it does not depend on the choice of $\gamma$.
Moreover, $\dd_pf$ is a Lipschitz function on $\Sigma'_p$.
It follows that the function $\dd_pf\:\Sigma_p'\to\RR$ can be uniquely extended to a Lipschitz function $\dd_pf\:\Sigma_p\to\RR$.
Further, we can extend it to the tangent space by setting 
\[(\dd_pf)(r\cdot \xi)
\df
r\cdot (\dd_pf)(\xi)\]
for any $r\ge 0$ and $\xi\in\Sigma_p$.
The obtained function $\dd_pf\:\T_p\to\RR$ is Lipschitz;
it is called the \index{differential}\emph{differential} of $f$ at $p$.

\begin{thm}{Exercise}\label{ex:df(xi)}
Let $f$ be a semiconcave function on an Alexandrov space.
Suppose $\gamma_1$ and $\gamma_2$ are geodesics that start at $p\z\in \Dom f$;
denote by $\theta$ the angle between $\gamma_1$ and $\gamma_2$ at $p$.
Show that 
\[|(f\circ\gamma_1)^+(0)-(f\circ\gamma_2)^+(0)|\le L\cdot \theta,\]
where $L$ is the Lipschitz constant of $f$ in a neighborhood of $p$.
\end{thm}

\begin{thm}{Exercise (First variation formula)} \label{ex:d(distfun)}
Let $p$ and $q$ be distinct points in an Alexandrov space~$\spc{A}$.
Show the following.

\begin{subthm}{ex:d(distfun):<}
$\dd_p\distfun_q(v)\le -\langle\dir pq,v\rangle$
for any $v\in\T_p$.
\end{subthm}

\begin{subthm}{ex:d(distfun):=}
Suppose $\spc{A}$ is proper.
Let $\Uparrow_p^q$ be the set of all direction of  geodesics from $p$ to $q$.
Then
\[\dd_p\distfun_q(v)=-\max_{\xi\in\Uparrow_p^q}\langle\xi,v\rangle\]
for any $v\in\T_p$.
\end{subthm}

\end{thm}

\section{Gradient}\label{sec:grad-def}

The following definition generalizes the gradient to semiconcave functions on Alexandrov space.
This generalization is not trivial even for concave functions on Euclidean space;
we suggest keeping this example in mind while reading further.

\begin{thm}{Definition}\label{def:grad} 
Let $f$ be a semiconcave function on an Alexandrov space.
A tangent vector $g\in \T_p$ is called a 
\index{gradient}\emph{gradient} of $f$ at $p$ 
(briefly,  $g\z=\nabla_p f$\index{19@$\nabla$ (gradient)}) if
\begin{subthm}{}
$(\dd_p f)(w)\le \<g,w\>$ for any $w\in \T_p$, and
\end{subthm}

\begin{subthm}{}
$(\dd_p f)(g) = \<g,g\>.$
\end{subthm}
\end{thm}

The following exercise provides a property of gradients that will play a key role in the proof of the first distance estimate (\ref{thm:dist-est}).

\begin{thm}{Exercise}\label{ex:monotonicity}
Let $f$ be a $\lambda$-concave function on an Alexandrov space.
Suppose that gradients $\nabla_xf$ and $\nabla_yf$ are defined.
Show that 
\[\<\dir{x}{y},\nabla_{x}f\>
+
\<\dir{y}{x},\nabla_{y}f\>
+
\lambda\cdot\dist{x}{y}{}\ge 0.\]
\end{thm}

\begin{figure}[ht!]
\centering
\includegraphics{mppics/pic-409}
\end{figure}

\begin{thm}{Proposition}\label{prop:grad-exist}
Suppose that a semiconcave function $f$ is defined in a neighborhood of a point $p$ in an Alexandrov space.
Then the gradient $\nabla_pf$ is uniquely defined.

Moreover, if $\dd_pf\le 0$, then we have $\nabla_pf=0$;
otherwise, $\nabla_pf\z=s\cdot \overline{\xi}$, where 
$s= \dd_pf(\overline{\xi})$
and
$\overline{\xi}\in \Sigma_p$ is the direction that maximize the value $\dd_pf(\xi)$ for $\xi\in \Sigma_p$.
\end{thm}


\begin{thm}{Key lemma}\label{lem:ohta} 
Let $f$ be a semiconcave function that is defined in a neighborhood of a point $p$
in an Alexandrov space $\spc{A}$. 
Then for any $u,v\in \T_p$, we have
\[s\cdot \sqrt{|u|^2+2\cdot\<u,v\> +|v|^2}
\ge 
(\dd_p f)(u)+(\dd_p f)(v),\]
where
\[s=\sup\set{(\dd_p f)(\xi)}{\xi\in\Sigma_p}.\]

\end{thm}

If $\T_p\iso\EE^m$ and $\dd_p f$ is a concave function,
then
\[2\cdot(\dd_p f)(\tfrac{u+v}2)\ge(\dd_p f)(u)+(\dd_p f)(v).\]
The latter implies the statement since $|u+v|=\sqrt{|u|^2+2\cdot\<u,v\> +|v|^2}$.
In general, $\T_p$ is not geodesic (and not even a length space), so concavity of $\dd_p f$ does not make  sense.
The key lemma however says  that in a certain sense $\dd_p f$ behaves as a concave function.

Solving the following exercise should help to find an approach to the key lemma.

\begin{thm}{Exercise}\label{ex:d(distfun):==}
Let $p$ and $q$ be distinct points in an Alexandrov space $\spc{A}$.
Suppose the geodesic $[pq]$ can be extended beyond $q$.

Show that
\[\dd_p\distfun_q(v)= -\langle\dir pq,v\rangle\]
for any $v\in\T_p$.
\end{thm}

\parit{Proof of \ref{lem:ohta}.}
We will assume that $\spc{A}$ is $\Alex0$ and $f$ is concave;
the general case requires only minor modifications.
We can assume that $v\ne 0$, $w\ne 0$, and $\alpha=\mangle(u,v)>0$; otherwise, the statement is trivial.

{

\begin{wrapfigure}{r}{34 mm}
\vskip-4mm
\centering
\includegraphics{mppics/pic-1205}
\vskip0mm
\end{wrapfigure}

Consider a model configuration of five points: $\tilde p$, $\tilde u$, $\tilde v$, $\tilde q$, $\tilde w\in\EE^2$ such that
\begin{itemize}
\item $\mangle\hinge{\tilde p}{\tilde u}{\tilde v}=\alpha$, 
\item $\dist{\tilde p}{\tilde u}{}=|u|$, 
\item $\dist{\tilde p}{\tilde v}{}=|v|$,
\end{itemize}
}
\begin{itemize}
\item $\tilde q$ lies on an extension of $[\tilde p\tilde v]$ so that $\tilde v$ is the midpoint of $[\tilde p\tilde q]$, 
\item $\tilde w$ is the midpoint between $\tilde u$ and ${\tilde v}$.
\end{itemize}
Note that 
\[\dist{\tilde p}{\tilde w}{}
=
\tfrac{1}{2}\cdot\sqrt{|u|^2+2\cdot\<u,v\>+|v|^2}.\eqlbl{eq:|p-w|=}\]

Since the geodesic space of directions $\Sigma'_p$ is dense in $\Sigma_p$,
we can assume that there are geodesics in the directions of $u$ and $v$.
Choose such geodesics $\gamma_u$ and $\gamma_v$ and assume that they are parametrized with speed $|u|$ and $|v|$ respectively.
For all small $t>0$, consider points $u_t,v_t,q_t,w_t\in \spc{A}$ such that
\begin{itemize}
\item $v_t=\gamma_v(t)$,\quad  $q_t=\gamma_v(2\cdot t)$
\item $u_t=\gamma_u(t)$.
\item $w_t$ is the midpoint of $[u_t v_t]$.
\end{itemize}
Clearly 
\[\dist{p}{u_t}{}=t\cdot |u|,\qquad \dist{p}{v_t}{}=t\cdot|v|,\qquad \dist{p}{q_t}{}=2\cdot t\cdot|v|.\] 
Since $\mangle(u,v)$ is defined, 
we have 
\[\dist{u_t}{v_t}{}=t\cdot\dist{\tilde u}{\tilde v}{}+o(t),
\qquad
\dist{u_t}{q_t}{}=t\cdot\dist{\tilde u}{\tilde q}{}+o(t).\]

From the point-on-side and hinge comparisons (\ref{point-on-side}$+$\ref{angle}), we have
\[\angk{v_t}p{w_t}
\ge
\angk{v_t}p{u_t}
\ge
\mangle\hinge{\tilde v}{\tilde p}{\tilde u}+\tfrac{o(t)}t\]
and
\[\angk{v_t}{q_t}{w_t}
\ge
\angk{v_t}{q_t}{u_t}
\ge
\mangle\hinge{\tilde v}{\tilde q}{\tilde u}+\tfrac{o(t)}t.\]
Clearly, 
$\mangle\hinge{\tilde v}{\tilde p}{\tilde u}+\mangle\hinge{\tilde v}{\tilde q}{\tilde u}=\pi$. 
From the adjacent angle comparison (\ref{2-sum}), 
$\angk{v_t}p{u_t}\z+\angk{v_t}{u_t}{q_t}\le \pi$.
Hence
$\angk{v_t}p{w_t}
\to
\mangle\hinge{\tilde v}{\tilde p}{\tilde w}$ as $t\to0+$
and thus 
\[\dist{p}{w_t}{}=t\cdot\dist{\tilde p}{\tilde w}{}+o(t).\]

Without loss of generality, we can assume that $f(p)=0$.
Since $f$ is concave, we have 
\begin{align*}
2\cdot f(w_t)&\ge f(u_t)+f(v_t)=
\\
&=t\cdot [(\dd_p f)(u)+(\dd_p f)(v)]+o(t).
\end{align*}
 
Applying concavity of $f$, we have
\begin{align*}
(\dd_p f)(\dir p{w_t})
&\ge 
\frac{f(w_t)}{\dist{p}{w_t}{}}
\ge 
\\
&\ge
\frac{t\cdot[(\dd_p f)(u)+(\dd_p f)(v)]+o(t)}{2\cdot t\cdot\dist{\tilde p}{\tilde w}{}+o(t)}.
\end{align*}
By \ref{eq:|p-w|=}, the key lemma follows.
\qeds

\parit{Proof of \ref{prop:grad-exist}; uniqueness.} 
If $g,g'\in \T_p$ are two gradients of $f$,
then 
\begin{align*}
\<g,g\>
&=(\dd_p f)(g)\le \<g,g'\>,
&
\<g',g'\>
&=(\dd_p f)(g')\le \<g,g'\>.
\end{align*}
Therefore,
\[\dist[2]{g}{g'}{}=\<g,g\>-2\cdot\<g,g'\>+\<g',g'\>\le0.\] 
It follows that $g=g'$.

\parit{Existence.} 
If $\dd_p f\le 0$, then one can take $\nabla_p f=\0$.

Suppose $s=\sup\set{(\dd_p f)(\xi)}{\xi\in\Sigma_p}>0$, 
it is sufficient to show that there is  $\overline{\xi}\in \Sigma_p$ such that 
\[
(\dd_p f)\left(\overline{\xi}\right)=s.
\eqlbl{overlinexi}
\]
Indeed, suppose $\overline{\xi}$ exists.
Applying \ref{lem:ohta} for $u=\overline{\xi}$, $v=\eps\cdot w$ with $\eps\to0+$, 
we get
\[(\dd_p f)(w)\le \<w,s\cdot\overline{\xi}\>\] 
for any $w\in\T_p$;
that is, $s\cdot\overline{\xi}$ is the gradient at $p$.

Take a sequence of directions $\xi_n\in \Sigma_p$, such that $(\dd_p f)(\xi_n)\to s$.
Applying \ref{lem:ohta} for $u=\xi_n$ and $v=\xi_m$, we get
\[s
\ge
\frac{(\dd_p f)(\xi_n)+(\dd_p f)(\xi_m)}{\sqrt{2+2\cdot\cos\mangle(\xi_n,\xi_m)}}.\]
Therefore $\mangle(\xi_n,\xi_m)\to0$ as $n,m\to\infty$;
that is, $\xi_1,\xi_2,\dots$ is a Cauchy sequence.
Clearly, $\overline{\xi}=\lim_n\xi_n$ meets \ref{overlinexi}.
\qeds

\begin{thm}{Exercise}\label{ex:convergence-grad}
Let $f$ and $g$ be locally Lipschitz semiconcave functions defined in a neighborhood of a point $p$ in an Alexandrov space.
Show that 
\[\dist[2]{\nabla_p f}{\nabla_p g}{\T_p}
\le 
s\cdot(|\nabla_p f|+|\nabla_p g|),\]
where
\[s=\sup\set{|(\dd_p f)(\xi)-(\dd_p g)(\xi)|}{\xi\in\Sigma_p}.\]

Conclude that if the sequence of restrictions $\dd_p f_n|_{\Sigma_p}$ converges uniformly, then $\nabla_pf_n$ converges as $n\to\infty$.
Here we assume that all functions $f_1$, $f_2,\dots$ are semiconcave and locally Lipschitz. 
\end{thm}

\begin{thm}{Exercise}\label{ex:semicontinuous-grad}
Let $f$ be a locally Lipschitz $\lambda$-concave function on an Alexandrov space $\spc{A}$.

\begin{subthm}{ex:semicontinuous-grad:>s}
Suppose $s\ge 0$.
Show that $|\nabla_xf|> s$ if and only if for some point $y$ we have
\[f(y)-f(x)>s\cdot \ell+\lambda\cdot \tfrac{\ell^2}2,\]
were $\ell=\dist{x}{y}{}$.
\end{subthm}

\begin{subthm}{ex:semicontinuous-grad:lim} Show that $x\mapsto|\nabla_xf|$ is lower semicontinuous;
that is,
\[|\nabla_{x_\infty}f|\le \liminf_{x_n\to x_\infty} |\nabla_{x_n}f|.\]

\end{subthm}

\end{thm}

%%!TEX root = the-gradient-flow.tex

\chapter{Gradient flow}\label{chap:GF}

\section{Velocity of curve}

Let $\alpha$ be a curve in an Alexandrov space $\spc{L}$.
If for any choice of 
geodesics $[p\,\alpha(t_0+\eps)]$ the vectors 
\[\tfrac{1}{\eps}\cdot\dist{p}{\alpha(t_0+\eps)}{}\cdot\dir p{\alpha(t_0+\eps)}\]
converge as $\eps\to 0+$, then their limit in $\T_p$ is called the \index{right derivative}\emph{right derivative} of $\alpha$ at $t_0$; it will be denoted by $\alpha^+(t_0)$.
In addition, $\alpha^+(t_0)\df0$
if $\tfrac{1}{\eps}\cdot\dist{p}{\alpha(t_0+\eps)}{}\to 0$ as $\eps\to 0+$.

The tangent vector $v=\dist px{}\cdot\dir px$ can be called \index{logarithm}\emph{logarithm} of $x$ at $p$ (briefly, \index{$v=\log_p x$ (logarithm)}$\log_p x$);
it is a tangent vector at $p$ of a geodesic path from $p$ to $x$.\label{page:log}


\begin{thm}{Claim}\label{clm:fa'=dfa'}
Let $\alpha$ be a curve in an Alexandrov space $\spc{L}$.
Suppose $f$ a semiconcave Lipschitz function
defined in a neighborhood of $p\z=\alpha(0)$,
and $\alpha^+(0)$ is defined.
Then 
\[(f\circ\alpha)^+(0)
=
(\dd_pf)(\alpha^+(0)).\]
\end{thm}

\parit{Proof.}
Without loss of generality, we can assume that $f(p)=0$.
Suppose $f$ and therefore $\dd_pf$ are $L$-Lipschitz.

Choose a constant-speed geodesic $\gamma$ that starts from $p$,
such that the distance
$s=\dist{\alpha^+(0)}{\gamma^+(0)}{\T_p}$
is small.
By the definition of differential,
\[(f\circ\gamma)^+(0)=\dd_pf(\gamma^+(0)).\]

By comparison and the definition of $\alpha^+$,
\[\dist{\alpha(\eps)}{\gamma(\eps)}{\spc{L}}\le s\cdot\eps+o(\eps)\]
for $\eps>0$.
Therefore,
\[|f\circ\alpha(\eps)-f\circ\gamma(\eps)|\le L\cdot s\cdot\eps+o(\eps).\]

Suppose $(f\circ\alpha)^+(0)$ is defined.
Then
\[|(f\circ\alpha)^+(0)-(f\circ\gamma)^+(0)|\le L\cdot s.\]
Since $\dd_pf$ is $L$-Lipschitz, we also get 
\[|\dd_pf(\alpha^+(0))-\dd_pf(\gamma^+(0))|\le L\cdot s.\]
It follows that the needed identity holds up to error $2\cdot L\cdot s$.
The statement follows since $s>0$ can be chosen arbitrarily.

The same argument is applicable if in the place of $(f\circ\alpha)^+(0)$
we use any limit of $\tfrac1{\eps_n}\cdot [f\circ\alpha(\eps_n)-f(p)]$ for a sequence $\eps_n\to 0+$.
It proves that all such limits coincide; in particular, $(f\circ\alpha)^+(0)$ is defined and equals to $(\dd_pf)(\alpha^+(0))$.
\qeds


\section{Gradient curves}

\begin{thm}{Definition}\label{def:grad-curve}
Let $f\:\spc{L}\subto\RR$ be a locally Lipschitz and semiconcave function on an Alexandrov space
$\spc{L}$.

A locally Lipschitz curve $\alpha\:[t_{\min},t_{\max})\to\Dom f$ will be called an \index{gradient curve}\emph{$f$-gradient curve} if
\[\alpha^+=\nabla_{\alpha} f;\]
that is, for any $t\in[t_{\min},t_{\max})$, $\alpha^+(t)$ is defined and 
$\alpha^+(t)=\nabla_{\alpha(t)} f$.
\end{thm}

A complete proof of the following theorem is given in \cite{alexander-kapovitch-petrunin2024}; 
it mimics the proof of the standard Picard theorem on the existence  and uniqueness of solutions of ordinary differential equations.
We omit the proof of existence as it is rather lengthy;
the uniqueness will be proved in the next section.


\begin{thm}{Picard theorem}\label{thm:glob-exist-grad-curv}
Let $f\:\spc{L}\subto \RR$ be a locally Lipschitz and $\lambda$-concave function on an Alexandrov space $\spc{L}$.
Then for any $p\in \Dom f$, there are unique $t_{\max}\in(0,\infty]$ and $f$-gradient curve $\alpha\:[0,t_{\max})\to \spc{L}$ with $\alpha(0)=p$ such that any sequence $t_n\to t_{\max}-$, the sequence $\alpha(t_n)$ does not have a limit point in $\Dom f$.
\end{thm}

Note that the theorem says that the future of a gradient curve is determined by its present, but it says nothing about its past.

Here is an example showing that the past is not determined by the present.
Consider the function $f\:x\mapsto -|x|$ on the real line $\RR$.
The tangent space $\T_x\RR$ can be identified with $\RR$.
Note that 
\[\nabla_xf=
\begin{cases}
1&\text{if}\quad x<0,
\\
0&\text{if}\quad x=0,
\\
-1&\text{if}\quad x>0.
\end{cases}
\]
So, the $f$-gradient curves go to the origin with unit speed and then stand there forever.
In particular, if $\alpha$ is an $f$-gradient curve that starts at $x$,
then $\alpha(t)=0$ for any $t\ge |x|$.

Here is a slightly more interesting example;
it shows that gradient curves can merge even in the region where $|\nabla f|\z\ne 0$. 


\begin{wrapfigure}[8]{r}{34 mm}
\vskip-0mm
\centering
\includegraphics{mppics/pic-1215}
\vskip0mm
\end{wrapfigure}

\begin{thm}{Example}
Consider the function $f\:(x,y)\mapsto-|x|-|y|$ on the $(x,y)$-plane.
Note that $f$ is concave;
its gradient field is sketched on the figure.

Let $\alpha$ be an $f$-gradient curve that starts at $(x,y)$ for $x>y>0$.
Then 
\[\alpha(t)=
\begin{cases}
(x-t,y-t) &\text{for}\quad 0\le t\le  x-y,
\\
(x-t,0) &\text{for}\quad x-y\le t\le  x,
\\
(0,0) &\text{for}\quad x\le t.
\end{cases}
\]

\end{thm}


\section{Distance estimates}

\begin{thm}{Observation}\label{eq:fist-var-inq+}
Let $\alpha$ be a gradient curve of a $\lambda$-concave function $f$ 
defined on an Alexandrov space.
Choose a point $p$; let $\ell(t)\df\distfun_p\circ\alpha(t)$ and $q=\alpha(t_0)$.
Then 
\[
\ell^+(t_0)\le -\left({f(p)}-{f(q)}-\tfrac\lambda2\cdot\ell^2(t_0)\right)/\ell(t_0)
\]
\end{thm}

\parit{Proof.}
Let $\gamma$ be the unit-speed parametrization of $[qp]$ from $q$ to $p$, so $q=\gamma(0)$.
Then 
\begin{align*}
\ell^+(t_0)&=(\dd_q\distfun_p)(\nabla_qf)\le
\\
&\le -\langle\dir qp,\nabla_qf\rangle \le
\\
&\le -\dd_qf(\dir qp)=
\\
&=-(f\circ\gamma)^+(0)\le
\\
&\le -\left({f(p)}-{f(q)}-\tfrac\lambda2\cdot\ell^2(t_0)\right)/\ell(t_0)
\end{align*}
In the above calculations we consequently applied
\ref{clm:fa'=dfa'},
\ref{ex:d(distfun)},
the definition of gradient,
the definition of differential,
and concavity of $t\z\mapsto f\circ\gamma(t)-\tfrac \lambda2\cdot {t^2}$.
\qeds

Note that the following estimate implies uniqueness in the Picard theorem (\ref{thm:glob-exist-grad-curv}).

\begin{thm}{First distance estimate}\label{thm:dist-est}
Let $f$ be a $\lambda$-concave locally Lipschitz function on an Alexandrov space $\spc{L}$.
Then
\[\dist{\alpha(t)}{\beta(t)}{}
\le 
e^{\lambda\cdot t}\cdot\dist[{{}}]{\alpha(0)}{\beta(0)}{}\]
for any $t\ge 0$ and any two $f$-gradient curves $\alpha$ and $\beta$.

Moreover, the statement holds for a locally Lipschitz $\lambda$-concave function defined in an open domain if there is a geodesic $[\alpha(t)\,\beta(t)]$ in $\Dom f$ for any~$t$.
\end{thm}

\parit{Proof.} 
Fix a choice of geodesic $[\alpha(t)\,\beta(t)]$ for each $t$.
Let $\ell(t)=\dist{\alpha(t)}{\beta(t)}{}$. 
Note that
\[\ell^+(t)
\le-
\<\dir{\alpha(t)}{\beta(t)},\nabla_{\alpha(t)}f\>-\<\dir{\beta(t)}{\alpha(t)},\nabla_{\beta(t)}f\>
\le
\lambda\cdot\ell(t).\]
Here one has to apply \ref{eq:fist-var-inq+} for distance to the midpoint $m$ of $[\alpha(t)\,\beta(t)]$, and then apply the triangle inequality.
Hence the result. 
\qeds



The following exercise describes a global geometric property of a gradient curve without direct reference to its function.
It uses the notion of \textit{self-contracting curves} introduced by Aris Daniilidis, Olivier Ley, and St\'ephane Sabourau \cite{daniilidis-ley-sabourau}.

\begin{thm}{Exercise}\label{ex:elf-contracting}
Let $f\:\spc{L}\subto\RR$ be a locally Lipschitz and concave function on an Alexandrov space
$\spc{L}$.
Then 
\[\dist{\alpha(t_1)}{\alpha(t_3)}{\spc{L}}\ge \dist{\alpha(t_2)}{\alpha(t_3)}{\spc{L}}.\]
for any $f$-gradient curve $\alpha$ and $t_1\le t_2\le t_3$.
\end{thm}

\begin{thm}{Exercise}\label{ex:mayer}
Let $f$ be a locally Lipschitz concave function defined on an Alexandrov space $\spc{L}$.
Suppose $\hat\alpha\:[0,\ell]\to\spc{L}$ is an arc-length reparametrization of an $f$-gradient curve.
Show that $(f\circ\hat\alpha)$ is concave.
\end{thm}




The following exercise implies that gradient curves for a uniformly converging sequence of $\lambda$-concave functions converge to the gradient curves of the limit function.

\begin{thm}{Exercise}\label{lem:fg-dist-est}
Let $f$ and $g$ be $\lambda$-concave locally Lipschitz functions on an Alexandrov space $\spc{L}$.
Suppose
$\alpha,\beta\:[0,t_{\max})\to \spc{L}$ are respectively $f$- and $g$-gradient curves.
Assume $|f-g|<\eps$; let $\ell\:t\mapsto\dist{\alpha(t)}{\beta(t)}{}$.
Show that
\[\ell^+\le \lambda\cdot\ell+\tfrac{2\cdot\eps}{\ell}.\]

Conclude that if $\alpha(0)=\beta(0)$ and $t_{\max}<\infty$, then
\[\dist{\alpha(t)}{\beta(t)}{}
\le
\Const\cdot\sqrt{\eps\cdot t}\]
for some constant $\Const=\Const(t_{\max},\lambda)$.
\end{thm}

\section{Gradient flow}

Let $\spc{L}$ be an Alexandrov space 
and $f$ be a locally Lipschitz semiconcave function defined on an open subset of $\spc{L}$.
If there is an $f$-gradient curve $\alpha$ such that $\alpha(0)=x$ and $\alpha(t)=y$,
then we will write 
\[\GF^t_f(x)=y.\]
The partially defined map $\GF^t_f$ from $\spc{L}$ to itself is called the \index{gradient flow}\emph{$f$-gradient flow} for time $t$.
Note that
\[\GF^{t_1+t_2}_f=\GF_f^{t_1}\circ\GF_f^{t_2}.\]
In other words, one may think that gradient flow is an action of the \textit{semigroup} $(\RR_{\ge0},+)$ on the space.
 
From the first distance estimate \ref{thm:dist-est}, 
it follows that for any $t\ge 0$, the domain of definition of $\GF^t_f$ is an open subset of $\spc{L}$.
In some cases, it is globally defined.
For example, if $f$ is a $\lambda$-concave function defined on the whole space $\spc{L}$, then $\GF^t_f(x)$ is defined for all $x\in \spc{L}$ and $t\ge0$;
see \cite[16.19]{alexander-kapovitch-petrunin2024}.

Now let us reformulate the statements about gradient curves obtained earlier using this new terminology.
From the first distance estimate, we have the following.

\begin{thm}{Proposition}\label{prop:GF-is-lip}
Let $\spc{L}$ be an Alexandrov space 
and $f\:\spc{L}\to \RR$ be a semiconcave function.
Then the map $x\mapsto\GF^t_f(x)$ is locally Lipschitz.

Moreover, if $f$ is $\lambda$-concave, then $\GF^t_f$ is $e^{\lambda\cdot t}$-Lipschitz.
\end{thm}

The next proposition follows from \ref{lem:fg-dist-est}.

\begin{thm}{Proposition}\label{grad-curve-conv}
Let $\spc{L}$ be an Alexandrov space.
Suppose $f_n\:\spc{L}\to\RR$ is a sequence of
$\lambda$-concave functions 
that converges to $f_\infty\:\spc{L}\to \RR$. 
Then for any $x\in \spc{L}$ and $t\ge 0$, we have
\[\GF_{f_n}^t(x)\to \GF_{f_\infty}^t(x)\]
as $n\to \infty$.
\end{thm}

%??? do we need GH-limit version???

\section{Gradient exponent}\label{gexp}

One of the technical difficulties in Alexandrov's geometry comes from
nonextendability of geodesics. 
In particular, the exponential map, $\exp_p\:\T_p\to \spc{L}$, if defined in the usual way, can
be undefined in an arbitrary small neighborhood of the origin. 

We construct its analog, the \index{gradient exponential map}\emph{gradient exponential map} 
\[\gexp_p\:\T_p\to\spc{L},\]
which essentially solves this problem. 
It shares many properties with the ordinary exponential map, and better in certain respects,
even in the Riemannian universe.

Let $\spc{L}$ be Alexandrov's space and $p\in \spc{L}$, consider the function $f\z=\distfun_p^2/2$.
Suppose $i_{\lambda}\:\lambda\cdot \spc{L}\to \spc{L}$ sends a point in the rescaled copy $\lambda\cdot\spc{L}$ to the corresponding point in $\spc{L}$.
Consider the one parameter family of maps
$$\Phi^t_{f}\circ i_{e^t}\:e^t{\cdot} \spc{L}\to \spc{L}$$
where $\Phi^t_{f}$ denotes gradient flow. 
Note that $(e^t{\cdot} \spc{L},p)\GHto (\T_p,o_p)$ as $t\to\infty$.
Let us define the \textit{gradient exponential map} as the limit
\[\gexp_p=\lim_{t\to\infty}\Phi^t_{f}\circ i_{e^t}.\]

\begin{thm}{Proposition}\label{prop:gexp}
Let $\spc{L}$ be a proper $\Alex0$ space.
Then for any $p\in \spc{L}$ the gradient exponent $\gexp_p\:\T_p\to\spc{L}$ is defined.
Moreover, $\gexp_p$ is a short map and 
\[\gexp_p(\gamma^+(0))=\gamma(1)\]
for any geodesic path $\gamma$ that starts at $p$.
\end{thm}

The last statement in the proposition says that it is appropriate to use term \textit{exponent} for $\gexp$.


\parit{Proof.} 
Note that $f''\le 1$.
By the first distance estimate, we have that $\Phi^t_{f}$ is an $e^t$-Lipschitz.
Therefore, the compositions $\Phi^t_{f}\circ i_{e^t}\:e^t{\cdot} \spc{L}\to \spc{L}$ are short. 
Hence a partial limit $\gexp_p\:\T_p
\spc{L}\to \spc{L}$ exists, and it is a short map.

Clearly for any partial limit we have
\[\Phi^t_f\circ\gexp_p(v)=\gexp_p(e^t\cdot v).\]
Since $\Phi^t$ is $e^t$-Lipschitz, it follows that $\gexp_p$ is uniquely
defined.
\qeds

\section{Remarks}

??? gradient exponent for $\kappa\ne 0$
and for nonproper.

The gradient exponential map $\gexp_p$  for a point $p$ a Riemannian manifold $(M,g)$ coincides with the Riemannian exponential map inside the cut locus of $p$ but \emph{is different } from the  Riemannian exponential outside it.

quasigeodesics

%%!TEX root = the-splitting.tex
\chapter{Line splitting}\label{chap:splitting}

\section{Busemann function}

A \index{half-line}\emph{half-line}
\footnote{\red V: we used half-line in the other book but I would still like to change this to "ray" which is the established term in literature. A: Both terms are used, since we use line is bit more  natuaral to say half-line --- but will agree to change it if you want it.}
is a distance-preserving map
from $\RR_{\ge0}=[0,\infty)$ 
to a metric space.
In other words, a half-line is a geodesic defined on the real half-line $\RR_{\ge0}$.

If $\gamma\:[0,\infty)\to \spc{X}$ is a half-line,
then the limit 
\[\bus_\gamma(x)=\lim_{t\to\infty}\dist{\gamma(t)}{x}{}- t\eqlbl{eq:def:busemann*}\]
is called the \index{Busemann function}\emph{Busemann function} of $\gamma$.

The Busemann function $\bus_\gamma$ mimics behavior of the distance function from the ideal point of $\gamma$.

\begin{thm}{Proposition}\label{prop:busemann}
For any half-line $\gamma$ in a metric space $\spc{X}$,
its Busemann function $\bus_\gamma\:\spc{X}\to \RR$ 
is defined.
Moreover, $\bus_\gamma$ is $1$-Lipschitz and $\bus_\gamma (\gamma(t))=-t$ for any $t$.

\end{thm}

\parit{Proof.}
Since $t=\dist{\gamma(0)}{\gamma(t)}{}$, the triangle inequality implies that, the function
\[t\mapsto\dist{\gamma(t)}{x}{}- t\] 
is nonincreasing, and 
\[\dist{\gamma(t)}{x}{}- t\ge-\dist{\gamma(0)}{x}{}\]
for any $x\in \spc{X}$.
Therefore, the limit in \ref{eq:def:busemann*} is defined,
and it is 1-Lipschitz as a limit of 1-Lipschitz functions.
The last statement follows since 
$\dist{\gamma(t)}{\gamma(t_0)}{}\z=t-t_0$ for all large~$t$.
\qeds

\begin{thm}{Exercise}\label{ex:busemann-CBB}
Any Busemann function on an $\Alex0$ space is concave.
\end{thm}

\section{Splitting theorem}

A \index{line}\emph{line} is a distance-preserving map
from $\RR$ to a metric space.
In other words, a line is a geodesic defined on the real line $\RR$.

\begin{thm}{Exercise}\label{ex:bus+bus}
Let $\gamma$ be a line in a metric space $\spc{X}$.
Show that for any point $x$ we have
\[\bus_+(x)+\bus_-(x)\ge 0\]
where, $\bus_+$ and $\bus_-$, are the Busemann functions asociated with half-lines $\gamma:[0,\infty)\to \spc{L}$ and $\gamma:(-\infty,0]\to \spc{L}$ respectively.
\end{thm}


Let $\spc{X}$ be a metric space and $A,B\subset \spc{X}$.
We will write 
\[\spc{X}=A\oplus B\]\index{$A\oplus B$}
if there are projections $\proj_A\:\spc{X}\to A$ 
and 
$\proj_B\:\spc{X}\to B$
such that 
\[\dist[2]{x}{y}{}=\dist[2]{\proj_A(x)}{\proj_A(y)}{}+\dist[2]{\proj_B(x)}{\proj_B(y)}{}\]
for any two points $x,y\in \spc{X}$.

Note that if 
\[\spc{X}=A\oplus B\]
then 
\begin{itemize}
\item $A$ intersects $B$ at a single point,
\item both sets $A$ and $B$ are \index{convex set}\emph{convex sets} in $\spc{X}$;
the latter means that any geodesic with the ends in $A$ (or $B$) lies in $A$ (or $B$). 
\end{itemize}

\begin{thm}{Line splitting theorem}\label{thm:splitting}
Let $\gamma$ be a line in a $\Alex0$ space~$\spc{L}$. 
Then 
\[\spc{L}=\spc{L}'\oplus \gamma(\RR)\]
for some subset $\spc{L}'\subset \spc{L}$.
\end{thm}

\begin{thm}{Corollary}\label{cor:splitting}
Any $\Alex0$ space $\spc{L}$ splits isometrically as
\[
\spc{L}=\spc{L}'\oplus H
\]
where $H\subset \spc{L}$ is a subset isometric to a Hilbert space, and $\spc{L}'\subset \spc{L}$ is a convex subset that contains no lines. 
\end{thm}

The following lemma is closely related to the first distance estimate (\ref{thm:dist-est});
it is also a limit case of \ref{prop:gexp}.
The proof goes along the same lines.

\begin{thm}{Lemma}\label{lem:dist-estimate}
Suppose $f\:\spc{L}\to\RR$ be a concave 1-Lipschitz function on an $\Alex0$ space $\spc{L}$.
Consider two $f$-gradient curves $\alpha$ and~$\beta$.
Then for any $t, s\ge 0$ we have
\begin{align*}
&\dist[2]{\alpha(s)}{\beta(t)}{}
\le 
\dist[2]{p}{q}{}+
2\cdot(f(p)-f(q))\cdot(s-t)+ (s-t)^2,
\end{align*}
where $p=\alpha(0)$ and $q=\beta(0)$.
\end{thm}

\parit{Proof.}
Since $f$ is 1-Lipschitz, $|\nabla f|\le1$.
Therefore 
\[f\circ\beta(t)\le f(q)+t\]
for any $t\ge0$.

Set $\ell(t)=\dist{p}{\beta(t)}{}$.
Applying \ref{eq:fist-var-inq+}, we get
\begin{align*}
(\ell^2)^+(t)
&\le 2\cdot \left(f\circ\beta(t)-f(p)\right)\le 
\\
&\le2\cdot\left(f(q)+t-f(p)\right).
\end{align*}
Therefore 
\[\ell^2(t)-\ell^2(0)\le 2\cdot\left(f(q)-f(p)\right)\cdot t + t^2.\]
It proves the needed inequality in case $s=0$.
Combining it with the first distance estimate (\ref{thm:dist-est}), we get the result in case $s\le t$.
The case $s\ge t$ follows by switching the roles of $s$ and $t$.
\qeds


\parit{Proof of \ref{thm:splitting}.} Consider two Busemann functions, $\bus_+$ and $\bus_-$, asociated with half-lines $\gamma:[0,\infty)\to \spc{L}$ and $\gamma:(-\infty,0]\to \spc{L}$ respectively; that is,
\[
\bus_\pm(x)
\df
\lim_{t\to\infty}\dist{\gamma(\pm t)}{x}{}- t.
\]
According to \ref{ex:busemann-CBB}, 
both $\bus_+$ and $\bus_-$ are concave.

By \ref{ex:bus+bus}, $\bus_+(x)+\bus_-(x)\ge0$ for any $x\in \spc{L}$.
On the other hand, by \ref{comp-kappa}, 
$f(t)=\distfun_x^2\circ\gamma(t)$ 
is $2$-concave.
In particular, $f(t)\le t^2+at+b$ for some constants $a,b\in\RR$.  Therefore, for all large $t$
\[
\dist{\gamma( t)}{x}{}- t +\dist{\gamma(- t)}{x}{}- t\le \sqrt{ t^2+at+b}-t+\sqrt{ t^2-at+b}-t
\]

Passing to the limit as $t\to\infty$, we get that  $\bus_+(x)+\bus_-(x)\le 0$.
Hence
\[
\bus_+(x)+\bus_-(x)= 0
\]
for any $x\in \spc{L}$.
In particular, the functions $\bus_+$ and $\bus_-$ are \index{affine function}\emph{affine};
that is, they are convex and concave at the same time.

Note that for any $x$,
\begin{align*}
|\nabla_x \bus_\pm|
&=\sup\set{\dd_x\bus_\pm(\xi)}{\xi\in\Sigma_x}=
\\
&=\sup\set{-\dd_x\bus_\mp(\xi)}{\xi\in\Sigma_x}\equiv
\\
&\equiv1.
\end{align*}

Observe that $\alpha$ is a $\bus_\pm$-gradient curve
if and only if $\alpha$ is a geodesic such that $(\bus_\pm\circ\alpha)^+=1$.
Indeed, if $\alpha$ is a geodesic, then $(\bus_\pm\circ\alpha)^+\le 1$ and the equality holds only if $\nabla_\alpha\bus_\pm=\alpha^+$.
Now suppose $\nabla_\alpha\bus_\pm=\alpha^+$.
Then $|\alpha^+|\le 1$ and $(\bus_\pm\circ\alpha)^+=1$; therefore 
\begin{align*}
|t_0-t_1|&\ge \dist{\alpha(t_0)}{\alpha(t_1)}{}\ge
\\
&\ge|\bus_\pm\circ\alpha(t_0)-\bus_\pm\circ\alpha(t_1)=
\\
&=|t_0-t_1|.
\end{align*}

It follows that for any $t>0$, the $\bus_\pm$-gradient flows commute;
that is, 
\[\GF_{\bus_+}^t\circ\GF_{\bus_-}^t=\id_\spc{L}.\]
Setting
\[\GF^t=\left[\begin{matrix}
\GF_{\bus_+}^t&\hbox{if}\ t\ge0\\
\GF_{\bus_-}^{-t}&\hbox{if}\ t\le0
               \end{matrix}\right.\]
defines an $\RR$-action on $\spc{L}$.

Consider the level set $\spc{L}'=\bus_+^{-1}(0)=\bus_-^{-1}(0)$;
it is a closed convex subset of $\spc{L}$, and therefore forms an Alexandrov space.
Consider the map $h\:\spc{L}'\times \RR\to \spc{L}$ defined by $h\:(x,t)\mapsto \GF^t(x)$.
Note that $h$ is onto.
Applying \ref{lem:dist-estimate} for $\GF_{\bus_+}^t$ and $\GF_{\bus_-}^t$ shows that $h$ is distance non-expanding and non-contracting at the same time; that is, $h$ is an isometry.
\qeds

Recall that according our definition the real line $\RR$ is $\Alex1$.
However, most of $\Alex1$ spaces have diameter at most $\pi$;
see \ref{ex:RisCBB(1)}.

\begin{thm}{Exercise}\label{ex:cone-CBB}
Suppose $\spc{X}$ is a complete geodesic space.
Show that $\Cone\spc{X}$ is $\Alex0$ if and only if $\spc{X}$ is $\Alex1$ and $\diam\spc{X}\le \pi$.
\end{thm}

\section{Anti-sum}

Here we give a corollary of \ref{ex:convergence-grad}.
It will be used to prove basic properties of the tangent space.


\begin{thm}{Anti-sum lemma}\label{lem:minus-sum} 
Let $\spc{L}$ be an Alexandrov space and $p\in \spc{L}$.

Given two vectors $u,v\in \T_p$, there is a unique vector $w\in \T_p$ such that
\[\langle u,x\rangle +\langle v,x\rangle +\langle w,x\rangle \ge 0\]
for any $x\in \T_p$, and
\[\langle u,w\rangle +\langle v,w\rangle +\langle w,w\rangle =0.\]

\end{thm}

\begin{thm}{Exercise}\label{ex:|antisum|}
Suppose $u,v, w\in \T_p$ are as in \ref{lem:minus-sum}.
Show that 
\[|w|^2\le |u|^2+|v|^2+2\cdot\langle u,v\rangle.\]

\end{thm}

If $\T_p$ were geodesic, then the lemma would follow from the existence  of the gradient, applied to the function $\T_p\to \RR$ defined by $x\mapsto -(\langle u,x\rangle +\langle v,x\rangle )$ which is concave.
However, the tangent space $\T_p$ might fail to be geodesic; see  Halbeisen's example \cite{alexander-kapovitch-petrunin2024}.

Applying the above lemma for $u=v$, we have the following statement.

\begin{thm}{Existence of polar vector}\label{cor:polar}
Let $\spc{L}$ be an Alexandrov space 
and $p\in \spc{L}$. 
Given a vector $u\in \T_p$,  there is a unique vector $u^*\in\T_p$ such that $\langle u^*,u^*\rangle +\langle u,u^*\rangle = 0$ and
$u^*$ is \index{polar vectors}\emph{polar} to $u$;
that is,
\[\langle u^*,x\rangle +\langle u,x\rangle \ge 0\]
for any $x\in \T_p$.
\end{thm}

\parit{Proof of \ref{lem:minus-sum}.}
By \ref{ex:d(distfun):==}, we can choose two sequences of points $a_n,b_n$ such that 
\begin{align*}
\dd_p\distfun_{a_n}(w)&=-\langle\dir{p}{a_n},w\rangle
\\
\dd_p\distfun_{b_n}(w)&=-\langle\dir{p}{b_n},w\rangle
\end{align*}
for any $w\in\T_p$ and $\dir{p}{a_n}\to u/|u|$, $\dir{p}{b_n}\to v/|v|$ as $n\to \infty$

Consider a sequence of functions 
\[f_n=|u|\cdot\distfun_{a_n}+|v|\cdot\distfun_{b_n}.\]
Note that 
\[(\dd_pf_n)(x)=-|u|\cdot\langle \dir{p}{a_n},x\rangle -|v|\cdot\langle \dir{p}{b_n},x\rangle .\]
Thus we have the following uniform convergence for $x\in\Sigma_p$:
\[(\dd_pf_n)(x)\to-\langle u,x\rangle -\langle v,x\rangle \]
as $n\to\infty$,
According to \ref{ex:convergence-grad}, 
the sequence $\nabla_pf_n$ converges.
Let 
\[w=\lim_{n\to\infty}\nabla_pf_n.\]
By the definition of gradient,
\[\begin{aligned}
\langle w,w\rangle &=\lim_{n\to\infty}\langle \nabla_pf_n,\nabla_pf_n\rangle =
&&&%right side
\langle w,x\rangle &=\lim_{n\to\infty}\langle \nabla_pf_n,x\rangle \ge
\\%second line
&=\lim_{n\to\infty}(\dd_p f_n)(\nabla_p f_n)
=
&&&%second line right side
&\ge
\lim_{n\to\infty}(\dd_pf_n)(x)
=
\\%line 3
&=-\langle u,w\rangle -\langle v,w\rangle ,
&&&%line 3 right side
&=-\langle u,x\rangle -\langle v,x\rangle .
\end{aligned}\]
\qedsf

\section{Linear subspace}

\begin{thm}{Definition}\label{def:opp+Lin}
Let $\spc{L}$ be an Alexandrov space, $p\in \spc{L}$ and $u,v\in\T_p$.
We say that vectors $u$ and $v$ are \index{opposite vectors}\emph{opposite}\label{def:opposite:page} to each other, (briefly, $u+v=0$) if $|u|=|v|=0$ or $\mangle(u,v)=\pi$ and $|u|=|v|$.

The subcone
\[\Lin_p=\set{v\in\T_p}{\exists\ w\in\T_p\quad \text{such that}\quad w+v=0}\]
will be called the \index{linear subspace}\emph{linear subspace} of $\T_p$.
\end{thm}

Soon we will introduce a natural linear structure on $\Lin_p$.

\begin{thm}{Proposition}\label{prop:opposite}
Let $\spc{L}$ be an Alexandrov space and $p\in \spc{L}$.
Given two vectors $u,v\in\T_p$, the following statements are equivalent:
\begin{subthm}{opposite} $u+v=0$;
\end{subthm}
\begin{subthm}{<x,u>} $\langle u,x\rangle +\langle v,x\rangle =0$ for any $x\in\T_p$;
\end{subthm}
\begin{subthm}{<xi,u>} $\langle u,\xi\rangle +\langle v,\xi\rangle =0$ for any $\xi\in\Sigma_p$.
\end{subthm}
\end{thm}

\parit{Proof.}
The equivalence  \ref{SHORT.<x,u>}$\Leftrightarrow$\ref{SHORT.<xi,u>} is trivial.

The condition $u+v=0$ is equivalent to 
$\langle u,u\rangle =-\langle u,v\rangle =\langle v,v\rangle$;
thus,
\ref{SHORT.<x,u>}$\Rightarrow$\ref{SHORT.opposite}.

Recall that $\T_p$ has nonnegative curvature.
Note that the hinges $\hinge 0ux$ and $\hinge 0vx$ are adjacent.
By \ref{ex:adjacent-CBB}, $\mangle\hinge 0ux+\mangle\hinge 0vx=\pi$;
hence \ref{SHORT.opposite}$\Rightarrow$\ref{SHORT.<x,u>}.
\qeds

\begin{thm}{Exercise}\label{prop:two-opp}
Let $\spc{L}$  be an Alexandrov space and $p\in \spc{L}$.
Then for any three vectors $u,v,w\in\T_p$, if $u+v=0$ and $u+ w=0$ then $v=w$.
\end{thm}

Let $u\in \Lin_p$; that is, $u+v=0$ for some $v\in\T_p$.
Given $s<0$, let 
\[s\cdot u\df (-s)\cdot v.\]
So we can multiply any vector in $\Lin_p$ by any real number (positive and negative).
By \ref{prop:two-opp}, this multiplication is uniquely defined;
by \ref{prop:opposite}; we have identity
\[\langle -v,x\rangle=-\langle v,x\rangle.\]


\begin{thm}{Exercise}\label{ex:3<,>=0}
Suppose $u,v,w\in\T_p$ are as in \ref{lem:minus-sum}.
Show that
\[\langle u,x\rangle +\langle v,x\rangle +\langle w,x\rangle = 0\]
for any $x\in \Lin_p$.
\end{thm}

\begin{thm}{Exercise}\label{ex:-u}
Let $\spc{L}$ be an Alexandrov space,
$p\in \spc{L}$ and $u\in \T_p$.
Suppose $u^*\in \T_p$ is provided by \ref{cor:polar};
that is, 
\[\langle u^*,u^*\rangle +\langle u,u^*\rangle = 0
\quad\text{and}\quad
\langle u^*,x\rangle +\langle u,x\rangle \ge 0
\]
for any $x\in \T_p$.
Show that 
\[u=-u^*\quad\Longleftrightarrow\quad|u|=|u^*|.\]
\end{thm}

\begin{thm}{Theorem}\label{thm:lin-subcone}
Let $p$ be a point in an Alexandrov space. 
Then $\Lin_p$ is isometric to a Hilbert space.
\end{thm}

\parit{Proof.}
Note that $\Lin_p$ is a closed subset of $\T_p$;
in particular, it is complete.

If any two vectors in $\Lin_p$ can be connected by a geodesic in $\Lin_p$,
then the statement follows from the splitting theorem (\ref{thm:splitting}).
By Menger's lemma (\ref{lem:mid>geod}), it is sufficient to show that for any two vectors $x,y\in\Lin_p$
there is a midpoint $w\in \Lin_p$.

Choose $w\in \T_p$ to be the anti-sum of $u=-\tfrac{1}{2}\cdot x$ and $v=-\tfrac{1}{2}\cdot y$;
see \ref{lem:minus-sum}.
By \ref{ex:|antisum|} and \ref{ex:3<,>=0},
\begin{align*}
|w|^2&\le \tfrac14\cdot |x|^2+\tfrac14\cdot|y|^2+\tfrac12\cdot\langle x,y\rangle,
\\
\langle w,x\rangle&= \tfrac12\cdot|x|^2+\tfrac12\cdot\langle x,y\rangle,
\\
\langle w,y\rangle&= \tfrac12\cdot|y|^2+\tfrac12\cdot\langle x,y\rangle,
\end{align*}
It follows that 
\begin{align*}
|x-w|^2
&= |x|^2+|w|^2-2\cdot\langle w,x\rangle\le
\\
&\le \tfrac14\cdot |x|^2+\tfrac14\cdot|y|^2-\tfrac12\cdot\langle x,y\rangle=
\\
&=\tfrac14\cdot|x-y|^2.
\end{align*}
That is, $|x-w|\le \tfrac12\cdot|x-y|$.
Similarly, we get $|y-w|\le \tfrac12\cdot|x-y|$.
Therefore $w$ is a midpoint of $x$ and $y$.
In addition we get equality 
\[|w|^2= \tfrac14\cdot |x|^2+\tfrac14\cdot|y|^2+\tfrac12\cdot\langle x,y\rangle.\]

It remains to show that $w\in\Lin_p$.
Let $w^*$ be the polar vector provided by \ref{cor:polar}.
Note that 
\[|w^*|\le |w|,
\quad
\langle w^*,x\rangle+\langle w,x\rangle=0,
\quad
\langle w^*,y\rangle+\langle w,y\rangle=0.
\]
The same calculation as above shows that $w^*$ is a midpoint of $-x$ and $-y$ and 
\[|w^*|^2= \tfrac14\cdot |x|^2+\tfrac14\cdot|y|^2+\tfrac12\cdot\langle x,y\rangle=|w|^2.\]
By \ref{ex:-u}, $w=-w^*$;
hence $w\in\Lin_p$.
\qeds

\begin{thm}{Lemma}\label{ex:grad-dist:G-delta}
Given a point $p$ in an Alexandrov space $\spc{L}$,
let $f\z=\distfun_p$, and let $S$ be the subset of points $x\in\spc{L}$ such that $|\nabla_xf|=1$.
Then $S$ is a dense G-delta set.

\end{thm}

\parit{Proof.}
Let $S_n\subset \spc{L}$ be defined by inequality $|\nabla_xf|>1-\tfrac1n$.
By \ref{ex:semicontinuous-grad:>s}, $S_n$ is open.

Choose a point $q\ne p$.
Observe that $|\nabla_xf|=1$ for any point $x\in\left]pq\right[$.
It follows that $S_n$ is dense in $\spc{L}$.

Since $S=\bigcap_nS_n$, the lemma follows.
\qeds


\begin{thm}{Exercise}\label{ex:grad-dist}
Let $p$, $f$, and $S$ be as in \ref{ex:grad-dist:G-delta}.

\begin{subthm}{ex:grad-dist:lin}
Show that 
\[\nabla_xf+\dir xp=0\]
for any 
$x\in S$;
in particular, $\dir xp\in \Lin_x$.
\end{subthm}

\begin{subthm}{ex:grad-dist:|grad|=1}
Show that if $|\nabla_xf|=1$, then $\dd_xf(w)= \langle\nabla_xf,w\rangle$ for any $w\in \T_x$.
\end{subthm}
\end{thm}

Note that \ref{ex:grad-dist} implies the following.

\begin{thm}{Corollary}\label{cor:euclid-subcone}
Given a countable set of points $X$ in an Alexandrov space $\spc{L}$
there is a G-delta dense set $S\subset\spc{L}$
such that 
$\dir sx\in \Lin_s$
for any $s\in S$ and $x\in X$.
\end{thm}

\section{Comments}

The splitting theorem has an interesting history that starts with Stefan Cohn-Vossen \cite{cohn-vossen_line};
who proved its $2$-dimensional case.
For Riemannian manifolds of higher dimensions 
it was proved by Victor Toponogov \cite{toponogov-globalization+splitting}.
Then it was generalized by Anatoliy Milka \cite{milka-line}
to Alexandrov spaces;
historically, it was the first result about Alexandrov spaces of dimension higher than 2.
Nearly the same proof is used in \cite[1.5]{burago-burago-ivanov}.

Further generalizations of the splitting theorem for Riemannian manifolds with nonnegative Ricci curvature were obtained by Jeff Cheeger and Detlef Gromoll \cite{cheeger-gromoll-split}.
This was further generalized by Jeff Cheeger and Toby Colding for limits of Riemannian manifolds with almost nonnegative Ricci curvature \cite{cheeger-colding-alm-rigidity} and to their synthetic generalizations, so-called {}\emph{RCD spaces}, by Nicola Gigli \cite{gigli2013splitting, gigli-splitting-overview}.
Jost-Hinrich Eschenburg obtained an analogous result for  Lorentzian manifolds \cite{eshenburg-split}, that is, pseudo-Riemannian manifolds of signature $(1,n)$.

The presented proof is close in spirit to the proof given by Cheeger and Gromoll \cite{cheeger-gromoll-split};
it is taken from our book \cite{alexander-kapovitch-petrunin2024}.

\begin{thm}{Open question}
Let $p$ be a point in an Alexandrov space $\spc{L}$.
Suppose that $0\ne v\in \Lin_p$.
Is it true that the tangent space $\T_p$ splits in the direction of $v$?
\end{thm}

Halbeisen's example \cite{alexander-kapovitch-petrunin2024,halbeisen} shows that compactness of space of directions is essential in the proof that space of directions is $\pi$-geodesic (see \ref{thm:finite-space-of-directions}).

\begin{thm}{Open question}\label{open:Halb-proper}
Let $\spc{L}$ be a proper Alexandrov space.
Is it true that for any $p\in \spc{L}$, the tangent space $\T_p$ is a length space?
\end{thm}

%%%%%%%%%%%%%%%%%%%%%%%%%%%%%%%%%%%%%%%%%%%%%%%%%%

\chapter{Dimension and volume}\label{chap:dim}

\section{Linear dimension}

Let $\spc{L}$ be an Alexandrov space.
Let us define its \index{linear dimension}\emph{linear dimension} \index{$\LinDim \spc{L}$}$\LinDim \spc{L}$ as the least upper bound on integers $m$ such that 
the Euclidean space $\EE^m$ is isometric to a subspace of the tangent space $\T_p\spc{L}$ at some point $p\in \spc{L}$.
If not stated otherwise, dimension of an Alexandrov space is its linear dimension.

If not stated otherwise, dimension will mean linear dimension.
In Section~\ref{sec:all-dim}, we will show that linear dimension of Alexandrov space coincides with all reasonable dimensions;
after that we will use \index{$\dim \spc{L}$}$\dim$ for $\LinDim$.

\begin{thm}{(\textit{n}+1)-comparison}
Let $\spc{L}$ be an $\Alex0$ space.
Then for any finite set of points $p,x_1,\dots,x_n\in \spc{L}$, there is a model configuration 
$\tilde p,\tilde x_1,\dots,\tilde x_n\in \EE^m$ such that 
\[|\tilde p-\tilde x_i|_{\EE^m}=| p- x_i|_{\spc{L}}
\quad\text{and}\quad
|\tilde x_i-\tilde x_j|_{\EE^m}\ge |x_i- x_j|_{\spc{L}}\]
for any $i$ and $j$.
Moreover, we can assume that $m\le \LinDim\spc{L}$. 
\end{thm}

\parit{Proof.}
By \ref{cor:euclid-subcone}, we can choose a point $p'$ arbitrarily close to $p$ so that 
$\Lin_{p'}\ni \dir{p'}{x_i}$ for any $i$.
Let us identify $\EE^m$ with a subspace of $\Lin_{p'}$ spanned by $\dir{p'}{x_1},\dots,\dir{p'}{x_n}$.
Note that $m\le \LinDim\spc{L}$.

Set $\tilde p'=0\in \EE^m$ and $\tilde x_i=\dist{p'}{x_n}{}\cdot\dir{p'}{x_n}\in \EE^m$ for every $i$.
Note that 
\[|\tilde p'-\tilde x_i|_{\EE^m}=| p'- x_i|_{\spc{L}}\]
for every $i$.
Applying the comparison $\mangle\hinge {p'}{x_i}{x_j}\ge \angk {p'}{x_i}{x_j}$, we get
\[|\tilde x_i-\tilde x_j|_{\EE^m}\ge |x_i- x_j|_{\spc{L}}\]
for any $i$ and $j$.
Passing to a limit configuration as $p'\to p$ we get the result.
\qeds

\begin{thm}{Exercise}\label{ex:tangent=Em}
Let $\spc{L}$ be an $\Alex0$ space.
Suppose $\LinDim\spc{L}\z=m<\infty$.
Show that $\T_p\spc{L}\iso \EE^m$ for a G-delta dense set of points $p\in\spc{L}$.
\end{thm}

\begin{thm}{Exercise}\label{ex:dim=1}
Show that a 1-dimensional Alexandrov space is homeomorphic to a 1-dimensional manifold, possibly with nonempty boundary.
\end{thm}


\begin{thm}{Exercise}\label{ex:resporka}
Let $\spc{L}$ be an $\Alex0$ space.

Show that $\LinDim \spc{L}\ge m$ if and only if for some $m+2$ points $p$, $a_0,\z\dots, a_{m}\in \spc{L}$
we have
\[\angk p{a_i}{a_j}>\tfrac\pi2\]
for any pair $i\ne j$.
\end{thm}

\section{Space of directions}

A metric space $\spc{X}$ will be called $\ell$-geodesic 
if any two points $x,y\in\spc{X}$ such that $\dist{x}{y}{}<\ell$ can be connected by a geodesic.
For instance, any geodesic space is $\infty$-geodesic.

\begin{thm}{Theorem}\label{thm:finite-space-of-directions}
Let $\spc{L}$ be a finite-dimensional Alexandrov space.
Then for any point $p\in \spc{L}$, its space of directions $\Sigma_p$ is a compact $\pi$-geodesic space.
\end{thm}


\begin{thm}{Exercise}\label{ex:finite-tan}
Let $p$ be a point in a finite-dimensional Alexandrov space $\spc{L}$.
Prove the following.
\begin{subthm}{ex:finite-tan:tan}
The tangent space $\T_p$ is a proper $\Alex0$ space.
\end{subthm}

\begin{subthm}{ex:finite-space-of-directions-dim}
$\LinDim\Sigma_p=\LinDim\spc{L}-1$.
\end{subthm}

\begin{subthm}{ex:finite-tan:sigma}
If $\LinDim \spc{L}>1$, then $\Sigma_p$ is geodesic.
\end{subthm}


\end{thm}

Using \ref{ex:finite-space-of-directions-dim}, one can prove results for all finite dimensional Alexandrov spaces via induction on  dimension.
Such proofs will be indicated below.

\parit{Proof of \ref{thm:finite-space-of-directions}.}
Choose $\eps>0$; suppose $\spc{L}$ is $m$-dimensional.
Assume can choose $n$ directions $\xi_1,\dots, \xi_n\in \Sigma_p$ such that $\mangle(\xi_i,\xi_j)\z>\eps$ for any $i\ne j$.
Without loss of generality, we may assume that each direction is geodesic;
that is, there is a point $x_i\in \spc{L}$ such that $\xi_i=\dir p{x_i}$.

Choose $y_i\in [px_i]$ such that $\dist{p}{y_i}{}=r$ for each $i$ and small fixed $r>0$.
Since $r$ is small, we can assume that $\angk p{y_i}{y_j}>\eps$ for any $i\ne j$.
By \ref{cor:euclid-subcone}, we can choose $p'$ arbitrarily close to $p$ such that $\dir{p'}{y_i}\in \Lin_{p'}$ for any $i$.
Since  $\dist{p'}{p}{}$ is small, $\angk {p'}{y_i}{y_j}>\eps$ for any $i\ne j$.
By comparison, 
\[\mangle \hinge{p'}{y_i}{y_j}>\eps.\]
Therefore $n\le \pack_\eps\SSS^{m-1}$,
where \index{$\pack_\eps\spc{X}$}$\pack_\eps\spc{X}$ is the exact upper bound on the number of points $x_1,\z\dots,x_k\in \spc{X}$ such that $\dist{x_i}{x_j}{}\ge\eps$ if $i\ne j$.

Since $\SSS^{m-1}$ is compact, $\pack_\eps\SSS^{m-1}<\infty$.
By the definition, the space of directions $\Sigma_p$ is complete. 
Applying \ref{ex:pack-net}, we get that  $\Sigma_p$ is compact.

It remains to prove the following claim.

\begin{clm}{}
If $\Sigma_p$ is compact, then it is $\pi$-geodesic
\end{clm}

Choose two geodesic directions $\xi=\dir px$ and $\zeta=\dir py$;
let 
\[\alpha\z=\tfrac12\cdot \mangle \hinge pxy=\tfrac12\cdot \dist{\xi}{\zeta}{\Sigma_p}.\]

Suppose $\alpha<\pi/2$.
Let us show that it is sufficient to construct an \index{almost midpoint}\emph{almost midpoint} $\mu\z=\dir pz$ of $\xi$ and $\zeta$ in $\Sigma_p$;
that is, we need to show that for any $\eps>0$ there is a geodesic $[pz]$ such that
\[\mangle\hinge pxz\le \alpha+\eps
\quad\text{and}\quad
\mangle\hinge pyz\le \alpha+\eps.\]
Indeed, once it is done, the compactness of $\Sigma_p$ can be used to get an actual midpoint for any two directions in $\Sigma_p$.
After that Menger's lemma (\ref{lem:mid>geod}) will finish the proof.

Choose a sequence of small positive numbers $r_n\to0$
Consider sequnces $x_n\z\in [px]$ and $y_n\z\in [py]$ such that $\dist{p}{x_n}{}=\dist{p}{y_n}{}=r_n$.
Let $m_n$ be a midpoint of $[x_n\,y_n]$.
%??? we use here that the directions $\xi=\dir px$ and $\zeta=\dir py$ are not opposite???

Since $\Sigma_p$ is compact, we can pass to a sequence of $r_n$ such that 
$\dir{p}{m_n}$ converges;
denote its limit by $\mu$.
Choose a geodesic $[pz]$ that runs at small angle from $\mu$.
Let us show that $\dir pz$ is the needed almost midpoint.

Evidently, $\angk p{x_n}{m_n}=\angk p{y_n}{m_n}$.
By \ref{ex:alex-lemma-cat}, we have
\[\angk p{x_n}{m_n}+\angk p{y_n}{m_n}\le \angk p{x_n}{y_n}.\]

Let $z_n\in [pz]$ be the point such that $\dist{p}{z_n}{}=\dist{p}{m_n}{}$.
By construction, for all large $n$, we have $\mangle\hinge pz{m_n}\approx0$  with arbitrary small given error.
By comparison, the value $\frac{\dist{z_n}{m_n}{}}{\dist{p}{z_n}{}}$ can be assumed to be arbitrary small for all large $n$.
Applying this observation and the definition of angle measure, we also have the following approximations
\begin{align*}
\angk p{x_n}{y_n}&\approx \mangle\hinge p{x_n}{y_n},
\\
\angk p{x_n}{m_n}\approx\angk p{x_n}{z_n}&\approx\mangle\hinge p{x_n}{z_n},
\\
\angk p{m_n}{y_n}\approx\angk p{z_n}{y_n}&\approx\mangle\hinge p{z_n}{y_n},
\end{align*}
again, with arbitrary given error and all large $n$.
It follows that $\dir pz$ is an almost midpoint of $\dir px$ and $\dir py$, as required.
\qeds

In the above proof, the angles $\mangle\hinge pxz$ and $\mangle\hinge pyz$ have lower bounds by 
the comparison, but we needed upper bounds that were extracted from the definition of angle measure and compactness of space of directions.

\section{Right-inverse theorem}

\begin{thm}{Theorem}\label{thm:right-inverse}
Suppose $p,a_0,\dots,a_m$ be points in an Alexandrov space $\spc{L}$ such
\[\angk p{a_i}{a_j}>\tfrac\pi2\]
for any $i\ne j$.
Then the map $f\:\spc{L}\to\RR^m$ defined by
\[f\:x\mapsto (\dist{a_1}{x}{},\dots,\dist{a_m}{x}{})\]
has a left inverse defined in a neighborhood of $f(p)$.
\end{thm}

In the proof we construct a local right inverse $\map$ of $f$ around $f(p)$.
The construction uses gradient flow for suitably chosen family of functions.
The structure of the proof can be seen in the following exercise,
more details are given in the hints.

\begin{thm}{Exercise}\label{ex:proof-right-inverse}
Suppose $p,a_0,\dots,a_m\in\spc{L}$ and $f\:\spc{L}\to\RR$ are as in \ref{thm:right-inverse}.
Assume $\eps>0$ is sufficiently small.
Given $\bm{y}=(y_1,y_2,\dots,y_m)\in \RR^m$, 
consider the function on $\spc{L}$ defined by
\[f_{\bm{y}}(x)=\min\{\,0, \dist{a_1}{x}{}-y_1,\dots,\dist{a_m}{x}{}-y_m\,\}+\eps\cdot\dist{a_0}{x}{}.\]

\begin{subthm}{ex:proof-right-inverse:grad}
There is $r>0$ such that 
Show that $f_{\bm{y}}$ is $\lambda$-concave in $\oBall(p,r)$ for some $\lambda$ and
\begin{enumerate}[(i)]
\item\label{111} $(\dd_x\distfun_{a_i})(\nabla_x f_{\bm{y}})<-\tfrac{1}{10}\cdot\eps^2$ if $\dist{a_i}{x}{}>y_i$ and
\item\label{222} $(\dd_x\distfun_{a_i})(\nabla_x f_{\bm{y}})>\tfrac{1}{10}\cdot\eps^2$ if 
\[\dist{a_i}{x}{}-y_i=\min_j\{\dist{a_j}{x}{}\z-y_j\}<0.\]
\end{enumerate}
for any $x\in \oBall(p,r)$.

\end{subthm}

\begin{subthm}{ex:proof-right-inverse:alpha}
Let $\alpha_{\bm{y}}$ be $f_{\bm{y}}$-gradient curve that starts at $p$.
Use \ref{SHORT.ex:proof-right-inverse:grad} to show that 
if for some $\bm{y}\in\RR^m$ and $t_0\le\tfrac{r}{2}$ we have
$|\distfun_{\bm{a}}{p}-\bm{y}|
\le
\tfrac{\eps^2}{10}\cdot t_0$, then 
$
\distfun_{\bm{a}}{[\alpha_{\bm{y}}(t_0)]}
= 
\bm{y}$.
\end{subthm}

\begin{subthm}{ex:proof-right-inverse:end}
Let $t_0(\bm{y})=\tfrac{10}{\eps^2}\cdot|\dist{\bm{a}}{p}{}-\bm{y}|$.
Use \ref{lem:fg-dist-est} to show that the map
\[\map\:{\bm{y}}\mapsto \alpha_{\bm{y}}\circ t_0(\bm{y})\]
continuous in $\Omega=\oBall(\dist{\bm{a}}{p}{},\tfrac{\eps^2\cdot r}{20} )\subset\RR^m$
and $f\circ \Phi(\bm{y})=\bm{y}$ for any $\bm{y}\in \Omega$.
This finishes the proof of \ref{thm:right-inverse}.
\end{subthm}

\end{thm}

%??? I think that since this is used later it should be proved and not left as a reference A: If we add a solution, then that is OK, is not it? in any case, the idea is more tranparent in the exercise and if needed one can read the solution. But lets do the real solution, not just a hint.

\section{Distance chart}

\begin{thm}{Theorem}\label{thm:dist-chart}
Suppose $p,a_0,\dots,a_m$ be points in an $m$-dimensional Alexandrov space $\spc{L}$ such
\[\angk p{a_i}{a_j}>\tfrac\pi2\]
for any $i\ne j$.
Then the map $f\:\spc{L}\to\RR^m$ defined by
\[f\:x\mapsto (\dist{a_1}{x}{},\dots,\dist{a_m}{x}{})\]
gives a bi-Lipschitz embedding of a neighborhood $\Omega$ of $p$;
the restriction $f|_\Omega$ is called \emph{distance chart} at $p$.
\end{thm}

The following exercise guides you to prove the theorem.

\begin{thm}{Exercise}\label{ex:proof-dist-chart}
Suppose $p,a_0,\dots,a_m\in\spc{L}$ and $f\:\spc{L}\to\RR$ are as in \ref{thm:right-inverse}.
Show that there is $\eps>0$ such that one of the following $m$ inequalities hold
\begin{align*}
\mangle\hinge xy{a_1}&<\tfrac\pi2-\eps,\ \dots,\  \mangle\hinge xy{a_m}<\tfrac\pi2-\eps,
\\
\mangle\hinge yx{a_1}&<\tfrac\pi2-\eps,\ \dots,\ \mangle\hinge yx{a_m}<\tfrac\pi2-\eps
\end{align*}
for any two points $x,y$ in a sufficiently small neighborhood of $p$.
Use it to prove \ref{thm:dist-chart}.
\end{thm}

\section{Volume}

Fix a positive integer $m$.
The $m$-dimensional Hausdorff measure of a Borel set $B$ in a metric space will be called its \index{volume}\emph{$m$-volume}; it will be denoted by $\vol_m B$.
We assume that the Hausdorff measure is calibrated so that the unit cube in $\EE^m$ has unit volume.

This definition will be applied mostly to subsets in $m$-dimensional Alexandrov spaces.
In this case, we may write $\vol B$ instead of $\vol_m B$.


\begin{thm}{Bishop--Gromov inequality}\label{inq:BG}
Let $\spc{L}$ be an $\Alex0$ space.
Suppose $\dim \spc{L}=m<\infty$.
Then 
\[\vol \oBall(p,r)\le \omega_m\cdot r^m,\]
where $\omega_m$ denotes the volume of the unit ball in $\EE^m$.
Moreover, the function 
\[r\mapsto \frac{\vol B(p,r)}{r^m}\]
is nonincreasing.
\end{thm}

\parit{Proof.}
Given $x\in\spc{L}$ choose a geodesic path $\gamma_x$ from $p$ to $x$.
Recall that $\log_p\:\spc{L}\to \T_p$ can be defined by $\log_p\:x\mapsto \gamma_x^+(0)$.
By comparison, $\log_p$ is distance-noncontracting.
Note that $\log_p$ maps $\oBall(p,r)_{\spc{L}}$ to $\oBall(0,r)_{\T_p}$.

\begin{wrapfigure}{r}{44 mm}
\vskip-0mm
\centering
\includegraphics{mppics/pic-803}
\vskip1mm
\end{wrapfigure}

If $\T_p\iso \EE^m$, then $\vol\oBall(0,r)_{\T_p}\z=\omega_m\cdot r^m$,
and the first statement follows.

If $\T_p$ is not isometric to $\EE^m$, then by \ref{ex:tangent=Em}, we can find a point $p'$ arbitrarily close to $p$ such that $\T_{p'}\iso \EE^m$.
If $\eps>\dist{p}{p'}{}$, then $\oBall(p,r)\subset \oBall(p',r+\eps)$.
Therefore,
\[\vol \oBall(p,r)\le \omega_m\cdot (r+\eps)^m\]
for any $\eps>0$.
Hence the first statement follows.

Now, suppose $0<r_1<r_2$.
Consider the map $w\: \spc{L}\to \spc{L}$ defined by $w\:x\mapsto \gamma_x(\tfrac {r_1}{r_2})$.
(The map $w$ mimics the dilation with center at $p$ and coefficient $\tfrac {r_1}{r_2}$.)
By comparison,
\[\dist{w(x)}{w(y)}{}\ge \tfrac {r_1}{r_2}\cdot \dist{x}{y}{}.\]
Observe that $\oBall(p,r_1) \supset w[\oBall(p,r_2)]$.
Therefore, 
\[\vol \oBall(p,r_1)\ge (\tfrac {r_1}{r_2})^m\cdot\vol \oBall(p,r_2).\]
\qedsf

The following exercise generalizes the Bishop--Gromov inequality to $\Alex{-1}$ case. 
It is sufficient for most applications, but a more exact statement will be given in \ref{inq:BG+} which also includes the case of  $\Alex{1}$ spaces.

\begin{thm}{Exersice}\label{ex:BG}
Let $\spc{L}$ be an $\Alex{-1}$ space.
Suppose $\spc{L}=m<\infty$.
Show that
\[\vol \oBall(p,r)\le \omega_m\cdot(\sinh r)^m,\]
where $\omega_m$ denotes the volume of the unit ball in $\EE^m$.
Moreover, the function 
\[r\mapsto \frac{\vol B(p,r)}{(\sinh r)^m}\]
is nonincreasing.
\end{thm}

\section{Other dimensions}\label{sec:all-dim}

Next we want to show that \textit{all reasonable definitions of dimension give the same result for Alexandrov spaces}.
More precisely, we have the following theorem; compare to \cite[15.16]{alexander-kapovitch-petrunin2024}.
We refer to \cite{hurewicz-wallman} for definitions of \index{Lebesgue coverning dimension}\emph{Lebesgue coverning dimension} \index{$\TopDim$ (topological dimension)}$\TopDim$ and \index{Hausdorff dimension}\emph{Hausdorff dimension} \index{$\HausDim$ (Hausdorff dimension)}$\HausDim$.

\begin{thm}{Theorem}\label{thm:dim=dim}
For any Alexandrov space $\spc{L}$, we have
\[\LinDim \spc{L}=\TopDim \spc{L}=\HausDim \spc{L}.\]
\end{thm}

\parit{Proof.}
Suppose $\LinDim\spc{L}\ge m$.
The right inverse theorem implies that $\spc{L}$ contains a subset homeomorphic to an open ball in $\EE^m$.
It follows that
\[\TopDim\spc{L}\ge \LinDim\spc{L}.\]

By Szpilrajn's theorem \cite[theorems V 8 and VII 2]{hurewicz-wallman}, 
\[\HausDim\spc{L}\ge\TopDim\spc{L}.\]

Finally, by the Bishop--Gromov inequality (\ref{inq:BG} and \ref{ex:BG}), we get that 
\[\LinDim \spc{L}\ge \HausDim\spc{L}.\]
\qedsf

\begin{thm}{Exercise}\label{ex:dim=dim}
Let $\Omega$ be an open subset of Alexandrov space $\spc{L}$.
Show that 
\[\LinDim \spc{L}=\LinDim \Omega=\TopDim \Omega=\HausDim \Omega.\]
\end{thm}

\section{Comments}

Let us state a version of Bishop--Gromov inequality for $\Alex\kappa$ spaces.
Its proof requires additionally the so-called \textit{coarea formula} for Alexandrov spaces. 
The weaker inequality from \ref{ex:BG} is sufficient for the sequel.

\begin{thm}{Bishop--Gromov inequality}\label{inq:BG+}
Let $p$ be a point in an $m$-dimensional $\Alex\kappa$ space.
Consider the function $v(r)\z=\vol_m\oBall(p,r)$;
denote by $\tilde v(r)$ the volume of $r$ ball in $\MM^m(\kappa)$.
Then 
\[v(r)\le \tilde v(r)\]
for $r>0$ and the function 
\[r\mapsto \frac{v(r)}{\tilde v(r)}\] is nonincreasing.
If $\kappa>0$, then one has to assume that $r<\tfrac\pi{\sqrt\kappa}$.
\end{thm}

This inequality was originally proved for Riemannian manifolds with lower Ricci curvature.
The first part is also called \emph{Bishop's inequality}.
It is due to Richard Bishop; see \cite{bishop1964} and \cite[Corollary 4, p. 256]{bishop-crittenden}.
The second part is due to Michael Gromov \cite{gromov1981}.

Theorem~\ref{thm:dim=dim}, was ssentially proved by Conrad Plaut \cite{plaut:dimension}.
At that time, it was not known whether
\[\LinDim\spc{L}=\infty\quad \Rightarrow\quad \TopDim\spc{L}=\infty\]
for any Alexandrov space $\spc{L}$.
The latter implication was proved by Grigory Perelman and the second author \cite{perelman-petrunin:qg}.


%%!TEX root = the-volume.tex

\chapter{Limit spaces}\label{chap:lim}\label{chap:stability}


Here we will show that lower curvature bound in the sense of Alexandrov survives under Gromov--Hausdorff limit,
present Perelman's construction of strictly concave functions, and
prove Gromov's selection theorem.

The suvival of curvature bound provides the main source of applications of Alexandrov geometry;
as an illustration we prove the homotopy stability theorem (\ref{thm:h-stability}) and deduce the homotopy finiteness theorem (\ref{thm:h-finiteness}) from it.



\section{Survival of curvature bounds}

\begin{thm}{Theorem}\label{thm:CBB-closed}
Let $\spc{X}_n\z\to \spc{X}_\infty$ be a convergence in the sense of Gromov--Hausdorff.
Suppose that for each $n$, the space $\spc{X}_n$ has curvature $\ge\kappa$ in the sense of Alexandrov.
Then the same holds for~$\spc{X}_\infty$.
\end{thm}

\parit{Proof}.
Choose a quadruple of points $p_\infty, x_\infty,y_\infty,z_\infty\in \spc{X}_\infty$.

By the definition of Gromov--Hausdorff convergence, we can choose points $p_n$,  $x_n$, $y_n$, $z_n\in \spc{X}_n$ for each $n$
that converge to $p_\infty$, $x_\infty$, $y_\infty$, $z_\infty\in \spc{X}_\infty$, respectively.
In particular, each of the 6 distances between pairs of $p_n$, $x_n$, $y_n$, $z_n$ converge to the distance between the corresponding pairs of $p_\infty, x_\infty,y_\infty,z_\infty$.

Since $\MM^2(\kappa)$-comparison holds for $p_n$, $x_n$, $y_n$, $z_n\z\in \spc{X}_n$,
passing to the limit, we get the $\MM^2(\kappa)$-comparison for $p_\infty$,  $x_\infty$, $y_\infty$, $z_\infty$.
\qeds

\begin{thm}[!]{Exercise}\label{ex:dim-lim}
Suppose that a sequence $\spc{A}_1,\spc{A}_2,\dots$ of $\Alex\kappa$ spaces converges to $\spc{A}_\infty$ in the sense of Gromov--Hausdorff.
Show that $\spc{A}_\infty$ is $\Alex\kappa$ and
\[\dim \spc{A}_\infty\le \liminf_{n\to\infty} \dim \spc{A}_n.\]
\end{thm}

\section{Gromov's selection theorem}

\begin{thm}{Gromov's selection theorem}\label{thm:gromov-compactness}
Let $m$ be a positive integer, and let $D,\kappa\in\RR$.
Then any sequence of $m$-dimensional $\Alex\kappa$ spaces with diameters at most $D$
has a converging subsequence in the sense of Gromov--Hausdorff.
\end{thm}

\parit{Proof of \ref{thm:gromov-compactness}.}
Denote by $\bm{K}$ the set of all isometry classes of $\Alex0$ spaces with dimension $\le m$ and diameter $\le D$.
By \ref{ex:dim-lim}, $\bm{K}$ is a closed subset of $\GH$.

Choose a space $\spc{A}\in \bm{K}$;
suppose $x_1,\dots,x_n\in \spc{A}$ is a collection of points such that $\dist{x_i}{x_j}{}> \eps$ for all $i\ne j$.
Note that the balls $B_i=\oBall(x_i,\tfrac\eps2)$ do not overlap.

By \ref{thm:right-inverse}, $\vol \spc{A}>0$.
By Bishop--Gromov inequality, $\vol \spc{A}<\infty$,
and if $\eps<D$, then 
\[\vol B_i\ge (\tfrac\eps{2\cdot D})^m\cdot\vol \spc{A}\]
for any $i$.
It follows that $n\le (\tfrac{2\cdot D}\eps)^m$;
that is, 
\[\pack_\eps\spc{A}\le  N(\eps)\df(\tfrac{2\cdot D}\eps)^m\]
for all small $\eps>0$.

Choose a maximal $\eps$-packing $x_1,\z\dots,x_n\in \spc{A}$.
By \ref{ex:pack-net}, $\spc{F}_\eps\z\df\{x_1,\z\dots,x_n\}$ is an $\eps$-net of $\spc{A}$.
Observe that $\dist{\spc{F}_\eps}{\spc{A}}{\GH}\le \eps$.
Further, note that the set $\bm{F}_\eps$ of finite metric spaces with diameter $\le D$ and at most $N(\eps)$ points forms a compact subset in $\GH$.

Summarizing, for any $\eps>0$ we can find a compact $\eps$-net $\bm{F}_\eps\subset \GH$ of $\bm{K}$.
Since $\GH$ is complete (\ref{prop:complete}), it remains to apply \ref{ex:net:compact}.

We finished the proof of the case $\kappa=0$.
In the general case, applying rescaling, we can assume that $\kappa=-1$ and then argue as before, using \ref{ex:BG} instead of \ref{inq:BG}.
\qeds

\begin{thm}[!]{Exercise}\label{ex:pack-vol}

\begin{subthm}{ex:pack-vol:pack}
Let $\spc{A}$ be an $m$-dimensional $\Alex0$ space with diameter $\le D$.
Suppose $\vol\spc{A}\ge v_0>0$.
Show that 
\[\pack_\eps\spc{A}\ge \frac\Const{\eps^m}\]
for some constant $\Const=\Const(m,D,v_0)>0$.
\end{subthm}


\begin{subthm}{ex:pack-vol:dim}
Conclude that if $\spc{A}_n$ is a sequence of $m$-dimensional $\Alex0$ spaces with diameter $\le D$, and volume $\ge v_0$, then its Gromov--Hausdorff limit $\spc{A}_\infty$ (if it exists) has dimension~$m$.
\end{subthm}
\end{thm}

\begin{thm}{Exercise}\label{ex:diam-compact:GH}
Show that any sequence of $m$-dimensional $\Alex\kappa$ spaces with marked points contains a subsequence pointed-converging in the sense of Gromov--Hausdorff (see Section~\ref{sec:Gromov--Hausdorff}).

\end{thm}

%%!TEX root = the-homot-finite.tex
\section{Controlled concavity}

Alexandrov spaces have plenty of semiconcave functions;
for instance, square of distance function. 
The following theorem provides a source of strictly concave functions  defined in a small open sets of finite-dimensional Alexandrov spaces. 

\begin{thm}{Theorem}
\label{thm:strictly-concave}
Let $\spc{L}$ be a complete finite-dimensional Alexandrov  space.
Then for any point $p\in \spc{L}$, there is  a strictly concave function $f$ defined in an
open neighborhood of $p$.

Moreover, given $0\ne v\in T_p$, the differential, $\dd_p f$, can be chosen
arbitrarily close to $x\mapsto -\<v,x\>$.
\end{thm}

\parit{Proof.} 
Fix small $r>0$ and large $c$;
consider the real-to-real function 
$$\phi_{r,c}(x)=(x-r)- c\cdot(x-r)^2/r,$$
so we have 
$\phi_{r,c}(r)=0$,
$\phi_{r,c}'(r)=1$,
and $\phi_{r,c}''(r)=- {2c}/{r}$. 

\begin{wrapfigure}{o}{44 mm}
\vskip-0mm
\centering
\includegraphics{mppics/pic-901}
\vskip1mm
\end{wrapfigure}

Let $\gamma$ be a unit-speed geodesic, fix a point $q$ and let 
$$\alpha(t)=\mangle(\gamma^+(t),\dir{\gamma(t)}{q}).$$
Recall that $r$ is small.
If $\dist q{\gamma(t)}{}$ is sufficiently close to
$r$, then direct calculations show that
$$(\phi_{r,c}\circ\distfun_q\circ\gamma)''(t)
\le 
\frac{3-c\cdot \cos^2[\alpha(t)]}{r}.$$
(Since $c$ is large, this inequality implies that $\phi_{r,c}\circ\distfun_q\circ\gamma$ is strictly concave at $t$ unless $\alpha(t)\approx\tfrac\pi2$.) 

Now, assume $\{q_1,\dots, q_N\}$ is a finite set of points such that $\dist p{q_i}{}=r$ for any $i$. 
For a geodesic $\gamma$, set $\alpha_i(t)=\mangle(\gamma^+(t),\dir {\gamma(t)}{q_i})$. 
Assume we have a collection $\{q_i\}$ such
that 
\[\max_i\{|\alpha_i(t)-\tfrac\pi2|\}\ge\eps>0\]
for any geodesic $\gamma$ in $\oBall(p,\eps)$. 
We can assume that $c>3N/\cos^2\eps$;
then the inequality above implies that the function
$$f=\sum_i \phi_{r,c}\circ\distfun_{q_i}$$
is strictly concave in $\oBall(p,\eps')$ for some positive $\eps'<\eps$.

The same argument as in \ref{ex:pack-vol} shows that for small $r>0$, one can
choose $N\ge \Const/\delta^{m-1}$ points $\{q_i\}$ such that $\dist{p}{q_i}{}=r$
and $\angk p{q_j}{q_i}>\delta$ (here $\Const=\Const(\Sigma_p)>0$).
On the other hand, suppose $\gamma$ runs from $x$ to~$y$.
If $|\alpha_i(t)- \tfrac\pi2|<\eps\ll\delta$, then applying the ($n$+1)-point comparison to $\gamma(t)$, $x$, $y$ and all $\{q_i\}$ we get that
$N\le \Const(m)/\delta^{m-2}$. 
Therefore, for small $\delta>0$ and yet smaller $\eps>0$, the set $\{q_i\}$ forms the needed collection.

If $r$ is small, then points $q_i$ can be chosen so that all directions
$\dir p {q_i}$ will be $\eps$-close to a given direction $\xi$ and
therefore the second property follows.
\qeds

The function $f$ in \ref{thm:strictly-concave} can be chosen to have maximum value $0$ at $p$,
$f(p)=0$ and with $\dd_p f(x)\approx-|x|$.
It can be constructed by taking the minimum of the functions in the theorem.
Then the set $K=\set{x\in\spc{L}}{f(x)\ge -\eps}$ forms a closed convex neighborhood of $p$ for any small $\eps>0$, so we get the following.


\begin{thm}{Corollary}\label{cor:convex-nbhd}
Any point $p$ of a finite-dimensional Alexandrov space admits an arbitrary small convex closed neighborhood $K$ and a strictly concave function $f$ defined in a neighborhood of $K$ such that $p$ is the maximum point of $f$
and $f|_{\partial K}=0$.
\end{thm}

\section{Liftings}

Suppose that the Gromov--Haudorff distance $\dist{\spc{L}}{\spc{L}'}{\GH}$ is sufficienlty small, so we may think that both spaces $\spc{L}$ and $\spc{L}'$ lie at small Hausdorff distance in an ambient metric space $\spc{W}$.
In particular, we may choose a small $\eps>0$, so that for any point $p\in \spc{L}$, there is a point $p'\in \spc{L}'$ such that $\dist{p}{p'}{\spc{W}}<\eps$;
the point $p'$ will be called a \index{lifting}\emph{lifting} (or \emph{$\eps$-lifting}) of $p$ in $\spc{L}'$.
We may choose a lifting $p'\in\spc{L}'$ for every point $p\in\spc{L}$, 
in this case the map $p\mapsto p'$ is called a {}\emph{($\eps$-)lifting map}.

Note that the lifting is not uniquely defined.
The lifting maps is not assumed to be continuous.
When we talk about liftings, we assume that $\eps>0$, the inclusions $\spc{L},\spc{L}'\hookrightarrow\spc{W}$,
as well as $\spc{W}$ are chosen.

Let $\spc{L}$ be  a compact $m$-dimensional Alexandrov space.
Suppose $\spc{L}'$ is another compact $m$-dimensional Alexandrov space such that $\dist{\spc{L}}{\spc{L}'}{\GH}$ is sufficiently small --- smaller than some $\eps=\eps(\spc{L})>0$.
Then the construction in $\spc{L}$ from the previous section  
can be repeated in $\spc{L}'$ for the liftings of all points and the same function $\phi$.
It produces a strictly concave function defined in a controlled neighborhood of the lifting $p'$ of $p$.

The result of this and related constructions will be called \index{lifting}\emph{liftings},
say we can talk about a lifting from $\spc{L}$ to $\spc{L}'$ of a function provided by \ref{thm:strictly-concave} (if the Gromov--Hausdorff distance $\dist{\spc{L}}{\spc{L}'}{\GH}$ is small, then these liftings are stricly concave)
and a lifting of a convex neighborhood from \ref{cor:convex-nbhd}.
Here one cannot use \ref{thm:strictly-concave} and \ref{cor:convex-nbhd} as black boxes --- one has to understand the construction, but it is straightforward.

\section{Nerves}

Let $\{\Omega_1,\dots,\Omega_k\}$ be a finite open cover of a compact metric space $\spc{X}$.
Consider an abstract simplicial complex $\spc{N}$, with one vertex $v_i$ for each set $\Omega_i$ such that a simplex with vertices $v_{i_1},\dots, v_{i_m}$ is included in $\spc{N}$ if 
the intersection $\Omega_{i_1}\cap\dots\cap \Omega_{i_m}$ is nonempty.
\begin{figure}[ht!]
\vskip-0mm
\centering
\includegraphics{mppics/pic-1402}
\end{figure}
The obtained simplicial complex $\spc{N}$ is called the \index{nerve}\emph{nerve} of the covering $\{\Omega_i\}$.
Evidently $\spc{N}$ is a finite simplicial complex ---
it is a subcomplex of a simplex with the vertices $\{v_1,\dots,v_k\}$.
Recall that $\Star_{v_i}$ denotes the union of all simplexes in $\spc{N}$ that shares vertex $v_i$.

The next statement follows from \cite[4G.3]{hatcher}.


\begin{thm}{Nerve theorem}\label{thm:nerve}
Let $\{\Omega_1,\dots,\Omega_k\}$ be an open cover of a compact metric space $\spc{X}$
and let $\spc{N}$ be the corresponging nerve with vertices $\{v_1,\dots,v_k\}$.
Suppose that every nonempty finite intersection $\Omega_{\alpha_1}\cap\z\dots\cap\Omega_{\alpha_k}$ is contractible.
Then $\spc{X}$ is homotopy equivalent to the nerve $\spc{N}$ of the cover.

Moreover homotopy equivalences  $a\:\spc{X}\to \spc{N}$ and $b\:\spc{N}\to\spc{X}$ can be chosen so that 
if $x\in \Omega_i$, then $a(x)\in \Star_{v_i}$,
and if $y\in\spc{N}$ lies in the simplex with vertices $v_{i_1},\dots, v_{i_m}$, then $b(y)\in \Omega_{i_1}\cup\dots\cup \Omega_{i_m}$.
\end{thm}

%???Вить, посмотри на это утверждение --- оно мне не сильно нравится.


\section{Homotopy stability}

\begin{thm}{Theorem}\label{thm:h-stability}
Let $\spc{L}_1,\spc{L}_2,\dots$, and $\spc{L}_\infty$ be $m$-dimensional $\Alex\kappa$ spaces, and $m<\infty$.
Suppose $\spc{L}_n\z\GHto \spc{L}_\infty$ as $n\to \infty$.
Then $\spc{L}_\infty$ is homotopically equivalent to $\spc{L}_n$ for all large $n$.

Moreover, given $\eps>0$ there are maps $h_n\:\spc{L}_\infty\to \spc{L}_n$ that are homotopy equivalences and $\eps$-liftings for all large $n$.
\end{thm}

Applying this theorem with the Gromov's selection theorem (\ref{thm:gromov-compactness}) and Exercise \ref{ex:pack-vol}, we get the following.


\begin{thm}{Theorem}\label{thm:h-finiteness}
There are only finitely many homotopy types of $m$-dimensional $\Alex\kappa$ spaces with diameter $\le D$, and volume $\ge v_0$;
here we assume that an integer $m$, and $v_0>0$ and $D>0$ are given.
\end{thm}

\parit{Proof of \ref{thm:h-finiteness} modulo \ref{thm:h-stability}.}
Assume the contrary, then we can choose a sequence of spaces $\spc{L}_1,\spc{L}_2,\dots$ that have different homotopy types and satisfy the assumptions of the theorem.
By Gromov's compactness theorem, we can assume that $\spc{L}_n$ converges to say $\spc{L}_\infty$ in the sense of Gromov--Hausdorff.

By \ref{ex:pack-vol}, $\dim \spc{L}_\infty=m$.
It remains to apply \ref{thm:h-stability}.
\qeds

\parit{Proof of \ref{thm:h-stability}.}
Since $\spc{L}_\infty$ is compact, applying \ref{cor:convex-nbhd}, we can find a finite open cover of $\spc{L}_\infty$ by convex open sets $\Omega_1,\dots, \Omega_k$ such that 
for each $\Omega_i$ there is a strictly concave function $f_i$ that is defined in a neighborhood of $\bar \Omega_i$ and such that $f_i|_{\partial \Omega_i}=0$.

Subtracting from functions $f_i$ some small value $\eps>0$,
we can ensure that $\bigcap_{i\in S}\Omega_{i}\ne \emptyset$ if and only if $\bigcap_{i\in S}\bar\Omega_{i}\ne \emptyset$.

Suppose that $W=\bigcap_{i\in S}\Omega_{i}\ne \emptyset$.
Then $W$ is contractible.
Indeed the function 
\[f_S\df\min_{i\in S} f_i\]
is strictly concave and it vanished on the boundary of $W$.
The $f_S$-gradient flow $(t,x)\mapsto \GF_{f_S}^t(x)$ defines a homotopy
$[0,\infty)\times W\to W$.
By the first distance estimate (\ref{thm:dist-est}), $\GF_{f_S}^t(x)$ converges to the (necessarily unique) maximum point of $f_S$ as $t\to\infty$.
Therefore, in the obtained homotoly we can parametrize $[0,\infty)$ by $[0,1)$ and extend the homotopy by continiously to $[0,1]$;
thus we get that $W$ is contractible.
In other words, the cover $\{\Omega_1,\dots, \Omega_k\}$ meets the assumptions of the nerve theorem (\ref{thm:nerve}).

The functions $f_i$ and sets $\Omega_i$ can be lifted to $\spc{L}_n$ keeping their properties for all large $n$. 
More precisely, there are liftings $f_{i,n}$ of all $f_i$ to $\spc{L}_n$ which are strictly concave for all large $n$ and such that $\bar\Omega_{i,n}=\set{x\in \spc{L}_n}{f_{i,n}(x)\ge 0}$ is a compact convex set and $\Omega_{i,n}\z=\set{x\in \spc{L}_n}{f_{i,n}(x)> 0}$ is an open convex set for each $i$.

Notice that $\{\Omega_{1,n},\dots,\Omega_{k,n}\}$ is an open cover of $\spc{L}_n$ for all large~$n$.
Indeed suppose we have $p_n\in \spc{L}_n\setminus(\Omega_{1,n}\cup\dots\cup\Omega_{k,n})$ for arbitrary large $n$.
Since $\spc{L}_\infty$ is compact, there is a limit point $p_\infty\in \spc{L}_\infty$ for a subsequnce of $p_n$.
But $p_\infty\in\Omega_i$ for some $i$ and therefore $p_n\in \Omega_{i,n}$ for arbitrary large $n$ --- a contradiction.

In a similar fashion, we can show that if $n$ is large, then any collection $\{\Omega_{i,n}\}_{i\in S}$ has a common point in $\spc{L}_n$ 
if and only if $\{\Omega_{i}\}_{i\in S}$ has a common point in $\spc{L}_\infty$.
Here we have to use that $\bigcap_{i\in S}\Omega_{i}\ne \emptyset$ if and only if $\bigcap_{i\in S}\bar\Omega_{i}\ne \emptyset$.

It follows that for any large $n$ the covers 
\begin{itemize}
\item $\{\Omega_{1},\dots,\Omega_{k}\}$ of $\spc{L}_\infty$ and 
\item $\{\Omega_{1,n},\dots,\Omega_{k,n}\}$ of $\spc{L}_n$.
\end{itemize}
have the same nerve.
By the nerve theorem (\ref{thm:nerve}), $\spc{L}_n$ and $\spc{L}_\infty$ are homotopically equivalent for all large $n$ --- a contradiction.
\qeds

\section{Comments}

Gromov's selection theorem provides the main source of applications of Alexandrov spaces to Riemannian geometry.
The homotopy-type finiteness theorem (\ref{thm:h-finiteness})  illustrates this technique.

Originally, Gromov's selection theorem was proved for Riemannian manifolds with a lower bound on Ricci curvature \cite{gromov1981}.
It motivates the study of the so-called $\mathrm{CD}(K,m)$ spaces; $\mathrm{CD}$ stands for curvature-dimension condition.
This theory has serious applications in Alexandrov geometry;
in particular, it provides a version of Liouville theorem about phase-space volume of geodesic flow in Alexandrov space \cite{brue-mondino-semola}.

The construction of strictly concave function is due to Grigory Perelman \cite{perelman1993,perelman-petrunin}.

Let us list some results that can be proved by applying Gromov's selection theorem
in the same fashion as in the proof of homotopy-type finiteness theorem (\ref{thm:h-finiteness}).

\begin{thm}{Betti-number theorem}
There is a constant $\Const=\Const(m,D,\kappa)$ such that 
\[\beta_0(M)+\beta_1(M)+\dots+\beta_m(M)\le \Const\]
for any closed $m$-dimensional Riemannian manifold $M$ with sectional curvature $\ge \kappa$ and diameter $\le D$.
Here $\beta_i(M)$ denotes $i^\text{th}$ Betti number of $M$.
\end{thm}

Gromov's original proof \cite{gromov-1981} of the Betti-number theorem did not use Alexandrov geometry directly;
but it is quite natural to prove it via Gromov's selection theorem.
The following result proved the second author \cite{petrunin2008}, and it uses the same technique.

\begin{thm}{Scalar curvature bound}
There is a constant $\Const=\Const(m,D,\kappa)$ such that 
\[\int_M\Sc\le \Const\]
for any closed $m$-dimensional Riemannian manifold $M$ with sectional curvature $\ge \kappa$ and diameter $\le D$.
Here $\Sc$ denotes the scalar curvature.
\end{thm}

The following theorem is a more exact version of \ref{thm:h-stability}.
Its close relative (\ref{thm:spherical-nbhd}) will play an important role in the following lecture.

\begin{thm}{Stability theorem}\label{thm:stability}
Let $\spc{L}_1,\spc{L}_2,\dots$, and $\spc{L}_\infty$ be  $m$-dimensional $\Alex\kappa$ spaces, and $m<\infty$.
Suppose $\spc{L}_n\GHto \spc{L}_\infty$ as $n\to \infty$.
Then $\spc{L}_\infty$ is homeomorphic to $\spc{L}_n$ for all large $n$.

Moreover, given $\eps>0$ there are maps $h_n\:\spc{L}_\infty\to \spc{L}_n$ that are homeomorphisms and $\eps$-liftings for all large $n$.
\end{thm}

This theorem was proved by Grigory Perelman \cite{perelman1991};
the proof was rewritten with more details by the first author \cite{kapovitch}.
Perelman have made an informal annoncemnt that the homeomorphisms in the theorem can be assumed to be bi-Lipschitz with constants that depend on $\spc{L}_\infty$;
he refused to write the proof, and so it save to consider it as a conjecture.

The last statement in the theorem implies the following finiteness result.

\begin{thm}{Homeomorphism-type finiteness}
There are only finitely many homeomorphism types of closed $m$-dimensional manifolds that admit a Riemannian metric with sectional curvature $\ge \kappa$, and diameter $\le D$.
\end{thm}

Applying several results in differential topology, this statement can be improved to diffeomorphism-type finiteness in all dimensions $m$ except $m=4$; see \cite{kirby-siebenmann} and  \cite{moise,thurston} for cases $m\ge 5$ and $m\le 3$, respectively.



%%!TEX root = the-boundary.tex
\chapter{Boundary}\label{chap:bry}

This lecture defines the boundary of a finite-dimensional Alexandrov space.
After discussing its properties, we prove the doubling theorem (\ref{thm:doubling:doubling}).

\section{Definition}

Let us give an inductive definition of the boundary of finite-dimensional Alexandrov spaces.

Suppose $\spc{A}$ is a 1-dimensional Alexandrov space.
By Exercise~\ref{ex:dim=1},
$\spc{A}$ is homeomorphic to a 1-dimensional manifold (possibly with non-empty boundary).
This  allows us to define the boundary $\partial\spc{A}\subset \spc{A}$ as the boundary of the manifold.

Now assume that the notion of boundary is defined in dimensions $1,\dots,m-1$.
Suppose  $\spc{A}$ is $m$-dimensional Alexandrov space.
We say that $p\in \spc{A}$ belongs to the boundary (briefly $p\in \partial \spc{A}$) if 
$\partial\Sigma_p\ne\emptyset$.
By \ref{thm:finite-space-of-directions} and \ref{ex:finite-space-of-directions-dim}, $\Sigma_p$ is an $(m-1)$-dimensional Alexandrov space;
therefore its boundary is already defined and hence this inductive definition makes sense.

It is instructive to check the following statements.
\begin{itemize}
\item For a closed convex set $K\subset \EE^m$ with non-empty interior, the topological boundary of $K$ as a subset of $\EE^m$ coincides with the boundary $K$ described above.
\item If $\spc{A}\iso\spc{A}_1\times\spc{A}_2$ is a finite-dimensional Alexandrov space,
then
\[\partial \spc{A}=(\partial\spc{A}_1\times\spc{A}_2)\,\cup\,(\spc{A}_1\times\partial\spc{A}_2)\]
\item If $\Cone\Sigma$ is an $\Alex0$ space of dimensions $\ge 2$  (this necessarily implies that   $\Cone\Sigma$  is  $\Alex1 $ then
\[\partial \Cone\Sigma=\Cone\partial\Sigma,\]
where $\Cone\partial\Sigma=\set{s\cdot \xi\in\Cone\Sigma }{\xi\in \partial\Sigma}$.
\end{itemize}


\section{Conic neighborhoods}

The following statement \cite{perelman1993} is a close relative of Perelman's stability theorem \ref{thm:stability}.
% but its proof is  simpler .
We are going to use this result without proof.

Recall that the logarithm $\log_px\:\spc{A}\to \T_p$ is defined on page \pageref{page:log}.

\begin{thm}{Theorem}\label{thm:spherical-nbhd}
For any point $p$ in a finite-dimensional Alexandrov space $\spc{A}$
and all sufficiently small $\eps>0$
there is a homeomorphism $h_\eps\:\oBall(p,\eps)_{\spc{A}}\to \oBall(0,\eps)_{\T_p}$ such that $0=h_\eps(p)$.

Moreover, we may assume that
\[
\sup_{x\in \oBall(p,\eps)}\{\,\tfrac1\eps\cdot\dist{\log_px}{h_\eps(x)}{\T_p}\,\}\to 0
\quad\text{as}\quad
\eps\to 0.\]
\end{thm}
Note that the last condition automatically implies that  $h_\eps$ as an $o(\eps)$ G-H approximation.

The above theorem is often used together with the \textit{uniqueness of conic neighborhoods} stated below.

Suppose that an open  neighborhood $U$ of a point $x$ in a metric space $\spc{X}$
% openness is very important. it's false otherwise
admits a homeomorphism to $\Cone\Sigma$ such that $x$ is mapped to the origin of the cone.
In this case, we say that $U$ has a \index{conic neighborhood}\emph{conic neighborhood} of~$x$.

\begin{thm}{Uniqueness of conic neighborhoods}\label{lem:kwun}
Any two conic neighborhoods of a given point in a metric space are \index{pointed homeomorphic}\emph{pointed homeomorphic}; that is, there is a homeomorphism between neighborhoods that maps the origin of one cone to the origin of the other.
\end{thm}

\begin{thm}{Advanced exercise}\label{ex:conic}
Prove \ref{lem:kwun} or read the proof in \cite{kwun1964}.
\end{thm}


\begin{thm}{Exercise}\label{ex:conic-tangent}
Suppose $x\mapsto x'$ is a homeomorphism between finite-dimensional Alexandrov spaces $\spc{A}$ and $\spc{A}'$. Show that 

\begin{subthm}{ex:conic-tangen:tangent}
$\T_x\cong \T_{x'}$ (here and below $\cong $ means homeomorphic) 
\end{subthm}

\begin{subthm}{ex:conic-tangen:dir}
$\Susp\Sigma_x\cong \Susp\Sigma_{x'}$.
\end{subthm}

\begin{subthm}{ex:conic-tangen:example}
but in general $\Sigma_x\ncong\Sigma_{x'}$.
\end{subthm}

\end{thm}



\section{Topology}

The following theorem states that boundary is a topological invariant, despite our definition having used geometry.

\begin{thm}{Theorem}\label{thm:top-bry}
Let $\spc{A}$ and $\spc{A}'$ be homeomorphic finite-dimensional Alexandrov spaces.
Then $\dim \spc{A}=\dim\spc{A}'$ and
\[\partial\spc{A}\ne \emptyset
\quad\iff\quad
\partial\spc{A}'\ne \emptyset
\]
\end{thm}

While working on the proof, keep in mind that there are pairs of spaces $\spc{K}_1$ and $\spc{K}_2$ such that $\spc{K}_1\ncong \spc{K}_2$, but $\RR\times \spc{K}_1\cong \RR\times \spc{K}_2$.
Suspension over the Poincaré homology sphere with $\SSS^4$ is one of the examples; compare to \ref{ex:conic-tangen:example}.

Let $\spc{A}$ be an $m$-dimensional Alexandrov space and $m<\infty$.
Define \index{rank}\emph{rank} of $\spc{A}$ (briefly, \index{$\rank\spc{A}$}$\rank\spc{A}$) as the minimal value $k$ such that $\spc{A}$ splits isometrically as $\RR^{m-k}\times \spc{K}$;
here $\spc{K}$ is a $k$-dimensional Alexandrov space.

In the following proof we will apply induction on the rank of $\spc{A}$.


\parit{Proof.}
The first statement follows from \ref{thm:dim=dim}.

Suppose we have a counterexample, say $\partial \spc{A}\ne \emptyset$, but $\partial \spc{A}'=\emptyset$.
Let $k\df\rank \spc{A}$ and $k'\df\rank \spc{A}'$.
We can assume that the pair $(k,k')$ is minimal in lexicographic order;
in particular, $k$ is minimal.
Let $x\mapsto x'$ be a homeomorphism from $\spc{A}$ to $\spc{A}'$.

Choose $x\in \partial \spc{A}$.
Since $\partial \spc{A}'=\emptyset$, we have $x'\notin \partial \spc{A}'$.
Note that 
\[\rank \T_x\le k
\quad\text{and}\quad
\rank \T_{x'}\le k',
\]
By \ref{ex:conic-tangen:tangent}, $\T_x\cong\T_{x'}$.
Note that $\partial \T_x\ne\emptyset$ and $\partial \T_{x'}=\emptyset$.
Therefore, we may assume that $\spc{A}$ and $\spc{A}'$ are Euclidean cones
and the homeomorphism sends the origin to the origin.
The remaining part of the proof is divided into three cases.

\parit{Case 1.}
Suppose $k>1$.
Let $\spc{A}\iso \RR^{m-k}\times \spc{C}$, where $\spc{C}$ a $k$-dimensional $\Alex0$ cone.
Observe that $\rank\T_y\le\rank\spc{A}$ for any $y\in\spc{A}$ and the equality holds only if $y$ projects to the origin of $\spc{C}$.

Since $k>1$ we can find $z\in\partial\spc{C}$ such that $z\ne 0$.
Choose $y$ that projects to $z$;
in particular, $\rank\T_y<\rank\spc{A}$.
By \ref{ex:conic-tangen:tangent}, $\T_y\cong\T_{y'}$,
$\partial  \T_y\ne\emptyset$ and $\partial \T_{y'}=\emptyset$.
The latter contradicts the minimality of $k$.

\parit{Case 2.} Suppose $k\le1$ and $k'>1$.
Since $\partial \spc{A}\ne \emptyset$, we get that $k=1$;
therefore, $\spc{A}=\RR^{m-1}\times\RR_{\ge0}$.

Let $\spc{A}'\iso \RR^{m-k'}\times \spc{C}'$, where $\spc{C}'$ a $k'$-dimensional $\Alex0$ cone.
Since $\partial\spc{A}\cong\RR^{m-1}$,
the image of $\partial\spc{A}$ in $\spc{A}'$ does not lie in $\RR^{m-k'}\z\times\{0\}$.
In other words, we can choose $y\in \partial \spc{A}$ such that its image $y'\in \spc{A}'$ has a nonzero projection in $\spc{C}'$.
Observe that $\T_y\cong\T_{y'}$,
\[
\rank\T_y\le k=1,
\quad
\rank\T_{y'}< k',
\quad
\partial \T_y=\emptyset,
\quad\text{and}\quad
\partial \T_{y'}\ne \emptyset\]
--- a contradiction.

\parit{Case 3.}
Suppose $k\le 1$ and $k'\le 1$.
Since $\partial \spc{A}\ne \emptyset$, $k=1$.
By \ref{ex:dim=1}, $\spc{A}\z\cong \RR^{m-1}\times\RR_{\ge0}$.
Therefore, $\spc{A}'\cong\RR^m$, and $\spc{A}\ncong\spc{A}'$ --- a contradiction.
\qeds

\begin{thm}{Exercise}\label{ex:bry2bry}
Let $x\mapsto x'$ be a homeomorphism $\Omega\to\Omega'$
between open subsets in finite-dimensional Alexandrov spaces $\spc{A}$ and $\spc{A}'$.
Show that $x\in \partial \spc{A}$ if and only if $x'\in \partial \spc{A}'$.

\end{thm}

\begin{thm}{Exercise}\label{ex:bry-closed}
Show that boundary of a finite-dimensional Alexandrov space is a closed subset.
\end{thm}

\section{Tangent space}

Spaces of directions and tangent spaces of an Alexandrov space have already been defined in \ref{sec:space+directions} and \ref{sec: tangent space}.
Let us extend these definitions to subsets of an Alexandrov space.

Let $X$ be a subset in a finite-dimensional Alexandrov space $\spc{A}$.
Choose $p\in \spc{A}$ and $\xi\in \Sigma_p$.
Suppose $\xi$ is a limit of directions $\dir{p}{x_n}$ for a sequence $x_1,x_2,\dots{}\in X$ that converges to $p$.
Then we say that $\xi$ is in the \index{space of directions}\emph{space of directions} from $p$ to $X$;
briefly \index{$\Sigma_p$ (space of directions)}$\xi\in\Sigma_pX$.

Further, $\Cone(\Sigma_pX)$ will be called the \index{tangent space}\emph{tangent space} to $X$ at $p$;
it will be denoted by \index{$\T_p$ (tangent space)}$\T_pX$.

Note that $\Sigma_pX$ is a subset of $\Sigma_p$ and $\T_pX$ is a subcone in $\T_p$

\begin{thm}{Theorem}\label{thm:partial-Sigma}
For any finite-dimensional Alexandrov space $\spc{A}$, we have
\[\partial (\Sigma_p\spc{A})=\Sigma_p(\partial\spc{A})
\quad\text{and}\quad
\partial(\T_p\spc{A})=\T_p(\partial\spc{A}).\]
\end{thm}

\parit{Proof.}
Choose a sequence $x_n\in \partial \spc{A}$ such that $x_n\to p$ and $\dir p{x_n}\to\xi$.

Let $\eps_n=2\cdot \dist{p}{x_n}{}$,
and let $h_{\eps_n}\:\oBall(p,\eps_n)_{\spc{A}}\to \oBall(0,\eps_n)_{\T_p}$ be the homeomorphisms provided by \ref{thm:spherical-nbhd};
in particular, $\tfrac2{\eps_n}\cdot h_{\eps_n}(x_n)\to \xi$ as $n\to\infty$.
By \ref{ex:bry2bry}, $h_{\eps_n}(x_n)\in \partial \T_p$.
By \ref{ex:bry-closed}, $\xi\in \partial \T_p$.
Therefore,
\[\partial (\Sigma_p\spc{A})\supset\Sigma_p(\partial\spc{A})
\quad\text{and}\quad
\partial(\T_p\spc{A})\supset\T_p(\partial\spc{A}).\]

Similarly, choose $\xi\in\partial\Sigma_p$.
Let $h_{\eps_n}\:\oBall(p,\eps_n)_{\spc{A}}\to \oBall(0,\eps_n)_{\T_p}$ be the homeomorphisms provided by \ref{thm:spherical-nbhd} for a sequence $\eps_n\to 0$ as $n\to\infty$.
By \ref{ex:bry2bry}, $x_n=h_{\eps_n}^{-1}(\tfrac{\eps_n}2\cdot\xi)\in \partial \spc{A}$.
By \ref{thm:spherical-nbhd}, $\dir p{x_n}\to \xi$.
Hence
\[\partial (\Sigma_p\spc{A})\subset\Sigma_p(\partial\spc{A})
\quad\text{and}\quad\partial(\T_p\spc{A})\subset\T_p(\partial\spc{A}).\]
\qedsf

\section{Doubling}

Let $A$ be a closed subset in a metric space $\spc{X}$.
The \index{doubling}\emph{doubling} $\spc{W}$ of $\spc{X}$ across $A$ is two copies of $\spc{X}$ glued along $A$;
more precisely, the underlying set of $\spc{W}$ is the quotient $\spc{X}\times\{0,1\}/\sim$, where $(a,0)\sim (a,1)$ for any $a\in A$ and $\spc{W}$ is equipped with the minimal metric such that both maps $\spc{X}\to \spc{W}$ defined by $x\mapsto (x,0)$ and $x\mapsto (x,1)$ are distance-preserving.

Alternatively, one may say that $\spc{W}$ is equipped with the maximal metric such that the projection $\proj\:\spc{W}\to\spc{A}$ defined by $(x,i)\mapsto x$ is a short map. 
The metric on $\spc{W}$ can also be defined explicitly as
\[\dist{(x,i)}{(y,j)}{\spc{W}}=
\begin{cases}
\dist{x}{y}{\spc{X}}&\text{if}\quad i= j.
\\
\inf\set{\dist{x}{a}{\spc{X}}+\dist{y}{a}{\spc{X}}}{a\in A}&\text{if}\quad i\ne j.
\end{cases}
\]

\begin{thm}{Theorem}\label{thm:doubling}
Let $\spc{A}$ be a finite-dimensional Alexandrov space with non-empty boundary.
Suppose $f\z=\tfrac12\cdot\distfun_p^2$ for some $p\in \spc{A}$.
Then

\begin{subthm}{thm:doubling:concave}
If $\dim \spc{A}\ge 2$, then
$\distfun_{\partial \Sigma_x}(\xi)\le \tfrac\pi2$ for any $x\in\partial \spc{A}$ and $\xi\in \Sigma_x$.
Moreover, if $\distfun_{\partial \Sigma_x}(\xi)= \tfrac\pi2$, then $\mangle(\xi,\zeta)\le\tfrac\pi2$ for any $\zeta\in \Sigma_x$. 
\end{subthm}

\begin{subthm}{thm:partial-grad:grad}
$\nabla_xf\in \partial\T_x$ for any $x\in\partial \spc{A}$.
\end{subthm}

\begin{subthm}{thm:partial-grad:flow}
If $\alpha$ is an $f$-gradient curve that starts at $x\in \partial \spc{A}$, then $\alpha(t)\in \partial \spc{A}$ for any $t$.
Moreover, if $p\in \partial \spc{A}$, then $\gexp_p(v)\in \partial \spc{A}$ for any $v\in\partial\T_p$.
\end{subthm}

\begin{subthm}{thm:doubling:doubling}
The doubling $\spc{W}$ of $\spc{A}$ across $\partial \spc{A}$ is an Alexandrov space with the same curvature bound.
\end{subthm}

\end{thm}

Part \ref{SHORT.thm:doubling:doubling} is called the \index{doubling theorem}\emph{doubling theorem}.

\parit{Proof.}
We will denote by 
\ref{SHORT.thm:doubling:concave}$_m,\dots,$\ref{SHORT.thm:doubling:doubling}$_m$ the corresponding statement assuming $m=\dim\spc{A}$.

The proof goes by induction on $m$.
Statement \ref{SHORT.thm:doubling:doubling}$_1$ follows from \ref{ex:dim=1} --- this is the base.
The induction step is a combination of the implications below.

\parit{\ref{SHORT.thm:doubling:doubling}$_{m-1}\Rightarrow$\ref{SHORT.thm:doubling:concave}$_m$.}
Suppose $m=2$, then $\dim\Sigma_x=1$; see \ref{ex:finite-space-of-directions-dim}.
By \ref{ex:dim=1}, $\Sigma_x$ isometric to a line segment $[0,\ell]$;
we need to show that $\ell\le\pi$.

Assume $\ell>\pi$, then the tangent space $\T_x=\Cone\Sigma_x$ has several different lines thru the origin.
Recall that $\T_x$ is an Alexandrov space; see \ref{ex:finite-tan}.
By \ref{cor:splitting}, $\T_x$ is isometric to the Euclidean plane;
the latter contradicts that $\Sigma_x$ is a line segment.

Now suppose $m>2$, so $\dim \Sigma_x>1$.
Assume $\distfun_{\partial \Sigma_x}(\xi)> \tfrac\pi2$ for some $\xi$.
By \ref{SHORT.thm:doubling:doubling}$_{m-1}$, the doubling $\Xi$ of $\Sigma_x$ is $\Alex1$.
Denote by $\xi_0$ and $\xi_1$ the points in $\Xi$ that correspond to $\xi$.
Observe that $\dist{\xi_0}{\xi_1}{\Xi}>\pi$.
The latter contradicts \ref{ex:RisCBB(1)}.

Finally, if $\distfun_{\partial \Sigma_x}(\xi)= \tfrac\pi2$, then $\dist{\xi_0}{\xi_1}{\Xi}=\pi$.
Therefore, $\Cone \Xi$ contains a line in the directions of $\xi_0$ and $\xi_1$;
in other words, $\Xi$ is a spherical suspension with poles $\xi_0$ and $\xi_1$.
In particular, every point of $\Xi$ lies on distance at most $\tfrac\pi2$ from $\xi_0$ or $\xi_1$.
The natural projection $\Xi\to \Sigma_x$ does not increase distances and sends both  $\xi_0$ and $\xi_1$ to $\xi$.
Therefore, the second statement of \ref{SHORT.thm:doubling:concave}$_m$ follows.

\parit{\ref{SHORT.thm:doubling:doubling}$_{m-1}+$\ref{SHORT.thm:doubling:concave}$_{m-1}+$\ref{SHORT.thm:doubling:concave}$_m\Rightarrow$\ref{SHORT.thm:partial-grad:grad}$_m$.}
We can assume that $s=\nabla_xf\ne 0$.
By \ref{prop:grad-exist}, $\nabla_xf\z=s\cdot \overline{\xi}$, where $s=\dd_xf(\overline{\xi})>0$ and $\overline{\xi}\in\Sigma_p$ is the direction that maximizes $\dd_xf(\overline{\xi})$.

Let $\zeta\in \partial\Sigma_x$ be a direction that minimizes the angle $\mangle(\overline{\xi},\zeta)$.
It is sufficient to show that $\zeta=\overline{\xi}$.

Assume $\zeta\ne \overline{\xi}$;
let $\eta=\dir[\Sigma_x]\zeta{\overline{\xi}}$.
By \ref{SHORT.thm:doubling:concave}$_m$, $\mangle(\overline{\xi},\zeta)\le \tfrac\pi2$ and
\ref{SHORT.thm:doubling:concave}$_{m-1}$ implies that 
\[\mangle(\eta,\nu)\le \tfrac\pi2\eqlbl{eq:<pi/2}\]
for any $\nu\in \Sigma_\zeta\Sigma_x$ (if $m=2$, then the last statement is evident). 

Let $\phi\:\Sigma_x\to\RR$ be restriction of $\dd_xf$ to $\Sigma_x$.
Applying \ref{ex:d(distfun):<} and \ref{eq:<pi/2}, we get that $\dd_{\bar \xi}\phi(\eta)\le 0$.
Since $\dd_xf$ is concave, we have that $\phi''+\phi\le 0$.
If $\phi(\zeta)\le 0$, then it implies that $\phi(\overline{\xi})\le 0$ --- a contradiction to the fact that $s>0$.
If $\phi(\zeta)> 0$, then $\phi(\overline{\xi})<\phi(\zeta)$ --- a contradiction again.

\parit{\ref{SHORT.thm:partial-grad:grad}$_m\Rightarrow$\ref{SHORT.thm:partial-grad:flow}$_m$.}
Let $\alpha$ be an $f$-gradient curve and $\ell(t)=\distfun_{\partial \spc{A}}\alpha(t)$.

Choose $t$;
let $x=\alpha(t)$ and $y\in \partial\spc{A}$ be a closest point to $x$.
By \ref{SHORT.thm:partial-grad:grad}$_m$, we have that $\nabla_y f\in\partial \T_y$.
Since the distance $\dist{x}{y}{}$ is minimal, 
we get $\langle \dir yx,v\rangle\le 0$ for any $v\in \partial \T_y$.
In particular,
\[\langle \dir yx,\nabla_y f\rangle\le 0\]
Applying Exercise~\ref{ex:monotonicity} to $x$ and $y$, 
we get
\[\ell'(t)\le \ell(t)\]
if the left-hand side is defined.
Since $\ell$ is Lipschitz, $\ell'$ is defined almost everywhere.
Integrating the inequality, we get 
\[\ell(t)\le e^t\cdot\ell(0)\]
for any $t\ge 0$.
In particular, if $\ell(0)=0$, then $\ell(t)=0$ for any $t\ge 0$.
Since $\partial\spc{A}$ is closed (\ref{ex:bry-closed}), the statement follows.

\parit{\ref{SHORT.thm:partial-grad:flow}$_{m}+$\ref{SHORT.thm:doubling:doubling}$_{m-1}\Rightarrow$\ref{SHORT.thm:doubling:doubling}$_m$.}
We will consider the case $\kappa=0$;
other cases can be done in the same way, but formulas get more complicated.

Denote by $\spc{A}_0$ and $\spc{A}_1$ the two copies of $\spc{A}$ in $\spc{W}$;
let us keep the notation $\partial \spc{A}$ for the common boundary of $\spc{A}_0$ and $\spc{A}_1$.

\begin{clm}{}
Let $\gamma$ be a geodesic in $\spc{W}$.
Then either $\gamma$ has at most one interior point in $\partial \spc{A}$ or
$\gamma\subset \partial \spc{A}$.
\end{clm}

\begin{wrapfigure}{r}{45mm}
\vskip-2mm
\centering
\includegraphics{mppics/pic-1315}
\end{wrapfigure}

Indeed, assume $\gamma$ shares at least two points with $\partial \spc{A}$, say $x=\gamma(t_1)$ and $y=\gamma(t_2)$ and these are not endpoints of $\gamma$.
Remove from $\gamma$ the set $\gamma\cap \spc{A}_1$
and exchange it to its reflection across $\partial\spc{A}$;
denote the obtained curve by $\hat\gamma$.

Any arc of $\hat\gamma$ with one endpoint in $\partial \spc{A}$
is a geodesic in $\spc{A}_0$.
Since $x,y\in \partial \spc{A}$, the arc of $\hat\gamma$ behind $y$ lies in the image of map $t\mapsto \GF^t_{f_x}(y)$, where $f_x=\tfrac12\cdot\distfun^2_x$.
By \ref{SHORT.thm:partial-grad:flow}, this arc lies in $\partial\spc{A}$.

Now choose a point $z$ on this arc, so $z\in \partial\spc{A}$.
Applying the same argument, we get that the arc of $\hat\gamma$ before $y$ lies in $\partial\spc{A}$.
Hence the claim follows.\claimqeds

Choose a point $p$ in $\spc{W}$;
let $f\df\tfrac12\cdot\distfun_p^2$.
It is sufficient to show that $(f\circ\gamma)''\le 1$ for any $t$.
If $p\in \partial\spc{A}$, then the statement follows from function comparison in $\spc{A}_0$ and $\spc{A}_1$.
So, we can assume that $p\in \spc{A}_0\setminus \partial\spc{A}$.

If $\gamma$ lies in $\partial \spc{A}$, then this inequality follows from the comparison in~$\spc{A}_0$.

\begin{wrapfigure}{r}{55mm}
\vskip-2mm
\centering
\includegraphics{mppics/pic-1325}
\end{wrapfigure}

Choose $y=\gamma(t_0)$; without loss of generality we can assume that $t_0=0$.

If $y\z\in \spc{A}_0\setminus\partial\spc{A}$, then $(f\z\circ\gamma)''(0)\le 1$ in the barrier sense;
it follows from the comparison in $\spc{A}_0$.

Assume $y\in \spc{A}_1\setminus\partial\spc{A}$.
Suppose $[py]$ crosses $\partial\spc{A}$ at $x$.
Let $\Sigma_x$ be the space of directions of $\spc{A}$ at $x$,
and let $\Xi$ be its doubling.
As before, we denote by $\Sigma_0$ and $\Sigma_1$ two copies of $\Sigma_x$ in  $\Xi$
and keep notation $\partial\Sigma_x$ for their common boundary.
By \ref{SHORT.thm:doubling:doubling}$_{m-1}$, $\Xi$ is $\Alex1$.

The directions $\dir x{y}$ and $\dir xp$ lie on opposite sides from $\Xi$ and
\[\dist{\dir x{y}}{\dir xp}\Xi\ge \pi.\]
Otherwise, we could choose a direction $\xi\in\partial\Sigma$ such that
\[\dist{\dir x{y}}{\xi}\Xi+\dist{\xi}{\dir xp}\Xi<\pi.\]
Furthermore, we could consider the radial curve $\alpha(t)=\gexp_x(t\cdot \xi)$.
By \ref{SHORT.thm:partial-grad:flow}$_m$, $\alpha$ lies in $\partial \spc{A}$.
By \ref{prop:gexp}
\[\dist{p}{\alpha(s)}{\spc{A}_0}
+\dist{y}{\alpha(s)}{\spc{A}_1}
<\dist{p}{y}{\spc{W}}\]
for small values $s>0$
--- a contradition.

$\Cone \Xi$ contains a line with directions $\dir x{y}$ and $\dir xp$.
By the splitting theorem, $\Cone \Xi$ split in these directions;
in particular, 
\[\dist{\dir x{y}}{\xi}{}+\dist{\xi}{\dir xp}{}=\pi.\]
for any $\xi\in\Xi$.
It follows that for any $\xi\in\Xi$ there is $\xi'\in\partial\Sigma_x$ such that 
$\xi$ and $\xi'$ lie on some geodesic $[\dir x{y} \dir xp]_\Xi$.

Fix $t\approx 0$ such that $t\ne 0$; let $z=\gamma(t)$.
Choose such $\xi'$ for $\xi=\dir xz$.
Consider the radial curve $\alpha(s)\df\gexp_x(s\cdot\xi')$.
Let us show that 
\[
\begin{aligned}
\dist{p}{z}{\spc{W}}
&\le \dist{p}{\alpha(s)}{\spc{A}_0}+ \dist{\alpha(s)}{z}{\spc{A}_1}\le
\\
&\le\side\hinge yp{z}.
\end{aligned}
\eqlbl{eq:gamma''}
\]
for suitable value $s$.

The first inequality in \ref{eq:gamma''} is evident.
Set $\phi=\mangle\hinge{x}{y}{z}$ and $\psi\z=\mangle(\dir xp,\xi')$.
The choice of $s$ comes from the model configuration $\tilde p$, $\tilde x$, $\tilde y$, $\tilde w$, $\tilde z\in \EE^2$ such that
\begin{align*}
\tilde x&\in [\tilde p\tilde y],
&
\dist{\tilde p}{\tilde x}{}&=\dist{ p}{x}{},
&
\dist{\tilde p}{\tilde y}{}&=\dist{p}{y}{},
&
\dist{\tilde x}{\tilde z}{}&=\dist{x}{z}{},
\\
\tilde w&\in [\tilde p\tilde z],
&
\mangle\hinge{\tilde x}{\tilde y}{\tilde z}&=\phi,
&
\mangle\hinge{\tilde x}{\tilde p}{\tilde w}&=\psi, 
&
s&=\dist{\tilde x}{\tilde w}{}.
\end{align*}
\begin{figure}[ht!]
\vskip-0mm
\centering
\includegraphics{mppics/pic-1014}
\end{figure}

\noindent
By \ref{prop:gexp}, we get 
\begin{align*}
\dist{p}{\alpha(s)}{\spc{A}_0}&\le \dist{\tilde p}{\tilde w}{},
\\
\dist{\alpha(s)}{z}{\spc{A}_1}&\le\dist{\tilde w}{\tilde z}{};
\end{align*}
by the comparison, 
\[\dist{\tilde p}{\tilde z}{}\le \side\hinge ypz.\]

\begin{thm}{Exercise}\label{ex:pz<ypz}
Prove the last inequality.
\end{thm}

Hence we get $(f\circ\gamma)''(0)\le 1$ in the barrier sense.

Finally if $\gamma(0)\in\partial\spc{A}$, then splitting argument shows that 
\[(f\circ\gamma)^+(0)+(f\circ\gamma)^-(0)\le 0.\]

Summarizing, we get that $(f\circ\gamma)''\le 1$ on every arc of $\gamma$ that lies entirely in $\spc{A}_0$ or $\spc{A}_1$.
If $\gamma$ crosses $\partial \spc{A}$, then we know that it happens only once and at the crossing moment $t_0$ 
we have $f\circ\gamma^+(t_0)\z+f\circ\gamma^-(t_0)\z\le 0$.
All this implies that $(f\circ\gamma)''\le 1$.
\qeds

\begin{thm}{Exercise}\label{ex:bry-connected}
Let $\spc{A}$ be a finite-dimensional $\Alex1$ space of dimension $\ge 2$ with non-empty boundary $\partial\spc{A}$.
Show that $\partial\spc{A}$ is connected.
\end{thm}


\begin{thm}{Exercise}\label{ex:dist-to-bry}
Let $\spc{A}$ be an $m$-dimensional $\Alex0$ space with non-empty boundary $\partial\spc{A}$
for $2\le m<\infty$.
Show that the distance function to the boundary
\[\distfun_{\partial\spc{A}}\:\spc{A}\to\RR\]
is concave.
\end{thm}

\begin{thm}{Exercise}\label{ex:liberman}
Let $\spc{A}$ be a finite-dimensional $\Alex0$ space with non-empty boundary $\partial\spc{A}$.
Suppose $\gamma$ is a geodesic in $\partial\spc{A}$ with the induced length metric.
Show that the function $t\mapsto \tfrac12\cdot\distfun_p^2\circ\gamma(t)$ is 1-concave for any point $p$. 
\end{thm}

\begin{thm}{Exercise}\label{ex:native}
Let $\spc{W}$ be a doubling of finite-dimensional Alexandrov space $\spc{A}$ across its boundary,
and let $proj\:\spc{W}\to\spc{A}$ be the natural projection.
Suppose $f\:\spc{A}\to\RR$ is a $\lambda$-concave function.
Show that $f\circ\proj\:\spc{W}\to\RR$ is $\lambda$-concave if and only if $\nabla_xf\in \partial \T_x$ 
for any $x\in\partial \spc{A}$.
\end{thm}



\section{Remarks}

It easily follows by induction on dimension  that the doubling of a finite-dimensional Alexandrov space across its boundary results in an Alexandrov space without boundary.
This observation can often be used to reduce a statement about general finite-dimensional Alexandrov spaces to  Alexandrov spaces without boundary.

For spaces without boundary the following tools become available.

\begin{thm}{Fundamental-class lemma}\label{lem:fund-class}
Any compact finite-dimensional Alexandrov space $\spc{A}$ without boundary has a fundamental class with $\ZZ/2$ coefficients;
that is, if $\spc{A}$ is $m$-dimensional, then
\[H^m(\spc{A},\ZZ/2)=\ZZ/2.\]

\end{thm}

This lemma was proved by Karsten Grove and Peter Petersen \cite{grove-petersen1993}.
Originally it was stated for Alexander--Spanier cohomology. We do not make this distinction  because for compact Alexandrov spaces it is the same as singular cohomology. Indeed,  both cohomology theories are homotopy invariant \cite[Chapter 6]{Spanier}, compact Alexandrov spaces are homotopy equivalent to finite simplicial complexes \ref{thm:finite-dim-hom-simplicial} and  for paracompact  CW complexes  Alexander--Spanier cohomology is isomorphic to \v{C}ech  and singular cohomolgy \cite[Chapter 6]{Spanier}.

This lemma implies, for example, that on finite-dimensional Alexandrov spaces without boundary 
the gradient flow for a $\lambda$-concave function is an onto map;
in other words, gradient curves can be extended into the past.
It is also used in the proof of the following version of the domain invariance theorem \cite[Theorem 3.2]{kapovitch-zhu}.

\begin{thm}{Domain invariance}\label{thm-inv-domain}
Let $\spc{A}_1$ and $\spc{A}_2$ be two $m$-dimensional Alexandrov spaces with empty boundary; $m$ is finite.
Suppose $\Omega_1$ is an open subset in $\spc{A}_1$ and $f\:\Omega_1\to \spc{A}_2$ is an injective continuous map.
Then $f(\Omega_1)$ is open in $\spc{A}_2$.
\end{thm}

Theorem~\ref{thm:spherical-nbhd} can be used to prove the following. 

\begin{thm}{Topological stratification}\label{thm:top-stratification}
Any $m$-dimensional Alexandrov space with $m<\infty$ can be subdivided into topological manifolds $S_0,\z\dots,S_m$ such that for every $i$ we have $\dim S_i=i$ or $S_i=\emptyset$.
Moreover,
\begin{subthm}{}
the closure of $S_{m-1}$ is the boundary of the space, and
\end{subthm}

\begin{subthm}{}
$S_{m-2}=\emptyset$.
\end{subthm}

\end{thm}

Let us mention that this statement implies that a compact finite-dimensional Alexandrov space has the homotopy type of a finite CW complex,
but it seems to be unknown if it has to be homeomorphic to a CW complex.

The stratification theorem~\ref{thm:top-stratification} can be sharpened as follows.

\begin{thm}{Boundary characterization}
Let $\spc{A}$ be an $m$-dimensional Alexandrov space with $m<\infty$.
Then the following statements are equivalent.

\begin{subthm}{item-boundary} $p\in \partial \spc{A}$;
\end{subthm}

\begin{subthm}{item-contractible} $\Sigma_p$ is contractible;
\end{subthm}

\begin{subthm}{item-space-dir-homology} $\tilde H_{m-1}(\Sigma_p,\ZZ/2)= 0$;
\end{subthm}

\begin{subthm}{item-local-homology} $H_m(\spc{A},\spc{A}\setminus \{p\},\ZZ/2)= 0$;
\end{subthm}

\end{thm}

Let $f$ be a semiconcave function.
A point $p\in \Dom f$ is called \index{critical point}\emph{critical} point of $f$ if $\dd_pf\le 0$; 
otherwise it is called \index{regular point}\emph{regular}.

The following statement \footnote{\red add reference? А: Добрый Витя обещал найти :)} plays a technical role in the proof of stability theorem,
but it is also a useful technical tool on its own.

\begin{thm}{Morse lemma}
Let $f$ be a semiconcave function on a finite-dimensional Alexandrov space without boundary.
Suppose $K$ is a compact set of regular points of $f$ in its level set $f=a$.
Then an open neighborhood $\Omega$ of $K$ admits a homeomorphism $x\mapsto (h(x),f(x))$ to a product space $\Lambda\times (a-\eps,a+\eps)$.
\end{thm}

Subsets in Alexandrov spaces that satisfy the condition in \ref{thm:partial-grad:flow} are called extremal.
More precisely, a subset $E$ is \index{extremal set}\emph{extremal} if for any $x\in E$
and $f$-gradient curve that starts in $E$ remains in $E$;
here $f$ is arbitrary function of the form $\tfrac12\cdot \distfun_p^2$. %{\red V: should we add the condition that $E$ is closed?} A: No, it follows.}

Extremal subsets were introduced by Grigory Perelman and the second author \cite{perelman-petrunin}.
They will pop up in the next lecture.

The following conjecture is one of the oldest questions in Alexandrov geometry that remains open.

\begin{thm}{Conjecture}
Let $S$ be a component of the boundary of a finite-dimensional Alexandrov space.
Then $S$ equipped with the induced length metric is an Alexandrov space with the same curvature bound.
\end{thm}

The doubling theorem has several generalizations \cite{petrunin1997,ge-li} that allow to glue nonidentical spaces.


%%!TEX root = the-quotients.tex
\chapter{Quotients}\label{chap:L/G}

This lecture gives several applications of Alexandrov geometry to isometric group actions.

\section{Quotient space}

Suppose that a group $G$ acts isometrically on a metric space $\spc{X}$.
Note that
\[\dist{G\cdot x}{G\cdot y}{\spc{X}/G}
\df
\inf
\set{\dist{x}{g\cdot y}{\spc{X}}}{g\in G}\]
defines a semimetric on the orbit space $\spc{X}/G$.
Moreover, if the orbits of the action are closed,
then it is a genuine metric.

\begin{thm}{Theorem}\label{thm:CBB/G}
Suppose that a group $G$ acts isometrically on a proper $\Alex0$ space $\spc{A}$, and $G$ has closed orbits.
Then the quotient space $\spc{A}/G$ is $\Alex0$.

\end{thm}

A more general formulation will be given in \ref{thm:submetry-CBB-1}.

\parit{Proof.}
Denote by $\sigma\:\spc{A}\to \spc{A}/G$ the quotient map.

Fix a quadruple of points $p,x_1,x_2,x_3\in \spc{A}/G$.
Choose $\hat p\in \spc{A}$ such that $\sigma(\hat{p})=p$.
Since $\spc{A}$ is proper, we can choose  points $\hat{x}_i\in \spc{A}$ such that $\sigma(\hat x_i)=x_i$ and
\[\dist{p}{x_i}{\spc{A}/G}
=
\dist{\hat{p}}{\hat{x}_i}{\spc{A}}\]
for all $i$.

Note that 
\[\dist{x_i}{x_j}{\spc{A}/G}
\le 
\dist{\hat{x}_i}{\hat{x}_j}{\spc{A}}
\]
for all $i$ and $j$.
Therefore 
\[\angk p{x_i}{x_j}
\le
\angk {\hat{p}}{\hat{x}_i}{\hat{x}_j}
\eqlbl{eq:angles-M-L}\]
for all $i$ and $j$.

By $\EE^2$-comparison in $\spc{A}$,
we have
\[\angk {\hat{p}}{\hat{x}_1}{\hat{x}_2}
+\angk {\hat{p}}{\hat{x}_2}{\hat{x}_3}
+\angk {\hat{p}}{\hat{x}_3}{\hat{x}_1}
\le 
2\cdot\pi.\]
Applying  \ref{eq:angles-M-L}, 
we get 
\[\angk p{x_1}{x_2}
+\angk p{x_2}{x_3}
+\angk p{x_3}{x_1}\le 2\cdot\pi;\]
that is,
the $\EE^2$-comparison holds for any quadruple in $\spc{A}/G$.
\qeds

\begin{thm}{Very advanced exercise}\label{ex:Hilbert/G}
Let $G$ be a compact Lie group with a bi-invariant Riemannian metric.
Show that $G$ is isometric to a quotient of a Hilbert space by an isometric group action.

Conclude that $G$ is $\Alex0$.
\end{thm}

\section{Submetries}

A map $\sigma\:\spc{X}\to\spc{Y}$ between metric spaces $\spc{X}$ and $\spc{Y}$
is called a \index{submetry}\emph{submetry} if 
\[\sigma(\oBall(p,r)_\spc{X})=\oBall(\sigma(p),r)_{\spc{Y}}\]
for any $p\in \spc{X}$ and $r\ge 0$.

Suppose $G$ and $\spc{A}$ are as in \ref{thm:CBB/G}.
Observe that the quotient map $\sigma\:\spc{A}\to \spc{A}/G$ is a submetry.
The following two exercises show that this is not the only source of submetries. 

\begin{thm}{Exercise}\label{ex:sumbetries(S^2)}
Construct submetries
\begin{subthm}{ex:sumbetries(S^2):1}
$\sigma_1\:\mathbb{S}^2\to[0,\pi]$,
\end{subthm}
\begin{subthm}{ex:sumbetries(S^2):2}
$\sigma_2\:\mathbb{S}^2\to[0,\tfrac\pi2]$,
\end{subthm}
\begin{subthm}{ex:sumbetries(S^2):n}
$\sigma_n\:\mathbb{S}^2\to[0,\tfrac\pi n]$ (for integer $n\ge 1$)
\end{subthm}
such that the fibers $\sigma_n^{-1}\{x\}$ are connected for any $x$.
\end{thm}

\begin{thm}{Exercise}\label{ex:sumbetries(E^2)}
Let $\sigma\:\EE^2\to [0,\infty)$ be a submetry.
Show that $K\z=\sigma^{-1}\{0\}$ is a closed convex set without interior points and $\sigma(x)\z=\distfun_Kx$.
\end{thm}

The proof of \ref{thm:CBB/G} works for submetries;
that is, \textit{if $\sigma\:\spc{A}\to\spc{B}$ is a submetry and $\spc{A}$ is a proper $\Alex0$ space, then so is $\spc{B}$}.
Theorem \ref{thm:CBB/G} admits a straightforward generalization to $\Alex{-1}$ case.

In the $\Alex1$ case, the proof produces a slightly weaker statement ---  \textit{$\SSS^2$-comparison holds for a quartuple $p,x_1,x_2,x_3$ in the quotient of $\Alex1$ if $\dist{p}{x_i}{}<\tfrac\pi 2$ for each $i$}.
In particular, the quotient space is \textit{locally} $\Alex1$.
But since $\Alex1$ space is geodesic, then so is its quotient.
Therefore, the globalization theorem implies that it is globally $\Alex1$.
The same holds for the targets of submetries from an  $\Alex1$ space.
With a bit of extra work, one can extend the statement to nonproper spaces \cite[8.34]{alexander-kapovitch-petrunin2024}.
Thus, we have the following.

\begin{thm}{Theorem}\label{thm:submetry-CBB-1}
Let $\sigma\:\spc{A}\to\spc{B}$ be a submetry.
If $\spc{A}$ is $\Alex\kappa$ space, then so is $\spc{B}$.

In particular, if $G$ acts isometrically on an $\Alex\kappa$ space $\spc{A}$, and $G$ has closed orbits.
Then the quotient space $\spc{A}/G$ is $\Alex\kappa$.
\end{thm}

\section{Hopf's conjecture}

\textit{Does $\mathbb{S}^2\times\mathbb{S}^2$ admit a Riemannian metric with positive sectional curvature?} \index{Hopf's conjecture}\emph{Hopf's conjecture} says that the answer should be negative.
Let us take a close look at the following partial result obtained by Wu-Yi Hsiang and Bruce Kleiner \cite{hsiang-kleiner}.

\begin{thm}{Theorem}\label{thm:hsiang-kleiner}
There is no Riemannian metric on $\SSS^2\times\SSS^2$ with sectional curvature $\ge 1$ and a nontrivial isometric $\SSS^1$-action.
\end{thm}

Reacall that a group action $G\acts\spc{X}$ is called \index{effective action}\emph{effective} if for any $g\in G$ there is $x\in\spc{X}$ such that $g\cdot x\ne x$.

\begin{thm}{Key lemma}\label{lem:S^3/S^1}
Suppose $\SSS^1\acts\SSS^3$ is an effective isometric action without fixed points
and $\Sigma=\SSS^3/\SSS^1$ is its quotient space.
Then there is a distance noncontracting map $\Sigma\to \tfrac12\cdot \SSS^2$, where $\tfrac12\cdot \SSS^2$ is the standard 2-sphere rescaled with a factor $\tfrac12$.
\end{thm}

The proof of the lemma is guided by the following exercise.

\begin{thm}{Exercise}\label{ex:S^3/S^1}
Suppose $\SSS^1\acts\SSS^3$ is an effective isometric action without fixed points.
Let us think   of $\SSS^3$ as the unit sphere in $\RR^4$.

\begin{subthm}{ex:S^3/S^1:pq}
Show that one can identify $\RR^4$ with $\CC^2$ so that the action
is given by matrix multiplication
\[\left(\begin{matrix}
u^p&0\\
0& u^q
\end{matrix}
\right),\]
where $(p,q)$ is a pair of relatively prime positive integers and $u\in \SSS^1=\set{z\in\CC}{|z|=1}$.
In particular, our $\SSS^1$ is a subgroup of the torus that acts by
matrix multiplication
\[\left(\begin{matrix}
v&0\\
0& w
\end{matrix}
\right),\]
where  $v,w\in \SSS^1$.
\end{subthm}

\smallskip

\noindent Fix $p$ and $q$ as above.
Let $\Sigma_{p,q}=\SSS^3/\SSS^1$ be the quotient space.

\smallskip

\begin{subthm}{ex:S^3/S^1:sphere}
Show that the $\Sigma_{p,q}=\SSS^3/\SSS^1$ is a topological sphere with $\SSS^1$-symmetry.
This symmetry has two fixed points, north pole and south pole, that correspond to the orbits of $(1,0)$ and $(0,1)$ in $\SSS^3$.
\end{subthm}

\smallskip

\noindent Denote by $S(r)$ the circle of radius $r$ with the center at the north pole of $\Sigma_{p,q}$.

\begin{subthm}{ex:S^3/S^1:a}
Denote by $T(r)$ the inverse image $T(r)$ in $\SSS^3$, and let $a(r)$ be its area.
Show that $T(r)$ is an orbit of the torus action and
\[a(r)=\pi^2\cdot\sin r\cdot \cos r.\]

\end{subthm}

\smallskip

\begin{subthm}{ex:S^3/S^1:b}
Let $b_{p,q}(r)$ be the length of the $\SSS^1$-orbit in $\SSS^3$ that corresponds to a point on $S(r)$. 
Show that
\[b_{p,q}=\pi\cdot\sqrt{(p\cdot \sin r)^2+(q\cdot \cos r)^2}.\]
\end{subthm}

\smallskip

\begin{subthm}{ex:S^3/S^1:c}
Let $c_{p,q}(r)$ be the length of $S(r)$.
Show that $a(r)=c_{p,q}(r)\cdot b_{p,q}(r)$.
\end{subthm}

\smallskip

\begin{subthm}{ex:S^3/S^1:cc}
Show that $c_{p,q}(r)\le c_{1,1}(r)$ for any pair $(p,q)$ of relatively prime positive integers.
Use it to construct a distance noncontracting map $\Sigma_{p,q}\to \tfrac12\cdot \SSS^2\iso\Sigma_{1,1}$.
\end{subthm}

\end{thm}

\parit{Proof of \ref{thm:hsiang-kleiner}.}
Assume $\spc{B}=(\SSS^2\times\SSS^2,g)$ is a counterexample.
By the Toponogov theorem, $\spc{B}$ is $\Alex1$.
By \ref{thm:CBB/G}, the quotient space $\spc{A}\z=\spc{B}/\SSS^1$ is $\Alex1$;
evidently, $\spc{A}$ is 3-dimensional.

Denote by $F\subset \spc{B}$ the fixed point set of the $\SSS^1$-action.
Then $\chi(\spc{B})\z=\chi(F)$.
Each connected component of $F$ is either an isolated point or a 2-dimensional geodesic submanifold in $\spc{B}$;
the latter has to have positive curvature, and therefore it is homeomorphic to $\SSS^2$ or $\RP^2$.
Notice that 
\begin{itemize}
 \item each isolated point contributes 1 to the Euler characteristic of~$\spc{B}$,
 \item each sphere contributes 2 to the Euler characteristic of $\spc{B}$, and
 \item each projective plane contributes 1 to the Euler characteristic of~$\spc{B}$.
\end{itemize}
Since $\chi(\spc{B})=4$, we are in one of the following three cases:
\begin{enumerate}
 \item\label{case1} $F$ has exactly 4 isolated points,
 \item\label{case2} $F$ has one 2-dimensional submanifold and at least 2 isolated points,
 \item\label{case3} $F$ has at least two 2-dimensional submanifolds.
\end{enumerate}
In each case we will arrive at a contradiction.

\parit{Case \ref{case1}.}
Suppose $F$ has exactly 4 isolated points $x_1$, $x_2$, $x_3$, and $x_4$.
Denote by $y_1$, $y_2$, $y_3$, and $y_4$ the corresponding points in $\spc{A}$.
Note that $\Sigma_{y_i}\spc{A}$ is isometric to a quotient of $\SSS^3$ by an isometric $\SSS^1$-action without fixed points.

By \ref{ex:S^3/S^1}, each angle $\mangle\hinge{y_i}{y_j}{y_k}\le \tfrac\pi2$ for any three distinct points 
$y_i$, $y_j$, $y_k$.
In particular, all four triangles $[y_1y_2y_3]$, $[y_1y_2y_4]$, $[y_1y_3y_4]$, and $[y_2y_3y_4]$ are nondegenerate.
By the comparison, the sum of angles in each triangle is strictly greater than $\pi$.

Denote by $\omega$ the sum of all 12 angles in the 4 triangles $[y_1y_2y_3]$, $[y_1y_2y_4]$, $[y_1y_3y_4]$, and $[y_2y_3y_4]$.
From above,
\[\omega>4\cdot\pi.\]

On the other hand, by \ref{ex:S^3/S^1} any triangle in $\Sigma_{y_1}\spc{A}$ has perimeter at most $\pi$.
In particular, 
\[\mangle\hinge{y_1}{y_2}{y_3}+\mangle\hinge{y_1}{y_3}{y_4}+\mangle\hinge{y_1}{y_4}{y_2}\le \pi.\]
Apply the same argument in $\Sigma_{y_2}\spc{A}$, $\Sigma_{y_3}\spc{A}$, and $\Sigma_{y_4}\spc{A}$;
adding the results, we get 
\[\omega\le 4\cdot\pi\]
--- a contradiction.

\parit{Case \ref{case2}.}
Suppose $F$ contains one surface $S$.
Then the projection of $S$ to $\spc{A}$ forms its boundary $\partial \spc{A}$.
The doubling $\spc{W}$ of $\spc{A}$ across its boundary has at least 4 singular points --- each singular point of $\spc{A}$ corresponds to two singular points of $\spc{W}$.

By the doubling theorem, $\spc{W}$ is a $\Alex1$ space.
Therefore we arrive at a contradiction in the same way as in the first case.

\parit{Case \ref{case3}.} Impossible by \ref{ex:bry-connected}.
\qeds

\section{Erdős' problem rediscovered}

A point $p$ in an Alexandrov space is called \index{extremal point}\emph{extremal} if $\mangle\hinge pxy\le \tfrac\pi2$ for any hinge $\hinge pxy$ with the vertex at $p$; equivalently, $\diam \Sigma_p\le \pi/2$.

\begin{thm}{Theorem}\label{thm:extr-point}
Let $\spc{A}$ be a compact $m$-dimensional $\Alex0$ space.
Then it has at most $2^m$ extremal points.
\end{thm}

\parit{Proof of \ref{thm:extr-point}.}
Let $\{p_1,\dots,p_N\}$ be extremal points in $\spc{A}$.
For each $p_i$ consider its open \index{Voronoi domain}\emph{Voronoi domain} $V_i$; that is, 
\[V_i=\set{x\in \spc{A}}{\dist{p_i}{x}{}<\dist{p_j}{x}{}\ \text{for any}\ j\not=i}.\]
Clearly $V_i\cap V_j=\emptyset$ if $i\not=j$.

Suppose  $0<\alpha\le 1$.
Given a point $x\in\spc{A}$, choose a geodesic $[p_ix]$ and denote by $x_i$ the point on $[p_ix]$ such that $\dist{p_i}{x_i}{}=\alpha\cdot\dist{p_i}{x}{}$;
let $\map_i\:x\to x_i$ be the corresponding map.
By the comparison, 
\[\dist{x_i}{y_i}{}\ge\alpha\cdot \dist{x}{y}{}\]
for any $x$, $y$, and $i$.
Therefore 
\[\vol(\map_i \spc{A})\ge\alpha^m\cdot\vol \spc{A}.\]

Suppose $\alpha<\tfrac12$.
Then $x_i\in V_i$ for any $x\in \spc{A}$.
Indeed, assume $x_i\notin V_i$,
then there is $p_j$ such that $\dist{p_i}{x_i}{}\ge\dist{p_j}{x_i}{}$.
Then by comparison, we have $\angk{p_j}{p_i}{x}_{\EE^2}>\tfrac\pi2$;
that is, $p_j$ is not an extremal point.

It follows that $\vol V_i\ge\alpha^m\cdot\vol \spc{A}$
for any $0<\alpha<\tfrac12$; hence 
\[\vol V_i\ge\tfrac1{2^m}\cdot\vol \spc{A}.\]
Since $V_1,\dots,V_N$ are disjoint subsets of $\spc{A}$, we have $N\le 2^m$.
\qeds


\section{Crystallographic actions}

An isometric action $\Gamma\acts \EE^m$ is called \index{crystallographic action}\emph{crystallographic} if it is 
\index{properly discontinuous}\emph{properly discontinuous} (that is, for any compact set $K\subset \EE^m$ and $x\z\in \EE^m$ there are only finitely many elements $g\in \Gamma$ such that $g\cdot x\in K$) and \emph{cocompact} (that is, the quotient space $\spc{A}=\EE^m/\Gamma$ is compact).

Let $F$ be a maximal finite subgroup of $\Gamma$;
that is, if $F<H<\Gamma$ for a finite group $H$, then $F=H$.
Denote by $\mathfrak{M}(\Gamma)$ the number of maximal finite subgroups of $\Gamma$ up to conjugation.

\begin{thm}{Open question}
Let $\Gamma\acts \EE^m$ be a crystallographic action.
Is it true that $\mathfrak{M}(\Gamma)\le 2^m$?
\end{thm}

Note that any finite subgroup $F$ of $\Gamma$ fixes an affine subspace $A_F$ in $\EE^m$.
If $F$ is maximal, then $A_F$ completely describes $F$.
Indeed, since the action is properly discontinuous, the subgroup of $\Gamma$ that fix $A_F$ has to be finite.
This subgroup must contain $F$, but since $F$ is maximal, it must coinside with $F$. 

Denote by $\mathfrak{M}_k(\Gamma)$ the number of maximal finite subgroups $F<\Gamma$ (up to conjugation) such that $\dim A_F=k$.

Choose a finite subgroup $F<\Gamma$; consider a conjugate subgroup $F'=g \cdot F \cdot g^{-1}$.
Note that $A_{F'}=g\cdot A_F$.
In particular, the subspaces $A_F$ and $A_{F'}$ have the same image in the quotient space $\spc{A}=\EE^m/\Gamma$.
Therefore, to count subgroups up to conjugation, we need to count the images of their fixed sets.
By the lemma below (\ref{lem:extr/G}), $\mathfrak{M}_0(\Gamma)$ cannot exceed the number of extremal points in $\spc{A}=\EE^m/\Gamma$.
Combining this observation with \ref{thm:extr-point}, we get the following.

\begin{thm}{Proposition}\label{prop:2m}
Let $\Gamma\acts \EE^m$ be a crystallographic action.
Then $\mathfrak{M}_0(\Gamma)\le 2^m$.
\end{thm}

\begin{thm}{Lemma}\label{lem:extr/G}
Let $\Gamma\acts \EE^m$ be a crystallographic action and $F$ be a maximal finite subgroup of $\Gamma$ that fixes an isolated point $p$.
Then the image of $p$ in the quotient space $\spc{A}=\EE^m/\Gamma$ is an extremal point.
\end{thm}

\parit{Proof.}
Let $q$ be the image of $p$.
Suppose $q$ is not extremal;
that is, $\mangle \hinge q{y_1}{y_2}>\tfrac\pi2$ for some hinge $\hinge q{y_1}{y_2}$ in $\spc{A}$.

Choose the inverse images $x_1,x_2\in \EE^m$ of $y_1,y_2\in \spc{A}$ such that $\dist{p}{x_i}{\EE^m}=\dist{q}{y_i}{\spc{A}}$.
Note that $\mangle \hinge p{x_1}{x_2}\ge \mangle \hinge q{y_1}{y_2}>\tfrac\pi2$.
Moreover, since $p$ is fixed by $F$, we have
\[\mangle \hinge p{x_1}{g\cdot x_2}>\tfrac\pi2
\eqlbl{eq:>pi/2}\]
for any $g\in F$.

Denote by $z$ the barycenter of the orbit $F\cdot x_2$.
Note that $z$ is a fixed point of $F$.
By \ref{eq:>pi/2}, $z\ne p$;
so $F$ must fix the line $pz$.
But $p$ is an isolated fixed point of $F$ --- a contradiction.
\qeds

\begin{thm}{Exercise}\label{ex:number(m-1)}
Let $\Gamma\acts \EE^m$ be a crystallographic action.
Show that
\begin{subthm}{ex:number(m-1):2}
$\mathfrak{M}_{m-1}(\Gamma)\le 2$, and
\end{subthm}

\begin{subthm}{ex:number(m-1):1}
if $\mathfrak{M}_{m-1}(\Gamma)=1$, then $\mathfrak{M}_0(\Gamma)\le 2^{m-1}$.
\end{subthm}

Construct  crystallographic actions with equalities in \ref{SHORT.ex:number(m-1):2} and \ref{SHORT.ex:number(m-1):1}.
\end{thm}

\section{Remarks}

Submetries were introduced by Valerii Berestovskii \cite{berestovskii1987} and have attracted attention in various contexts of differential and metric geometry.



A more general form of Theorem \ref{thm:hsiang-kleiner} was found by Karsten Grove and Burkhard Wilking \cite{grove-wilking};
it classifies isometric $\SSS^1$ actions on  4-dimensional manifolds with nonnegative sectional curvature.
This proof is as beautiful as the original work of Wu-Yi Hsiang and Bruce Kleiner.

It is expected that \textit{no $\Alex1$ space with a nontrivial isometric $\SSS^1$-action can be homeomorphic to $\SSS^2\times\SSS^2$};
so \ref{thm:hsiang-kleiner} holds for general $\Alex1$ space.
The proof of \ref{thm:hsiang-kleiner} would work if we had the following generalization of \ref{lem:S^3/S^1};
see \cite{harvey-searle}.

\begin{thm}{Open question}
Let $\Sigma$ be an $\Alex1$ space homeomorphic to $\SSS^3$.
Suppose $\SSS^1$ acts on $\Sigma$ isometrically and without fixed points.
Is it true that any triangle in $\Sigma/\SSS^1$ has perimeter at most $\pi$?

And if the answer is, is there a distance-noncontracting map
\[\Sigma/\SSS^1\z\to \tfrac12\cdot\SSS^2?\]
\end{thm}


\begin{thm}{Advanced exercise}\label{ex:S1actsS3}
Suppose $\SSS^1$ acts isometrically on an $\Alex1$ space $\spc{A}$ that is homeomorphic to $\SSS^3$.
Assume its fixed-point set is a closed local geodesic $\gamma$.
Show that
\[\length\gamma\le2\cdot\pi.\]
\end{thm}

An analogous question for a $\ZZ_2$-action is open \cite{petrunin-involution}.

Theorem \ref{thm:extr-point} is a translation of the following classical problem in discrete geometry to Alexandrov's language.

\begin{thm}{Problem}\label{erdos-problem}
Let $F$ be a set of points in $\EE^m$ such that any triangle formed by three distinct points in $F$ has no obtuse angles.
Then  $|F|\le2^m$.
Moreover, if $|F|=2^m$, then $F$ consists of the vertices of an $m$-dimensional rectangle.
\end{thm}

This problem was posed by Paul Erdős \cite{erdos} and solved by Ludwig Danzer and Branko Gr\"unbaum \cite{danzer-gruenbaum}.
Grigory Perelman noticed that, after proper definitions, the same proof works in Alexandrov spaces \cite{perelman-Erdos}; thus, it proves \ref{thm:extr-point}.
Applying the our argument to the convex hull of $F$ in \ref{erdos-problem} proves that $|F|\le 2^m$;
the case of equality requires more work.

Compact $m$-dimensional $\Alex0$ spaces with the maximal number of extremal points include $m$-dimensional rectangles and the quotients of flat tori by reflections across a point.
(This action has $2^m$ isolated fixed points; each corresponds to an extremal point in the quotient space $\spc{A}=\TT^m/\ZZ_2$.)
Nina Lebedeva has proved \cite{lebedeva2015} that \textit{every $m$-dimensional $\Alex0$ space with $2^m$ extremal points is a quotient of Euclidean space by a crystallographic action}.

The extremal subsets of Alexandrov space were brifly discussed in \ref{sec:bry-remarks}.
The following definition is more relevant to isometric group actions.

A closed subset $E$ in a finite-dimensional Alexandrov space is called
\index{extremal set}\emph{extremal} if $\mangle\hinge pxy\z\le \tfrac\pi2$ for any $x\notin E$ and $p\in E$ such that $\dist{x}{p}{}$ takes a minimal value.
An extremal set is called \index{minimal extremal set}\emph{minimal} if it contains no proper extremal subsets.

For example, the whole space and the empty set are extremal.
Also, every vertex, edge, or face (as well as their unions) of the cube is an extremal subset of the cube.
Vertices of the cube are its only minimal extremal subsets.

Counting maximal finite subgroups in a crystallographic group $\Gamma$ (up to conjugation) is equivalent to counting the minimal extremal subsets in the quotient space $\spc{A}=\EE^m/\Gamma$.
So, \ref{prop:2m} would follow from the next conjecture.

\begin{thm}{Conjecture}
Any $m$-dimensional compact $\Alex0$ space has at most $2^m$ minimal extremal subset.
\end{thm}

Let us mention another related conjecture.
An extremal set is called \index{primitive extremal set}\emph{primitive} if it contains no proper extremal subsets with nonempty relative interior.
For example, each face of $m$-dimensional cube is its primitive extremal subset;
therefore the cube has exactly $3^m$ primitive extremal subset, including the empty set and the whole cube.

\begin{thm}{Conjecture}
Any $m$-dimensional compact $\Alex0$ space has at most $3^m$ minimal extremal subset.
\end{thm}

Some crude estimates on number of extremal subsets follow from the idea in Gromov's Betti number theorem \ref{thm:betti}.


%\chapter{CBB: definition}

\section{Distances and geodesics}

\parbf{Distances.}
The distance between two points $x$ and $y$ in a metric space $\spc{X}$ will be denoted by $\dist{x}{y}{}$ or $\dist{x}{y}{\spc{X}}$.
The latter notation is used if we need to emphasize 
that the distance is taken in the space~${\spc{X}}$.
The function $(x,y)\mapsto \dist{x}{y}{\spc{X}}$ is called \index{metric}\emph{metric};
it has to meet the following conditions for any three points $x,y,z\in \spc{X}$:

\begin{subthm}{metric>=0}
$\dist{x}{y}{\spc{X}}\ge 0$,
\end{subthm}

\begin{subthm}{metric=0} $\dist{x}{y}{\spc{X}}= 0$ $\iff$ $x=y$,
\end{subthm}

\begin{subthm}{metric:sym} $\dist{x}{y}{\spc{X}}=\dist{y}{x}{\spc{X}}$,
\end{subthm}

\begin{subthm}{metric:triangle} $\dist{x}{y}{\spc{X}}+\dist{y}{z}{\spc{X}}\ge\dist{x}{z}{\spc{X}}$.
\end{subthm}

\parbf{Geodesics.}
Let $\II$\index{$\II$} be a real interval. 
A distance-preserving map $\gamma$ from $\II$ to a metric space $\spc{X}$ is called a \index{geodesic}\emph{geodesic}%
\footnote{Others call it differently: \textit{shortest path}, \textit{minimizing geodesic}.
Also, note that the meaning of the term \textit{geodesic} is different from what is used in Riemannian geometry, altho they are closely related.}; 
in other words, $\gamma\:\II\to \spc{X}$ is a geodesic if 
\[\dist{\gamma(s)}{\gamma(t)}{\spc{X}}=|s-t|\]
for any pair $s,t\in \II$.

If $\gamma\:[a,b]\to \spc{X}$ is a geodesic such that $p=\gamma(a)$, $q=\gamma(b)$, then we say that $\gamma$ is a geodesic from $p$ to $q$.
In this case, the image of $\gamma$ is denoted by $[p q]$\index{$[{*}{*}]$}, and, with abuse of notations, we also call it a \index{geodesic}\emph{geodesic}.
We may write $[p q]_{\spc{X}}$ 
to emphasize that the geodesic $[p q]$ is in the space  ${\spc{X}}$.

In general, a geodesic from $p$ to $q$ need not exist and if it exists, it need not  be unique.  
However, once we write $[p q]$ we assume that we have chosen such geodesic.

\parbf{Geodesic path.}
A \index{geodesic path}\emph{geodesic path} is a geodesic with constant-speed parameterization by the unit interval $[0,1]$.

\parbf{Geodesic space.}
A metric space is called \index{geodesic space}\emph{geodesic} if any pair of its points can be joined by a geodesic.

\section{Baby Toponogov}

Recall that \index{polyhedral space}\emph{polyhedral space} is a geodesic space that admits a finite triangulation such that each simplex is isometric to a simplex in a Euclidean space.
If, in addition, it is homeomorphic to a surface (without boundary), then it is called a \index{polyhedral surface}\emph{polyhedral surface}.
A point on a polyhedral surface with nonzero curvature is called an \index{essential vertex}\emph{essential vertex}.
Any other point on the surface will be called \index{regular point}\emph{regular}.
Note that \textit{any regular point has a neighborhood that is isometric to an open set in the Euclidean plane}.

\begin{thm}{Exercise}\label{ex:poly+geod}
Let $P$ be a non-negatively curved polyhedral surface.

\begin{subthm}{}
Show that a geodesic in $P$ cannot pass thru an essential vertex.
\end{subthm}

\begin{subthm}{}
Show that if two geodesics in $P$ intersect at two points, 
then these are the endpoints for both geodesics.
\end{subthm}

\end{thm}

The next theorem gives a global geometric property of non-negatively curved polyhedral surfaces.

Given a hinge $\hinge pxy$ in a non-negatively curved polyhedral surface $P$, denote by $\mangle\hinge pxy$ the minimal angle that the hinge cuts from $P$ at~$p$.
(Soon we will give a more general definition of $\mangle\hinge pxy$; see \ref{sec:angles}.)

\begin{thm}{Theorem}\label{thm:poly-cbb}
Let $P$ be a polyhedral surface.
Assume $P$ has non-negative curvature at each point (see \ref{sec:Alexandrov-existence}).
Then 
\[\mangle\hinge pxy\ge\angk pxy\]
for any hinge $\hinge pxy$ in $P$.
\end{thm}

The following exercise will be used in the proof.

\begin{thm}{Exercise}\label{ex:concave-loc}
Let $f\:[0,\ell]\to\RR$ be a continuous function such that for any $t\in \left]0,\ell\right[$ there is a linear function $h$ that locally supports $f$ from above;
that is, $h(t_0)=f(t_0)$, and there is $\eps>0$ such that $h(t)\ge f(t)$ if $|t-t_0|<\eps$.
Show that $f$ is concave.
\end{thm}


\parit{Proof.}
Let $[pxy]$ be a triangle in $P$ and let $[\tilde p\tilde x\tilde y]$ be the model triangle of $[pxy]$.
Set $\ell=|x-y|_P=|\tilde x-\tilde y|_{\EE^2}$.

Denote by $\gamma(t)$ and $\tilde \gamma(t)$ the geodesics $[xy]$ and $[\tilde x\tilde y]$ parametrized by length starting from $x$ and $\tilde x$, respectively.
Observe that it is sufficient to show that 
$$| p- \gamma(t)|\le|\tilde p-\tilde \gamma(t)| 
\eqlbl{eq:comp-gamma}$$
for any $t$ in $[0,\ell]$.

We may assume that $p$ is a regular point;
otherwise, move it slightly and apply approximation.


From the cosine law, we get that the function 
$$\tilde f(t)=|\tilde p-\tilde \gamma(t)|^2-t^2$$
is linear.
Consider the function
$$f(t)=|p- \gamma(t)|^2-t^2.$$
Note that $f(0)=\tilde f(0)$, $f(\ell)=\tilde f(\ell)$, and the inequality~\ref{eq:comp-gamma} is equivalent to
$$f(t)\ge \tilde f(t).
\eqlbl{eq:comp-f}$$
By Jensen's inequality, \ref{eq:comp-f} holds if $f$ is concave.

By \ref{ex:poly+geod}, 
$\gamma(t_0)$ is regular.
Since $p$ is regular,
a geodesic $[p\gamma(t)]$ contains only regular points.
Therefore for small $\eps>0$,
 the $\eps$-neighborhood of $[p\gamma(t)]$, say $\Omega$, contains only regular points. 
We may assume that $\Omega$ is homeomorphic to a disc;
in this case, there is a locally distance-preserving embedding $\iota\:\Omega\to\EE^2$.
Note the image $\iota[p\gamma(t)]$ is a line segment that 
and $\iota(\Omega)$ is the $\eps$-neighborhood of $\iota[p\gamma(t)]$ in $\EE^2$;
in particular, $\iota(\Omega)$ is convex.
Thus $\iota(\Omega)$ contains a triangle with  base $\iota[\gamma(t_0-\eps)\ \gamma(t_0+\eps)]$  and vertex $\iota(p)$.

Clearly, for any $t\in[t_0-\eps,t_0+\eps]$ 
we have 
$$|\iota(p)-\iota(\gamma(t))|\ge|p-\gamma(t)|.$$
Note that
the function
$$h(t)= |\iota(p)-\iota(\gamma(t))|^2-t^2$$
is linear.
From above, $h$ supports $f$ locally  at $t_0$.
It remains to apply~\ref{ex:concave-loc}.
\qeds

\section{Definition}

\begin{thm}{Definition}\label{def:CBB}
A metric space $\spc{X}$ has {}\emph{nonnegative curvature} in the sense of Alexandrov if the inequality 
\[\angk  pxy_{\EE^2}+\angk pyz_{\EE^2}+\angk pzx_{\EE^2}
\le 
2\cdot\pi
\eqlbl{eq:CBB-comparison}\]
holds for any quadruple $p,x,y,z\in\spc{X}$ such that each model angle in \ref{eq:CBB-comparison} is defined. 

The inequality \ref{eq:CBB-comparison} is called \index{4-point comparison}\emph{4-point comparison} for the quadruple $p,x,y,z$.
If instead of $\EE^2$, we use $\SSS^2$ or $\HH^2$, then we get the definition of
$\CBB(1)$ and $\CBB(-1)$ comparisons.
(Note that $\angk  pxy_{\EE^2}$ and $\angk  pxy_{\HH^2}$ are defined if $p\ne x$, $p\ne y$,
but for $\angk  pxy_{\SSS^2}$ we need in addition, $\dist{p}{x}{}+\dist{p}{y}{}+\dist{x}{y}{}<2\cdot\pi$.)

More generally, one may apply this definition to $\MM^2(\kappa)$ --- the model plane of curvature $\kappa$, defined as follows:
$\MM^2(0)=\EE^2$,
if $\kappa>0$, then $\MM^2(\kappa)$ is the sphere of radius $\tfrac{1}{\sqrt{\kappa}}$ and if $\kappa<0$, then it is Lobachevsky plane rescaled by factor $\tfrac{1}{\sqrt{-\kappa}}$.
This way we define $\CBB(\kappa)$ comparison for any real $\kappa$.
\end{thm}

While this definition can be applied to any metric space,
it is usually applied to geodesic spaces (or, at least, length spaces that will be defined later).

\begin{thm}{Exercise}
Show that Euclidean space $\EE^n$ is $\CBB(0)$.
\end{thm}


\begin{thm}{Exercise}\label{ex:polyCBB}
Show that a polyhedral surface is $\CBB(0)$ if and only if it has nonnegative curvature in the sense of \ref{sec:Alexandrov-existence}. 
\end{thm}





\section{Comments}

The first synthetic description of curvature is due to Abraham Wald \cite{wald}; 
it was given in a lone publication on a ``coordinateless description of Gauss surfaces'' published in 1936.
In 1941, similar definitions were rediscovered by Alexandr Alexandrov \cite{alexandrov:def}.

In Alexandrov's work, the first applications of this approach were given.
Mainly: the main part of \ref{thm:alexandrov+pogorelov} \cite{alexandrov-1941,alexandrov-1941convex}
and the {}\emph{gluing theorem} \cite{alexandrov-1946}, which gave a flexible tool to modify non-negatively curved metrics on a sphere.
These two results together formed the foundation of the branch of geometry now called {}\emph{Alexandrov geometry};
they gave  a very intuitive geometric tool to study embeddings and bending of surfaces in Euclidean space and changed the subject dramatically.

In particular, the existence of bending of a large spherical dome (sphere with a small disc removed) easily follows from these two theorems; moreover, it provides an intuitive description of such bending that can be extended to a closed convex surface.






%%!TEX root = invitation-CBB.tex
\chapter{Polyhedral surfaces}\label{chap:alex-embedding}

In this lecture we discuss intrisic geometry of surfaces of convex polyhedra and convex bodies.
Furhter, we prove the Cauchy theorem, and then modify the proof to get the Alexandrov uniqueness theorem.

\section{Surface of convex polyhedron}

Let us define a \index{convex body}\emph{convex body} as a compact convex subset in $\EE^3$ with nonempty interior.
The \index{surface}\emph{surface} of a convex body is defined as its boundary equipped with the induced length metric.

\begin{thm}{Exercise}\label{ex:surf-S2}
Show that the surface of a convex body is homeomorphic to the 2-dimensional sphere.
\end{thm}

A \index{convex polyhedron}\emph{convex polyhedron} is a convex body with a finite number of extremal points, called its \index{vertex}\emph{vertices}.

The surface, say $P$, of a convex polyhedron $K$ admits a finite triangulation such that each triangle is isometric to a plane triangle.
In other words, $P$ is a closed \index{polyhedral surface}\emph{polyhedral surface};
that is, it is a 2-dimensional manifold with a length metric that admits a finite triangulation such that each triangle is isometric to a solid plane triangle.
A \index{triangulation}\emph{triangulation} of a polyhedral surface will always be assumed to satisfy this condition.

The total angle around a vertex $v$ in $P$ is defined as the sum of angles at $v$ of all triangles in the triangulation that contain $v$.

If a point $p\in P$ is not a vertex of $K$,
then
\begin{itemize}
\item $p$ lies in the interior of a face of $K$, and its neighborhood in $P$ is a piece of plane, or
\item $p$ lies on an edge, and its neighborhood is two half-planes glued along the boundary.
\end{itemize}
In both cases, a neighborhood of $p$ in $P$ (with the induced length metric) is isometric to an open domain of the plane.
In this case, the total angle around $p$ will be defined to be $2\cdot\pi$.

\begin{thm}{Claim}\label{clm:total-angle}
Let $P$ be the surface of a convex polyhedron $K$.
Then, the total angle around any point $p\in P$ cannot exceed $2\cdot\pi$.
\end{thm}

The proof relies on the triangle inequality for angles (or the spherical triangle inequality).
It follows from \ref{claim:angle-3angle-inq}, but our proof of this statement is a straightforward generalization of the argument in the classical geometry textbook \cite[§ 47]{kiselev-stereo-en} that proves the following statement.

\begin{thm}{Spherical triangle inequality}\label{ex:angle-triangle}
Let $w_1,w_2,w_3$ be unit vectors in $\EE^3$.
Denote by $\alpha_{i,j}$ the angle between the vectors $v_i$ and $v_j$.
Then
$$\alpha_{1,3}\le \alpha_{1,2}+\alpha_{2,3}.$$
Moreover, in the case of equality, the three solid angles spanned by $w_1$, $w_2$, and $w_3$ form a plane.
\end{thm}

\parit{Proof of \ref{clm:total-angle}.}
Consider the intersection of $K$ with a small sphere centered at~$p$;
it is a convex spherical polygon, say $F$.
Applying rescaling we may assume that the sphere has unit radius.
Then we need to show that the perimeter of $F$ does not exceed $2\cdot\pi$.

\begin{wrapfigure}{o}{22mm}
\vskip-4mm
\centering
\includegraphics{mppics/pic-1103}
\end{wrapfigure}

Note that $F$ lies in a hemisphere, say $H$.
Moreover, there is a decreasing sequence of convex spherical polygons
\[H=H_0\supset\dots\supset H_n=F,\]
such that $H_{i+1}$ is obtained from $H_{i}$ by cutting along a chord.

By the spherical triangle inequality (\ref{ex:angle-triangle}), we have
\[
2\cdot\pi=\perim H=\perim H_0\ge\dots\ge\perim H_n=\perim F
\]
--- hence the result.
\qeds

\section{Curvature}

Let $p$ be a point on a polyhedral surface, and $\theta_p$ is the total angle around $p$.
The value $2\cdot \pi -\theta_p$ is called the \index{curvature}\emph{curvature} of the polyhedral surface at $p$.

Note that if $p$ is not a vertex in a triangulation of $P$, then its curvature is zero.
A vertex of a triangulation of a polyhedral surface is called \index{essential vertex}\emph{essential} if its curvature is not $0$.

\begin{thm}{Exercise}\label{ex:vertex-essential-vertex}
Let $v$ be a point on the surface $P$ of a convex polyhedron $K$.
Show that $v$ is a vertex of $K$ if and only if
$v$ is an essential vertex of $P$.
\end{thm}


\begin{thm}{Exercise}\label{ex:geodesic-vertex}
Show that geodesics on a closed polyhedral surface with nonnegative curvature may have essential vertices only at their ends.
\end{thm}

\begin{thm}{Exercise}\label{pr:tetrahedron}
Assume that the surface of a nondegenerate tetrahedron $T$ has curvature $\pi$ at each of its vertices.
Show that

\begin{subthm}{pr:tetrahedron:=}
all faces of $T$ are congruent;
\end{subthm}

\begin{subthm}{pr:tetrahedron:perp} the line containing midpoints of opposite edges of $T$ intersects these edges at right angles.
\end{subthm}

\end{thm}

\begin{thm}{Exercise}\label{ex:gauss-bonnet}
Show that sum of curvatures of a closed polyhedral surface $P$ equals to $2\cdot\pi\cdot\chi(P)$,
where $\chi(P)$ denotes the Euler characteristic of $P$.
\end{thm}


Claim~\ref{clm:total-angle} says that \textit{surfaces of convex polyhedra have nonnegative curvature} in the sense of the above definition.
Now we show that this definition agrees with the 4-point comparison.

\begin{thm}{Proposition}\label{prop:poly-CBB}
A polyhedral surface with nonnegative curvature at each vertex is $\Alex0$.
\end{thm}

\parit{Proof.}
Denote the surface by $P$.
By \ref{comp-kappa}, it is sufficient to check that
$\distfun_p^2\circ\gamma$ is 1-concave for any geodesic $\gamma$ and any point $p$ in $P$.

We can assume that $p$ is not a vertex;
the vertex case can be done by approximation.
By \ref{ex:geodesic-vertex}, $\gamma$ does not contain vertices.

Given a point $x=\gamma(t_0)$, choose a geodesic $[px]$.
Again, by \ref{ex:geodesic-vertex}, $[px]$ does not contain vertices.
Therefore a small neighborhood of $U\supset [px]$ can be unfolded on a plane;
that is, there is an injective length-preserving map $z\mapsto \tilde z$
of $U$ into the Euclidean plane.
This way we map part of $\gamma$ in $U$ to a line segment $\tilde\gamma$.
Let
\[\tilde f(t)\df\tfrac12\cdot\distfun_{\tilde p}^2\circ\tilde \gamma(t).\]
Since the geodesic $[px]$ maps to a line segment, we have $\tilde f(t_0)= f(t_0)$.
Furthermore, since the unfolding $z\mapsto \tilde z$ preserves lengths of curves, we get
$\tilde f(t)\ge f(t)$ if $t$ if the left-hand side is defined.
That is, $\tilde f$ is a local upper barrier of $f$ at $t_0$.
Evidently, $\tilde f''\equiv 1$; therefore $f''\le 1$.
It remains to apply \ref{comp-kappa}.
\qeds

\begin{thm}{Exercise}\label{ex:poly-CBB}
Prove the converse to the proposition;
that is, show that if a poyhedral surface is $\Alex0$, then it has nonnegative curvature in the sense defined in this section.
\end{thm}

\section{Surface of convex body}

\begin{thm}{Advanced exercise}\label{ex:surface-covergence}
Let $K_1,K_2,\dots,$ and $K_\infty$ be convex bodies in $\EE^m$.
Denote by $P_n$ the surface of $K_n$ with induced length metric.
Suppose $K_n\z\to K_\infty$ in the sense of Hausdorff.
Show that $P_n\to P_\infty$ in the sense of Gromov--Hausdorff.
\end{thm}

Any convex body is a Hausdorff limit of a sequence of convex polyhedra.
Therefore, the next proposition follows from \ref{prop:poly-CBB}, \ref{ex:surface-covergence}, and \ref{thm:CBB-closed}.

\begin{thm}{Proposition}\label{prop:conv-surf-CBB(0)}
The surface of a convex body in $\EE^3$ is $\Alex0$.
\end{thm}

\begin{thm}{Very advanced exercise}\label{ex:liberman+milka}
Let $P$ be the surface of a nondegenerate convex body $K\subset\EE^3$;
we assume that $P$ is equipped with the induced length metric.

\begin{subthm}{ex:liberman+milka:liberman}
Show that any geodesic $\gamma$ in $P$ is one-sided differentiable as a curve in $\EE^3$.
\end{subthm}

\begin{subthm}{ex:liberman+milka:convex}
Suppose a plane $\Pi$ cuts from $P$ a disc $\Delta$, and the reflection of $\Delta$ across $\Pi$ lies in $K$.
Show that $\Delta$ is a convex subset of $P$;
that is, if a geodesic has ends in $\Delta$, then it completely lies in $\Delta$.
\end{subthm}


\begin{subthm}{ex:liberman+milka:milka}
Let $\gamma_1$ and $\gamma_2$ be geodesic paths in $P$ that start at one point $p\z=\gamma_1(0)\z=\gamma_2(0)$.
Suppose $x_i=\gamma_i(1)$, and $y_i\z=p+\gamma_i^+(0)$.
Show that
\[\dist{x_1}{x_2}{P}\le \dist{y_1}{y_2}{W},\]
where $W$ is the complement to the interior of $K$.
\end{subthm}

\end{thm}


\section{Cauchy theorem}

Recall that \textit{surfaces} of convex polyhedrons are considered with the induced length metric.
 
\begin{thm}{Theorem}\label{thm:cauchy}
Let $K$ and $K'$ be convex polyhedrons in $\EE^3$;
denote their surfaces 
by $P$ and $P'$.
Suppose there is an isometry $P\to P'$ that sends each face of $K$ to a face of $K'$.
Then $K$ is congruent to $K'$; moreover the isometry $P\to P'$ can be extended to a motion of $\EE^3$ that maps $K$  to $K'$.
\end{thm}

\parit{Proof modulo two lemmas.}
Consider the graph $\Gamma$ formed by the edges of $K$;
the edges of $K'$ form a naturally isomorphic graph.
 
For an edge $e$ in $\Gamma$, denote by $\alpha_e$ and $\alpha'_e$ the dihedral angles in $K$ and $K'$, respectively.
Mark $e$ by plus if $\alpha_e < \alpha'_e$ and by minus if $\alpha_e > \alpha'_e$.

Let us remove from $\Gamma$ everything that is not marked;
that is, leave only the edges marked by $(+)$ or $(-)$ and their endpoints.
If $\Gamma$ is an empty graph, then the theorem follows.
Let us assume the contrary.

The graph $\Gamma$ is embedded into $P$, which is homeomorphic to the sphere.
In particular, the edges coming from one vertex have a natural cyclic order. 
Given a vertex $v$ of $\Gamma$, count the \textit{number of sign changes} around $v$;
that is, the number of consequent pairs edges with different signs. 

\begin{thm}{Local lemma}\label{lem:local}
For any vertex of  $\Gamma$ the number of sign changes is at least $4$.
\end{thm}

In other words, at each vertex of $\Gamma$, one can choose 4 edges marked by $(+)$, $(-)$, $(+)$, $(-)$ in the same cyclical order.
Note that the local lemma contradicts the following.

\begin{thm}{Global lemma}\label{lem:global}
Let $\Gamma$ be a nonempty planar graph.
Then it is impossible to mark all of the edges of $\Gamma$ by $(+)$ or $(-)$
in such a way  that the number of sign changes around each vertex of $\Gamma$ is at least $4$.
\end{thm}

It remains to prove these two lemmas.
\qeds


\section{Arm lemma}

\begin{thm}{Lemma}\label{lem:arm}
Assume that $A=[a_0 a_1\dots a_n]$ is a convex polygon in $\EE^2$
and $A'=[a'_0 a'_1\dots a'_n]$ be a polygonal line in $\EE^3$
such that 
$|a_i-a_{i+1}|=|a'_i-a'_{i+1}|$ for any $i\in\{0,\dots,n-1\}$
and 
$\measuredangle a_i\le \measuredangle a'_i$
for each $i\in\{1,\dots,n-1\}$.
Then 
$$|a_0-a_n|\le |a'_0-a'_n|$$
and equality holds if and only if $A$ is congruent to $A'$.
\end{thm}

One may view the polygonal lines $[a_0a_1\dots a_n]$ and $[a'_0a'_1\dots a'_n]$ as a robot's arm in two positions.
Informally speaking, the arm lemma says that when the arm opens,
the distance between the shoulder and tip of a finger increases;
assuming that starting position a convex plane polygon.

\begin{thm}{Exercise}\label{ex:arm-nonconvex}
Show that the arm lemma does not hold if 
instead of the convexity,
one only the local convexity;
that is, if we assume only that the polygonal line $a_0 a_1\dots a_n$ turns only left.
\end{thm}

\begin{thm}{Exercise}\label{ex:cauchy}
Suppose $A=[a_1\dots a_n]$ and $A'=[a'_1\dots a'_n]$ be noncongruent convex plane polygons with equal corresponding sides.
Mark each vertex $a_i$ with plus (minus) if the interior angle of $A$ at $a_i$ is smaller (respectively bigger) than the interior angle of $A'$ at $a_i'$.
Show that there are at least 4 sign changes around $A$. %+PIC

Give an example showing the statement does not hold without assuming convexity.

\end{thm}

\parit{Proof.}
We will view $\EE^2$ as the $xy$-plane in~$\EE^3$; 
so both $A$ and $A'$ lie in~$\EE^3$.

Let $a_m$ be the vertex of $A$ that lies on the maximal distance to the line $(a_0a_n)$.
Let us shift indexes of $a_i$ and $a'_i$ down by $m$,
so that 
\begin{align*}
a_{-m}&:=a_0,
&&\dots
&
a_{0}&:=a_m,
&&\dots
&
a_k&:=a_n,
\\
a'_{-m}&:=a'_0,
&&\dots
&
a'_{0}&:=a'_m,
&&\dots
&
a'_k&:=a'_n,
\end{align*}
where $k=n-m$.
(Here the symbol ``$:=$'' means an assignment as in programming.)

Without loss of generality, we may assume that
\begin{itemize}
\item $a_0=a'_0$ and they both coincide with the origin in $\EE^3$;
\item all $a_i$ lie in the $xy$-plane and the $x$-axis is parallel to the line $a_{-m}a_k$;
\item the angle $\measuredangle a'_0$ lies in $xy$-plane and contains the angle $\measuredangle a_0$ inside so that the directions to $a'_{-1}$,$a_{-1}$, $a_{1}$ and $a'_{1}$ from $a_0$ appear in the same cyclic order.
\end{itemize}

Denote by $x_i$ and $x'_i$ the projections of $a_i$ and $a'_i$ to the $x$-axis.
We can assume in addition that $x_k\ge x_{-m}$.
In this case,
$$|a_k-a_{-m}|=x_k-x_{-m}.$$
Since the projection is a distance non-expanding, we also have
$$|a'_k-a'_{-m}|\ge x'_k-x'_{-m}.$$ 

\begin{wrapfigure}{r}{60mm}
\vskip-5mm
\centering
\includegraphics{mppics/pic-30}
\vskip3mm
\end{wrapfigure}

Therefore it is sufficient to show
that 
$$x'_k-x'_{-m}\ge x_k-x_{-m}.$$
The latter holds if
$$x'_i-x'_{i-1}\ge x_i-x_{i-1}.\eqlbl{eq:|bb|=<|aa|}$$
for each $i$.
It remains to prove \ref{eq:|bb|=<|aa|}.

Let us assume that $i>0$; 
the case $i\le 0$ is similar.
Denote by $\sigma_i$ ($\sigma'_i$) the angle between the vector $w_i=a_{i}-a_{i-1}$ (respectively $w_i'=a'_{i}-a'_{i-1}$) and the $x$-axis.
Note that
$$\begin{aligned}
x_i-x_{i-1}&=|a_i-a_{i-1}|\cdot\cos\sigma_i,
\\
x'_i-x'_{i-1}&=|a_i-a_{i-1}|\cdot\cos\sigma'_i
\end{aligned}
\eqlbl{eq:proj}$$
for each $i>0$.
By construction $\sigma_1\ge \sigma'_1$.
Note that $\measuredangle (w_{i-1},w_i)\z=\pi -\measuredangle a_i$.
From convexity of $[a_1 a_1\dots a_i]$, we have
$$\sigma_i=\sigma_1+(\pi-\measuredangle a_1)+\dots+(\pi-\measuredangle a_i)$$
 for any $i>0$.
Since $\measuredangle (w'_{i-1},w'_i)=\pi -\measuredangle a'_i$,
applying the triangle inequality for angles (\ref{ex:angle-triangle}) several times,
we get
$$\sigma'_i\le\sigma'_1+(\pi-\measuredangle a'_1)+\dots+(\pi-\measuredangle a'_i).$$
Since $\measuredangle a'_j\ge \measuredangle a_j$ for each $j$, we get
$\sigma'_i\le \sigma_i$, and therefore
\[\cos \sigma'_i\ge \cos\sigma_i\]
Applying \ref{eq:proj}, we get \ref{eq:|bb|=<|aa|}.

In the case of equality, we have $\sigma_i=\sigma'_i$,
which implies $\measuredangle a_i=\measuredangle a'_i$ for each $i$.
This also implies that all $A'$ is a convex polygonal line in the $xy$-plane.
The latter easily follows from the equality case in \ref{ex:angle-triangle}.
\qeds

\begin{thm}{Advanced exercise}\label{ex:arm'}
Let $A$ and $A'$ be as in the arm lemma (\ref{lem:arm}).

\begin{subthm}{ex:bow'+}
Suppose that $\measuredangle \hinge{a_n}{a_{n-1}}{a_0}\le\tfrac\pi2$.
Show that $\measuredangle \hinge{a_0}{a_1}{a_n}\ge \measuredangle \hinge{a_0'}{a_1'}{a_n'}$.
\end{subthm}

\begin{subthm}{ex:bow'-} Show that the inequality $\measuredangle \hinge{a_0}{a_1}{a_n}\ge \measuredangle \hinge{a_0'}{a_1'}{a_n'}$ does not hold in general.
\end{subthm}

\end{thm}

\section{Proof of local lemma}
 
\parit{Proof of the local lemma (\ref{lem:local}).}
Assume that the local lemma does not hold at the vertex $v$ of $\Gamma$.
Choose a plane that cuts from $P$ a small pyramid $\Delta$ with the vertex~$v$.
One can choose two points $a$ and $b$ on the base of $\Delta$
so that on one side of the segments $[va]$ and $[vb]$ we have only pluses
and on the other side only minuses.

The base of $\Delta$ has two polygonal lines with ends at $a$ and $b$.
Choose the one that has only pluses;
denote it by $a_0 a_1 \dots a_n$;
so $a=a_0$ and $b=a_n$.
Denote by $a'_0 a'_1 \dots a'_n$
the corresponding line in $P'$;
let $a'=a'_0$ and $b'=a'_n$.

{

\begin{wrapfigure}{r}{40mm}
\vskip-0mm
\centering
\includegraphics{mppics/pic-40}
\vskip-0mm
\end{wrapfigure}

Since each marked edge passing thru $a_i$ has a $(+)$ on it or nothing, 
we have 
$$\measuredangle a_i\le\measuredangle a'_i$$
for each $i$.

}

\begin{thm}{Exercise}\label{ex:a<a}
Prove the last statement. 
\end{thm}

By the construction we have $|a_i-a_{i-1}|=|a'_i-a'_{i-1}|$ for all $i$.
By the arm lemma (\ref{lem:arm}), 
we get 
\[|a-b|\le |a'-b'|.
\eqlbl{clm:ab<ab}\]

Swap $K$ and $K'$ and repeat the same construction for a plane passing thru $a'$ and $b'$.
We get
\[|a-b|\ge |a'-b'|.
\eqlbl{clm:ab>ab}\]

The inequalities
\ref{clm:ab<ab} and \ref{clm:ab>ab} 
together imply $|a-b|=|a'-b'|$.
The equality case in the arm lemma implies that no edge at $v$ is marked;
that is, $v$ is not a vertex of $\Gamma$
--- a contradiction.
\qeds

From the proof, it follows that the local lemma is indeed local --- it works for two nonconguent convex polyhedral angles with equal corresponding faces.
Use this observation in the following exercise.

\begin{wrapfigure}{r}{20mm}
\vskip-0mm
\centering
\includegraphics{mppics/pic-10}
\bigskip
\includegraphics{mppics/pic-20}
\vskip-0mm
\end{wrapfigure}

\begin{thm}{Exercise}\label{ex:disc-bend}
Consider two polyhedral discs in $\EE^3$ glued from regular polygons by the rule on the diagrams.
Assume that each disc is part of a surface of a convex polyhedron.

\begin{subthm}{}
The first configuration is rigid; that is, one can not fix the position of the pentagon and continuously move the remaining 5 vertices in a new position so that each triangle moves by a one-parameter family of isometries of $\EE^3$.
\end{subthm}

\begin{subthm}{}
Show that the second configuration has a rotational symmetry with the axis passing thru the midpoint of the marked edge.
\end{subthm}

\end{thm}

\section{Proof of global lemma}

It is instructive to do the next exercise before diving into the proof.

\begin{thm}{Exercise}\label{ex:octahedron}
Try to mark the edges of an octahedron
by pluses and minuses
such that there would be 4 sign changes at each vertex.

Show that this is impossible.
\end{thm}

The proof of the global lemma is based on counting the sign changes
in two ways;
first while walking around each vertex of $\Gamma$
and second while moving around each of the regions separated by $\Gamma$
on the surface~$P$.
If two edges are adjacent at a vertex,
then they are also adjacent in a region.
The converse is true as well.
Therefore, both countings give the same number.

\parit{Proof of \ref{lem:global}.}
We can assume that $\Gamma$ is connected;
that is, one can get from any vertex to any other vertex by walking along edges.
(If not, pass to a connected component of $\Gamma$.)

We can assume that $\Gamma$ is embedded in the sphere.
Denote by $k$ and $l$ the number of vertices and edges in $\Gamma$.
Denote by $m$ the number of \textit{regions} that $\Gamma$ cuts from $P$.
Since $\Gamma$ is connected, each region is homeomorphic to an open disc.

\begin{thm}{Exercise}\label{ex:disc}
Prove the last statement.
\end{thm}

Now we can apply Euler's formula
$$k-l+m=2.
\eqlbl{eq:cauchy:euler}$$

Denote by $s$ the total number of sign changes in $\Gamma$ for all vertices. 
By the local lemma (\ref{lem:local}), we have 
$$ 4\cdot k\le s.\eqlbl{eq:S>=4k}$$

Let us get an upper bound on $s$ by counting the number of sign changes when one walks around
each region. 
Denote by $m_n$ the number of regions bounded by $n$ edges;
if an edge appears twice when it is counted twice.
Note that each region is bounded by at least $3$ edges;
therefore
$$m=m_3+m_4+m_5+\dots\eqlbl{eq:3-4-5}$$
Since edge belongs to exactly two regions, we get
$$2\cdot l=3\cdot m_3+ 4\cdot m_4+5\cdot m_5+\dots$$
Combining this with Euler's formula \ref{eq:cauchy:euler}, we get
$$4\cdot k=8+2\cdot m_3+4\cdot m_4+6\cdot m_5+8\cdot m_6+\dots
\eqlbl{eq:k=2+}$$
Observe that the number of sign changes in $n$-gon regions has to be even and $\le n$.
Therefore
$$s \le 2\cdot m_3 + 4\cdot m_4 + 4\cdot m_5 + 6\cdot m_6+\dots
\eqlbl{eq:23-44-45}$$
Clearly, \ref{eq:S>=4k} and \ref{eq:23-44-45} contradict \ref{eq:k=2+}.
\qeds


\section{Alexandrov uniqueness theorem}

Alexandrov's uniqueness theorem states that the conclusion of the Cauchy theorem (\ref{thm:cauchy}) still holds without the face-to-face assumption.

\begin{thm}{Theorem}\label{thm:alexandrov-uni'}
Any two convex polyhedrons in $\EE^3$ with isometric surfaces are congruent.

Moreover, any isometry between surfaces of convex polyhedrons can be extended to an isometry of the whole $\EE^3$.
\end{thm}

Instead of proof we list the modifications needed in the proof of Cauchy's theorem.

\parit{List of modifications in the proof of \ref{thm:cauchy}.}
Suppose $\iota\:P\z\to P'$  is an isometry between surfaces of $K$ and $K'$.
Mark in $P$ all the edges of $K$ and all the inverse images of edges in $K'$.
It might happen that an edge in $K'$ does not correspond to an edge in $K$;
it this case its inverse image in $K$ will be called a \index{fake edge}\emph{fake edge} of $K$.

The marked lines divide $P$ into convex polygons, and the restriction of $\iota$ to each polygon is a rigid motion.
These polygons should be used instead of faces in the proof of \ref{thm:cauchy}.

A vertex of the obtained graph can be a vertex of $K$, or it can be a fake vertex;
that is, it might be an intersection of an edge and a fake edge.

\begin{figure}[ht!]
\vskip-0mm
\centering
\includegraphics{mppics/pic-50}
\vskip-0mm
\end{figure}

For the first type of vertex, the local lemma can be proved the same way.
For a fake vertex $v$, it is easy to see that both parts of the edge coming thru $v$ are marked with minus
while both of the fake edges at $v$ are marked with plus.
Therefore, the local lemma holds for the fake vertices as well.

The remaining parts of the proof need no modifications.
\qeds

\begin{thm}{Exercise}\label{pr:K-P-simmetry}
Let $K$ be a convex polyhedron in $\EE^3$;
denote by $P$ its surface.
Show that each isometry $\iota\:P\z\to P$,
can be extended to an isometry of $\EE^3$.
\end{thm}


\section{Remarks}

This lecture contains selected material from Alexandrov's book~\cite{alexandrov}.

In Euclid's Elements, 
solids were claimed to be equal if the same holds for their faces, but no proof was given.
Adrien-Marie Legendre became interested in this problem towards the end of the 18th century.
He discussed it with his colleague Joseph-Louis Lagrange, who suggested this problem to Augustin-Louis Cauchy in 1813; soon he proved it \cite{cauchy}.
This theorem is included in many popular books \cite{aigner-zigler,dolbilin,tabacnikov-fuks}.
The key observation that the face-to-face condition can be removed was made by
Alexandr Alexandrov \cite{alexandrov-1941}.

\parit{Arm lemma.}
Cauchy's proof \cite{cauchy}
also used a version of the arm lemma, but its proof contained a small mistake that was corrected in a century \cite{sabitov}.

Several proofs of the arm lemma can be found in the letters between Isaac Schoenberg and Stanisław Zaremba \cite{schoenberg-zaremba}.

The following variation of the arm lemma makes sense for nonconvex spherical polygons.
It is due to Viktor Zalgaller \cite{zalgaller}.
It can be used instead of the standard arm lemma.

\begin{thm}{Another arm lemma}
Let $A=[a_1\dots a_n]$ and $A'\z=[a'_1\z\dots a'_n]$ be two spherical $n$-gons (not necessarily convex).
Assume that $A$ lies in a half-sphere,
the corresponding sides of $A$ and $A'$ are equal
and each angle of $A$ is at least the corresponding angle in $A'$.
Then $A$ is congruent to~$A'$. 
\end{thm}

Another close relative of the arm lemma is Reshetnyak's majorization theorem \cite{reshetnyak}.

\parit{Global lemma.}
A more visual proof of the global lemma is given in \cite[II \S 1.3]{alexandrov}.
This argument reused by Anton Klyachko \cite{klyachko} in his \index{car-crash lemma}\emph{car-crash lemma}.

\parit{Approximation.}
Proposition \ref{prop:conv-surf-CBB(0)} generalizes to boundaries of convex bodies  in $\EE^m$ for any $m\ge 2$.
This could be considered as a partial case of the conjecture about boundary of Alexandrov space; see \ref{conj:bry}.
Another partial case is proved by the authors and Stephanie Alexander \cite{alexander-kapovitch-petrunin-2008}.

\parit{Existance theorem.}
\ref{ex:surf-S2} and \ref{prop:poly-CBB} imply that the surface of a convex body is a sphere with nonnegative curvature in the sense of Alexandrov.
The celebrated theorem of Alexandrov states that the converse also holds if we allow degeneration of convex bodies to plane figures;
the surface of a plane figure is defined as its doubling across the boundary.
In other words, any $\Alex0$ metric on the two-sphere is isometric to a surface of a (possibly degenerate) convex body.
Moreover this convex body is unique up to congruence.
The last part is due to Alexei Pogorelov \cite{pogorelov}.

Originally, Alexandrov proved the statement for polyhedral metrics on the sphere; this proof is sketched in the appendix.
Then he used approximation to extend the result to  arbitrary $\Alex0$ metrics on the sphere.



\chapter{Misc}

\section{Existence}\label{sec:Alexandrov-existence}

\begin{thm}{Theorem}\label{thm:exist}
A polyhedral metric on the sphere is isometric to the surface of a convex polyhedron (possibly degenerate to a flat polygon) if and only if it has nonnegative curvature at each point.
\end{thm}

\begin{wrapfigure}{r}{30mm}
\vskip-5mm
\centering
\includegraphics{mppics/pic-1010}
\vskip-0mm
\end{wrapfigure}

By \ref{thm:alexandrov-uni'}, a convex polyhedron is completely defined by the intrinsic metric of its surface.
By \ref{thm:exist}, it follows that knowing the metric we could find the position of the edges.
However, in practice, it is not easy to do.

For example, the surface glued from a rectangle as shown on the diagram defines a tetrahedron.
Some of the glued lines appear inside facets of the tetrahedron and some edges (dashed lines) do not follow the sides of the rectangle.

\paragraph{Space of polyhedrons.}
Let us denote by $\mathbf{K}$ the space of all convex polyhedrons in the Euclidean space,
including polyhedrons that degenerate to a plane polygon.
Polyhedra in $\mathbf{K}$ will be considered up to a motion of the space,
and the whole space $\mathbf{K}$ will be considered with Hausdorff distance up to a motion of the space;
that is, the distance between $K$ and $K'$ is the exact lower bound on Hausdorff distance from $\iota(K)$ to $K'$, where $\iota$ is arbitrary motion of $\EE^3$.

Further, denote by $\mathbf{K}_n$ the polyhedrons in $\mathbf{K}$ with exactly $n$ vertices.
Since any polyhedron has at least 3 vertices, the space $\mathbf{K}$ admits a subdivision into a countable number of subsets $\mathbf{K}_3,\mathbf{K}_4,\dots$

\paragraph{Space of polyhedral metrics.}
The space of polyhedral metrics on the sphere with nonnegative curvature will be denoted by $\mathbf{P}$.
The metrics in $\mathbf{P}$ will be considered up to an isometry, and the whole space $\mathbf{P}$ will be equipped with the topology induced by the Gromov--Hausdorff metric.

The subset of $\mathbf{P}$ of all metrics with exactly $n$ essential vertices will be denoted by $\mathbf{P}_n$.
It is easy to see that any metric in $\mathbf{P}$ has at least 3 essential vertices.
Therefore $\mathbf{P}$ is subdivided into countably many subsets
 $\mathbf{P}_3,\mathbf{P}_4,\dots$

\paragraph{From a polyhedron to its surface.}

By \ref{prop:poly-CBB}, passing from a polyhedron to its surface defines a map
\[\iota\:\mathbf{K}\to \mathbf{P}.\]

By \ref{ex:vertex-essential-vertex}, the number of vertices of a polyhedron is equal to the number of essential vertices on its surface.
In other words, $\iota(\mathbf{K}_n)\subset \mathbf{P}_n$ for any $n\ge 3$.

Using the introduced notation, we can unite \ref{thm:alexandrov-uni'} and \ref{thm:exist} in the following more exact statement.

\begin{thm}{Reformulation}
For any integer $n\ge 3$,
the map $\iota$ induces a bijection between $\mathbf{K}_n$ and~$\mathbf{P}_n$.
\end{thm}

The proof is based on a construction of a one-parameter family of polyhedrons that starts at an arbitrary polyhedron
and ends at a polyhedron with its surface isometric to the given one.
This type of argument is called the \textit{continuity method}; it is often used in the theory of differential equations.


\parit{Sketch.}
By \ref{thm:alexandrov-uni'}, the map $\iota\:\mathbf{K}_n\to\mathbf{P}_n$ is injective.
Let us prove that it is surjective.

\begin{thm}{Lemma}
For any integer $n\ge 3$, the space $\mathbf{P}_n$ is connected.
\end{thm}

The proof of this lemma is not complicated, but it requires ingenuity;
it can be done by the direct construction of a one-parameter family of metrics in $\mathbf{P}_n$ that connects two given metrics.
Such a family can be obtained by а sequential application of the following construction and its inverse.

Let $P\in\mathbf{P}_n$.
Suppose $v$ and $w$ are essential vertices in $P$.
Let us cut $P$ along a geodesic from $v$ to $w$.
Note that the geodesic cannot pass thru an essential vertex of $P$.
Further, note that there is a three-parameter family of patches that can be used to patch the cut so that the obtained metric remains in $\mathbf{P}_n$;
in particular, the obtained metric has exactly $n$ essential vertices (after the patching, the vertices $v$ and $w$ may become inessential).


\begin{thm}{Lemma}
The map $\iota\:\mathbf{K}_n\to\mathbf{P}_n$ is open,
that is, it maps any open set in $\mathbf{K}_n$ to an open set in $\mathbf{P}_n$.

In particular, for any $n\ge 3$, the image $\iota(\mathbf{K}_n)$ is open in~$\mathbf{P}_n$.
\end{thm}

This statement is very close to the so-called \textit{invariance of domain theorem};
the latter states that a continuous injective map between manifolds of the same dimension is open.

Recall that $\iota$ is injective.
The proof of the invariance of domain theorem can be adapted to our case since both spaces $\mathbf{K}_n$ and $\mathbf{P}_n$ are $(3\cdot n-6)$-dimensional and both look like manifolds, altho, formally speaking, they are \textit{not} manifolds.
In a more technical language, $\mathbf{K}_n$ and $\mathbf{P}_n$ have the natural structure of $(3\cdot n-6)$-dimensional \textit{orbifolds},
and the map $\iota$ respects the \textit{orbifold structure}.

We will only show that both spaces $\mathbf{K}_n$ and $\mathbf{P}_n$ are $(3\cdot n-6)$-dimensional.

Choose  $K\in\mathbf{K}_n$.
Note that $K$ is uniquely determined by the $3\cdot n$ coordinates of its $n$ vertices.
We can assume that the first vertex is the origin, the second has two vanishing coordinates and the third has one vanishing coordinate; therefore, all polyhedrons in $\mathbf{K}_n$ that lie sufficiently close to $K$ can be described by $3\cdot n-6$ parameters.
If $K$ has no symmetries, then this description can be made one-to-one;
in this case, a neighborhood of $K$ in $\mathbf{K}_n$ is a $(3\cdot n-6)$-dimensional manifold.
If $K$ has a nontrivial symmetry group, then this description is not one-to-one but it does not have an impact on the dimension of~$\mathbf{K}_n$.

The case of polyhedral metrics is analogous.
We need to construct a subdivision of the sphere into plane triangles using only essential vertices.
By Euler's formula, there are exactly $3\cdot n-6$ edges in this subdivision.
Note that the lengths of edges completely describe the metric, and slight changes in these lengths produce a metric with the same property.
Again, if $P$ has no symmetries, then this description is one-to-one.

\begin{thm}{Lemma}
The map $\iota\:\mathbf{K}_n\to\mathbf{P}_n$ is closed;
that is, the image of a closed set in $\mathbf{K}_n$ is closed in $\mathbf{P}_n$.

In particular, for any $n\ge 3$, the set $\iota(\mathbf{K}_n)$ is closed in~$\mathbf{P}_n$.
\end{thm}

Choose a closed set $Z$ in $\mathbf{K}_n$.
Denote by $\bar Z$ the closure of $Z$ in $\mathbf{K}$; note that $Z=\mathbf{K}_n\cap \bar Z$.
Assume $K_1,K_2,\dots\in Z$ is a sequence of polyhedrons that converges to a polyhedron $K_\infty\in\bar Z$.
By \ref{lem:H>GH}, $\iota(K_n)$ converges to $\iota(K_\infty)$ in $\mathbf{P}$.
In particular, $\iota(\bar Z)$ is closed in $\mathbf{P}$.

Since $\iota(\mathbf{K}_n)\subset \mathbf{P}_n$ for any $n\ge 3$, we have $\iota (Z)=\iota(\bar Z)\cap \mathbf{P}_n$;
that is, $\iota (Z)$ is closed in $\mathbf{P}_n$.

\medskip

Summarizing, $\iota(\mathbf{K}_n)$ is a nonempty closed and open set in $\mathbf{P}_n$, and $\mathbf{P}_n$ is connected for any $n\ge 3$.
Therefore, $\iota(\mathbf{K}_n)=\mathbf{P}_n$; that is, $\iota\:\mathbf{K}_n\z\to\mathbf{P}_n$ is surjective.
\qeds

\section{Approximation}

By now, the embedding theorem is proved for polyhedral metrics on the sphere.
The general case is done by approximation, using the following statement.

\begin{thm}{Proposition}\label{prop:H>GH}
Let $K_1,K_2,\dots$ be a sequence of convex bodies that converge to $K_\infty$ in the sense of Hausdorff.
Then the surface of $K_n$ converges to the surface of $K_\infty$ in the sense of Gromov--Hausdorff.
\end{thm}

If $K_\infty$ is nondegenerate, then the statement follows from \ref{lem:H>GH}.
The degenerate case is left as an exercise.

Let $\spc{X}_\infty$ be an $\Alex0$ space that is homeomorphic to $\SSS^2$.
Suppose that $\spc{X}_\infty$ is a Gromov--Hausdorff limit of a sequence of spheres with polyhedral metrics $\spc{X}_1,\spc{X}_2,\dots$
By \ref{thm:exist}, there is a sequence of convex polyhedra $K_1,K_2,\dots$ with surfaces isometric to $\spc{X}_1,\spc{X}_2,\dots$, respectively.
Note that  $\diam K_n\le \diam \spc{X}_n$ for any $n$.
Therefore we can assume that all polyhedra $K_1,K_2,\dots$ lie in a closed ball of sufficiently large radius.

Applying Blaschke selection theorem, we can pass to a subsequence of $K_1,K_2,\dots$ that converges in the sense of Hausdorff; denote its limit by $K_\infty$.
By \ref{prop:H>GH} the surface of $K_\infty$ is isometric to $\spc{X}_\infty$.

Therefore it remains to prove the following lemma.

\begin{thm}{Lemma}\label{lem:GH-approximation}
Let $\spc{X}$ be an $\Alex0$ space that is homeomorphic to $\SSS^2$.
Then there is a sphere with polyhedral metrics $\spc{X}'$
that is arbitrarily close to $\spc{X}$ in the sense of Gromov--Hausdorff.
\end{thm}

\parit{Proof with two cheatings.}
Suppose we can triangulate $\spc{X}_\infty$ by small geodesic triangles;
that is, we can choose a finite set of points $p_1,\dots,p_n\z\in \spc{X}_\infty$ and some geodesics $[p_ip_j]$ that cut $\spc{X}_\infty$ into regions of small diameter bounded by geodesic triangles $[p_ip_jp_k]$.
(This is the first chating, the actual proof uses a triangulation with a weaker property.)

Observe that total angle around each $p_i$ cannot exceed $2\cdot \pi$.
That is, suppose $p_{j_1},\dots,p_{j_k}$ are points connected to $p_i$ by geodesics.
Assume that they are ordered in the natural cyclic order.
Then
\[\mangle\hinge{p_i}{p_{j_1}}{p_{j_2}}+\dots+\mangle\hinge{p_i}{p_{j_{k-1}}}{p_{j_k}}+\mangle\hinge{p_i}{p_{j_{k}}}{p_{j_1}}\le 2\cdot\pi.\]
By comparison, we get
\[\angk{p_i}{p_{j_1}}{p_{j_2}}+\dots+
\angk{p_i}{p_{j_{k-1}}}{p_{j_k}}+\angk{p_i}{p_{j_{k}}}{p_{j_1}}\le 2\cdot\pi.\eqlbl{eq:sum<=<2pi}\]

Now let us exchange each triangle by its model triangle.
That is, consider a model triangle for each region in the subdivision of $\spc{X}$ and glue them together by the same rule.
By \ref{eq:sum<=<2pi}, the obtained polyhedral surface $\spc{X}'$ has nonnegative curvature.
It remains to show that this way we can produce $\spc{X}'$ arbitrarily close to $\spc{X}$.

Denote by $p_i\to p_i'$ the natural map; it takes $p_i$ in $\spc{X}$ and returns the corresponding point in $\spc{X}'$.
Observe that
\[\dist{p_i'}{p_j'}{\spc{X}'}
\le
\dist{p_i}{p_j}{\spc{X}}.\eqlbl{eq:|pp|}\]
Indeed, choose a geodesic $\gamma$ from $p_i$ to $p_j$.
Let $p_i=x_0,x_1,\dots,x_n=p_j$ be the points of intersections of $\gamma$ with the edges of the triangulation listed as they appear on $\gamma$.
For each $i$, denote by $x_i'$ the corresponding point in $\spc{X}'$.
By comparison, we get
\[\dist{x_k'}{x_{k-1}'}{\spc{X}'}
\le
\dist{x_k}{x_{k-1}}{\spc{X}}.\]
for each $k$.
Therefore, \ref{eq:|pp|} follows.

Suppose $\eps>0$ is small, the points $p_1,\dots,p_n$ form an $\eps$-net in $\spc{X}$, all edges of the triangulation are smaller than $\eps$ and
\[\dist{p_i'}{p_j'}{\spc{X}'}
\ge
\dist{p_i}{p_j}{\spc{X}} -100\cdot \eps.\eqlbl{eq:|pp|>=}\]
Then, together with the inequality above it proves the lemma.

Now let us assume that the sides of model triangles in $\spc{X}'$ are geodesics.
(This is the second cheating; the sides of the model triangles are local geodesics in $\spc{X}'$,
but not necessarily geodesic; that is, they do not have to be length-minimizing.
The actual proof does not use this assumption.)

Choose a geodesic $\gamma'$ from $p_i'$ to $p_j'$ in $\spc{X}'$.
Note that $\gamma'$ visits each triangle in the triangulation of $\spc{X}'$ at most once.

Let $p_i'=x_0',x_1',\dots,x_n'\z=p_j'$ be the points of intersections of $\gamma'$ with the edges of the triangulation listed from $p_i'$ to $p_j'$.
For each $i$, denote by $x_i$ the corresponding point in $\spc{X}$.
Let $\Delta_k'$ be the triangle that contains arc $[x'_{k-1}x'_k]$ of $\gamma'$ and $\Delta_k$ the corresponding triangle in~$\spc{X}$.
Note that
\[\dist{x_k'}{x_{k-1}'}{\spc{X}'}
\ge
\dist{x_k}{x_{k-1}}{\spc{X}} -\eps\cdot K(\Delta_k),
\eqlbl{eq:|xx|<}\]
where $K(\Delta_k)$ denotes the access of $\Delta_k$;
that is, the sum of its internal angles minus $\pi$.

Euler's formula and \ref{eq:sum<=<2pi} imply that the sum of all accesses is at most $4\cdot\pi$.
Therefore, summing up \ref{eq:|xx|<}, we get
\[\dist{p_i'}{p_j'}{\spc{X}'}
\ge
\dist{p_i}{p_j}{\spc{X}}-4\cdot \pi\cdot \eps.\]
Whence \ref{eq:|pp|>=} follows.
\qeds

\section{Comments}

\parit{Existence theorem.}
This theorem was proved by Alexandr Alexandrov~\cite{alexandrov-1941}.
Our sketch is taken from \cite{lebedeva-petrunin};
a complete proof is nicely written in~\cite{alexandrov}.
In the original proof, the spaces $\mathbf{K}_n$ and $\mathbf{P}_n$ were modified so the they become $(3\cdot n-6)$-dimensional manifolds.
It was done by introducing extra structure (for $\mathbf{K}_n$ it is orientation + a marked vertex and an edge) that \textit{brakes symmetries} of the spaces.
After that one could apply the domain invariance theorem directly.
Alternatively, one may first remove from $\mathbf{K}_n$ and $\mathbf{P}_n$ elements (polyhedron or surface)with nontrivial symmetries (after that the spaces become manifolds) and show that any symmetric polyhedron (or surface) can be approximated by a non-symmetric polyhedron (or surface).

A very different proof was found by Yuri Volkov in his thesis \cite{volkov};
it uses a deformation of three-dimensional polyhedral space.

%P and \Sigma???


%\appendix
%\chapter[Alexandrov's embedding theorem]{Alexandrov's embedding theorem\\ \textsc{\normalsize by Nina Lebedeva and Anton Petrunin}}\label{chap:embedding}

\section{Introduction}

Intrinsic distance between two points on the surface of a convex polyhedron is defined as the length of a shortest curve on the surface between these points.

Recall that the sum of angles at the tip of a convex polyhedral angle is less than $2\cdot\pi$;
this statement can be found in a school textbook \cite[§~48]{kiselev-stereo-en}.

It is easy to see that the surface of a convex polyhedron is homeomorphic to the sphere.
Therefore the statements above imply that the surface of a convex polyhedron equipped with its intrinsic metric is an example of a \textit{polyhedral metric on the sphere with the sum of angles around each vertex at most $2\cdot\pi$};
a metric is called \emph{polyhedral} if the sphere admits a triangulation such that every triangle is congruent to a plane triangle.

Alexandrov's theorem states that the converse holds if one includes in the consideration \textit{twice covered polygons}.
In other words, we assume that a polyhedron can degenerate to a plane polygon;
in this case, its surface is defined as two copies of the polygon glued along their boundary.

Further, we assume that a polyhedron can degenerate to a plane polygon.

\pagebreak

\begin{thm}{Alexandrov's theorem}
\begin{enumerate}[I.]
\item\label{thm:exist}
A polyhedral metric on the sphere is isometric to the surface of a convex polyhedron if and only if the sum of angles around each of its vertex is not greater than $2\cdot\pi$.

\item\label{thm:unique} 
Moreover, a convex polyhedron is defined up to congruence by the intrinsic metric on its surface.
\end{enumerate}

\end{thm}

A. D. Alexandrov has many remarkable theorems, but in our opinion, this theorem is the most remarkable.
At the same time, its proof is elementary;
it could be explained to anyone familiar with basic topology.

This theorem has many applications.
In particular, it is used in the proof of its generalization \cite{alexandrov-1948} that gives a complete description of intrinsic metrics on the sphere that are isometric to convex surfaces in the Euclidean space.
The latter statement is fundamental in a branch of modern mathematics --- the so-called \emph{Alexandrov geometry}.

The first part is central; it is called the \emph{existence theorem}.
The second part is called the \emph{uniqueness theorem}; it is a slight variation of Cauchy's theorem about polyhedrons.
(There is another uniqueness theorem of Alexandrov that generalizes Minkowski's theorem about  polyhedrons.)

According to the theorem, a convex polyhedron is completely defined by the intrinsic metric of its surface.
In particular, knowing the metric we could find the position of the edges.
However, in practice, it is not easy to do.
For example, the surface glued from a rectangle as shown on the diagram defines a tetrahedron.
Some of the glued lines appear inside facets of the tetrahedron and some edges (dashed lines) do not follow the sides of the rectangle.

{

\begin{wrapfigure}{r}{30mm}
\vskip-3mm
\centering
\includegraphics{mppics/pic-15}
\vskip-0mm
\end{wrapfigure}

The theorem was proved by A. D. Alexandrov in 1941 \cite{alexandrov-1941};
we will present a sketch of his proof.
A complete proof is nicely written by A. D. Alexandrov in his book~\cite{alexandrov}.
Yet another proof was found by Yu.~A.~Volkov in his thesis \cite{volkov};
it uses a deformation of three-dimensional polyhedral space.

}

\section{Space of polyhedrons and metrics}

\paragraph{Space of polyhedrons.}
Let us denote by $\Phi$ the space of all convex polyhedrons in the Euclidean space,
including polyhedrons that degenerate to a plane polygon.
Polyhedra in $\Phi$ will be considered up to a motion of the space, 
and the whole space $\Phi$ will be considered with the natural topology (an intuitive meaning of closeness of two polyhedrons should be sufficient).  

Further, denote by $\Phi_n$ the polyhedrons in $\Phi$ with exactly $n$ vertices.
Since any polyhedron has at least 3 vertices, the space $\Phi$ admits a subdivision into a countable number of subsets $\Phi_3,\Phi_4,\dots$

\paragraph{Space of polyhedral metrics.}
The space of polyhedral metrics on the sphere with the sum of angles around each point at most $2\cdot\pi$ will be denoted by $\Psi$.
The metrics in $\Psi$ will be considered up to an isometry, and the whole space $\Psi$ will be equipped with the natural topology (again, an intuitive meaning of closeness of two metrics is sufficient).

A point on the sphere with the sum of angles strictly less than $2\cdot\pi$ will be called an \emph{essential vertex}.
The subset of $\Psi$ of all metrics with exactly $n$ essential vertices will be denoted by $\Psi_n$.
It is easy to see that any metric in $\Psi$ has at least 3 essential vertices.
Therefore $\Psi$ is subdivided into countably many subsets
 $\Psi_3,\Psi_4,\dots$

\paragraph{From a polyhedron to its surface.}

Recall that the surface of a convex polyhedron is a sphere with a polyhedral metric such that the sum of angles around each point is at most $2\cdot\pi$.
Therefore passing from a polyhedron to its surface defines a map
\[\iota\:\Phi\to \Psi.\]

Note that the number of vertices of a polyhedron is equal to the number of essential vertices of its surface.
In other words, $\iota(\Phi_n)\subset \Psi_n$ for any $n\ge 3$.

\section{About the proof}

Using the notation introduced in the previous section, we can give the following more exact formulation of Alexandrov's theorem: 

\begin{thm}{Reformulation}
For any integer $n\ge 3$,
the map $\iota$ is a bijection from $\Phi_n$ to~$\Psi_n$.
\end{thm}

We sketch the original proof of A. D. Alexandrov.
It is based on the  construction of a one-parameter family of polyhedrons that starts at arbitrary polyhedron
and ends at a polyhedron with its surface isometric to the given one.
This type of argument is called the \emph{continuity method}; it is often used in the theory of differential equations.

\medskip

The two parts of the first formulation will be proved separately.

\parit{Part \ref{thm:unique}.} Let us show that the map $\iota\:\Phi_n\to\Psi_n$ is injective;
in other words, a convex polyhedron is defined by the intrinsic metric on its surface up to a motion of the space.

The last statement is analogous to the Cauchy theorem about polyhedrons,
and the proof goes along the same lines. 

The Cauchy theorem states that facets of a polyhedron together with the gluing rule completely describe a convex polyhedron;
its proof is given in many classical popular texts \cite{aigner-zigler,dolbilin,tabacnikov-fuks}.

\medskip

\parit{Part \ref{thm:exist}.}
Let us prove that $\iota\:\Phi_n\to\Psi_n$ is surjective.
This part of the proof is subdivided into the following lemmas:

\begin{thm}{Lemma}
For any integer $n\ge 3$, the space $\Psi_n$ is connected.
\end{thm}

The proof of this lemma is not complicated, but it requires ingenuity;
it can be done by the direct construction of a one-parameter family of metrics in $\Psi_n$ that connects two given metrics.
Such a family can be obtained by а sequential application of the following construction and its inverse.

Let $M$ be a sphere with metric from $\Psi_n$.
Suppose $v$ and $w$ are essential vertices in $M$.
Let us cut $M$ along a shortest line from $v$ to~$w$.
Note that the shortest line cannot pass thru an essential vertex of $M$.
Further, note that there is a three-parameter family of patches that can be used to patch the cut so that the obtained metric remains in $\Psi_n$;
in particular, the obtained metric has exactly $n$ essential vertices (after the patching, the vertices $v$ and $w$ may become inessential).


\begin{thm}{Lemma}
The map $\iota\:\Phi_n\to\Psi_n$ is open, 
that is, it maps any open set in $\Phi_n$ to an open set in $\Psi_n$.

In particular, for any $n\ge 3$, the image $\iota(\Phi_n)$ is open in~$\Psi_n$.
\end{thm}

This statement is very close to the so-called \emph{invariance of domain theorem};
the latter states that a continuous injective map between manifolds of the same dimension is open.

According to part \ref{thm:unique}, $\iota$ is injective.
The proof of the invariance of domain theorem can be adapted to our case since both spaces $\Phi_n$ and $\Psi_n$ are $(3\cdot n-6)$-dimensional and both look like manifolds, altho, formally speaking, they are \emph{not} manifolds.
In a more technical language, $\Phi_n$ and $\Psi_n$ have the natural structure of $(3\cdot n-6)$-dimensional \emph{orbifolds},
and the map $\iota$ respects the \emph{orbifold structure}.

We will only show that both spaces $\Phi_n$ and $\Psi_n$ are $(3\cdot n-6)$-dimensional.

Choose a polyhedron $P$ in $\Phi_n$.
Note that $P$ is uniquely determined by the $3\cdot n$ coordinates of its $n$ vertices.
We can assume that the first vertex is the origin, the second has two vanishing coordinates and the third has one vanishing coordinate; therefore, all polyhedrons in $\Phi_n$ that lie sufficiently close to $P$ can be described by $3\cdot n-6$ parameters.
If $P$ has no symmetries then this description can be made one-to-one;
in this case, a neighborhood of $P$ in $\Phi_n$ is a $(3\cdot n-6)$-dimensional manifold.
If $P$ has a nontrivial symmetry group, then this description is not one-to-one but it does not have an impact on the dimension of $\Phi_n$.

The case of polyhedral metrics is analogous.
We need to construct a subdivision of the sphere into plane triangles using only essential vertices.
By Euler's formula, there are exactly $3\cdot n-6$ edges in this subdivision.
Note that the lengths of edges completely describe the metric, and slight changes of these lengths produce a metric with the same property.

\begin{thm}{Lemma}
The map $\iota\:\Phi_n\to\Psi_n$ is closed;
that is, the image of a closed set in $\Phi_n$ is closed in $\Psi_n$.

In particular, for any $n\ge 3$, the set $\iota(\Phi_n)$ is closed in~$\Psi_n$.
\end{thm}

Choose a closed set $Z$ in $\Phi_n$.
Denote by $\bar Z$ the closure of $Z$ in $\Phi$; note that $Z=\Phi_n\cap \bar Z$.
Assume $P_1,P_2,\dots\in Z$ is a sequence of polyhedrons that converges to a polyhedron $P_\infty\in\bar Z$.
Note that $\iota(P_n)$ converges to $\iota(P_\infty)$ in $\Psi$.
In particular, $\iota(\bar Z)$ is closed in $\Psi$.

Since $\iota(\Phi_n)\subset \Psi_n$ for any $n\ge 3$, we have  $\iota (Z)=\iota(\bar Z)\cap \Psi_n$;
that is, $\iota (Z)$ is closed in $\Psi_n$. 

\medskip

Summarizing, $\iota(\Phi_n)$ is a nonempty closed and open set in $\Psi_n$, and $\Psi_n$ is connected for any $n\ge 3$.
Therefore, $\iota(\Phi_n)=\Psi_n$; that is, $\iota\:\Phi_n\z\to\Psi_n$ is surjective.
\qeds

\parbf{Acknowledgments.} We want to thank Stephanie Alexander, Yuri Burago, and Jules %Kiyoshi
Tsukahara for help. 
The authors were partially supported by RFBR grant 20-01-00070 and NSF grant DMS-2005279.


\backmatter

%%!TEX root = the-sols.tex

\chapter{Semisolutions}

\parbf{\ref{ex:compact+connceted}.}
Choose a sequence of positive numbers $\varepsilon_n\to 0$ and a finite $\varepsilon_n$-net $N_n$ of $K$ for each $n$.
%???eps-net are not defined!!!
We can assume that $\eps_0>\diam K$, and $N_0$ is a one-point set.
If $\dist{x}{y}{}<\eps_k$ for some $x\in N_{k+1}$ and $y\in N_{k}$, then connect them by a curve of length at most $\eps_k$.

Let $K'$ be the union of all these curves and $K$.
Show that $K'$ is compact and path-connected.

\parit{Source:} This problem is due to Eugene Bilokopytov \cite{bilokopytov}.

\parbf{\ref{ex:compact=>complete}.}
Choose a Cauchy sequence $x_n$ in $(\spc{X},\|*\z-*\|)$; it is sufficient to show that a subsequence of $x_n$ converges.

Observe that the sequence $x_n$ is Cauchy in $(\spc{X},|*-*|)$;
denote its limit by $x_\infty$.

Passing to a subsequence, we can assume that $\|x_n-x_{n+1}\|\z<\tfrac1{2^n}$.
It follows that there is a 1-Lipschitz path $\gamma$ in $(\spc{X},\|*-*\|)$ such that $x_n=\gamma(\tfrac1{2^n})$ for each $n$ and $x_\infty=\gamma(0)$.
Therefore,
\begin{align*}
\|x_\infty-x_n\|&\le \length\gamma|_{[0,\frac1{2^n}]}\le \tfrac1{2^n}.
\end{align*}
In particular, $x_n$ converges to $x_\infty$ in $(\spc{X},\|*\z-*\|)$.

\parit{Source:} \cite[Corollary]{hu-kirk}; see also \cite[Lemma 2.3]{petrunin-stadler}.

\parbf{\ref{ex:compact-length}.}
Given a pair of points $p$ and $q$, choose a sequence of paths $\gamma_n$ from $p$ to $q$ such that
\[\length\gamma_n\to \dist pq{}
\quad\text{as}\quad
n\to\infty;\]
these paths exist since we are in a length space.
Note that we can assume that each $\gamma_n$ is parametrized proportionally to the arc length;
in particular, $\gamma_n$ are equicontinuous.
Show that paths $\gamma_n$ lie in a closed ball, say $\cBall[p,r]$ of some radius $r<\infty$.
Since the space is proper, $\cBall[p,r]$ is compact.
By the Arzelà--Ascoli theorem, we can pass to a converging subsequence of $\gamma_n$.
Show that its limit is a geodesic path from $p$ to $q$.

\parbf{\ref{ex:menger}.}
Choose a sequence $\eps_n>0$ that converges to zero very fast, say such that $\sum_n10^n\cdot \eps_n$ is small.
Follow the argument in the proof of Menger's lemma, taking $\eps_n$-midpoints at the $n^{\text{th}}$ stage.

\parbf{\ref{ex:k-><mono}.}
Let us write the Riemannian metric on $\MM^2(\kappa)$ in polar coordinates $(\theta,r)$;
it has the form 
$(\begin{smallmatrix}
h^2&0
\\
0&1
\end{smallmatrix})$, where $h=h(\kappa,r)\ge 0$.
Calculate $h(\kappa,r)$.
Show that for fixed $r$, the function $r\mapsto h(\kappa,r)$ is nonincreasing in the domain of definition.
Suppose $\kappa<\Kappa$, consider the partially defined map $\MM^2(\kappa)\to\MM^2(\Kappa)$ that sends a point to the point with the same polar coordinates.
Show that this map is short in the domain of definition.
Use it to prove the statement in the exercise.


\parbf{\ref{ex:angkK}.} Show and use that 
$\angk p{x}{y}_{\SSS^2}-\angk p{x}{y}_{\EE^2}=O(\dist[2]{p}{x}{}+\dist[2]{p}{y}{})$
and 
$\angk p{x}{y}_{\EE^2}-\angk p{x}{y}_{\HH^2}=O(\dist[2]{p}{x}{}+\dist[2]{p}{y}{})$.

\parbf{\ref{ex:undefined-angle}.}
Consider a hinge in the plane $\RR^2$ with a metric defined by norm, say by the $\ell^\infty$-norm.

\parbf{\ref{ex:adjacent-angles}.}
Assume $\mangle\hinge pxz+\mangle\hinge pyz<\pi$.
By \ref{claim:angle-3angle-inq}, $\mangle\hinge pxy<\pi$.
Therefore,
$\angk p{\bar x}{\bar y}<\pi$
for some $\bar x\in \left]px\right]$ and $\bar y\in \left]py\right]$.
Hence 
\[\dist p{\bar x}{}+\dist {\bar y}p{}<\dist {\bar x}{\bar y}{}\]
--- a contradiction.

\parbf{\ref{ex:first-var}.}
Denote by $\alpha$ the arc-length parametrization of $[qp]$ from $q$ to $p$.
Choose $\eps>0$.
Observe that 
\[\dist[2]{\gamma(t)}{\alpha(\tfrac1\eps\cdot t)}{}\le t^2\cdot(1-\tfrac2\eps\cdot\cos\phi+\tfrac1{\eps^2})+o(t^2),\]
where $\phi=\mangle\hinge q p x$.
By the triangle  inequality
\[\dist{p}{\gamma(t)}{}\le \dist{\gamma(t)}{\alpha(\tfrac1\eps\cdot t)}{}+\dist{q}{p}{}-\tfrac1\eps\cdot t.\]
Conclude that
\[\dist{p}{\gamma(t)}{}
\le
\dist{q}{p}{}-t\cdot \cos \phi+\delta(\eps)\cdot t+o(t),\]
where $\delta(\eps)\to 0$ as $\eps\to0$.
The statement follows since $\eps>0$ is arbitrary.

\parbf{\ref{ex:generalized-selection}.}
Since the space is proper, it is separable; 
that is, we can choose an countable everywhere dense set $\{x_1,x_2,\dots\}$.

Let $A_1,A_2,\dots$ be a sequence of closed sets.
Applying the diagonal procedure, we can pass to a subsequence such that for each $i$ the sequence $\distfun_{A_n}x_i$ converges as $n\to\infty$;
denote its limit by $f(x_i)$.

Since $\distfun_{A_n}$ is $1$-Lipschitz for any $n$, we have 
\[|f(x_i)-f(x_j)|\le \dist{x_i}{x_j}{}\]
for all $i$ and $j$.
Suppose $f(x_i)<\infty$ for some $i$; note that the same holds for any $i$.
Therefore, the function $f$ can be extended to a continuous function defined on the whole ambient space.
Show that $A_\infty=f^{-1}\{0\}$ is the limit of $A_n$ in the sense of Hausdorff.

If $f(x_i)=\infty$ for some $i$, then the same holds for any $i$.
Show that in this case $A_n\to\emptyset$ in the sense of Hausdorff.

\parbf{\ref{ex:Haus-conv}.}
Apply the definition of Hausdorff distance (\ref{def:hausdorff-convergence}).

\parbf{\ref{ex:geod-closed}.}
Given $x_\infty,y_\infty\in\spc{X}_\infty$, choose $x_n,y_n\in \spc{X}_n$ such that $x_n\to x_\infty$ and $y_n\to y_\infty$.
Let $z_n$ be the midpoint of $[x_ny_n]$.
Since $\spc{X}_\infty$ is proper, we can choose a subsequence of $z_m$ that converges to a point, say $z_\infty\in \spc{X}_\infty$.
Note that $z_\infty$ is a midpoint of $x_\infty$ and $y_\infty$, then apply Menger's lemma (\ref{lem:mid>geod}).

\parbf{\ref{ex:non-contracting-map}.}
Given a pair of points $x_0,y_0\in \spc{K}$, 
consider two sequences $x_0,x_1,\dots$ and $y_0,y_1,\dots$
such that $x_{n+1}=f(x_n)$ and $y_{n+1}\z=f(y_n)$ for each $n$.

Since $\spc{K}$ is compact, 
we can choose an increasing sequence of integers $n_k$
such that both sequences $(x_{n_i})_{i=1}^\infty$ and $(y_{n_i})_{i=1}^\infty$
converge.
In particular, both are Cauchy;
that is,
\[
|x_{n_i}-x_{n_j}|_{\spc{K}}\to 0 
\quad\text{and}\quad
|y_{n_i}-y_{n_j}|_{\spc{K}}\to 0
\]
as $\min\{i,j\}\to\infty$.

Since $f$ is distance-noncontracting, 
\[
|x_0-x_{|n_i-n_j|}|
\le 
|x_{n_i}-x_{n_j}|
\]
for any $i$ and $j$.
Therefore, there is a sequence $m_i\to\infty$ such that
\[
x_{m_i}\to x_0\quad\text{and}\quad y_{m_i}\to y_0
\leqno({*})\]
as $i\to\infty$.

Since $f$ is distance-noncontracting, the sequence $\ell_n=|x_n-y_n|_{\spc{K}}$ is nondecreasing.
By $({*})$,  $\ell_{m_i}\to\ell_0$ as $m_i\to\infty$.
It follows that 
\[\ell_0=\ell_1=\dots\]
In particular, 
\[|x_0-y_0|_{\spc{K}}=\ell_0=\ell_1=|f(x_0)-f(y_0)|_{\spc{K}}\]
for any pair of points $(x_0,y_0)$ in $\spc{K}$.
That is, the map $f$ is distance-preserving; hence $f$ is injective.
From $({*})$, we also get that $f(\spc{K})$ is everywhere dense.
Since $\spc{K}$ is compact $f\:\spc{K}\to \spc{K}$ is surjective --- hence the result.

\parit{Remarks.}
This is a basic lemma in the introduction to Gromov--Hausdorff distance \cite[see 7.3.30 in][]{burago-burago-ivanov}.
The presented proof is not quite standard;
I learned it from Travis Morrison, 
a student in my MASS class at Penn State, Fall 2011.

Note that this exercise implies that \textit{any surjective non-expanding map from a compact metric space to itself is an isometry}.

\parbf{\ref{ex:GH-po}.}
The only-if part is trivial. 
Let us prove the if part.

If $\dist{\spc{X}_n}{\spc{X}_\infty}{\GH}\not\to 0$, then we can pass to a subsequence such that $\dist{\spc{X}_n}{\spc{X}_\infty}{\GH}\ge\eps$ for some $\eps>0$.
Show that we can pass to a subsequence again, so that $\spc{X}_n$ converges in the sense of Gromov--Hausdorff, say to $\spc{Y}$.
Observe that $\spc{Y}\le \spc{X}_\infty$ and $\spc{X}_\infty\le\spc{Y}$.
By \ref{ex:non-contracting-map}, $\spc{Y}\iso \spc{X}_\infty$ --- a contradiction.

\parbf{\ref{ex:compact-GH}.} Show and use that $\dist{\spc{X}_\infty}{\spc{X}_\infty'}{\GH}<\eps$ for any $\eps>0$.



\parbf{\ref{ex:GH-noncompact}.  } \parit{\ref{SHORT.ex:GH-noncompact:proper}}
 Consider the graphs of the following functions with the induced metric from $\RR^2$.
\[
x\mapsto \cos x+\cos \tfrac x\pi
\quad\text{and}\quad
x\mapsto \cos x+\sin \tfrac x\pi.
\]


\parit{\ref{SHORT.ex:GH-noncompact:bounded}}
For every rational number  $q\in[1,2]$ consider an interval of length $q$. Let $\spc{X}$ be obtained by identifying all  initial points of  the intervals to one point and all  end points to another.

Let $\spc{Y}$ be constructed in the same way but skipping the interval of length $1.5$.



\parbf{\ref{ex:Euclid-is-CBB}.}
The 4-point comparison (\ref{def:CBB}) reduces our question to the following.
\textit{Any spherical triangle has perimeter at most $2\cdot\pi$.}
Choose a spherical triangle $[xyz]$.
Let $x'$ be the antipode of $x$; that is $x'=-x$.
The spherical triangle inequality (\ref{claim:angle-3angle-inq} or \ref{ex:angle-triangle}) implies that
\[\dist{x}{z}{\mathbb{S}^2}\le \dist{y}{x'}{\mathbb{S}^2}+\dist{x'}{z}{\mathbb{S}^2}.\]
Observe that 
\[
\dist{x}{y}{\mathbb{S}^2}+\dist{y}{x'}{\mathbb{S}^2}=\pi,
\quad\text{and}\quad
\dist{x}{z}{\mathbb{S}^2}+\dist{z}{x'}{\mathbb{S}^2}=\pi.
\]
Hence
\[\dist{x}{y}{\mathbb{S}^2}+\dist{x}{z}{\mathbb{S}^2}+\dist{y}{z}{\mathbb{S}^2}\le2\cdot \pi.\]

\parbf{\ref{ex:(3+1)-expanding}.} For the only-if part consider the following two cases.

If $\angk p{x_1}{x_2}+\angk p{x_2}{x_3}\ge \pi$, then choose two model triangles $[qy_1y_2]\z=\modtrig(px_1x_2)$ and $[qy_2y_3]=\modtrig(px_2x_y)$ that lie on the opposite sides of $[qy_2]$.
By the comparison, $\dist{y_1}{y_3}{}\ge \dist{x_1}{x_3}{}$.
Therefore the obtained configuration meets all the conditions.

If $\angk p{x_1}{x_2}+\angk p{x_2}{x_3}\ge \pi$, then choose two model triangles $[qy_1y_2]\z=\modtrig(px_1x_2)$
and take $y_3$ on the extension of $[y_1q]$ behind $q$ such that $\dist{q}{y_3}{}=\dist{p}{x_3}{}$.
Then $\mangle \hinge q{y_2}{y_3}\ge \angk p{x_2}{x_3}$, therefore $\dist{y_2}{y_3}{}\ge \dist{x_2}{x_3}{}$.
Further, $\dist{y_2}{y_3}{}=\dist{x_2}{p}{}+\dist{p}{x_3}{} \ge \dist{x_2}{x_3}{}$,
and again, the obtained configuration meets all the conditions.

To prove the if part, choose a configuration $q,y_1,y_2,y_3$ that meets all the conditions and maximize the sum
\[\dist{y_1}{y_2}{}+\dist{y_2}{y_3}{}+\dist{y_3}{y_1}{}.\]
Show that $q$ lies in the solid triangle $y_1y_2y_3$;
in particular 
\[\mangle \hinge q{y_1}{y_2}+\mangle \hinge q{y_2}{y_3}+ \mangle \hinge q{y_3}{y_1}=2\cdot\pi.\]
Moreover, $\dist{q}{y_i}{}=\dist{p}{x_i}{}$ for each $i$.
Applying that increasing the opposite side in a plane triangle increases the corresponding angle, we get 
\[\angk  p{x_1}{x_2}+\angk p{x_2}{x_3}+\angk p{x_3}{x_1}
\le 
2\cdot\pi.
\]

\parbf{\ref{ex:alex-lemma-cat}.}
Consider model triangles $[\tilde p\tilde x\tilde z]=\modtrig(pxz)$ and $[\tilde p\tilde y\tilde z]=\modtrig(pyz)$
that share side $[\tilde p\tilde z]$ and lie on its opposite sides.
Note that 
\begin{align*}
\dist{\tilde x}{\tilde y}{\EE^2}
&\ge \dist{\tilde x}{\tilde y}{\EE^2}+\dist{\tilde x}{\tilde y}{\EE^2}=
\\
&=\dist{x}{z}{\spc{X}}+\dist{z}{y}{\spc{X}}=
\\
&=\dist{x}{y}{\spc{X}},
\end{align*}
where $\spc{X}$ is our metric space.
It remains to apply the monotonicity of angle in a triangle with respect to its opposite side. 


\parbf{\ref{ex:noncreasing}.}
Apply \ref{clm:angle-mono}.

\parbf{\ref{ex:0-angle}.}
Without loss of generality, we can assume that $\dist{p}{x}{}\le \dist{p}{y}{}$.
Choose $\bar x\in [px]$;
let $\bar y\in [px]$ be such that $\dist{p}{\bar x}{}=\dist{p}{\bar y}{}$.
Apply \ref{clm:angle-mono} to show that $\bar x=\bar y$.
Conclude that $[px]\subset [py]$.

\parbf{\ref{ex:pi-angle}.}
Assume that there are two distinct geodesics from $z$ to $x$.
Then we can choose distinct points $p$ and $q$ on these geodesics such that $\dist{z}{p}{}=\dist{z}{q}{}$.
Observe that $\angk zpq>0$.
By the triangle inequality, we get 
\[\dist{x}{p}{}+\dist{p}{y}{}\le \dist{x}{p}{}+\dist{p}{z}{}+\dist{z}{y}{}=\dist{x}{z}{}+\dist{z}{y}{}\]
Observe that $\angk zxy=\pi$.
Therefore $\mangle\hinge zxy=\pi$ for any geodesic $[zx]$.

\parbf{\ref{ex:adjacent-CBB}.}
By \ref{ex:adjacent-angles}, we have
\[\mangle\hinge pxz+\mangle\hinge pyz\ge \pi.\]
Since $z\in \left]xy\right[$ we have 
\[\angk z{\bar x}{\bar y}=\pi\]
for any $\bar x\in \left[xz\right[$ and $\bar y\in \left]zy\right]$.
By comparison, we have that 
\[\angk z{\bar x}{\bar p}+\angk z{\bar p}{\bar y}\le\pi\]
for any $\bar p\in \left]zp\right]$.
Passing to the limit as
$\dist{z}{\bar x}{}\to 0$,
$\dist{z}{\bar y}{}\to 0$, and
$\dist{z}{\bar p}{}\to 0$,
we get the statement.

\parbf{\ref{ex:pxyvw}.} 
Without loss of generality, we can assume that $x$, $v$, $w$, and $y$ appear on 
$[xy]$ in this order.
By \ref{clm:angle-mono},
\[
\angk xyp\ge \angk xwp \ge\angk xvp.
\]
Hence, $\Rightarrow$ follows.

By Alexandrov's lemma,
\begin{align*}
\angk xyp=\angk xvp
\quad&\Longleftrightarrow\quad
\angk yxp=\angk yvp,
\\
\angk xyp=\angk xwp
\quad&\Longleftrightarrow\quad
\angk yxp=\angk ywp.
\end{align*}
Whence, $\Leftarrow$ follows.

\parbf{\ref{ex:angle-lim}.} Suppose $\mangle \hinge {x_\infty}{y_\infty}{z_\infty}>\alpha$.
Then we can choose $\bar y_\infty\in\left]x_\infty y_\infty\right]$
and $\bar z_\infty\in\left]x_\infty z_\infty\right]$ such that 
$\angk{x_\infty}{\bar y_\infty}{\bar z_\infty}>\alpha$.
Now choose $\bar y_n\in\left]x_n y_n\right]$ and $\bar y_n\in\left]x_n z_n\right]$ such that $\bar y_n\to \bar y_\infty$ and $\bar z_n\to \bar z_\infty$.
Observe that 
\[\liminf_{n\to\infty}\mangle \hinge {x_n}{y_n}{z_n}\ge\liminf_{n\to\infty}\angk{x_n}{\bar y_n}{\bar z_n} \ge \alpha,\]
hence the result.

\parbf{\ref{ex:urysohn}.}
The Urysohn space provides an example;
see for example \cite[Lecture 2]{petrunin2023pure}.

\parbf{\ref{ex:normCBB}.}
Choose a triangle $[0vw]$.
Note that $m=\tfrac12(v+w)$ is the midpoint of $[vw]$.

Use comparison, to show that
\[2\cdot |\tfrac12(v+w)|^2+2\cdot |\tfrac12(v-w)|^2\ge |v|^2+|w|^2.\]

Note this inequality implies the opposite one;
it follows if we rewrite it via $x=\tfrac12(v+w)$ and $y=\tfrac12(v-w)$.
Hence we have 
\[2\cdot |\tfrac12(v+w)|^2+2\cdot |\tfrac12(v-w)|^2= |v|^2+|w|^2\]
for any $v,w$.
That is, the norm is quadratic and the statement follows.

\parbf{\ref{ex:alm-min}.}
Suppose such a point does not exist;
that is, for any $p\in \spc{X}$ there is a point $p'$ such that $r(p')\le  (1-\eps)\cdot r(p)$ and $\dist p{p'}{}<\tfrac{1}{\eps}\cdot r(p)$.
Construct a sequence of points $p_0,p_1,\dots$ such that $p_n=p_{n-1}'$ for any~$n$.
Show that this sequence is Cauchy; denote its limit by $p_\infty$.
Arrive at a contradiction by showing that $r(p_\infty)\le0$.

\parbf{\ref{ex:CBB(1)notitCBB(0)}.}
Note that $\spc{X}$ has no defined sphericlal model angles;
therefore it has curvature $\ge 1$.

However, $\spc{X}$ does not have curvature $\ge 0$ since
\[\angk  p{x_1}{x_2}_{\EE^2}=\angk  p{x_2}{x_3}_{\EE^2}=\angk  p{x_1}{x_3}_{\EE^2}=\pi.\]

\parbf{\ref{ex:RisCBB(1)}.}
Suppose $\mangle\hinge mxp\ne 0$ and $\mangle\hinge mxp\ne\pi$, or equivalently $\mangle\hinge mxq\ne0$.

We can assume that $\dist pq{}$ only slightly exceeds $\pi$,
so $\dist pm{}<\pi$ and $\dist qm{}<\pi$.
We can also assume that $\dist xm{}<\pi$.
Use the comparison to show that 
\[\dist px{}+\dist qx{} < \dist pq,\]
and arrive at a contradiction with the triangle inequality.

Extend $[pq]$ to a maximal local geodesic $\gamma$.
It might be a closed or a line segment.
Argue as above to show that any point lies on $\gamma$ and make a conclusion.

\parbf{\ref{ex:perim-k>0}.}
Arguing by contradiction, suppose 
\[\dist{p}{q}{}+\dist{q}{r}{}+\dist{r}{p}{}> 2\cdot\pi\eqlbl{eq:perimeter-of-triange<2pi}\] 
for $p,q,r\in \spc{A}$. 
Rescaling the space slightly, we can assume that $\diam\spc{A}<\pi$,
but the inequality \ref{eq:perimeter-of-triange<2pi} still holds.
By \ref{clm:K>k},
after rescaling $\spc{A}$ is still $\Alex1$.

Take $z_0\in [q r]$ on maximal distance from $p$.
Consider the following model configuration:
two geodesics $[\tilde p\tilde z_0]$, $[\tilde q\tilde r]$ in $\mathbb{S}^2$ such that 
\begin{align*}
\dist{\tilde p}{\tilde z_0}{}&=\dist{p}{z_0}{},
&  
\dist{\tilde q}{\tilde r}{}&=\dist{q}{r}{},
\\ 
\dist{\tilde z_0}{\tilde q}{}&=\dist{z_0}{q}{},
&  
\dist{\tilde z_0}{\tilde r}{}&=\dist{z_0}{q}{},
\end{align*}
and 
\[\mangle\hinge{\tilde z_0}{\tilde q}{\tilde p}
=\mangle\hinge{\tilde z_0}{\tilde r}{\tilde p}
=\tfrac\pi2.\]

Let $\tilde z\in [\tilde q\tilde r]$,
and let $z\in [q r]$ be the corresponding point.
By comparison, $\dist pz{}\le\dist {\tilde p}{\tilde z}{}$ for points $z$ near $z_0$.
Moreover, this inequality holds as far as 
\[\dist{\tilde p}{\tilde z_0}{}+\dist{\tilde z_0}{\tilde z}{}+\dist{\tilde p}{\tilde z}{}<2\cdot\pi.\]
But this inequality holds for all $\tilde z$ since  $\dist{\tilde p}{\tilde z_0}{}<\pi$, $\dist{\tilde z_0}{\tilde q}{}<\pi$, and $\dist{\tilde z_0}{\tilde r}{}<\pi$.
Hence we get $\dist pq{}\le\dist {\tilde p}{\tilde q}{}$ and $\dist pr{}\le\dist {\tilde p}{\tilde r}{}$.
The latter contradicts \ref{eq:perimeter-of-triange<2pi}.

\parbf{\ref{ex:dir-compact}.}
Suppose $\dir p{x_n}\not\to\dir p{x_\infty}$.
Since $\Sigma_p$ is compact, we may pass to a converging subsequence of $\dir p{x_n}$;
denote by $\xi$ its limit.
We may assume that $\mangle (\dir p{x_\infty},\xi)>0$.

Denote by $\gamma_n$ and $\gamma_\infty$ the arc-length parametrization of $[px_n]$ and $[px_\infty]$ from $p$.
Choose a geodesic $\alpha$ that starts from $p$ and goes in a direction sufficiently close to $\xi$.
By comparison we can choose $\alpha$ so that
\[\dist{\alpha(t)}{\gamma_n(t)}{}<\eps\cdot t\]
for all large $n$ and all sufficiently small $t$.
Moreover, we can assume that
\[\dist{\alpha(t)}{\gamma_\infty(t)}{}>a\cdot t\]
for some fixed $a>0$ and all small $t$.
These two inequalities imply 
that 
\[\dist{\gamma_n(t)}{\gamma_\infty(t)}{}>\tfrac a2\cdot t\]
for all small $t$ and all large $n$.
On the other hand, by assumption, $\dist{\gamma_n(t)}{\gamma_\infty(t)}{}\to0$ as $n\to\infty$ --- a contradiction.

\parit{Comments.}
The compactness of $\Sigma_p$ is necessary.
An example can be built using iterated warped product of line segments and applying \cite[Theorem 1.2]{alexander-bishop2004}.
The space $\spc{A}$ can be assumed to be compact.


\parbf{\ref{ex:geodesic-cone}.}
Note that any point of $\Cone \spc{X}$ can be connected to the origin by a geodesic.
Given a nonzero element $v\in\Cone \spc{X}$, denote by $v'$ its projection in $\spc{X}$.

Suppose $\spc{X}$ is $\pi$-geodesic.
Choose two nonzero elements $v,w\in\Cone \spc{X}$; let $\alpha=\mangle(v,w)=\dist{v'}{w'}{\spc{X}}$.
If $\alpha\ge \pi$, then the product of geodesics $[v0]\cup [0w]$ forms a geodesic $[vw]$.
If $\alpha<\pi$, there is a geodesic $\gamma\:[0,\alpha]\to \spc{X}$ from $v'$ to $w'$.
Consider hinge $\hinge {\tilde o}{\tilde v}{\tilde w}$ in the plane 
such that $\mangle\hinge {\tilde o}{\tilde v}{\tilde w}=\alpha$, $\dist{\tilde o}{\tilde v}{}=|v|$, and $\dist{\tilde o}{\tilde w}{}=|w|$.
Let $t\mapsto (\phi(t),r(t))$ be geodesic $[\tilde v\tilde w]$ written in polar coordinates with origin $\tilde o$, so that $\phi(0)=0$.
Show that $t\mapsto r(t)\cdot\gamma\circ\phi(t)$ is a geodesic from $v$ to $w$;
here we identify $\spc{X}$ with the unit sphere in $\Cone \spc{X}$.

To prove the converse, try to reverse the steps in the argument above.

\parbf{\ref{ex:GHto-tangent}.}
Let  $\spc{A}_n=\lambda_n\cdot\spc{A}$.
Note that for any $n$ the space $\Sigma_p\spc{A}$ is identical to $\Sigma_{\iota_n(p)} \spc{A}_n$.
In particular, we can identify isometrically $\T_p\spc{A}$ with $\T_{\iota_n(p)}(\lambda\cdot \spc{A})$.
So for any geodesic $\gamma$ that starts at $p$, the vector $\gamma^+(0)$ corresponds to $\frac{1}{\lambda}\cdot(\iota_n\circ\gamma)^+(0))$.

Consider the logarithm maps $f_n=\log_{\iota_n(p)}\:\spc{A}_n\to T_p\spc{A}$.
We claim that this sequence of maps satisfies the assumptions of Lemma~\ref{lem:almost-isom-pointed};
the condition in \ref{SHORT.lem:almost-isom-pointed-basepoint} is evident.  

Note that it is sufficient to check the conditions in \ref{SHORT.lem:almost-isom-pointed-b} and \ref{SHORT.lem:almost-isom-pointed-c} only for $R=1$. 

Choose $\eps>0$.
By compactness of $\Sigma_p$ we can find a finite $\eps$-net $\xi_1,\dots,\xi_N$ in $\Sigma_p$. Moreover, without loss of generality we can assume that these directions are geodesic;
that is, there exist geodesics $\gamma_1,\ldots, \gamma_N$ starting at $p$ such that $\xi_i=\gamma_i^+(0)$ for each $i$.

Choose $T>0$ such that all $\gamma_i$ are defined on $[0,T]$.
Apply the comparison to show that for any $\lambda_n>\frac{1}{T}$ the image under $f_n$ of the union $\bigcup_N\gamma_i([0,T])$ is an $\eps$-net in $\oBall(0,1)_{T_p}$.
This proves \ref{SHORT.lem:almost-isom-pointed-c}.

By comparison, we have that
\[\dist{\xi_i}{\xi_j}{\Sigma_p}\ge \angk p{\gamma_i(t_i)}{\gamma_j(t_j)}<\eps\]
for all $i\ne j$ and any $t_i,t_j\in (0,T]$.
By the definition of an angle, we can assume that $T$ have been chosen so that in addition 
\[\dist{\xi_i}{\xi_j}{\Sigma_p}\le \angk p{\gamma_i(t)}{\gamma_j(t)}+\eps\]
for all $i\ne j$ and any $t\in (0,T]$.

By construction of the map $f_n$ this implies that 
\[|\dist{x}{x'}{\spc{A}_n}-\dist{f_n(x)}{f_n(x')}{T_p}|<\eps\]
for all $\lambda_n>\frac{1}{T}$ and all points $x,x'$ in $\bigcup_N\gamma_i([0,\frac{1}{\lambda_n}])\subset \oBall(p,1)_{\spc{A}_n}$.
  
Now hinge comparison and the triangle inequality imply that the same  holds for arbitrary points $x,x'$  in  $\oBall(p,1)_{\spc{A}_n}$ with $\eps$ replaced by $3\eps$.
This verifies \ref{SHORT.lem:almost-isom-pointed-b}.

\parbf{\ref{ex:distfun-semiconcave}.} From \ref{comp-kappa}, this inequality follows in the sense of distributions, and hence in any other sense.

\parbf{\ref{ex:df(xi)}.}
Since angles are defined, it follows that 
\[\dist{\gamma_1(t)}{\gamma_2(t)}{}\le \theta\cdot t\]
for all small $t>0$.     
Since $f$ is $L$-Lipschitz, we get 
\[|f(\gamma_1(t))-f(\gamma_2(t))|\le L\cdot \theta\cdot t,\]
hence the statement.

\parbf{\ref{ex:d(distfun)}}; \ref{SHORT.ex:d(distfun):<}
Note that we can assume there is a geodesic in the direction of $v$, and apply \ref{ex:first-var}.

\parit{\ref{SHORT.ex:d(distfun):=}.}
By \ref{SHORT.ex:d(distfun):<}, $\dd_p\distfun_q(v)\le-\max_{\xi\in\Uparrow_p^q}\langle\xi,v\rangle$.
Suppose this inequality is strict for some $v$.
We can assume that $|v|=1$ and there is a geodesic, say $\gamma$ in the direction of $v$.
Let $\dd_p\distfun_q(v)=-\cos\alpha_0$ for some $\alpha\in [0,\pi]$.
Note that any geodesic from $p$ to $q$ makes angle bigger than $\alpha_0$ with $\gamma$.


The function $f=\distfun_q\circ\gamma$ is Lipschitz.
By Rademacher's theorem it is differentiable almost everywhere;
moreover, 
\[f(t)-f(0)=\int_0^t f'(t)\cdot dt.\]
Suppose $f'(t)$ is defined.
Use \ref{SHORT.ex:d(distfun):<} to show that 
$f'(t)=-\cos\alpha(t)$, where $\alpha(t)$ is the angle between $\gamma$ and any geodesic from $\gamma(t)$ to $q$.
Note that we can choose a sequence $t_n\to 0$ such that 
\[\lim_{n\to\infty}\alpha(t_n) \le \alpha_0.\]
Consider a sequence of geodsics $[p\,\gamma(t_n)]$.
Since the space is proper, we can pass to its convergent subsequence.
Its limit is a geodesic from $p$ to $q$, denote it by $[pq]$.

Use \ref{ex:angle-lim} to show that $[pq]$ makes an angle at most $\alpha_0$ with $\gamma$ --- a contradiction.
 
\parbf{\ref{ex:monotonicity}.}
Let $\gamma\:[0,\ell]\to \spc{A}$ be the geodesic $[xy]$ parametrized from $x$ to $y$,
and let $\phi=f\circ\gamma$.
Observe that 
\[\phi'(0)=\dd_xf(\dir xy)\le \<\dir{x}{y},\nabla_{x}f\>.\]
The same way we get $-\phi'(\ell)\le \<\dir{y}{x},\nabla_{y}f\>$.
Since $f$ is $\lambda$-concave, we have
\begin{align*}
f(y)&\le f(x)+\phi'(0)\cdot \ell+\tfrac\lambda2\cdot\ell^2,
\\
f(x)&\le f(y)-\phi'(\ell)\cdot \ell+\tfrac\lambda2\cdot\ell^2.
\end{align*}
Hence the statement follows.

\parbf{\ref{ex:d(distfun):==}.}
If the space is proper, then the statement follows from \ref{SHORT.ex:d(distfun):=} and \ref{ex:pi-angle}.

To do the general case argue by contradiction.
Let $z$ be a point on the extension of $[pq]$ behind $q$;
it exists by the assumption.
Note that we can assume that $|v|=1$ and it is a direction of a geodesic, say $[px]$.

Show that for there is a sequence $x_n\in \left]px\right]$ such that $\dist{p}{x_n}{}\to0$ ad
$\mangle \hinge q{x_n}p>\eps$ for each $n$ and some fixed $\eps>0$.
Observe that $\mangle\hinge q{x_n}z\z<\pi-\eps$; therefore
\[\dist{z}{x_n}{}<\dist{x_n}{q}{}+\dist{q}z{}-\delta\]
for each $n$ and some fixed $\delta>0$.
Pass to the limit as $x_n\to p$ and arrive at a contradiction.

\parbf{\ref{ex:convergence-grad}.}
Note that
$|(\dd_p f)(v)-(\dd_p g)(v)|\le s\cdot|v|$
for any $v\in \T_p$.
From the definition of gradient (\ref{def:grad}) we have:
\begin{align*}
&(\dd_p f)(\nabla_p g)\le\<\nabla_p f,\nabla_p g\>,
&&(\dd_p g)(\nabla_p f)\le\<\nabla_p f,\nabla_p g\>,
\\
&(\dd_p f)(\nabla_p f)=\<\nabla_p f,\nabla_p f\>,
&&(\dd_p g)(\nabla_p g)=\<\nabla_p g,\nabla_p g\>.
\end{align*}
Therefore,
\begin{align*}
&\dist[2]{\nabla_pf}{\nabla_pg}{}
=\<\nabla_p f,\nabla_p f\>+\<\nabla_p g,\nabla_p g\>-2\cdot\<\nabla_p f,\nabla_p g\>
\le
\\
&\le (\dd_p f)(\nabla_p f)+(\dd_p g)(\nabla_p g)-
(\dd_p f)(\nabla_p g)-(\dd_p g)(\nabla_p f)
\le
\\
&\le s\cdot(|\nabla_p f|+|\nabla_p g|).
\end{align*}

\parbf{\ref{ex:semicontinuous-grad}.}
Suppose $|\nabla_xf|> s$.
Then we can choose a geodesic $\gamma$ that starts at $x$ such that 
$(f\circ\gamma)^+(0)>s$.
In particular, there is $\eps>0$ such that
\[f\circ\gamma(t)>(s+\eps)\cdot t+o(t),\]
hence the only-if part follows.

Now suppose $f(y)-f(x)>s\cdot \ell+\lambda\cdot \tfrac{\ell^2}2$,
were $\ell=\dist{x}{y}{}$.
Let $\gamma\:[0,\ell]\to \spc{A}$ be a geodesic from $x$ to $y$.
Since $f\circ\gamma$ is $\lambda$-concave, we have
\[f\circ\gamma(\ell)\le f\circ\gamma(0)+(f\circ\gamma)^+(0)\cdot\ell+\lambda\cdot \tfrac{\ell^2}2.\]
It follows that 
\[\dd_x(\dir xy)=(f\circ\gamma)^+(0)>s,\]
and by \ref{prop:grad-exist}, $|\nabla_x f|>s$.

\parbf{\ref{ex:elf-contracting}.}
Note that $f\circ\alpha$ is a nondecreasing function.
Apply \ref{ex:d(distfun):<} and the definition of gradient to show that
\[
-\dd_{\alpha(t)}\distfun_{\alpha(t_3)}(\nabla_{\alpha(t)}f)
\ge
\langle \nabla_{\alpha(t)},\dir{\alpha(t)}{\alpha(t_3)}\rangle
\ge
\dd_{\alpha(t)}(\dir{\alpha(t)}{\alpha(t_3)})
\ge0
\]
for any $t<t_3$.
Conclude that the function 
$t\mapsto \distfun_{\alpha(t_3)}\circ\alpha(t)$ is noncreasing for $t\le t_3$.

\parbf{\ref{ex:mayer}.}
For any $s>s_0$,
\begin{align*}
(f\circ\hat\alpha)^+(s_0)&=|\nabla_{\hat\alpha(s_0)}f|
\ge
\\
&\ge
(d_{\hat\alpha(s_0)}f)(\dir{\hat\alpha(s_0)}{\hat\alpha(s)})
\ge
\\
&\ge
\frac{f\circ\hat\alpha(s)-f\circ\hat\alpha(s_0)}{\dist{\hat\alpha(s)}{\hat\alpha(s_0)}{}}.
\end{align*} 
Since $s-s_0\ge\dist{\hat\alpha(s)}{\hat\alpha(s_0)}{}$, for any $s>s_0$ we have 
\[(f\circ\hat\alpha)^+(s_0)\ge
\frac{f\circ\hat\alpha(s)-f\circ\hat\alpha(s_0)}{s-s_0}.\]

\parbf{\ref{lem:fg-dist-est}.}
Fix $t$, and let $p=\alpha(t)$ and $q=\beta(t)$.
Apply \ref{eq:fist-var-inq+} to get
\begin{align*}
 \ell^+
&\le -\<\dir{p}{q},\nabla_{p}f\>
-\<\dir{q}{p},\nabla_{q}g\>
\le
\\
&\le -{\left({f(q)}-{f(p)}-\lambda\cdot\tfrac{\ell^2}2\right)}/{\ell}
-{\left({g(p)}-{g(q)}-\lambda\cdot\tfrac{\ell^2}2\right)}/{\ell}\le
\\
&\le \lambda\cdot\ell+\tfrac{2\cdot\eps}{\ell}.
\end{align*}
Integrating this inequality, we get the second statement.

\parbf{\ref{ex:busemann-CBB}.} Apply \ref{ex:distfun-semiconcave}.

\parbf{\ref{ex:bus+bus}.} By the triangle inequality, 
\[\dist{\gamma(-t)}{x}{}+\dist{\gamma(t)}{x}{}-2\cdot t\ge 0\]
for any $t\ge 0$.
Passing to the limit as $t\to\infty$, we get the result.

\parbf{\ref{ex:cone-CBB}.}
Suppose $\Cone\spc{X}$ is $\Alex0$.
Observe that two half-lines in $\Cone\spc{X}$ that start from the origin and go into directions $x$ and $y\in\spc{X}$ form a line if and only if $\dist{x}{y}{\spc{X}}\ge \pi$.
Apply the splitting theorem to show that for any $x\in \spc{X}$ there is at most one point $y$ such that $\dist{x}{y}{\spc{X}}\ge \pi$ and in this case we have equality.
Conclude that $\diam \spc{X}\z\le \pi$.

Now choose a quadruple of points $p,x_1,x_2,x_3\in \spc{X}$;
we will identify $\spc{X}$ with the unit sphere in $\Cone\spc{X}$.
Suppose $\dist{p}{x_i}{}<\tfrac\pi2$ for any $i$.
Consider the following points in the cone: $y_i=\tfrac1{\cos \dist{p}{x_i}{\spc{X}}}\cdot x_i$, and $q=p$.
Show that $\EE^2$-comparison for $q,y_1,y_2,y_3$ in $\Cone\spc{X}$ implies $\SSS^2$-comparsion for $p,x_1,x_2,x_3$ in $\spc{X}$.
Conclude that $\spc{X}$ is locally $\Alex1$. 
Apply the globalization theorem (\ref{thm:globalization+}).

Now assume $\spc{X}$ is $\Alex1$ and $\diam\spc{X}\le \pi$.
By \ref{ex:perim-k>0}, the perimeter of any triangle in $\spc{X}$ is at most $2\cdot\pi$.
We need to check $\EE^2$-comparison for a given quadruple of points $q,y_1,y_2,y_3$ in $\Cone\spc{X}$.
We can assume that none of these points is the origin; otherwise perturb them a bit.

Set $x_i=y_i/|y_i|$ for each $i$ and $p=q/|q|$; we can assume that $p,x_1,x_2,x_3$ are distinct in $\spc{X}$, which is the unit sphere in $\Cone\spc{X}$.

Assume the model triangles $\modtrig(px_1x_2)$, $\modtrig(px_2x_3)$, and $\modtrig(px_3x_1)$ are defined;
that is, perimeters triangles $[px_1x_2]$, $[px_2x_3]$, and $[px_3x_1]$ are strictly less than $2\cdot\pi$. 
Note that $\EE^3\iso\Cone\SSS^2$.
Use this together with the $\SSS^2$-comparison for $p,x_1,x_2,x_3$ in $\spc{X}$ to show that $\EE^2$-comparison holds for $q,y_1,y_2,y_3$ in $\Cone\spc{X}$.

Finally, if some of the model triangles are not defined, consider rescaling of $\spc{X}$ with a coefficient $\lambda$ slightly smaller than 1.
Apply the argument above to show that the comparison holds for the corresponding points in $\Cone(\lambda\cdot\spc{X})$ and pass to the limit as $\lambda\to 1$.

\parit{Comment.}
The last part of the proof is close to the argument in \ref{thm:CBB-closed}.

\parbf{\ref{ex:|antisum|}.}
Observe that
\begin{align*}
\langle u,u\rangle+\langle v,u\rangle+\langle w,u\rangle &\ge 0,
\\
\langle u,v\rangle+\langle v,v\rangle+\langle w,v\rangle &\ge 0,
\\
\langle u,w\rangle+\langle v,w\rangle+\langle w,w\rangle &= 0.
\end{align*}
Add the first two inequalities and subtract the last identity.

\parbf{\ref{prop:two-opp}.}
Apply \ref{prop:opposite} to show that 
$\langle v,v\rangle =\langle v,w\rangle=\langle w,w\rangle$,
and use it.

\parbf{\ref{ex:3<,>=0}.} Show and use that
\[\langle u,x\rangle +\langle v,x\rangle +\langle w,x\rangle \ge 0\]
and
\[\langle u,-x\rangle +\langle v,-x\rangle +\langle w,-x\rangle \ge 0.\]

\parbf{\ref{ex:-u}.} Part $\Rightarrow$ is evident.
To prove part $\Leftarrow$, observe that 
\[\langle u^*,u^*\rangle =-\langle u,u^*\rangle\le \langle u,u\rangle\]
and since $|u|=|u^*|$, we have equality.

\parbf{\ref{ex:grad-dist}.}
Apply \ref{ex:-u}.

\parbf{\ref{ex:tangent=Em}.}
By \ref{ex:diam-compact:proper}, $\spc{A}$ is \emph{separable}; that is, it contains a countable dense set of points.
Apply \ref{cor:euclid-subcone} to this set.

\parbf{\ref{ex:dim=1}.} Argue as in \ref{ex:RisCBB(1)}.

\parbf{\ref{ex:resporka}.} The only-if part is trivial.
Suppose the configuration $p$, $a_0,\z\dots, a_{m}\in \spc{A}$ meets the condition.
By \ref{ex:grad-dist} the directions $\dir q{a_0},\z\dots,\dir q{a_m}\in \Lin_q$ for G-delta dense set of points $q\in \spc{A}$.
If $q$ is sufficiently close to $p$, then $\angk q{a_i}{a_j}>\tfrac\pi2$,
and therefore, $\mangle\hinge q{a_i}{a_j}>\tfrac\pi2$ for $i\ne j$.
Conclude that $\dim\Lin_q\ge m$ in this case.

\parbf{\ref{ex:finite-tan}}; 
\ref{SHORT.ex:finite-tan:tan}. Apply \ref{ex:geodesic-cone}, \ref{prop:Tan-is-CBB(0)}, and \ref{thm:finite-space-of-directions}.

\parit{\ref{SHORT.ex:finite-space-of-directions-dim}.}
Apply \ref{ex:resporka} to show that $\LinDim\T_p=\LinDim\spc{A}$ (argue as in \ref{prop:Tan-is-CBB(0)}).

\parit{\ref{SHORT.ex:finite-tan:sigma}.}
By \ref{thm:finite-space-of-directions} for any two points $\xi,\zeta\in\Sigma_p$ such that $\dist{\xi}{\zeta}{\Sigma_p}<\pi$ there is a geodesic $[\xi\zeta]_{\Sigma_p}$.
Suppose $\dist{\xi}{\zeta}{\Sigma_p}\ge\pi$, then $\T_p$ contains a line thru the origin in the directions $\xi$ and $\zeta$.
By \ref{SHORT.ex:finite-tan:tan} we can apply the splitting theorem (\ref{thm:splitting}) to $\T_p$.
We get that $\Sigma_p$ is a spherical suspension with poles $\xi$ and $\zeta$.
Hence, $\dist\xi\zeta{}=\pi$ and there is a geodesic $[\xi\zeta]$.


\parbf{\ref{ex:proof-right-inverse}}; \ref{SHORT.ex:proof-right-inverse:grad}.
By \ref{ex:distfun-semiconcave}, each function $\distfun_{a_i}$ is semiconcave in a small neighborhood of $p$.
Therefore we can choose $\lambda$ and $r>0$ so that $f_{\bm{y}}$ is $\lambda$-concave in $\oBall(p,r)$; further we will assume that $r$ is sufficiently small.
Choose $\alpha>0$ such that $\angk{x}{a_i}{a_j}>\tfrac\pi2+\alpha$ for all $i\ne j$;
we may assume that $\alpha<\tfrac{1}{10}$;
in particular,
\[(\dd_x\distfun_{a_j}{}{})(\dir{x}{a_i})
\ge
-\cos\angk{x}{a_i}{a_j}
>
\tfrac\alpha2\eqlbl{inq-a_j}\]
for $j\ne i$.

By the definition of gradient and \ref{ex:d(distfun):<}, we have
\begin{align*}
-(\dd_x\distfun_{a_i}{}{})(\nabla_x f_{\bm{y}})
&\ge
\<\dir x{a_i},\nabla_x f_{\bm{y}}\>
\ge
\\
&\ge
(\dd_xf_{\bm{y}})(\dir x{a_i}).
\end{align*}
If $\dist{a_i}{x}{}>y_i$, then 
\[\dd_xf_{\bm{y}}=\sigma+\eps\cdot \dd_x\distfun_{a_0},\]
where $\sigma$ is a minimum of a subset of the following functions
$0$, and $\dd_x\distfun_{a_j}$ for $0\ne j\ne i$.
By \ref{inq-a_j}, 
\[(\dd_x\distfun_{a_i}{}{})(\nabla_x f_{\bm{y}})< -\tfrac\alpha2\cdot\eps.\]
Hence (\ref{111}) holds for all sufficiently small $\eps>0$.

Now assume that $\dist{a_i}{x}{}-y_i=\min_j\{\dist{a_j}{x}{}\z-y_j\}<0$.
Then
\begin{align*}
\dd_x f_{\bm{y}}
&=
\min_{i\in S} \{\,\dd_x\distfun_{a_j}\,\}+\eps\cdot \dd_x\distfun_{a_0}
\le
\\
&\le
\dd_x \distfun_{a_i}{}{}+\eps\cdot(\dd_p\distfun_{a_0}{}{}),
\end{align*}
where $j\in S$ if and only if $\dist{a_i}{x}{}-y_i=\dist{a_j}{x}{}-y_j$.
Applying \ref{inq-a_j}, we get
\begin{align*}
(\dd_x \distfun_{a_i}{}{})(\nabla_x f_{\bm{y}})
&\ge 
\dd_xf_{\bm{y}}(\nabla_x f_{\bm{y}}) -\eps\cdot(\dd_x \distfun_{a_0}{}{})(\nabla_x f_{\bm{y}}) 
\ge 
\\
&\ge
\left[(\dd_xf_{\bm{y}})(\dir x{a_0})\right]^2-2\cdot \eps
\ge
\\
&\ge
\left[\tfrac\alpha2-\eps\right]^2-2\cdot \eps.
\end{align*}
Thus, (\ref{222}) holds for all sufficiently small $\eps>0$. 

\parit{\ref{SHORT.ex:proof-right-inverse:alpha}}
Consider the following real-to-real functions:
\[\begin{aligned}
\phi(t)
&\df
\max_{i}\{\dist{a_i}{\alpha_{\bm{y}}(t)}{}-y_i\},
\\
\psi(t)
&\df
\min_{i}\{\dist{a_i}{\alpha_{\bm{y}}(t)}{}-y_i\}.
\end{aligned}\eqlbl{eq:xy-def}\]
Use \ref{SHORT.ex:proof-right-inverse:grad}, to show that for $t\in[0,t_0]$, we have $\phi^+(t)<-\tfrac{1}{10}\cdot\eps^2$ if $\phi(t)>0$
and $\psi^+(t)>\tfrac{1}{10}\cdot\eps^2$ if $\psi(t)<0$.
Conclude that $\phi(t_0)=\psi(t_0)=0$; hence the result.


\parit{\ref{SHORT.ex:proof-right-inverse:end}}
A straightforward application of \ref{lem:fg-dist-est} and a reformulation of \ref{SHORT.ex:proof-right-inverse:alpha}.

\parbf{\ref{ex:proof-dist-chart}.}
Apply the (\textit{n}+1)-comparison (\ref{thm:n+1}) to show that at least one of the inequalities
\[
\mangle\hinge xy{a_0}<\tfrac\pi2-\eps,\ \dots,\  \mangle\hinge xy{a_m}<\tfrac\pi2-\eps,
\]
holds.
Similarty, we get that at least one of the inequalities
\[
\mangle\hinge yx{a_0}<\tfrac\pi2-\eps,\ \dots,\  \mangle\hinge yx{a_m}<\tfrac\pi2-\eps,
\]
holds.

Suppose our statement does not hold for $x$ and $y$ in a sufficiently small neighborhood of $p$.
It follows that 
\[\mangle\hinge yx{a_0}<\tfrac\pi2-\eps
\quad\text{and}\quad
\mangle\hinge yx{a_0}<\tfrac\pi2-\eps.
\eqlbl{eq:a0}
\]
Note that $\dist{x}{y}{}$ is small compared to $\dist{a_0}{x}{}$ and $\dist{a_0}{y}{}$.
Therefore, the comparison contradicts \ref{eq:a0}. 

By the construction, $f$ is Lipschitz.
From above, we can choose $i>0$ so that $\mangle\hinge xy{a_i}<\tfrac\pi2-\eps$ (if $\mangle\hinge yx{a_i}<\tfrac\pi2-\eps$, then swap $x$ and $y$).
By comparison, there is $c>0$ such that $\dist{a_i}{y}{}\le \dist{a_i}{x}{}+c\cdot \dist{x}{y}{}$.
Hence $f$ is bi-Lipschitz, and now \ref{thm:right-inverse} implies \ref{thm:dist-chart}.


 
\parbf{\ref{ex:diam-compact:proper}.}
Reuse the argument from  the first part of the proof of Bishop--Gromov inequality.

\parbf{\ref{ex:BG}.} 
You should follow the proof Bishop--Gromov inequality, plus prove the following two inequalities 
\begin{align*}
\sinh r_2\cdot \dist{\log_p x}{\log_p y}{\T_p} &\ge\dist{x}{y}{\spc{A}}
\\
\sinh r_2\cdot\dist{w(x)}{w(y)}{\spc{A}} &\ge \sinh r_1\cdot\dist{x}{y}{\spc{A}}
\end{align*}
for any $x,y\in\oBall(p,r)$.

\parbf{\ref{ex:dim=dim}.} 
Suppose $K$ is a compact set in $\spc{A}$ such that $\HausDim K\ge m$.
Use the map $w$ from the proof of the Bishop--Gromov inequality (\ref{inq:BG} and \ref{ex:BG}) to show that any open ball in $\spc{A}$ contains a compact set $K'$ such that $\HausDim K'\ge m$.

Use this in addition to the arguments in \ref{thm:dim=dim}. 

\parbf{\ref{ex:dim-lim}.}
Apply \ref{ex:resporka}.

\parbf{\ref{ex:net}};
\ref{SHORT.ex:net:finite}.
Suppose $X$ is compact.
Then for any $\eps>0$ any cover of $X$ by open $\eps$-balls have a finite subcover.
Note that the centers of these balls is an $\eps$-net of $X$.

Suppose $X$ has a finite $\eps$-net.
Show that any sequence $x_n$ of points in $X$ has a subsequence such that all of its points lie in one $\eps$-ball.
Apply this statement for $\eps=\tfrac1n$ together with the diagonal procedure.

\parit{\ref{SHORT.ex:net:compact}.}
Let $Z$ be a compact $\eps$-net of $X$.
By \ref{SHORT.ex:net:finite}, $Z$ admits a finite $\eps$-net $F$.
Note that $F$ is a $2\cdot\eps$-net of $X$.
Since $\eps>0$ is arbitrary, we get the result.


\parbf{\ref{ex:pack-net}.} If $x_1,\dots,x_n$ is not an $\eps$-net, then there is a point $y$ such that $\dist{x_i}{y}{}\ge\eps$ for any $i$.
Therefore $x_1,\dots,x_n$ is not a maximal packing --- a contradiction.

\parbf{\ref{ex:pack-vol}}; \ref{SHORT.ex:pack-vol:pack}
Apply the Bishop--Gromov inequality (\ref{inq:BG}).

\parit{\ref{SHORT.ex:pack-vol:dim}}
By \ref{ex:dim-lim}, $\dim\spc{A}_\infty\le m$.
To show that $\dim\spc{A}_\infty\ge m$,
apply \ref{cor:euclid-subcone} to a maximal packing and use the estimate in \ref{SHORT.ex:pack-vol:pack}.

\parit{Comment.}
A stronger statement holds 
\[\vol_m\spc{A}_\infty=\lim_{n\to\infty} \vol_m\spc{A}_n;\]
in other words, if $\bm{K}\subset \GH$ denotes the set of isometry classes of all compact $\Alex\kappa$ spaces with dimension $\le m$, then the function
$\vol_m\:\bm{K}\to \RR$ is continuous.


\parbf{\ref{ex:diam-compact:GH}.}
Argue as in \ref{thm:gromov-compactness} to construct a Gromov--Hausdorff convergence of $\cBall(p_n,R)_{\spc{A}_n}$ for given $R>0$, then apply the diagonal procedure to construct the needed convergence.

\parbf{\ref{ex:no-conc}.}
Consider the infinite product $\SSS^1\times ({\tfrac 12}\cdot \SSS^1)\times ({\tfrac 14}\cdot \SSS^1)\times\dots$

\parbf{\ref{ex:conic}.}
Let $V$ and $W$ be two conic neighborhoods of a point~$p$.
Without loss of generality, we may assume that $V\Subset W$;
that is, the closure of $V$ lies in $W$.

Construct a sequence of embeddings $f_n\:V\to W$
such that 
\begin{itemize}
\item 
For any compact set $K\subset V$ 
there is a positive integer $n=n_K$ such that 
$f_n(k)=f_m(k)$ for any $k\in K$ and $m, n \ge n_K$.
\item For any point $w\in W$ there is a point $v\in V$ such that $f_n(v)=w$ for all large $n$.
\end{itemize}

Note that once such a sequence is constructed, $f\:V\to W$ defined by $f(v)=f_n(v)$ for all large values of $n$ gives the needed homeomorphism.

The sequence $f_n$ can be constructed recursively
\[f_{n+1}=\Psi_n\circ f_n\circ \Phi_n,\]
where $\Phi_n\:V\to V$ 
and $\Psi_n\:W\to W$ 
are homeomorphisms
of the form 
\[\Phi_n(x)=\phi_n(x)\ast x\quad \text{and}\quad \Phi_n(x)=\psi_n(x)\star x,\]
where $\phi_n\:V\to \RR_{\ge 0}$, $\psi_n\:W\to \RR_{\ge 0}$ are suitable continuous functions;
``$\ast$'' and ``$\star$'' denote the multiplications in the cone structures of $V$ and $W$ respectively.

\parit{Comment.} If it is hard to follow, read the original proof by Kyung Whan Kwun \cite{kwun1964}.

\parbf{\ref{ex:conic-tangent}}; \ref{SHORT.ex:conic-tangen:tangent}. Apply \ref{thm:spherical-nbhd} and \ref{lem:kwun}.

\parit{\ref{SHORT.ex:conic-tangen:dir}.} Apply \ref{SHORT.ex:conic-tangen:tangent}.

\parit{\ref{SHORT.ex:conic-tangen:example}.} Recall that the Poincaré homology sphere can be obtained as a quotient space $\Sigma=\SSS^3/\Gamma$ by an isometric action of a finite group $\Gamma$  --- the so-called binary icosahedral group.
By the double suspension theorem,  $\Susp^2\Sigma\cong\SSS^5$.
Note that $\Susp^2\Sigma$ is an Alexandrov space and it has a point with space of directions isometric to $\Susp\Sigma$.
Observe that $\Susp\Sigma$ is not a manifold; in particular $\Susp\Sigma\ncong\SSS^4$.
Therefore the pair $\Susp^2\Sigma$ and $\SSS^5$ provides the needed example.

\parbf{\ref{ex:bry2bry}.} Apply \ref{thm:spherical-nbhd}, \ref{lem:kwun}, and \ref{thm:top-bry}.

\parbf{\ref{ex:bry-closed}.}
Let $\spc{A}$ be a finite-dimensional Alexandrov space.
Choose $x\in\spc{A}$.
By \ref{thm:spherical-nbhd}, a neighborhood $U\ni x$ is homeomorphic to $\T_x$.
Therefore \ref{ex:bry2bry}, implies that $U\cap\partial\spc{A}=\emptyset$ if and only if $x\notin \partial\spc{A}$;
that is, the complement $\spc{A}\setminus\partial\spc{A}$ is open, and therefore, $\spc{A}$ is closed.

\parbf{\ref{ex:pz<ypz}.}
Consider the model triangle $[\tilde x\tilde y\tilde z']=\modtrig(xyz)$.
\begin{figure}[ht!]
\vskip-0mm
\centering
\includegraphics{mppics/pic-1015}
\end{figure}

Show that 
\[\dist{\tilde p}{\tilde z}{}\le \dist{\tilde p}{\tilde z'}{}\le\side\hinge yp{z}.\]


\parbf{\ref{ex:bry-connected}.}
Assume $\spc{A}$ has at least two boundary components, say $A$ and $B$.
Denote by $\gamma$ a geodesic that minimizes the distance from $A$ to $B$.

Let 
\[\dots,\spc{A}_{-1},\spc{A}_{0},\spc{A}_{1},\dots\]
be a two-sided infinite sequence of copies on $\partial\spc{A}$.
Let us glue $\spc{A}_{i}$ to $\spc{A}_{i+1}$ along $A$ if $i$ is even and along $B$ if $i$ is odd.

By the doubling theorem, every point in the obtained space $\spc{N}$ has a neighborhood that is isometric to a neighborhood of the corresponding point in $\spc{A}$ or its doubling.
By the globalization theorem, $\spc{N}$ is $\Alex1$.

Note that the copies of $\gamma$ in $\spc{A}_{i}$ form a line in $\spc{N}$.
By the splitting theorem, $\spc{N}$ is isometric to a product $\spc{N}'\oplus \RR$.
Since $\dim\spc{N}>1$, Exercise~\ref{ex:dim=1} implies that $\diam\spc{N}\le \pi$ --- a contradiction.

\parbf{\ref{ex:dist-to-bry}.} Choose $x$ on $\gamma$;
we can assume that $x=\gamma(0)$.
Let $y\in \partial \spc{A}$ be a closest point to $x$.
Let $\alpha=\mangle(\dir xy,\gamma^+(0)$.

Suppose $x\notin \partial \spc{A}$.
Show that $\T_y=\RR_{\ge0}\times\T_y\partial \spc{A}$
and $\dir yx\perp \T_y\partial \spc{A}$.

Given a vector $v\in \T_y$, denote by $\bar v$ its projection to $\T_y\partial \spc{A}$.
Apply the comparison and \ref{prop:gexp} to show that 
\[\dist{\gamma(t)}{\gexp_y(\overline{\log_x\gamma(t)})}{}\le \dist{x}{y}{}+t\cdot\cos\alpha.\]
Conclude that $\gamma''(0)\le 0$ in the barrier sense.


\parbf{\ref{ex:liberman}.}
Suppose $\gamma$ is defined on the interval $[0,\ell]$.
Assume that the function $\rho\:t\mapsto \tfrac12\cdot\distfun_p^2\circ\gamma(t)$ is not $1$-concave.
Let $\bar\rho\:[0,\ell]\to\RR$ be the minimal $1$-concave function such that $\bar\rho\ge \rho$.
Note that $\bar\rho=\rho$ at the ends of $[0,\ell]$.

Consider the curve $\bar\gamma(t)\df \GF_f^{s(t)}\gamma(t)$;
where $f=\tfrac12\cdot\distfun_p^2$ and $s(t)\z=\ln\circ\bar\rho(t)-\ln\circ\rho(t)$.
Use the first distance estimate to show that $\length\bar\gamma<\length\gamma$ and arrive at a contradiction.

\parit{Comment.}
The statement was proved by Grigory Perelman and the second author \cite{perelman-petrunin};
it generalizes a theorem of Joseph Liberman \cite{liberman} about geodesics on convex surfaces.
The original Liberman's version of the following geometric statement.
\textit{Suppose that $C$ is the cone over $\gamma$ with the vertex at $p$,
where $\gamma$ is a geodesic on a convex surface and $p$ is a point in the convex body bounded by the surface.
Then after unfolding $C$ into plane, $\gamma$ becomes a locally convex curve.}
It is instructive to check that this formulation is equivalent to ours for convex bodies.

\parbf{\ref{ex:native}.}
Choose a geodesic $\gamma$ in $\spc{W}$.
Arguing as in the proof of \ref{thm:doubling:doubling}, we get 
that $\gamma$ can cross the common boundary of two halves $\spc{A}_0$ and $\spc{A}_1$ of $\spc{W}$ at most once, or it lies in the common boundary.

In the later case $\lambda$-concavity of $f\circ\proj\circ\gamma$ follows from $\lambda$-concavity of $f$.
In the former case the convexity has to be checked only at the point of crossing;
we may assume that it happens at $x=\gamma(0)$.
Since $\nabla_xf\in\partial\T_x$ for any $x\in\partial\spc{A}$ the $f$-gradient flows agree on $\spc{A}_0$ and $\spc{A}_1$.

Assume $f\circ\proj\circ\gamma$ is not $\lambda$-concavity at $0$.
Apply $f$-gradinent flow to shorten $\gamma$ keeping its ends as in the proof of \ref{ex:liberman},
and arrive at a contradiction.

\parbf{\ref{ex:Hilbert/G}.} Read \cite[Section 4]{terng-thorbergsson} and/or the solution for ``Quotient of the Hilbert space'' in \cite{petrunin2020}.

\parbf{\ref{ex:sumbetries(S^2)}}; \ref{SHORT.ex:sumbetries(S^2):1}.
Choose an isometric $\SSS^1$-action on $\SSS^2$ that fixes the poles of the sphere.
Consider the projection to the quotient space $\sigma_1\:\SSS^2\z\to \SSS^2/\SSS^1=[0,\pi]$.

\parit{\ref{SHORT.ex:sumbetries(S^2):2}.}
Take a half-circle $\gamma$ on $\SSS^2$ and define 
$\sigma_2(x)\df\distfun_\gamma(x)_{\SSS^2}$.

\parit{\ref{SHORT.ex:sumbetries(S^2):n}.}
Consider the subdivision of $\SSS^2$ into $\SSS^1$-orbits of the action from~\ref{SHORT.ex:sumbetries(S^2):1}.
Cut $\SSS^2$ into two hemispheres by meridians rotate one hemisphere by an angle $\alpha=\pi/n$ and glue it back.
Observe that there is a submetry $\sigma_n$ such that the inverse image $\sigma_n^{-1}\{y\}$ is a union of the arcs from the original $\SSS^1$-orbits.

Note that for $n=2$ we get the solution in \ref{SHORT.ex:sumbetries(S^2):2}.

\parbf{\ref{ex:sumbetries(E^2)}.}
Show that for any $x\in\EE^2$ there is a half-line $H\ni x$ such that 
the restriction $\sigma|_H$ is an isometry.
Suppose such a half-line $H$ starts at $p$ and passes thru $q$.
Show that $\langle x-p,q-p \rangle\le 0$ for any $x\in \sigma^{-1}\{0\}$.
Conclude that $\sigma^{-1}\{0\}$ is a convex closed set.
Finally use the definition of submetry to show that  $\sigma^{-1}\{0\}$ has no interior points. 

\parbf{\ref{ex:S^3/S^1}};
\ref{SHORT.ex:S^3/S^1:pq}.
Our $\SSS^1$ is a commutative subgroup of $\SO(3)$.
Therefore it is a subgroup of a maximal torus in $\SO(3)$.
Show that the described torus action is induced by a maximal torus in $\SO(3)$.
Use that maximal tori in $\SO(3)$ are conjugate.

\parit{\ref{SHORT.ex:S^3/S^1:sphere}.}
Cut $\SSS^3$ into two solid tori the Clifford torus $\tfrac1{\sqrt2}\cdot \SSS^1\times \SSS^1$.
Observe that the quotient of each solid torus is a disc;
conclude that $\Sigma_{p,q}$ is a sphere.
The torus action on $\SSS^3$ induce the needed $\SSS^1$-cation on $\Sigma_{p,q}$.

\parit{\ref{SHORT.ex:S^3/S^1:a}+\ref{SHORT.ex:S^3/S^1:b}+\ref{SHORT.ex:S^3/S^1:c}.} Straightforward calculations.

\parit{\ref{SHORT.ex:S^3/S^1:cc}.}
Consider the map $\Sigma_{p,q}\to\Sigma_{1,1}$ that sends poles to poles,
preserve the distance to the poles and respects the $\SSS^1$ action.

\parbf{\ref{ex:number(m-1)}};
\ref{SHORT.ex:number(m-1):2}.
Suppose $\#_{m-1}(\Gamma)\ge 3$;
that is $\spc{A}=\EE^m/\Gamma$ has at least 3 boundary components.
Follow Case~3 in the proof \ref{thm:hsiang-kleiner} to glue a train-space from copies of $\spc{A}$ using two of these components.
Show that the obtained space splits and arrive at a contradiction.

(Alternatively, apply a similar construction to all components of the boundary.
Show that the obtained space has {}\emph{exponential volume growth};
that is, there is $a>1$ such that $\vol \oBall(p,r)>a^r$ for all large~$r$.
Arrive at a contradiction with the Bishop--Gromov inequality.)

\parit{\ref{SHORT.ex:number(m-1):1}.}
Apply the doubling theorem as in Case~2 in the proof \ref{thm:hsiang-kleiner}.

\parbf{\ref{ex:S1actsS3}.}
Show that the quotient space $\Delta=\spc{A}/\mathbb{S}^1$ is an $\Alex1$ disc and $\gamma$ projects isometrically to $\partial\Delta$.
It remains to show that the perimeter of $\Delta$ cannot exceed $2\cdot\pi$.
The latter follows from \cite[3.3.5]{petrunin:survey};
it states that if $\Delta$ as an $m$-dimensional $\Alex1$ space, then $\vol_{m-1}\partial \Delta\le \vol_{m-1}\partial \mathbb{S}^{m-1}$.

\parbf{\ref{ex:surf-S2}.}
We can assume that the origin lies in the interior of the convex body.
Consider the central projection from its surface, say $\Sigma$, to the sphere $\SSS^2$ centered at the origin.
Show that this projection $\Sigma\to \SSS^2$ is a homeomorphism.

\parbf{\ref{ex:vertex-essential-vertex}.}
Follow the argument in \ref{clm:total-angle}.
Show that the inequality is strict if and only if $F$ has opposite points.


\parbf{\ref{ex:geodesic-vertex}.}
Suppose a geodesic $\gamma$ passes thru a vertex $v$.
Denote by $\alpha$ and $\beta$ the angles that $\gamma$ cuts at $v$.
Since $v$ is essential, $\alpha+\beta<2\cdot\pi$.
Therefore $\alpha<\pi$ or $\beta<\pi$.
Arrive at a contradiction by showing that $\gamma$ is not length-minimizing.

\parbf{\ref{pr:tetrahedron}}; \ref{SHORT.pr:tetrahedron:=}.
Cut the surface of $T$ along three edges coming from one vertex $v_1$ and unfold the obtained surface onto the plane.
Show that this way we get a triangle, the three vertices correspond to $v_1$ and the midpoints of sides correspond to the remaining three vertices.
Make a conclusion.

\parit{\ref{SHORT.pr:tetrahedron:perp}}.
Suppose that $0,v_1,v_2,v_3\in\RR^3$ are the vertices of $T$.
From \ref{SHORT.pr:tetrahedron:=}, we have that 
\[|v_1|=|v_2-v_3|,\quad |v_2|=|v_3-v_1|,\quad|v_3|=|v_1-v_1|.\]
Use it to show that $\langle v_1,v_2+v_3-v_1\rangle=0$.
Make a conclusion.

\parbf{\ref{ex:poly-CBB}.}
We need to show that if a polyhedral surface is $\Alex0$, then the total angle $\theta$ at every vertex $p$ it at most $2\cdot\pi$.

Assume that $\theta>2\cdot\pi$,
let $\phi=\max\{\,\pi,\tfrac13\cdot\theta\,\}$.
Note that we can choose three points $x_1$, $x_2$, and $x_3$ close to $p$ such that 
$\mangle \hinge p{x_i}{x_j}=\phi$ for $i\ne j$.
Since the points $x_i$ are close to $p$, we have $\mangle \hinge p{x_i}{x_j}=\angk p{x_i}{x_j}$.
The latter contradicts $\EE^2$-comparison. 

\parbf{\ref{ex:surface-covergence}.}
We will use that the closest-point projection from the Euclidean space to a convex body is \index{short map}\emph{short};
that is, distance-nonexpanding \cite[13.3]{petrunin-zamora}.

Assume $K_\infty$ is nondegenerate.
Without loss of generality, we may assume that 
\[\cBall(0,r)\subset K_\infty\subset\cBall(0,1)\]
for some $r>0$.
Note that there is a sequence $\eps_n\to 0$ such that 
\[ K_n\subset(1+\eps_n)\cdot K_\infty
\quad\text{and}\quad
K_\infty\subset(1+\eps_n)\cdot K_n\]
for each large $n$.

Given $x\in K_n$, denote by $g_n(x)$ the closest-point projection of $(1+\eps_n)\cdot x$ to $K_\infty$.
Similarly, given $x\in K_\infty$, denote by $h_n(x)$ the closest point projection of $(1+\eps_n)\cdot x$ to $K_n$.
Note that 
\begin{align*}
\dist{g_n(x)}{g_n(y)}{}&\le (1+\eps_n)\cdot\dist{x}{y}{}
\intertext{and}
\dist{h_n(x)}{h_n(y)}{}&\le (1+\eps_n)\cdot\dist{x}{y}{}.
\end{align*}

Denote by $\Sigma_\infty$ and $\Sigma_n$ the surface of $K_\infty$ and $K_n$ respectively. 
The above inequalities imply 
\begin{align*}
\dist{g_n(x)}{g_n(y)}{\Sigma_\infty}&\le (1+\eps_n)\cdot\dist{x}{y}{\Sigma_n}
\intertext{for any $x,y\in \Sigma_n$, and}
\dist{h_n(x)}{h_n(y)}{\Sigma_n}&\le (1+\eps_n)\cdot\dist{x}{y}{\Sigma_\infty}.
\end{align*}
for any $x,y\in \Sigma_\infty$.

Note that the maps $g_n$ and $h_n$ are onto.
Apply \ref{ex:GH-po} to finish the proof.

Alternatively, since the closest-point projection cannot increase the length of curve, we also get
\begin{align*}
\dist{x}{h_n\circ g_n(x)}{\Sigma_\infty}&\le 10\cdot \eps_n
\\
\dist{y}{g_n\circ h_n(y)}{\Sigma_n}&\le 10\cdot \eps_n.
\end{align*}
for all large $n$.
Therefore, $g_n$ is a $\delta_n$-isometry $\Sigma_n\to\Sigma_\infty$ for a sequence $\delta_n\to 0$.

\parit{Comments.}
More generally, if a sequence of $m$-dimensional $\Alex\kappa$ spaces $\spc{A}_1,\spc{A}_2,\dots$ converges to $\spc{A}_\infty$ and $\dim \spc{A}_\infty=m<\infty$,
then $\partial \spc{A}_n$ equipped with the induced length metrics converge to  $\partial \spc{A}_\infty$.
This statement is a partial case of the theorem about extremal subsets proved by the second author \cite[1.2]{petrunin1997}.

\parbf{\ref{ex:liberman+milka}}; \ref{SHORT.ex:liberman+milka:liberman}.
By \ref{ex:liberman}, the function $f_p\:t\mapsto \distfun_p\circ\gamma(t)$ is semiconcave for any $p\in K$.
In particular, one-sided derivatives $f_p^+(t)$ are defined for every $t$.

Given $x=\gamma(t)$, choose three points $p_1,p_2,p_3\in K$ in general position;
that is, the four points $x,p_1,p_2,p_3$ do not lie in one plane.
Observe that the distance functions $\distfun_{p_i}$ give smooth coordinates in a neighborhood of $x$.
From above the functions $f_{p_i}$ have one-sided derivatives at $t$.
Since the coordinates are smooth we get that $\gamma^+(t)$ is defined as well.

\parit{\ref{SHORT.ex:liberman+milka:milka}.}
If the plane $py_1y_2$ supports $K$, then 
$\mangle\hinge p{y_1}{y_2}_{\EE^3}=\mangle\hinge p{x_1}{x_2}_S$.
In this case, the statement follows from \ref{prop:conv-surf-CBB(0)}.

Now suppose that the line segment $[y_1y_2]_{\EE^3}$ intersects $K$.
Choose a geodesic $[y_1y_2]_W$;
note that it contains a point of $K$, say $z$.
Now consider a one-parameter family of points 
$y_i(t)\df \gamma(t)+\gamma^+(t)\z\cdot (1-t)\z\cdot \dist{p}{x_i}{S}$.
Note that this family is not continuous.

Show that for any point $p\in K$, the function $t\mapsto \dist{p}{\gamma_i(t)}{\EE^3}$ is nonincreasing.
Conclude that the function $t\mapsto \dist{p}{\gamma_i(t)}{W}$ is nonincreasing for any $p\in S$.
Therefore, 
\begin{align*}
\dist{y_1}{y_2}{W}
&=\dist{y_1(0)}{y_2(0)}{W}=
\\
&=\dist{y_1(0)}{z}{W}+\dist{y_2(0)}{z}{W}\ge
\\
&\ge\dist{y_1(1)}{z}{W}+\dist{y_2(1)}{z}{W}\ge 
\\
&\ge\dist{x_1}{x_2}{S}.
\end{align*}
The last inequality follows since the closest point projection $W\to S$ is short.

It remains to consider the case when the plane $py_1y_2$ does not support $K$,
and $[y_1y_2]_{\EE^3}$ does not intersect $K$.
In this case the plane $py_1y_2$ intersects $K$ along a convex figure $F$ that lies in the solid triangle 
$py_1y_2$ and contains its vertex $p$.

Choose points $y_1'\in [py_1]_{\EE^3}$ and $y_2'\in [py_2]_{\EE^3}$ such that $[y_1'y_2']$ touches $F$.
Denote by $x_1'\in [px_1]_{S}$ and $x_2'\in [px_2]_{S}$ the corresponding points;
that is, $\dist{p}{y_1'}{\EE^3}=\dist{p}{x_1'}S$ and $\dist{p}{y_2'}{\EE^3}=\dist{p}{x_2'}S$.
From the above, we have that $\dist{y_1'}{y_2'}{\EE^3}\ge\dist{x_1'}{x_2'}S$;
in other words, 
\[\angk p{y_1'}{y_2'}\ge \angk p{x_1'}{x_2'};\]
here we think that $[p{y_1'}{y_2'}]$ is a triangle in $\EE^3$, but $[p{x_1'}{x_2'}]$ is a triangle in $S$.
Note that 
\[\angk p{y_1'}{y_2'}=\angk p{y_1}{y_2}
\quad\text{and}\quad
\angk p{x_1}{x_2}\le \angk p{x_1'}{x_2'};
\]
the second inequality follows from \ref{ex:noncreasing}.
Hence the remaining case follows.

\parit{Comments.}
Part~\ref{SHORT.ex:liberman+milka:liberman} is the so-called Liberman lemma --- the main tools in studying geodesics on convex surfaces.
It was originally proved by Joseph Liberman \cite{liberman}; the proof of \ref{ex:liberman} is its generalization. 

Part~\ref{SHORT.ex:liberman+milka:milka} is the result of Anatolii Milka \cite[Theorem 2]{milka1982}.



{
\documentclass[twoside]{book}

%\newcommand{\spell}[2]{#1} %spell
\newcommand{\spell}[2]{#2} %notes


\def\thetitle{A journey into Alexandrov geometry:\\
curvature bounded below}
\def\theauthors{Vitali Kapovitch and Anton Petrunin}

\usepackage{lectures}
\usepackage[colorlinks=true,
citecolor=black,
linkcolor=black,
anchorcolor=black,
filecolor=black,
menucolor=black,
urlcolor=black,
pdftitle={\thetitle},
pdfsubject={Geometry},
pdfauthor={\theauthors}
]{hyperref}
\makeindex

%\usepackage[x-1a]{pdfx}

%\overfullrule=100mm
\def\red{\color{red}}
\begin{document}

\spell{\pagestyle{empty}\renewcommand\includegraphics[2][{}]{}\def\emph{\textit}\renewcommand\footnote[1]{\ (#1)}\renewcommand\z{}\renewcommand\section[1]{SECTION. {#1} SECTION.}}{}

\frontmatter
\title{\thetitle}
\author{\theauthors}
\date{}
\maketitle
\thispagestyle{empty}

\mainmatter
\newpage
\tableofcontents

\chapter*{Preface}

As in our previous invitation \cite{alexander-kapovitch-petrunin-2019},
we try to demonstrate the beauty and power of Alexandrov geometry by reaching interesting applications and theorems with a minimum of preparation.
This time we do spaces with curvature bounded below in the sense of Alexandrov.

This subject is more technical, this time we jumped over proofs of couple of technical results,
namely existence part in generalized Picard's theorem (\ref{thm:glob-exist-grad-curv})
and Perelman's theorem about conic neighborhoods (\ref{thm:spherical-nbhd}).
The rest of our presentation is nearly rigorous.

\medskip 

In Lecture~\ref{chap:prelim}, we discuss necessary preliminaries and fix notations.

Lecture~\ref{chap:defs} introduces the main object of our study --- spaces with curvature bounded below in the sense of Alexandrov.

In Lecture~\ref{chap:globalization} we formulate and prove the globalization theorem --- local Alexandrov condition implies global.
To simplify the presentation we consider only compact case, but this case is leading.

In Lecture~\ref{chap:derivative} we do beginning of calculus --- tangent space and space of directions, differential, and gradient.

Lecture~\ref{chap:GF} introduces gradient flow --- this is the main technical tool in the theory.

Lecture~\ref{chap:splitting} proves the line splitting theorem.
It provides the first application of gradient flow.

In Lecture~\ref{chap:dim} we introduce and discuss dimension of Alexandrov spaces,
introduce volume,
and prove the Bishop--Gromov inequality.

Lecture~\ref{chap:lim} shows that lower curvature bound survives in the Gromov--Hausdorff limit and proves Gromov's selection theorem.
Further we do Perelman's construction of strictly concave functions and apply it with Gromov's selection theorem to prove the homotopy finiteness theorem.
This proof illustrates the main source of applications of Alexandrov geometry.

In Lecture~\ref{chap:bry} we introduce boundary of finite-dimensioanal Alexandrov space and prove the doubling theorem.

Lecture~\ref{chap:L/G} we show that quotient Alexandrov space by isometric group action is an Alexandrov space and give several applications of this statement.
These proofs illustrate another source of applications of Alexandrov geometry.

Lecture~\ref{chap:convex-body} brings us back to the original object of study of Alexandrov.
We show that surface of a convex body in Euclidean space is an Alexandrov space.
This is historically the first serious application of Alexandrov geometry.

Finally, Appendix~\ref{chap:embedding} sketches Alexandrov embedding theorem of convex polyhedra.
Historically, this theorem is the first remarkable result in Alexandrov geometry that dates back to 1941.
The proof is very well written by Alexandrov, but we decided to include its sketch here due to its beauty and importance.
This appendix was written by Nina Lebedeva and the second author for a book about .

Let us give a list of available texts on Alexandrov spaces with curvature bounded below: 
\begin{itemize}
\item The first introduction to Alexandrov geometry is given in the original paper of Yuriy Burago, Michael Gromov, and Grigory Perelman \cite{burago-gromov-perelman} 
and its extension \cite{perelman1991} written by Perelman.
\item A brief and reader-friendly introduction was written by Katsuhiro Shiohama \cite[Sections 1--8]{shiohama}.
\item Another reader-friendly introduction, written by Dmiti Burago, Yuriy
Burago, and Sergei Ivanov \cite[Chapter 10]{burago-burago-ivanov}.
\item Survey by Conrad Plaut \cite{plaut:survey}.
\item Survey by the second author \cite{petrunin:survey}.
\end{itemize}

\parbf{Acknowledgments.}
Our notes were shaped in a number of lectures given by the authors
at different occasions in Penn State, including the MASS program,
at the Summer School ``Algebra and Geometry'' in Yaroslavl,
at SPbSU,
and University of Toronto.
We want to thank these institution for hospitality and support.

We were partially supported by the following grants:
Vitali Kapovitch ---   NSERC Discovery grants;
Anton Petrunin --- 
NSF grant DMS-2005279. %??? check!!!




%%!TEX root = the-prelim.tex

\chapter{Preliminaries}\label{chap:prelim}

\section{Prerequisites}

We assume that the reader is familiar with the following topics in metric geometry:
\begin{itemize}
\item Compactness and proper metric spaces;
recall that a metric space is \index{proper space}\emph{proper} if all its closed balls (with finite radius) are compact.
\item Complete metric spaces and completion.
\item Curves, semicontinuity of length and rectifiability.
\item Hausdorff and Gromov--Hausdorff convergence.
These are discussed briefly in \ref{sec:Hausdorff convergence}--\ref{sec:Gromov--Hausdorff-metric}.
The definitions are there, but it would be hard to follow without prior experience.
\end{itemize}
These topics are treated in \cite{burago-burago-ivanov} and \cite{petrunin2023pure}.
Occasionally, we use the Baire category theorem and Rademacher's theorem, but these could be used as black boxes.

We use some topology. 
Most of the time, any introductory text in algebraic topology should be sufficient.
For some examples, we use more advanced results, but these could also be used as black boxes.

Since most of the applications come from Riemannian geometry, it is better to be familiar with the Toponogov comparison theorem and related topics.
The classical book by Jeff Cheeger and David Ebin \cite{cheeger-ebin} contains more than you will need.

\section{Notations}

The distance between two points $x$ and $y$ in a metric space $\spc{X}$ will be denoted by \index{$\dist{x}{y}{}=\dist{x}{y}{\spc{X}}$ (distance)}$\dist{x}{y}{}$ or $\dist{x}{y}{\spc{X}}$.\label{page:|x-y|X}
The latter notation is used if we need to emphasize 
that the distance is taken in the space~${\spc{X}}$.

Given radius $r\in[0,\infty]$ and center $x\in \spc{X}$, the sets
\begin{align*}
\oBall(x,r)&=\set{y\in \spc{X}}{\dist{x}{y}{}<r},
\\
\cBall[x,r]&=\set{y\in \spc{X}}{\dist{x}{y}{}\le r}
\end{align*}
are called, respectively, the \index{open ball}\emph{open} and  the \index{closed ball}\emph{closed  balls}.
The notations $\oBall(x,r)_{\spc{X}}$ and $\cBall[x,r]_{\spc{X}}$
might be used if we need to emphasize that these balls are taken in the metric space $\spc{X}$.

We will denote by \index{$\SSS^n$, $\EE^n$, $\HH^n$, and $\MM^n(\kappa)$}$\SSS^n$, $\EE^n$, and $\HH^n$ the $n$-dimensional sphere (with angle metric), 
Euclidean space, and Lobachevsky space respectively.
More generally, $\MM^n(\kappa)$ will denote the \index{model!space}\emph{model $n$-space} of curvature $\kappa$;
that is,
\begin{itemize}
\item if $\kappa>0$, then $\MM^n(\kappa)$ is the $n$-sphere of radius $\tfrac{1}{\sqrt{\kappa}}$, so $\SSS^n=\MM^n(1)$
\item $\MM^n(0)=\EE^n$,
\item if $\kappa<0$, then $\MM^n(\kappa)$ is the Lobachevsky $n$-space $\HH^n$ rescaled by factor $\tfrac{1}{\sqrt{-\kappa}}$;
in particular $\MM^n(-1)=\HH^n$.
\end{itemize}

\section{Length spaces}\label{sec:length}

Let $\spc{X}$ be a metric space.
If for any $\eps>0$ and any pair of points $x,y\in\spc{X}$, there is a path $\alpha$ connecting $x$ to $y$ such that
\[\length\alpha< \dist{x}{y}{}+\eps,\]
then $\spc{X}$ is called a \index{length space}\emph{length space} and the metric on $\spc{X}$ is called a \index{length metric}\emph{length metric}.\label{page:length metric}

\begin{thm}{Exercise}\label{ex:compact+connceted}
Let $\spc{X}$ be a complete length space.
Show that for any compact subset $K\subset\spc{X}$
there is a compact path-connected subset $K'\subset\spc{X}$ that contains $K$.  
\end{thm}

\parbf{Induced length metric.}
Directly from the definition, it follows that if $\alpha\:[0,1]\to\spc{X}$ is a path from $x$ to $y$ 
(that is, $\alpha(0)=x$ and $\alpha(1)=y$), then 
\[\length\alpha\ge \dist{x}{y}{}.\]
Set 
\[\yetdist{x}{y}{}=\inf\{\,\length\alpha\,\}\]
where the greatest lower bound is taken for all paths from $x$ to~$y$.
It is straightforward to check that $(x,y)\mapsto \yetdist{x}{y}{}$ is an \emph{$\infty$-metric};
that is, $(x,y)\mapsto \yetdist{x}{y}{}$ is a metric in the extended positive reals $[0,\infty]$. 
The metric $\yetdist{*}{*}{}$ is called the \index{induced length metric}\emph{induced length metric}.

\begin{thm}{Exercise}\label{ex:compact=>complete}
Suppose $(\spc{X},\dist{*}{*}{})$ is a complete metric space.
Show that $(\spc{X},\yetdist{*}{*}{})$ is complete;
that is, any Cauchy sequence of points in $(\spc{X},\yetdist{*}{*}{})$ converges in $(\spc{X},\yetdist{*}{*}{})$.
\end{thm}

Let $A$ be a subset of a metric space $\spc{X}$.
Given two points $x,y\in A$,
consider the value
\[\dist{x}{y}{A}=\inf_{\alpha}\{\,\length\alpha\,\},\]
where the greatest lower bound is taken for all paths $\alpha$ from $x$ to $y$ in~$A$.
In other words, $\dist{*}{*}{A}$ denotes the induced length metric on the subspace $A$.
(The notation $\dist{*}{*}{A}$ conflicts with the previously defined notation for distance $\dist{x}{y}{\spc{X}}$ in a metric space $\spc{X}$.
However, most of the time we will work with ambient length spaces where the meaning will be unambiguous.)

\section{Geodesics}

Let $\spc{X}$ be a metric space 
and $\II$\index{$\II$ (real interval)} a real interval. 
A distance-preserving map $\gamma\:\II\to \spc{X}$ is called a \index{geodesic}\emph{geodesic}%
\footnote{Others call it differently: \textit{shortest path}, \textit{minimizing geodesic}.
Also, note that the meaning of the term \textit{geodesic} is different from what is used in Riemannian geometry, altho they are closely related.}; 
in other words, $\gamma\:\II\z\to \spc{X}$ is a geodesic if 
\[\dist{\gamma(s)}{\gamma(t)}{}=|s-t|\]
for any pair $s,t\in \II$.

If $\gamma\:[a,b]\to \spc{X}$ is a geodesic such that $p=\gamma(a)$, $q=\gamma(b)$, then we say that $\gamma$ is a geodesic from $p$ to $q$.
In this case, the image of $\gamma$ is denoted by $[p q]$\index{$[pq]$ (geodesic)}, and, with abuse of notations, we also call it a \index{geodesic}\emph{geodesic}.
We may write $[p q]_{\spc{X}}$ 
to emphasize that the geodesic $[p q]$ is in the space  ${\spc{X}}$.

In general, a geodesic from $p$ to $q$ need not exist and if it exists, it need not  be unique;
for example, any meridian is a geodesic between poles on the sphere.
However, once we write $[p q]$ we assume that we have chosen such a geodesic.

A \index{geodesic!path}\emph{geodesic path} is a geodesic with constant-speed parameterization by the unit interval $[0,1]$.

A metric space is called \index{geodesic!space}\emph{geodesic} if any pair of its points can be joined by a geodesic.

Evidently, any geodesic space is a length space.

\begin{thm}{Exercise}\label{ex:compact-length}
Show that any proper length space is geodesic.
\end{thm}

\section{Menger's lemma}

\begin{thm}{Lemma}\label{lem:mid>geod}
Let $\spc{X}$ be a complete metric space.
Assume that for any pair of points $x,y\in \spc{X}$, 
there is a midpoint~$z$.
Then $\spc{X}$ is a geodesic space.

\end{thm}

This lemma is due to Karl Menger \cite[Section 6]{menger}.

%???+PIC!!!

\parit{Proof.}
Choose $x,y\in \spc{X}$;
set $\gamma(0)=x$, and $\gamma(1)=y$.

\begin{figure}[ht!]
\vskip-0mm
\centering
\includegraphics{mppics/pic-104}
\end{figure}

Let $\gamma(\tfrac12)$ be a midpoint between $\gamma(0)$ and $\gamma(1)$.
Further, let $\gamma(\frac14)$ 
and $\gamma(\frac34)$ be midpoints between the pairs $(\gamma(0),\gamma(\tfrac12))$ 
and $(\gamma(\tfrac12),\gamma(1))$ respectively.
Applying the above procedure recursively,
on the $n$-th step we define $\gamma(\tfrac{k}{2^n})$,
for every odd integer $k$ such that $0<\tfrac k{2^n}<1$, 
as a midpoint of the already defined
$\gamma(\tfrac{k-1}{2^n})$ and $\gamma(\tfrac{k+1}{2^n})$.

This way we define $\gamma(t)$ for all dyadic rationals $t$ in $[0,1]$.
Moreover, $\gamma$ has Lipschitz constant $\dist{x}{y}{}$.
Since $\spc{X}$ is complete, the map $\gamma$ can be extended continuously to $[0,1]$.
Moreover,
\[
\length\gamma\le \dist{x}{y}{}.
\]
Therefore $\gamma$ is a geodesic path from $x$ to $y$.
\qedsf

\begin{thm}{Exercise}\label{ex:menger}
Let $\spc{X}$ be a complete metric space.
Assume that for any pair of points $x,y\in \spc{X}$, 
there is an \index{almost midpoint}\emph{almost midpoint};
that is, given $\eps>0$, there is a point $z$ such that 
\[\dist{x}{z}{}<\tfrac12\cdot\dist{x}{y}{}+\eps 
\quad\text{and}\quad
\dist{y}{z}{}<\tfrac12\cdot\dist{x}{y}{}+\eps.\]
Show that $\spc{X}$ is a length space.
\end{thm}


\section{Triangles and model tangles}

\parbf{Triangles.}
Given a triple of distinct points $p,q,r$ in a metric space $\spc{X}$, a choice of geodesics $([q r], [r p], [p q])$ will be called a \index{triangle}\emph{triangle}; we will use the short notation 
$\trig p q r=\trig p q r_{\spc{X}}=([q r], [r p], [p q])$\index{$\trig p q r=\trig p q r_{\spc{X}}$ (triangle)}.

Given a triple $p,q,r\in \spc{X}$ there may be no triangle 
$\trig p q r$ simply because one of the pairs of these points cannot be joined by a geodesic.
Also, many different triangles with these vertices may exist, any of which can be denoted by $\trig p q r$.
If we write $\trig p q r$, it means that we have chosen such a triangle.


\parbf{Model triangles.}
Given three points $p,q,r$ in a metric space $\spc{X}$,
let us define its \index{model!triangle}\emph{model triangle} $\trig{\tilde p}{\tilde q}{\tilde r}$ 
(briefly, 
$\trig{\tilde p}{\tilde q}{\tilde r}=\modtrig(p q r)_{\EE^2}$%
\index{$\modtrig$ (model triangle)}) to be a triangle in the Euclidean plane $\EE^2$ such that
\begin{align*}\dist{\tilde p}{\tilde q}{\EE^2}&=\dist{p}{q}{\spc{X}},
&
\quad\dist{\tilde q}{\tilde r}{\EE^2}&=\dist{q}{r}{\spc{X}},
&
\quad\dist{\tilde r}{\tilde p}{\EE^2}&=\dist{r}{p}{\spc{X}}.
\end{align*}

In the same way, we can define the \index{hyperbolic model triangle}\emph{hyperbolic} and the \index{spherical model triangles}\emph{spherical model triangles} $\modtrig(p q r)_{\HH^2}$, $\modtrig(p q r)_{\SSS^2}$
in the Lobachevsky plane $\HH^2$ and the unit sphere~$\SSS^2$.
In the latter case, the model triangle is said to be defined if in addition
\[\dist{p}{q}{}+\dist{q}{r}{}+\dist{r}{p}{}< 2\cdot\pi.\]
In this case, the model triangle again exists and is unique up to an isometry of~$\SSS^2$.

\parbf{Model angles.}
If 
$\trig{\tilde p}{\tilde q}{\tilde r}=\modtrig(p q r)_{\EE^2}$ 
and $\dist{p}{q}{},\dist{p}{r}{}>0$, 
the angle measure of 
$\trig{\tilde p}{\tilde q}{\tilde r}$ at $\tilde p$ 
will be called the \index{model!angle}\emph{model angle} of the triple $p$, $q$, $r$ and will be denoted by
$\angk p q r_{\EE^2}$%
\index{$\angk{p}{q}{r}$ (model angle)}.\label{page:model-angle}

For example, if $\dist{p}{q}{}=\dist{q}{r}{}=\dist{r}{p}{}$, then $\angk p q r_{\EE^2}=\tfrac\pi3$ regardless of existence and relative position of geodesics $[pq]$ and $[pr]$.

The same way we define $\angk p q r_{\MM^2(\kappa)}$;
in particular, $\angk p q r_{\HH^2}$ and $\angk p q r_{\SSS^2}$.
We may use the notation $\angk p q r$ if it is evident which of the model spaces is meant.

\begin{thm}{Exercise}\label{ex:k-><mono}
Show that for any triple of point $p$, $q$, and $r$,
the function
\[\kappa\mapsto \angk p q r_{\MM^2(\kappa)}\]
is nondecreasing in its domain of definition.
\end{thm}


\section{Hinges and their angle measure}\label{sec:angles}

\parbf{Hinges.} Let $p,x,y\in \spc{X}$ be a triple of points such that $p$ is distinct from $x$ and~$y$.
A pair of geodesics $([p x],[p y])$ will be called  a \index{hinge}\emph{hinge} and will be denoted by 
$\hinge p x y=([p x],[p y])$\index{$\hinge p x y$ (hinge)}.

\parbf{Angles.}
The angle measure of a hinge $\hinge p x y$ is defined as the following limit
\[\mangle\hinge p x y=\lim_{\bar x,\bar y\to p} \angk p{\bar x}{\bar y},\]
where $\bar x\in\left]p x\right]$ and $\bar y\in\left]p y\right]$.

Note that if $\mangle\hinge p x y$ is defined, then
\[0\le \mangle\hinge p x y\le \pi.\]

\begin{thm}{Exercise}\label{ex:angkK}
Suppose that in the above definition, one uses spherical or hyperbolic model angles instead of Euclidean.
Show that it does not change the value $\mangle\hinge p x y$.
\end{thm}


\begin{thm}{Exercise}\label{ex:undefined-angle}
Give an example of a hinge $\hinge p x y$ in a metric space with an undefined angle measure $\mangle\hinge p x y$.
\end{thm}

\section{Triangle inequality for angles}

\begin{thm}{Proposition}\label{claim:angle-3angle-inq}
Let  $[px_1]$, $[px_2]$, and $[px_3]$ be three geodesics in a metric space.
Suppose all the angle measures $\alpha_{i j}=\mangle\hinge p {x_i}{x_j}$ are defined.
Then 
\[\alpha_{13}\le \alpha_{12}+\alpha_{23}.\]

\end{thm}



\parit{Proof.}
Since $\alpha_{13}\le\pi$, we can assume that $\alpha_{12}+\alpha_{23}< \pi$.
Denote by $\gamma_i$ the unit-speed parametrization of $[px_i]$ from $p$ to $x_i$.
Given any $\eps>0$, for all sufficiently small $t,\tau,s\in\RR_{\ge0}$ we have
\begin{align*}
\dist{\gamma_1(t)}{\gamma_3(\tau)}{}
&\le 
\dist{\gamma_1(t)}{\gamma_2(s)}{}+\dist{\gamma_2(s)}{\gamma_3(\tau)}{}<\\
&<
\sqrt{t^2+s^2-2\cdot t\cdot  s\cdot \cos(\alpha_{12}+\eps)} +
\\
&\quad+\sqrt{s^2+\tau^2-2\cdot s\cdot \tau\cdot \cos(\alpha_{23}+\eps)}\le
\end{align*}

\begin{wrapfigure}{o}{30 mm}
\vskip-6mm
\centering
\includegraphics{mppics/pic-615}
\vskip6mm
\end{wrapfigure}

Below we define 
$s(t,\tau)$ so that for 
$s=s(t,\tau)$, this chain of inequalities can be continued as follows:
\[\le
\sqrt{t^2+\tau^2-2\cdot t\cdot \tau\cdot \cos(\alpha_{12}+\alpha_{23}+2\cdot \eps)}.
\]

Thus for any $\eps>0$, 
\[\alpha_{13}\le \alpha_{12}+\alpha_{23}+2\cdot \eps.\]
Hence the result follows.

To define $s(t,\tau)$, consider three half-lines $\tilde \gamma_1$, $\tilde \gamma_2$, $\tilde \gamma_3$ on a Euclidean plane starting at one point, such that
$\mangle(\tilde \gamma_1,\tilde \gamma_2)\z=\alpha_{12}+\eps$,
$\mangle(\tilde \gamma_2,\tilde \gamma_3)\z=\alpha_{23}+\eps$,
and $\mangle(\tilde \gamma_1,\tilde \gamma_3)\z=\alpha_{12}\z+\alpha_{23}\z+2\cdot \eps$.
We parametrize each half-line by the distance from the starting point.
Given two positive numbers $t,\tau\in\RR_{\ge0}$, let $s=s(t,\tau)$ be 
the number such that 
$\tilde \gamma_2(s)\in[\tilde \gamma_1(t)\ \tilde \gamma_3(\tau)]$. 
Clearly, $s\le\max\{t,\tau\}$, so $t,\tau,s$ may be taken sufficiently small.
\qeds 

\begin{thm}{Exercise}\label{ex:adjacent-angles}
Prove that the sum of adjacent angles is at least $\pi$.

More precisely: suppose two hinges $\hinge pxz$ and $\hinge pyz$ are \index{adjacent hinges}\emph{adjacent};
that is, they share side $[pz]$, and the union of two sides $[px]$ and $[py]$ form a geodesic $[xy]$.
Show that
\[\mangle\hinge pxz+\mangle\hinge pyz\ge \pi\]
whenever  each angle on the left-hand side is defined.

Give an example showing that the inequality can be strict.
\end{thm}

\begin{thm}{Exercise}\label{ex:first-var}
Assume that the angle measure of $\hinge q p x$ is defined.
Let $\gamma$ be the unit speed parametrization of $[qx]$ from $q$ to $x$.
Show that
\[\dist{p}{\gamma(t)}{}
\le
\dist{q}{p}{}-t\cdot \cos(\mangle\hinge q p x)+o(t).\]

\end{thm}

\section{Hausdorff convergence}\label{sec:Hausdorff convergence}

\begin{thm}{Definition}\label{def:gen-Haus-conv}
Let $A_1,A_2,\dots$ be a sequence of closed sets in a metric space $\spc{X}$.
We say that the sequence $A_n$ \index{Hausdorff!limit}\emph{converges} to a closed set $A_\infty$ in the {}\emph{sense of Hausdorff} if, for any $x\in\spc{X}$, we have
$\distfun_{A_n}(x)\z\to \distfun_{A_\infty}(x)$ as $n\to\infty$.
\end{thm}

For example, suppose $\spc{X}$ is the Euclidean plane and $A_n$ is the circle with radius $n$ and center at the point $(0,n)$; it converges to the $x$-axis.

\begin{figure}[ht!]
\vskip-0mm
\centering
\includegraphics{mppics/pic-415}
\end{figure}

Further, consider the sequence of one-point sets $B_n=\{(n,0)\}$ in the Euclidean plane.
It converges to the empty set;
indeed, for any point $x$ we have $\distfun_{B_n}(x)\to\infty$ as $n\to \infty$ and $\distfun_{\emptyset}(x)= \infty$ for any~$x$.

The following exercise is an extension of the so-called Blaschke selection theorem to our version of Hausdorff convergence.

\begin{thm}{Exercise}\label{ex:generalized-selection}
Show that any sequence of closed sets in a proper metric space has a convergent subsequence in the sense of Hausdorff.
\end{thm}

\section{Hausdorff metric}

\begin{thm}{Definition}\label{def:hausdorff-convergence}
Let $A$ and $B$ be two non-empty compact subsets of a metric space $\spc{X}$.
Then the \index{Hausdorff!distance}\emph{Hausdorff distance} between $A$ and $B$ is defined as 
$$|A-B|_{\Haus\spc{X}}
\df
\sup_{x\in \spc{X}}\{\,|\distfun_A(x)-\distfun_B(x)|\,\}.
$$

\end{thm}

The following observation gives a useful reformulation of the definition:

\begin{thm}{Observation}\label{obs:Haus-nbhds}
Suppose $A$ and $B$ be two compact subsets of a metric space $\spc{X}$.
Then $|A-B|_{\Haus\spc{X}}< R$ if and only if and only if 
$B$ lies in an $R$-neighborhood of $A$, 
and 
$A$ lies in an $R$-neighborhood of~$B$.
\end{thm}

The following exercise implies that Hausdorff convergence of compact subsets is the convergence in Hausdorff metric.

\begin{thm}{Exercise}\label{ex:Haus-conv}
Let $A_1,A_2,\dots,$ and $A_\infty$ be compact non-empty sets in a metric space $\spc{X}$.
Show that $\dist{A_n}{A_\infty}{\Haus\spc{X}}\to 0$ as $n\to\infty$
if and only if $A_n\to A_\infty$ in the sense of Hausdorff.
\end{thm}

\section{Gromov--Hausdorff convergence}\label{sec:Gromov--Hausdorff}

Let $\spc{X}_1,\spc{X}_2,\dots,$ and $\spc{X}_\infty$ be a sequence of complete metric spaces.
Suppose that there is a metric on the disjoint union 
\[\bm{X}=\bigsqcup_{n\in \NN\cup\{\infty\}} \spc{X}_n\] 
that satisfies the following property:

\begin{thm}{Property}\label{propery:GH}
The restriction of the metric on each $\spc{X}_n$ and $\spc{X}_\infty$ coincides with its original metric, 
and $\spc{X}_n\to \spc{X}_\infty$ as subsets in $\bm{X}$ in the sense of Hausdorff.
\end{thm}

In this case we say that the metric on $\bm{X}$ \textit{defines} a \index{Gromov--Hausdorff limit}\emph{convergence} $\spc{X}_n\z\to \spc{X}_\infty$ in the {}\emph{sense of Gromov--Hausdorff}.
The metric on  $\bigsqcup \spc{X}_n$ makes it possible to talk about limits of sequences $x_n\in \spc{X}_n$ as $n\to\infty$, as well as weak limits of a sequence of Borel measures $\mu_n$ on $\spc{X}_n$ and so on.

The limit space is not uniquely defined by the sequence.
For example, if each space $\spc{X}_n$ in the sequence is isometric to the half-line, then its limit might be isometric to the half-line or the whole line.
The first convergence is evident and the second could be guessed from the diagram.

\begin{figure}[ht!]
\vskip-0mm
\centering
\includegraphics{mppics/pic-500}
\end{figure}

Note that any sequence of spaces has an empty space as its limit in some  Gromov--Hausdorff convergence.
Exercise \ref{ex:compact-GH} states that if the limit is non-empty and compact, then it is unique up to isometry. 

\begin{thm}{Exercise}\label{ex:geod-closed}
Let $\spc{X}_1,\spc{X}_2,\dots$ be a sequence of geodesic metric spaces.
Suppose $\spc{X}_n\to \spc{X}_\infty$ is a convergence in the sense of Gromov--Hausdorff.
Assume $\spc{X}_\infty$ is proper, show that it is geodesic.
\end{thm}

\parbf{Pointed convergence.}
Often the isometry class of the limit can be fixed by marking a point $p_n$ in each space $\spc{X}_n$.
We say that $(\spc{X}_n,p_n)$ converges to $(\spc{X}_\infty,p_\infty)$ if there is a metric on $\bm{X}$ as in \ref{propery:GH} such that $p_n\to p_\infty$.
This is called \index{pointed convergence}\emph{pointed Gromov--Hausdorff convergence}.
For example, the sequence $(\spc{X}_n,p_n)=(\RR_{\ge0},0)$ converges to $(\RR_{\ge0},0)$, while $(\spc{X}_n,p_n)=(\RR_{\ge0},n)$ converges to $(\RR,0)$ as $n\to \infty$.

\section{Gromov--Hausdorff metric}\label{sec:Gromov--Hausdorff-metric}

In this section we cook up a metric space out of all compact non-empty metric spaces
that defines Gromov--Hausdorff convergence.
We want to define the metric on the set of \textit{isometry classes} of compact metric spaces.
Further, the term \textit{metric space} might also stand for its \textit{isometry class}.

The obtained metric is called the Gromov--Hausdorff metric;
the corresponding metric space will be denoted by $\GH$.
This distance is defined as the maximal metric such that \textit{the distance between subspaces in a metric space is not greater than the Hausdorff distance between them}.
Here is a formal definition.

\begin{thm}{Definition}\label{def:GH}
The \index{Gromov--Hausdorff distance}\emph{Gromov--Hausdorff distance} $|\spc{X}-\spc{Y}|_{\GH}$ between compact metric spaces $\spc{X}$ and $\spc{Y}$
is defined by the following
relation.
 
Given  $r > 0$, we have $|\spc{X}-\spc{Y}|_{\GH} < r$ if and only if there exists a metric
space $\spc{W}$ and subspaces $\spc{X}'$ and $\spc{Y}'$ in $\spc{W}$ that are isometric to $\spc{X}$ and $\spc{Y}$,
respectively, such that $|\spc{X}'-\spc{Y}'|_{\Haus\spc{W}} < r$. 
(Here $|\spc{X}'-\spc{Y}'|_{\Haus\spc{W}}$ denotes the Hausdorff distance between sets $\spc{X}'$ and $\spc{Y}'$ in $\spc{W}$.)
\end{thm}

For the proof of the following statement we refer to \cite{burago-burago-ivanov} and \cite{petrunin2023pure}.

\begin{thm}{Proposition}\label{prop:complete}
$\GH$ is a complete metric space.
\end{thm}

Note that this means in particular that if $X,Y$ are compact and $|\spc{X}-\spc{Y}|_{\GH}=0$ then $X$ and $Y$ are isometric.

Gromov--Hausdorff convergence of compact spaces has particularly nice properties.
From the technical point of view, they follow from the next statement, which we formulate as an exercise.

\begin{thm}{Exercise}\label{ex:non-contracting-map}
Let $f$ be a distance noncontracting map from 
a compact metric space $\spc{K}$ to itself.
Show that $f$ is an isometry; that is, it is a distance-preserving bijection.
\end{thm}

For two metric spaces $\spc{X}$ and $\spc{Y}$,
we write $\spc{X}\le \spc{Y}+\eps$ if
there is a map $f\:\spc{X}\to \spc{Y}$ such that 
\[\dist{x}{x'}{\spc{X}}\le \dist{f(x)}{f(x')}{\spc{Y}}+\eps\]
for any $x,x'\in \spc{X}$.

\begin{thm}{Exercise}\label{ex:GH-po}
Let $\spc{X}_1,\spc{X}_2,\dots,$ and $\spc{X}_\infty$ are compact metric spaces.
Show that there is a Gromov--Hausdorff convergence $\spc{X}_n\to\spc{X}_\infty$ if and only if for some sequence $\eps_n\to 0$,
we have 
\[\spc{X}_\infty\le \spc{X}_n+\eps_n\quad\text{and}\quad \spc{X}_n\le \spc{X}_\infty+\eps_n.\]
\end{thm}

\begin{thm}{Exercise}\label{ex:compact-GH}
Let $\spc{X}_1,\spc{X}_2,\dots$ be a sequence of metric spaces.
Suppose $\spc{X}_\infty$ and $\spc{X}_\infty'$ are non-empty limit spaces for some Gromov--Hausdorff convergences of $\spc{X}_n$.
Assume $\spc{X}_\infty$ is compact, show that it is isometric to~$\spc{X}_\infty'$.
\end{thm}

\section{Almost isometries}

\begin{thm}{Definition}
Let $\spc{X}$ and $\spc{Y}$ be metric spaces.
A map $f\:\spc{X}\to\spc{Y}$
is called an \index{isometry!$\eps$-isometry}\emph{$\eps$-isometry}
if the following two conditions hold:

\begin{subthm}{}
$f(\spc{X})$ is an \index{$\eps$-net}\emph{$\eps$-net} in $\spc{Y}$; that is, for any $y\in \spc{Y}$ there is $x\in \spc{X}$ such that $\dist{f(x)}{y}{\spc{Y}}<\eps$.
\end{subthm}

\begin{subthm}{}
$\bigl|\dist{f(x)}{f(x')}{\spc{Y}}-\dist{x}{x'}{\spc{X}}\bigr|\le \eps$ for any $x,x'\in\spc{X}$.
\end{subthm}

\end{thm}

When dealing with Gromov--Hausdorff convergence the following lemma is often useful as it allows to bypass constructing explicit metrics on the disjoint unions of $\spc{X}_1,\spc{X}_2,\dots$, and $\spc{X}_\infty$

\begin{thm}{Lemma}\label{lem:almost-isom}
Let $\spc{X}_1,\spc{X}_2,\dots$, and $\spc{X}_\infty$ be complete metric spaces,
and let $\eps_n\to\0+$ as $n\to\infty$.
Suppose that either 
\begin{subthm}{lem:almost-isom-a}
for each $n$ there is an $\eps_n$-isometry $f_n\:\spc{X}_n\to\spc{X}_\infty$, or
\end{subthm}
\begin{subthm}{lem:almost-isom-b}
for each $n$ there is an $\eps_n$-isometry $h_n\:\spc{X}_\infty\to\spc{X}_n$.
\end{subthm}
Then there is a Gromov--Hausdorff convergence $\spc{X}_n\to \spc{X}_\infty$.

Furthermore, a partial converse also holds.

\begin{subthm}{lem:almost-isom-c}
Suppose we have a Gromov--Hausdorff convergence $\spc{X}_n\to \spc{X}_\infty$ and $\spc{X}_\infty$ is compact. Then there exist $\eps_n\to\0+$ as $n\to\infty$ and  $\eps_n$-isometris $f_n\:\spc{X}_n\to\spc{X}_\infty$ (and $h_n\:\spc{X}_\infty\to\spc{X}_n$)
such that $x_n\in \spc{X}_n$ converges to $x_\infty \in  \spc{X}_\infty$ with respect to the  convergence $\spc{X}_n\z\to \spc{X}_\infty$ if and only if $f_n(x_n)\to x_\infty$ (respectively, $\dist{h_n(x_\infty) }{x_n}{\spc{X}_n}\to 0$) as $n\to\infty$.
\end{subthm}
\end{thm}


\parit{Proof.}
To prove part \ref{SHORT.lem:almost-isom-a} let us construct a common space $\bm{X}$ for the spaces $\spc{X}_1,\spc{X}_2,\dots$, and $\spc{X}_\infty$
by taking the metric $\rho$ on the disjoint union $\spc{X}_\infty\sqcup\spc{X}_1\sqcup\spc{X}_2\sqcup\dots$ that is defined the following way:
\begin{align*}
\dist{x_n}{y_n}{\bm{X}}&=\dist{x_n}{y_n}{\spc{X}_n},
\\
\dist{x_\infty}{y_\infty}{\bm{X}}&=\dist{x_\infty}{y_\infty}{\spc{X}_\infty},
\\
\dist{x_n}{x_\infty}{\bm{X}}&=\inf\set{\dist{x_n}{y_n}{\spc{X}_n}+\eps_n+\dist{x_\infty}{f(y_n)}{\spc{X}_\infty}}{{y_n}\in \spc{X}_n},
\\
\dist{x_n}{x_m}{\bm{X}}&=\inf\set{\dist{x_n}{y_\infty}{\bm{X}}+\dist{x_m}{y_\infty}{\bm{X}}}{y_\infty\in\spc{X}_\infty},
\end{align*}
where we assume that $x_\infty,y_\infty\in \spc{X}_\infty$, and $x_n,y_n\in \spc{X}_n$ for each $n$. 
It remains to observe that this indeed defines a metric on $\bm{X}$, and $\spc{X}_n\to \spc{X}_\infty$ in the sense of Hausdorff.

The proof of the second part is analogous; one only needs to change one line in the definition of the metric to the following:
\[\dist{x_n}{x_\infty}{\bm{X}}=\inf\set{\dist{x_n}{h(y_\infty)}{\spc{X}_n}+\eps_n+\dist{x_\infty}{y_\infty}{\spc{X}_\infty}}{{y_\infty}\in \spc{X}_\infty}.\]

We leave part \ref{SHORT.lem:almost-isom-c} as an exercise.
\qedsf

Lemma~\ref{lem:almost-isom} has a natural analogue for pointed convergence.
For simplicity we only state part \ref{SHORT.lem:almost-isom-a} of the lemma.
Parts \ref{SHORT.lem:almost-isom-b} and \ref{SHORT.lem:almost-isom-c} can be rephrased similarly.



\begin{thm}{Lemma}\label{lem:almost-isom-pointed}
Let $(\spc{X}_1, p_1),(\spc{X}_2,p_2) ,\dots$, let $(\spc{X}_\infty, p_\infty)$ be pointed metric spaces, and let $\eps(n,R)\to\0+$ as $n\to\infty$ for any fixed $R>0$.
Suppose that for each $n$ there is a map $f_n\:\spc{X}_n\to\spc{X}_\infty$ such that


\begin{subthm}{}
$f_n(p_n)\to p_\infty$
\end{subthm}

\begin{subthm}{}
$\bigl|\dist{f_n(x)}{f_n(x')}{\spc{X}_\infty }-\dist{x}{x'}{\spc{X}_n}\bigr|\le \eps(n,R)$ for any $x,x'\z\in \oBall(p_n,R)$.
\end{subthm}

\begin{subthm}{}
For any $x \in \oBall(p_\infty,R)$ there is $x_n\in \oBall(p_n,R)$ such that $\dist{x}{f_n(x_n)}{}\le \eps(n,R)$
\end{subthm}

Then there is a pointed  Gromov--Hausdorff convergence $(\spc{X}_n,p_n)\z\to (\spc{X}_\infty,p_\infty)$.
\end{thm}

The proofs of \ref{lem:almost-isom-pointed} and \ref{lem:almost-isom} are analogous;
we leave the former to the reader.



\section{Comments}

In principle, our definition of Gromov--Hausdorff distance works for complete metric spaces that are not necessarily compact.
However, according to the following exercise, it only defines a \emph{semimetric}; that is, zero Gromov--Hausdorff distance does not imply that the spaces are isometric.
For that reason it is not in use.

\begin{thm}{Exercise}\label{ex:GH-noncompact}
Construct two nonisometric proper (noncompact) metric spaces with vanishing Gromov--Hausdorff distance.
\end{thm}


%%%%%%%%%%%%%%%%%%%%%%%%%%%%
%\chapter{Definitions}

The first synthetic description of curvature is due to Abraham Wald \cite{wald} published in 1936;
it was his student work, written under the supervision of Karl Menger. 
This publication was not noticed for about 50 years \cite{berestovskii}.
In 1941, similar definitions were rediscovered by Alexandr Alexandrov \cite{alexandrov:def}.



\section{Wald's approach}

Abraham Wald noticed that given a \textit{typical} metric on the quadruple of points $\spc{X}\z=\{x_1,x_2,x_3,x_4\}$ there is a closed interval,
say 
\[[\kappa_{\min}(x_1,x_2,x_3,x_4),\kappa_{\max}(x_1,x_2,x_3,x_4)]\subset \RR\]
such that there is a \textit{model configuration} in $\MM^3(\kappa)$;
that is, $\tilde x_1$, $\tilde x_2$, $\tilde x_3$, $\tilde x_4\in\MM^3(\kappa)$ such that
\[\dist{\tilde x_i}{\tilde x_j}{\MM^3(\kappa)}=\dist{x_i}{x_j}{\spc{X}}\]
for all $i$ and $j$.


\begin{wrapfigure}{r}{33mm}
\vskip-2mm
\centering
\includegraphics{mppics/pic-710}
\end{wrapfigure}

In $\MM^3(\kappa_{\min})$ and $\MM^3(\kappa_{\max})$, the points $\tilde x_1,\tilde x_2,\tilde x_3,\tilde x_4$ form degenerate tetrahedrons shown on the diagram (for $\kappa_{\min}$ it is a convex quadrangle and for $\kappa_{\max}$ --- a triangle with a point inside).
In the interior of the interval, the tetrahedron is nondegenerate.

Moreover, one can use $[-\infty,\infty)$ instead of $\RR$ 
and let
\[\kappa_{\min}(x_1,x_2,x_3,x_4)=-\infty\]
if there is \textit{almost} model quadruple in
$\MM^3(\kappa)$ for $\kappa\to -\infty$;
that is, for any $\eps>0$ there is a quadruple
$\tilde x_1,\tilde x_2,\tilde x_3,\tilde x_4\in\MM^3(\kappa)$
such that $\kappa\le -\tfrac1\eps$, and
\[\dist{\tilde x_i}{\tilde x_j}{\MM^3(\kappa)}\lege\dist{x_i}{x_j}{\spc{X}}\pm\eps\]
for all $i$ and $j$.
In this case the interval 
\[[\kappa_{\min}(x_1,x_2,x_3,x_4),\kappa_{\max}(x_1,x_2,x_3,x_4)]\subset [-\infty,\infty)\]
is defined for \textit{any} quadruple.

\begin{thm}{Exercise}
Let $x_1,x_2,x_3,x_4$ be a quadruple in a metric space such that $\kappa_{\min}(x_1,x_2,x_3,x_4)=-\infty$.
Show that two maximal numbers from the following three are equal to each other.
\begin{align*}
a&=\dist{x_1}{x_2}{}+\dist{x_3}{x_4}{},
\\
b&=\dist{x_1}{x_3}{}+\dist{x_2}{x_4}{},
\\
c&=\dist{x_1}{x_4}{}+\dist{x_2}{x_3}{}.
\end{align*}


\end{thm}


\begin{thm}{Exercise}
Suppose that $x_1,x_2,x_3,x_4$ in a metric space
such that
\begin{align*}
\dist{x_1}{x_2}{}=\dist{x_1}{x_3}{}=\dist{x_1}{x_4}{}&=1,
\\
\dist{x_2}{x_3}{}=\dist{x_3}{x_4}{}=\dist{x_4}{x_1}{}&=2.
\end{align*}
Show that 
\[\kappa_{\min}(x_1,x_2,x_3,x_4)=\kappa_{\max}(x_1,x_2,x_3,x_4)=-\infty.\]
\end{thm}

\begin{thm}{Exercise}
Let $x_1,x_2,x_3,x_4$ be a quadruple in $\EE^2$.
Suppose that $x_3$ lie on the line thru $x_1$ and $x_2$,
but $x_4$ does not.
Show that 
\[\kappa_{\min}(x_1,x_2,x_3,x_4)=\kappa_{\max}(x_1,x_2,x_3,x_4)=0.\]
\end{thm}

\begin{thm}{Wald-style definition}
Let $\kappa\in \RR$.
A metric space $\spc{X}$ has curvature $\ge\kappa$ (or $\le\kappa$) 
if for any quadruple $x_1,x_2,x_3,x_4\in \spc{X}$ we have 
$\kappa_{\max}(x_1,x_2,x_3,x_4)\ge \kappa$ (or $\kappa_{\min}(x_1,x_2,x_3,x_4)\le \kappa$ respectively). 
\end{thm}

This definition is given for its historical value.
It will not be used further in the sequel.
We will use another definition that is very close, but not equivalent.

\section{Substance}\label{sec:manifesto}

Consider the space $\mathcal{M}_4$ of all isometry classes of 4-point metric spaces.
Each element in $\mathcal{M}_4$ can be described by 6 numbers 
 --- the distances between all 6 pairs of its points, say $\ell_{i,j}$ for $1\le i< j\le 4$ modulo permutations of the index set $(1,2,3,4)$.
These 6 numbers are subject to 12 triangle inequalities; that is,
\[\ell_{i,j}+\ell_{j,k}\ge \ell_{i,k}\]
holds for all $i$, $j$ and $k$, where we assume that $\ell_{j,i}=\ell_{i,j}$, and $\ell_{i,i}=0$.

{

\begin{wrapfigure}{o}{33mm}
\vskip-3mm
\centering
\includegraphics{mppics/pic-700}
\end{wrapfigure}

The space $\mathcal{M}_4$ comes with topology.
It can be defined as a quotient topology of the cone in $\RR^6$ by permutations of the 4 points of the space.

Consider the subset $\mathcal{E}_4\subset \mathcal{M}_4$ of all isometry classes of 4-point metric spaces that admit isometric embeddings into Euclidean space.

}

\begin{thm}{Claim}\label{clm:two-components-of-M4}
The complement $\mathcal{M}_4\setminus \mathcal{E}_4$ has two connected components.
\end{thm}

\begin{thm}{Exercise}
Spend 10 minutes trying to prove the claim.
\end{thm}


The definition of Alexandrov spaces is based on the claim above.
Let us denote one of the components by $\mathcal{P}_4$ and the other by~$\mathcal{N}_4$.
Here $\mathcal{P}$ and $\mathcal{N}$ stand for {}\emph{positive} 
and {}\emph{negative curvature} because spheres have no quadruples of type $\mathcal{N}_4$ and 
hyperbolic space
has no quadruples of type~$\mathcal{P}_4$.

A metric space that has no quadruples of points of type $\mathcal{P}_4$ or $\mathcal{N}_4$
respectively 
is called an Alexandrov space with non-positive or non-negative curvature (briefly.

\begin{wrapfigure}{r}{33mm}
\vskip-0mm
\centering
\includegraphics{mppics/pic-710}
\end{wrapfigure}

Let us describe the subdivision into  $\mathcal{P}_4$, $\mathcal{E}_4$, and $\mathcal{N}_4$ intuitively.
Imagine that you move out of $\mathcal{E}_4$ --- your path is a one-parameter family of 4-point metric spaces.
The last thing you see in $\mathcal{E}_4$ is one of the two plane configurations shown on the diagram.
If you see the right configuration then you move into $\mathcal{N}_4$;
if it is the one on the left, then you move into $\mathcal{P}_4$.
More degenerate pictures can be avoided; for example, a triangle with a point on a side.
From such a configuration one may move in $\mathcal{N}_4$ and $\mathcal{P}_4$ (as well as come back to $\mathcal{E}_4$).

Here is an exercise, solving which would force you to rebuild a considerable part of Alexandrov geometry.
It is wise to spend some time thinking about this it before proceeding.

\begin{thm}{Advanced exercise}\label{ex:convex-set}
Assume $\spc{X}$ is a complete metric space with length metric (see Section~\ref{sec:length}), 
containing only quadruples of type~$\mathcal{E}_4$.
Show that $\spc{X}$ is isometric to a convex set in a Hilbert space.
\end{thm}

If in the definition above, we take $\MM^3(\kappa)$ instead of $\EE^3$.
Then we will arrive at Wald's definition of curvature bounded below and above by $\kappa$.
The parameter $\kappa$ has three interesting choices $-1$, $0$, and $1$;
the rest can be obtained from these three applying rescaling.

Again, the definition that we are going to use is not equivalent.


\section{Embedding theorem}

The following theorem is historically the first remarkable result in Alexandrov geometry.
The main part of the following theorem is due to Alexandr Alexandrov~\cite{alexandrov-1948}.
The last part is very difficult; it was proved by Aleksei Pogorelov~\cite{pogorelov}.

\begin{thm}{Theorem}\label{thm:alexandrov+pogorelov}
A metric space $\spc{X}$ is isometric to the surface of a convex body in the Euclidean space if and only if $\spc{X}$ is an $\Alex0$ space that is homeomorphic to $\SSS^2$.

Moreover, $\spc{X}$ determines the convex body up to congruence.
\end{thm}

The convex body above is a compact convex subset in $\EE^3$;
we assume that it does not lie in a line but might degenerate to a plane figure, say $F$.
In the latter case, its surface is defined as two copies of $F$ glued along the boundary.
For nondegenerate convex body $B$, its surface is its boundary $\partial B$ equipped with the induced length metric. 

The only-if part of the theorem is the simplest; we will give a complete proof of it eventually.
The if part will be sketched.
We will not touch the last part.

%%!TEX root = the-definitions.tex
\chapter{Definitions}\label{chap:defs}

In this lecture we prove equivalence of several definitions of Alexandrov space.


\section{Four-point comparison}\label{sec:4-point}

Recall that $\angk  pxy$ denotes the model angle; see page \pageref{page:model-angle}.

Let $p,x,y,z$ be a quadruple of points in a metric space.
If the inequality 
\[\angk  pxy_{\EE^2}+\angk pyz_{\EE^2}+\angk pzx_{\EE^2}
\le 
2\cdot\pi
\eqlbl{eq:CBB-comparison}\]
holds, then we say that the quadruple meets \index{comparison}\emph{$\EE^2$-comparison}.
If the left-hand side is undefined, then we assume that the comparison holds.

\begin{thm}{Exercise}\label{ex:CBB+-}
Suppose $\EE^2$-comparison holds for quadruple $p,x_1,x_2,x_3$.
Show that $\EE^2$-comparison holds for quadruple $q,y_1,y_2,y_3$ if
\[\dist{q}{y_i}{}\ge\dist{p}{x_i}{}\qquad\text{and}\qquad\dist{y_i}{y_j}{}\le\dist{x_i}{x_j}{}\]
for all $i$ and $j$.
\end{thm}

Instead of $\EE^2$, we can use $\SSS^2$ or $\HH^2$.
This way we get the definition of $\SSS^2$- or $\HH^2$-comparisons.
Recall that $\angk  pxy_{\EE^2}$ and $\angk  pxy_{\HH^2}$ are defined if $p\ne x$, $p\ne y$,
but for $\angk  pxy_{\SSS^2}$ we require in addition that
\[\dist{p}{x}{}+\dist{p}{y}{}+\dist{x}{y}{}<2\cdot\pi;\]
if this does not hold for one of the angles, then we assume that $\SSS^2$-comparison holds for this quadruple.

More generally, one may apply this definition to $\MM^2(\kappa)$ and  define $\MM^2(\kappa)$-comparison for any real $\kappa$.
However, if you see $\MM^2(\kappa)$-comparison, it is safe to assume that $\kappa=-1$, $0$, or $1$;
applying rescaling, the $\MM^2(\kappa)$-comparison can be reduced to these three cases.

\begin{thm}{Definition}\label{def:CBB}
A metric space $\spc{X}$ has {}\emph{curvature $\ge\kappa$} in the sense of Alexandrov
if $\MM^2(\kappa)$-comparison
holds for any quadruple of points in $\spc{X}$.
\end{thm}

While this definition can be applied to any metric space,
we will use it mostly for geodesic spaces that are complete (and often compact or proper). 
If a complete geodesic space has curvature $\ge\kappa$ in the sense of Alexandrov, 
then it will be called an $\Alex\kappa$ space; here $\Alex\kappa$ is an adjective.
An $\spc{X}$ is $\Alex\kappa$ for some $\kappa$, then we say that $\spc{X}$ is an \index{Alexandrov space}\emph{Alexandrov space}.

It is common practice in Alexandrov geometry to write proofs for nonnegative curvature and 
leave the general curvature bound as an exercise. These generalizations are usually straightforward. We will add notes when they are not.
We will also often formulate statements just for $\kappa=0$ even when they admit straightforward generalizations to arbitrary curvature bounds;
see \cite{alexander-kapovitch-petrunin2024} for a more formal tratment.


\begin{thm}{Exercise}\label{ex:Euclid-is-CBB}
Show that $\EE^n$ is $\Alex0$.
\end{thm}

\begin{thm}{Exercise}\label{ex:(3+1)-expanding}
Show that a metric space $\spc{X}$ has nonnegative curvature in the sense of Alexandrov
if and only if for any quadruple of points $p,x_1,x_2,x_3\in \spc{X}$ 
there is a quadruple of points $q,y_1,y_2,y_3\in\EE^3$
such that 
\[\dist{p}{x_i}{\spc{X}}\ge\dist{q}{y_i}{\EE^2} 
\quad \text{and}\quad
\dist{x_i}{x_j}{\spc{X}}\le\dist{y_i}{y_j}{\EE^2}\] 
for all $i$ and $j$.
\end{thm}

\section{Alexandrov's lemma}

Recall that $[xy]$ denotes a geodesic from $x$ to $y$;
set  
\index{10@$\left]x y\right]$, $\left[x y\right[$, $\left]x y\right[$}
\[
\left]x y\right]=[xy]\setminus\{x\},
\quad
\left[x y\right[=[xy]\setminus\{y\},
\quad
\left]x y\right[=[xy]\setminus\{x,y\}.\]

\begin{thm}{Lemma}
\index{Alexandrov's lemma}
\label{lem:alex}  
Let $p,x,y,z$ be distinct points in a metric space such that $z\in \left]x y\right[$.
Then 
the following expressions have the same sign:

\begin{subthm}{lem-alex-difference}
$\angk x p y
-\angk x p z$,
\end{subthm} 

\begin{subthm}{lem-alex-angle}
$\angk z p x
+\angk z p y -\pi$.
\end{subthm}

\begin{wrapfigure}{r}{25mm}
\vskip-6mm
\centering
\includegraphics{mppics/pic-730}
\end{wrapfigure}

The same holds for the hyperbolic and spherical model angles, 
but in the latter case, one has to assume in addition that
\[\dist{p}{x}{}+\dist{p}{y}{}+\dist{x}{y}{}< 2\cdot\pi.\]

\end{thm}

In the proof we will apply the following statement from elementary geometry.

\begin{thm}{Angle monotonicity}\label{angle-monotonicity}
Increasing the opposite side in a plane triangle increases the corresponding angle, and the other way around.

Moreover, the same statement holds for spherical and hyperbolic triangles.
\end{thm}


\parit{Proof.} 
Consider the model triangle $\trig{\tilde x}{\tilde p}{\tilde z}=\modtrig(x p z)$.
Take 
a point $\tilde y$ on the extension of 
$[\tilde x \tilde z]$ beyond $\tilde z$ so that $\dist{\tilde x}{\tilde y}{}=\dist{x}{y}{}$ (and therefore $\dist{\tilde x}{\tilde z}{}=\dist{x}{z}{}$). 

\begin{wrapfigure}{r}{33mm}
\vskip-0mm
\centering
\includegraphics{mppics/pic-740}
\end{wrapfigure}

By the angle monotonicity (\ref{angle-monotonicity}),
the following expressions have the same sign:
\begin{enumerate}[(i)]
\item $\mangle\hinge{\tilde x}{\tilde p}{\tilde y}-\angk{x}{p}{y}$,
\item $\dist{\tilde p}{\tilde y}{}-\dist{p}{y}{}$,
\item $\mangle\hinge{\tilde z}{\tilde p}{\tilde y}-\angk{z}{p}{y}$.
\end{enumerate}
Since 
\[\mangle\hinge{\tilde x}{\tilde p}{\tilde y}=\mangle\hinge{\tilde x}{\tilde p}{\tilde z}=\angk{x}{p}{z}\]
and
\[ \mangle\hinge{\tilde z}{\tilde p}{\tilde y}
=\pi-\mangle\hinge{\tilde z}{\tilde x}{\tilde p}
=\pi-\angk{z}{x}{p},\]
the statement follows.


The spherical and hyperbolic cases can be proved along the same lines.
\qeds

\begin{thm}{Exercise}\label{ex:alex-lemma-cat}
Assume $p,x,y,z$ are as in Alexandrov's lemma (\ref{lem:alex}).
Show that
\[\angk p x y
\ge
\angk p x z + \angk p z y,\]
with equality if and only if the expressions in \ref{SHORT.lem-alex-difference} and \ref{SHORT.lem-alex-angle} in Alexandrov's lemma vanish.
\end{thm}

Note that 
\[p\in\left]x y\right[
\quad\Longrightarrow\quad
\angk pxy=\pi.
\]
Applying it with Alexandrov's lemma and $\EE^2$-comparison, we get the following.

\begin{thm}{Claim}\label{clm:angle-mono}
If $p,x,y,z$ are points in an $\Alex0$ space.
Suppose $p\in\left]x y\right[$, then 
\[\angk xyz\le \angk xpz.\]
\end{thm}

\begin{wrapfigure}{r}{25mm}
\vskip-0mm
\centering
\includegraphics{mppics/pic-750}
\end{wrapfigure}

\begin{thm}{Exercise}\label{ex:noncreasing}
Let $\hinge p x y$ be a hinge in an $\Alex0$ space.
Consider the function
\[f\:(\dist{p}{\bar x}{},\dist{p}{\bar y}{})\mapsto \angk p{\bar x}{\bar y},\]
where $\bar x\in\left]p x\right]$ and $\bar y\in\left]p y\right]$.
Show that $f$ is nonincreasing in each argument.
\end{thm}

This exercise implies the following.

\begin{thm}{Claim}\label{clm:angle-defined}
The angle measure of any hinge in an $\Alex0$ 
space is defined and  is at least as large as the corresponding model angle;
that is,
\[\mangle\hinge p x y\ge \angk p x y\]
for any hinge $\hinge p x y$ in an $\Alex0$.

\end{thm}

\begin{thm}{Exercise}\label{ex:0-angle}
Let $\hinge p x y$ be a hinge in an $\Alex0$ space.
Suppose $\mangle\hinge p x y=0$; show that $[px]\subset [py]$ or $[py]\subset [px]$.

Conclude that geodesics in $\Alex0$ space cannot \emph{bifurcate};
that is, if two geodesics $[px]$ and $[py]$ share a nontrivial arc with an end at $p$, then $[px]\subset [py]$ or $[py]\subset [px]$.
\end{thm}

\begin{thm}{Exercise}\label{ex:pi-angle}
Let $[xy]$ be a geodesic in an $\Alex0$ space.
Suppose $z\in \left]xy\right[$. Show that there is a unique geodesic $[xz]$ and $[xz]\subset [xy]$.
\end{thm}

Recall that adjacent hinges are defined in \ref{ex:adjacent-angles}.

\begin{thm}{Exercise}\label{ex:adjacent-CBB}
Let $\hinge pxz$ and $\hinge pyz$ be adjacent hinges in an $\Alex0$ 
space.
Show that
\[\mangle\hinge pxz+\mangle\hinge pyz= \pi.\]
\end{thm}


\begin{thm}{Exercise}\label{ex:pxyvw}
Let $\spc{A}$ be an $\Alex0$ 
space.
Show that  
\[
\angk xyp=\angk xvp
\quad\Longleftrightarrow\quad
\angk xyp=\angk xwp
\]
for any points
$p,x,y,v,w$ in $\spc{A}$ such that $v,w\in \left]xy\right[$.
\end{thm}

\begin{thm}{Exercise}\label{ex:angle-lim}
Let $\spc{A}$ be an $\Alex0$ space.
Suppose hinges $\hinge {x_n}{y_n}{z_n}$ in $\spc{A}$ converge to a hinge $\hinge {x_\infty}{y_\infty}{z_\infty}$;
that is, geodesics $[x_ny_n]$ and $[x_nz_n]$ converge to the geodesics $[x_\infty y_\infty]$ and $[x_\infty z_\infty]$ in the sense of Hausdorff.
Show that 
\[\liminf_{n\to\infty}\mangle \hinge {x_n}{y_n}{z_n}\ge \mangle \hinge {x_\infty}{y_\infty}{z_\infty}.\]
\end{thm}

The last inequality might be strict;
for example, on the surface of convex polyhedron, which is a $\Alex0$ space by \ref{prop:conv-surf-CBB(0)}.

\section{Hinge comparison}

Let $\hinge pxy$ be a hinge in an $\Alex0$ space $\spc{A}$.
By \ref{ex:noncreasing}, the angle measure $\mangle\hinge pxy$ is defined and
\[\mangle\hinge pxy\ge \angk pxy.\]
Further, according to \ref{ex:adjacent-CBB}, we have 
\[\mangle\hinge pxz+\mangle\hinge pyz=\pi\]
for adjacent hinges $\hinge pxz$ and $\hinge pyz$ in $\spc{A}$.

The following theorem provides a converse.

\begin{thm}{Theorem}\label{thm:angle-cbb}
A complete geodesic space $\spc{A}$ is $\Alex0$ if the following conditions hold.

\begin{subthm}{angle-a}
For any hinge $\hinge x p y$ in $\spc{A}$, the angle 
$\mangle\hinge x p y$ is defined and 
\[\mangle\hinge x p y\ge\angk x p y.\]
\end{subthm}

\begin{subthm}{angle-b}
For any two adjacent hinges $\hinge pxz$ and $\hinge pyz$ in $\spc{A}$, we have
\[\mangle\hinge pxz+\mangle\hinge pyz\le\pi.\]
\end{subthm}

\end{thm}

\parit{Proof.}
Consider a point  $w\in \mathopen{]} p z \mathclose{[}$ close to $p$.
From \ref{SHORT.angle-b}, it follows that 
\[\mangle\hinge w x z+ \mangle\hinge w x{p}\le\pi\quad \text{and}\quad \mangle\hinge w y z + \mangle\hinge w y{p}\le\pi.\]

\begin{wrapfigure}{o}{30 mm}
\vskip-0mm
\centering
\includegraphics{mppics/pic-805}
\vskip4mm
\end{wrapfigure}

Since $\mangle\hinge w x y\le \mangle\hinge w x p +\mangle\hinge w y{p}$ (see \ref{claim:angle-3angle-inq}), we get 
\[\mangle\hinge w x z+ \mangle\hinge w y z +\mangle\hinge w x y
\le
2\cdot\pi.\]
Applying \ref{SHORT.angle-a}, 
\[\angk w x z
+ \angk w y z 
+\angk w x y
\le
2\cdot\pi.\]
Passing to the limits as $w\to p$, we have
\[\angk p x z 
+ \angk p y z 
+\angk p x y
\le
2\cdot\pi.\]
\qedsf

\section{Equivalent conditions}

The following theorem summarizes \ref{clm:angle-mono}, \ref{clm:angle-defined}, \ref{ex:adjacent-CBB}, and \ref{thm:angle-cbb}.

\begin{thm}{Theorem}\label{thm:defs_of_alex} 
Let $\spc{A}$ be a complete geodesic space.
Then the following conditions are equivalent.

\begin{subthm}{cbb}
$\spc{A}$ is $\Alex0$.
\end{subthm}
 

\begin{subthm}{2-sum} 
(adjacent angle comparison\index{comparison!adjacent angle comparison})
\[\angk z p x
+\angk z p y\le \pi\]
for any geodesic $[x y]$ and point $z\in \mathopen{]}x y\mathclose{[}$, $z\ne p$ in $\spc{A}$.
\end{subthm}

\begin{subthm}{point-on-side}
(\index{comparison!point-on-side comparison}point-on-side comparison)
\[\angk x p y\le\angk x p z\]
for any geodesic $[x y]$ and $z\in \mathopen{]}x y\mathclose{[}$ in $\spc{A}$.
\end{subthm}

\begin{subthm}{angle}(hinge comparison\index{comparison!hinge comparison})
\index{hinge!comparison}
the angle $\mangle\hinge x p y$ is defined for any hinge $\hinge x p y$ in $\spc{A}$.
Moreover, 
\[\mangle\hinge x p y\ge\angk x p y\]
for any hinge $\hinge x p y$, and
\[\mangle\hinge z p y + \mangle\hinge z p x\le\pi\]
for any adjacent hinges $\hinge z p y$ and $\hinge z p x$.
\end{subthm}

Moreover, the implications \ref{SHORT.cbb}$\Rightarrow$\ref{SHORT.2-sum}$\Rightarrow$\ref{SHORT.point-on-side}$\Rightarrow$\ref{SHORT.angle} hold in any space, not necessarily a geodesic one.
\end{thm}

\begin{thm}{Advanced Exercise}\label{ex:urysohn}
Construct a complete geodesic space $\spc{X}$ that is not $\Alex0$, but satisfies the following weaker version of the adjacent angle comparison \ref{2-sum}.

For any three points $p,x,y\in \spc{X}$ there is a geodesic $[x y]$ such that for any $z\in \left]x y\right[$
\[\angk{z}{p}{x}+\angk{z}{p}{y}
\le
\pi.\]
\end{thm}

\begin{thm}{Exercise}\label{ex:normCBB}
Let $\spc{W}$ be $\RR^n$ with the metric induced by a norm.
Show that if $\spc{W}$ is $\Alex0$, then $\spc{W}$ is isometric to the Euclidean space~$\EE^n$.
\end{thm}

\section{Function comparison}\label{Function comparison}

\parbf{Real-to-real functions.}
Choose $\lambda\in \RR$.
Let $s\:\II\to\RR$ be a locally Lipschitz function defined on an interval $\II$.
The following statement are equivalent;
if one (and therefore any) of them holds for $s$, then we say that $s$ is \index{91@$\lambda$-concave function}\emph{$\lambda$-concave}.
\begin{itemize}
\item We have inequality $s''\le \lambda$, where the second derivative $s''$ is understood in the sense of distributions.
\item The function $t\mapsto s(t)-\lambda\cdot\tfrac{t^2}2$ is concave.
\item The \index{Jensen inequality}\emph{Jensen inequality}
\[s(a\cdot t_0+(1-a)\cdot t_1)\ge a\cdot s(t_0)+(1-a)\cdot s(t_1)+\tfrac\lambda2\cdot a\cdot(1-a)\cdot(t_1-t_0)^2 \]
holds for any $t_0,t_1\in \II$ and $a\in[0,1]$.
\item for any $t_0\in \II$ there is a quadratic polynomial $\ell=\tfrac\lambda2\cdot t^2+a\cdot t+b$ (it is called a \index{barrier}\emph{barrier}) that supports (locally) $s$ at $t_0$ from above;
that is, $\ell(t_0)\z= s(t_0)$ and $\ell(t)\ge s(t)$ for any $t$ (in a neighborhood of $t_0$)
\end{itemize}

To prove equivalence, approximate $f$ by smooth functions taking a convolutions $f_n=f*k_n$ for a suitable sequence of kernels $k_n$.
Note that all the conditions are equivalent for $f_n$;
passing to the limit we get the same for $f$.

\begin{thm}{Exercise}\label{ex:concave'}
Show that $\lambda$-concave functions are one-sided differentiable.
\end{thm}

The following exercise implies that if the function defined on an open interval, then the Lipschitz condition can be dropped from the definition of $\lambda$-concavity.

\begin{thm}{Exercise}\label{ex:concave-open}
Suppose a real-to-real function $f$ is defined on an open inerval and satisfies one the Jensen inequality stated above.
Show that $f$ is locally Lipscitz.
\end{thm}

\parbf{Functions on metric spaces.}
A function on a metric space $\spc{A}$ will usually mean a \textit{locally Lipschitz real-valued function defined on an open subset of $\spc{A}$}.
The domain of a function $f$ will be denoted by $\Dom f$.

We say that $f$ is \index{91@$\lambda$-concave function}\emph{$\lambda$-concave} (briefly $f''\le \lambda$) if
for any unit-speed geodesic $\gamma\:\II\z\to \Dom f$
the real-to-real function $t\mapsto f\circ\gamma(t)$ is $\lambda$-concave.

The following proposition is simple but conceptual ---
it reduces a global comparison to an infinitesimal condition on distance functions.

\begin{thm}{Proposition}\label{comp-kappa}
A complete geodesic space $\spc{A}$ is $\Alex0$ if and only if $f''\le 1$ for any function $f$ of the form
\[f\:x\mapsto \tfrac12\cdot\dist[2]{p}{x}{}.\] 
\end{thm} 

\parit{Proof.}
Choose a unit-speed geodesic $\gamma$ in $\spc{A}$ and two points $x=\gamma(t_0)$, $y=\gamma(t_1)$ for some $t_0<t_1$.
Consider the model triangle $\trig{\tilde p}{\tilde x}{\tilde y}\z=\modtrig(p x y)$.
Let $\tilde \gamma\:[t_0,t_1]\to\EE^2$ be the unit-speed parametrization of $[\tilde x \tilde y]$ from $\tilde x$ to $\tilde y$.

Set
\begin{align*} 
\tilde r(t)&\df\dist{\tilde p}{\tilde\gamma(t)}{},
& 
r(t)&\df\dist{p}{\gamma(t)}{}.
\end{align*}
Clearly, $\tilde r(t_0)=r(t_0)$ and $\tilde r(t_1)=r(t_1)$.
Note that the point-on-side comparison (\ref{point-on-side}) says that the implication
\[t_0\le t\le t_1
\qquad\Longrightarrow\qquad
\tilde r(t)\le r(t)
\eqlbl{eq:r=<r}\]
holds for any $\gamma$ and $t_0<t_1$.

Jensen's inequality for the function $h$ is equivalent to \ref{eq:r=<r}.
Hence the proposition follows.
\qeds

\section{Semiconcave functions}\label{sec:Semiconcave functions}

Recall that $\lambda$-concave functions were defined in Section \ref{Function comparison},
and when we say \textit{function} we usually mean a \textit{locally Lipschitz function defined on an open domain}.

Let $f$ be a locally Lipschitz real-valued function defined in an open subset $\Dom f$ of an Alexandrov space $\spc{A}$.
Suppose $\phi$ is a continuous function defined in $\Dom f$.
We will write $f''\le \phi$ if for any point $x\in \Dom f$ and any $\eps>0$ there is a neighborhood $U\ni x$ such that
the restriction $f|_U$ is $(\phi(x)+\eps)$-concave.

If $f''\le \phi$ for some continuous function $\phi$, then $f$ is called  \index{semiconcave function}\emph{semiconcave}.

\begin{thm}{Exercise}\label{ex:distfun-semiconcave}
Let $f$ be a \emph{distance function} on an $\Alex0$ space $\spc{A}$;
that is, $f(x)\equiv\dist{p}{x}{}$ for some $p\in \spc{A}$.
Show that $f''\le \tfrac1f$.
In particular, $f$ is semiconcave in $\spc{A}\setminus\{p\}$.
\end{thm}

Proposition~\ref{comp-kappa} admits the following generalization.
The is nearly the same, but the formulas are getting more complicated.

\begin{thm}{Proposition}
A complete geodesic space $\spc{A}$ is $\Alex{\mp1}$
if $f''\z\le \pm f$ for any function of the type $f=\cosh\circ\distfun_p$ (respectively, $f=-\cos\circ\distfun_p|_{\oBall(p,\pi)}$).
\end{thm}

The geometric meaning of these inequalities remains the same:
\textit{distance functions are more concave than distance functions in $\MM^2(\kappa)$}.

\section{Remarks}

Note that Alexandrov's lemma is a result in neutral geometry;
it has the following useful variation; see \cite[10.2]{alexander-kapovitch-petrunin2024} or \cite[3.3]{alexander-kapovitch-kirszbraun}.

\begin{thm}{Overlap lemma}\label{lem:extend-overlap}
Let $\tilde x^1$, $\tilde x^2$, $\tilde x^3$, $\tilde p^1$, $\tilde p^2$, ans $\tilde p^3$ be points in $\EE^2$, $\SSS^2$, or $\HH^2$.
Assume that, for any permutation $\{i,j,k\}$ of $\{1,2,3\}$, we have
\begin{enumerate}[(i)]

\item
\label{no-overlap:px=px}
$\dist{\tilde p^i}{\tilde x^\kay}{}=\dist{\tilde p^j}{\tilde x^\kay}{}$,
%$\dist{\tilde p^i}{\tilde x^\kay}{}=\dist{\tilde p^j}{\tilde x^\kay}{}$,

\item
\label{no-overlap:orient-1}
$\tilde p^i$ and $\tilde x^i$ lie in the same closed half-space determined by $[\tilde x^j\tilde x^\kay]$,
\end{enumerate}

If no pair of triangles $\trig{\tilde p^i}{\tilde x^j}{\tilde x^\kay}$ overlap,
then
\[\mangle{\tilde p^1} +\mangle {\tilde p^2}+\mangle{\tilde p^3}> 2\cdot\pi,\]
where $\mangle\tilde p^i\df\mangle\hinge{\tilde p^i}{\tilde x^\kay}{\tilde x^j}$
for a permutation $\{i,j,k\}$ of $\{1,2,3\}$.
\end{thm}

The condition \ref{SHORT.angle-b} in \ref{thm:angle-cbb} might be superfluous.
This is a long-standing open problem possibly dating back to Alexandrov \cite[footnote in 4.1.5]{burago-burago-ivanov}.
Let us state it formally.

\begin{thm}{Open question}\label{open:hinge-}
Let $\spc{A}$ be a complete geodesic space (you can also assume that $\spc{A}$ is homeomorphic to $\mathbb{S}^2$ or $\RR^2$)
such that for any hinge $\hinge x p y$ in $\spc{A}$,
the angle $\mangle\hinge x p y$ is defined and
\[\mangle\hinge x p y\ge\angk x p y.\]
Is it true that $\spc{A}$ is an Alexandrov space?
\end{thm}

Our 4-point comparison in Section~\ref{sec:4-point} is closely related to the so-called $\CAT$ comparison, which defines an \textit{upper} curvature bound in the sense of Alexandrov;
this is the subject of our previous  book  \cite{alexander-kapovitch-petrunin-2019}.

In both comparisons we check certain conditions on the 6 distances between pairs of points in a 4-point set.
Michael Gromov \cite[Section 1.19$_+$]{gromov1999} suggested considering other conditions of that type for $n$-point subsets;
see \cite{toyoda,lebedeva-petrunin-zolotov,lebedeva2019,petrunin2017,lebedeva-petrunin2024,lebedeva-petrunin2023,lebedeva-petrunin2021,lebedeva-petrunin2025,eskenazis-mendel-naor,gromov2001} for the development of this idea.

One coul define Alxandrov space as a complete \textit{length} space with curvature $\ge \kappa$.
This condition is more natural and general, but many statements can be reduced to the geodesic case.
In particular, suppose $\spc{A}$ is a complete length space with curvature $\ge \kappa$,
then 
\textit{$\spc{A}$ can be isometrically embedded into an $\Alex\kappa$ space} --- the ultrapower of $\spc{A}$; see \cite[4.11+8.4]{alexander-kapovitch-petrunin2024}.
Also, by Plaut's theorem, any point $p$ in $\spc{A}$ can be connected by geodesics to \textit{most} of points in $\spc{A}$
\cite[8.11]{alexander-kapovitch-petrunin2024}; compare to \ref{ex:grad-dist:geod}.

%%!TEX root = the-globalization.tex
\chapter{Globalization}\label{chap:globalization}

The globalization theorem states that a locally Alexandrov space is globally Alexandrov.
We start with the simplest meaningful case of this theorem and indicate a way to extend.

\section{Globalization}

A complete geodesic metric space $\spc{A}$ is \index{locally $\Alex0$}\emph{locally $\Alex0$} if any point $p\in\spc{A}$ admits a neighborhood $U\ni p$ such that the $\EE^2$-comparison holds for any quadruple of points in $U$.

\begin{thm}{Globalization theorem}\label{thm:glob} 
Any compact locally $\Alex0$ space is $\Alex0$.
\end{thm}

\parit{Proof modulo the key lemma.}
Note that condition \ref{angle-b} holds in $\spc{A}$ (the proof is the same).
It remains to check \ref{angle-a};
that is,
\[\mangle\hinge x p y\ge\angk x p y
\eqlbl{eq:mod-angle-CBB-comp-glob}\]
for any hinge $\hinge x p y$ in $\spc{A}$.

First note that \ref{eq:mod-angle-CBB-comp-glob} holds for hinges in a small neighborhood of any point;
this can be proved the same way as \ref{clm:angle-defined} and \ref{ex:adjacent-CBB}, applying the local version of the $\EE^2$-comparison.
Since $\spc{A}$ is compact, there is $\eps>0$ such that \ref{eq:mod-angle-CBB-comp-glob} holds if $\dist{x}{p}{}+\dist{p}{y}{}<\eps$.
Applying the key lemma several times we get that \ref{eq:mod-angle-CBB-comp-glob} holds for any given hinge.
\qeds

\begin{thm}{Key lemma}\label{key-lem:globalization} 
Let $\spc{A}$ be locally $\Alex0$. 
Assume that the comparison
\[\mangle\hinge x p q
\ge\angk x p q\]
holds for any hinge $\hinge x p q$ with 
$\dist{x}{y}{}+\dist{x}{q}{}
<
\frac{2}{3}\cdot\ell$.
Then the comparison
\[\mangle\hinge x p q
\ge\angk x p q\] 
holds for any hinge $\hinge x p q$ with $\dist{x}{ p}{}+\dist{x}{q}{}<\ell$.
\end{thm}

Let $\hinge x p q$ be a hinge in $\spc{A}$.
Denote by $\side \hinge x p q$ its \index{$\side \hinge x p q$ (model side)}\index{model!side}\emph{model side};
this is the opposite side in a flat triangle with the same angle and two adjacent sides as in $\hinge x p q$.

\begin{wrapfigure}{r}{44mm}
\centering
\includegraphics{mppics/pic-105}
\end{wrapfigure}

More precisely,
consider the model hinge $\hinge {\tilde x} {\tilde p} {\tilde q}$ in $\EE^2$ that is defined by 
\begin{align*}
\mangle\hinge {\tilde x} {\tilde p} {\tilde q}_{\EE^2}&=\mangle\hinge x p q_{\spc{A}},
\\
\dist{\tilde x} {\tilde p}{\EE^2}&=\dist{x} {p}{\spc{A}},
\\
\dist{\tilde x} {\tilde q}{\EE^2}&=\dist{x} {q}{\spc{A}};
\intertext{then}
\side \hinge x p q_{\spc{A}}
&\df
\dist{\tilde p}{\tilde q}{\EE^2}.
\end{align*}

Note that 
\[\side \hinge x p q \ge\dist{p}{q}{}
\quad\Longleftrightarrow\quad
\mangle\hinge x p q\ge \angk x p q.
\]
We will use it in the following proof.

\parit{Proof.} 
It is sufficient to prove the inequality
\[\side \hinge x p q
\ge\dist{p}{q}{}\eqlbl{eq:thm:=def-loc*}\] 
for any hinge $\hinge x p q$ with $\dist{x}{p}{}+\dist{x}{q}{}<\ell$.

Consider a hinge $\hinge x p q$ such that 
\[\tfrac{2}{3}\cdot\ell \le\dist{p}{x}{}\z+\dist{x}{q}{}< \ell.\]
First, let us construct a new hinge $\hinge{x'}p q$ with
\[
\dist{p}{x}{}+\dist{x}{q}{}\ge\dist{p}{x'}{}+\dist{x'}{q}{},
\eqlbl{eq:thm:=def-loc-fourstar}\]
such that 
\[\side \hinge x p q
\ge\side \hinge{x'}p q.
\eqlbl{eq:thm:=def-loc-fivestar}\]

\parit{Construction.}
Assume $\dist{x}{q}{}\ge\dist{x}{p}{}$; otherwise, switch the roles of $p$ and $q$.
Take $x'\in [x q]$ such that 
\[\dist{p}{x}{}+3\cdot\dist[{{}}]{x}{x'}{}
=\tfrac{2}{3}\cdot\ell. \eqlbl{3|xx'|}\]
Choose a geodesic $[x' p]$ and consider the  hinge $\hinge{x'}p q$ formed by $[x'p]$ and $[x' q]\subset [x q]$.
The triangle inequality implies \ref{eq:thm:=def-loc-fourstar}.
Further, note that 
\begin{align*}
\dist{p}{x}{}\z+\dist{x}{x'}{}&<\tfrac{2}{3}\cdot\ell,
&
\dist{p}{x'}{}\z+\dist{x'}{x}{}&<\tfrac{2}{3}\cdot\ell.
\end{align*}
In particular, 
\[\mangle\hinge x p{x'}
\ge\angk x p{x'}
\quad \text{and}\quad 
\mangle\hinge {x'}p x
\ge\angk {x'}p x.
\eqlbl{eq:thm:=def-loc-threestar}\]

{

\begin{wrapfigure}{r}{30 mm}
\vskip-0mm
\centering
\includegraphics{mppics/pic-820}
\vskip-4mm
\end{wrapfigure}

Now, let 
$\trig{\tilde x}{\tilde x'}{\tilde p}=\modtrig(x x' p)$.
Take $\tilde  q$ on the extension of $[\tilde  x\tilde  x']$ beyond $x'$ such that $\dist{\tilde x}{\tilde q}{}\z=\dist{x}{q}{}$ (and therefore $\dist{\tilde x'}{\tilde q}{}=\dist{x'}{q}{}$).
By~\ref{eq:thm:=def-loc-threestar},
\[\mangle\hinge x p q
=\mangle\hinge  x p{x'}\ge\angk x p{x'}\quad \Rightarrow\quad 
\side \hinge x q p\ge\dist{\tilde p}{\tilde q}{}.\]
Hence
\begin{align*}
\mangle\hinge{\tilde x'}{\tilde p}{\tilde q}&= 
\pi
-\angk{x'}p x
\ge
\\
&\ge
\pi-\mangle\hinge{x'}p x
=
\\
&=
\mangle\hinge{x'}p q,
\end{align*}
and \ref{eq:thm:=def-loc-fivestar} follows.

}

\medskip

Let us continue the proof.
Set $x_0=x$.
Let us apply inductively the above construction to get a sequence of hinges  $\hinge{x_n}p q$ with $x_{n+1}=x_n'$.
From \ref{eq:thm:=def-loc-fivestar}, we have that the sequence  $s_n\z=\side \hinge{x_n}p q$ is nonincreasing.
\begin{figure}[ht!]
\centering
\includegraphics{mppics/pic-825}
\end{figure}

The sequence might terminate at some $n$ only if $\dist{p}{x_n}{}+\dist{x_n}{q}{}\z< \tfrac{2}{3}\cdot\ell $.
In this case, by the assumptions of the lemma, $\side \hinge{x_n}p q\ge\dist{p}{q}{}$.
Since the sequence $s_n$ is nonincreasing, inequality \ref{eq:thm:=def-loc*} follows.

Otherwise, the sequence $r_n=\dist{p}{x_n}{}+\dist{x_n}{q}{}$ is nonincreasing, and $r_n\ge\tfrac{2}{3}\cdot\ell$ for all $n$.
Note that by construction, the distances
$\dist{x_n}{x_{n+1}}{}$, $\dist{x_n}{p}{}$, and $\dist{x_n}{q}{}$ are bounded away from zero for all large $n$.
Indeed, since on each step, we move $x_n$ toward to the point $p$ or $q$ that is further away, the distances $\dist{x_n}{p}{}$ and $\dist{x_n}{q}{}$ become about the same.
Namely, by \ref{3|xx'|}, we have that $\dist{p}{x_n}{}-\dist{x_n}{q}{}\le \tfrac29\cdot\ell$ for all large $n$.
Since $\dist{p}{x_n}{}+\dist{x_n}{q}{}\ge \tfrac23\cdot\ell$, we have $\dist{x_n}{p}{}\ge \tfrac\ell{100}$ and $\dist{x_n}{q}{}\ge \tfrac\ell{100}$.
Further, since $r_n\ge\tfrac{2}{3}\cdot\ell$, \ref{3|xx'|} implies that $\dist{x_n}{x_{n+1}}{}>\tfrac\ell{100}$.


Since the sequence $r_n$ is nonincreasing, it converges.
In particular, $r_n-r_{n+1}\to 0$ as $n\to\infty$.
It follows that $\angk{x_n}{p_n}{x_{n+1}}\to \pi$,
where $p_n=p$ if $x_{n+1}\in [x_nq]$, and otherwise $p_n=q$.
Since $\mangle\hinge{x_n}{p_n}{x_{n+1}}\ge\angk{x_n}{p_n}{x_{n+1}}$, we have
$\mangle\hinge{x_n}{p_n}{x_{n+1}}\to \pi$  as $n\to\infty$.

It follows that
\[r_n-s_n=\dist{p}{x_n}{}+\dist{x_n}{q}{}-\side \hinge{x_n}p q\to 0.\] 
Together with the triangle inequality
\[
\dist{p}{x_n}{}+\dist{x_n}{q}{}\ge\dist{p}{q}{}
\]
this yields
\[\lim_{n\to\infty}\side \hinge{x_n}p q\ge \dist{p}{q}{}.\]
Finally, the monotonicity of the sequence $s_n=\side \hinge{x_n}p q$ implies \ref{eq:thm:=def-loc*}.
\qeds

\section{General case}

The globalization theorem  can be generalized to any curvature bound $\kappa$.
The case $\kappa\le 0$ is proved in the same way, but the case $\kappa>0$ requires modifications.

The compactness condition in our version of the theorem can be traded for completeness.
The proof uses the following statement where $r(x)$ measures the size of a neighborhood of $x$ where the comparison holds.

\begin{thm}{Exercise}\label{ex:alm-min}
Let $\spc{X}$ be a complete metric space.
Suppose $r\:\spc{X}\to \RR$ is a positive continuous function.
Show that for any $\eps>0$ there is a point $p\in \spc{X}$ such that 
\[r(x)> (1-\eps)\cdot r(p)\] 
for any $x\in \cBall[p,\tfrac{1}{\eps}\cdot r(p)]$.

\end{thm}

This implies the following general version of the globalization theorem.

\begin{thm}{Theorem}\label{thm:globalization+}
Any locally $\Alex\kappa$ length space is $\Alex\kappa$.
\end{thm}

By \ref{ex:k-><mono}, we have
\[\angk x y z_{\MM^2(\kappa)}\le \angk x y z_{\MM^2(\Kappa)}\]
if $\kappa\le \Kappa$ and the right-hand side is defined.
It follows that a $\Alex\Kappa$ space is \textit{locally} $\Alex\kappa$.
Therefore, the globalization theorem implies the following.

\begin{thm}{Claim}\label{clm:K>k}
If $\Kappa>\kappa$, then any $\Alex\Kappa$ space is $\Alex\kappa$.
\end{thm}

In other words the expression \textit{curvature bounded below by $\kappa$} makes sense for geodesic spaces.
However, by the following exercise, it does not make much sense in general.

\begin{thm}{Exercise}\label{ex:CBB(1)notitCBB(0)}
Let $\spc{X}$ be the set $\{p,x_1,x_2,x_3\}$ with the metric defined by
\[\dist{p}{x_i}{}=\pi,\quad\dist{x_i}{x_j}{}=2\cdot\pi\]
for all $i\ne j$.
Show that $\spc{X}$ has curvature $\ge 1$, but does not have curvature $\ge 0$.
\end{thm}

\begin{thm}{Exercise}\label{ex:RisCBB(1)}
Let $p$ and $q$ be points in an $\Alex1$ space $\spc{A}$.
Suppose $\dist{p}{q}{}>\pi$.
Denote by $m$ the midpoint of $[pq]$.
Show that for any hinge $\hinge mxp$ we have
either $\mangle\hinge mxp=0$ or $\mangle\hinge mxp=\pi$.

Conclude that $\spc{A}$ is isometric to a line interval or a circle.

\end{thm}

\begin{thm}{Exercise}\label{ex:perim-k>0}
Suppose  
$\spc{A}$ is an $\Alex1$
and $\diam \spc{A}\le \pi$.
Show that 
\[\dist{x}{y}{}+\dist{y}{z}{}+\dist{z}{x}{}\le 2\cdot\pi\]
for any triple of points $x,y,z\in \spc{A}$.
\end{thm}


\section{Remarks}

The following question about \ref{angle-a} was stated in \cite[footnote in 4.1.5]{burago-burago-ivanov} but this is a long-standing open problem (possibly dating back to Alexandrov).

\begin{thm}{Open question}\label{open:hinge-}
Let $\spc{A}$ be a complete geodesic space (you can also assume that $\spc{A}$ is homeomorphic to $\mathbb{S}^2$ or $\RR^2$) 
such that for any hinge $\hinge x p y$ in $\spc{A}$, 
the angle $\mangle\hinge x p y$ is defined and 
\[\mangle\hinge x p y\ge\angk x p y.\]
Is it true that $\spc{A}$ is an Alexandrov space?
\end{thm}

The globalization theorem is also known as the \textit{generalized Toponogov theorem}.
Its two-dimensional case was proved by Paolo Pizzetti \cite{pizzetti};
later it was reproved independently by Alexandr Alexandrov \cite{alexandrov:devel}. %is it right ref?? 
Victor Toponogov \cite{toponogov-globalization+splitting} proved it for Riemannian manifolds of all dimensions.
For Alexandrov spaces of all dimensions, the theorem first appears in the paper of Michael Gromov, Yuriy Burago, and Grigory Perelman \cite{burago-gromov-perelman}.
Their statement is slightly more general than \ref{thm:globalization+}; it is for complete length spaces.
Another version for noncomplete, but geodesic spaces was proved by the second author \cite{petrunin:globalization}.


We took the proof from our book \cite{alexander-kapovitch-petrunin2024}, but reduced generality to compact nonnegatively curved spaces.
This proof is based on simplifications obtained by Conrad Plaut \cite{plaut:dimension} and Dmitry Burago, Yuriy Burago, and Sergei Ivanov \cite{burago-burago-ivanov}.
The same proof was rediscovered independently by Urs Lang and Viktor Schroeder \cite{lang-schroeder:globalization}.
Another simplified argument was found by Katsuhiro Shiohama \cite{shiohama}.





%%!TEX root = the-calculus.tex

\chapter{Calculus}\label{chap:derivative}

This lecture defines several notions related to the first-order derivatives in Alexandrov spaces;
this includes space of directions, tangent space, differential, and gradient.

\section{Space of directions} 
\label{sec:space+directions}

Let $\spc{A}$ be an Alexandrov space.
By \ref{ex:noncreasing}, the angle measure of any hinge in is defined.
Given $p\in \spc{A}$, consider the set $\mathfrak{S}_p$ of all nontrivial geodesics starting at $p$.
By \ref{claim:angle-3angle-inq}, the triangle inequality holds for $\mangle$ on $\mathfrak{S}_p$,
that is, $(\mathfrak{S}_p,\mangle)$ 
forms a \index{semimetric}\emph{semimetric} space;
that is, $\mangle$ behaves like a metric, but might vanish for distinct directions. 

The metric space corresponding to  $(\mathfrak{S}_p,\mangle)$ is called the \index{70@$\Sigma_p'$ (geodesic directions)}\index{space of geodesic directions}\emph{space of geodesic directions} at $p$, denoted by $\Sigma'_p$ or $\Sigma'_p\spc{A}$.
The elements of $\Sigma'_p$ are called \index{geodesic!direction}\emph{geodesic directions} at $p$.
Each geodesic direction is formed by an equivalence class of geodesics starting from $p$ 
for the equivalence relation 
\[[px]\sim[py]\quad \iff\quad \mangle\hinge pxy=0;\]
the direction of $[px]$ is denoted by $\dir px $.\index{40@$\dir{p}{q}$ (direction)}
By \ref{ex:0-angle}, 
\[[px]\sim[py]
\quad\iff\quad
[px]\subset [py]
\quad\text{or}\quad
[px]\supset[py].
\]
 
The completion of $\Sigma'_p$ is called the \index{space of directions}\emph{space of directions} at $p$ and is denoted by \index{70@$\Sigma_p$ (space of directions)}$\Sigma_p$ or $\Sigma_p\spc{A}$.
The elements of $\Sigma_p$ are called \index{direction}\emph{directions} at $p$.

\begin{thm}{Exercise}\label{ex:dir-compact}
Let $\spc{A}$ be an Alexandrov space.
Assume that a sequence of geodesics $[px_n]$ converge to a geodesic $[px_\infty]$ in the sense of Hausdorff,
and $x_\infty\ne p$.
Suppose $\Sigma_p$ is compact.
Show that $\dir p{x_n}\z\to\dir p{x_\infty}$ as $n\to\infty$.

\end{thm}


\section{Tangent space}\label{sec: tangent space}

The \index{65@$\Cone$}\index{cone}\emph{Euclidean cone} $\spc{V}=\Cone\spc{X}$
over a metric space $\spc{X}$
is defined as the metric space whose underlying set consists of
equivalence classes in
$[0,\infty)\times \spc{X}$ with the equivalence relation ``$\sim$'' given by $(0,p)\sim (0,q)$ for any points $p,q\in\spc{X}$,
and whose metric is given by the cosine rule
\[
\dist{(s,p)}{(t,q)}{\spc{V}} 
=
\sqrt{s^2+t^2-2\cdot s\cdot t\cdot \cos\theta},
\]
where $\theta= \min\{\pi, \dist{p}{q}{\spc{X}}\}$.

The leading example is
\[\Cone\SSS^n\iso\EE^{n+1};\]
here ``$\iso$'' stands for ``isometric to''. 
Now let us extend several notions from Euclidean space to Euclidean cones. 

The point in $\spc{V}$ that corresponds $(t,x)\z\in[0,\infty)\times \spc{X}$ will be denoted by $t\cdot x$.
The point in $\spc{V}$ formed by the equivalence class of $\{\0\}\times\spc{X}$ is called the \index{origin}\emph{origin} of the cone and is denoted by $\0$ or $\0_{\spc{V}}$.
For $v\in\spc{V}$ the distance $\dist{\0}{v}{\spc{V}}$ is called the \index{norm}\emph{norm} of $v$ and is denoted by $|v|$ or $|v|_{\spc{V}}$.
The \index{scalar product}\emph{scalar product} $\<v,w\>$
of $v=s\cdot p$ and $w=t\cdot q$
is defined by 
\[\<v,w\>
\df |v|\cdot|w|\cdot\cos\theta
\]
where $\theta= \min\{\pi, \dist{p}{q}{\spc{X}}\}$.
The value $\theta$ is undefined if $v=\0$ or $w=\0$;
in these cases we set $\<v,w\>\df0$.

\begin{thm}{Exercise}\label{ex:geodesic-cone}
Show that $\Cone\spc{X}$ is geodesic if and only if $\spc{X}$ is \index{91@$\ell$-geodesic space}\emph{$\pi$-geodesic};
that is, any two points $x,y\in \spc{X}$ such that $\dist{x}{y}{\spc{X}}<\pi$ can be joined by a geodesic in $\spc{X}$.
\end{thm}

\parbf{Tangent space.}
The Euclidean cone $\Cone\Sigma_p$ over the space of directions $\Sigma_p$ is called the \index{tangent space}\emph{tangent space} at $p$ and is denoted by \index{70@$\T_p$ (tangent space)}$\T_p$ or $\T_p\spc{A}$.
The elements of $\T_p\spc{A}$ will be called \index{tangent vector}\emph{tangent vectors} at $p$
(despite that $\T_p$ is only a cone --- not a vector space).
The space of directions $\Sigma_p$ can be (and will be) identified with the unit sphere in~$\T_p$;
that is, with the set $\set{v\in\T_p}{|v|=1}$.

\begin{thm}{Proposition}\label{prop:Tan-is-CBB(0)}
Any tangent space to an Alexandrov space has nonnegative curvature in the sense of Alexandrov.
\end{thm}

Halbeisen's example \cite{alexander-kapovitch-petrunin2024} shows that the tangent space $\T_p$ at some point of Alexandrov space might fail to be geodesic;
in this case $\T_p$ is \textit{not} $\Alex0$.

\parit{Proof.}
Consider the tangent space $\T_p=\Cone \Sigma_p$ of an Alexandrov space $\spc{A}$ at a point $p$.
We need to show that the $\EE^2$-comparison holds for a given quadruple $v_0$, $v_1$, $v_2$, $v_3\in \T_p$.

Recall that the space of geodesic directions $\Sigma_p'$ is dense in $\Sigma_p$.
It follows that the subcone $\T'_p=\Cone\Sigma_p'$ is dense in $\T_p$.
Therefore, it is sufficient to consider the case $v_0$, $v_1$, $v_2$, $v_3\in \T'_p$.

For each $i$, choose a geodesic $\gamma_i$ from $p$ in the direction of $v_i$;
reparametrize each $\gamma_i$ so that it has speed $|v_i|$.
Since the angles are defined, we have
\[\dist{\gamma_i(\eps)}{\gamma_j(\eps)}{\spc{A}}=\eps\cdot\dist{v_i}{v_j}{\T_p}+o(\eps)
\eqlbl{eq:gamma-v}\]
for $\eps>0$.
The quadruple $\gamma_0(\eps)$, $\gamma_1(\eps)$, $\gamma_2(\eps)$, $\gamma_3(\eps)$ meets the $\MM^2(\kappa)$-comparison.
After rescaling all the distances by $\tfrac1\eps$, it becomes the $\MM^2(\eps^2\cdot\kappa)$-comparison.
Passing to the limit as $\eps\to 0$ and applying \ref{eq:gamma-v}, we get the $\EE^2$-comparison for $v_0$, $v_1$, $v_2$, $v_3$.
\qeds


\begin{thm}{Exercise}\label{ex:GHto-tangent}
Let $p$ be a point in an Alexandrov space $\spc{A}$,
and let $\lambda_n\to\infty$.
Suppose $\Sigma_p$ is compact.
Show that there is a pointed Gromov--Hausdorff convergence $(\lambda_n\cdot \spc{A},p)\z\to (\T_p,0)$.
Moreover, for any geodesic $\gamma$ that starts at $p$, we have
\[\iota_n\circ\gamma(t/\lambda_n)\to t\cdot \gamma^+(0),\]
where $\iota_n$ sends a point in $\spc{A}$ to the corresponding point in $\lambda_n\cdot\spc{A}$.
\end{thm}

\section{Semiconcave functions}\label{sec:Semiconcave functions}

Recall that $\lambda$-concave functions were defined in Section \ref{Function comparison},
and when we say \textit{function} we usually mean a \textit{locally Lipschitz function defined on an open domain}.

Let $f$ be a locally Lipschitz real-valued function defined in an open subset $\Dom f$ of an Alexandrov space $\spc{A}$.
Suppose $\phi$ is a continuous function defined in $\Dom f$.
We will write $f''\le \phi$ if for any point $x\in \Dom f$ and any $\eps>0$ there is a neighborhood $U\ni x$ such that 
the restriction $f|_U$ is $(\phi(x)+\eps)$-concave.


If $f''\le \phi$ for some continuous function $\phi$, then $f$ is called  \index{semiconcave function}\emph{semiconcave}.


\begin{thm}{Exercise}\label{ex:distfun-semiconcave}
Let $f$ be a \emph{distance function} on an $\Alex0$ space $\spc{A}$;
that is, $f(x)\equiv\dist{p}{x}{}$ for some $p\in \spc{A}$.
Show that $f''\le \tfrac1f$.
In particular, $f$ is semiconcave in $\spc{A}\setminus\{p\}$.
\end{thm}


\section{Differential}\label{sec:differential}
\index{differential of a function}

Let $f$ be a semiconcave function on an Alexandrov space $\spc{A}$, and $p\z\in \Dom f$.
Choose a unit-speed geodesic $\gamma$ that starts at $p$;
let $\xi\in\Sigma_p$ be its direction.
Define 
\[(\dd_pf)(\xi)\df(f\circ\gamma)^+(0),\]
here $(f\circ\gamma)^+$ denotes the \index{right derivative}\emph{right derivative} of $(f\circ\gamma)$;
it is defined since $f$ is semiconcave.

By the following exercise, the value $(\dd_pf)(\xi)$ is defined; that is, it does not depend on the choice of $\gamma$.
Moreover, $\dd_pf$ is a Lipschitz function on $\Sigma'_p$.
It follows that the function $\dd_pf\:\Sigma_p'\to\RR$ can be uniquely extended to a Lipschitz function $\dd_pf\:\Sigma_p\to\RR$.
Further, we can extend it to the tangent space by setting 
\[(\dd_pf)(r\cdot \xi)
\df
r\cdot (\dd_pf)(\xi)\]
for any $r\ge 0$ and $\xi\in\Sigma_p$.
The obtained function $\dd_pf\:\T_p\to\RR$ is Lipschitz;
it is called the \index{differential}\emph{differential} of $f$ at $p$.

\begin{thm}{Exercise}\label{ex:df(xi)}
Let $f$ be a semiconcave function on an Alexandrov space.
Suppose $\gamma_1$ and $\gamma_2$ are geodesics that start at $p\z\in \Dom f$;
denote by $\theta$ the angle between $\gamma_1$ and $\gamma_2$ at $p$.
Show that 
\[|(f\circ\gamma_1)^+(0)-(f\circ\gamma_2)^+(0)|\le L\cdot \theta,\]
where $L$ is the Lipschitz constant of $f$ in a neighborhood of $p$.
\end{thm}

\begin{thm}{Exercise (First variation formula)} \label{ex:d(distfun)}
Let $p$ and $q$ be distinct points in an Alexandrov space~$\spc{A}$.
Show the following.

\begin{subthm}{ex:d(distfun):<}
$\dd_p\distfun_q(v)\le -\langle\dir pq,v\rangle$
for any $v\in\T_p$.
\end{subthm}

\begin{subthm}{ex:d(distfun):=}
Suppose $\spc{A}$ is proper.
Let $\Uparrow_p^q$ be the set of all direction of  geodesics from $p$ to $q$.
Then
\[\dd_p\distfun_q(v)=-\max_{\xi\in\Uparrow_p^q}\langle\xi,v\rangle\]
for any $v\in\T_p$.
\end{subthm}

\end{thm}

\section{Gradient}\label{sec:grad-def}

The following definition generalizes the gradient to semiconcave functions on Alexandrov space.
This generalization is not trivial even for concave functions on Euclidean space;
we suggest keeping this example in mind while reading further.

\begin{thm}{Definition}\label{def:grad} 
Let $f$ be a semiconcave function on an Alexandrov space.
A tangent vector $g\in \T_p$ is called a 
\index{gradient}\emph{gradient} of $f$ at $p$ 
(briefly,  $g\z=\nabla_p f$\index{19@$\nabla$ (gradient)}) if
\begin{subthm}{}
$(\dd_p f)(w)\le \<g,w\>$ for any $w\in \T_p$, and
\end{subthm}

\begin{subthm}{}
$(\dd_p f)(g) = \<g,g\>.$
\end{subthm}
\end{thm}

The following exercise provides a property of gradients that will play a key role in the proof of the first distance estimate (\ref{thm:dist-est}).

\begin{thm}{Exercise}\label{ex:monotonicity}
Let $f$ be a $\lambda$-concave function on an Alexandrov space.
Suppose that gradients $\nabla_xf$ and $\nabla_yf$ are defined.
Show that 
\[\<\dir{x}{y},\nabla_{x}f\>
+
\<\dir{y}{x},\nabla_{y}f\>
+
\lambda\cdot\dist{x}{y}{}\ge 0.\]
\end{thm}

\begin{figure}[ht!]
\centering
\includegraphics{mppics/pic-409}
\end{figure}

\begin{thm}{Proposition}\label{prop:grad-exist}
Suppose that a semiconcave function $f$ is defined in a neighborhood of a point $p$ in an Alexandrov space.
Then the gradient $\nabla_pf$ is uniquely defined.

Moreover, if $\dd_pf\le 0$, then we have $\nabla_pf=0$;
otherwise, $\nabla_pf\z=s\cdot \overline{\xi}$, where 
$s= \dd_pf(\overline{\xi})$
and
$\overline{\xi}\in \Sigma_p$ is the direction that maximize the value $\dd_pf(\xi)$ for $\xi\in \Sigma_p$.
\end{thm}


\begin{thm}{Key lemma}\label{lem:ohta} 
Let $f$ be a semiconcave function that is defined in a neighborhood of a point $p$
in an Alexandrov space $\spc{A}$. 
Then for any $u,v\in \T_p$, we have
\[s\cdot \sqrt{|u|^2+2\cdot\<u,v\> +|v|^2}
\ge 
(\dd_p f)(u)+(\dd_p f)(v),\]
where
\[s=\sup\set{(\dd_p f)(\xi)}{\xi\in\Sigma_p}.\]

\end{thm}

If $\T_p\iso\EE^m$ and $\dd_p f$ is a concave function,
then
\[2\cdot(\dd_p f)(\tfrac{u+v}2)\ge(\dd_p f)(u)+(\dd_p f)(v).\]
The latter implies the statement since $|u+v|=\sqrt{|u|^2+2\cdot\<u,v\> +|v|^2}$.
In general, $\T_p$ is not geodesic (and not even a length space), so concavity of $\dd_p f$ does not make  sense.
The key lemma however says  that in a certain sense $\dd_p f$ behaves as a concave function.

Solving the following exercise should help to find an approach to the key lemma.

\begin{thm}{Exercise}\label{ex:d(distfun):==}
Let $p$ and $q$ be distinct points in an Alexandrov space $\spc{A}$.
Suppose the geodesic $[pq]$ can be extended beyond $q$.

Show that
\[\dd_p\distfun_q(v)= -\langle\dir pq,v\rangle\]
for any $v\in\T_p$.
\end{thm}

\parit{Proof of \ref{lem:ohta}.}
We will assume that $\spc{A}$ is $\Alex0$ and $f$ is concave;
the general case requires only minor modifications.
We can assume that $v\ne 0$, $w\ne 0$, and $\alpha=\mangle(u,v)>0$; otherwise, the statement is trivial.

{

\begin{wrapfigure}{r}{34 mm}
\vskip-4mm
\centering
\includegraphics{mppics/pic-1205}
\vskip0mm
\end{wrapfigure}

Consider a model configuration of five points: $\tilde p$, $\tilde u$, $\tilde v$, $\tilde q$, $\tilde w\in\EE^2$ such that
\begin{itemize}
\item $\mangle\hinge{\tilde p}{\tilde u}{\tilde v}=\alpha$, 
\item $\dist{\tilde p}{\tilde u}{}=|u|$, 
\item $\dist{\tilde p}{\tilde v}{}=|v|$,
\end{itemize}
}
\begin{itemize}
\item $\tilde q$ lies on an extension of $[\tilde p\tilde v]$ so that $\tilde v$ is the midpoint of $[\tilde p\tilde q]$, 
\item $\tilde w$ is the midpoint between $\tilde u$ and ${\tilde v}$.
\end{itemize}
Note that 
\[\dist{\tilde p}{\tilde w}{}
=
\tfrac{1}{2}\cdot\sqrt{|u|^2+2\cdot\<u,v\>+|v|^2}.\eqlbl{eq:|p-w|=}\]

Since the geodesic space of directions $\Sigma'_p$ is dense in $\Sigma_p$,
we can assume that there are geodesics in the directions of $u$ and $v$.
Choose such geodesics $\gamma_u$ and $\gamma_v$ and assume that they are parametrized with speed $|u|$ and $|v|$ respectively.
For all small $t>0$, consider points $u_t,v_t,q_t,w_t\in \spc{A}$ such that
\begin{itemize}
\item $v_t=\gamma_v(t)$,\quad  $q_t=\gamma_v(2\cdot t)$
\item $u_t=\gamma_u(t)$.
\item $w_t$ is the midpoint of $[u_t v_t]$.
\end{itemize}
Clearly 
\[\dist{p}{u_t}{}=t\cdot |u|,\qquad \dist{p}{v_t}{}=t\cdot|v|,\qquad \dist{p}{q_t}{}=2\cdot t\cdot|v|.\] 
Since $\mangle(u,v)$ is defined, 
we have 
\[\dist{u_t}{v_t}{}=t\cdot\dist{\tilde u}{\tilde v}{}+o(t),
\qquad
\dist{u_t}{q_t}{}=t\cdot\dist{\tilde u}{\tilde q}{}+o(t).\]

From the point-on-side and hinge comparisons (\ref{point-on-side}$+$\ref{angle}), we have
\[\angk{v_t}p{w_t}
\ge
\angk{v_t}p{u_t}
\ge
\mangle\hinge{\tilde v}{\tilde p}{\tilde u}+\tfrac{o(t)}t\]
and
\[\angk{v_t}{q_t}{w_t}
\ge
\angk{v_t}{q_t}{u_t}
\ge
\mangle\hinge{\tilde v}{\tilde q}{\tilde u}+\tfrac{o(t)}t.\]
Clearly, 
$\mangle\hinge{\tilde v}{\tilde p}{\tilde u}+\mangle\hinge{\tilde v}{\tilde q}{\tilde u}=\pi$. 
From the adjacent angle comparison (\ref{2-sum}), 
$\angk{v_t}p{u_t}\z+\angk{v_t}{u_t}{q_t}\le \pi$.
Hence
$\angk{v_t}p{w_t}
\to
\mangle\hinge{\tilde v}{\tilde p}{\tilde w}$ as $t\to0+$
and thus 
\[\dist{p}{w_t}{}=t\cdot\dist{\tilde p}{\tilde w}{}+o(t).\]

Without loss of generality, we can assume that $f(p)=0$.
Since $f$ is concave, we have 
\begin{align*}
2\cdot f(w_t)&\ge f(u_t)+f(v_t)=
\\
&=t\cdot [(\dd_p f)(u)+(\dd_p f)(v)]+o(t).
\end{align*}
 
Applying concavity of $f$, we have
\begin{align*}
(\dd_p f)(\dir p{w_t})
&\ge 
\frac{f(w_t)}{\dist{p}{w_t}{}}
\ge 
\\
&\ge
\frac{t\cdot[(\dd_p f)(u)+(\dd_p f)(v)]+o(t)}{2\cdot t\cdot\dist{\tilde p}{\tilde w}{}+o(t)}.
\end{align*}
By \ref{eq:|p-w|=}, the key lemma follows.
\qeds

\parit{Proof of \ref{prop:grad-exist}; uniqueness.} 
If $g,g'\in \T_p$ are two gradients of $f$,
then 
\begin{align*}
\<g,g\>
&=(\dd_p f)(g)\le \<g,g'\>,
&
\<g',g'\>
&=(\dd_p f)(g')\le \<g,g'\>.
\end{align*}
Therefore,
\[\dist[2]{g}{g'}{}=\<g,g\>-2\cdot\<g,g'\>+\<g',g'\>\le0.\] 
It follows that $g=g'$.

\parit{Existence.} 
If $\dd_p f\le 0$, then one can take $\nabla_p f=\0$.

Suppose $s=\sup\set{(\dd_p f)(\xi)}{\xi\in\Sigma_p}>0$, 
it is sufficient to show that there is  $\overline{\xi}\in \Sigma_p$ such that 
\[
(\dd_p f)\left(\overline{\xi}\right)=s.
\eqlbl{overlinexi}
\]
Indeed, suppose $\overline{\xi}$ exists.
Applying \ref{lem:ohta} for $u=\overline{\xi}$, $v=\eps\cdot w$ with $\eps\to0+$, 
we get
\[(\dd_p f)(w)\le \<w,s\cdot\overline{\xi}\>\] 
for any $w\in\T_p$;
that is, $s\cdot\overline{\xi}$ is the gradient at $p$.

Take a sequence of directions $\xi_n\in \Sigma_p$, such that $(\dd_p f)(\xi_n)\to s$.
Applying \ref{lem:ohta} for $u=\xi_n$ and $v=\xi_m$, we get
\[s
\ge
\frac{(\dd_p f)(\xi_n)+(\dd_p f)(\xi_m)}{\sqrt{2+2\cdot\cos\mangle(\xi_n,\xi_m)}}.\]
Therefore $\mangle(\xi_n,\xi_m)\to0$ as $n,m\to\infty$;
that is, $\xi_1,\xi_2,\dots$ is a Cauchy sequence.
Clearly, $\overline{\xi}=\lim_n\xi_n$ meets \ref{overlinexi}.
\qeds

\begin{thm}{Exercise}\label{ex:convergence-grad}
Let $f$ and $g$ be locally Lipschitz semiconcave functions defined in a neighborhood of a point $p$ in an Alexandrov space.
Show that 
\[\dist[2]{\nabla_p f}{\nabla_p g}{\T_p}
\le 
s\cdot(|\nabla_p f|+|\nabla_p g|),\]
where
\[s=\sup\set{|(\dd_p f)(\xi)-(\dd_p g)(\xi)|}{\xi\in\Sigma_p}.\]

Conclude that if the sequence of restrictions $\dd_p f_n|_{\Sigma_p}$ converges uniformly, then $\nabla_pf_n$ converges as $n\to\infty$.
Here we assume that all functions $f_1$, $f_2,\dots$ are semiconcave and locally Lipschitz. 
\end{thm}

\begin{thm}{Exercise}\label{ex:semicontinuous-grad}
Let $f$ be a locally Lipschitz $\lambda$-concave function on an Alexandrov space $\spc{A}$.

\begin{subthm}{ex:semicontinuous-grad:>s}
Suppose $s\ge 0$.
Show that $|\nabla_xf|> s$ if and only if for some point $y$ we have
\[f(y)-f(x)>s\cdot \ell+\lambda\cdot \tfrac{\ell^2}2,\]
were $\ell=\dist{x}{y}{}$.
\end{subthm}

\begin{subthm}{ex:semicontinuous-grad:lim} Show that $x\mapsto|\nabla_xf|$ is lower semicontinuous;
that is,
\[|\nabla_{x_\infty}f|\le \liminf_{x_n\to x_\infty} |\nabla_{x_n}f|.\]

\end{subthm}

\end{thm}

%%!TEX root = the-gradient-flow.tex

\chapter{Gradient flow}\label{chap:GF}

\section{Velocity of curve}

Let $\alpha$ be a curve in an Alexandrov space $\spc{L}$.
If for any choice of 
geodesics $[p\,\alpha(t_0+\eps)]$ the vectors 
\[\tfrac{1}{\eps}\cdot\dist{p}{\alpha(t_0+\eps)}{}\cdot\dir p{\alpha(t_0+\eps)}\]
converge as $\eps\to 0+$, then their limit in $\T_p$ is called the \index{right derivative}\emph{right derivative} of $\alpha$ at $t_0$; it will be denoted by $\alpha^+(t_0)$.
In addition, $\alpha^+(t_0)\df0$
if $\tfrac{1}{\eps}\cdot\dist{p}{\alpha(t_0+\eps)}{}\to 0$ as $\eps\to 0+$.

The tangent vector $v=\dist px{}\cdot\dir px$ can be called \index{logarithm}\emph{logarithm} of $x$ at $p$ (briefly, \index{$v=\log_p x$ (logarithm)}$\log_p x$);
it is a tangent vector at $p$ of a geodesic path from $p$ to $x$.\label{page:log}


\begin{thm}{Claim}\label{clm:fa'=dfa'}
Let $\alpha$ be a curve in an Alexandrov space $\spc{L}$.
Suppose $f$ a semiconcave Lipschitz function
defined in a neighborhood of $p\z=\alpha(0)$,
and $\alpha^+(0)$ is defined.
Then 
\[(f\circ\alpha)^+(0)
=
(\dd_pf)(\alpha^+(0)).\]
\end{thm}

\parit{Proof.}
Without loss of generality, we can assume that $f(p)=0$.
Suppose $f$ and therefore $\dd_pf$ are $L$-Lipschitz.

Choose a constant-speed geodesic $\gamma$ that starts from $p$,
such that the distance
$s=\dist{\alpha^+(0)}{\gamma^+(0)}{\T_p}$
is small.
By the definition of differential,
\[(f\circ\gamma)^+(0)=\dd_pf(\gamma^+(0)).\]

By comparison and the definition of $\alpha^+$,
\[\dist{\alpha(\eps)}{\gamma(\eps)}{\spc{L}}\le s\cdot\eps+o(\eps)\]
for $\eps>0$.
Therefore,
\[|f\circ\alpha(\eps)-f\circ\gamma(\eps)|\le L\cdot s\cdot\eps+o(\eps).\]

Suppose $(f\circ\alpha)^+(0)$ is defined.
Then
\[|(f\circ\alpha)^+(0)-(f\circ\gamma)^+(0)|\le L\cdot s.\]
Since $\dd_pf$ is $L$-Lipschitz, we also get 
\[|\dd_pf(\alpha^+(0))-\dd_pf(\gamma^+(0))|\le L\cdot s.\]
It follows that the needed identity holds up to error $2\cdot L\cdot s$.
The statement follows since $s>0$ can be chosen arbitrarily.

The same argument is applicable if in the place of $(f\circ\alpha)^+(0)$
we use any limit of $\tfrac1{\eps_n}\cdot [f\circ\alpha(\eps_n)-f(p)]$ for a sequence $\eps_n\to 0+$.
It proves that all such limits coincide; in particular, $(f\circ\alpha)^+(0)$ is defined and equals to $(\dd_pf)(\alpha^+(0))$.
\qeds


\section{Gradient curves}

\begin{thm}{Definition}\label{def:grad-curve}
Let $f\:\spc{L}\subto\RR$ be a locally Lipschitz and semiconcave function on an Alexandrov space
$\spc{L}$.

A locally Lipschitz curve $\alpha\:[t_{\min},t_{\max})\to\Dom f$ will be called an \index{gradient curve}\emph{$f$-gradient curve} if
\[\alpha^+=\nabla_{\alpha} f;\]
that is, for any $t\in[t_{\min},t_{\max})$, $\alpha^+(t)$ is defined and 
$\alpha^+(t)=\nabla_{\alpha(t)} f$.
\end{thm}

A complete proof of the following theorem is given in \cite{alexander-kapovitch-petrunin2024}; 
it mimics the proof of the standard Picard theorem on the existence  and uniqueness of solutions of ordinary differential equations.
We omit the proof of existence as it is rather lengthy;
the uniqueness will be proved in the next section.


\begin{thm}{Picard theorem}\label{thm:glob-exist-grad-curv}
Let $f\:\spc{L}\subto \RR$ be a locally Lipschitz and $\lambda$-concave function on an Alexandrov space $\spc{L}$.
Then for any $p\in \Dom f$, there are unique $t_{\max}\in(0,\infty]$ and $f$-gradient curve $\alpha\:[0,t_{\max})\to \spc{L}$ with $\alpha(0)=p$ such that any sequence $t_n\to t_{\max}-$, the sequence $\alpha(t_n)$ does not have a limit point in $\Dom f$.
\end{thm}

Note that the theorem says that the future of a gradient curve is determined by its present, but it says nothing about its past.

Here is an example showing that the past is not determined by the present.
Consider the function $f\:x\mapsto -|x|$ on the real line $\RR$.
The tangent space $\T_x\RR$ can be identified with $\RR$.
Note that 
\[\nabla_xf=
\begin{cases}
1&\text{if}\quad x<0,
\\
0&\text{if}\quad x=0,
\\
-1&\text{if}\quad x>0.
\end{cases}
\]
So, the $f$-gradient curves go to the origin with unit speed and then stand there forever.
In particular, if $\alpha$ is an $f$-gradient curve that starts at $x$,
then $\alpha(t)=0$ for any $t\ge |x|$.

Here is a slightly more interesting example;
it shows that gradient curves can merge even in the region where $|\nabla f|\z\ne 0$. 


\begin{wrapfigure}[8]{r}{34 mm}
\vskip-0mm
\centering
\includegraphics{mppics/pic-1215}
\vskip0mm
\end{wrapfigure}

\begin{thm}{Example}
Consider the function $f\:(x,y)\mapsto-|x|-|y|$ on the $(x,y)$-plane.
Note that $f$ is concave;
its gradient field is sketched on the figure.

Let $\alpha$ be an $f$-gradient curve that starts at $(x,y)$ for $x>y>0$.
Then 
\[\alpha(t)=
\begin{cases}
(x-t,y-t) &\text{for}\quad 0\le t\le  x-y,
\\
(x-t,0) &\text{for}\quad x-y\le t\le  x,
\\
(0,0) &\text{for}\quad x\le t.
\end{cases}
\]

\end{thm}


\section{Distance estimates}

\begin{thm}{Observation}\label{eq:fist-var-inq+}
Let $\alpha$ be a gradient curve of a $\lambda$-concave function $f$ 
defined on an Alexandrov space.
Choose a point $p$; let $\ell(t)\df\distfun_p\circ\alpha(t)$ and $q=\alpha(t_0)$.
Then 
\[
\ell^+(t_0)\le -\left({f(p)}-{f(q)}-\tfrac\lambda2\cdot\ell^2(t_0)\right)/\ell(t_0)
\]
\end{thm}

\parit{Proof.}
Let $\gamma$ be the unit-speed parametrization of $[qp]$ from $q$ to $p$, so $q=\gamma(0)$.
Then 
\begin{align*}
\ell^+(t_0)&=(\dd_q\distfun_p)(\nabla_qf)\le
\\
&\le -\langle\dir qp,\nabla_qf\rangle \le
\\
&\le -\dd_qf(\dir qp)=
\\
&=-(f\circ\gamma)^+(0)\le
\\
&\le -\left({f(p)}-{f(q)}-\tfrac\lambda2\cdot\ell^2(t_0)\right)/\ell(t_0)
\end{align*}
In the above calculations we consequently applied
\ref{clm:fa'=dfa'},
\ref{ex:d(distfun)},
the definition of gradient,
the definition of differential,
and concavity of $t\z\mapsto f\circ\gamma(t)-\tfrac \lambda2\cdot {t^2}$.
\qeds

Note that the following estimate implies uniqueness in the Picard theorem (\ref{thm:glob-exist-grad-curv}).

\begin{thm}{First distance estimate}\label{thm:dist-est}
Let $f$ be a $\lambda$-concave locally Lipschitz function on an Alexandrov space $\spc{L}$.
Then
\[\dist{\alpha(t)}{\beta(t)}{}
\le 
e^{\lambda\cdot t}\cdot\dist[{{}}]{\alpha(0)}{\beta(0)}{}\]
for any $t\ge 0$ and any two $f$-gradient curves $\alpha$ and $\beta$.

Moreover, the statement holds for a locally Lipschitz $\lambda$-concave function defined in an open domain if there is a geodesic $[\alpha(t)\,\beta(t)]$ in $\Dom f$ for any~$t$.
\end{thm}

\parit{Proof.} 
Fix a choice of geodesic $[\alpha(t)\,\beta(t)]$ for each $t$.
Let $\ell(t)=\dist{\alpha(t)}{\beta(t)}{}$. 
Note that
\[\ell^+(t)
\le-
\<\dir{\alpha(t)}{\beta(t)},\nabla_{\alpha(t)}f\>-\<\dir{\beta(t)}{\alpha(t)},\nabla_{\beta(t)}f\>
\le
\lambda\cdot\ell(t).\]
Here one has to apply \ref{eq:fist-var-inq+} for distance to the midpoint $m$ of $[\alpha(t)\,\beta(t)]$, and then apply the triangle inequality.
Hence the result. 
\qeds



The following exercise describes a global geometric property of a gradient curve without direct reference to its function.
It uses the notion of \textit{self-contracting curves} introduced by Aris Daniilidis, Olivier Ley, and St\'ephane Sabourau \cite{daniilidis-ley-sabourau}.

\begin{thm}{Exercise}\label{ex:elf-contracting}
Let $f\:\spc{L}\subto\RR$ be a locally Lipschitz and concave function on an Alexandrov space
$\spc{L}$.
Then 
\[\dist{\alpha(t_1)}{\alpha(t_3)}{\spc{L}}\ge \dist{\alpha(t_2)}{\alpha(t_3)}{\spc{L}}.\]
for any $f$-gradient curve $\alpha$ and $t_1\le t_2\le t_3$.
\end{thm}

\begin{thm}{Exercise}\label{ex:mayer}
Let $f$ be a locally Lipschitz concave function defined on an Alexandrov space $\spc{L}$.
Suppose $\hat\alpha\:[0,\ell]\to\spc{L}$ is an arc-length reparametrization of an $f$-gradient curve.
Show that $(f\circ\hat\alpha)$ is concave.
\end{thm}




The following exercise implies that gradient curves for a uniformly converging sequence of $\lambda$-concave functions converge to the gradient curves of the limit function.

\begin{thm}{Exercise}\label{lem:fg-dist-est}
Let $f$ and $g$ be $\lambda$-concave locally Lipschitz functions on an Alexandrov space $\spc{L}$.
Suppose
$\alpha,\beta\:[0,t_{\max})\to \spc{L}$ are respectively $f$- and $g$-gradient curves.
Assume $|f-g|<\eps$; let $\ell\:t\mapsto\dist{\alpha(t)}{\beta(t)}{}$.
Show that
\[\ell^+\le \lambda\cdot\ell+\tfrac{2\cdot\eps}{\ell}.\]

Conclude that if $\alpha(0)=\beta(0)$ and $t_{\max}<\infty$, then
\[\dist{\alpha(t)}{\beta(t)}{}
\le
\Const\cdot\sqrt{\eps\cdot t}\]
for some constant $\Const=\Const(t_{\max},\lambda)$.
\end{thm}

\section{Gradient flow}

Let $\spc{L}$ be an Alexandrov space 
and $f$ be a locally Lipschitz semiconcave function defined on an open subset of $\spc{L}$.
If there is an $f$-gradient curve $\alpha$ such that $\alpha(0)=x$ and $\alpha(t)=y$,
then we will write 
\[\GF^t_f(x)=y.\]
The partially defined map $\GF^t_f$ from $\spc{L}$ to itself is called the \index{gradient flow}\emph{$f$-gradient flow} for time $t$.
Note that
\[\GF^{t_1+t_2}_f=\GF_f^{t_1}\circ\GF_f^{t_2}.\]
In other words, one may think that gradient flow is an action of the \textit{semigroup} $(\RR_{\ge0},+)$ on the space.
 
From the first distance estimate \ref{thm:dist-est}, 
it follows that for any $t\ge 0$, the domain of definition of $\GF^t_f$ is an open subset of $\spc{L}$.
In some cases, it is globally defined.
For example, if $f$ is a $\lambda$-concave function defined on the whole space $\spc{L}$, then $\GF^t_f(x)$ is defined for all $x\in \spc{L}$ and $t\ge0$;
see \cite[16.19]{alexander-kapovitch-petrunin2024}.

Now let us reformulate the statements about gradient curves obtained earlier using this new terminology.
From the first distance estimate, we have the following.

\begin{thm}{Proposition}\label{prop:GF-is-lip}
Let $\spc{L}$ be an Alexandrov space 
and $f\:\spc{L}\to \RR$ be a semiconcave function.
Then the map $x\mapsto\GF^t_f(x)$ is locally Lipschitz.

Moreover, if $f$ is $\lambda$-concave, then $\GF^t_f$ is $e^{\lambda\cdot t}$-Lipschitz.
\end{thm}

The next proposition follows from \ref{lem:fg-dist-est}.

\begin{thm}{Proposition}\label{grad-curve-conv}
Let $\spc{L}$ be an Alexandrov space.
Suppose $f_n\:\spc{L}\to\RR$ is a sequence of
$\lambda$-concave functions 
that converges to $f_\infty\:\spc{L}\to \RR$. 
Then for any $x\in \spc{L}$ and $t\ge 0$, we have
\[\GF_{f_n}^t(x)\to \GF_{f_\infty}^t(x)\]
as $n\to \infty$.
\end{thm}

%??? do we need GH-limit version???

\section{Gradient exponent}\label{gexp}

One of the technical difficulties in Alexandrov's geometry comes from
nonextendability of geodesics. 
In particular, the exponential map, $\exp_p\:\T_p\to \spc{L}$, if defined in the usual way, can
be undefined in an arbitrary small neighborhood of the origin. 

We construct its analog, the \index{gradient exponential map}\emph{gradient exponential map} 
\[\gexp_p\:\T_p\to\spc{L},\]
which essentially solves this problem. 
It shares many properties with the ordinary exponential map, and better in certain respects,
even in the Riemannian universe.

Let $\spc{L}$ be Alexandrov's space and $p\in \spc{L}$, consider the function $f\z=\distfun_p^2/2$.
Suppose $i_{\lambda}\:\lambda\cdot \spc{L}\to \spc{L}$ sends a point in the rescaled copy $\lambda\cdot\spc{L}$ to the corresponding point in $\spc{L}$.
Consider the one parameter family of maps
$$\Phi^t_{f}\circ i_{e^t}\:e^t{\cdot} \spc{L}\to \spc{L}$$
where $\Phi^t_{f}$ denotes gradient flow. 
Note that $(e^t{\cdot} \spc{L},p)\GHto (\T_p,o_p)$ as $t\to\infty$.
Let us define the \textit{gradient exponential map} as the limit
\[\gexp_p=\lim_{t\to\infty}\Phi^t_{f}\circ i_{e^t}.\]

\begin{thm}{Proposition}\label{prop:gexp}
Let $\spc{L}$ be a proper $\Alex0$ space.
Then for any $p\in \spc{L}$ the gradient exponent $\gexp_p\:\T_p\to\spc{L}$ is defined.
Moreover, $\gexp_p$ is a short map and 
\[\gexp_p(\gamma^+(0))=\gamma(1)\]
for any geodesic path $\gamma$ that starts at $p$.
\end{thm}

The last statement in the proposition says that it is appropriate to use term \textit{exponent} for $\gexp$.


\parit{Proof.} 
Note that $f''\le 1$.
By the first distance estimate, we have that $\Phi^t_{f}$ is an $e^t$-Lipschitz.
Therefore, the compositions $\Phi^t_{f}\circ i_{e^t}\:e^t{\cdot} \spc{L}\to \spc{L}$ are short. 
Hence a partial limit $\gexp_p\:\T_p
\spc{L}\to \spc{L}$ exists, and it is a short map.

Clearly for any partial limit we have
\[\Phi^t_f\circ\gexp_p(v)=\gexp_p(e^t\cdot v).\]
Since $\Phi^t$ is $e^t$-Lipschitz, it follows that $\gexp_p$ is uniquely
defined.
\qeds

\section{Remarks}

??? gradient exponent for $\kappa\ne 0$
and for nonproper.

The gradient exponential map $\gexp_p$  for a point $p$ a Riemannian manifold $(M,g)$ coincides with the Riemannian exponential map inside the cut locus of $p$ but \emph{is different } from the  Riemannian exponential outside it.

quasigeodesics

%%!TEX root = the-splitting.tex
\chapter{Line splitting}\label{chap:splitting}

\section{Busemann function}

A \index{half-line}\emph{half-line}
\footnote{\red V: we used half-line in the other book but I would still like to change this to "ray" which is the established term in literature. A: Both terms are used, since we use line is bit more  natuaral to say half-line --- but will agree to change it if you want it.}
is a distance-preserving map
from $\RR_{\ge0}=[0,\infty)$ 
to a metric space.
In other words, a half-line is a geodesic defined on the real half-line $\RR_{\ge0}$.

If $\gamma\:[0,\infty)\to \spc{X}$ is a half-line,
then the limit 
\[\bus_\gamma(x)=\lim_{t\to\infty}\dist{\gamma(t)}{x}{}- t\eqlbl{eq:def:busemann*}\]
is called the \index{Busemann function}\emph{Busemann function} of $\gamma$.

The Busemann function $\bus_\gamma$ mimics behavior of the distance function from the ideal point of $\gamma$.

\begin{thm}{Proposition}\label{prop:busemann}
For any half-line $\gamma$ in a metric space $\spc{X}$,
its Busemann function $\bus_\gamma\:\spc{X}\to \RR$ 
is defined.
Moreover, $\bus_\gamma$ is $1$-Lipschitz and $\bus_\gamma (\gamma(t))=-t$ for any $t$.

\end{thm}

\parit{Proof.}
Since $t=\dist{\gamma(0)}{\gamma(t)}{}$, the triangle inequality implies that, the function
\[t\mapsto\dist{\gamma(t)}{x}{}- t\] 
is nonincreasing, and 
\[\dist{\gamma(t)}{x}{}- t\ge-\dist{\gamma(0)}{x}{}\]
for any $x\in \spc{X}$.
Therefore, the limit in \ref{eq:def:busemann*} is defined,
and it is 1-Lipschitz as a limit of 1-Lipschitz functions.
The last statement follows since 
$\dist{\gamma(t)}{\gamma(t_0)}{}\z=t-t_0$ for all large~$t$.
\qeds

\begin{thm}{Exercise}\label{ex:busemann-CBB}
Any Busemann function on an $\Alex0$ space is concave.
\end{thm}

\section{Splitting theorem}

A \index{line}\emph{line} is a distance-preserving map
from $\RR$ to a metric space.
In other words, a line is a geodesic defined on the real line $\RR$.

\begin{thm}{Exercise}\label{ex:bus+bus}
Let $\gamma$ be a line in a metric space $\spc{X}$.
Show that for any point $x$ we have
\[\bus_+(x)+\bus_-(x)\ge 0\]
where, $\bus_+$ and $\bus_-$, are the Busemann functions asociated with half-lines $\gamma:[0,\infty)\to \spc{L}$ and $\gamma:(-\infty,0]\to \spc{L}$ respectively.
\end{thm}


Let $\spc{X}$ be a metric space and $A,B\subset \spc{X}$.
We will write 
\[\spc{X}=A\oplus B\]\index{$A\oplus B$}
if there are projections $\proj_A\:\spc{X}\to A$ 
and 
$\proj_B\:\spc{X}\to B$
such that 
\[\dist[2]{x}{y}{}=\dist[2]{\proj_A(x)}{\proj_A(y)}{}+\dist[2]{\proj_B(x)}{\proj_B(y)}{}\]
for any two points $x,y\in \spc{X}$.

Note that if 
\[\spc{X}=A\oplus B\]
then 
\begin{itemize}
\item $A$ intersects $B$ at a single point,
\item both sets $A$ and $B$ are \index{convex set}\emph{convex sets} in $\spc{X}$;
the latter means that any geodesic with the ends in $A$ (or $B$) lies in $A$ (or $B$). 
\end{itemize}

\begin{thm}{Line splitting theorem}\label{thm:splitting}
Let $\gamma$ be a line in a $\Alex0$ space~$\spc{L}$. 
Then 
\[\spc{L}=\spc{L}'\oplus \gamma(\RR)\]
for some subset $\spc{L}'\subset \spc{L}$.
\end{thm}

\begin{thm}{Corollary}\label{cor:splitting}
Any $\Alex0$ space $\spc{L}$ splits isometrically as
\[
\spc{L}=\spc{L}'\oplus H
\]
where $H\subset \spc{L}$ is a subset isometric to a Hilbert space, and $\spc{L}'\subset \spc{L}$ is a convex subset that contains no lines. 
\end{thm}

The following lemma is closely related to the first distance estimate (\ref{thm:dist-est});
it is also a limit case of \ref{prop:gexp}.
The proof goes along the same lines.

\begin{thm}{Lemma}\label{lem:dist-estimate}
Suppose $f\:\spc{L}\to\RR$ be a concave 1-Lipschitz function on an $\Alex0$ space $\spc{L}$.
Consider two $f$-gradient curves $\alpha$ and~$\beta$.
Then for any $t, s\ge 0$ we have
\begin{align*}
&\dist[2]{\alpha(s)}{\beta(t)}{}
\le 
\dist[2]{p}{q}{}+
2\cdot(f(p)-f(q))\cdot(s-t)+ (s-t)^2,
\end{align*}
where $p=\alpha(0)$ and $q=\beta(0)$.
\end{thm}

\parit{Proof.}
Since $f$ is 1-Lipschitz, $|\nabla f|\le1$.
Therefore 
\[f\circ\beta(t)\le f(q)+t\]
for any $t\ge0$.

Set $\ell(t)=\dist{p}{\beta(t)}{}$.
Applying \ref{eq:fist-var-inq+}, we get
\begin{align*}
(\ell^2)^+(t)
&\le 2\cdot \left(f\circ\beta(t)-f(p)\right)\le 
\\
&\le2\cdot\left(f(q)+t-f(p)\right).
\end{align*}
Therefore 
\[\ell^2(t)-\ell^2(0)\le 2\cdot\left(f(q)-f(p)\right)\cdot t + t^2.\]
It proves the needed inequality in case $s=0$.
Combining it with the first distance estimate (\ref{thm:dist-est}), we get the result in case $s\le t$.
The case $s\ge t$ follows by switching the roles of $s$ and $t$.
\qeds


\parit{Proof of \ref{thm:splitting}.} Consider two Busemann functions, $\bus_+$ and $\bus_-$, asociated with half-lines $\gamma:[0,\infty)\to \spc{L}$ and $\gamma:(-\infty,0]\to \spc{L}$ respectively; that is,
\[
\bus_\pm(x)
\df
\lim_{t\to\infty}\dist{\gamma(\pm t)}{x}{}- t.
\]
According to \ref{ex:busemann-CBB}, 
both $\bus_+$ and $\bus_-$ are concave.

By \ref{ex:bus+bus}, $\bus_+(x)+\bus_-(x)\ge0$ for any $x\in \spc{L}$.
On the other hand, by \ref{comp-kappa}, 
$f(t)=\distfun_x^2\circ\gamma(t)$ 
is $2$-concave.
In particular, $f(t)\le t^2+at+b$ for some constants $a,b\in\RR$.  Therefore, for all large $t$
\[
\dist{\gamma( t)}{x}{}- t +\dist{\gamma(- t)}{x}{}- t\le \sqrt{ t^2+at+b}-t+\sqrt{ t^2-at+b}-t
\]

Passing to the limit as $t\to\infty$, we get that  $\bus_+(x)+\bus_-(x)\le 0$.
Hence
\[
\bus_+(x)+\bus_-(x)= 0
\]
for any $x\in \spc{L}$.
In particular, the functions $\bus_+$ and $\bus_-$ are \index{affine function}\emph{affine};
that is, they are convex and concave at the same time.

Note that for any $x$,
\begin{align*}
|\nabla_x \bus_\pm|
&=\sup\set{\dd_x\bus_\pm(\xi)}{\xi\in\Sigma_x}=
\\
&=\sup\set{-\dd_x\bus_\mp(\xi)}{\xi\in\Sigma_x}\equiv
\\
&\equiv1.
\end{align*}

Observe that $\alpha$ is a $\bus_\pm$-gradient curve
if and only if $\alpha$ is a geodesic such that $(\bus_\pm\circ\alpha)^+=1$.
Indeed, if $\alpha$ is a geodesic, then $(\bus_\pm\circ\alpha)^+\le 1$ and the equality holds only if $\nabla_\alpha\bus_\pm=\alpha^+$.
Now suppose $\nabla_\alpha\bus_\pm=\alpha^+$.
Then $|\alpha^+|\le 1$ and $(\bus_\pm\circ\alpha)^+=1$; therefore 
\begin{align*}
|t_0-t_1|&\ge \dist{\alpha(t_0)}{\alpha(t_1)}{}\ge
\\
&\ge|\bus_\pm\circ\alpha(t_0)-\bus_\pm\circ\alpha(t_1)=
\\
&=|t_0-t_1|.
\end{align*}

It follows that for any $t>0$, the $\bus_\pm$-gradient flows commute;
that is, 
\[\GF_{\bus_+}^t\circ\GF_{\bus_-}^t=\id_\spc{L}.\]
Setting
\[\GF^t=\left[\begin{matrix}
\GF_{\bus_+}^t&\hbox{if}\ t\ge0\\
\GF_{\bus_-}^{-t}&\hbox{if}\ t\le0
               \end{matrix}\right.\]
defines an $\RR$-action on $\spc{L}$.

Consider the level set $\spc{L}'=\bus_+^{-1}(0)=\bus_-^{-1}(0)$;
it is a closed convex subset of $\spc{L}$, and therefore forms an Alexandrov space.
Consider the map $h\:\spc{L}'\times \RR\to \spc{L}$ defined by $h\:(x,t)\mapsto \GF^t(x)$.
Note that $h$ is onto.
Applying \ref{lem:dist-estimate} for $\GF_{\bus_+}^t$ and $\GF_{\bus_-}^t$ shows that $h$ is distance non-expanding and non-contracting at the same time; that is, $h$ is an isometry.
\qeds

Recall that according our definition the real line $\RR$ is $\Alex1$.
However, most of $\Alex1$ spaces have diameter at most $\pi$;
see \ref{ex:RisCBB(1)}.

\begin{thm}{Exercise}\label{ex:cone-CBB}
Suppose $\spc{X}$ is a complete geodesic space.
Show that $\Cone\spc{X}$ is $\Alex0$ if and only if $\spc{X}$ is $\Alex1$ and $\diam\spc{X}\le \pi$.
\end{thm}

\section{Anti-sum}

Here we give a corollary of \ref{ex:convergence-grad}.
It will be used to prove basic properties of the tangent space.


\begin{thm}{Anti-sum lemma}\label{lem:minus-sum} 
Let $\spc{L}$ be an Alexandrov space and $p\in \spc{L}$.

Given two vectors $u,v\in \T_p$, there is a unique vector $w\in \T_p$ such that
\[\langle u,x\rangle +\langle v,x\rangle +\langle w,x\rangle \ge 0\]
for any $x\in \T_p$, and
\[\langle u,w\rangle +\langle v,w\rangle +\langle w,w\rangle =0.\]

\end{thm}

\begin{thm}{Exercise}\label{ex:|antisum|}
Suppose $u,v, w\in \T_p$ are as in \ref{lem:minus-sum}.
Show that 
\[|w|^2\le |u|^2+|v|^2+2\cdot\langle u,v\rangle.\]

\end{thm}

If $\T_p$ were geodesic, then the lemma would follow from the existence  of the gradient, applied to the function $\T_p\to \RR$ defined by $x\mapsto -(\langle u,x\rangle +\langle v,x\rangle )$ which is concave.
However, the tangent space $\T_p$ might fail to be geodesic; see  Halbeisen's example \cite{alexander-kapovitch-petrunin2024}.

Applying the above lemma for $u=v$, we have the following statement.

\begin{thm}{Existence of polar vector}\label{cor:polar}
Let $\spc{L}$ be an Alexandrov space 
and $p\in \spc{L}$. 
Given a vector $u\in \T_p$,  there is a unique vector $u^*\in\T_p$ such that $\langle u^*,u^*\rangle +\langle u,u^*\rangle = 0$ and
$u^*$ is \index{polar vectors}\emph{polar} to $u$;
that is,
\[\langle u^*,x\rangle +\langle u,x\rangle \ge 0\]
for any $x\in \T_p$.
\end{thm}

\parit{Proof of \ref{lem:minus-sum}.}
By \ref{ex:d(distfun):==}, we can choose two sequences of points $a_n,b_n$ such that 
\begin{align*}
\dd_p\distfun_{a_n}(w)&=-\langle\dir{p}{a_n},w\rangle
\\
\dd_p\distfun_{b_n}(w)&=-\langle\dir{p}{b_n},w\rangle
\end{align*}
for any $w\in\T_p$ and $\dir{p}{a_n}\to u/|u|$, $\dir{p}{b_n}\to v/|v|$ as $n\to \infty$

Consider a sequence of functions 
\[f_n=|u|\cdot\distfun_{a_n}+|v|\cdot\distfun_{b_n}.\]
Note that 
\[(\dd_pf_n)(x)=-|u|\cdot\langle \dir{p}{a_n},x\rangle -|v|\cdot\langle \dir{p}{b_n},x\rangle .\]
Thus we have the following uniform convergence for $x\in\Sigma_p$:
\[(\dd_pf_n)(x)\to-\langle u,x\rangle -\langle v,x\rangle \]
as $n\to\infty$,
According to \ref{ex:convergence-grad}, 
the sequence $\nabla_pf_n$ converges.
Let 
\[w=\lim_{n\to\infty}\nabla_pf_n.\]
By the definition of gradient,
\[\begin{aligned}
\langle w,w\rangle &=\lim_{n\to\infty}\langle \nabla_pf_n,\nabla_pf_n\rangle =
&&&%right side
\langle w,x\rangle &=\lim_{n\to\infty}\langle \nabla_pf_n,x\rangle \ge
\\%second line
&=\lim_{n\to\infty}(\dd_p f_n)(\nabla_p f_n)
=
&&&%second line right side
&\ge
\lim_{n\to\infty}(\dd_pf_n)(x)
=
\\%line 3
&=-\langle u,w\rangle -\langle v,w\rangle ,
&&&%line 3 right side
&=-\langle u,x\rangle -\langle v,x\rangle .
\end{aligned}\]
\qedsf

\section{Linear subspace}

\begin{thm}{Definition}\label{def:opp+Lin}
Let $\spc{L}$ be an Alexandrov space, $p\in \spc{L}$ and $u,v\in\T_p$.
We say that vectors $u$ and $v$ are \index{opposite vectors}\emph{opposite}\label{def:opposite:page} to each other, (briefly, $u+v=0$) if $|u|=|v|=0$ or $\mangle(u,v)=\pi$ and $|u|=|v|$.

The subcone
\[\Lin_p=\set{v\in\T_p}{\exists\ w\in\T_p\quad \text{such that}\quad w+v=0}\]
will be called the \index{linear subspace}\emph{linear subspace} of $\T_p$.
\end{thm}

Soon we will introduce a natural linear structure on $\Lin_p$.

\begin{thm}{Proposition}\label{prop:opposite}
Let $\spc{L}$ be an Alexandrov space and $p\in \spc{L}$.
Given two vectors $u,v\in\T_p$, the following statements are equivalent:
\begin{subthm}{opposite} $u+v=0$;
\end{subthm}
\begin{subthm}{<x,u>} $\langle u,x\rangle +\langle v,x\rangle =0$ for any $x\in\T_p$;
\end{subthm}
\begin{subthm}{<xi,u>} $\langle u,\xi\rangle +\langle v,\xi\rangle =0$ for any $\xi\in\Sigma_p$.
\end{subthm}
\end{thm}

\parit{Proof.}
The equivalence  \ref{SHORT.<x,u>}$\Leftrightarrow$\ref{SHORT.<xi,u>} is trivial.

The condition $u+v=0$ is equivalent to 
$\langle u,u\rangle =-\langle u,v\rangle =\langle v,v\rangle$;
thus,
\ref{SHORT.<x,u>}$\Rightarrow$\ref{SHORT.opposite}.

Recall that $\T_p$ has nonnegative curvature.
Note that the hinges $\hinge 0ux$ and $\hinge 0vx$ are adjacent.
By \ref{ex:adjacent-CBB}, $\mangle\hinge 0ux+\mangle\hinge 0vx=\pi$;
hence \ref{SHORT.opposite}$\Rightarrow$\ref{SHORT.<x,u>}.
\qeds

\begin{thm}{Exercise}\label{prop:two-opp}
Let $\spc{L}$  be an Alexandrov space and $p\in \spc{L}$.
Then for any three vectors $u,v,w\in\T_p$, if $u+v=0$ and $u+ w=0$ then $v=w$.
\end{thm}

Let $u\in \Lin_p$; that is, $u+v=0$ for some $v\in\T_p$.
Given $s<0$, let 
\[s\cdot u\df (-s)\cdot v.\]
So we can multiply any vector in $\Lin_p$ by any real number (positive and negative).
By \ref{prop:two-opp}, this multiplication is uniquely defined;
by \ref{prop:opposite}; we have identity
\[\langle -v,x\rangle=-\langle v,x\rangle.\]


\begin{thm}{Exercise}\label{ex:3<,>=0}
Suppose $u,v,w\in\T_p$ are as in \ref{lem:minus-sum}.
Show that
\[\langle u,x\rangle +\langle v,x\rangle +\langle w,x\rangle = 0\]
for any $x\in \Lin_p$.
\end{thm}

\begin{thm}{Exercise}\label{ex:-u}
Let $\spc{L}$ be an Alexandrov space,
$p\in \spc{L}$ and $u\in \T_p$.
Suppose $u^*\in \T_p$ is provided by \ref{cor:polar};
that is, 
\[\langle u^*,u^*\rangle +\langle u,u^*\rangle = 0
\quad\text{and}\quad
\langle u^*,x\rangle +\langle u,x\rangle \ge 0
\]
for any $x\in \T_p$.
Show that 
\[u=-u^*\quad\Longleftrightarrow\quad|u|=|u^*|.\]
\end{thm}

\begin{thm}{Theorem}\label{thm:lin-subcone}
Let $p$ be a point in an Alexandrov space. 
Then $\Lin_p$ is isometric to a Hilbert space.
\end{thm}

\parit{Proof.}
Note that $\Lin_p$ is a closed subset of $\T_p$;
in particular, it is complete.

If any two vectors in $\Lin_p$ can be connected by a geodesic in $\Lin_p$,
then the statement follows from the splitting theorem (\ref{thm:splitting}).
By Menger's lemma (\ref{lem:mid>geod}), it is sufficient to show that for any two vectors $x,y\in\Lin_p$
there is a midpoint $w\in \Lin_p$.

Choose $w\in \T_p$ to be the anti-sum of $u=-\tfrac{1}{2}\cdot x$ and $v=-\tfrac{1}{2}\cdot y$;
see \ref{lem:minus-sum}.
By \ref{ex:|antisum|} and \ref{ex:3<,>=0},
\begin{align*}
|w|^2&\le \tfrac14\cdot |x|^2+\tfrac14\cdot|y|^2+\tfrac12\cdot\langle x,y\rangle,
\\
\langle w,x\rangle&= \tfrac12\cdot|x|^2+\tfrac12\cdot\langle x,y\rangle,
\\
\langle w,y\rangle&= \tfrac12\cdot|y|^2+\tfrac12\cdot\langle x,y\rangle,
\end{align*}
It follows that 
\begin{align*}
|x-w|^2
&= |x|^2+|w|^2-2\cdot\langle w,x\rangle\le
\\
&\le \tfrac14\cdot |x|^2+\tfrac14\cdot|y|^2-\tfrac12\cdot\langle x,y\rangle=
\\
&=\tfrac14\cdot|x-y|^2.
\end{align*}
That is, $|x-w|\le \tfrac12\cdot|x-y|$.
Similarly, we get $|y-w|\le \tfrac12\cdot|x-y|$.
Therefore $w$ is a midpoint of $x$ and $y$.
In addition we get equality 
\[|w|^2= \tfrac14\cdot |x|^2+\tfrac14\cdot|y|^2+\tfrac12\cdot\langle x,y\rangle.\]

It remains to show that $w\in\Lin_p$.
Let $w^*$ be the polar vector provided by \ref{cor:polar}.
Note that 
\[|w^*|\le |w|,
\quad
\langle w^*,x\rangle+\langle w,x\rangle=0,
\quad
\langle w^*,y\rangle+\langle w,y\rangle=0.
\]
The same calculation as above shows that $w^*$ is a midpoint of $-x$ and $-y$ and 
\[|w^*|^2= \tfrac14\cdot |x|^2+\tfrac14\cdot|y|^2+\tfrac12\cdot\langle x,y\rangle=|w|^2.\]
By \ref{ex:-u}, $w=-w^*$;
hence $w\in\Lin_p$.
\qeds

\begin{thm}{Lemma}\label{ex:grad-dist:G-delta}
Given a point $p$ in an Alexandrov space $\spc{L}$,
let $f\z=\distfun_p$, and let $S$ be the subset of points $x\in\spc{L}$ such that $|\nabla_xf|=1$.
Then $S$ is a dense G-delta set.

\end{thm}

\parit{Proof.}
Let $S_n\subset \spc{L}$ be defined by inequality $|\nabla_xf|>1-\tfrac1n$.
By \ref{ex:semicontinuous-grad:>s}, $S_n$ is open.

Choose a point $q\ne p$.
Observe that $|\nabla_xf|=1$ for any point $x\in\left]pq\right[$.
It follows that $S_n$ is dense in $\spc{L}$.

Since $S=\bigcap_nS_n$, the lemma follows.
\qeds


\begin{thm}{Exercise}\label{ex:grad-dist}
Let $p$, $f$, and $S$ be as in \ref{ex:grad-dist:G-delta}.

\begin{subthm}{ex:grad-dist:lin}
Show that 
\[\nabla_xf+\dir xp=0\]
for any 
$x\in S$;
in particular, $\dir xp\in \Lin_x$.
\end{subthm}

\begin{subthm}{ex:grad-dist:|grad|=1}
Show that if $|\nabla_xf|=1$, then $\dd_xf(w)= \langle\nabla_xf,w\rangle$ for any $w\in \T_x$.
\end{subthm}
\end{thm}

Note that \ref{ex:grad-dist} implies the following.

\begin{thm}{Corollary}\label{cor:euclid-subcone}
Given a countable set of points $X$ in an Alexandrov space $\spc{L}$
there is a G-delta dense set $S\subset\spc{L}$
such that 
$\dir sx\in \Lin_s$
for any $s\in S$ and $x\in X$.
\end{thm}

\section{Comments}

The splitting theorem has an interesting history that starts with Stefan Cohn-Vossen \cite{cohn-vossen_line};
who proved its $2$-dimensional case.
For Riemannian manifolds of higher dimensions 
it was proved by Victor Toponogov \cite{toponogov-globalization+splitting}.
Then it was generalized by Anatoliy Milka \cite{milka-line}
to Alexandrov spaces;
historically, it was the first result about Alexandrov spaces of dimension higher than 2.
Nearly the same proof is used in \cite[1.5]{burago-burago-ivanov}.

Further generalizations of the splitting theorem for Riemannian manifolds with nonnegative Ricci curvature were obtained by Jeff Cheeger and Detlef Gromoll \cite{cheeger-gromoll-split}.
This was further generalized by Jeff Cheeger and Toby Colding for limits of Riemannian manifolds with almost nonnegative Ricci curvature \cite{cheeger-colding-alm-rigidity} and to their synthetic generalizations, so-called {}\emph{RCD spaces}, by Nicola Gigli \cite{gigli2013splitting, gigli-splitting-overview}.
Jost-Hinrich Eschenburg obtained an analogous result for  Lorentzian manifolds \cite{eshenburg-split}, that is, pseudo-Riemannian manifolds of signature $(1,n)$.

The presented proof is close in spirit to the proof given by Cheeger and Gromoll \cite{cheeger-gromoll-split};
it is taken from our book \cite{alexander-kapovitch-petrunin2024}.

\begin{thm}{Open question}
Let $p$ be a point in an Alexandrov space $\spc{L}$.
Suppose that $0\ne v\in \Lin_p$.
Is it true that the tangent space $\T_p$ splits in the direction of $v$?
\end{thm}

Halbeisen's example \cite{alexander-kapovitch-petrunin2024,halbeisen} shows that compactness of space of directions is essential in the proof that space of directions is $\pi$-geodesic (see \ref{thm:finite-space-of-directions}).

\begin{thm}{Open question}\label{open:Halb-proper}
Let $\spc{L}$ be a proper Alexandrov space.
Is it true that for any $p\in \spc{L}$, the tangent space $\T_p$ is a length space?
\end{thm}

%%%%%%%%%%%%%%%%%%%%%%%%%%%%%%%%%%%%%%%%%%%%%%%%%%

\chapter{Dimension and volume}\label{chap:dim}

\section{Linear dimension}

Let $\spc{L}$ be an Alexandrov space.
Let us define its \index{linear dimension}\emph{linear dimension} \index{$\LinDim \spc{L}$}$\LinDim \spc{L}$ as the least upper bound on integers $m$ such that 
the Euclidean space $\EE^m$ is isometric to a subspace of the tangent space $\T_p\spc{L}$ at some point $p\in \spc{L}$.
If not stated otherwise, dimension of an Alexandrov space is its linear dimension.

If not stated otherwise, dimension will mean linear dimension.
In Section~\ref{sec:all-dim}, we will show that linear dimension of Alexandrov space coincides with all reasonable dimensions;
after that we will use \index{$\dim \spc{L}$}$\dim$ for $\LinDim$.

\begin{thm}{(\textit{n}+1)-comparison}
Let $\spc{L}$ be an $\Alex0$ space.
Then for any finite set of points $p,x_1,\dots,x_n\in \spc{L}$, there is a model configuration 
$\tilde p,\tilde x_1,\dots,\tilde x_n\in \EE^m$ such that 
\[|\tilde p-\tilde x_i|_{\EE^m}=| p- x_i|_{\spc{L}}
\quad\text{and}\quad
|\tilde x_i-\tilde x_j|_{\EE^m}\ge |x_i- x_j|_{\spc{L}}\]
for any $i$ and $j$.
Moreover, we can assume that $m\le \LinDim\spc{L}$. 
\end{thm}

\parit{Proof.}
By \ref{cor:euclid-subcone}, we can choose a point $p'$ arbitrarily close to $p$ so that 
$\Lin_{p'}\ni \dir{p'}{x_i}$ for any $i$.
Let us identify $\EE^m$ with a subspace of $\Lin_{p'}$ spanned by $\dir{p'}{x_1},\dots,\dir{p'}{x_n}$.
Note that $m\le \LinDim\spc{L}$.

Set $\tilde p'=0\in \EE^m$ and $\tilde x_i=\dist{p'}{x_n}{}\cdot\dir{p'}{x_n}\in \EE^m$ for every $i$.
Note that 
\[|\tilde p'-\tilde x_i|_{\EE^m}=| p'- x_i|_{\spc{L}}\]
for every $i$.
Applying the comparison $\mangle\hinge {p'}{x_i}{x_j}\ge \angk {p'}{x_i}{x_j}$, we get
\[|\tilde x_i-\tilde x_j|_{\EE^m}\ge |x_i- x_j|_{\spc{L}}\]
for any $i$ and $j$.
Passing to a limit configuration as $p'\to p$ we get the result.
\qeds

\begin{thm}{Exercise}\label{ex:tangent=Em}
Let $\spc{L}$ be an $\Alex0$ space.
Suppose $\LinDim\spc{L}\z=m<\infty$.
Show that $\T_p\spc{L}\iso \EE^m$ for a G-delta dense set of points $p\in\spc{L}$.
\end{thm}

\begin{thm}{Exercise}\label{ex:dim=1}
Show that a 1-dimensional Alexandrov space is homeomorphic to a 1-dimensional manifold, possibly with nonempty boundary.
\end{thm}


\begin{thm}{Exercise}\label{ex:resporka}
Let $\spc{L}$ be an $\Alex0$ space.

Show that $\LinDim \spc{L}\ge m$ if and only if for some $m+2$ points $p$, $a_0,\z\dots, a_{m}\in \spc{L}$
we have
\[\angk p{a_i}{a_j}>\tfrac\pi2\]
for any pair $i\ne j$.
\end{thm}

\section{Space of directions}

A metric space $\spc{X}$ will be called $\ell$-geodesic 
if any two points $x,y\in\spc{X}$ such that $\dist{x}{y}{}<\ell$ can be connected by a geodesic.
For instance, any geodesic space is $\infty$-geodesic.

\begin{thm}{Theorem}\label{thm:finite-space-of-directions}
Let $\spc{L}$ be a finite-dimensional Alexandrov space.
Then for any point $p\in \spc{L}$, its space of directions $\Sigma_p$ is a compact $\pi$-geodesic space.
\end{thm}


\begin{thm}{Exercise}\label{ex:finite-tan}
Let $p$ be a point in a finite-dimensional Alexandrov space $\spc{L}$.
Prove the following.
\begin{subthm}{ex:finite-tan:tan}
The tangent space $\T_p$ is a proper $\Alex0$ space.
\end{subthm}

\begin{subthm}{ex:finite-space-of-directions-dim}
$\LinDim\Sigma_p=\LinDim\spc{L}-1$.
\end{subthm}

\begin{subthm}{ex:finite-tan:sigma}
If $\LinDim \spc{L}>1$, then $\Sigma_p$ is geodesic.
\end{subthm}


\end{thm}

Using \ref{ex:finite-space-of-directions-dim}, one can prove results for all finite dimensional Alexandrov spaces via induction on  dimension.
Such proofs will be indicated below.

\parit{Proof of \ref{thm:finite-space-of-directions}.}
Choose $\eps>0$; suppose $\spc{L}$ is $m$-dimensional.
Assume can choose $n$ directions $\xi_1,\dots, \xi_n\in \Sigma_p$ such that $\mangle(\xi_i,\xi_j)\z>\eps$ for any $i\ne j$.
Without loss of generality, we may assume that each direction is geodesic;
that is, there is a point $x_i\in \spc{L}$ such that $\xi_i=\dir p{x_i}$.

Choose $y_i\in [px_i]$ such that $\dist{p}{y_i}{}=r$ for each $i$ and small fixed $r>0$.
Since $r$ is small, we can assume that $\angk p{y_i}{y_j}>\eps$ for any $i\ne j$.
By \ref{cor:euclid-subcone}, we can choose $p'$ arbitrarily close to $p$ such that $\dir{p'}{y_i}\in \Lin_{p'}$ for any $i$.
Since  $\dist{p'}{p}{}$ is small, $\angk {p'}{y_i}{y_j}>\eps$ for any $i\ne j$.
By comparison, 
\[\mangle \hinge{p'}{y_i}{y_j}>\eps.\]
Therefore $n\le \pack_\eps\SSS^{m-1}$,
where \index{$\pack_\eps\spc{X}$}$\pack_\eps\spc{X}$ is the exact upper bound on the number of points $x_1,\z\dots,x_k\in \spc{X}$ such that $\dist{x_i}{x_j}{}\ge\eps$ if $i\ne j$.

Since $\SSS^{m-1}$ is compact, $\pack_\eps\SSS^{m-1}<\infty$.
By the definition, the space of directions $\Sigma_p$ is complete. 
Applying \ref{ex:pack-net}, we get that  $\Sigma_p$ is compact.

It remains to prove the following claim.

\begin{clm}{}
If $\Sigma_p$ is compact, then it is $\pi$-geodesic
\end{clm}

Choose two geodesic directions $\xi=\dir px$ and $\zeta=\dir py$;
let 
\[\alpha\z=\tfrac12\cdot \mangle \hinge pxy=\tfrac12\cdot \dist{\xi}{\zeta}{\Sigma_p}.\]

Suppose $\alpha<\pi/2$.
Let us show that it is sufficient to construct an \index{almost midpoint}\emph{almost midpoint} $\mu\z=\dir pz$ of $\xi$ and $\zeta$ in $\Sigma_p$;
that is, we need to show that for any $\eps>0$ there is a geodesic $[pz]$ such that
\[\mangle\hinge pxz\le \alpha+\eps
\quad\text{and}\quad
\mangle\hinge pyz\le \alpha+\eps.\]
Indeed, once it is done, the compactness of $\Sigma_p$ can be used to get an actual midpoint for any two directions in $\Sigma_p$.
After that Menger's lemma (\ref{lem:mid>geod}) will finish the proof.

Choose a sequence of small positive numbers $r_n\to0$
Consider sequnces $x_n\z\in [px]$ and $y_n\z\in [py]$ such that $\dist{p}{x_n}{}=\dist{p}{y_n}{}=r_n$.
Let $m_n$ be a midpoint of $[x_n\,y_n]$.
%??? we use here that the directions $\xi=\dir px$ and $\zeta=\dir py$ are not opposite???

Since $\Sigma_p$ is compact, we can pass to a sequence of $r_n$ such that 
$\dir{p}{m_n}$ converges;
denote its limit by $\mu$.
Choose a geodesic $[pz]$ that runs at small angle from $\mu$.
Let us show that $\dir pz$ is the needed almost midpoint.

Evidently, $\angk p{x_n}{m_n}=\angk p{y_n}{m_n}$.
By \ref{ex:alex-lemma-cat}, we have
\[\angk p{x_n}{m_n}+\angk p{y_n}{m_n}\le \angk p{x_n}{y_n}.\]

Let $z_n\in [pz]$ be the point such that $\dist{p}{z_n}{}=\dist{p}{m_n}{}$.
By construction, for all large $n$, we have $\mangle\hinge pz{m_n}\approx0$  with arbitrary small given error.
By comparison, the value $\frac{\dist{z_n}{m_n}{}}{\dist{p}{z_n}{}}$ can be assumed to be arbitrary small for all large $n$.
Applying this observation and the definition of angle measure, we also have the following approximations
\begin{align*}
\angk p{x_n}{y_n}&\approx \mangle\hinge p{x_n}{y_n},
\\
\angk p{x_n}{m_n}\approx\angk p{x_n}{z_n}&\approx\mangle\hinge p{x_n}{z_n},
\\
\angk p{m_n}{y_n}\approx\angk p{z_n}{y_n}&\approx\mangle\hinge p{z_n}{y_n},
\end{align*}
again, with arbitrary given error and all large $n$.
It follows that $\dir pz$ is an almost midpoint of $\dir px$ and $\dir py$, as required.
\qeds

In the above proof, the angles $\mangle\hinge pxz$ and $\mangle\hinge pyz$ have lower bounds by 
the comparison, but we needed upper bounds that were extracted from the definition of angle measure and compactness of space of directions.

\section{Right-inverse theorem}

\begin{thm}{Theorem}\label{thm:right-inverse}
Suppose $p,a_0,\dots,a_m$ be points in an Alexandrov space $\spc{L}$ such
\[\angk p{a_i}{a_j}>\tfrac\pi2\]
for any $i\ne j$.
Then the map $f\:\spc{L}\to\RR^m$ defined by
\[f\:x\mapsto (\dist{a_1}{x}{},\dots,\dist{a_m}{x}{})\]
has a left inverse defined in a neighborhood of $f(p)$.
\end{thm}

In the proof we construct a local right inverse $\map$ of $f$ around $f(p)$.
The construction uses gradient flow for suitably chosen family of functions.
The structure of the proof can be seen in the following exercise,
more details are given in the hints.

\begin{thm}{Exercise}\label{ex:proof-right-inverse}
Suppose $p,a_0,\dots,a_m\in\spc{L}$ and $f\:\spc{L}\to\RR$ are as in \ref{thm:right-inverse}.
Assume $\eps>0$ is sufficiently small.
Given $\bm{y}=(y_1,y_2,\dots,y_m)\in \RR^m$, 
consider the function on $\spc{L}$ defined by
\[f_{\bm{y}}(x)=\min\{\,0, \dist{a_1}{x}{}-y_1,\dots,\dist{a_m}{x}{}-y_m\,\}+\eps\cdot\dist{a_0}{x}{}.\]

\begin{subthm}{ex:proof-right-inverse:grad}
There is $r>0$ such that 
Show that $f_{\bm{y}}$ is $\lambda$-concave in $\oBall(p,r)$ for some $\lambda$ and
\begin{enumerate}[(i)]
\item\label{111} $(\dd_x\distfun_{a_i})(\nabla_x f_{\bm{y}})<-\tfrac{1}{10}\cdot\eps^2$ if $\dist{a_i}{x}{}>y_i$ and
\item\label{222} $(\dd_x\distfun_{a_i})(\nabla_x f_{\bm{y}})>\tfrac{1}{10}\cdot\eps^2$ if 
\[\dist{a_i}{x}{}-y_i=\min_j\{\dist{a_j}{x}{}\z-y_j\}<0.\]
\end{enumerate}
for any $x\in \oBall(p,r)$.

\end{subthm}

\begin{subthm}{ex:proof-right-inverse:alpha}
Let $\alpha_{\bm{y}}$ be $f_{\bm{y}}$-gradient curve that starts at $p$.
Use \ref{SHORT.ex:proof-right-inverse:grad} to show that 
if for some $\bm{y}\in\RR^m$ and $t_0\le\tfrac{r}{2}$ we have
$|\distfun_{\bm{a}}{p}-\bm{y}|
\le
\tfrac{\eps^2}{10}\cdot t_0$, then 
$
\distfun_{\bm{a}}{[\alpha_{\bm{y}}(t_0)]}
= 
\bm{y}$.
\end{subthm}

\begin{subthm}{ex:proof-right-inverse:end}
Let $t_0(\bm{y})=\tfrac{10}{\eps^2}\cdot|\dist{\bm{a}}{p}{}-\bm{y}|$.
Use \ref{lem:fg-dist-est} to show that the map
\[\map\:{\bm{y}}\mapsto \alpha_{\bm{y}}\circ t_0(\bm{y})\]
continuous in $\Omega=\oBall(\dist{\bm{a}}{p}{},\tfrac{\eps^2\cdot r}{20} )\subset\RR^m$
and $f\circ \Phi(\bm{y})=\bm{y}$ for any $\bm{y}\in \Omega$.
This finishes the proof of \ref{thm:right-inverse}.
\end{subthm}

\end{thm}

%??? I think that since this is used later it should be proved and not left as a reference A: If we add a solution, then that is OK, is not it? in any case, the idea is more tranparent in the exercise and if needed one can read the solution. But lets do the real solution, not just a hint.

\section{Distance chart}

\begin{thm}{Theorem}\label{thm:dist-chart}
Suppose $p,a_0,\dots,a_m$ be points in an $m$-dimensional Alexandrov space $\spc{L}$ such
\[\angk p{a_i}{a_j}>\tfrac\pi2\]
for any $i\ne j$.
Then the map $f\:\spc{L}\to\RR^m$ defined by
\[f\:x\mapsto (\dist{a_1}{x}{},\dots,\dist{a_m}{x}{})\]
gives a bi-Lipschitz embedding of a neighborhood $\Omega$ of $p$;
the restriction $f|_\Omega$ is called \emph{distance chart} at $p$.
\end{thm}

The following exercise guides you to prove the theorem.

\begin{thm}{Exercise}\label{ex:proof-dist-chart}
Suppose $p,a_0,\dots,a_m\in\spc{L}$ and $f\:\spc{L}\to\RR$ are as in \ref{thm:right-inverse}.
Show that there is $\eps>0$ such that one of the following $m$ inequalities hold
\begin{align*}
\mangle\hinge xy{a_1}&<\tfrac\pi2-\eps,\ \dots,\  \mangle\hinge xy{a_m}<\tfrac\pi2-\eps,
\\
\mangle\hinge yx{a_1}&<\tfrac\pi2-\eps,\ \dots,\ \mangle\hinge yx{a_m}<\tfrac\pi2-\eps
\end{align*}
for any two points $x,y$ in a sufficiently small neighborhood of $p$.
Use it to prove \ref{thm:dist-chart}.
\end{thm}

\section{Volume}

Fix a positive integer $m$.
The $m$-dimensional Hausdorff measure of a Borel set $B$ in a metric space will be called its \index{volume}\emph{$m$-volume}; it will be denoted by $\vol_m B$.
We assume that the Hausdorff measure is calibrated so that the unit cube in $\EE^m$ has unit volume.

This definition will be applied mostly to subsets in $m$-dimensional Alexandrov spaces.
In this case, we may write $\vol B$ instead of $\vol_m B$.


\begin{thm}{Bishop--Gromov inequality}\label{inq:BG}
Let $\spc{L}$ be an $\Alex0$ space.
Suppose $\dim \spc{L}=m<\infty$.
Then 
\[\vol \oBall(p,r)\le \omega_m\cdot r^m,\]
where $\omega_m$ denotes the volume of the unit ball in $\EE^m$.
Moreover, the function 
\[r\mapsto \frac{\vol B(p,r)}{r^m}\]
is nonincreasing.
\end{thm}

\parit{Proof.}
Given $x\in\spc{L}$ choose a geodesic path $\gamma_x$ from $p$ to $x$.
Recall that $\log_p\:\spc{L}\to \T_p$ can be defined by $\log_p\:x\mapsto \gamma_x^+(0)$.
By comparison, $\log_p$ is distance-noncontracting.
Note that $\log_p$ maps $\oBall(p,r)_{\spc{L}}$ to $\oBall(0,r)_{\T_p}$.

\begin{wrapfigure}{r}{44 mm}
\vskip-0mm
\centering
\includegraphics{mppics/pic-803}
\vskip1mm
\end{wrapfigure}

If $\T_p\iso \EE^m$, then $\vol\oBall(0,r)_{\T_p}\z=\omega_m\cdot r^m$,
and the first statement follows.

If $\T_p$ is not isometric to $\EE^m$, then by \ref{ex:tangent=Em}, we can find a point $p'$ arbitrarily close to $p$ such that $\T_{p'}\iso \EE^m$.
If $\eps>\dist{p}{p'}{}$, then $\oBall(p,r)\subset \oBall(p',r+\eps)$.
Therefore,
\[\vol \oBall(p,r)\le \omega_m\cdot (r+\eps)^m\]
for any $\eps>0$.
Hence the first statement follows.

Now, suppose $0<r_1<r_2$.
Consider the map $w\: \spc{L}\to \spc{L}$ defined by $w\:x\mapsto \gamma_x(\tfrac {r_1}{r_2})$.
(The map $w$ mimics the dilation with center at $p$ and coefficient $\tfrac {r_1}{r_2}$.)
By comparison,
\[\dist{w(x)}{w(y)}{}\ge \tfrac {r_1}{r_2}\cdot \dist{x}{y}{}.\]
Observe that $\oBall(p,r_1) \supset w[\oBall(p,r_2)]$.
Therefore, 
\[\vol \oBall(p,r_1)\ge (\tfrac {r_1}{r_2})^m\cdot\vol \oBall(p,r_2).\]
\qedsf

The following exercise generalizes the Bishop--Gromov inequality to $\Alex{-1}$ case. 
It is sufficient for most applications, but a more exact statement will be given in \ref{inq:BG+} which also includes the case of  $\Alex{1}$ spaces.

\begin{thm}{Exersice}\label{ex:BG}
Let $\spc{L}$ be an $\Alex{-1}$ space.
Suppose $\spc{L}=m<\infty$.
Show that
\[\vol \oBall(p,r)\le \omega_m\cdot(\sinh r)^m,\]
where $\omega_m$ denotes the volume of the unit ball in $\EE^m$.
Moreover, the function 
\[r\mapsto \frac{\vol B(p,r)}{(\sinh r)^m}\]
is nonincreasing.
\end{thm}

\section{Other dimensions}\label{sec:all-dim}

Next we want to show that \textit{all reasonable definitions of dimension give the same result for Alexandrov spaces}.
More precisely, we have the following theorem; compare to \cite[15.16]{alexander-kapovitch-petrunin2024}.
We refer to \cite{hurewicz-wallman} for definitions of \index{Lebesgue coverning dimension}\emph{Lebesgue coverning dimension} \index{$\TopDim$ (topological dimension)}$\TopDim$ and \index{Hausdorff dimension}\emph{Hausdorff dimension} \index{$\HausDim$ (Hausdorff dimension)}$\HausDim$.

\begin{thm}{Theorem}\label{thm:dim=dim}
For any Alexandrov space $\spc{L}$, we have
\[\LinDim \spc{L}=\TopDim \spc{L}=\HausDim \spc{L}.\]
\end{thm}

\parit{Proof.}
Suppose $\LinDim\spc{L}\ge m$.
The right inverse theorem implies that $\spc{L}$ contains a subset homeomorphic to an open ball in $\EE^m$.
It follows that
\[\TopDim\spc{L}\ge \LinDim\spc{L}.\]

By Szpilrajn's theorem \cite[theorems V 8 and VII 2]{hurewicz-wallman}, 
\[\HausDim\spc{L}\ge\TopDim\spc{L}.\]

Finally, by the Bishop--Gromov inequality (\ref{inq:BG} and \ref{ex:BG}), we get that 
\[\LinDim \spc{L}\ge \HausDim\spc{L}.\]
\qedsf

\begin{thm}{Exercise}\label{ex:dim=dim}
Let $\Omega$ be an open subset of Alexandrov space $\spc{L}$.
Show that 
\[\LinDim \spc{L}=\LinDim \Omega=\TopDim \Omega=\HausDim \Omega.\]
\end{thm}

\section{Comments}

Let us state a version of Bishop--Gromov inequality for $\Alex\kappa$ spaces.
Its proof requires additionally the so-called \textit{coarea formula} for Alexandrov spaces. 
The weaker inequality from \ref{ex:BG} is sufficient for the sequel.

\begin{thm}{Bishop--Gromov inequality}\label{inq:BG+}
Let $p$ be a point in an $m$-dimensional $\Alex\kappa$ space.
Consider the function $v(r)\z=\vol_m\oBall(p,r)$;
denote by $\tilde v(r)$ the volume of $r$ ball in $\MM^m(\kappa)$.
Then 
\[v(r)\le \tilde v(r)\]
for $r>0$ and the function 
\[r\mapsto \frac{v(r)}{\tilde v(r)}\] is nonincreasing.
If $\kappa>0$, then one has to assume that $r<\tfrac\pi{\sqrt\kappa}$.
\end{thm}

This inequality was originally proved for Riemannian manifolds with lower Ricci curvature.
The first part is also called \emph{Bishop's inequality}.
It is due to Richard Bishop; see \cite{bishop1964} and \cite[Corollary 4, p. 256]{bishop-crittenden}.
The second part is due to Michael Gromov \cite{gromov1981}.

Theorem~\ref{thm:dim=dim}, was ssentially proved by Conrad Plaut \cite{plaut:dimension}.
At that time, it was not known whether
\[\LinDim\spc{L}=\infty\quad \Rightarrow\quad \TopDim\spc{L}=\infty\]
for any Alexandrov space $\spc{L}$.
The latter implication was proved by Grigory Perelman and the second author \cite{perelman-petrunin:qg}.


%%!TEX root = the-volume.tex

\chapter{Limit spaces}\label{chap:lim}\label{chap:stability}


Here we will show that lower curvature bound in the sense of Alexandrov survives under Gromov--Hausdorff limit,
present Perelman's construction of strictly concave functions, and
prove Gromov's selection theorem.

The suvival of curvature bound provides the main source of applications of Alexandrov geometry;
as an illustration we prove the homotopy stability theorem (\ref{thm:h-stability}) and deduce the homotopy finiteness theorem (\ref{thm:h-finiteness}) from it.



\section{Survival of curvature bounds}

\begin{thm}{Theorem}\label{thm:CBB-closed}
Let $\spc{X}_n\z\to \spc{X}_\infty$ be a convergence in the sense of Gromov--Hausdorff.
Suppose that for each $n$, the space $\spc{X}_n$ has curvature $\ge\kappa$ in the sense of Alexandrov.
Then the same holds for~$\spc{X}_\infty$.
\end{thm}

\parit{Proof}.
Choose a quadruple of points $p_\infty, x_\infty,y_\infty,z_\infty\in \spc{X}_\infty$.

By the definition of Gromov--Hausdorff convergence, we can choose points $p_n$,  $x_n$, $y_n$, $z_n\in \spc{X}_n$ for each $n$
that converge to $p_\infty$, $x_\infty$, $y_\infty$, $z_\infty\in \spc{X}_\infty$, respectively.
In particular, each of the 6 distances between pairs of $p_n$, $x_n$, $y_n$, $z_n$ converge to the distance between the corresponding pairs of $p_\infty, x_\infty,y_\infty,z_\infty$.

Since $\MM^2(\kappa)$-comparison holds for $p_n$, $x_n$, $y_n$, $z_n\z\in \spc{X}_n$,
passing to the limit, we get the $\MM^2(\kappa)$-comparison for $p_\infty$,  $x_\infty$, $y_\infty$, $z_\infty$.
\qeds

\begin{thm}[!]{Exercise}\label{ex:dim-lim}
Suppose that a sequence $\spc{A}_1,\spc{A}_2,\dots$ of $\Alex\kappa$ spaces converges to $\spc{A}_\infty$ in the sense of Gromov--Hausdorff.
Show that $\spc{A}_\infty$ is $\Alex\kappa$ and
\[\dim \spc{A}_\infty\le \liminf_{n\to\infty} \dim \spc{A}_n.\]
\end{thm}

\section{Gromov's selection theorem}

\begin{thm}{Gromov's selection theorem}\label{thm:gromov-compactness}
Let $m$ be a positive integer, and let $D,\kappa\in\RR$.
Then any sequence of $m$-dimensional $\Alex\kappa$ spaces with diameters at most $D$
has a converging subsequence in the sense of Gromov--Hausdorff.
\end{thm}

\parit{Proof of \ref{thm:gromov-compactness}.}
Denote by $\bm{K}$ the set of all isometry classes of $\Alex0$ spaces with dimension $\le m$ and diameter $\le D$.
By \ref{ex:dim-lim}, $\bm{K}$ is a closed subset of $\GH$.

Choose a space $\spc{A}\in \bm{K}$;
suppose $x_1,\dots,x_n\in \spc{A}$ is a collection of points such that $\dist{x_i}{x_j}{}> \eps$ for all $i\ne j$.
Note that the balls $B_i=\oBall(x_i,\tfrac\eps2)$ do not overlap.

By \ref{thm:right-inverse}, $\vol \spc{A}>0$.
By Bishop--Gromov inequality, $\vol \spc{A}<\infty$,
and if $\eps<D$, then 
\[\vol B_i\ge (\tfrac\eps{2\cdot D})^m\cdot\vol \spc{A}\]
for any $i$.
It follows that $n\le (\tfrac{2\cdot D}\eps)^m$;
that is, 
\[\pack_\eps\spc{A}\le  N(\eps)\df(\tfrac{2\cdot D}\eps)^m\]
for all small $\eps>0$.

Choose a maximal $\eps$-packing $x_1,\z\dots,x_n\in \spc{A}$.
By \ref{ex:pack-net}, $\spc{F}_\eps\z\df\{x_1,\z\dots,x_n\}$ is an $\eps$-net of $\spc{A}$.
Observe that $\dist{\spc{F}_\eps}{\spc{A}}{\GH}\le \eps$.
Further, note that the set $\bm{F}_\eps$ of finite metric spaces with diameter $\le D$ and at most $N(\eps)$ points forms a compact subset in $\GH$.

Summarizing, for any $\eps>0$ we can find a compact $\eps$-net $\bm{F}_\eps\subset \GH$ of $\bm{K}$.
Since $\GH$ is complete (\ref{prop:complete}), it remains to apply \ref{ex:net:compact}.

We finished the proof of the case $\kappa=0$.
In the general case, applying rescaling, we can assume that $\kappa=-1$ and then argue as before, using \ref{ex:BG} instead of \ref{inq:BG}.
\qeds

\begin{thm}[!]{Exercise}\label{ex:pack-vol}

\begin{subthm}{ex:pack-vol:pack}
Let $\spc{A}$ be an $m$-dimensional $\Alex0$ space with diameter $\le D$.
Suppose $\vol\spc{A}\ge v_0>0$.
Show that 
\[\pack_\eps\spc{A}\ge \frac\Const{\eps^m}\]
for some constant $\Const=\Const(m,D,v_0)>0$.
\end{subthm}


\begin{subthm}{ex:pack-vol:dim}
Conclude that if $\spc{A}_n$ is a sequence of $m$-dimensional $\Alex0$ spaces with diameter $\le D$, and volume $\ge v_0$, then its Gromov--Hausdorff limit $\spc{A}_\infty$ (if it exists) has dimension~$m$.
\end{subthm}
\end{thm}

\begin{thm}{Exercise}\label{ex:diam-compact:GH}
Show that any sequence of $m$-dimensional $\Alex\kappa$ spaces with marked points contains a subsequence pointed-converging in the sense of Gromov--Hausdorff (see Section~\ref{sec:Gromov--Hausdorff}).

\end{thm}

%%!TEX root = the-homot-finite.tex
\section{Controlled concavity}

Alexandrov spaces have plenty of semiconcave functions;
for instance, square of distance function. 
The following theorem provides a source of strictly concave functions  defined in a small open sets of finite-dimensional Alexandrov spaces. 

\begin{thm}{Theorem}
\label{thm:strictly-concave}
Let $\spc{L}$ be a complete finite-dimensional Alexandrov  space.
Then for any point $p\in \spc{L}$, there is  a strictly concave function $f$ defined in an
open neighborhood of $p$.

Moreover, given $0\ne v\in T_p$, the differential, $\dd_p f$, can be chosen
arbitrarily close to $x\mapsto -\<v,x\>$.
\end{thm}

\parit{Proof.} 
Fix small $r>0$ and large $c$;
consider the real-to-real function 
$$\phi_{r,c}(x)=(x-r)- c\cdot(x-r)^2/r,$$
so we have 
$\phi_{r,c}(r)=0$,
$\phi_{r,c}'(r)=1$,
and $\phi_{r,c}''(r)=- {2c}/{r}$. 

\begin{wrapfigure}{o}{44 mm}
\vskip-0mm
\centering
\includegraphics{mppics/pic-901}
\vskip1mm
\end{wrapfigure}

Let $\gamma$ be a unit-speed geodesic, fix a point $q$ and let 
$$\alpha(t)=\mangle(\gamma^+(t),\dir{\gamma(t)}{q}).$$
Recall that $r$ is small.
If $\dist q{\gamma(t)}{}$ is sufficiently close to
$r$, then direct calculations show that
$$(\phi_{r,c}\circ\distfun_q\circ\gamma)''(t)
\le 
\frac{3-c\cdot \cos^2[\alpha(t)]}{r}.$$
(Since $c$ is large, this inequality implies that $\phi_{r,c}\circ\distfun_q\circ\gamma$ is strictly concave at $t$ unless $\alpha(t)\approx\tfrac\pi2$.) 

Now, assume $\{q_1,\dots, q_N\}$ is a finite set of points such that $\dist p{q_i}{}=r$ for any $i$. 
For a geodesic $\gamma$, set $\alpha_i(t)=\mangle(\gamma^+(t),\dir {\gamma(t)}{q_i})$. 
Assume we have a collection $\{q_i\}$ such
that 
\[\max_i\{|\alpha_i(t)-\tfrac\pi2|\}\ge\eps>0\]
for any geodesic $\gamma$ in $\oBall(p,\eps)$. 
We can assume that $c>3N/\cos^2\eps$;
then the inequality above implies that the function
$$f=\sum_i \phi_{r,c}\circ\distfun_{q_i}$$
is strictly concave in $\oBall(p,\eps')$ for some positive $\eps'<\eps$.

The same argument as in \ref{ex:pack-vol} shows that for small $r>0$, one can
choose $N\ge \Const/\delta^{m-1}$ points $\{q_i\}$ such that $\dist{p}{q_i}{}=r$
and $\angk p{q_j}{q_i}>\delta$ (here $\Const=\Const(\Sigma_p)>0$).
On the other hand, suppose $\gamma$ runs from $x$ to~$y$.
If $|\alpha_i(t)- \tfrac\pi2|<\eps\ll\delta$, then applying the ($n$+1)-point comparison to $\gamma(t)$, $x$, $y$ and all $\{q_i\}$ we get that
$N\le \Const(m)/\delta^{m-2}$. 
Therefore, for small $\delta>0$ and yet smaller $\eps>0$, the set $\{q_i\}$ forms the needed collection.

If $r$ is small, then points $q_i$ can be chosen so that all directions
$\dir p {q_i}$ will be $\eps$-close to a given direction $\xi$ and
therefore the second property follows.
\qeds

The function $f$ in \ref{thm:strictly-concave} can be chosen to have maximum value $0$ at $p$,
$f(p)=0$ and with $\dd_p f(x)\approx-|x|$.
It can be constructed by taking the minimum of the functions in the theorem.
Then the set $K=\set{x\in\spc{L}}{f(x)\ge -\eps}$ forms a closed convex neighborhood of $p$ for any small $\eps>0$, so we get the following.


\begin{thm}{Corollary}\label{cor:convex-nbhd}
Any point $p$ of a finite-dimensional Alexandrov space admits an arbitrary small convex closed neighborhood $K$ and a strictly concave function $f$ defined in a neighborhood of $K$ such that $p$ is the maximum point of $f$
and $f|_{\partial K}=0$.
\end{thm}

\section{Liftings}

Suppose that the Gromov--Haudorff distance $\dist{\spc{L}}{\spc{L}'}{\GH}$ is sufficienlty small, so we may think that both spaces $\spc{L}$ and $\spc{L}'$ lie at small Hausdorff distance in an ambient metric space $\spc{W}$.
In particular, we may choose a small $\eps>0$, so that for any point $p\in \spc{L}$, there is a point $p'\in \spc{L}'$ such that $\dist{p}{p'}{\spc{W}}<\eps$;
the point $p'$ will be called a \index{lifting}\emph{lifting} (or \emph{$\eps$-lifting}) of $p$ in $\spc{L}'$.
We may choose a lifting $p'\in\spc{L}'$ for every point $p\in\spc{L}$, 
in this case the map $p\mapsto p'$ is called a {}\emph{($\eps$-)lifting map}.

Note that the lifting is not uniquely defined.
The lifting maps is not assumed to be continuous.
When we talk about liftings, we assume that $\eps>0$, the inclusions $\spc{L},\spc{L}'\hookrightarrow\spc{W}$,
as well as $\spc{W}$ are chosen.

Let $\spc{L}$ be  a compact $m$-dimensional Alexandrov space.
Suppose $\spc{L}'$ is another compact $m$-dimensional Alexandrov space such that $\dist{\spc{L}}{\spc{L}'}{\GH}$ is sufficiently small --- smaller than some $\eps=\eps(\spc{L})>0$.
Then the construction in $\spc{L}$ from the previous section  
can be repeated in $\spc{L}'$ for the liftings of all points and the same function $\phi$.
It produces a strictly concave function defined in a controlled neighborhood of the lifting $p'$ of $p$.

The result of this and related constructions will be called \index{lifting}\emph{liftings},
say we can talk about a lifting from $\spc{L}$ to $\spc{L}'$ of a function provided by \ref{thm:strictly-concave} (if the Gromov--Hausdorff distance $\dist{\spc{L}}{\spc{L}'}{\GH}$ is small, then these liftings are stricly concave)
and a lifting of a convex neighborhood from \ref{cor:convex-nbhd}.
Here one cannot use \ref{thm:strictly-concave} and \ref{cor:convex-nbhd} as black boxes --- one has to understand the construction, but it is straightforward.

\section{Nerves}

Let $\{\Omega_1,\dots,\Omega_k\}$ be a finite open cover of a compact metric space $\spc{X}$.
Consider an abstract simplicial complex $\spc{N}$, with one vertex $v_i$ for each set $\Omega_i$ such that a simplex with vertices $v_{i_1},\dots, v_{i_m}$ is included in $\spc{N}$ if 
the intersection $\Omega_{i_1}\cap\dots\cap \Omega_{i_m}$ is nonempty.
\begin{figure}[ht!]
\vskip-0mm
\centering
\includegraphics{mppics/pic-1402}
\end{figure}
The obtained simplicial complex $\spc{N}$ is called the \index{nerve}\emph{nerve} of the covering $\{\Omega_i\}$.
Evidently $\spc{N}$ is a finite simplicial complex ---
it is a subcomplex of a simplex with the vertices $\{v_1,\dots,v_k\}$.
Recall that $\Star_{v_i}$ denotes the union of all simplexes in $\spc{N}$ that shares vertex $v_i$.

The next statement follows from \cite[4G.3]{hatcher}.


\begin{thm}{Nerve theorem}\label{thm:nerve}
Let $\{\Omega_1,\dots,\Omega_k\}$ be an open cover of a compact metric space $\spc{X}$
and let $\spc{N}$ be the corresponging nerve with vertices $\{v_1,\dots,v_k\}$.
Suppose that every nonempty finite intersection $\Omega_{\alpha_1}\cap\z\dots\cap\Omega_{\alpha_k}$ is contractible.
Then $\spc{X}$ is homotopy equivalent to the nerve $\spc{N}$ of the cover.

Moreover homotopy equivalences  $a\:\spc{X}\to \spc{N}$ and $b\:\spc{N}\to\spc{X}$ can be chosen so that 
if $x\in \Omega_i$, then $a(x)\in \Star_{v_i}$,
and if $y\in\spc{N}$ lies in the simplex with vertices $v_{i_1},\dots, v_{i_m}$, then $b(y)\in \Omega_{i_1}\cup\dots\cup \Omega_{i_m}$.
\end{thm}

%???Вить, посмотри на это утверждение --- оно мне не сильно нравится.


\section{Homotopy stability}

\begin{thm}{Theorem}\label{thm:h-stability}
Let $\spc{L}_1,\spc{L}_2,\dots$, and $\spc{L}_\infty$ be $m$-dimensional $\Alex\kappa$ spaces, and $m<\infty$.
Suppose $\spc{L}_n\z\GHto \spc{L}_\infty$ as $n\to \infty$.
Then $\spc{L}_\infty$ is homotopically equivalent to $\spc{L}_n$ for all large $n$.

Moreover, given $\eps>0$ there are maps $h_n\:\spc{L}_\infty\to \spc{L}_n$ that are homotopy equivalences and $\eps$-liftings for all large $n$.
\end{thm}

Applying this theorem with the Gromov's selection theorem (\ref{thm:gromov-compactness}) and Exercise \ref{ex:pack-vol}, we get the following.


\begin{thm}{Theorem}\label{thm:h-finiteness}
There are only finitely many homotopy types of $m$-dimensional $\Alex\kappa$ spaces with diameter $\le D$, and volume $\ge v_0$;
here we assume that an integer $m$, and $v_0>0$ and $D>0$ are given.
\end{thm}

\parit{Proof of \ref{thm:h-finiteness} modulo \ref{thm:h-stability}.}
Assume the contrary, then we can choose a sequence of spaces $\spc{L}_1,\spc{L}_2,\dots$ that have different homotopy types and satisfy the assumptions of the theorem.
By Gromov's compactness theorem, we can assume that $\spc{L}_n$ converges to say $\spc{L}_\infty$ in the sense of Gromov--Hausdorff.

By \ref{ex:pack-vol}, $\dim \spc{L}_\infty=m$.
It remains to apply \ref{thm:h-stability}.
\qeds

\parit{Proof of \ref{thm:h-stability}.}
Since $\spc{L}_\infty$ is compact, applying \ref{cor:convex-nbhd}, we can find a finite open cover of $\spc{L}_\infty$ by convex open sets $\Omega_1,\dots, \Omega_k$ such that 
for each $\Omega_i$ there is a strictly concave function $f_i$ that is defined in a neighborhood of $\bar \Omega_i$ and such that $f_i|_{\partial \Omega_i}=0$.

Subtracting from functions $f_i$ some small value $\eps>0$,
we can ensure that $\bigcap_{i\in S}\Omega_{i}\ne \emptyset$ if and only if $\bigcap_{i\in S}\bar\Omega_{i}\ne \emptyset$.

Suppose that $W=\bigcap_{i\in S}\Omega_{i}\ne \emptyset$.
Then $W$ is contractible.
Indeed the function 
\[f_S\df\min_{i\in S} f_i\]
is strictly concave and it vanished on the boundary of $W$.
The $f_S$-gradient flow $(t,x)\mapsto \GF_{f_S}^t(x)$ defines a homotopy
$[0,\infty)\times W\to W$.
By the first distance estimate (\ref{thm:dist-est}), $\GF_{f_S}^t(x)$ converges to the (necessarily unique) maximum point of $f_S$ as $t\to\infty$.
Therefore, in the obtained homotoly we can parametrize $[0,\infty)$ by $[0,1)$ and extend the homotopy by continiously to $[0,1]$;
thus we get that $W$ is contractible.
In other words, the cover $\{\Omega_1,\dots, \Omega_k\}$ meets the assumptions of the nerve theorem (\ref{thm:nerve}).

The functions $f_i$ and sets $\Omega_i$ can be lifted to $\spc{L}_n$ keeping their properties for all large $n$. 
More precisely, there are liftings $f_{i,n}$ of all $f_i$ to $\spc{L}_n$ which are strictly concave for all large $n$ and such that $\bar\Omega_{i,n}=\set{x\in \spc{L}_n}{f_{i,n}(x)\ge 0}$ is a compact convex set and $\Omega_{i,n}\z=\set{x\in \spc{L}_n}{f_{i,n}(x)> 0}$ is an open convex set for each $i$.

Notice that $\{\Omega_{1,n},\dots,\Omega_{k,n}\}$ is an open cover of $\spc{L}_n$ for all large~$n$.
Indeed suppose we have $p_n\in \spc{L}_n\setminus(\Omega_{1,n}\cup\dots\cup\Omega_{k,n})$ for arbitrary large $n$.
Since $\spc{L}_\infty$ is compact, there is a limit point $p_\infty\in \spc{L}_\infty$ for a subsequnce of $p_n$.
But $p_\infty\in\Omega_i$ for some $i$ and therefore $p_n\in \Omega_{i,n}$ for arbitrary large $n$ --- a contradiction.

In a similar fashion, we can show that if $n$ is large, then any collection $\{\Omega_{i,n}\}_{i\in S}$ has a common point in $\spc{L}_n$ 
if and only if $\{\Omega_{i}\}_{i\in S}$ has a common point in $\spc{L}_\infty$.
Here we have to use that $\bigcap_{i\in S}\Omega_{i}\ne \emptyset$ if and only if $\bigcap_{i\in S}\bar\Omega_{i}\ne \emptyset$.

It follows that for any large $n$ the covers 
\begin{itemize}
\item $\{\Omega_{1},\dots,\Omega_{k}\}$ of $\spc{L}_\infty$ and 
\item $\{\Omega_{1,n},\dots,\Omega_{k,n}\}$ of $\spc{L}_n$.
\end{itemize}
have the same nerve.
By the nerve theorem (\ref{thm:nerve}), $\spc{L}_n$ and $\spc{L}_\infty$ are homotopically equivalent for all large $n$ --- a contradiction.
\qeds

\section{Comments}

Gromov's selection theorem provides the main source of applications of Alexandrov spaces to Riemannian geometry.
The homotopy-type finiteness theorem (\ref{thm:h-finiteness})  illustrates this technique.

Originally, Gromov's selection theorem was proved for Riemannian manifolds with a lower bound on Ricci curvature \cite{gromov1981}.
It motivates the study of the so-called $\mathrm{CD}(K,m)$ spaces; $\mathrm{CD}$ stands for curvature-dimension condition.
This theory has serious applications in Alexandrov geometry;
in particular, it provides a version of Liouville theorem about phase-space volume of geodesic flow in Alexandrov space \cite{brue-mondino-semola}.

The construction of strictly concave function is due to Grigory Perelman \cite{perelman1993,perelman-petrunin}.

Let us list some results that can be proved by applying Gromov's selection theorem
in the same fashion as in the proof of homotopy-type finiteness theorem (\ref{thm:h-finiteness}).

\begin{thm}{Betti-number theorem}
There is a constant $\Const=\Const(m,D,\kappa)$ such that 
\[\beta_0(M)+\beta_1(M)+\dots+\beta_m(M)\le \Const\]
for any closed $m$-dimensional Riemannian manifold $M$ with sectional curvature $\ge \kappa$ and diameter $\le D$.
Here $\beta_i(M)$ denotes $i^\text{th}$ Betti number of $M$.
\end{thm}

Gromov's original proof \cite{gromov-1981} of the Betti-number theorem did not use Alexandrov geometry directly;
but it is quite natural to prove it via Gromov's selection theorem.
The following result proved the second author \cite{petrunin2008}, and it uses the same technique.

\begin{thm}{Scalar curvature bound}
There is a constant $\Const=\Const(m,D,\kappa)$ such that 
\[\int_M\Sc\le \Const\]
for any closed $m$-dimensional Riemannian manifold $M$ with sectional curvature $\ge \kappa$ and diameter $\le D$.
Here $\Sc$ denotes the scalar curvature.
\end{thm}

The following theorem is a more exact version of \ref{thm:h-stability}.
Its close relative (\ref{thm:spherical-nbhd}) will play an important role in the following lecture.

\begin{thm}{Stability theorem}\label{thm:stability}
Let $\spc{L}_1,\spc{L}_2,\dots$, and $\spc{L}_\infty$ be  $m$-dimensional $\Alex\kappa$ spaces, and $m<\infty$.
Suppose $\spc{L}_n\GHto \spc{L}_\infty$ as $n\to \infty$.
Then $\spc{L}_\infty$ is homeomorphic to $\spc{L}_n$ for all large $n$.

Moreover, given $\eps>0$ there are maps $h_n\:\spc{L}_\infty\to \spc{L}_n$ that are homeomorphisms and $\eps$-liftings for all large $n$.
\end{thm}

This theorem was proved by Grigory Perelman \cite{perelman1991};
the proof was rewritten with more details by the first author \cite{kapovitch}.
Perelman have made an informal annoncemnt that the homeomorphisms in the theorem can be assumed to be bi-Lipschitz with constants that depend on $\spc{L}_\infty$;
he refused to write the proof, and so it save to consider it as a conjecture.

The last statement in the theorem implies the following finiteness result.

\begin{thm}{Homeomorphism-type finiteness}
There are only finitely many homeomorphism types of closed $m$-dimensional manifolds that admit a Riemannian metric with sectional curvature $\ge \kappa$, and diameter $\le D$.
\end{thm}

Applying several results in differential topology, this statement can be improved to diffeomorphism-type finiteness in all dimensions $m$ except $m=4$; see \cite{kirby-siebenmann} and  \cite{moise,thurston} for cases $m\ge 5$ and $m\le 3$, respectively.



%%!TEX root = the-boundary.tex
\chapter{Boundary}\label{chap:bry}

This lecture defines the boundary of a finite-dimensional Alexandrov space.
After discussing its properties, we prove the doubling theorem (\ref{thm:doubling:doubling}).

\section{Definition}

Let us give an inductive definition of the boundary of finite-dimensional Alexandrov spaces.

Suppose $\spc{A}$ is a 1-dimensional Alexandrov space.
By Exercise~\ref{ex:dim=1},
$\spc{A}$ is homeomorphic to a 1-dimensional manifold (possibly with non-empty boundary).
This  allows us to define the boundary $\partial\spc{A}\subset \spc{A}$ as the boundary of the manifold.

Now assume that the notion of boundary is defined in dimensions $1,\dots,m-1$.
Suppose  $\spc{A}$ is $m$-dimensional Alexandrov space.
We say that $p\in \spc{A}$ belongs to the boundary (briefly $p\in \partial \spc{A}$) if 
$\partial\Sigma_p\ne\emptyset$.
By \ref{thm:finite-space-of-directions} and \ref{ex:finite-space-of-directions-dim}, $\Sigma_p$ is an $(m-1)$-dimensional Alexandrov space;
therefore its boundary is already defined and hence this inductive definition makes sense.

It is instructive to check the following statements.
\begin{itemize}
\item For a closed convex set $K\subset \EE^m$ with non-empty interior, the topological boundary of $K$ as a subset of $\EE^m$ coincides with the boundary $K$ described above.
\item If $\spc{A}\iso\spc{A}_1\times\spc{A}_2$ is a finite-dimensional Alexandrov space,
then
\[\partial \spc{A}=(\partial\spc{A}_1\times\spc{A}_2)\,\cup\,(\spc{A}_1\times\partial\spc{A}_2)\]
\item If $\Cone\Sigma$ is an $\Alex0$ space of dimensions $\ge 2$  (this necessarily implies that   $\Cone\Sigma$  is  $\Alex1 $ then
\[\partial \Cone\Sigma=\Cone\partial\Sigma,\]
where $\Cone\partial\Sigma=\set{s\cdot \xi\in\Cone\Sigma }{\xi\in \partial\Sigma}$.
\end{itemize}


\section{Conic neighborhoods}

The following statement \cite{perelman1993} is a close relative of Perelman's stability theorem \ref{thm:stability}.
% but its proof is  simpler .
We are going to use this result without proof.

Recall that the logarithm $\log_px\:\spc{A}\to \T_p$ is defined on page \pageref{page:log}.

\begin{thm}{Theorem}\label{thm:spherical-nbhd}
For any point $p$ in a finite-dimensional Alexandrov space $\spc{A}$
and all sufficiently small $\eps>0$
there is a homeomorphism $h_\eps\:\oBall(p,\eps)_{\spc{A}}\to \oBall(0,\eps)_{\T_p}$ such that $0=h_\eps(p)$.

Moreover, we may assume that
\[
\sup_{x\in \oBall(p,\eps)}\{\,\tfrac1\eps\cdot\dist{\log_px}{h_\eps(x)}{\T_p}\,\}\to 0
\quad\text{as}\quad
\eps\to 0.\]
\end{thm}
Note that the last condition automatically implies that  $h_\eps$ as an $o(\eps)$ G-H approximation.

The above theorem is often used together with the \textit{uniqueness of conic neighborhoods} stated below.

Suppose that an open  neighborhood $U$ of a point $x$ in a metric space $\spc{X}$
% openness is very important. it's false otherwise
admits a homeomorphism to $\Cone\Sigma$ such that $x$ is mapped to the origin of the cone.
In this case, we say that $U$ has a \index{conic neighborhood}\emph{conic neighborhood} of~$x$.

\begin{thm}{Uniqueness of conic neighborhoods}\label{lem:kwun}
Any two conic neighborhoods of a given point in a metric space are \index{pointed homeomorphic}\emph{pointed homeomorphic}; that is, there is a homeomorphism between neighborhoods that maps the origin of one cone to the origin of the other.
\end{thm}

\begin{thm}{Advanced exercise}\label{ex:conic}
Prove \ref{lem:kwun} or read the proof in \cite{kwun1964}.
\end{thm}


\begin{thm}{Exercise}\label{ex:conic-tangent}
Suppose $x\mapsto x'$ is a homeomorphism between finite-dimensional Alexandrov spaces $\spc{A}$ and $\spc{A}'$. Show that 

\begin{subthm}{ex:conic-tangen:tangent}
$\T_x\cong \T_{x'}$ (here and below $\cong $ means homeomorphic) 
\end{subthm}

\begin{subthm}{ex:conic-tangen:dir}
$\Susp\Sigma_x\cong \Susp\Sigma_{x'}$.
\end{subthm}

\begin{subthm}{ex:conic-tangen:example}
but in general $\Sigma_x\ncong\Sigma_{x'}$.
\end{subthm}

\end{thm}



\section{Topology}

The following theorem states that boundary is a topological invariant, despite our definition having used geometry.

\begin{thm}{Theorem}\label{thm:top-bry}
Let $\spc{A}$ and $\spc{A}'$ be homeomorphic finite-dimensional Alexandrov spaces.
Then $\dim \spc{A}=\dim\spc{A}'$ and
\[\partial\spc{A}\ne \emptyset
\quad\iff\quad
\partial\spc{A}'\ne \emptyset
\]
\end{thm}

While working on the proof, keep in mind that there are pairs of spaces $\spc{K}_1$ and $\spc{K}_2$ such that $\spc{K}_1\ncong \spc{K}_2$, but $\RR\times \spc{K}_1\cong \RR\times \spc{K}_2$.
Suspension over the Poincaré homology sphere with $\SSS^4$ is one of the examples; compare to \ref{ex:conic-tangen:example}.

Let $\spc{A}$ be an $m$-dimensional Alexandrov space and $m<\infty$.
Define \index{rank}\emph{rank} of $\spc{A}$ (briefly, \index{$\rank\spc{A}$}$\rank\spc{A}$) as the minimal value $k$ such that $\spc{A}$ splits isometrically as $\RR^{m-k}\times \spc{K}$;
here $\spc{K}$ is a $k$-dimensional Alexandrov space.

In the following proof we will apply induction on the rank of $\spc{A}$.


\parit{Proof.}
The first statement follows from \ref{thm:dim=dim}.

Suppose we have a counterexample, say $\partial \spc{A}\ne \emptyset$, but $\partial \spc{A}'=\emptyset$.
Let $k\df\rank \spc{A}$ and $k'\df\rank \spc{A}'$.
We can assume that the pair $(k,k')$ is minimal in lexicographic order;
in particular, $k$ is minimal.
Let $x\mapsto x'$ be a homeomorphism from $\spc{A}$ to $\spc{A}'$.

Choose $x\in \partial \spc{A}$.
Since $\partial \spc{A}'=\emptyset$, we have $x'\notin \partial \spc{A}'$.
Note that 
\[\rank \T_x\le k
\quad\text{and}\quad
\rank \T_{x'}\le k',
\]
By \ref{ex:conic-tangen:tangent}, $\T_x\cong\T_{x'}$.
Note that $\partial \T_x\ne\emptyset$ and $\partial \T_{x'}=\emptyset$.
Therefore, we may assume that $\spc{A}$ and $\spc{A}'$ are Euclidean cones
and the homeomorphism sends the origin to the origin.
The remaining part of the proof is divided into three cases.

\parit{Case 1.}
Suppose $k>1$.
Let $\spc{A}\iso \RR^{m-k}\times \spc{C}$, where $\spc{C}$ a $k$-dimensional $\Alex0$ cone.
Observe that $\rank\T_y\le\rank\spc{A}$ for any $y\in\spc{A}$ and the equality holds only if $y$ projects to the origin of $\spc{C}$.

Since $k>1$ we can find $z\in\partial\spc{C}$ such that $z\ne 0$.
Choose $y$ that projects to $z$;
in particular, $\rank\T_y<\rank\spc{A}$.
By \ref{ex:conic-tangen:tangent}, $\T_y\cong\T_{y'}$,
$\partial  \T_y\ne\emptyset$ and $\partial \T_{y'}=\emptyset$.
The latter contradicts the minimality of $k$.

\parit{Case 2.} Suppose $k\le1$ and $k'>1$.
Since $\partial \spc{A}\ne \emptyset$, we get that $k=1$;
therefore, $\spc{A}=\RR^{m-1}\times\RR_{\ge0}$.

Let $\spc{A}'\iso \RR^{m-k'}\times \spc{C}'$, where $\spc{C}'$ a $k'$-dimensional $\Alex0$ cone.
Since $\partial\spc{A}\cong\RR^{m-1}$,
the image of $\partial\spc{A}$ in $\spc{A}'$ does not lie in $\RR^{m-k'}\z\times\{0\}$.
In other words, we can choose $y\in \partial \spc{A}$ such that its image $y'\in \spc{A}'$ has a nonzero projection in $\spc{C}'$.
Observe that $\T_y\cong\T_{y'}$,
\[
\rank\T_y\le k=1,
\quad
\rank\T_{y'}< k',
\quad
\partial \T_y=\emptyset,
\quad\text{and}\quad
\partial \T_{y'}\ne \emptyset\]
--- a contradiction.

\parit{Case 3.}
Suppose $k\le 1$ and $k'\le 1$.
Since $\partial \spc{A}\ne \emptyset$, $k=1$.
By \ref{ex:dim=1}, $\spc{A}\z\cong \RR^{m-1}\times\RR_{\ge0}$.
Therefore, $\spc{A}'\cong\RR^m$, and $\spc{A}\ncong\spc{A}'$ --- a contradiction.
\qeds

\begin{thm}{Exercise}\label{ex:bry2bry}
Let $x\mapsto x'$ be a homeomorphism $\Omega\to\Omega'$
between open subsets in finite-dimensional Alexandrov spaces $\spc{A}$ and $\spc{A}'$.
Show that $x\in \partial \spc{A}$ if and only if $x'\in \partial \spc{A}'$.

\end{thm}

\begin{thm}{Exercise}\label{ex:bry-closed}
Show that boundary of a finite-dimensional Alexandrov space is a closed subset.
\end{thm}

\section{Tangent space}

Spaces of directions and tangent spaces of an Alexandrov space have already been defined in \ref{sec:space+directions} and \ref{sec: tangent space}.
Let us extend these definitions to subsets of an Alexandrov space.

Let $X$ be a subset in a finite-dimensional Alexandrov space $\spc{A}$.
Choose $p\in \spc{A}$ and $\xi\in \Sigma_p$.
Suppose $\xi$ is a limit of directions $\dir{p}{x_n}$ for a sequence $x_1,x_2,\dots{}\in X$ that converges to $p$.
Then we say that $\xi$ is in the \index{space of directions}\emph{space of directions} from $p$ to $X$;
briefly \index{$\Sigma_p$ (space of directions)}$\xi\in\Sigma_pX$.

Further, $\Cone(\Sigma_pX)$ will be called the \index{tangent space}\emph{tangent space} to $X$ at $p$;
it will be denoted by \index{$\T_p$ (tangent space)}$\T_pX$.

Note that $\Sigma_pX$ is a subset of $\Sigma_p$ and $\T_pX$ is a subcone in $\T_p$

\begin{thm}{Theorem}\label{thm:partial-Sigma}
For any finite-dimensional Alexandrov space $\spc{A}$, we have
\[\partial (\Sigma_p\spc{A})=\Sigma_p(\partial\spc{A})
\quad\text{and}\quad
\partial(\T_p\spc{A})=\T_p(\partial\spc{A}).\]
\end{thm}

\parit{Proof.}
Choose a sequence $x_n\in \partial \spc{A}$ such that $x_n\to p$ and $\dir p{x_n}\to\xi$.

Let $\eps_n=2\cdot \dist{p}{x_n}{}$,
and let $h_{\eps_n}\:\oBall(p,\eps_n)_{\spc{A}}\to \oBall(0,\eps_n)_{\T_p}$ be the homeomorphisms provided by \ref{thm:spherical-nbhd};
in particular, $\tfrac2{\eps_n}\cdot h_{\eps_n}(x_n)\to \xi$ as $n\to\infty$.
By \ref{ex:bry2bry}, $h_{\eps_n}(x_n)\in \partial \T_p$.
By \ref{ex:bry-closed}, $\xi\in \partial \T_p$.
Therefore,
\[\partial (\Sigma_p\spc{A})\supset\Sigma_p(\partial\spc{A})
\quad\text{and}\quad
\partial(\T_p\spc{A})\supset\T_p(\partial\spc{A}).\]

Similarly, choose $\xi\in\partial\Sigma_p$.
Let $h_{\eps_n}\:\oBall(p,\eps_n)_{\spc{A}}\to \oBall(0,\eps_n)_{\T_p}$ be the homeomorphisms provided by \ref{thm:spherical-nbhd} for a sequence $\eps_n\to 0$ as $n\to\infty$.
By \ref{ex:bry2bry}, $x_n=h_{\eps_n}^{-1}(\tfrac{\eps_n}2\cdot\xi)\in \partial \spc{A}$.
By \ref{thm:spherical-nbhd}, $\dir p{x_n}\to \xi$.
Hence
\[\partial (\Sigma_p\spc{A})\subset\Sigma_p(\partial\spc{A})
\quad\text{and}\quad\partial(\T_p\spc{A})\subset\T_p(\partial\spc{A}).\]
\qedsf

\section{Doubling}

Let $A$ be a closed subset in a metric space $\spc{X}$.
The \index{doubling}\emph{doubling} $\spc{W}$ of $\spc{X}$ across $A$ is two copies of $\spc{X}$ glued along $A$;
more precisely, the underlying set of $\spc{W}$ is the quotient $\spc{X}\times\{0,1\}/\sim$, where $(a,0)\sim (a,1)$ for any $a\in A$ and $\spc{W}$ is equipped with the minimal metric such that both maps $\spc{X}\to \spc{W}$ defined by $x\mapsto (x,0)$ and $x\mapsto (x,1)$ are distance-preserving.

Alternatively, one may say that $\spc{W}$ is equipped with the maximal metric such that the projection $\proj\:\spc{W}\to\spc{A}$ defined by $(x,i)\mapsto x$ is a short map. 
The metric on $\spc{W}$ can also be defined explicitly as
\[\dist{(x,i)}{(y,j)}{\spc{W}}=
\begin{cases}
\dist{x}{y}{\spc{X}}&\text{if}\quad i= j.
\\
\inf\set{\dist{x}{a}{\spc{X}}+\dist{y}{a}{\spc{X}}}{a\in A}&\text{if}\quad i\ne j.
\end{cases}
\]

\begin{thm}{Theorem}\label{thm:doubling}
Let $\spc{A}$ be a finite-dimensional Alexandrov space with non-empty boundary.
Suppose $f\z=\tfrac12\cdot\distfun_p^2$ for some $p\in \spc{A}$.
Then

\begin{subthm}{thm:doubling:concave}
If $\dim \spc{A}\ge 2$, then
$\distfun_{\partial \Sigma_x}(\xi)\le \tfrac\pi2$ for any $x\in\partial \spc{A}$ and $\xi\in \Sigma_x$.
Moreover, if $\distfun_{\partial \Sigma_x}(\xi)= \tfrac\pi2$, then $\mangle(\xi,\zeta)\le\tfrac\pi2$ for any $\zeta\in \Sigma_x$. 
\end{subthm}

\begin{subthm}{thm:partial-grad:grad}
$\nabla_xf\in \partial\T_x$ for any $x\in\partial \spc{A}$.
\end{subthm}

\begin{subthm}{thm:partial-grad:flow}
If $\alpha$ is an $f$-gradient curve that starts at $x\in \partial \spc{A}$, then $\alpha(t)\in \partial \spc{A}$ for any $t$.
Moreover, if $p\in \partial \spc{A}$, then $\gexp_p(v)\in \partial \spc{A}$ for any $v\in\partial\T_p$.
\end{subthm}

\begin{subthm}{thm:doubling:doubling}
The doubling $\spc{W}$ of $\spc{A}$ across $\partial \spc{A}$ is an Alexandrov space with the same curvature bound.
\end{subthm}

\end{thm}

Part \ref{SHORT.thm:doubling:doubling} is called the \index{doubling theorem}\emph{doubling theorem}.

\parit{Proof.}
We will denote by 
\ref{SHORT.thm:doubling:concave}$_m,\dots,$\ref{SHORT.thm:doubling:doubling}$_m$ the corresponding statement assuming $m=\dim\spc{A}$.

The proof goes by induction on $m$.
Statement \ref{SHORT.thm:doubling:doubling}$_1$ follows from \ref{ex:dim=1} --- this is the base.
The induction step is a combination of the implications below.

\parit{\ref{SHORT.thm:doubling:doubling}$_{m-1}\Rightarrow$\ref{SHORT.thm:doubling:concave}$_m$.}
Suppose $m=2$, then $\dim\Sigma_x=1$; see \ref{ex:finite-space-of-directions-dim}.
By \ref{ex:dim=1}, $\Sigma_x$ isometric to a line segment $[0,\ell]$;
we need to show that $\ell\le\pi$.

Assume $\ell>\pi$, then the tangent space $\T_x=\Cone\Sigma_x$ has several different lines thru the origin.
Recall that $\T_x$ is an Alexandrov space; see \ref{ex:finite-tan}.
By \ref{cor:splitting}, $\T_x$ is isometric to the Euclidean plane;
the latter contradicts that $\Sigma_x$ is a line segment.

Now suppose $m>2$, so $\dim \Sigma_x>1$.
Assume $\distfun_{\partial \Sigma_x}(\xi)> \tfrac\pi2$ for some $\xi$.
By \ref{SHORT.thm:doubling:doubling}$_{m-1}$, the doubling $\Xi$ of $\Sigma_x$ is $\Alex1$.
Denote by $\xi_0$ and $\xi_1$ the points in $\Xi$ that correspond to $\xi$.
Observe that $\dist{\xi_0}{\xi_1}{\Xi}>\pi$.
The latter contradicts \ref{ex:RisCBB(1)}.

Finally, if $\distfun_{\partial \Sigma_x}(\xi)= \tfrac\pi2$, then $\dist{\xi_0}{\xi_1}{\Xi}=\pi$.
Therefore, $\Cone \Xi$ contains a line in the directions of $\xi_0$ and $\xi_1$;
in other words, $\Xi$ is a spherical suspension with poles $\xi_0$ and $\xi_1$.
In particular, every point of $\Xi$ lies on distance at most $\tfrac\pi2$ from $\xi_0$ or $\xi_1$.
The natural projection $\Xi\to \Sigma_x$ does not increase distances and sends both  $\xi_0$ and $\xi_1$ to $\xi$.
Therefore, the second statement of \ref{SHORT.thm:doubling:concave}$_m$ follows.

\parit{\ref{SHORT.thm:doubling:doubling}$_{m-1}+$\ref{SHORT.thm:doubling:concave}$_{m-1}+$\ref{SHORT.thm:doubling:concave}$_m\Rightarrow$\ref{SHORT.thm:partial-grad:grad}$_m$.}
We can assume that $s=\nabla_xf\ne 0$.
By \ref{prop:grad-exist}, $\nabla_xf\z=s\cdot \overline{\xi}$, where $s=\dd_xf(\overline{\xi})>0$ and $\overline{\xi}\in\Sigma_p$ is the direction that maximizes $\dd_xf(\overline{\xi})$.

Let $\zeta\in \partial\Sigma_x$ be a direction that minimizes the angle $\mangle(\overline{\xi},\zeta)$.
It is sufficient to show that $\zeta=\overline{\xi}$.

Assume $\zeta\ne \overline{\xi}$;
let $\eta=\dir[\Sigma_x]\zeta{\overline{\xi}}$.
By \ref{SHORT.thm:doubling:concave}$_m$, $\mangle(\overline{\xi},\zeta)\le \tfrac\pi2$ and
\ref{SHORT.thm:doubling:concave}$_{m-1}$ implies that 
\[\mangle(\eta,\nu)\le \tfrac\pi2\eqlbl{eq:<pi/2}\]
for any $\nu\in \Sigma_\zeta\Sigma_x$ (if $m=2$, then the last statement is evident). 

Let $\phi\:\Sigma_x\to\RR$ be restriction of $\dd_xf$ to $\Sigma_x$.
Applying \ref{ex:d(distfun):<} and \ref{eq:<pi/2}, we get that $\dd_{\bar \xi}\phi(\eta)\le 0$.
Since $\dd_xf$ is concave, we have that $\phi''+\phi\le 0$.
If $\phi(\zeta)\le 0$, then it implies that $\phi(\overline{\xi})\le 0$ --- a contradiction to the fact that $s>0$.
If $\phi(\zeta)> 0$, then $\phi(\overline{\xi})<\phi(\zeta)$ --- a contradiction again.

\parit{\ref{SHORT.thm:partial-grad:grad}$_m\Rightarrow$\ref{SHORT.thm:partial-grad:flow}$_m$.}
Let $\alpha$ be an $f$-gradient curve and $\ell(t)=\distfun_{\partial \spc{A}}\alpha(t)$.

Choose $t$;
let $x=\alpha(t)$ and $y\in \partial\spc{A}$ be a closest point to $x$.
By \ref{SHORT.thm:partial-grad:grad}$_m$, we have that $\nabla_y f\in\partial \T_y$.
Since the distance $\dist{x}{y}{}$ is minimal, 
we get $\langle \dir yx,v\rangle\le 0$ for any $v\in \partial \T_y$.
In particular,
\[\langle \dir yx,\nabla_y f\rangle\le 0\]
Applying Exercise~\ref{ex:monotonicity} to $x$ and $y$, 
we get
\[\ell'(t)\le \ell(t)\]
if the left-hand side is defined.
Since $\ell$ is Lipschitz, $\ell'$ is defined almost everywhere.
Integrating the inequality, we get 
\[\ell(t)\le e^t\cdot\ell(0)\]
for any $t\ge 0$.
In particular, if $\ell(0)=0$, then $\ell(t)=0$ for any $t\ge 0$.
Since $\partial\spc{A}$ is closed (\ref{ex:bry-closed}), the statement follows.

\parit{\ref{SHORT.thm:partial-grad:flow}$_{m}+$\ref{SHORT.thm:doubling:doubling}$_{m-1}\Rightarrow$\ref{SHORT.thm:doubling:doubling}$_m$.}
We will consider the case $\kappa=0$;
other cases can be done in the same way, but formulas get more complicated.

Denote by $\spc{A}_0$ and $\spc{A}_1$ the two copies of $\spc{A}$ in $\spc{W}$;
let us keep the notation $\partial \spc{A}$ for the common boundary of $\spc{A}_0$ and $\spc{A}_1$.

\begin{clm}{}
Let $\gamma$ be a geodesic in $\spc{W}$.
Then either $\gamma$ has at most one interior point in $\partial \spc{A}$ or
$\gamma\subset \partial \spc{A}$.
\end{clm}

\begin{wrapfigure}{r}{45mm}
\vskip-2mm
\centering
\includegraphics{mppics/pic-1315}
\end{wrapfigure}

Indeed, assume $\gamma$ shares at least two points with $\partial \spc{A}$, say $x=\gamma(t_1)$ and $y=\gamma(t_2)$ and these are not endpoints of $\gamma$.
Remove from $\gamma$ the set $\gamma\cap \spc{A}_1$
and exchange it to its reflection across $\partial\spc{A}$;
denote the obtained curve by $\hat\gamma$.

Any arc of $\hat\gamma$ with one endpoint in $\partial \spc{A}$
is a geodesic in $\spc{A}_0$.
Since $x,y\in \partial \spc{A}$, the arc of $\hat\gamma$ behind $y$ lies in the image of map $t\mapsto \GF^t_{f_x}(y)$, where $f_x=\tfrac12\cdot\distfun^2_x$.
By \ref{SHORT.thm:partial-grad:flow}, this arc lies in $\partial\spc{A}$.

Now choose a point $z$ on this arc, so $z\in \partial\spc{A}$.
Applying the same argument, we get that the arc of $\hat\gamma$ before $y$ lies in $\partial\spc{A}$.
Hence the claim follows.\claimqeds

Choose a point $p$ in $\spc{W}$;
let $f\df\tfrac12\cdot\distfun_p^2$.
It is sufficient to show that $(f\circ\gamma)''\le 1$ for any $t$.
If $p\in \partial\spc{A}$, then the statement follows from function comparison in $\spc{A}_0$ and $\spc{A}_1$.
So, we can assume that $p\in \spc{A}_0\setminus \partial\spc{A}$.

If $\gamma$ lies in $\partial \spc{A}$, then this inequality follows from the comparison in~$\spc{A}_0$.

\begin{wrapfigure}{r}{55mm}
\vskip-2mm
\centering
\includegraphics{mppics/pic-1325}
\end{wrapfigure}

Choose $y=\gamma(t_0)$; without loss of generality we can assume that $t_0=0$.

If $y\z\in \spc{A}_0\setminus\partial\spc{A}$, then $(f\z\circ\gamma)''(0)\le 1$ in the barrier sense;
it follows from the comparison in $\spc{A}_0$.

Assume $y\in \spc{A}_1\setminus\partial\spc{A}$.
Suppose $[py]$ crosses $\partial\spc{A}$ at $x$.
Let $\Sigma_x$ be the space of directions of $\spc{A}$ at $x$,
and let $\Xi$ be its doubling.
As before, we denote by $\Sigma_0$ and $\Sigma_1$ two copies of $\Sigma_x$ in  $\Xi$
and keep notation $\partial\Sigma_x$ for their common boundary.
By \ref{SHORT.thm:doubling:doubling}$_{m-1}$, $\Xi$ is $\Alex1$.

The directions $\dir x{y}$ and $\dir xp$ lie on opposite sides from $\Xi$ and
\[\dist{\dir x{y}}{\dir xp}\Xi\ge \pi.\]
Otherwise, we could choose a direction $\xi\in\partial\Sigma$ such that
\[\dist{\dir x{y}}{\xi}\Xi+\dist{\xi}{\dir xp}\Xi<\pi.\]
Furthermore, we could consider the radial curve $\alpha(t)=\gexp_x(t\cdot \xi)$.
By \ref{SHORT.thm:partial-grad:flow}$_m$, $\alpha$ lies in $\partial \spc{A}$.
By \ref{prop:gexp}
\[\dist{p}{\alpha(s)}{\spc{A}_0}
+\dist{y}{\alpha(s)}{\spc{A}_1}
<\dist{p}{y}{\spc{W}}\]
for small values $s>0$
--- a contradition.

$\Cone \Xi$ contains a line with directions $\dir x{y}$ and $\dir xp$.
By the splitting theorem, $\Cone \Xi$ split in these directions;
in particular, 
\[\dist{\dir x{y}}{\xi}{}+\dist{\xi}{\dir xp}{}=\pi.\]
for any $\xi\in\Xi$.
It follows that for any $\xi\in\Xi$ there is $\xi'\in\partial\Sigma_x$ such that 
$\xi$ and $\xi'$ lie on some geodesic $[\dir x{y} \dir xp]_\Xi$.

Fix $t\approx 0$ such that $t\ne 0$; let $z=\gamma(t)$.
Choose such $\xi'$ for $\xi=\dir xz$.
Consider the radial curve $\alpha(s)\df\gexp_x(s\cdot\xi')$.
Let us show that 
\[
\begin{aligned}
\dist{p}{z}{\spc{W}}
&\le \dist{p}{\alpha(s)}{\spc{A}_0}+ \dist{\alpha(s)}{z}{\spc{A}_1}\le
\\
&\le\side\hinge yp{z}.
\end{aligned}
\eqlbl{eq:gamma''}
\]
for suitable value $s$.

The first inequality in \ref{eq:gamma''} is evident.
Set $\phi=\mangle\hinge{x}{y}{z}$ and $\psi\z=\mangle(\dir xp,\xi')$.
The choice of $s$ comes from the model configuration $\tilde p$, $\tilde x$, $\tilde y$, $\tilde w$, $\tilde z\in \EE^2$ such that
\begin{align*}
\tilde x&\in [\tilde p\tilde y],
&
\dist{\tilde p}{\tilde x}{}&=\dist{ p}{x}{},
&
\dist{\tilde p}{\tilde y}{}&=\dist{p}{y}{},
&
\dist{\tilde x}{\tilde z}{}&=\dist{x}{z}{},
\\
\tilde w&\in [\tilde p\tilde z],
&
\mangle\hinge{\tilde x}{\tilde y}{\tilde z}&=\phi,
&
\mangle\hinge{\tilde x}{\tilde p}{\tilde w}&=\psi, 
&
s&=\dist{\tilde x}{\tilde w}{}.
\end{align*}
\begin{figure}[ht!]
\vskip-0mm
\centering
\includegraphics{mppics/pic-1014}
\end{figure}

\noindent
By \ref{prop:gexp}, we get 
\begin{align*}
\dist{p}{\alpha(s)}{\spc{A}_0}&\le \dist{\tilde p}{\tilde w}{},
\\
\dist{\alpha(s)}{z}{\spc{A}_1}&\le\dist{\tilde w}{\tilde z}{};
\end{align*}
by the comparison, 
\[\dist{\tilde p}{\tilde z}{}\le \side\hinge ypz.\]

\begin{thm}{Exercise}\label{ex:pz<ypz}
Prove the last inequality.
\end{thm}

Hence we get $(f\circ\gamma)''(0)\le 1$ in the barrier sense.

Finally if $\gamma(0)\in\partial\spc{A}$, then splitting argument shows that 
\[(f\circ\gamma)^+(0)+(f\circ\gamma)^-(0)\le 0.\]

Summarizing, we get that $(f\circ\gamma)''\le 1$ on every arc of $\gamma$ that lies entirely in $\spc{A}_0$ or $\spc{A}_1$.
If $\gamma$ crosses $\partial \spc{A}$, then we know that it happens only once and at the crossing moment $t_0$ 
we have $f\circ\gamma^+(t_0)\z+f\circ\gamma^-(t_0)\z\le 0$.
All this implies that $(f\circ\gamma)''\le 1$.
\qeds

\begin{thm}{Exercise}\label{ex:bry-connected}
Let $\spc{A}$ be a finite-dimensional $\Alex1$ space of dimension $\ge 2$ with non-empty boundary $\partial\spc{A}$.
Show that $\partial\spc{A}$ is connected.
\end{thm}


\begin{thm}{Exercise}\label{ex:dist-to-bry}
Let $\spc{A}$ be an $m$-dimensional $\Alex0$ space with non-empty boundary $\partial\spc{A}$
for $2\le m<\infty$.
Show that the distance function to the boundary
\[\distfun_{\partial\spc{A}}\:\spc{A}\to\RR\]
is concave.
\end{thm}

\begin{thm}{Exercise}\label{ex:liberman}
Let $\spc{A}$ be a finite-dimensional $\Alex0$ space with non-empty boundary $\partial\spc{A}$.
Suppose $\gamma$ is a geodesic in $\partial\spc{A}$ with the induced length metric.
Show that the function $t\mapsto \tfrac12\cdot\distfun_p^2\circ\gamma(t)$ is 1-concave for any point $p$. 
\end{thm}

\begin{thm}{Exercise}\label{ex:native}
Let $\spc{W}$ be a doubling of finite-dimensional Alexandrov space $\spc{A}$ across its boundary,
and let $proj\:\spc{W}\to\spc{A}$ be the natural projection.
Suppose $f\:\spc{A}\to\RR$ is a $\lambda$-concave function.
Show that $f\circ\proj\:\spc{W}\to\RR$ is $\lambda$-concave if and only if $\nabla_xf\in \partial \T_x$ 
for any $x\in\partial \spc{A}$.
\end{thm}



\section{Remarks}

It easily follows by induction on dimension  that the doubling of a finite-dimensional Alexandrov space across its boundary results in an Alexandrov space without boundary.
This observation can often be used to reduce a statement about general finite-dimensional Alexandrov spaces to  Alexandrov spaces without boundary.

For spaces without boundary the following tools become available.

\begin{thm}{Fundamental-class lemma}\label{lem:fund-class}
Any compact finite-dimensional Alexandrov space $\spc{A}$ without boundary has a fundamental class with $\ZZ/2$ coefficients;
that is, if $\spc{A}$ is $m$-dimensional, then
\[H^m(\spc{A},\ZZ/2)=\ZZ/2.\]

\end{thm}

This lemma was proved by Karsten Grove and Peter Petersen \cite{grove-petersen1993}.
Originally it was stated for Alexander--Spanier cohomology. We do not make this distinction  because for compact Alexandrov spaces it is the same as singular cohomology. Indeed,  both cohomology theories are homotopy invariant \cite[Chapter 6]{Spanier}, compact Alexandrov spaces are homotopy equivalent to finite simplicial complexes \ref{thm:finite-dim-hom-simplicial} and  for paracompact  CW complexes  Alexander--Spanier cohomology is isomorphic to \v{C}ech  and singular cohomolgy \cite[Chapter 6]{Spanier}.

This lemma implies, for example, that on finite-dimensional Alexandrov spaces without boundary 
the gradient flow for a $\lambda$-concave function is an onto map;
in other words, gradient curves can be extended into the past.
It is also used in the proof of the following version of the domain invariance theorem \cite[Theorem 3.2]{kapovitch-zhu}.

\begin{thm}{Domain invariance}\label{thm-inv-domain}
Let $\spc{A}_1$ and $\spc{A}_2$ be two $m$-dimensional Alexandrov spaces with empty boundary; $m$ is finite.
Suppose $\Omega_1$ is an open subset in $\spc{A}_1$ and $f\:\Omega_1\to \spc{A}_2$ is an injective continuous map.
Then $f(\Omega_1)$ is open in $\spc{A}_2$.
\end{thm}

Theorem~\ref{thm:spherical-nbhd} can be used to prove the following. 

\begin{thm}{Topological stratification}\label{thm:top-stratification}
Any $m$-dimensional Alexandrov space with $m<\infty$ can be subdivided into topological manifolds $S_0,\z\dots,S_m$ such that for every $i$ we have $\dim S_i=i$ or $S_i=\emptyset$.
Moreover,
\begin{subthm}{}
the closure of $S_{m-1}$ is the boundary of the space, and
\end{subthm}

\begin{subthm}{}
$S_{m-2}=\emptyset$.
\end{subthm}

\end{thm}

Let us mention that this statement implies that a compact finite-dimensional Alexandrov space has the homotopy type of a finite CW complex,
but it seems to be unknown if it has to be homeomorphic to a CW complex.

The stratification theorem~\ref{thm:top-stratification} can be sharpened as follows.

\begin{thm}{Boundary characterization}
Let $\spc{A}$ be an $m$-dimensional Alexandrov space with $m<\infty$.
Then the following statements are equivalent.

\begin{subthm}{item-boundary} $p\in \partial \spc{A}$;
\end{subthm}

\begin{subthm}{item-contractible} $\Sigma_p$ is contractible;
\end{subthm}

\begin{subthm}{item-space-dir-homology} $\tilde H_{m-1}(\Sigma_p,\ZZ/2)= 0$;
\end{subthm}

\begin{subthm}{item-local-homology} $H_m(\spc{A},\spc{A}\setminus \{p\},\ZZ/2)= 0$;
\end{subthm}

\end{thm}

Let $f$ be a semiconcave function.
A point $p\in \Dom f$ is called \index{critical point}\emph{critical} point of $f$ if $\dd_pf\le 0$; 
otherwise it is called \index{regular point}\emph{regular}.

The following statement \footnote{\red add reference? А: Добрый Витя обещал найти :)} plays a technical role in the proof of stability theorem,
but it is also a useful technical tool on its own.

\begin{thm}{Morse lemma}
Let $f$ be a semiconcave function on a finite-dimensional Alexandrov space without boundary.
Suppose $K$ is a compact set of regular points of $f$ in its level set $f=a$.
Then an open neighborhood $\Omega$ of $K$ admits a homeomorphism $x\mapsto (h(x),f(x))$ to a product space $\Lambda\times (a-\eps,a+\eps)$.
\end{thm}

Subsets in Alexandrov spaces that satisfy the condition in \ref{thm:partial-grad:flow} are called extremal.
More precisely, a subset $E$ is \index{extremal set}\emph{extremal} if for any $x\in E$
and $f$-gradient curve that starts in $E$ remains in $E$;
here $f$ is arbitrary function of the form $\tfrac12\cdot \distfun_p^2$. %{\red V: should we add the condition that $E$ is closed?} A: No, it follows.}

Extremal subsets were introduced by Grigory Perelman and the second author \cite{perelman-petrunin}.
They will pop up in the next lecture.

The following conjecture is one of the oldest questions in Alexandrov geometry that remains open.

\begin{thm}{Conjecture}
Let $S$ be a component of the boundary of a finite-dimensional Alexandrov space.
Then $S$ equipped with the induced length metric is an Alexandrov space with the same curvature bound.
\end{thm}

The doubling theorem has several generalizations \cite{petrunin1997,ge-li} that allow to glue nonidentical spaces.


%%!TEX root = the-quotients.tex
\chapter{Quotients}\label{chap:L/G}

This lecture gives several applications of Alexandrov geometry to isometric group actions.

\section{Quotient space}

Suppose that a group $G$ acts isometrically on a metric space $\spc{X}$.
Note that
\[\dist{G\cdot x}{G\cdot y}{\spc{X}/G}
\df
\inf
\set{\dist{x}{g\cdot y}{\spc{X}}}{g\in G}\]
defines a semimetric on the orbit space $\spc{X}/G$.
Moreover, if the orbits of the action are closed,
then it is a genuine metric.

\begin{thm}{Theorem}\label{thm:CBB/G}
Suppose that a group $G$ acts isometrically on a proper $\Alex0$ space $\spc{A}$, and $G$ has closed orbits.
Then the quotient space $\spc{A}/G$ is $\Alex0$.

\end{thm}

A more general formulation will be given in \ref{thm:submetry-CBB-1}.

\parit{Proof.}
Denote by $\sigma\:\spc{A}\to \spc{A}/G$ the quotient map.

Fix a quadruple of points $p,x_1,x_2,x_3\in \spc{A}/G$.
Choose $\hat p\in \spc{A}$ such that $\sigma(\hat{p})=p$.
Since $\spc{A}$ is proper, we can choose  points $\hat{x}_i\in \spc{A}$ such that $\sigma(\hat x_i)=x_i$ and
\[\dist{p}{x_i}{\spc{A}/G}
=
\dist{\hat{p}}{\hat{x}_i}{\spc{A}}\]
for all $i$.

Note that 
\[\dist{x_i}{x_j}{\spc{A}/G}
\le 
\dist{\hat{x}_i}{\hat{x}_j}{\spc{A}}
\]
for all $i$ and $j$.
Therefore 
\[\angk p{x_i}{x_j}
\le
\angk {\hat{p}}{\hat{x}_i}{\hat{x}_j}
\eqlbl{eq:angles-M-L}\]
for all $i$ and $j$.

By $\EE^2$-comparison in $\spc{A}$,
we have
\[\angk {\hat{p}}{\hat{x}_1}{\hat{x}_2}
+\angk {\hat{p}}{\hat{x}_2}{\hat{x}_3}
+\angk {\hat{p}}{\hat{x}_3}{\hat{x}_1}
\le 
2\cdot\pi.\]
Applying  \ref{eq:angles-M-L}, 
we get 
\[\angk p{x_1}{x_2}
+\angk p{x_2}{x_3}
+\angk p{x_3}{x_1}\le 2\cdot\pi;\]
that is,
the $\EE^2$-comparison holds for any quadruple in $\spc{A}/G$.
\qeds

\begin{thm}{Very advanced exercise}\label{ex:Hilbert/G}
Let $G$ be a compact Lie group with a bi-invariant Riemannian metric.
Show that $G$ is isometric to a quotient of a Hilbert space by an isometric group action.

Conclude that $G$ is $\Alex0$.
\end{thm}

\section{Submetries}

A map $\sigma\:\spc{X}\to\spc{Y}$ between metric spaces $\spc{X}$ and $\spc{Y}$
is called a \index{submetry}\emph{submetry} if 
\[\sigma(\oBall(p,r)_\spc{X})=\oBall(\sigma(p),r)_{\spc{Y}}\]
for any $p\in \spc{X}$ and $r\ge 0$.

Suppose $G$ and $\spc{A}$ are as in \ref{thm:CBB/G}.
Observe that the quotient map $\sigma\:\spc{A}\to \spc{A}/G$ is a submetry.
The following two exercises show that this is not the only source of submetries. 

\begin{thm}{Exercise}\label{ex:sumbetries(S^2)}
Construct submetries
\begin{subthm}{ex:sumbetries(S^2):1}
$\sigma_1\:\mathbb{S}^2\to[0,\pi]$,
\end{subthm}
\begin{subthm}{ex:sumbetries(S^2):2}
$\sigma_2\:\mathbb{S}^2\to[0,\tfrac\pi2]$,
\end{subthm}
\begin{subthm}{ex:sumbetries(S^2):n}
$\sigma_n\:\mathbb{S}^2\to[0,\tfrac\pi n]$ (for integer $n\ge 1$)
\end{subthm}
such that the fibers $\sigma_n^{-1}\{x\}$ are connected for any $x$.
\end{thm}

\begin{thm}{Exercise}\label{ex:sumbetries(E^2)}
Let $\sigma\:\EE^2\to [0,\infty)$ be a submetry.
Show that $K\z=\sigma^{-1}\{0\}$ is a closed convex set without interior points and $\sigma(x)\z=\distfun_Kx$.
\end{thm}

The proof of \ref{thm:CBB/G} works for submetries;
that is, \textit{if $\sigma\:\spc{A}\to\spc{B}$ is a submetry and $\spc{A}$ is a proper $\Alex0$ space, then so is $\spc{B}$}.
Theorem \ref{thm:CBB/G} admits a straightforward generalization to $\Alex{-1}$ case.

In the $\Alex1$ case, the proof produces a slightly weaker statement ---  \textit{$\SSS^2$-comparison holds for a quartuple $p,x_1,x_2,x_3$ in the quotient of $\Alex1$ if $\dist{p}{x_i}{}<\tfrac\pi 2$ for each $i$}.
In particular, the quotient space is \textit{locally} $\Alex1$.
But since $\Alex1$ space is geodesic, then so is its quotient.
Therefore, the globalization theorem implies that it is globally $\Alex1$.
The same holds for the targets of submetries from an  $\Alex1$ space.
With a bit of extra work, one can extend the statement to nonproper spaces \cite[8.34]{alexander-kapovitch-petrunin2024}.
Thus, we have the following.

\begin{thm}{Theorem}\label{thm:submetry-CBB-1}
Let $\sigma\:\spc{A}\to\spc{B}$ be a submetry.
If $\spc{A}$ is $\Alex\kappa$ space, then so is $\spc{B}$.

In particular, if $G$ acts isometrically on an $\Alex\kappa$ space $\spc{A}$, and $G$ has closed orbits.
Then the quotient space $\spc{A}/G$ is $\Alex\kappa$.
\end{thm}

\section{Hopf's conjecture}

\textit{Does $\mathbb{S}^2\times\mathbb{S}^2$ admit a Riemannian metric with positive sectional curvature?} \index{Hopf's conjecture}\emph{Hopf's conjecture} says that the answer should be negative.
Let us take a close look at the following partial result obtained by Wu-Yi Hsiang and Bruce Kleiner \cite{hsiang-kleiner}.

\begin{thm}{Theorem}\label{thm:hsiang-kleiner}
There is no Riemannian metric on $\SSS^2\times\SSS^2$ with sectional curvature $\ge 1$ and a nontrivial isometric $\SSS^1$-action.
\end{thm}

Reacall that a group action $G\acts\spc{X}$ is called \index{effective action}\emph{effective} if for any $g\in G$ there is $x\in\spc{X}$ such that $g\cdot x\ne x$.

\begin{thm}{Key lemma}\label{lem:S^3/S^1}
Suppose $\SSS^1\acts\SSS^3$ is an effective isometric action without fixed points
and $\Sigma=\SSS^3/\SSS^1$ is its quotient space.
Then there is a distance noncontracting map $\Sigma\to \tfrac12\cdot \SSS^2$, where $\tfrac12\cdot \SSS^2$ is the standard 2-sphere rescaled with a factor $\tfrac12$.
\end{thm}

The proof of the lemma is guided by the following exercise.

\begin{thm}{Exercise}\label{ex:S^3/S^1}
Suppose $\SSS^1\acts\SSS^3$ is an effective isometric action without fixed points.
Let us think   of $\SSS^3$ as the unit sphere in $\RR^4$.

\begin{subthm}{ex:S^3/S^1:pq}
Show that one can identify $\RR^4$ with $\CC^2$ so that the action
is given by matrix multiplication
\[\left(\begin{matrix}
u^p&0\\
0& u^q
\end{matrix}
\right),\]
where $(p,q)$ is a pair of relatively prime positive integers and $u\in \SSS^1=\set{z\in\CC}{|z|=1}$.
In particular, our $\SSS^1$ is a subgroup of the torus that acts by
matrix multiplication
\[\left(\begin{matrix}
v&0\\
0& w
\end{matrix}
\right),\]
where  $v,w\in \SSS^1$.
\end{subthm}

\smallskip

\noindent Fix $p$ and $q$ as above.
Let $\Sigma_{p,q}=\SSS^3/\SSS^1$ be the quotient space.

\smallskip

\begin{subthm}{ex:S^3/S^1:sphere}
Show that the $\Sigma_{p,q}=\SSS^3/\SSS^1$ is a topological sphere with $\SSS^1$-symmetry.
This symmetry has two fixed points, north pole and south pole, that correspond to the orbits of $(1,0)$ and $(0,1)$ in $\SSS^3$.
\end{subthm}

\smallskip

\noindent Denote by $S(r)$ the circle of radius $r$ with the center at the north pole of $\Sigma_{p,q}$.

\begin{subthm}{ex:S^3/S^1:a}
Denote by $T(r)$ the inverse image $T(r)$ in $\SSS^3$, and let $a(r)$ be its area.
Show that $T(r)$ is an orbit of the torus action and
\[a(r)=\pi^2\cdot\sin r\cdot \cos r.\]

\end{subthm}

\smallskip

\begin{subthm}{ex:S^3/S^1:b}
Let $b_{p,q}(r)$ be the length of the $\SSS^1$-orbit in $\SSS^3$ that corresponds to a point on $S(r)$. 
Show that
\[b_{p,q}=\pi\cdot\sqrt{(p\cdot \sin r)^2+(q\cdot \cos r)^2}.\]
\end{subthm}

\smallskip

\begin{subthm}{ex:S^3/S^1:c}
Let $c_{p,q}(r)$ be the length of $S(r)$.
Show that $a(r)=c_{p,q}(r)\cdot b_{p,q}(r)$.
\end{subthm}

\smallskip

\begin{subthm}{ex:S^3/S^1:cc}
Show that $c_{p,q}(r)\le c_{1,1}(r)$ for any pair $(p,q)$ of relatively prime positive integers.
Use it to construct a distance noncontracting map $\Sigma_{p,q}\to \tfrac12\cdot \SSS^2\iso\Sigma_{1,1}$.
\end{subthm}

\end{thm}

\parit{Proof of \ref{thm:hsiang-kleiner}.}
Assume $\spc{B}=(\SSS^2\times\SSS^2,g)$ is a counterexample.
By the Toponogov theorem, $\spc{B}$ is $\Alex1$.
By \ref{thm:CBB/G}, the quotient space $\spc{A}\z=\spc{B}/\SSS^1$ is $\Alex1$;
evidently, $\spc{A}$ is 3-dimensional.

Denote by $F\subset \spc{B}$ the fixed point set of the $\SSS^1$-action.
Then $\chi(\spc{B})\z=\chi(F)$.
Each connected component of $F$ is either an isolated point or a 2-dimensional geodesic submanifold in $\spc{B}$;
the latter has to have positive curvature, and therefore it is homeomorphic to $\SSS^2$ or $\RP^2$.
Notice that 
\begin{itemize}
 \item each isolated point contributes 1 to the Euler characteristic of~$\spc{B}$,
 \item each sphere contributes 2 to the Euler characteristic of $\spc{B}$, and
 \item each projective plane contributes 1 to the Euler characteristic of~$\spc{B}$.
\end{itemize}
Since $\chi(\spc{B})=4$, we are in one of the following three cases:
\begin{enumerate}
 \item\label{case1} $F$ has exactly 4 isolated points,
 \item\label{case2} $F$ has one 2-dimensional submanifold and at least 2 isolated points,
 \item\label{case3} $F$ has at least two 2-dimensional submanifolds.
\end{enumerate}
In each case we will arrive at a contradiction.

\parit{Case \ref{case1}.}
Suppose $F$ has exactly 4 isolated points $x_1$, $x_2$, $x_3$, and $x_4$.
Denote by $y_1$, $y_2$, $y_3$, and $y_4$ the corresponding points in $\spc{A}$.
Note that $\Sigma_{y_i}\spc{A}$ is isometric to a quotient of $\SSS^3$ by an isometric $\SSS^1$-action without fixed points.

By \ref{ex:S^3/S^1}, each angle $\mangle\hinge{y_i}{y_j}{y_k}\le \tfrac\pi2$ for any three distinct points 
$y_i$, $y_j$, $y_k$.
In particular, all four triangles $[y_1y_2y_3]$, $[y_1y_2y_4]$, $[y_1y_3y_4]$, and $[y_2y_3y_4]$ are nondegenerate.
By the comparison, the sum of angles in each triangle is strictly greater than $\pi$.

Denote by $\omega$ the sum of all 12 angles in the 4 triangles $[y_1y_2y_3]$, $[y_1y_2y_4]$, $[y_1y_3y_4]$, and $[y_2y_3y_4]$.
From above,
\[\omega>4\cdot\pi.\]

On the other hand, by \ref{ex:S^3/S^1} any triangle in $\Sigma_{y_1}\spc{A}$ has perimeter at most $\pi$.
In particular, 
\[\mangle\hinge{y_1}{y_2}{y_3}+\mangle\hinge{y_1}{y_3}{y_4}+\mangle\hinge{y_1}{y_4}{y_2}\le \pi.\]
Apply the same argument in $\Sigma_{y_2}\spc{A}$, $\Sigma_{y_3}\spc{A}$, and $\Sigma_{y_4}\spc{A}$;
adding the results, we get 
\[\omega\le 4\cdot\pi\]
--- a contradiction.

\parit{Case \ref{case2}.}
Suppose $F$ contains one surface $S$.
Then the projection of $S$ to $\spc{A}$ forms its boundary $\partial \spc{A}$.
The doubling $\spc{W}$ of $\spc{A}$ across its boundary has at least 4 singular points --- each singular point of $\spc{A}$ corresponds to two singular points of $\spc{W}$.

By the doubling theorem, $\spc{W}$ is a $\Alex1$ space.
Therefore we arrive at a contradiction in the same way as in the first case.

\parit{Case \ref{case3}.} Impossible by \ref{ex:bry-connected}.
\qeds

\section{Erdős' problem rediscovered}

A point $p$ in an Alexandrov space is called \index{extremal point}\emph{extremal} if $\mangle\hinge pxy\le \tfrac\pi2$ for any hinge $\hinge pxy$ with the vertex at $p$; equivalently, $\diam \Sigma_p\le \pi/2$.

\begin{thm}{Theorem}\label{thm:extr-point}
Let $\spc{A}$ be a compact $m$-dimensional $\Alex0$ space.
Then it has at most $2^m$ extremal points.
\end{thm}

\parit{Proof of \ref{thm:extr-point}.}
Let $\{p_1,\dots,p_N\}$ be extremal points in $\spc{A}$.
For each $p_i$ consider its open \index{Voronoi domain}\emph{Voronoi domain} $V_i$; that is, 
\[V_i=\set{x\in \spc{A}}{\dist{p_i}{x}{}<\dist{p_j}{x}{}\ \text{for any}\ j\not=i}.\]
Clearly $V_i\cap V_j=\emptyset$ if $i\not=j$.

Suppose  $0<\alpha\le 1$.
Given a point $x\in\spc{A}$, choose a geodesic $[p_ix]$ and denote by $x_i$ the point on $[p_ix]$ such that $\dist{p_i}{x_i}{}=\alpha\cdot\dist{p_i}{x}{}$;
let $\map_i\:x\to x_i$ be the corresponding map.
By the comparison, 
\[\dist{x_i}{y_i}{}\ge\alpha\cdot \dist{x}{y}{}\]
for any $x$, $y$, and $i$.
Therefore 
\[\vol(\map_i \spc{A})\ge\alpha^m\cdot\vol \spc{A}.\]

Suppose $\alpha<\tfrac12$.
Then $x_i\in V_i$ for any $x\in \spc{A}$.
Indeed, assume $x_i\notin V_i$,
then there is $p_j$ such that $\dist{p_i}{x_i}{}\ge\dist{p_j}{x_i}{}$.
Then by comparison, we have $\angk{p_j}{p_i}{x}_{\EE^2}>\tfrac\pi2$;
that is, $p_j$ is not an extremal point.

It follows that $\vol V_i\ge\alpha^m\cdot\vol \spc{A}$
for any $0<\alpha<\tfrac12$; hence 
\[\vol V_i\ge\tfrac1{2^m}\cdot\vol \spc{A}.\]
Since $V_1,\dots,V_N$ are disjoint subsets of $\spc{A}$, we have $N\le 2^m$.
\qeds


\section{Crystallographic actions}

An isometric action $\Gamma\acts \EE^m$ is called \index{crystallographic action}\emph{crystallographic} if it is 
\index{properly discontinuous}\emph{properly discontinuous} (that is, for any compact set $K\subset \EE^m$ and $x\z\in \EE^m$ there are only finitely many elements $g\in \Gamma$ such that $g\cdot x\in K$) and \emph{cocompact} (that is, the quotient space $\spc{A}=\EE^m/\Gamma$ is compact).

Let $F$ be a maximal finite subgroup of $\Gamma$;
that is, if $F<H<\Gamma$ for a finite group $H$, then $F=H$.
Denote by $\mathfrak{M}(\Gamma)$ the number of maximal finite subgroups of $\Gamma$ up to conjugation.

\begin{thm}{Open question}
Let $\Gamma\acts \EE^m$ be a crystallographic action.
Is it true that $\mathfrak{M}(\Gamma)\le 2^m$?
\end{thm}

Note that any finite subgroup $F$ of $\Gamma$ fixes an affine subspace $A_F$ in $\EE^m$.
If $F$ is maximal, then $A_F$ completely describes $F$.
Indeed, since the action is properly discontinuous, the subgroup of $\Gamma$ that fix $A_F$ has to be finite.
This subgroup must contain $F$, but since $F$ is maximal, it must coinside with $F$. 

Denote by $\mathfrak{M}_k(\Gamma)$ the number of maximal finite subgroups $F<\Gamma$ (up to conjugation) such that $\dim A_F=k$.

Choose a finite subgroup $F<\Gamma$; consider a conjugate subgroup $F'=g \cdot F \cdot g^{-1}$.
Note that $A_{F'}=g\cdot A_F$.
In particular, the subspaces $A_F$ and $A_{F'}$ have the same image in the quotient space $\spc{A}=\EE^m/\Gamma$.
Therefore, to count subgroups up to conjugation, we need to count the images of their fixed sets.
By the lemma below (\ref{lem:extr/G}), $\mathfrak{M}_0(\Gamma)$ cannot exceed the number of extremal points in $\spc{A}=\EE^m/\Gamma$.
Combining this observation with \ref{thm:extr-point}, we get the following.

\begin{thm}{Proposition}\label{prop:2m}
Let $\Gamma\acts \EE^m$ be a crystallographic action.
Then $\mathfrak{M}_0(\Gamma)\le 2^m$.
\end{thm}

\begin{thm}{Lemma}\label{lem:extr/G}
Let $\Gamma\acts \EE^m$ be a crystallographic action and $F$ be a maximal finite subgroup of $\Gamma$ that fixes an isolated point $p$.
Then the image of $p$ in the quotient space $\spc{A}=\EE^m/\Gamma$ is an extremal point.
\end{thm}

\parit{Proof.}
Let $q$ be the image of $p$.
Suppose $q$ is not extremal;
that is, $\mangle \hinge q{y_1}{y_2}>\tfrac\pi2$ for some hinge $\hinge q{y_1}{y_2}$ in $\spc{A}$.

Choose the inverse images $x_1,x_2\in \EE^m$ of $y_1,y_2\in \spc{A}$ such that $\dist{p}{x_i}{\EE^m}=\dist{q}{y_i}{\spc{A}}$.
Note that $\mangle \hinge p{x_1}{x_2}\ge \mangle \hinge q{y_1}{y_2}>\tfrac\pi2$.
Moreover, since $p$ is fixed by $F$, we have
\[\mangle \hinge p{x_1}{g\cdot x_2}>\tfrac\pi2
\eqlbl{eq:>pi/2}\]
for any $g\in F$.

Denote by $z$ the barycenter of the orbit $F\cdot x_2$.
Note that $z$ is a fixed point of $F$.
By \ref{eq:>pi/2}, $z\ne p$;
so $F$ must fix the line $pz$.
But $p$ is an isolated fixed point of $F$ --- a contradiction.
\qeds

\begin{thm}{Exercise}\label{ex:number(m-1)}
Let $\Gamma\acts \EE^m$ be a crystallographic action.
Show that
\begin{subthm}{ex:number(m-1):2}
$\mathfrak{M}_{m-1}(\Gamma)\le 2$, and
\end{subthm}

\begin{subthm}{ex:number(m-1):1}
if $\mathfrak{M}_{m-1}(\Gamma)=1$, then $\mathfrak{M}_0(\Gamma)\le 2^{m-1}$.
\end{subthm}

Construct  crystallographic actions with equalities in \ref{SHORT.ex:number(m-1):2} and \ref{SHORT.ex:number(m-1):1}.
\end{thm}

\section{Remarks}

Submetries were introduced by Valerii Berestovskii \cite{berestovskii1987} and have attracted attention in various contexts of differential and metric geometry.



A more general form of Theorem \ref{thm:hsiang-kleiner} was found by Karsten Grove and Burkhard Wilking \cite{grove-wilking};
it classifies isometric $\SSS^1$ actions on  4-dimensional manifolds with nonnegative sectional curvature.
This proof is as beautiful as the original work of Wu-Yi Hsiang and Bruce Kleiner.

It is expected that \textit{no $\Alex1$ space with a nontrivial isometric $\SSS^1$-action can be homeomorphic to $\SSS^2\times\SSS^2$};
so \ref{thm:hsiang-kleiner} holds for general $\Alex1$ space.
The proof of \ref{thm:hsiang-kleiner} would work if we had the following generalization of \ref{lem:S^3/S^1};
see \cite{harvey-searle}.

\begin{thm}{Open question}
Let $\Sigma$ be an $\Alex1$ space homeomorphic to $\SSS^3$.
Suppose $\SSS^1$ acts on $\Sigma$ isometrically and without fixed points.
Is it true that any triangle in $\Sigma/\SSS^1$ has perimeter at most $\pi$?

And if the answer is, is there a distance-noncontracting map
\[\Sigma/\SSS^1\z\to \tfrac12\cdot\SSS^2?\]
\end{thm}


\begin{thm}{Advanced exercise}\label{ex:S1actsS3}
Suppose $\SSS^1$ acts isometrically on an $\Alex1$ space $\spc{A}$ that is homeomorphic to $\SSS^3$.
Assume its fixed-point set is a closed local geodesic $\gamma$.
Show that
\[\length\gamma\le2\cdot\pi.\]
\end{thm}

An analogous question for a $\ZZ_2$-action is open \cite{petrunin-involution}.

Theorem \ref{thm:extr-point} is a translation of the following classical problem in discrete geometry to Alexandrov's language.

\begin{thm}{Problem}\label{erdos-problem}
Let $F$ be a set of points in $\EE^m$ such that any triangle formed by three distinct points in $F$ has no obtuse angles.
Then  $|F|\le2^m$.
Moreover, if $|F|=2^m$, then $F$ consists of the vertices of an $m$-dimensional rectangle.
\end{thm}

This problem was posed by Paul Erdős \cite{erdos} and solved by Ludwig Danzer and Branko Gr\"unbaum \cite{danzer-gruenbaum}.
Grigory Perelman noticed that, after proper definitions, the same proof works in Alexandrov spaces \cite{perelman-Erdos}; thus, it proves \ref{thm:extr-point}.
Applying the our argument to the convex hull of $F$ in \ref{erdos-problem} proves that $|F|\le 2^m$;
the case of equality requires more work.

Compact $m$-dimensional $\Alex0$ spaces with the maximal number of extremal points include $m$-dimensional rectangles and the quotients of flat tori by reflections across a point.
(This action has $2^m$ isolated fixed points; each corresponds to an extremal point in the quotient space $\spc{A}=\TT^m/\ZZ_2$.)
Nina Lebedeva has proved \cite{lebedeva2015} that \textit{every $m$-dimensional $\Alex0$ space with $2^m$ extremal points is a quotient of Euclidean space by a crystallographic action}.

The extremal subsets of Alexandrov space were brifly discussed in \ref{sec:bry-remarks}.
The following definition is more relevant to isometric group actions.

A closed subset $E$ in a finite-dimensional Alexandrov space is called
\index{extremal set}\emph{extremal} if $\mangle\hinge pxy\z\le \tfrac\pi2$ for any $x\notin E$ and $p\in E$ such that $\dist{x}{p}{}$ takes a minimal value.
An extremal set is called \index{minimal extremal set}\emph{minimal} if it contains no proper extremal subsets.

For example, the whole space and the empty set are extremal.
Also, every vertex, edge, or face (as well as their unions) of the cube is an extremal subset of the cube.
Vertices of the cube are its only minimal extremal subsets.

Counting maximal finite subgroups in a crystallographic group $\Gamma$ (up to conjugation) is equivalent to counting the minimal extremal subsets in the quotient space $\spc{A}=\EE^m/\Gamma$.
So, \ref{prop:2m} would follow from the next conjecture.

\begin{thm}{Conjecture}
Any $m$-dimensional compact $\Alex0$ space has at most $2^m$ minimal extremal subset.
\end{thm}

Let us mention another related conjecture.
An extremal set is called \index{primitive extremal set}\emph{primitive} if it contains no proper extremal subsets with nonempty relative interior.
For example, each face of $m$-dimensional cube is its primitive extremal subset;
therefore the cube has exactly $3^m$ primitive extremal subset, including the empty set and the whole cube.

\begin{thm}{Conjecture}
Any $m$-dimensional compact $\Alex0$ space has at most $3^m$ minimal extremal subset.
\end{thm}

Some crude estimates on number of extremal subsets follow from the idea in Gromov's Betti number theorem \ref{thm:betti}.


%\chapter{CBB: definition}

\section{Distances and geodesics}

\parbf{Distances.}
The distance between two points $x$ and $y$ in a metric space $\spc{X}$ will be denoted by $\dist{x}{y}{}$ or $\dist{x}{y}{\spc{X}}$.
The latter notation is used if we need to emphasize 
that the distance is taken in the space~${\spc{X}}$.
The function $(x,y)\mapsto \dist{x}{y}{\spc{X}}$ is called \index{metric}\emph{metric};
it has to meet the following conditions for any three points $x,y,z\in \spc{X}$:

\begin{subthm}{metric>=0}
$\dist{x}{y}{\spc{X}}\ge 0$,
\end{subthm}

\begin{subthm}{metric=0} $\dist{x}{y}{\spc{X}}= 0$ $\iff$ $x=y$,
\end{subthm}

\begin{subthm}{metric:sym} $\dist{x}{y}{\spc{X}}=\dist{y}{x}{\spc{X}}$,
\end{subthm}

\begin{subthm}{metric:triangle} $\dist{x}{y}{\spc{X}}+\dist{y}{z}{\spc{X}}\ge\dist{x}{z}{\spc{X}}$.
\end{subthm}

\parbf{Geodesics.}
Let $\II$\index{$\II$} be a real interval. 
A distance-preserving map $\gamma$ from $\II$ to a metric space $\spc{X}$ is called a \index{geodesic}\emph{geodesic}%
\footnote{Others call it differently: \textit{shortest path}, \textit{minimizing geodesic}.
Also, note that the meaning of the term \textit{geodesic} is different from what is used in Riemannian geometry, altho they are closely related.}; 
in other words, $\gamma\:\II\to \spc{X}$ is a geodesic if 
\[\dist{\gamma(s)}{\gamma(t)}{\spc{X}}=|s-t|\]
for any pair $s,t\in \II$.

If $\gamma\:[a,b]\to \spc{X}$ is a geodesic such that $p=\gamma(a)$, $q=\gamma(b)$, then we say that $\gamma$ is a geodesic from $p$ to $q$.
In this case, the image of $\gamma$ is denoted by $[p q]$\index{$[{*}{*}]$}, and, with abuse of notations, we also call it a \index{geodesic}\emph{geodesic}.
We may write $[p q]_{\spc{X}}$ 
to emphasize that the geodesic $[p q]$ is in the space  ${\spc{X}}$.

In general, a geodesic from $p$ to $q$ need not exist and if it exists, it need not  be unique.  
However, once we write $[p q]$ we assume that we have chosen such geodesic.

\parbf{Geodesic path.}
A \index{geodesic path}\emph{geodesic path} is a geodesic with constant-speed parameterization by the unit interval $[0,1]$.

\parbf{Geodesic space.}
A metric space is called \index{geodesic space}\emph{geodesic} if any pair of its points can be joined by a geodesic.

\section{Baby Toponogov}

Recall that \index{polyhedral space}\emph{polyhedral space} is a geodesic space that admits a finite triangulation such that each simplex is isometric to a simplex in a Euclidean space.
If, in addition, it is homeomorphic to a surface (without boundary), then it is called a \index{polyhedral surface}\emph{polyhedral surface}.
A point on a polyhedral surface with nonzero curvature is called an \index{essential vertex}\emph{essential vertex}.
Any other point on the surface will be called \index{regular point}\emph{regular}.
Note that \textit{any regular point has a neighborhood that is isometric to an open set in the Euclidean plane}.

\begin{thm}{Exercise}\label{ex:poly+geod}
Let $P$ be a non-negatively curved polyhedral surface.

\begin{subthm}{}
Show that a geodesic in $P$ cannot pass thru an essential vertex.
\end{subthm}

\begin{subthm}{}
Show that if two geodesics in $P$ intersect at two points, 
then these are the endpoints for both geodesics.
\end{subthm}

\end{thm}

The next theorem gives a global geometric property of non-negatively curved polyhedral surfaces.

Given a hinge $\hinge pxy$ in a non-negatively curved polyhedral surface $P$, denote by $\mangle\hinge pxy$ the minimal angle that the hinge cuts from $P$ at~$p$.
(Soon we will give a more general definition of $\mangle\hinge pxy$; see \ref{sec:angles}.)

\begin{thm}{Theorem}\label{thm:poly-cbb}
Let $P$ be a polyhedral surface.
Assume $P$ has non-negative curvature at each point (see \ref{sec:Alexandrov-existence}).
Then 
\[\mangle\hinge pxy\ge\angk pxy\]
for any hinge $\hinge pxy$ in $P$.
\end{thm}

The following exercise will be used in the proof.

\begin{thm}{Exercise}\label{ex:concave-loc}
Let $f\:[0,\ell]\to\RR$ be a continuous function such that for any $t\in \left]0,\ell\right[$ there is a linear function $h$ that locally supports $f$ from above;
that is, $h(t_0)=f(t_0)$, and there is $\eps>0$ such that $h(t)\ge f(t)$ if $|t-t_0|<\eps$.
Show that $f$ is concave.
\end{thm}


\parit{Proof.}
Let $[pxy]$ be a triangle in $P$ and let $[\tilde p\tilde x\tilde y]$ be the model triangle of $[pxy]$.
Set $\ell=|x-y|_P=|\tilde x-\tilde y|_{\EE^2}$.

Denote by $\gamma(t)$ and $\tilde \gamma(t)$ the geodesics $[xy]$ and $[\tilde x\tilde y]$ parametrized by length starting from $x$ and $\tilde x$, respectively.
Observe that it is sufficient to show that 
$$| p- \gamma(t)|\le|\tilde p-\tilde \gamma(t)| 
\eqlbl{eq:comp-gamma}$$
for any $t$ in $[0,\ell]$.

We may assume that $p$ is a regular point;
otherwise, move it slightly and apply approximation.


From the cosine law, we get that the function 
$$\tilde f(t)=|\tilde p-\tilde \gamma(t)|^2-t^2$$
is linear.
Consider the function
$$f(t)=|p- \gamma(t)|^2-t^2.$$
Note that $f(0)=\tilde f(0)$, $f(\ell)=\tilde f(\ell)$, and the inequality~\ref{eq:comp-gamma} is equivalent to
$$f(t)\ge \tilde f(t).
\eqlbl{eq:comp-f}$$
By Jensen's inequality, \ref{eq:comp-f} holds if $f$ is concave.

By \ref{ex:poly+geod}, 
$\gamma(t_0)$ is regular.
Since $p$ is regular,
a geodesic $[p\gamma(t)]$ contains only regular points.
Therefore for small $\eps>0$,
 the $\eps$-neighborhood of $[p\gamma(t)]$, say $\Omega$, contains only regular points. 
We may assume that $\Omega$ is homeomorphic to a disc;
in this case, there is a locally distance-preserving embedding $\iota\:\Omega\to\EE^2$.
Note the image $\iota[p\gamma(t)]$ is a line segment that 
and $\iota(\Omega)$ is the $\eps$-neighborhood of $\iota[p\gamma(t)]$ in $\EE^2$;
in particular, $\iota(\Omega)$ is convex.
Thus $\iota(\Omega)$ contains a triangle with  base $\iota[\gamma(t_0-\eps)\ \gamma(t_0+\eps)]$  and vertex $\iota(p)$.

Clearly, for any $t\in[t_0-\eps,t_0+\eps]$ 
we have 
$$|\iota(p)-\iota(\gamma(t))|\ge|p-\gamma(t)|.$$
Note that
the function
$$h(t)= |\iota(p)-\iota(\gamma(t))|^2-t^2$$
is linear.
From above, $h$ supports $f$ locally  at $t_0$.
It remains to apply~\ref{ex:concave-loc}.
\qeds

\section{Definition}

\begin{thm}{Definition}\label{def:CBB}
A metric space $\spc{X}$ has {}\emph{nonnegative curvature} in the sense of Alexandrov if the inequality 
\[\angk  pxy_{\EE^2}+\angk pyz_{\EE^2}+\angk pzx_{\EE^2}
\le 
2\cdot\pi
\eqlbl{eq:CBB-comparison}\]
holds for any quadruple $p,x,y,z\in\spc{X}$ such that each model angle in \ref{eq:CBB-comparison} is defined. 

The inequality \ref{eq:CBB-comparison} is called \index{4-point comparison}\emph{4-point comparison} for the quadruple $p,x,y,z$.
If instead of $\EE^2$, we use $\SSS^2$ or $\HH^2$, then we get the definition of
$\CBB(1)$ and $\CBB(-1)$ comparisons.
(Note that $\angk  pxy_{\EE^2}$ and $\angk  pxy_{\HH^2}$ are defined if $p\ne x$, $p\ne y$,
but for $\angk  pxy_{\SSS^2}$ we need in addition, $\dist{p}{x}{}+\dist{p}{y}{}+\dist{x}{y}{}<2\cdot\pi$.)

More generally, one may apply this definition to $\MM^2(\kappa)$ --- the model plane of curvature $\kappa$, defined as follows:
$\MM^2(0)=\EE^2$,
if $\kappa>0$, then $\MM^2(\kappa)$ is the sphere of radius $\tfrac{1}{\sqrt{\kappa}}$ and if $\kappa<0$, then it is Lobachevsky plane rescaled by factor $\tfrac{1}{\sqrt{-\kappa}}$.
This way we define $\CBB(\kappa)$ comparison for any real $\kappa$.
\end{thm}

While this definition can be applied to any metric space,
it is usually applied to geodesic spaces (or, at least, length spaces that will be defined later).

\begin{thm}{Exercise}
Show that Euclidean space $\EE^n$ is $\CBB(0)$.
\end{thm}


\begin{thm}{Exercise}\label{ex:polyCBB}
Show that a polyhedral surface is $\CBB(0)$ if and only if it has nonnegative curvature in the sense of \ref{sec:Alexandrov-existence}. 
\end{thm}





\section{Comments}

The first synthetic description of curvature is due to Abraham Wald \cite{wald}; 
it was given in a lone publication on a ``coordinateless description of Gauss surfaces'' published in 1936.
In 1941, similar definitions were rediscovered by Alexandr Alexandrov \cite{alexandrov:def}.

In Alexandrov's work, the first applications of this approach were given.
Mainly: the main part of \ref{thm:alexandrov+pogorelov} \cite{alexandrov-1941,alexandrov-1941convex}
and the {}\emph{gluing theorem} \cite{alexandrov-1946}, which gave a flexible tool to modify non-negatively curved metrics on a sphere.
These two results together formed the foundation of the branch of geometry now called {}\emph{Alexandrov geometry};
they gave  a very intuitive geometric tool to study embeddings and bending of surfaces in Euclidean space and changed the subject dramatically.

In particular, the existence of bending of a large spherical dome (sphere with a small disc removed) easily follows from these two theorems; moreover, it provides an intuitive description of such bending that can be extended to a closed convex surface.






%%%!TEX root = invitation-CBB.tex
\chapter{Polyhedral surfaces}\label{chap:alex-embedding}

In this lecture we discuss intrisic geometry of surfaces of convex polyhedra and convex bodies.
Furhter, we prove the Cauchy theorem, and then modify the proof to get the Alexandrov uniqueness theorem.

\section{Surface of convex polyhedron}

Let us define a \index{convex body}\emph{convex body} as a compact convex subset in $\EE^3$ with nonempty interior.
The \index{surface}\emph{surface} of a convex body is defined as its boundary equipped with the induced length metric.

\begin{thm}{Exercise}\label{ex:surf-S2}
Show that the surface of a convex body is homeomorphic to the 2-dimensional sphere.
\end{thm}

A \index{convex polyhedron}\emph{convex polyhedron} is a convex body with a finite number of extremal points, called its \index{vertex}\emph{vertices}.

The surface, say $P$, of a convex polyhedron $K$ admits a finite triangulation such that each triangle is isometric to a plane triangle.
In other words, $P$ is a closed \index{polyhedral surface}\emph{polyhedral surface};
that is, it is a 2-dimensional manifold with a length metric that admits a finite triangulation such that each triangle is isometric to a solid plane triangle.
A \index{triangulation}\emph{triangulation} of a polyhedral surface will always be assumed to satisfy this condition.

The total angle around a vertex $v$ in $P$ is defined as the sum of angles at $v$ of all triangles in the triangulation that contain $v$.

If a point $p\in P$ is not a vertex of $K$,
then
\begin{itemize}
\item $p$ lies in the interior of a face of $K$, and its neighborhood in $P$ is a piece of plane, or
\item $p$ lies on an edge, and its neighborhood is two half-planes glued along the boundary.
\end{itemize}
In both cases, a neighborhood of $p$ in $P$ (with the induced length metric) is isometric to an open domain of the plane.
In this case, the total angle around $p$ will be defined to be $2\cdot\pi$.

\begin{thm}{Claim}\label{clm:total-angle}
Let $P$ be the surface of a convex polyhedron $K$.
Then, the total angle around any point $p\in P$ cannot exceed $2\cdot\pi$.
\end{thm}

The proof relies on the triangle inequality for angles (or the spherical triangle inequality).
It follows from \ref{claim:angle-3angle-inq}, but our proof of this statement is a straightforward generalization of the argument in the classical geometry textbook \cite[§ 47]{kiselev-stereo-en} that proves the following statement.

\begin{thm}{Spherical triangle inequality}\label{ex:angle-triangle}
Let $w_1,w_2,w_3$ be unit vectors in $\EE^3$.
Denote by $\alpha_{i,j}$ the angle between the vectors $v_i$ and $v_j$.
Then
$$\alpha_{1,3}\le \alpha_{1,2}+\alpha_{2,3}.$$
Moreover, in the case of equality, the three solid angles spanned by $w_1$, $w_2$, and $w_3$ form a plane.
\end{thm}

\parit{Proof of \ref{clm:total-angle}.}
Consider the intersection of $K$ with a small sphere centered at~$p$;
it is a convex spherical polygon, say $F$.
Applying rescaling we may assume that the sphere has unit radius.
Then we need to show that the perimeter of $F$ does not exceed $2\cdot\pi$.

\begin{wrapfigure}{o}{22mm}
\vskip-4mm
\centering
\includegraphics{mppics/pic-1103}
\end{wrapfigure}

Note that $F$ lies in a hemisphere, say $H$.
Moreover, there is a decreasing sequence of convex spherical polygons
\[H=H_0\supset\dots\supset H_n=F,\]
such that $H_{i+1}$ is obtained from $H_{i}$ by cutting along a chord.

By the spherical triangle inequality (\ref{ex:angle-triangle}), we have
\[
2\cdot\pi=\perim H=\perim H_0\ge\dots\ge\perim H_n=\perim F
\]
--- hence the result.
\qeds

\section{Curvature}

Let $p$ be a point on a polyhedral surface, and $\theta_p$ is the total angle around $p$.
The value $2\cdot \pi -\theta_p$ is called the \index{curvature}\emph{curvature} of the polyhedral surface at $p$.

Note that if $p$ is not a vertex in a triangulation of $P$, then its curvature is zero.
A vertex of a triangulation of a polyhedral surface is called \index{essential vertex}\emph{essential} if its curvature is not $0$.

\begin{thm}{Exercise}\label{ex:vertex-essential-vertex}
Let $v$ be a point on the surface $P$ of a convex polyhedron $K$.
Show that $v$ is a vertex of $K$ if and only if
$v$ is an essential vertex of $P$.
\end{thm}


\begin{thm}{Exercise}\label{ex:geodesic-vertex}
Show that geodesics on a closed polyhedral surface with nonnegative curvature may have essential vertices only at their ends.
\end{thm}

\begin{thm}{Exercise}\label{pr:tetrahedron}
Assume that the surface of a nondegenerate tetrahedron $T$ has curvature $\pi$ at each of its vertices.
Show that

\begin{subthm}{pr:tetrahedron:=}
all faces of $T$ are congruent;
\end{subthm}

\begin{subthm}{pr:tetrahedron:perp} the line containing midpoints of opposite edges of $T$ intersects these edges at right angles.
\end{subthm}

\end{thm}

\begin{thm}{Exercise}\label{ex:gauss-bonnet}
Show that sum of curvatures of a closed polyhedral surface $P$ equals to $2\cdot\pi\cdot\chi(P)$,
where $\chi(P)$ denotes the Euler characteristic of $P$.
\end{thm}


Claim~\ref{clm:total-angle} says that \textit{surfaces of convex polyhedra have nonnegative curvature} in the sense of the above definition.
Now we show that this definition agrees with the 4-point comparison.

\begin{thm}{Proposition}\label{prop:poly-CBB}
A polyhedral surface with nonnegative curvature at each vertex is $\Alex0$.
\end{thm}

\parit{Proof.}
Denote the surface by $P$.
By \ref{comp-kappa}, it is sufficient to check that
$\distfun_p^2\circ\gamma$ is 1-concave for any geodesic $\gamma$ and any point $p$ in $P$.

We can assume that $p$ is not a vertex;
the vertex case can be done by approximation.
By \ref{ex:geodesic-vertex}, $\gamma$ does not contain vertices.

Given a point $x=\gamma(t_0)$, choose a geodesic $[px]$.
Again, by \ref{ex:geodesic-vertex}, $[px]$ does not contain vertices.
Therefore a small neighborhood of $U\supset [px]$ can be unfolded on a plane;
that is, there is an injective length-preserving map $z\mapsto \tilde z$
of $U$ into the Euclidean plane.
This way we map part of $\gamma$ in $U$ to a line segment $\tilde\gamma$.
Let
\[\tilde f(t)\df\tfrac12\cdot\distfun_{\tilde p}^2\circ\tilde \gamma(t).\]
Since the geodesic $[px]$ maps to a line segment, we have $\tilde f(t_0)= f(t_0)$.
Furthermore, since the unfolding $z\mapsto \tilde z$ preserves lengths of curves, we get
$\tilde f(t)\ge f(t)$ if $t$ if the left-hand side is defined.
That is, $\tilde f$ is a local upper barrier of $f$ at $t_0$.
Evidently, $\tilde f''\equiv 1$; therefore $f''\le 1$.
It remains to apply \ref{comp-kappa}.
\qeds

\begin{thm}{Exercise}\label{ex:poly-CBB}
Prove the converse to the proposition;
that is, show that if a poyhedral surface is $\Alex0$, then it has nonnegative curvature in the sense defined in this section.
\end{thm}

\section{Surface of convex body}

\begin{thm}{Advanced exercise}\label{ex:surface-covergence}
Let $K_1,K_2,\dots,$ and $K_\infty$ be convex bodies in $\EE^m$.
Denote by $P_n$ the surface of $K_n$ with induced length metric.
Suppose $K_n\z\to K_\infty$ in the sense of Hausdorff.
Show that $P_n\to P_\infty$ in the sense of Gromov--Hausdorff.
\end{thm}

Any convex body is a Hausdorff limit of a sequence of convex polyhedra.
Therefore, the next proposition follows from \ref{prop:poly-CBB}, \ref{ex:surface-covergence}, and \ref{thm:CBB-closed}.

\begin{thm}{Proposition}\label{prop:conv-surf-CBB(0)}
The surface of a convex body in $\EE^3$ is $\Alex0$.
\end{thm}

\begin{thm}{Very advanced exercise}\label{ex:liberman+milka}
Let $P$ be the surface of a nondegenerate convex body $K\subset\EE^3$;
we assume that $P$ is equipped with the induced length metric.

\begin{subthm}{ex:liberman+milka:liberman}
Show that any geodesic $\gamma$ in $P$ is one-sided differentiable as a curve in $\EE^3$.
\end{subthm}

\begin{subthm}{ex:liberman+milka:convex}
Suppose a plane $\Pi$ cuts from $P$ a disc $\Delta$, and the reflection of $\Delta$ across $\Pi$ lies in $K$.
Show that $\Delta$ is a convex subset of $P$;
that is, if a geodesic has ends in $\Delta$, then it completely lies in $\Delta$.
\end{subthm}


\begin{subthm}{ex:liberman+milka:milka}
Let $\gamma_1$ and $\gamma_2$ be geodesic paths in $P$ that start at one point $p\z=\gamma_1(0)\z=\gamma_2(0)$.
Suppose $x_i=\gamma_i(1)$, and $y_i\z=p+\gamma_i^+(0)$.
Show that
\[\dist{x_1}{x_2}{P}\le \dist{y_1}{y_2}{W},\]
where $W$ is the complement to the interior of $K$.
\end{subthm}

\end{thm}


\section{Cauchy theorem}

Recall that \textit{surfaces} of convex polyhedrons are considered with the induced length metric.
 
\begin{thm}{Theorem}\label{thm:cauchy}
Let $K$ and $K'$ be convex polyhedrons in $\EE^3$;
denote their surfaces 
by $P$ and $P'$.
Suppose there is an isometry $P\to P'$ that sends each face of $K$ to a face of $K'$.
Then $K$ is congruent to $K'$; moreover the isometry $P\to P'$ can be extended to a motion of $\EE^3$ that maps $K$  to $K'$.
\end{thm}

\parit{Proof modulo two lemmas.}
Consider the graph $\Gamma$ formed by the edges of $K$;
the edges of $K'$ form a naturally isomorphic graph.
 
For an edge $e$ in $\Gamma$, denote by $\alpha_e$ and $\alpha'_e$ the dihedral angles in $K$ and $K'$, respectively.
Mark $e$ by plus if $\alpha_e < \alpha'_e$ and by minus if $\alpha_e > \alpha'_e$.

Let us remove from $\Gamma$ everything that is not marked;
that is, leave only the edges marked by $(+)$ or $(-)$ and their endpoints.
If $\Gamma$ is an empty graph, then the theorem follows.
Let us assume the contrary.

The graph $\Gamma$ is embedded into $P$, which is homeomorphic to the sphere.
In particular, the edges coming from one vertex have a natural cyclic order. 
Given a vertex $v$ of $\Gamma$, count the \textit{number of sign changes} around $v$;
that is, the number of consequent pairs edges with different signs. 

\begin{thm}{Local lemma}\label{lem:local}
For any vertex of  $\Gamma$ the number of sign changes is at least $4$.
\end{thm}

In other words, at each vertex of $\Gamma$, one can choose 4 edges marked by $(+)$, $(-)$, $(+)$, $(-)$ in the same cyclical order.
Note that the local lemma contradicts the following.

\begin{thm}{Global lemma}\label{lem:global}
Let $\Gamma$ be a nonempty planar graph.
Then it is impossible to mark all of the edges of $\Gamma$ by $(+)$ or $(-)$
in such a way  that the number of sign changes around each vertex of $\Gamma$ is at least $4$.
\end{thm}

It remains to prove these two lemmas.
\qeds


\section{Arm lemma}

\begin{thm}{Lemma}\label{lem:arm}
Assume that $A=[a_0 a_1\dots a_n]$ is a convex polygon in $\EE^2$
and $A'=[a'_0 a'_1\dots a'_n]$ be a polygonal line in $\EE^3$
such that 
$|a_i-a_{i+1}|=|a'_i-a'_{i+1}|$ for any $i\in\{0,\dots,n-1\}$
and 
$\measuredangle a_i\le \measuredangle a'_i$
for each $i\in\{1,\dots,n-1\}$.
Then 
$$|a_0-a_n|\le |a'_0-a'_n|$$
and equality holds if and only if $A$ is congruent to $A'$.
\end{thm}

One may view the polygonal lines $[a_0a_1\dots a_n]$ and $[a'_0a'_1\dots a'_n]$ as a robot's arm in two positions.
Informally speaking, the arm lemma says that when the arm opens,
the distance between the shoulder and tip of a finger increases;
assuming that starting position a convex plane polygon.

\begin{thm}{Exercise}\label{ex:arm-nonconvex}
Show that the arm lemma does not hold if 
instead of the convexity,
one only the local convexity;
that is, if we assume only that the polygonal line $a_0 a_1\dots a_n$ turns only left.
\end{thm}

\begin{thm}{Exercise}\label{ex:cauchy}
Suppose $A=[a_1\dots a_n]$ and $A'=[a'_1\dots a'_n]$ be noncongruent convex plane polygons with equal corresponding sides.
Mark each vertex $a_i$ with plus (minus) if the interior angle of $A$ at $a_i$ is smaller (respectively bigger) than the interior angle of $A'$ at $a_i'$.
Show that there are at least 4 sign changes around $A$. %+PIC

Give an example showing the statement does not hold without assuming convexity.

\end{thm}

\parit{Proof.}
We will view $\EE^2$ as the $xy$-plane in~$\EE^3$; 
so both $A$ and $A'$ lie in~$\EE^3$.

Let $a_m$ be the vertex of $A$ that lies on the maximal distance to the line $(a_0a_n)$.
Let us shift indexes of $a_i$ and $a'_i$ down by $m$,
so that 
\begin{align*}
a_{-m}&:=a_0,
&&\dots
&
a_{0}&:=a_m,
&&\dots
&
a_k&:=a_n,
\\
a'_{-m}&:=a'_0,
&&\dots
&
a'_{0}&:=a'_m,
&&\dots
&
a'_k&:=a'_n,
\end{align*}
where $k=n-m$.
(Here the symbol ``$:=$'' means an assignment as in programming.)

Without loss of generality, we may assume that
\begin{itemize}
\item $a_0=a'_0$ and they both coincide with the origin in $\EE^3$;
\item all $a_i$ lie in the $xy$-plane and the $x$-axis is parallel to the line $a_{-m}a_k$;
\item the angle $\measuredangle a'_0$ lies in $xy$-plane and contains the angle $\measuredangle a_0$ inside so that the directions to $a'_{-1}$,$a_{-1}$, $a_{1}$ and $a'_{1}$ from $a_0$ appear in the same cyclic order.
\end{itemize}

Denote by $x_i$ and $x'_i$ the projections of $a_i$ and $a'_i$ to the $x$-axis.
We can assume in addition that $x_k\ge x_{-m}$.
In this case,
$$|a_k-a_{-m}|=x_k-x_{-m}.$$
Since the projection is a distance non-expanding, we also have
$$|a'_k-a'_{-m}|\ge x'_k-x'_{-m}.$$ 

\begin{wrapfigure}{r}{60mm}
\vskip-5mm
\centering
\includegraphics{mppics/pic-30}
\vskip3mm
\end{wrapfigure}

Therefore it is sufficient to show
that 
$$x'_k-x'_{-m}\ge x_k-x_{-m}.$$
The latter holds if
$$x'_i-x'_{i-1}\ge x_i-x_{i-1}.\eqlbl{eq:|bb|=<|aa|}$$
for each $i$.
It remains to prove \ref{eq:|bb|=<|aa|}.

Let us assume that $i>0$; 
the case $i\le 0$ is similar.
Denote by $\sigma_i$ ($\sigma'_i$) the angle between the vector $w_i=a_{i}-a_{i-1}$ (respectively $w_i'=a'_{i}-a'_{i-1}$) and the $x$-axis.
Note that
$$\begin{aligned}
x_i-x_{i-1}&=|a_i-a_{i-1}|\cdot\cos\sigma_i,
\\
x'_i-x'_{i-1}&=|a_i-a_{i-1}|\cdot\cos\sigma'_i
\end{aligned}
\eqlbl{eq:proj}$$
for each $i>0$.
By construction $\sigma_1\ge \sigma'_1$.
Note that $\measuredangle (w_{i-1},w_i)\z=\pi -\measuredangle a_i$.
From convexity of $[a_1 a_1\dots a_i]$, we have
$$\sigma_i=\sigma_1+(\pi-\measuredangle a_1)+\dots+(\pi-\measuredangle a_i)$$
 for any $i>0$.
Since $\measuredangle (w'_{i-1},w'_i)=\pi -\measuredangle a'_i$,
applying the triangle inequality for angles (\ref{ex:angle-triangle}) several times,
we get
$$\sigma'_i\le\sigma'_1+(\pi-\measuredangle a'_1)+\dots+(\pi-\measuredangle a'_i).$$
Since $\measuredangle a'_j\ge \measuredangle a_j$ for each $j$, we get
$\sigma'_i\le \sigma_i$, and therefore
\[\cos \sigma'_i\ge \cos\sigma_i\]
Applying \ref{eq:proj}, we get \ref{eq:|bb|=<|aa|}.

In the case of equality, we have $\sigma_i=\sigma'_i$,
which implies $\measuredangle a_i=\measuredangle a'_i$ for each $i$.
This also implies that all $A'$ is a convex polygonal line in the $xy$-plane.
The latter easily follows from the equality case in \ref{ex:angle-triangle}.
\qeds

\begin{thm}{Advanced exercise}\label{ex:arm'}
Let $A$ and $A'$ be as in the arm lemma (\ref{lem:arm}).

\begin{subthm}{ex:bow'+}
Suppose that $\measuredangle \hinge{a_n}{a_{n-1}}{a_0}\le\tfrac\pi2$.
Show that $\measuredangle \hinge{a_0}{a_1}{a_n}\ge \measuredangle \hinge{a_0'}{a_1'}{a_n'}$.
\end{subthm}

\begin{subthm}{ex:bow'-} Show that the inequality $\measuredangle \hinge{a_0}{a_1}{a_n}\ge \measuredangle \hinge{a_0'}{a_1'}{a_n'}$ does not hold in general.
\end{subthm}

\end{thm}

\section{Proof of local lemma}
 
\parit{Proof of the local lemma (\ref{lem:local}).}
Assume that the local lemma does not hold at the vertex $v$ of $\Gamma$.
Choose a plane that cuts from $P$ a small pyramid $\Delta$ with the vertex~$v$.
One can choose two points $a$ and $b$ on the base of $\Delta$
so that on one side of the segments $[va]$ and $[vb]$ we have only pluses
and on the other side only minuses.

The base of $\Delta$ has two polygonal lines with ends at $a$ and $b$.
Choose the one that has only pluses;
denote it by $a_0 a_1 \dots a_n$;
so $a=a_0$ and $b=a_n$.
Denote by $a'_0 a'_1 \dots a'_n$
the corresponding line in $P'$;
let $a'=a'_0$ and $b'=a'_n$.

{

\begin{wrapfigure}{r}{40mm}
\vskip-0mm
\centering
\includegraphics{mppics/pic-40}
\vskip-0mm
\end{wrapfigure}

Since each marked edge passing thru $a_i$ has a $(+)$ on it or nothing, 
we have 
$$\measuredangle a_i\le\measuredangle a'_i$$
for each $i$.

}

\begin{thm}{Exercise}\label{ex:a<a}
Prove the last statement. 
\end{thm}

By the construction we have $|a_i-a_{i-1}|=|a'_i-a'_{i-1}|$ for all $i$.
By the arm lemma (\ref{lem:arm}), 
we get 
\[|a-b|\le |a'-b'|.
\eqlbl{clm:ab<ab}\]

Swap $K$ and $K'$ and repeat the same construction for a plane passing thru $a'$ and $b'$.
We get
\[|a-b|\ge |a'-b'|.
\eqlbl{clm:ab>ab}\]

The inequalities
\ref{clm:ab<ab} and \ref{clm:ab>ab} 
together imply $|a-b|=|a'-b'|$.
The equality case in the arm lemma implies that no edge at $v$ is marked;
that is, $v$ is not a vertex of $\Gamma$
--- a contradiction.
\qeds

From the proof, it follows that the local lemma is indeed local --- it works for two nonconguent convex polyhedral angles with equal corresponding faces.
Use this observation in the following exercise.

\begin{wrapfigure}{r}{20mm}
\vskip-0mm
\centering
\includegraphics{mppics/pic-10}
\bigskip
\includegraphics{mppics/pic-20}
\vskip-0mm
\end{wrapfigure}

\begin{thm}{Exercise}\label{ex:disc-bend}
Consider two polyhedral discs in $\EE^3$ glued from regular polygons by the rule on the diagrams.
Assume that each disc is part of a surface of a convex polyhedron.

\begin{subthm}{}
The first configuration is rigid; that is, one can not fix the position of the pentagon and continuously move the remaining 5 vertices in a new position so that each triangle moves by a one-parameter family of isometries of $\EE^3$.
\end{subthm}

\begin{subthm}{}
Show that the second configuration has a rotational symmetry with the axis passing thru the midpoint of the marked edge.
\end{subthm}

\end{thm}

\section{Proof of global lemma}

It is instructive to do the next exercise before diving into the proof.

\begin{thm}{Exercise}\label{ex:octahedron}
Try to mark the edges of an octahedron
by pluses and minuses
such that there would be 4 sign changes at each vertex.

Show that this is impossible.
\end{thm}

The proof of the global lemma is based on counting the sign changes
in two ways;
first while walking around each vertex of $\Gamma$
and second while moving around each of the regions separated by $\Gamma$
on the surface~$P$.
If two edges are adjacent at a vertex,
then they are also adjacent in a region.
The converse is true as well.
Therefore, both countings give the same number.

\parit{Proof of \ref{lem:global}.}
We can assume that $\Gamma$ is connected;
that is, one can get from any vertex to any other vertex by walking along edges.
(If not, pass to a connected component of $\Gamma$.)

We can assume that $\Gamma$ is embedded in the sphere.
Denote by $k$ and $l$ the number of vertices and edges in $\Gamma$.
Denote by $m$ the number of \textit{regions} that $\Gamma$ cuts from $P$.
Since $\Gamma$ is connected, each region is homeomorphic to an open disc.

\begin{thm}{Exercise}\label{ex:disc}
Prove the last statement.
\end{thm}

Now we can apply Euler's formula
$$k-l+m=2.
\eqlbl{eq:cauchy:euler}$$

Denote by $s$ the total number of sign changes in $\Gamma$ for all vertices. 
By the local lemma (\ref{lem:local}), we have 
$$ 4\cdot k\le s.\eqlbl{eq:S>=4k}$$

Let us get an upper bound on $s$ by counting the number of sign changes when one walks around
each region. 
Denote by $m_n$ the number of regions bounded by $n$ edges;
if an edge appears twice when it is counted twice.
Note that each region is bounded by at least $3$ edges;
therefore
$$m=m_3+m_4+m_5+\dots\eqlbl{eq:3-4-5}$$
Since edge belongs to exactly two regions, we get
$$2\cdot l=3\cdot m_3+ 4\cdot m_4+5\cdot m_5+\dots$$
Combining this with Euler's formula \ref{eq:cauchy:euler}, we get
$$4\cdot k=8+2\cdot m_3+4\cdot m_4+6\cdot m_5+8\cdot m_6+\dots
\eqlbl{eq:k=2+}$$
Observe that the number of sign changes in $n$-gon regions has to be even and $\le n$.
Therefore
$$s \le 2\cdot m_3 + 4\cdot m_4 + 4\cdot m_5 + 6\cdot m_6+\dots
\eqlbl{eq:23-44-45}$$
Clearly, \ref{eq:S>=4k} and \ref{eq:23-44-45} contradict \ref{eq:k=2+}.
\qeds


\section{Alexandrov uniqueness theorem}

Alexandrov's uniqueness theorem states that the conclusion of the Cauchy theorem (\ref{thm:cauchy}) still holds without the face-to-face assumption.

\begin{thm}{Theorem}\label{thm:alexandrov-uni'}
Any two convex polyhedrons in $\EE^3$ with isometric surfaces are congruent.

Moreover, any isometry between surfaces of convex polyhedrons can be extended to an isometry of the whole $\EE^3$.
\end{thm}

Instead of proof we list the modifications needed in the proof of Cauchy's theorem.

\parit{List of modifications in the proof of \ref{thm:cauchy}.}
Suppose $\iota\:P\z\to P'$  is an isometry between surfaces of $K$ and $K'$.
Mark in $P$ all the edges of $K$ and all the inverse images of edges in $K'$.
It might happen that an edge in $K'$ does not correspond to an edge in $K$;
it this case its inverse image in $K$ will be called a \index{fake edge}\emph{fake edge} of $K$.

The marked lines divide $P$ into convex polygons, and the restriction of $\iota$ to each polygon is a rigid motion.
These polygons should be used instead of faces in the proof of \ref{thm:cauchy}.

A vertex of the obtained graph can be a vertex of $K$, or it can be a fake vertex;
that is, it might be an intersection of an edge and a fake edge.

\begin{figure}[ht!]
\vskip-0mm
\centering
\includegraphics{mppics/pic-50}
\vskip-0mm
\end{figure}

For the first type of vertex, the local lemma can be proved the same way.
For a fake vertex $v$, it is easy to see that both parts of the edge coming thru $v$ are marked with minus
while both of the fake edges at $v$ are marked with plus.
Therefore, the local lemma holds for the fake vertices as well.

The remaining parts of the proof need no modifications.
\qeds

\begin{thm}{Exercise}\label{pr:K-P-simmetry}
Let $K$ be a convex polyhedron in $\EE^3$;
denote by $P$ its surface.
Show that each isometry $\iota\:P\z\to P$,
can be extended to an isometry of $\EE^3$.
\end{thm}


\section{Remarks}

This lecture contains selected material from Alexandrov's book~\cite{alexandrov}.

In Euclid's Elements, 
solids were claimed to be equal if the same holds for their faces, but no proof was given.
Adrien-Marie Legendre became interested in this problem towards the end of the 18th century.
He discussed it with his colleague Joseph-Louis Lagrange, who suggested this problem to Augustin-Louis Cauchy in 1813; soon he proved it \cite{cauchy}.
This theorem is included in many popular books \cite{aigner-zigler,dolbilin,tabacnikov-fuks}.
The key observation that the face-to-face condition can be removed was made by
Alexandr Alexandrov \cite{alexandrov-1941}.

\parit{Arm lemma.}
Cauchy's proof \cite{cauchy}
also used a version of the arm lemma, but its proof contained a small mistake that was corrected in a century \cite{sabitov}.

Several proofs of the arm lemma can be found in the letters between Isaac Schoenberg and Stanisław Zaremba \cite{schoenberg-zaremba}.

The following variation of the arm lemma makes sense for nonconvex spherical polygons.
It is due to Viktor Zalgaller \cite{zalgaller}.
It can be used instead of the standard arm lemma.

\begin{thm}{Another arm lemma}
Let $A=[a_1\dots a_n]$ and $A'\z=[a'_1\z\dots a'_n]$ be two spherical $n$-gons (not necessarily convex).
Assume that $A$ lies in a half-sphere,
the corresponding sides of $A$ and $A'$ are equal
and each angle of $A$ is at least the corresponding angle in $A'$.
Then $A$ is congruent to~$A'$. 
\end{thm}

Another close relative of the arm lemma is Reshetnyak's majorization theorem \cite{reshetnyak}.

\parit{Global lemma.}
A more visual proof of the global lemma is given in \cite[II \S 1.3]{alexandrov}.
This argument reused by Anton Klyachko \cite{klyachko} in his \index{car-crash lemma}\emph{car-crash lemma}.

\parit{Approximation.}
Proposition \ref{prop:conv-surf-CBB(0)} generalizes to boundaries of convex bodies  in $\EE^m$ for any $m\ge 2$.
This could be considered as a partial case of the conjecture about boundary of Alexandrov space; see \ref{conj:bry}.
Another partial case is proved by the authors and Stephanie Alexander \cite{alexander-kapovitch-petrunin-2008}.

\parit{Existance theorem.}
\ref{ex:surf-S2} and \ref{prop:poly-CBB} imply that the surface of a convex body is a sphere with nonnegative curvature in the sense of Alexandrov.
The celebrated theorem of Alexandrov states that the converse also holds if we allow degeneration of convex bodies to plane figures;
the surface of a plane figure is defined as its doubling across the boundary.
In other words, any $\Alex0$ metric on the two-sphere is isometric to a surface of a (possibly degenerate) convex body.
Moreover this convex body is unique up to congruence.
The last part is due to Alexei Pogorelov \cite{pogorelov}.

Originally, Alexandrov proved the statement for polyhedral metrics on the sphere; this proof is sketched in the appendix.
Then he used approximation to extend the result to  arbitrary $\Alex0$ metrics on the sphere.



%\chapter{Misc}

\section{Existence}\label{sec:Alexandrov-existence}

\begin{thm}{Theorem}\label{thm:exist}
A polyhedral metric on the sphere is isometric to the surface of a convex polyhedron (possibly degenerate to a flat polygon) if and only if it has nonnegative curvature at each point.
\end{thm}

\begin{wrapfigure}{r}{30mm}
\vskip-5mm
\centering
\includegraphics{mppics/pic-1010}
\vskip-0mm
\end{wrapfigure}

By \ref{thm:alexandrov-uni'}, a convex polyhedron is completely defined by the intrinsic metric of its surface.
By \ref{thm:exist}, it follows that knowing the metric we could find the position of the edges.
However, in practice, it is not easy to do.

For example, the surface glued from a rectangle as shown on the diagram defines a tetrahedron.
Some of the glued lines appear inside facets of the tetrahedron and some edges (dashed lines) do not follow the sides of the rectangle.

\paragraph{Space of polyhedrons.}
Let us denote by $\mathbf{K}$ the space of all convex polyhedrons in the Euclidean space,
including polyhedrons that degenerate to a plane polygon.
Polyhedra in $\mathbf{K}$ will be considered up to a motion of the space,
and the whole space $\mathbf{K}$ will be considered with Hausdorff distance up to a motion of the space;
that is, the distance between $K$ and $K'$ is the exact lower bound on Hausdorff distance from $\iota(K)$ to $K'$, where $\iota$ is arbitrary motion of $\EE^3$.

Further, denote by $\mathbf{K}_n$ the polyhedrons in $\mathbf{K}$ with exactly $n$ vertices.
Since any polyhedron has at least 3 vertices, the space $\mathbf{K}$ admits a subdivision into a countable number of subsets $\mathbf{K}_3,\mathbf{K}_4,\dots$

\paragraph{Space of polyhedral metrics.}
The space of polyhedral metrics on the sphere with nonnegative curvature will be denoted by $\mathbf{P}$.
The metrics in $\mathbf{P}$ will be considered up to an isometry, and the whole space $\mathbf{P}$ will be equipped with the topology induced by the Gromov--Hausdorff metric.

The subset of $\mathbf{P}$ of all metrics with exactly $n$ essential vertices will be denoted by $\mathbf{P}_n$.
It is easy to see that any metric in $\mathbf{P}$ has at least 3 essential vertices.
Therefore $\mathbf{P}$ is subdivided into countably many subsets
 $\mathbf{P}_3,\mathbf{P}_4,\dots$

\paragraph{From a polyhedron to its surface.}

By \ref{prop:poly-CBB}, passing from a polyhedron to its surface defines a map
\[\iota\:\mathbf{K}\to \mathbf{P}.\]

By \ref{ex:vertex-essential-vertex}, the number of vertices of a polyhedron is equal to the number of essential vertices on its surface.
In other words, $\iota(\mathbf{K}_n)\subset \mathbf{P}_n$ for any $n\ge 3$.

Using the introduced notation, we can unite \ref{thm:alexandrov-uni'} and \ref{thm:exist} in the following more exact statement.

\begin{thm}{Reformulation}
For any integer $n\ge 3$,
the map $\iota$ induces a bijection between $\mathbf{K}_n$ and~$\mathbf{P}_n$.
\end{thm}

The proof is based on a construction of a one-parameter family of polyhedrons that starts at an arbitrary polyhedron
and ends at a polyhedron with its surface isometric to the given one.
This type of argument is called the \textit{continuity method}; it is often used in the theory of differential equations.


\parit{Sketch.}
By \ref{thm:alexandrov-uni'}, the map $\iota\:\mathbf{K}_n\to\mathbf{P}_n$ is injective.
Let us prove that it is surjective.

\begin{thm}{Lemma}
For any integer $n\ge 3$, the space $\mathbf{P}_n$ is connected.
\end{thm}

The proof of this lemma is not complicated, but it requires ingenuity;
it can be done by the direct construction of a one-parameter family of metrics in $\mathbf{P}_n$ that connects two given metrics.
Such a family can be obtained by а sequential application of the following construction and its inverse.

Let $P\in\mathbf{P}_n$.
Suppose $v$ and $w$ are essential vertices in $P$.
Let us cut $P$ along a geodesic from $v$ to $w$.
Note that the geodesic cannot pass thru an essential vertex of $P$.
Further, note that there is a three-parameter family of patches that can be used to patch the cut so that the obtained metric remains in $\mathbf{P}_n$;
in particular, the obtained metric has exactly $n$ essential vertices (after the patching, the vertices $v$ and $w$ may become inessential).


\begin{thm}{Lemma}
The map $\iota\:\mathbf{K}_n\to\mathbf{P}_n$ is open,
that is, it maps any open set in $\mathbf{K}_n$ to an open set in $\mathbf{P}_n$.

In particular, for any $n\ge 3$, the image $\iota(\mathbf{K}_n)$ is open in~$\mathbf{P}_n$.
\end{thm}

This statement is very close to the so-called \textit{invariance of domain theorem};
the latter states that a continuous injective map between manifolds of the same dimension is open.

Recall that $\iota$ is injective.
The proof of the invariance of domain theorem can be adapted to our case since both spaces $\mathbf{K}_n$ and $\mathbf{P}_n$ are $(3\cdot n-6)$-dimensional and both look like manifolds, altho, formally speaking, they are \textit{not} manifolds.
In a more technical language, $\mathbf{K}_n$ and $\mathbf{P}_n$ have the natural structure of $(3\cdot n-6)$-dimensional \textit{orbifolds},
and the map $\iota$ respects the \textit{orbifold structure}.

We will only show that both spaces $\mathbf{K}_n$ and $\mathbf{P}_n$ are $(3\cdot n-6)$-dimensional.

Choose  $K\in\mathbf{K}_n$.
Note that $K$ is uniquely determined by the $3\cdot n$ coordinates of its $n$ vertices.
We can assume that the first vertex is the origin, the second has two vanishing coordinates and the third has one vanishing coordinate; therefore, all polyhedrons in $\mathbf{K}_n$ that lie sufficiently close to $K$ can be described by $3\cdot n-6$ parameters.
If $K$ has no symmetries, then this description can be made one-to-one;
in this case, a neighborhood of $K$ in $\mathbf{K}_n$ is a $(3\cdot n-6)$-dimensional manifold.
If $K$ has a nontrivial symmetry group, then this description is not one-to-one but it does not have an impact on the dimension of~$\mathbf{K}_n$.

The case of polyhedral metrics is analogous.
We need to construct a subdivision of the sphere into plane triangles using only essential vertices.
By Euler's formula, there are exactly $3\cdot n-6$ edges in this subdivision.
Note that the lengths of edges completely describe the metric, and slight changes in these lengths produce a metric with the same property.
Again, if $P$ has no symmetries, then this description is one-to-one.

\begin{thm}{Lemma}
The map $\iota\:\mathbf{K}_n\to\mathbf{P}_n$ is closed;
that is, the image of a closed set in $\mathbf{K}_n$ is closed in $\mathbf{P}_n$.

In particular, for any $n\ge 3$, the set $\iota(\mathbf{K}_n)$ is closed in~$\mathbf{P}_n$.
\end{thm}

Choose a closed set $Z$ in $\mathbf{K}_n$.
Denote by $\bar Z$ the closure of $Z$ in $\mathbf{K}$; note that $Z=\mathbf{K}_n\cap \bar Z$.
Assume $K_1,K_2,\dots\in Z$ is a sequence of polyhedrons that converges to a polyhedron $K_\infty\in\bar Z$.
By \ref{lem:H>GH}, $\iota(K_n)$ converges to $\iota(K_\infty)$ in $\mathbf{P}$.
In particular, $\iota(\bar Z)$ is closed in $\mathbf{P}$.

Since $\iota(\mathbf{K}_n)\subset \mathbf{P}_n$ for any $n\ge 3$, we have $\iota (Z)=\iota(\bar Z)\cap \mathbf{P}_n$;
that is, $\iota (Z)$ is closed in $\mathbf{P}_n$.

\medskip

Summarizing, $\iota(\mathbf{K}_n)$ is a nonempty closed and open set in $\mathbf{P}_n$, and $\mathbf{P}_n$ is connected for any $n\ge 3$.
Therefore, $\iota(\mathbf{K}_n)=\mathbf{P}_n$; that is, $\iota\:\mathbf{K}_n\z\to\mathbf{P}_n$ is surjective.
\qeds

\section{Approximation}

By now, the embedding theorem is proved for polyhedral metrics on the sphere.
The general case is done by approximation, using the following statement.

\begin{thm}{Proposition}\label{prop:H>GH}
Let $K_1,K_2,\dots$ be a sequence of convex bodies that converge to $K_\infty$ in the sense of Hausdorff.
Then the surface of $K_n$ converges to the surface of $K_\infty$ in the sense of Gromov--Hausdorff.
\end{thm}

If $K_\infty$ is nondegenerate, then the statement follows from \ref{lem:H>GH}.
The degenerate case is left as an exercise.

Let $\spc{X}_\infty$ be an $\Alex0$ space that is homeomorphic to $\SSS^2$.
Suppose that $\spc{X}_\infty$ is a Gromov--Hausdorff limit of a sequence of spheres with polyhedral metrics $\spc{X}_1,\spc{X}_2,\dots$
By \ref{thm:exist}, there is a sequence of convex polyhedra $K_1,K_2,\dots$ with surfaces isometric to $\spc{X}_1,\spc{X}_2,\dots$, respectively.
Note that  $\diam K_n\le \diam \spc{X}_n$ for any $n$.
Therefore we can assume that all polyhedra $K_1,K_2,\dots$ lie in a closed ball of sufficiently large radius.

Applying Blaschke selection theorem, we can pass to a subsequence of $K_1,K_2,\dots$ that converges in the sense of Hausdorff; denote its limit by $K_\infty$.
By \ref{prop:H>GH} the surface of $K_\infty$ is isometric to $\spc{X}_\infty$.

Therefore it remains to prove the following lemma.

\begin{thm}{Lemma}\label{lem:GH-approximation}
Let $\spc{X}$ be an $\Alex0$ space that is homeomorphic to $\SSS^2$.
Then there is a sphere with polyhedral metrics $\spc{X}'$
that is arbitrarily close to $\spc{X}$ in the sense of Gromov--Hausdorff.
\end{thm}

\parit{Proof with two cheatings.}
Suppose we can triangulate $\spc{X}_\infty$ by small geodesic triangles;
that is, we can choose a finite set of points $p_1,\dots,p_n\z\in \spc{X}_\infty$ and some geodesics $[p_ip_j]$ that cut $\spc{X}_\infty$ into regions of small diameter bounded by geodesic triangles $[p_ip_jp_k]$.
(This is the first chating, the actual proof uses a triangulation with a weaker property.)

Observe that total angle around each $p_i$ cannot exceed $2\cdot \pi$.
That is, suppose $p_{j_1},\dots,p_{j_k}$ are points connected to $p_i$ by geodesics.
Assume that they are ordered in the natural cyclic order.
Then
\[\mangle\hinge{p_i}{p_{j_1}}{p_{j_2}}+\dots+\mangle\hinge{p_i}{p_{j_{k-1}}}{p_{j_k}}+\mangle\hinge{p_i}{p_{j_{k}}}{p_{j_1}}\le 2\cdot\pi.\]
By comparison, we get
\[\angk{p_i}{p_{j_1}}{p_{j_2}}+\dots+
\angk{p_i}{p_{j_{k-1}}}{p_{j_k}}+\angk{p_i}{p_{j_{k}}}{p_{j_1}}\le 2\cdot\pi.\eqlbl{eq:sum<=<2pi}\]

Now let us exchange each triangle by its model triangle.
That is, consider a model triangle for each region in the subdivision of $\spc{X}$ and glue them together by the same rule.
By \ref{eq:sum<=<2pi}, the obtained polyhedral surface $\spc{X}'$ has nonnegative curvature.
It remains to show that this way we can produce $\spc{X}'$ arbitrarily close to $\spc{X}$.

Denote by $p_i\to p_i'$ the natural map; it takes $p_i$ in $\spc{X}$ and returns the corresponding point in $\spc{X}'$.
Observe that
\[\dist{p_i'}{p_j'}{\spc{X}'}
\le
\dist{p_i}{p_j}{\spc{X}}.\eqlbl{eq:|pp|}\]
Indeed, choose a geodesic $\gamma$ from $p_i$ to $p_j$.
Let $p_i=x_0,x_1,\dots,x_n=p_j$ be the points of intersections of $\gamma$ with the edges of the triangulation listed as they appear on $\gamma$.
For each $i$, denote by $x_i'$ the corresponding point in $\spc{X}'$.
By comparison, we get
\[\dist{x_k'}{x_{k-1}'}{\spc{X}'}
\le
\dist{x_k}{x_{k-1}}{\spc{X}}.\]
for each $k$.
Therefore, \ref{eq:|pp|} follows.

Suppose $\eps>0$ is small, the points $p_1,\dots,p_n$ form an $\eps$-net in $\spc{X}$, all edges of the triangulation are smaller than $\eps$ and
\[\dist{p_i'}{p_j'}{\spc{X}'}
\ge
\dist{p_i}{p_j}{\spc{X}} -100\cdot \eps.\eqlbl{eq:|pp|>=}\]
Then, together with the inequality above it proves the lemma.

Now let us assume that the sides of model triangles in $\spc{X}'$ are geodesics.
(This is the second cheating; the sides of the model triangles are local geodesics in $\spc{X}'$,
but not necessarily geodesic; that is, they do not have to be length-minimizing.
The actual proof does not use this assumption.)

Choose a geodesic $\gamma'$ from $p_i'$ to $p_j'$ in $\spc{X}'$.
Note that $\gamma'$ visits each triangle in the triangulation of $\spc{X}'$ at most once.

Let $p_i'=x_0',x_1',\dots,x_n'\z=p_j'$ be the points of intersections of $\gamma'$ with the edges of the triangulation listed from $p_i'$ to $p_j'$.
For each $i$, denote by $x_i$ the corresponding point in $\spc{X}$.
Let $\Delta_k'$ be the triangle that contains arc $[x'_{k-1}x'_k]$ of $\gamma'$ and $\Delta_k$ the corresponding triangle in~$\spc{X}$.
Note that
\[\dist{x_k'}{x_{k-1}'}{\spc{X}'}
\ge
\dist{x_k}{x_{k-1}}{\spc{X}} -\eps\cdot K(\Delta_k),
\eqlbl{eq:|xx|<}\]
where $K(\Delta_k)$ denotes the access of $\Delta_k$;
that is, the sum of its internal angles minus $\pi$.

Euler's formula and \ref{eq:sum<=<2pi} imply that the sum of all accesses is at most $4\cdot\pi$.
Therefore, summing up \ref{eq:|xx|<}, we get
\[\dist{p_i'}{p_j'}{\spc{X}'}
\ge
\dist{p_i}{p_j}{\spc{X}}-4\cdot \pi\cdot \eps.\]
Whence \ref{eq:|pp|>=} follows.
\qeds

\section{Comments}

\parit{Existence theorem.}
This theorem was proved by Alexandr Alexandrov~\cite{alexandrov-1941}.
Our sketch is taken from \cite{lebedeva-petrunin};
a complete proof is nicely written in~\cite{alexandrov}.
In the original proof, the spaces $\mathbf{K}_n$ and $\mathbf{P}_n$ were modified so the they become $(3\cdot n-6)$-dimensional manifolds.
It was done by introducing extra structure (for $\mathbf{K}_n$ it is orientation + a marked vertex and an edge) that \textit{brakes symmetries} of the spaces.
After that one could apply the domain invariance theorem directly.
Alternatively, one may first remove from $\mathbf{K}_n$ and $\mathbf{P}_n$ elements (polyhedron or surface)with nontrivial symmetries (after that the spaces become manifolds) and show that any symmetric polyhedron (or surface) can be approximated by a non-symmetric polyhedron (or surface).

A very different proof was found by Yuri Volkov in his thesis \cite{volkov};
it uses a deformation of three-dimensional polyhedral space.

%P and \Sigma???



\chapter{Surface theory}\label{chap:surfaces}

This lecture is less rigorous;
it aims to demonstrate beauty of geometry of convex surfaces, which is the precursor of modern Alexandrov geometry.
For a deeper dive into this theory, we recommend turning to the classic and brilliantly written books by Alexandr Alexandrov \cite{alexandrov,alexandrov-1948}.
Also, the book by Alexey Pogorelov \cite{pogorelov1969} is very recommended, despite being a challenge to read.


\section{Polyhedral surfaces}

A \index{polyhedral surface}\emph{polyhedral surface} is defined as a 2-dimensional manifold (possibly with boundary) with a length metric that admits a finite triangulation such that each triangle is isometric to a solid plane triangle.
A \index{triangulation}\emph{triangulation} of a polyhedral surface will always be assumed to satisfy this condition.

Note that according to our definition, any polyhedral surface is compact.

Choose a point $p$ on a polyhedral surface $\spc{P}$.
We can assume that $p$ is a vertex of a triangulation $\spc{P}$;
it can be achieved by subdividing the triangulation.
Denote by $\theta_p$ the \emph{total angle} around $p$;
that is, the sum of all angles at $p$ in all the triangles that have $p$ as a vertex.

Note that $\theta_p$ does not depend on the choice of triangulation.
If $p$ is an interior point, then the value $2\cdot\pi-\theta_p$ is called \emph{curvature} at $p$.
If $p$ lies on the boundary of $\spc{P}$, then the value $\pi-\theta_p$ is called \emph{inner turn} at $p$.

A point with nonvanishing curvature or inner turn will be called an \emph{essential vertex} of the surface.
Observe that an essential vertex is a vertex in any triangulation.

\begin{thm}{Exercise}\label{ex:geodesic-vertex}
Show that geodesics on a polyhedral surface with nonnegative curvatures and nonnegative inner turns may have essential vertices only at their endpoints.
\end{thm}

The following statement is an analog of the Gauss--Bonnet formula.

\begin{thm}{Exercise}\label{ex:gauss-bonnet}
Let $K(\spc{P})$ and $T(\spc{P})$ denote the sum of curvatures of all interior points
and the sum of all inner turns of the boundary points a polyhedral surface $\spc{P}$.
Show that
\[K(\spc{P})+T(\spc{P})=2\cdot\pi\cdot\chi(\spc{P}),\]
where $\chi(\spc{P})$ denotes the Euler characteristic of $\spc{P}$.
\end{thm}

The following proposition states that this new definition of curvature agrees with the $\Alex0$ comparison.

\begin{thm}{Proposition}\label{prop:poly-CBB}
A polyhedral surface is $\Alex0$ if and only if it has nonnegative curvature at every inner point and and nonnegative inner turn at each boundary point.
\end{thm}

\parit{Proof.}
By \ref{comp-kappa}, it is sufficient to check that
$f=\tfrac12\cdot\distfun_p^2\circ\gamma$ is 1-concave for any geodesic $\gamma$ and any point $p$.

We can assume that $p$ is not a vertex and the endpoints of $\gamma$ are not vertices;
the vertex case can be done by approximation.
By \ref{ex:geodesic-vertex}, $\gamma$ does not contain vertices.

Given a point $x=\gamma(t_0)$, choose a geodesic $[px]$.
Again, by \ref{ex:geodesic-vertex}, $[px]$ does not contain vertices.
Therefore, a neighborhood $U\supset [px]$ can be unfolded on a plane;
that is, there is an injective length-preserving map $z\mapsto \tilde z$
of $U$ into the Euclidean plane.
This way we map the part of $\gamma$ in $U$ to a line segment $\tilde\gamma$.
Let
\[\tilde f(t)\df\tfrac12\cdot\distfun_{\tilde p}^2\circ\tilde \gamma(t).\]
Since the geodesic $[px]$ maps to a line segment, we have $\tilde f(t_0)= f(t_0)$.
Furthermore, since the unfolding $z\mapsto \tilde z$ preserves lengths of curves, we get
$\tilde f(t)\ge f(t)$ if $t$ is close to $t_0$.
That is, $\tilde f$ is a local upper barrier of $f$ at $t_0$; see \ref{Function comparison}.
Evidently, $\tilde f''(t)\equiv 1$.
Therefore, $f$ is 1-concave.

\begin{thm}{Exercise}\label{ex:poly-CBB}
The converse is left to the reader.\qeds
\end{thm}

\section{Approximation}

The following theorem is the main extra tool available in Alexandrov geometry of surfaces.
We will use this statement in the proof of \ref{cor:Alex0-convex} to reduce questions about $\Alex0$ surfaces to polyhedral surfaces with nonnegative curvature.

\begin{thm}{Theorem}\label{thm:approximation}
Any closed $\Alex0$ surface is a Gromov--Hausdorff limit of homeomorphic polyhedral surfaces with nonnegative curvature.
\end{thm}

The construction of polyhedral approximations is based on the following exercise.

\begin{thm}{Exercise}\label{ex:construction}
Let $\spc{P}$ be a closed $\Alex0$ surface.

\begin{subthm}{ex:approximation:nbhd}
Show that any point $p$ admits an arbitrary small closed convex polygonal neighborhood $N$;
that is, $N$ is convex and bounded by a broken geodesic.

\end{subthm}

\begin{subthm}{ex:approximation:triangulation}
Given $\delta>0$, show that $\spc{P}$ admits a triangulation $\tau$ by convex triangles
with positive inner turn at each vertex and diameter smaller than $\delta$.
\end{subthm}

\begin{subthm}{ex:approximation:poly}
Suppose that $v$ is a vertex of a triangulation $\tau$ of $\spc{P}$ by convex triangles.
Let $\theta_v$ be the sum of angles at $v$ in all the triangles of~$\tau$.
Show that $\theta_v\le 2\cdot\pi$.
\end{subthm}

\end{thm}

\parit{Construction.}
Let $\spc{P}$ be a closed $\Alex0$ surface.
By part \ref{SHORT.ex:approximation:triangulation}, we can triangulate $\spc{P}$ by small convex triangles, say diameter of each triangle is less than given $\delta>0$.
Let us exchange each triangle of the triangulation by its model solid triangle; denote by $\tilde {\spc{P}}_\delta$ the obtained polyhedral surface.
Note that $\tilde {\spc{P}}_\delta$ is homeomorphic to $\spc{P}$;
moreover, there is a homeomorphism $\spc{P}\to \tilde {\spc{P}}_\delta$ that sends a point $x\in \spc{P}$ to a point $\tilde x\in \tilde {\spc{P}}_\delta$ in the corresponding model triangle.

By the angle comparison (\ref{angle-a}) and part \ref{SHORT.ex:approximation:poly} of the exercise, the total angle around each vertex in $\tilde {\spc{P}}_\delta$ cannot exceed $2\cdot\pi$.
That is, the obtained polyhedral space $\tilde {\spc{P}}_\delta$ has nonnegative curvature.
\qeds

Observe that Theorem \ref{thm:approximation} follows from the following.

\begin{thm}{Claim}\label{clm:approximation}
If $\tilde {\spc{P}}_\delta$ is provides by the construction, then $\tilde {\spc{P}}_\delta\to\spc{P}$ as $\delta\to 0$ in the sense of Gromov--Hausdorff.
\end{thm}

This claim seems to be self-evident, but it is not;
a very smart proof was given by Alexandrov \cite[VII §~6]{alexandrov-1948}.
We will indicate an alternative proof based on the following exercise and two theorems which will be stated without a proof.
The first theorem is due to Yuri Burago, Mikhael Gromov, and Grigori Perelman \cite[10.8]{burago-gromov-perelman};
it is a generalization of Alexandrov's theorem for surfaces \cite[X §~2]{alexandrov-1948}.
The second theorem is due to Nan Li \cite[Corollary 0.1]{li}.

\begin{thm}{Theorem}\label{thm:cont-vol}
Let $\spc{X}_1, \spc{X}_2$ be a sequence of $n$-dimensional $\Alex\kappa$ spaces that converges to $\spc{X}_\infty$ in the sense of Gromov--Hausdorff.
Then the $n$-volume on $\spc{X}_i$ weakly converges to the $n$-volume on $\spc{X}_\infty$.
\end{thm}

\begin{thm}{Theorem}\label{thm:vol-short}
Let $\spc{X}$ be an Alexandrov space without boundary, and let $\spc{Y}$ be an arbitrary Alexandrov space.
Then any short volume-preserving map $\spc{X}\to\spc{Y}$ is an isometry.
\end{thm}

Suppose a convex solid triangle $\Delta$ in an $\Alex0$ surface has angles $\alpha$, $\beta$ and $\gamma$.
Let us define its excess by
\[\excess\Delta=\alpha+\beta+\gamma-\pi.\]
Since the angles of a model triangle sum up to $\pi$, by the angle comparison (\ref{angle-a}),
the excess is nonnegative.

\begin{thm}{Exercise}\label{ex:approximation}
Let $\tau$ be a triangulation of a closed $\Alex0$ surface $\spc{P}$ by convex triangles $\Delta_1,\dots,\Delta_n$.


%\begin{subthm}{ex:approximation:diangle}
%Let $\Upsilon$ be a topological disc in $\tilde{\spc{P}}$ bounded by a geodesic $\gamma_0$ and local geodesic $\gamma_1$ with common endpoints.
%Let $\omega$ be the sum of the curvatures of points in $\Upsilon$.
%Show that
%\[\cos \omega\cdot \length \gamma_1\le \length \gamma_0\]
%if $\omega<\pi$.
%\end{subthm}


\begin{subthm}{ex:approximation:excess}
Show that
\[\excess\Delta_1+\dots+\excess\Delta_n \le 2\cdot\pi\cdot \chi(\spc{P}),\]
where $\chi(\spc{P})$ denotes the Euler characteristic of $\spc{P}$.
\end{subthm}


\begin{subthm}{ex:approximation:length}
Let $x$ and $y$ be points on the sides of a triangle $\Delta_i$, and let $\tilde x$ and $\tilde y$ be the corresponding points in the corresponding triangle $\tilde \Delta$ in $\tilde{\spc{P}}$.
Show that
\[\dist{\tilde x}{\tilde y}{\tilde \Delta}\le\dist{x}{y}{\Delta}\le \dist{\tilde x}{\tilde y}{\tilde \Delta}+\excess\Delta\cdot \diam \Delta.\]

\end{subthm}

\begin{subthm}{ex:approximation:area}
Let $\Delta$ be a solid triangle in the triangulation $\tau$ of $\spc{P}$, and $\tilde \Delta$ --- the corresponding triangle in $\tilde{\spc{P}}$.
Show that
\[\area \tilde \Delta\le \area \Delta\le \area \tilde \Delta+\tfrac12\cdot\excess\Delta\cdot (\diam \Delta)^2.\]

\end{subthm}

\end{thm}

Note that part \ref{SHORT.ex:approximation:excess} implies that $\chi(\spc{P})\ge 0$.
Therefore, $\spc{P}$ is homeomorphic to a sphere, projctive plane, torus, or Klein bottle.
In the latter two cases, the construction produces a flat surface $\tilde{\spc{P}}_\delta$, which has to be isometric to $\spc{P}$.
Therefore the cases of sphere, projctive plane are more interesting.

\parit{Proof of \ref{clm:approximation}.}
Choose a sequence of positive numbers $\delta_n\to0$;
let $\tilde{\spc{P}}_{\delta_n}$ be polyhedral spaces provided by the construction and let $\tau_n$ be the corresponding triangulation.

According to part \ref{SHORT.ex:approximation:length} of the exercise, the spaces $\tilde{\spc{P}}_{\delta_n}$ have bounded diameter.
Therefore by Gromov's selection theorem, we can pass to a converging sequence of $\tilde{\spc{P}}_{\delta_n}$;
denote its Gromov--Hausdorff limit by $\tilde{\spc{P}}$.
Note that if $\tilde{\spc{P}}_{\delta}$ does not converge to $\spc{P}$, then we can assume that $\tilde{\spc{P}}$ is not isometric to $\spc{P}$.

Choose two points $x,y\in \spc{P}$, and connect them by a geodesic.
Denote by $s_1,\dots,s_m$ the points of the geodesic on the sides of the triangulation $\tau_n$;
we assume that these points appear in the same order on the geodesic.
Denote by $\tilde x$, $\tilde y$, and $\tilde s_1,\dots,\tilde s_m$ the corresponding points in $\tilde{\spc{P}}_{\delta_n}$.
By part \ref{SHORT.ex:approximation:length} of the exercise,
\[\dist{\tilde s_{i-1}}{\tilde s_i}{\tilde{\spc{P}}_{\delta_n}}\le\dist{s_{i-1}}{s_i}{\spc{P}}.\]
Note also that
\[\dist{\tilde x}{\tilde s_1}{\tilde{\spc{P}}_{\delta_n}}\le \delta_n
\quad\text{and}\quad
\dist{\tilde s_m}{\tilde y}{\tilde{\spc{P}}_{\delta_n}}\le \delta_n
\]
Therefore
\[\dist{\tilde x}{\tilde y}{\tilde{\spc{P}}_{\delta_n}}\le\dist{x}{y}{\spc{P}}+2\cdot\delta_n.\]

Passing to the limit, we get a short onto map $\spc{P}\to \tilde{\spc{P}}$.
On the other hand, applying parts \ref{SHORT.ex:approximation:excess} and \ref{SHORT.ex:approximation:area}, we get that
\[
\area \spc{P} -\pi\cdot\chi(\spc{P})\cdot \delta_n^2
\le
\area \tilde{\spc{P}}_{\delta_n}
\le
\area \spc{P}
\]
By Theorem \ref{thm:cont-vol}, $\area \spc{P}= \area \tilde{\spc{P}}$.
Applying Theorem \ref{thm:vol-short}, we get that the short map $\spc{P}\to \tilde{\spc{P}}$ is an isometry --- a contradiction.
\qeds

\parit{Remark.}
The main difficulty in the proof comes from nonconvexity of triangles in the triangulation of $\tilde{\spc{P}}_\delta$.
If these triangles would be convex, then the first estimate in parts \ref{SHORT.ex:approximation:length} and \ref{SHORT.ex:approximation:excess} would imply that $\tilde{\spc{P}}_\delta$ is close to $\spc{P}$ in the sense of Gromov--Hausdorff.













\section{Surface of polyhedrons and bodies}

Let us define a \index{convex body}\emph{convex body} as a compact convex subset in $\EE^3$ with a non-empty interior.
The \index{surface}\emph{surface} of a convex body is defined as its boundary equipped with the induced length metric.

\begin{thm}{Exercise}\label{ex:surf-S2}
Show that the surface of a convex body is homeomorphic to the 2-dimensional sphere.
\end{thm}

A \index{convex polyhedron}\emph{convex polyhedron} is a convex body with a finite number of extremal points, called its \index{vertex}\emph{vertices}.

Note that the surface of a convex polyhedron $K$ is a closed polyhedral surface.

\begin{thm}{Exercise}\label{pr:tetrahedron}
Assume that the surface of a nondegenerate tetrahedron $T$ has curvature $\pi$ at each of its vertices.
Show that

\begin{subthm}{pr:tetrahedron:=}
all faces of $T$ are congruent;
\end{subthm}

\begin{subthm}{pr:tetrahedron:perp} the line containing the midpoints of opposite edges of $T$ intersects these edges at right angles.
\end{subthm}

\end{thm}

\begin{thm}{Claim}\label{clm:total-angle}
The surface $\spc{P}$ of any convex polyhedron $K$ has nonnegative curvature.
Moreover, a point $v$ is a vertex of $K$ if and only if
$v$ is an essential vertex of $\spc{P}$.
\end{thm}

A proof is given in Kiselyov's school textbook \cite[§ 48]{kiselev-stereo-en};
one can also deduce it from \ref{claim:angle-3angle-inq}.

\begin{thm}{Exercise}\label{ex:surface-covergence}
Let $K_1,K_2,\dots,$ and $K_\infty$ be convex bodies in $\EE^m$.
Denote by $\spc{P}_n$ the surface of $K_n$.
Suppose $K_n\z\to K_\infty$ in the sense of Hausdorff.
Show that $\spc{P}_n\to \spc{P}_\infty$ in the sense of Gromov--Hausdorff.
\end{thm}

Since any convex body is a Hausdorff limit of a sequence of convex polyhedrons, the next proposition follows from \ref{prop:poly-CBB}, \ref{ex:surface-covergence}, and \ref{thm:CBB-closed}.

\begin{thm}{Proposition}\label{prop:conv-surf-CBB(0)}
The surface of a convex body in $\EE^3$ is $\Alex0$.
\end{thm}

\section{Uniqueness theorem}

\begin{thm}{Theorem}\label{thm:alexandrov-uni'}
Any two convex polyhedrons in $\EE^3$ with isometric surfaces are congruent.

Moreover, any isometry between the surfaces of convex polyhedrons can be extended to an isometry of the whole $\EE^3$.
\end{thm}

If one assumes that the isometry between the surfaces is face-to-face,
then we get an equivalent reformulation of Cauchy's theorem.
Cauchy's argument, with a small addition, proves \ref{thm:alexandrov-uni'}.

First, let us remind Cauchy's proof, assuming the reader knows it.
If not, then read it in one of the classical texts \cite{aigner-zigler,dolbilin,tabacnikov-fuks}.

\parit{Sketch of Cauchy's proof.}
Suppose $K$ and $K'$ are convex polyhedrons in $\EE^3$;
denote their surfaces
by $\spc{P}$ and $\spc{P}'$.
Suppose there is an isometry $\iota\:\spc{P}\to \spc{P}'$ that sends each face of $K$ to a face of $K'$.

Let us mark an edge of $K$ with ``$+$'' (or ``$-$'') if the dihedral angle at this edge in $K$ is smaller (respectively, bigger) than the corresponding angle in $K'$.
Further, we consider the  graph $\Gamma$ that is formed by all marked edges.
If $\Gamma$ is empty, then Cauchy's theorem follows; assume the contrary.

The graph $\Gamma$ is embedded into $\spc{P}$, which is homeomorphic to the sphere.
In particular, the edges coming from one vertex have a natural cyclic order.
Given a vertex $v$ of $\Gamma$, we can count the \textit{number of sign changes} around $v$;
that is, the number of consequent pairs of edges with different signs.

We need to show two statements:

\begin{thm}{Local lemma}
At any vertex of $\Gamma$, the number of sign changes is at least $4$.
\end{thm}

\begin{thm}{Global lemma}
No (nonempty) planar graph meets the condition of the local lemma.
\end{thm}

Once the lemmas are proved, Cauchy's theorem follows.
\qeds

Once more, the argument above is  written only to make sure we are on the same page;
it will not work without reading the actual proof.

\parit{Alexandrov's addition.}
We need to remove the assumption that the isometry $\iota\:\spc{P}\z\to \spc{P}'$ is face-to-face.
Mark in $\spc{P}$ all the edges of $K$ as we did above.
In addition, if an edge in $K'$ does not correspond to an edge of $K$, then mark its inverse image in $K$   with ``$-$''; these lines on $K$ will be referred to as \index{fake edges and vertices}\emph{fake edges}.

The marked lines divide $\spc{P}$ into convex polygons, and the restriction of $\iota$ to each polygon is a rigid motion.
These polygons should be used instead of faces in the Cauchy's argument.

A vertex of the obtained graph can be a vertex of $K$, or it can be a {}\emph{fake vertex};
that is, it might be an intersection of an edge and a fake edge.

\begin{figure}[ht!]
\vskip-0mm
\centering
\includegraphics{mppics/pic-50}
\vskip-0mm
\end{figure}

For a usual vertex, the local lemma can be proved the same way.
For a fake vertex $v$, it is easy to see that both parts of the edge coming thru $v$ are marked with minus
while both of the fake edges at $v$ are marked with plus.
Therefore, we still have at least four sign changes at $v$.
The remaining argument works as before.
\qeds

Let us also state the following result of Alexey Pogorelov \cite[chapter III]{pogorelov};
an alternative proof was found by Yurii Volkov \cite{volkov1968}.

\begin{thm}{Theorem}
Any two convex bodies in $\EE^3$ with isometric surfaces are congruent.

Moreover, any isometry between surfaces of convex bodies can be extended to an isometry of the whole $\EE^3$.
\end{thm}

At first glance, this theorem might look like a small improvement of Alexandrov's uniqueness,
but this improvement is huge.
The proof is quite hard.
Let us just mention that it would follow if any two polyhedra $K$ and $K'$  with close surfaces in the sense of Gromov--Hausdorff would be almost congruent;
that is, there is a motion $\mu$ of $\EE^3$ such that the Hausdorff distance from $K$ to $\mu(K')$ is small.


\section{Existence theorem}

By \ref{prop:poly-CBB}, \ref{clm:total-angle}, and \ref{ex:surf-S2}, the surface of a convex polyhedron is an $\Alex0$ and homeomorphic to the sphere.
Alexandrov's theorem states that the converse holds if one includes in the consideration \textit{twice covered polygons}.
In other words, we have to consider a plane polygon as a degenerate polyhedron;
in this case, its surface is defined as the doubling of the polygon across its boundary.

From now on, we assume that a polyhedron can degenerate to a plane polygon.

\begin{thm}{Theorem}\label{thm:alexandrov-first}
A polyhedral metric on the two-sphere is isometric to the surface of a convex polyhedron (possibly degenerate) if and only if it has nonnegative curvature.

\end{thm}

Applying the approximation theorem (\ref{thm:approximation}) and \ref{ex:surface-covergence}, we get the following statement.
Here we again assume that a convex body can degenerate to a convex plane figure,
and, in this case, its surface is defined as the doubling of the figure across its boundary.

\begin{thm}{Corollary}\label{cor:Alex0-convex}
A metric on the two-sphere is $\Alex0$ if and only if it is isometric to the surface of a convex body (possibly degenerate).

\end{thm}

The proof of the existence theorem will be discussed in the following two sections.
It is instructive to solve the following exercise before going further.

\begin{thm}{Exercise}\label{ex:alexandrov=<4}
Let $\spc{P}$ be the 2-sphere equipped with a polyhedral metric with nonnegative curvature.

\begin{subthm}{ex:alexandrov=<4:>=3}
Prove that $\spc{P}$ has at least 3 essential vertices.
\end{subthm}

\begin{subthm}{ex:alexandrov=<4:=3}
If $\spc{P}$ has exactly 3 essential vertices $u$, $v$, and $w$, then it is isometric to the doubling of the solid model triangle $\modtrig(uvw)$.
\end{subthm}

\begin{subthm}{ex:alexandrov=<4:4}
If $\spc{P}$ has exactly 4 essential vertices, then it is isometric to the surface of a tetrahedron (possibly degenerate to a quadrangle).
\end{subthm}

\end{thm}

\section{Reformulation}

In this section, we introduce several notions and use them to reformulate the existence theorem (\ref{thm:reformulation}).

\paragraph{Space of polyhedrons.}
Let us denote by $\bm{K}$ the space of all convex polyhedrons in the Euclidean space,
including polyhedrons that degenerate to a plane polygon.
Polyhedrons in $\bm{K}$ will be considered up to a motion of the space; we will not distinguish between a convex polyhedron and its congruence class.

The space $\bm{K}$ will be considered with the topology induced by the {}\emph{Hausdorff metric up to a motion};
that is, the distance between (equivalence classes of) polyhedrons $K$ and $L$ is defined by
\[\dist{K}{L}{}\df \inf_\mu \{\dist{K}{\mu(L)}{\Haus}\},\]
where $\mu$ runs among all motions of $\EE^3$.

We say that a polyhedron $K$ in $\bm{K}$ has \emph{no symmetries} if  $K\z\ne \mu(K)$ for any nontrivial motion $\mu$ of $\EE^3$.
The set of all polyhedrons without symmetry in $\bm{K}$ will be denoted by $\bm{K}^\circ$.
Observe that $\bm{K}^\circ$ is an open set in $\bm{K}$.

Further, denote by $\bm{K}_n$ the polyhedrons in $\bm{K}$ with exactly $n$ vertices, and let $\bm{K}_n^\circ=\bm{K}_n\cap \bm{K}^\circ$.
Since any polyhedron has at least 3 vertices, the space $\bm{K}$ admits a subdivision into a countable number of subsets $\bm{K}_3,\bm{K}_4,\dots$

\paragraph{Space of surfaces.}
The space of polyhedral surfaces with nonnegative curvature that are homeomorphic to the 2-sphere will be denoted by $\bm{P}$.
The surfaces in $\bm{P}$ will be considered up to an isometry, and the whole space $\bm{P}$ will be equipped with the natural topology induced by the Gromov--Hausdorff metric.

We say that a surface $\spc{P}$ in $\bm{P}$ has \emph{no symmetries} if there is no nontrivial isometry
$\mu\:\spc{P}\to \spc{P}$.
The set of all surfaces without symmetry in $\bm{P}$ will be denoted by $\bm{P}^\circ$.
Observe that $\bm{P}^\circ$ is an open set in $\bm{P}$.

The subset of $\bm{P}$ of all surfaces with exactly $n$ essential vertices will be denoted by $\bm{P}_n$; let $\bm{P}_n^\circ=\bm{P}_n\cap \bm{P}^\circ$.
By \ref{ex:alexandrov=<4:>=3}, any surface in $\bm{P}$ has at least 3 essential vertices.
Therefore $\bm{P}$ is subdivided into countably many subsets
 $\bm{P}_3,\bm{P}_4,\dots$

\paragraph{From a polyhedron to its surface.}
Recall that the surface of a convex polyhedron is a sphere with nonnegative curvature.
Therefore, passing from a polyhedron to its surface defines a map
\[\iota\:\bm{K}\to \bm{P}.\]

Note that the existence theorem (\ref{thm:alexandrov-first}) follows from the next statement.

\begin{thm}{Theorem}\label{thm:reformulation}
For any integer $n\ge 3$,
the map $\iota$ is a bijection from $\bm{K}_n$ to~$\bm{P}_n$.
\end{thm}

\section{About the proof of existence}

By \ref{ex:surface-covergence}, the map $\iota\:\bm{K}\to\bm{P}$ is continuous.
Combining \ref{clm:total-angle} with the uniqueness theorem (\ref{thm:alexandrov-uni'}), we get that $\iota(\bm{K}_n)\subset \bm{P}_n$ and the map $\iota\:\bm{K}_n\to\bm{P}_n$ is injective.
It remains to prove the following.

\begin{thm}{Claim}\label{clm:surjective}
For any $n\ge 3$, the map $\iota\:\bm{K}_n\to\bm{P}_n$ is surjective.
\end{thm}

The proof is based on the construction of a one-parameter family of polyhedrons that starts at an arbitrary polyhedron
and ends at a polyhedron with its surface isometric to the given surface $\spc{P}$.
This type of argument is called the \index{continuity method}\emph{continuity method}; it is often used in the theory of differential equations.

\medskip

Now let us get into details.
First, observe that the second part of the uniqueness theorem (\ref{thm:alexandrov-uni'}) implies that $\iota(\bm{K}_n^\circ)\subset \bm{P}_n^\circ$.

\begin{thm}{Lemma}\label{lem:connected}
For any integer $n\ge 4$, the space $\bm{P}_n^\circ$ is connected and dense in $\bm{P}_n$.
\end{thm}

Note that $\bm{P}_3^\circ=\emptyset$;
indeed the surface of a triangle admits a reflection symmetry.
The case $n=4$ can be deduced from \ref{ex:alexandrov=<4:4}; thus, we can assume that $n\ge 5$.

The second statement is proved by a general-position-type argument.

The proof of the first statement is not complicated, but it requires ingenuity;
it can be done by the direct construction of a one-parameter family of surfaces in $\bm{P}_n^\circ$ that connects two given surfaces.
Such a family can be obtained as a sequence of the following deformations (direct or reversed).

Start with a surface $\spc{P}$ from $\bm{P}_n^\circ$.
Suppose $v$ and $w$ are essential vertices in $\spc{P}$.
Let us cut $\spc{P}$ along a shortest path from $v$ to~$w$.
This way we obtain a sphere with a hole.
The hole can be patched by a disc so that the obtained surface remains in $\bm{P}_n$.
In particular, the obtained surface has exactly $n$ essential vertices;
note that after the patching, the vertices $v$ and $w$ may become inessential.
(There is a three-parameter family of such patches, so we have something to choose from.)
Choosing a one-parameter family of such patches, we can get a deformation of~$\spc{P}$.

Again, applying a general-position-type argument to the above construction, we get a path in $\bm{P}_n^\circ$, assuming that the starting and ending surfaces are in $\bm{P}_n^\circ$.

\begin{thm}{Lemma}\label{lem:open}
The map $\iota\:\bm{K}_n^\circ\to\bm{P}_n^\circ$ is open,
that is, it maps any open set in $\bm{K}_n^\circ$ to an open set in $\bm{P}_n^\circ$.

In particular, for any $n\ge 3$, the image $\iota(\bm{K}_n^\circ)$ is open in~$\bm{P}_n^\circ$.
\end{thm}

This statement follows from the so-called \index{invariance of domain}\emph{invariance of domain theorem},
which states that a \textit{continuous injective map between manifolds of the same dimension is open}.

Recall that $\iota$ defines a continuous and injective $\bm{K}_n^\circ\to\bm{P}_n^\circ$.
It remains to check that both spaces $\bm{K}_n^\circ$ and $\bm{P}_n^\circ$ are $(3\cdot n-6)$-dimensional manifolds.

Choose a polyhedron $K$ in $\bm{K}_n$.
It is uniquely determined by the $3\cdot n$ coordinates of its $n$ vertices.
We can assume that the first vertex is at the origin,
the second has a positive $x$-coordinate
and the remaining two coordinates vanish,
and the third has a vanishing $z$-coordinate and a positive $y$-coordinate.
Therefore, all polyhedrons in $\bm{K}_n$ that lie sufficiently close to $K$ can be described by $3\cdot n-6$ parameters.
If $K$ has no symmetries, then this description is one-to-one;
in this case, a neighborhood of $K$ in $\bm{K}_n$ admits a parametrization by an open set in $\RR^{3\cdot n-6}$.

The case of surfaces is analogous.
We need to construct a subdivision of the sphere into plane triangles using only essential vertices.
By Euler's formula, there are exactly $3\cdot n-6$ edges in this subdivision.
The lengths of the edges completely describe the surface $\spc{P}$ and any surface near by.
If the surface has no symmetries, then this description is one-to-one, and a neighborhood of $\spc{P}$ in $\bm{P}_n$ admits a parametrization by an open set in  $\RR^{3\cdot n-6}$.

\begin{thm}{Lemma}\label{lem:closed}
The map $\iota\:\bm{K}_n\to\bm{P}_n$ is closed;
that is, the image of a closed set in $\bm{K}_n$ is closed in $\bm{P}_n$.

In particular, for any $n\ge 3$, the set $\iota(\bm{K}_n)$ is closed in~$\bm{P}_n$.
\end{thm}

Choose a sequence of polyhedrons $K_1,K_2,\ldots$ in $\bm{K}_n$.
Assume that the sequence $\spc{P}_i=\iota(K_n)$ converges in $\bm{P}_n$ as $i\to \infty$;
denote its limit by $\spc{P}_\infty$.
We need to construct a polyhedron $K_\infty\in \bm{K}_n$ such that $\iota(K_\infty)=\spc{P}_\infty$;
let us do it.

Passing to a subsequence, we can assume that $K_i$ converges in $\bm{K}$;
denote the limit polyhedron by $K_\infty$.
Since $\iota$ is continuous, $\iota(K_i)$ converges to $\iota(K_\infty)$ in~$\bm{P}$; so, $\iota(K_\infty)=\spc{P}_\infty$.
Recall that $\iota(\bm{K}_m)\subset\bm{P}_m$ for each $m$; therefore, $K_\infty\in \bm{K}_n$.


\parit{Proof of \ref{clm:surjective}.}
The case $n\le 4$ is already solved in \ref{ex:alexandrov=<4}; so we assume that $n\ge 5$.
By \ref{lem:closed} and \ref{lem:open},
$\iota(\bm{K}_n^\circ)$ is a non-empty closed and open set in $\bm{P}_n^\circ$, and $\bm{P}_n^\circ$ is connected.
Therefore, $\iota(\bm{K}_n^\circ)=\bm{P}_n^\circ$.

By \ref{lem:closed}, $\iota(\bm{K}_n)$ is closed in $\bm{P}_n$.
By \ref{lem:connected}, $\bm{P}_n^\circ$ is dense in $\bm{P}_n$.
Since $\iota(\bm{K}_n^\circ)=\bm{P}_n^\circ$, we have $\bm{P}_n^\circ\subset \iota(\bm{K}_n)$;
therefore, $\iota(\bm{K}_n)=\bm{P}_n$;
that is, $\iota\:\bm{K}_n\z\to\bm{P}_n$ is surjective.
\qeds

\section{Ambient space}

On one hand the Alexandrov surface theory is simpler since it has extra tools,
On the other hand, this tool comes with extra structure, which makes the theory more complicated.
The following result of Joseph Liberman \cite{liberman} gives an example.

\begin{thm}{Theorem}
Any geodesic in the surface of a convex body is one-sided differentiable as a curve in $\EE^3$.
\end{thm}

\parit{Proof.}
Let $\gamma$ be a geodesic on the surface of a convex body $K$.
Choose $p\in K$.
By \ref{ex:liberman}, the function $f_p\:t\mapsto \distfun_p\circ\gamma(t)$ is semiconcave for any $p\in K$.
In particular, one-sided derivatives $f_p^+(t)$ are defined for every $t$.

Given $x=\gamma(t)$, choose three points $p_1,p_2,p_3\in K$ in general position;
that is, the four points $x,p_1,p_2,p_3$ do not lie in one plane.
Observe that the distance functions $\distfun_{p_i}$ give smooth coordinates in a neighborhood of $x$.
From above the functions $f_{p_i}$ have one-sided derivatives at $t$.
Since the coordinates are smooth, we get that $\gamma^+(t)$ is defined as well.
\qeds




\begin{thm}{Exercise}\label{ex:convex}
Suppose a plane $\Pi$ cuts from the surface of a convex body $K$ a disc $\Delta$, and the reflection of $\Delta$ across $\Pi$ lies in $K$.
Show that $\Delta$ is a convex subset of the surface;
that is, if a geodesic has endpoints in $\Delta$, then it completely lies in $\Delta$.
\end{thm}

The following exercise gives a more exact version of comparison for convex surfaces;
it is due to Anatolii Milka \cite[Theorem 2]{milka1982}.

\begin{thm}{Very advanced exercise}\label{ex:milka}
Let $\spc{P}$ be the surface of a nondegenerate convex body $K\subset\EE^3$,
and let $\gamma_1$ and $\gamma_2$ be geodesic paths in $\spc{P}$ that start at one point $p\z=\gamma_1(0)\z=\gamma_2(0)$.
Suppose $x_i=\gamma_i(1)$, and $y_i\z=p+\gamma_i^+(0)$.
Show that
\[\dist{x_1}{x_2}{\spc{P}}\le \dist{y_1}{y_2}{W},\]
where $W$ is the complement to the interior of $K$.

\end{thm}



\section{Remarks}


The statement of Cauchy's theorem was conjectured by Adrien-Marie Legendre at the end of the 18$^\text{th}$ century;
a formulation was given in the first edition of his geometry textbook \cite{legendre}.
It was motivated by a vague definition in Euclid's Elements, which could be interpreted as
\textit{polyhedrons are equal if the same holds for their faces}.

The local lemma was already known to Legendre.
Legendre discussed this question with his colleague Joseph-Louis Lagrange, who suggested this problem to Augustin-Louis Cauchy in 1813; soon he solved it \cite{cauchy}.

The key observation that the face-to-face condition can be removed was made by
Alexandr Alexandrov in 1941; in the same paper he proved the uniqueness theorem \cite{alexandrov-1941}.
A quite different proof was found by Yurii Volkov in his thesis \cite{volkov}; it uses a deformation of three-dimensional polyhedral space.
(Be aware that the proof of this theorem given in the book by Igor Pak contains an essential mistake \cite{petrunin-2023}.)

In Cauchy's proof \cite{cauchy}, it was deducted from an analog of the following lemma.
Cauchy made a small mistake in its proof that was fixed in a century \cite{sabitov}.
Several proofs of the arm lemma can be found in the letters between Isaac Schoenberg and Stanisław Zaremba \cite{schoenberg-zaremba}.

\begin{thm}{Arm lemma}\label{lem:arm}
Assume that $A=[a_0 a_1\dots a_n]$ is a convex polygon in $\EE^2$
and $A'=[a'_0 a'_1\dots a'_n]$ is a polygonal line in $\EE^3$
such that
$|a_i-a_{i+1}|=|a'_i-a'_{i+1}|$ for any $i\in\{0,\dots,n-1\}$
and
$\measuredangle a_i\le \measuredangle a'_i$
for each $i\in\{1,\dots,n-1\}$.
Then
$$|a_0-a_n|\le |a'_0-a'_n|$$
and equality holds if and only if $A$ is congruent to $A'$.
\end{thm}

The following variation of the arm lemma makes sense for nonconvex spherical polygons.
It is due to Viktor Zalgaller \cite{zalgaller}.
It can be used instead of the standard arm lemma.

\begin{thm}{Another arm lemma}
Let $A=[a_1\dots a_n]$ and $A'\z=[a'_1\z\dots a'_n]$ be two spherical $n$-gons (not necessarily convex).
Assume that $A$ lies in a half-sphere,
the corresponding sides of $A$ and $A'$ are equal,
and each angle of $A$ is at least the corresponding angle in $A'$.
Then $A$ is congruent to~$A'$.
\end{thm}

Another close relative of the arm lemma is Reshetnyak's majorization theorem \cite{reshetnyak}.

Alexandrov gave two proofs of the global lemma \cite[2.1.2 and 2.1.3]{alexandrov}.
The first is combinatorial, and the second is more visual.
The argument in the second proof was reused by Anton Klyachko \cite{klyachko} in his \index{car-crash lemma}\emph{car-crash lemma}.

Proposition \ref{prop:conv-surf-CBB(0)} generalizes to the boundaries of convex bodies  in $\EE^m$ for any $m\ge 2$.
It could be considered as a partial case of the conjecture about the boundary of Alexandrov space; see \ref{conj:bry}.
Another partial case, for Riemannian manifolds with boundary, is proved by the authors and Stephanie Alexander \cite{alexander-kapovitch-petrunin-2008}.


\begin{wrapfigure}{r}{30mm}
\vskip-3mm
\centering
\includegraphics{mppics/pic-15}
\vskip-0mm
\end{wrapfigure}

According to the uniqueness theorem, a convex polyhedron is completely defined by the intrinsic metric of its surface.
In particular, knowing the metric, we could find the position of the edges.
However, in practice, it is not easy to do.
For example, the surface glued from a rectangle, as shown in the picture, defines a tetrahedron.
Some of the glued lines appear inside the facets of the tetrahedron, and some edges (dashed lines) do not follow the sides of the rectangle.


%\chapter[Alexandrov's embedding theorem]{Alexandrov's embedding theorem\\ \textsc{\normalsize by Nina Lebedeva and Anton Petrunin}}\label{chap:embedding}

\section{Introduction}

Intrinsic distance between two points on the surface of a convex polyhedron is defined as the length of a shortest curve on the surface between these points.

Recall that the sum of angles at the tip of a convex polyhedral angle is less than $2\cdot\pi$;
this statement can be found in a school textbook \cite[§~48]{kiselev-stereo-en}.

It is easy to see that the surface of a convex polyhedron is homeomorphic to the sphere.
Therefore the statements above imply that the surface of a convex polyhedron equipped with its intrinsic metric is an example of a \textit{polyhedral metric on the sphere with the sum of angles around each vertex at most $2\cdot\pi$};
a metric is called \emph{polyhedral} if the sphere admits a triangulation such that every triangle is congruent to a plane triangle.

Alexandrov's theorem states that the converse holds if one includes in the consideration \textit{twice covered polygons}.
In other words, we assume that a polyhedron can degenerate to a plane polygon;
in this case, its surface is defined as two copies of the polygon glued along their boundary.

Further, we assume that a polyhedron can degenerate to a plane polygon.

\pagebreak

\begin{thm}{Alexandrov's theorem}
\begin{enumerate}[I.]
\item\label{thm:exist}
A polyhedral metric on the sphere is isometric to the surface of a convex polyhedron if and only if the sum of angles around each of its vertex is not greater than $2\cdot\pi$.

\item\label{thm:unique} 
Moreover, a convex polyhedron is defined up to congruence by the intrinsic metric on its surface.
\end{enumerate}

\end{thm}

A. D. Alexandrov has many remarkable theorems, but in our opinion, this theorem is the most remarkable.
At the same time, its proof is elementary;
it could be explained to anyone familiar with basic topology.

This theorem has many applications.
In particular, it is used in the proof of its generalization \cite{alexandrov-1948} that gives a complete description of intrinsic metrics on the sphere that are isometric to convex surfaces in the Euclidean space.
The latter statement is fundamental in a branch of modern mathematics --- the so-called \emph{Alexandrov geometry}.

The first part is central; it is called the \emph{existence theorem}.
The second part is called the \emph{uniqueness theorem}; it is a slight variation of Cauchy's theorem about polyhedrons.
(There is another uniqueness theorem of Alexandrov that generalizes Minkowski's theorem about  polyhedrons.)

According to the theorem, a convex polyhedron is completely defined by the intrinsic metric of its surface.
In particular, knowing the metric we could find the position of the edges.
However, in practice, it is not easy to do.
For example, the surface glued from a rectangle as shown on the diagram defines a tetrahedron.
Some of the glued lines appear inside facets of the tetrahedron and some edges (dashed lines) do not follow the sides of the rectangle.

{

\begin{wrapfigure}{r}{30mm}
\vskip-3mm
\centering
\includegraphics{mppics/pic-15}
\vskip-0mm
\end{wrapfigure}

The theorem was proved by A. D. Alexandrov in 1941 \cite{alexandrov-1941};
we will present a sketch of his proof.
A complete proof is nicely written by A. D. Alexandrov in his book~\cite{alexandrov}.
Yet another proof was found by Yu.~A.~Volkov in his thesis \cite{volkov};
it uses a deformation of three-dimensional polyhedral space.

}

\section{Space of polyhedrons and metrics}

\paragraph{Space of polyhedrons.}
Let us denote by $\Phi$ the space of all convex polyhedrons in the Euclidean space,
including polyhedrons that degenerate to a plane polygon.
Polyhedra in $\Phi$ will be considered up to a motion of the space, 
and the whole space $\Phi$ will be considered with the natural topology (an intuitive meaning of closeness of two polyhedrons should be sufficient).  

Further, denote by $\Phi_n$ the polyhedrons in $\Phi$ with exactly $n$ vertices.
Since any polyhedron has at least 3 vertices, the space $\Phi$ admits a subdivision into a countable number of subsets $\Phi_3,\Phi_4,\dots$

\paragraph{Space of polyhedral metrics.}
The space of polyhedral metrics on the sphere with the sum of angles around each point at most $2\cdot\pi$ will be denoted by $\Psi$.
The metrics in $\Psi$ will be considered up to an isometry, and the whole space $\Psi$ will be equipped with the natural topology (again, an intuitive meaning of closeness of two metrics is sufficient).

A point on the sphere with the sum of angles strictly less than $2\cdot\pi$ will be called an \emph{essential vertex}.
The subset of $\Psi$ of all metrics with exactly $n$ essential vertices will be denoted by $\Psi_n$.
It is easy to see that any metric in $\Psi$ has at least 3 essential vertices.
Therefore $\Psi$ is subdivided into countably many subsets
 $\Psi_3,\Psi_4,\dots$

\paragraph{From a polyhedron to its surface.}

Recall that the surface of a convex polyhedron is a sphere with a polyhedral metric such that the sum of angles around each point is at most $2\cdot\pi$.
Therefore passing from a polyhedron to its surface defines a map
\[\iota\:\Phi\to \Psi.\]

Note that the number of vertices of a polyhedron is equal to the number of essential vertices of its surface.
In other words, $\iota(\Phi_n)\subset \Psi_n$ for any $n\ge 3$.

\section{About the proof}

Using the notation introduced in the previous section, we can give the following more exact formulation of Alexandrov's theorem: 

\begin{thm}{Reformulation}
For any integer $n\ge 3$,
the map $\iota$ is a bijection from $\Phi_n$ to~$\Psi_n$.
\end{thm}

We sketch the original proof of A. D. Alexandrov.
It is based on the  construction of a one-parameter family of polyhedrons that starts at arbitrary polyhedron
and ends at a polyhedron with its surface isometric to the given one.
This type of argument is called the \emph{continuity method}; it is often used in the theory of differential equations.

\medskip

The two parts of the first formulation will be proved separately.

\parit{Part \ref{thm:unique}.} Let us show that the map $\iota\:\Phi_n\to\Psi_n$ is injective;
in other words, a convex polyhedron is defined by the intrinsic metric on its surface up to a motion of the space.

The last statement is analogous to the Cauchy theorem about polyhedrons,
and the proof goes along the same lines. 

The Cauchy theorem states that facets of a polyhedron together with the gluing rule completely describe a convex polyhedron;
its proof is given in many classical popular texts \cite{aigner-zigler,dolbilin,tabacnikov-fuks}.

\medskip

\parit{Part \ref{thm:exist}.}
Let us prove that $\iota\:\Phi_n\to\Psi_n$ is surjective.
This part of the proof is subdivided into the following lemmas:

\begin{thm}{Lemma}
For any integer $n\ge 3$, the space $\Psi_n$ is connected.
\end{thm}

The proof of this lemma is not complicated, but it requires ingenuity;
it can be done by the direct construction of a one-parameter family of metrics in $\Psi_n$ that connects two given metrics.
Such a family can be obtained by а sequential application of the following construction and its inverse.

Let $M$ be a sphere with metric from $\Psi_n$.
Suppose $v$ and $w$ are essential vertices in $M$.
Let us cut $M$ along a shortest line from $v$ to~$w$.
Note that the shortest line cannot pass thru an essential vertex of $M$.
Further, note that there is a three-parameter family of patches that can be used to patch the cut so that the obtained metric remains in $\Psi_n$;
in particular, the obtained metric has exactly $n$ essential vertices (after the patching, the vertices $v$ and $w$ may become inessential).


\begin{thm}{Lemma}
The map $\iota\:\Phi_n\to\Psi_n$ is open, 
that is, it maps any open set in $\Phi_n$ to an open set in $\Psi_n$.

In particular, for any $n\ge 3$, the image $\iota(\Phi_n)$ is open in~$\Psi_n$.
\end{thm}

This statement is very close to the so-called \emph{invariance of domain theorem};
the latter states that a continuous injective map between manifolds of the same dimension is open.

According to part \ref{thm:unique}, $\iota$ is injective.
The proof of the invariance of domain theorem can be adapted to our case since both spaces $\Phi_n$ and $\Psi_n$ are $(3\cdot n-6)$-dimensional and both look like manifolds, altho, formally speaking, they are \emph{not} manifolds.
In a more technical language, $\Phi_n$ and $\Psi_n$ have the natural structure of $(3\cdot n-6)$-dimensional \emph{orbifolds},
and the map $\iota$ respects the \emph{orbifold structure}.

We will only show that both spaces $\Phi_n$ and $\Psi_n$ are $(3\cdot n-6)$-dimensional.

Choose a polyhedron $P$ in $\Phi_n$.
Note that $P$ is uniquely determined by the $3\cdot n$ coordinates of its $n$ vertices.
We can assume that the first vertex is the origin, the second has two vanishing coordinates and the third has one vanishing coordinate; therefore, all polyhedrons in $\Phi_n$ that lie sufficiently close to $P$ can be described by $3\cdot n-6$ parameters.
If $P$ has no symmetries then this description can be made one-to-one;
in this case, a neighborhood of $P$ in $\Phi_n$ is a $(3\cdot n-6)$-dimensional manifold.
If $P$ has a nontrivial symmetry group, then this description is not one-to-one but it does not have an impact on the dimension of $\Phi_n$.

The case of polyhedral metrics is analogous.
We need to construct a subdivision of the sphere into plane triangles using only essential vertices.
By Euler's formula, there are exactly $3\cdot n-6$ edges in this subdivision.
Note that the lengths of edges completely describe the metric, and slight changes of these lengths produce a metric with the same property.

\begin{thm}{Lemma}
The map $\iota\:\Phi_n\to\Psi_n$ is closed;
that is, the image of a closed set in $\Phi_n$ is closed in $\Psi_n$.

In particular, for any $n\ge 3$, the set $\iota(\Phi_n)$ is closed in~$\Psi_n$.
\end{thm}

Choose a closed set $Z$ in $\Phi_n$.
Denote by $\bar Z$ the closure of $Z$ in $\Phi$; note that $Z=\Phi_n\cap \bar Z$.
Assume $P_1,P_2,\dots\in Z$ is a sequence of polyhedrons that converges to a polyhedron $P_\infty\in\bar Z$.
Note that $\iota(P_n)$ converges to $\iota(P_\infty)$ in $\Psi$.
In particular, $\iota(\bar Z)$ is closed in $\Psi$.

Since $\iota(\Phi_n)\subset \Psi_n$ for any $n\ge 3$, we have  $\iota (Z)=\iota(\bar Z)\cap \Psi_n$;
that is, $\iota (Z)$ is closed in $\Psi_n$. 

\medskip

Summarizing, $\iota(\Phi_n)$ is a nonempty closed and open set in $\Psi_n$, and $\Psi_n$ is connected for any $n\ge 3$.
Therefore, $\iota(\Phi_n)=\Psi_n$; that is, $\iota\:\Phi_n\z\to\Psi_n$ is surjective.
\qeds

\parbf{Acknowledgments.} We want to thank Stephanie Alexander, Yuri Burago, and Jules %Kiyoshi
Tsukahara for help. 
The authors were partially supported by RFBR grant 20-01-00070 and NSF grant DMS-2005279.


\backmatter

%%!TEX root = the-sols.tex

\chapter{Semisolutions}

\parbf{\ref{ex:compact+connceted}.}
Choose a sequence of positive numbers $\varepsilon_n\to 0$ and a finite $\varepsilon_n$-net $N_n$ of $K$ for each $n$.
%???eps-net are not defined!!!
We can assume that $\eps_0>\diam K$, and $N_0$ is a one-point set.
If $\dist{x}{y}{}<\eps_k$ for some $x\in N_{k+1}$ and $y\in N_{k}$, then connect them by a curve of length at most $\eps_k$.

Let $K'$ be the union of all these curves and $K$.
Show that $K'$ is compact and path-connected.

\parit{Source:} This problem is due to Eugene Bilokopytov \cite{bilokopytov}.

\parbf{\ref{ex:compact=>complete}.}
Choose a Cauchy sequence $x_n$ in $(\spc{X},\|*\z-*\|)$; it is sufficient to show that a subsequence of $x_n$ converges.

Observe that the sequence $x_n$ is Cauchy in $(\spc{X},|*-*|)$;
denote its limit by $x_\infty$.

Passing to a subsequence, we can assume that $\|x_n-x_{n+1}\|\z<\tfrac1{2^n}$.
It follows that there is a 1-Lipschitz path $\gamma$ in $(\spc{X},\|*-*\|)$ such that $x_n=\gamma(\tfrac1{2^n})$ for each $n$ and $x_\infty=\gamma(0)$.
Therefore,
\begin{align*}
\|x_\infty-x_n\|&\le \length\gamma|_{[0,\frac1{2^n}]}\le \tfrac1{2^n}.
\end{align*}
In particular, $x_n$ converges to $x_\infty$ in $(\spc{X},\|*\z-*\|)$.

\parit{Source:} \cite[Corollary]{hu-kirk}; see also \cite[Lemma 2.3]{petrunin-stadler}.

\parbf{\ref{ex:compact-length}.}
Given a pair of points $p$ and $q$, choose a sequence of paths $\gamma_n$ from $p$ to $q$ such that
\[\length\gamma_n\to \dist pq{}
\quad\text{as}\quad
n\to\infty;\]
these paths exist since we are in a length space.
Note that we can assume that each $\gamma_n$ is parametrized proportionally to the arc length;
in particular, $\gamma_n$ are equicontinuous.
Show that paths $\gamma_n$ lie in a closed ball, say $\cBall[p,r]$ of some radius $r<\infty$.
Since the space is proper, $\cBall[p,r]$ is compact.
By the Arzelà--Ascoli theorem, we can pass to a converging subsequence of $\gamma_n$.
Show that its limit is a geodesic path from $p$ to $q$.

\parbf{\ref{ex:menger}.}
Choose a sequence $\eps_n>0$ that converges to zero very fast, say such that $\sum_n10^n\cdot \eps_n$ is small.
Follow the argument in the proof of Menger's lemma, taking $\eps_n$-midpoints at the $n^{\text{th}}$ stage.

\parbf{\ref{ex:k-><mono}.}
Let us write the Riemannian metric on $\MM^2(\kappa)$ in polar coordinates $(\theta,r)$;
it has the form 
$(\begin{smallmatrix}
h^2&0
\\
0&1
\end{smallmatrix})$, where $h=h(\kappa,r)\ge 0$.
Calculate $h(\kappa,r)$.
Show that for fixed $r$, the function $r\mapsto h(\kappa,r)$ is nonincreasing in the domain of definition.
Suppose $\kappa<\Kappa$, consider the partially defined map $\MM^2(\kappa)\to\MM^2(\Kappa)$ that sends a point to the point with the same polar coordinates.
Show that this map is short in the domain of definition.
Use it to prove the statement in the exercise.


\parbf{\ref{ex:angkK}.} Show and use that 
$\angk p{x}{y}_{\SSS^2}-\angk p{x}{y}_{\EE^2}=O(\dist[2]{p}{x}{}+\dist[2]{p}{y}{})$
and 
$\angk p{x}{y}_{\EE^2}-\angk p{x}{y}_{\HH^2}=O(\dist[2]{p}{x}{}+\dist[2]{p}{y}{})$.

\parbf{\ref{ex:undefined-angle}.}
Consider a hinge in the plane $\RR^2$ with a metric defined by norm, say by the $\ell^\infty$-norm.

\parbf{\ref{ex:adjacent-angles}.}
Assume $\mangle\hinge pxz+\mangle\hinge pyz<\pi$.
By \ref{claim:angle-3angle-inq}, $\mangle\hinge pxy<\pi$.
Therefore,
$\angk p{\bar x}{\bar y}<\pi$
for some $\bar x\in \left]px\right]$ and $\bar y\in \left]py\right]$.
Hence 
\[\dist p{\bar x}{}+\dist {\bar y}p{}<\dist {\bar x}{\bar y}{}\]
--- a contradiction.

\parbf{\ref{ex:first-var}.}
Denote by $\alpha$ the arc-length parametrization of $[qp]$ from $q$ to $p$.
Choose $\eps>0$.
Observe that 
\[\dist[2]{\gamma(t)}{\alpha(\tfrac1\eps\cdot t)}{}\le t^2\cdot(1-\tfrac2\eps\cdot\cos\phi+\tfrac1{\eps^2})+o(t^2),\]
where $\phi=\mangle\hinge q p x$.
By the triangle  inequality
\[\dist{p}{\gamma(t)}{}\le \dist{\gamma(t)}{\alpha(\tfrac1\eps\cdot t)}{}+\dist{q}{p}{}-\tfrac1\eps\cdot t.\]
Conclude that
\[\dist{p}{\gamma(t)}{}
\le
\dist{q}{p}{}-t\cdot \cos \phi+\delta(\eps)\cdot t+o(t),\]
where $\delta(\eps)\to 0$ as $\eps\to0$.
The statement follows since $\eps>0$ is arbitrary.

\parbf{\ref{ex:generalized-selection}.}
Since the space is proper, it is separable; 
that is, we can choose an countable everywhere dense set $\{x_1,x_2,\dots\}$.

Let $A_1,A_2,\dots$ be a sequence of closed sets.
Applying the diagonal procedure, we can pass to a subsequence such that for each $i$ the sequence $\distfun_{A_n}x_i$ converges as $n\to\infty$;
denote its limit by $f(x_i)$.

Since $\distfun_{A_n}$ is $1$-Lipschitz for any $n$, we have 
\[|f(x_i)-f(x_j)|\le \dist{x_i}{x_j}{}\]
for all $i$ and $j$.
Suppose $f(x_i)<\infty$ for some $i$; note that the same holds for any $i$.
Therefore, the function $f$ can be extended to a continuous function defined on the whole ambient space.
Show that $A_\infty=f^{-1}\{0\}$ is the limit of $A_n$ in the sense of Hausdorff.

If $f(x_i)=\infty$ for some $i$, then the same holds for any $i$.
Show that in this case $A_n\to\emptyset$ in the sense of Hausdorff.

\parbf{\ref{ex:Haus-conv}.}
Apply the definition of Hausdorff distance (\ref{def:hausdorff-convergence}).

\parbf{\ref{ex:geod-closed}.}
Given $x_\infty,y_\infty\in\spc{X}_\infty$, choose $x_n,y_n\in \spc{X}_n$ such that $x_n\to x_\infty$ and $y_n\to y_\infty$.
Let $z_n$ be the midpoint of $[x_ny_n]$.
Since $\spc{X}_\infty$ is proper, we can choose a subsequence of $z_m$ that converges to a point, say $z_\infty\in \spc{X}_\infty$.
Note that $z_\infty$ is a midpoint of $x_\infty$ and $y_\infty$, then apply Menger's lemma (\ref{lem:mid>geod}).

\parbf{\ref{ex:non-contracting-map}.}
Given a pair of points $x_0,y_0\in \spc{K}$, 
consider two sequences $x_0,x_1,\dots$ and $y_0,y_1,\dots$
such that $x_{n+1}=f(x_n)$ and $y_{n+1}\z=f(y_n)$ for each $n$.

Since $\spc{K}$ is compact, 
we can choose an increasing sequence of integers $n_k$
such that both sequences $(x_{n_i})_{i=1}^\infty$ and $(y_{n_i})_{i=1}^\infty$
converge.
In particular, both are Cauchy;
that is,
\[
|x_{n_i}-x_{n_j}|_{\spc{K}}\to 0 
\quad\text{and}\quad
|y_{n_i}-y_{n_j}|_{\spc{K}}\to 0
\]
as $\min\{i,j\}\to\infty$.

Since $f$ is distance-noncontracting, 
\[
|x_0-x_{|n_i-n_j|}|
\le 
|x_{n_i}-x_{n_j}|
\]
for any $i$ and $j$.
Therefore, there is a sequence $m_i\to\infty$ such that
\[
x_{m_i}\to x_0\quad\text{and}\quad y_{m_i}\to y_0
\leqno({*})\]
as $i\to\infty$.

Since $f$ is distance-noncontracting, the sequence $\ell_n=|x_n-y_n|_{\spc{K}}$ is nondecreasing.
By $({*})$,  $\ell_{m_i}\to\ell_0$ as $m_i\to\infty$.
It follows that 
\[\ell_0=\ell_1=\dots\]
In particular, 
\[|x_0-y_0|_{\spc{K}}=\ell_0=\ell_1=|f(x_0)-f(y_0)|_{\spc{K}}\]
for any pair of points $(x_0,y_0)$ in $\spc{K}$.
That is, the map $f$ is distance-preserving; hence $f$ is injective.
From $({*})$, we also get that $f(\spc{K})$ is everywhere dense.
Since $\spc{K}$ is compact $f\:\spc{K}\to \spc{K}$ is surjective --- hence the result.

\parit{Remarks.}
This is a basic lemma in the introduction to Gromov--Hausdorff distance \cite[see 7.3.30 in][]{burago-burago-ivanov}.
The presented proof is not quite standard;
I learned it from Travis Morrison, 
a student in my MASS class at Penn State, Fall 2011.

Note that this exercise implies that \textit{any surjective non-expanding map from a compact metric space to itself is an isometry}.

\parbf{\ref{ex:GH-po}.}
The only-if part is trivial. 
Let us prove the if part.

If $\dist{\spc{X}_n}{\spc{X}_\infty}{\GH}\not\to 0$, then we can pass to a subsequence such that $\dist{\spc{X}_n}{\spc{X}_\infty}{\GH}\ge\eps$ for some $\eps>0$.
Show that we can pass to a subsequence again, so that $\spc{X}_n$ converges in the sense of Gromov--Hausdorff, say to $\spc{Y}$.
Observe that $\spc{Y}\le \spc{X}_\infty$ and $\spc{X}_\infty\le\spc{Y}$.
By \ref{ex:non-contracting-map}, $\spc{Y}\iso \spc{X}_\infty$ --- a contradiction.

\parbf{\ref{ex:compact-GH}.} Show and use that $\dist{\spc{X}_\infty}{\spc{X}_\infty'}{\GH}<\eps$ for any $\eps>0$.



\parbf{\ref{ex:GH-noncompact}.  } \parit{\ref{SHORT.ex:GH-noncompact:proper}}
 Consider the graphs of the following functions with the induced metric from $\RR^2$.
\[
x\mapsto \cos x+\cos \tfrac x\pi
\quad\text{and}\quad
x\mapsto \cos x+\sin \tfrac x\pi.
\]


\parit{\ref{SHORT.ex:GH-noncompact:bounded}}
For every rational number  $q\in[1,2]$ consider an interval of length $q$. Let $\spc{X}$ be obtained by identifying all  initial points of  the intervals to one point and all  end points to another.

Let $\spc{Y}$ be constructed in the same way but skipping the interval of length $1.5$.



\parbf{\ref{ex:Euclid-is-CBB}.}
The 4-point comparison (\ref{def:CBB}) reduces our question to the following.
\textit{Any spherical triangle has perimeter at most $2\cdot\pi$.}
Choose a spherical triangle $[xyz]$.
Let $x'$ be the antipode of $x$; that is $x'=-x$.
The spherical triangle inequality (\ref{claim:angle-3angle-inq} or \ref{ex:angle-triangle}) implies that
\[\dist{x}{z}{\mathbb{S}^2}\le \dist{y}{x'}{\mathbb{S}^2}+\dist{x'}{z}{\mathbb{S}^2}.\]
Observe that 
\[
\dist{x}{y}{\mathbb{S}^2}+\dist{y}{x'}{\mathbb{S}^2}=\pi,
\quad\text{and}\quad
\dist{x}{z}{\mathbb{S}^2}+\dist{z}{x'}{\mathbb{S}^2}=\pi.
\]
Hence
\[\dist{x}{y}{\mathbb{S}^2}+\dist{x}{z}{\mathbb{S}^2}+\dist{y}{z}{\mathbb{S}^2}\le2\cdot \pi.\]

\parbf{\ref{ex:(3+1)-expanding}.} For the only-if part consider the following two cases.

If $\angk p{x_1}{x_2}+\angk p{x_2}{x_3}\ge \pi$, then choose two model triangles $[qy_1y_2]\z=\modtrig(px_1x_2)$ and $[qy_2y_3]=\modtrig(px_2x_y)$ that lie on the opposite sides of $[qy_2]$.
By the comparison, $\dist{y_1}{y_3}{}\ge \dist{x_1}{x_3}{}$.
Therefore the obtained configuration meets all the conditions.

If $\angk p{x_1}{x_2}+\angk p{x_2}{x_3}\ge \pi$, then choose two model triangles $[qy_1y_2]\z=\modtrig(px_1x_2)$
and take $y_3$ on the extension of $[y_1q]$ behind $q$ such that $\dist{q}{y_3}{}=\dist{p}{x_3}{}$.
Then $\mangle \hinge q{y_2}{y_3}\ge \angk p{x_2}{x_3}$, therefore $\dist{y_2}{y_3}{}\ge \dist{x_2}{x_3}{}$.
Further, $\dist{y_2}{y_3}{}=\dist{x_2}{p}{}+\dist{p}{x_3}{} \ge \dist{x_2}{x_3}{}$,
and again, the obtained configuration meets all the conditions.

To prove the if part, choose a configuration $q,y_1,y_2,y_3$ that meets all the conditions and maximize the sum
\[\dist{y_1}{y_2}{}+\dist{y_2}{y_3}{}+\dist{y_3}{y_1}{}.\]
Show that $q$ lies in the solid triangle $y_1y_2y_3$;
in particular 
\[\mangle \hinge q{y_1}{y_2}+\mangle \hinge q{y_2}{y_3}+ \mangle \hinge q{y_3}{y_1}=2\cdot\pi.\]
Moreover, $\dist{q}{y_i}{}=\dist{p}{x_i}{}$ for each $i$.
Applying that increasing the opposite side in a plane triangle increases the corresponding angle, we get 
\[\angk  p{x_1}{x_2}+\angk p{x_2}{x_3}+\angk p{x_3}{x_1}
\le 
2\cdot\pi.
\]

\parbf{\ref{ex:alex-lemma-cat}.}
Consider model triangles $[\tilde p\tilde x\tilde z]=\modtrig(pxz)$ and $[\tilde p\tilde y\tilde z]=\modtrig(pyz)$
that share side $[\tilde p\tilde z]$ and lie on its opposite sides.
Note that 
\begin{align*}
\dist{\tilde x}{\tilde y}{\EE^2}
&\ge \dist{\tilde x}{\tilde y}{\EE^2}+\dist{\tilde x}{\tilde y}{\EE^2}=
\\
&=\dist{x}{z}{\spc{X}}+\dist{z}{y}{\spc{X}}=
\\
&=\dist{x}{y}{\spc{X}},
\end{align*}
where $\spc{X}$ is our metric space.
It remains to apply the monotonicity of angle in a triangle with respect to its opposite side. 


\parbf{\ref{ex:noncreasing}.}
Apply \ref{clm:angle-mono}.

\parbf{\ref{ex:0-angle}.}
Without loss of generality, we can assume that $\dist{p}{x}{}\le \dist{p}{y}{}$.
Choose $\bar x\in [px]$;
let $\bar y\in [px]$ be such that $\dist{p}{\bar x}{}=\dist{p}{\bar y}{}$.
Apply \ref{clm:angle-mono} to show that $\bar x=\bar y$.
Conclude that $[px]\subset [py]$.

\parbf{\ref{ex:pi-angle}.}
Assume that there are two distinct geodesics from $z$ to $x$.
Then we can choose distinct points $p$ and $q$ on these geodesics such that $\dist{z}{p}{}=\dist{z}{q}{}$.
Observe that $\angk zpq>0$.
By the triangle inequality, we get 
\[\dist{x}{p}{}+\dist{p}{y}{}\le \dist{x}{p}{}+\dist{p}{z}{}+\dist{z}{y}{}=\dist{x}{z}{}+\dist{z}{y}{}\]
Observe that $\angk zxy=\pi$.
Therefore $\mangle\hinge zxy=\pi$ for any geodesic $[zx]$.

\parbf{\ref{ex:adjacent-CBB}.}
By \ref{ex:adjacent-angles}, we have
\[\mangle\hinge pxz+\mangle\hinge pyz\ge \pi.\]
Since $z\in \left]xy\right[$ we have 
\[\angk z{\bar x}{\bar y}=\pi\]
for any $\bar x\in \left[xz\right[$ and $\bar y\in \left]zy\right]$.
By comparison, we have that 
\[\angk z{\bar x}{\bar p}+\angk z{\bar p}{\bar y}\le\pi\]
for any $\bar p\in \left]zp\right]$.
Passing to the limit as
$\dist{z}{\bar x}{}\to 0$,
$\dist{z}{\bar y}{}\to 0$, and
$\dist{z}{\bar p}{}\to 0$,
we get the statement.

\parbf{\ref{ex:pxyvw}.} 
Without loss of generality, we can assume that $x$, $v$, $w$, and $y$ appear on 
$[xy]$ in this order.
By \ref{clm:angle-mono},
\[
\angk xyp\ge \angk xwp \ge\angk xvp.
\]
Hence, $\Rightarrow$ follows.

By Alexandrov's lemma,
\begin{align*}
\angk xyp=\angk xvp
\quad&\Longleftrightarrow\quad
\angk yxp=\angk yvp,
\\
\angk xyp=\angk xwp
\quad&\Longleftrightarrow\quad
\angk yxp=\angk ywp.
\end{align*}
Whence, $\Leftarrow$ follows.

\parbf{\ref{ex:angle-lim}.} Suppose $\mangle \hinge {x_\infty}{y_\infty}{z_\infty}>\alpha$.
Then we can choose $\bar y_\infty\in\left]x_\infty y_\infty\right]$
and $\bar z_\infty\in\left]x_\infty z_\infty\right]$ such that 
$\angk{x_\infty}{\bar y_\infty}{\bar z_\infty}>\alpha$.
Now choose $\bar y_n\in\left]x_n y_n\right]$ and $\bar y_n\in\left]x_n z_n\right]$ such that $\bar y_n\to \bar y_\infty$ and $\bar z_n\to \bar z_\infty$.
Observe that 
\[\liminf_{n\to\infty}\mangle \hinge {x_n}{y_n}{z_n}\ge\liminf_{n\to\infty}\angk{x_n}{\bar y_n}{\bar z_n} \ge \alpha,\]
hence the result.

\parbf{\ref{ex:urysohn}.}
The Urysohn space provides an example;
see for example \cite[Lecture 2]{petrunin2023pure}.

\parbf{\ref{ex:normCBB}.}
Choose a triangle $[0vw]$.
Note that $m=\tfrac12(v+w)$ is the midpoint of $[vw]$.

Use comparison, to show that
\[2\cdot |\tfrac12(v+w)|^2+2\cdot |\tfrac12(v-w)|^2\ge |v|^2+|w|^2.\]

Note this inequality implies the opposite one;
it follows if we rewrite it via $x=\tfrac12(v+w)$ and $y=\tfrac12(v-w)$.
Hence we have 
\[2\cdot |\tfrac12(v+w)|^2+2\cdot |\tfrac12(v-w)|^2= |v|^2+|w|^2\]
for any $v,w$.
That is, the norm is quadratic and the statement follows.

\parbf{\ref{ex:alm-min}.}
Suppose such a point does not exist;
that is, for any $p\in \spc{X}$ there is a point $p'$ such that $r(p')\le  (1-\eps)\cdot r(p)$ and $\dist p{p'}{}<\tfrac{1}{\eps}\cdot r(p)$.
Construct a sequence of points $p_0,p_1,\dots$ such that $p_n=p_{n-1}'$ for any~$n$.
Show that this sequence is Cauchy; denote its limit by $p_\infty$.
Arrive at a contradiction by showing that $r(p_\infty)\le0$.

\parbf{\ref{ex:CBB(1)notitCBB(0)}.}
Note that $\spc{X}$ has no defined sphericlal model angles;
therefore it has curvature $\ge 1$.

However, $\spc{X}$ does not have curvature $\ge 0$ since
\[\angk  p{x_1}{x_2}_{\EE^2}=\angk  p{x_2}{x_3}_{\EE^2}=\angk  p{x_1}{x_3}_{\EE^2}=\pi.\]

\parbf{\ref{ex:RisCBB(1)}.}
Suppose $\mangle\hinge mxp\ne 0$ and $\mangle\hinge mxp\ne\pi$, or equivalently $\mangle\hinge mxq\ne0$.

We can assume that $\dist pq{}$ only slightly exceeds $\pi$,
so $\dist pm{}<\pi$ and $\dist qm{}<\pi$.
We can also assume that $\dist xm{}<\pi$.
Use the comparison to show that 
\[\dist px{}+\dist qx{} < \dist pq,\]
and arrive at a contradiction with the triangle inequality.

Extend $[pq]$ to a maximal local geodesic $\gamma$.
It might be a closed or a line segment.
Argue as above to show that any point lies on $\gamma$ and make a conclusion.

\parbf{\ref{ex:perim-k>0}.}
Arguing by contradiction, suppose 
\[\dist{p}{q}{}+\dist{q}{r}{}+\dist{r}{p}{}> 2\cdot\pi\eqlbl{eq:perimeter-of-triange<2pi}\] 
for $p,q,r\in \spc{A}$. 
Rescaling the space slightly, we can assume that $\diam\spc{A}<\pi$,
but the inequality \ref{eq:perimeter-of-triange<2pi} still holds.
By \ref{clm:K>k},
after rescaling $\spc{A}$ is still $\Alex1$.

Take $z_0\in [q r]$ on maximal distance from $p$.
Consider the following model configuration:
two geodesics $[\tilde p\tilde z_0]$, $[\tilde q\tilde r]$ in $\mathbb{S}^2$ such that 
\begin{align*}
\dist{\tilde p}{\tilde z_0}{}&=\dist{p}{z_0}{},
&  
\dist{\tilde q}{\tilde r}{}&=\dist{q}{r}{},
\\ 
\dist{\tilde z_0}{\tilde q}{}&=\dist{z_0}{q}{},
&  
\dist{\tilde z_0}{\tilde r}{}&=\dist{z_0}{q}{},
\end{align*}
and 
\[\mangle\hinge{\tilde z_0}{\tilde q}{\tilde p}
=\mangle\hinge{\tilde z_0}{\tilde r}{\tilde p}
=\tfrac\pi2.\]

Let $\tilde z\in [\tilde q\tilde r]$,
and let $z\in [q r]$ be the corresponding point.
By comparison, $\dist pz{}\le\dist {\tilde p}{\tilde z}{}$ for points $z$ near $z_0$.
Moreover, this inequality holds as far as 
\[\dist{\tilde p}{\tilde z_0}{}+\dist{\tilde z_0}{\tilde z}{}+\dist{\tilde p}{\tilde z}{}<2\cdot\pi.\]
But this inequality holds for all $\tilde z$ since  $\dist{\tilde p}{\tilde z_0}{}<\pi$, $\dist{\tilde z_0}{\tilde q}{}<\pi$, and $\dist{\tilde z_0}{\tilde r}{}<\pi$.
Hence we get $\dist pq{}\le\dist {\tilde p}{\tilde q}{}$ and $\dist pr{}\le\dist {\tilde p}{\tilde r}{}$.
The latter contradicts \ref{eq:perimeter-of-triange<2pi}.

\parbf{\ref{ex:dir-compact}.}
Suppose $\dir p{x_n}\not\to\dir p{x_\infty}$.
Since $\Sigma_p$ is compact, we may pass to a converging subsequence of $\dir p{x_n}$;
denote by $\xi$ its limit.
We may assume that $\mangle (\dir p{x_\infty},\xi)>0$.

Denote by $\gamma_n$ and $\gamma_\infty$ the arc-length parametrization of $[px_n]$ and $[px_\infty]$ from $p$.
Choose a geodesic $\alpha$ that starts from $p$ and goes in a direction sufficiently close to $\xi$.
By comparison we can choose $\alpha$ so that
\[\dist{\alpha(t)}{\gamma_n(t)}{}<\eps\cdot t\]
for all large $n$ and all sufficiently small $t$.
Moreover, we can assume that
\[\dist{\alpha(t)}{\gamma_\infty(t)}{}>a\cdot t\]
for some fixed $a>0$ and all small $t$.
These two inequalities imply 
that 
\[\dist{\gamma_n(t)}{\gamma_\infty(t)}{}>\tfrac a2\cdot t\]
for all small $t$ and all large $n$.
On the other hand, by assumption, $\dist{\gamma_n(t)}{\gamma_\infty(t)}{}\to0$ as $n\to\infty$ --- a contradiction.

\parit{Comments.}
The compactness of $\Sigma_p$ is necessary.
An example can be built using iterated warped product of line segments and applying \cite[Theorem 1.2]{alexander-bishop2004}.
The space $\spc{A}$ can be assumed to be compact.


\parbf{\ref{ex:geodesic-cone}.}
Note that any point of $\Cone \spc{X}$ can be connected to the origin by a geodesic.
Given a nonzero element $v\in\Cone \spc{X}$, denote by $v'$ its projection in $\spc{X}$.

Suppose $\spc{X}$ is $\pi$-geodesic.
Choose two nonzero elements $v,w\in\Cone \spc{X}$; let $\alpha=\mangle(v,w)=\dist{v'}{w'}{\spc{X}}$.
If $\alpha\ge \pi$, then the product of geodesics $[v0]\cup [0w]$ forms a geodesic $[vw]$.
If $\alpha<\pi$, there is a geodesic $\gamma\:[0,\alpha]\to \spc{X}$ from $v'$ to $w'$.
Consider hinge $\hinge {\tilde o}{\tilde v}{\tilde w}$ in the plane 
such that $\mangle\hinge {\tilde o}{\tilde v}{\tilde w}=\alpha$, $\dist{\tilde o}{\tilde v}{}=|v|$, and $\dist{\tilde o}{\tilde w}{}=|w|$.
Let $t\mapsto (\phi(t),r(t))$ be geodesic $[\tilde v\tilde w]$ written in polar coordinates with origin $\tilde o$, so that $\phi(0)=0$.
Show that $t\mapsto r(t)\cdot\gamma\circ\phi(t)$ is a geodesic from $v$ to $w$;
here we identify $\spc{X}$ with the unit sphere in $\Cone \spc{X}$.

To prove the converse, try to reverse the steps in the argument above.

\parbf{\ref{ex:GHto-tangent}.}
Let  $\spc{A}_n=\lambda_n\cdot\spc{A}$.
Note that for any $n$ the space $\Sigma_p\spc{A}$ is identical to $\Sigma_{\iota_n(p)} \spc{A}_n$.
In particular, we can identify isometrically $\T_p\spc{A}$ with $\T_{\iota_n(p)}(\lambda\cdot \spc{A})$.
So for any geodesic $\gamma$ that starts at $p$, the vector $\gamma^+(0)$ corresponds to $\frac{1}{\lambda}\cdot(\iota_n\circ\gamma)^+(0))$.

Consider the logarithm maps $f_n=\log_{\iota_n(p)}\:\spc{A}_n\to T_p\spc{A}$.
We claim that this sequence of maps satisfies the assumptions of Lemma~\ref{lem:almost-isom-pointed};
the condition in \ref{SHORT.lem:almost-isom-pointed-basepoint} is evident.  

Note that it is sufficient to check the conditions in \ref{SHORT.lem:almost-isom-pointed-b} and \ref{SHORT.lem:almost-isom-pointed-c} only for $R=1$. 

Choose $\eps>0$.
By compactness of $\Sigma_p$ we can find a finite $\eps$-net $\xi_1,\dots,\xi_N$ in $\Sigma_p$. Moreover, without loss of generality we can assume that these directions are geodesic;
that is, there exist geodesics $\gamma_1,\ldots, \gamma_N$ starting at $p$ such that $\xi_i=\gamma_i^+(0)$ for each $i$.

Choose $T>0$ such that all $\gamma_i$ are defined on $[0,T]$.
Apply the comparison to show that for any $\lambda_n>\frac{1}{T}$ the image under $f_n$ of the union $\bigcup_N\gamma_i([0,T])$ is an $\eps$-net in $\oBall(0,1)_{T_p}$.
This proves \ref{SHORT.lem:almost-isom-pointed-c}.

By comparison, we have that
\[\dist{\xi_i}{\xi_j}{\Sigma_p}\ge \angk p{\gamma_i(t_i)}{\gamma_j(t_j)}<\eps\]
for all $i\ne j$ and any $t_i,t_j\in (0,T]$.
By the definition of an angle, we can assume that $T$ have been chosen so that in addition 
\[\dist{\xi_i}{\xi_j}{\Sigma_p}\le \angk p{\gamma_i(t)}{\gamma_j(t)}+\eps\]
for all $i\ne j$ and any $t\in (0,T]$.

By construction of the map $f_n$ this implies that 
\[|\dist{x}{x'}{\spc{A}_n}-\dist{f_n(x)}{f_n(x')}{T_p}|<\eps\]
for all $\lambda_n>\frac{1}{T}$ and all points $x,x'$ in $\bigcup_N\gamma_i([0,\frac{1}{\lambda_n}])\subset \oBall(p,1)_{\spc{A}_n}$.
  
Now hinge comparison and the triangle inequality imply that the same  holds for arbitrary points $x,x'$  in  $\oBall(p,1)_{\spc{A}_n}$ with $\eps$ replaced by $3\eps$.
This verifies \ref{SHORT.lem:almost-isom-pointed-b}.

\parbf{\ref{ex:distfun-semiconcave}.} From \ref{comp-kappa}, this inequality follows in the sense of distributions, and hence in any other sense.

\parbf{\ref{ex:df(xi)}.}
Since angles are defined, it follows that 
\[\dist{\gamma_1(t)}{\gamma_2(t)}{}\le \theta\cdot t\]
for all small $t>0$.     
Since $f$ is $L$-Lipschitz, we get 
\[|f(\gamma_1(t))-f(\gamma_2(t))|\le L\cdot \theta\cdot t,\]
hence the statement.

\parbf{\ref{ex:d(distfun)}}; \ref{SHORT.ex:d(distfun):<}
Note that we can assume there is a geodesic in the direction of $v$, and apply \ref{ex:first-var}.

\parit{\ref{SHORT.ex:d(distfun):=}.}
By \ref{SHORT.ex:d(distfun):<}, $\dd_p\distfun_q(v)\le-\max_{\xi\in\Uparrow_p^q}\langle\xi,v\rangle$.
Suppose this inequality is strict for some $v$.
We can assume that $|v|=1$ and there is a geodesic, say $\gamma$ in the direction of $v$.
Let $\dd_p\distfun_q(v)=-\cos\alpha_0$ for some $\alpha\in [0,\pi]$.
Note that any geodesic from $p$ to $q$ makes angle bigger than $\alpha_0$ with $\gamma$.


The function $f=\distfun_q\circ\gamma$ is Lipschitz.
By Rademacher's theorem it is differentiable almost everywhere;
moreover, 
\[f(t)-f(0)=\int_0^t f'(t)\cdot dt.\]
Suppose $f'(t)$ is defined.
Use \ref{SHORT.ex:d(distfun):<} to show that 
$f'(t)=-\cos\alpha(t)$, where $\alpha(t)$ is the angle between $\gamma$ and any geodesic from $\gamma(t)$ to $q$.
Note that we can choose a sequence $t_n\to 0$ such that 
\[\lim_{n\to\infty}\alpha(t_n) \le \alpha_0.\]
Consider a sequence of geodsics $[p\,\gamma(t_n)]$.
Since the space is proper, we can pass to its convergent subsequence.
Its limit is a geodesic from $p$ to $q$, denote it by $[pq]$.

Use \ref{ex:angle-lim} to show that $[pq]$ makes an angle at most $\alpha_0$ with $\gamma$ --- a contradiction.
 
\parbf{\ref{ex:monotonicity}.}
Let $\gamma\:[0,\ell]\to \spc{A}$ be the geodesic $[xy]$ parametrized from $x$ to $y$,
and let $\phi=f\circ\gamma$.
Observe that 
\[\phi'(0)=\dd_xf(\dir xy)\le \<\dir{x}{y},\nabla_{x}f\>.\]
The same way we get $-\phi'(\ell)\le \<\dir{y}{x},\nabla_{y}f\>$.
Since $f$ is $\lambda$-concave, we have
\begin{align*}
f(y)&\le f(x)+\phi'(0)\cdot \ell+\tfrac\lambda2\cdot\ell^2,
\\
f(x)&\le f(y)-\phi'(\ell)\cdot \ell+\tfrac\lambda2\cdot\ell^2.
\end{align*}
Hence the statement follows.

\parbf{\ref{ex:d(distfun):==}.}
If the space is proper, then the statement follows from \ref{SHORT.ex:d(distfun):=} and \ref{ex:pi-angle}.

To do the general case argue by contradiction.
Let $z$ be a point on the extension of $[pq]$ behind $q$;
it exists by the assumption.
Note that we can assume that $|v|=1$ and it is a direction of a geodesic, say $[px]$.

Show that for there is a sequence $x_n\in \left]px\right]$ such that $\dist{p}{x_n}{}\to0$ ad
$\mangle \hinge q{x_n}p>\eps$ for each $n$ and some fixed $\eps>0$.
Observe that $\mangle\hinge q{x_n}z\z<\pi-\eps$; therefore
\[\dist{z}{x_n}{}<\dist{x_n}{q}{}+\dist{q}z{}-\delta\]
for each $n$ and some fixed $\delta>0$.
Pass to the limit as $x_n\to p$ and arrive at a contradiction.

\parbf{\ref{ex:convergence-grad}.}
Note that
$|(\dd_p f)(v)-(\dd_p g)(v)|\le s\cdot|v|$
for any $v\in \T_p$.
From the definition of gradient (\ref{def:grad}) we have:
\begin{align*}
&(\dd_p f)(\nabla_p g)\le\<\nabla_p f,\nabla_p g\>,
&&(\dd_p g)(\nabla_p f)\le\<\nabla_p f,\nabla_p g\>,
\\
&(\dd_p f)(\nabla_p f)=\<\nabla_p f,\nabla_p f\>,
&&(\dd_p g)(\nabla_p g)=\<\nabla_p g,\nabla_p g\>.
\end{align*}
Therefore,
\begin{align*}
&\dist[2]{\nabla_pf}{\nabla_pg}{}
=\<\nabla_p f,\nabla_p f\>+\<\nabla_p g,\nabla_p g\>-2\cdot\<\nabla_p f,\nabla_p g\>
\le
\\
&\le (\dd_p f)(\nabla_p f)+(\dd_p g)(\nabla_p g)-
(\dd_p f)(\nabla_p g)-(\dd_p g)(\nabla_p f)
\le
\\
&\le s\cdot(|\nabla_p f|+|\nabla_p g|).
\end{align*}

\parbf{\ref{ex:semicontinuous-grad}.}
Suppose $|\nabla_xf|> s$.
Then we can choose a geodesic $\gamma$ that starts at $x$ such that 
$(f\circ\gamma)^+(0)>s$.
In particular, there is $\eps>0$ such that
\[f\circ\gamma(t)>(s+\eps)\cdot t+o(t),\]
hence the only-if part follows.

Now suppose $f(y)-f(x)>s\cdot \ell+\lambda\cdot \tfrac{\ell^2}2$,
were $\ell=\dist{x}{y}{}$.
Let $\gamma\:[0,\ell]\to \spc{A}$ be a geodesic from $x$ to $y$.
Since $f\circ\gamma$ is $\lambda$-concave, we have
\[f\circ\gamma(\ell)\le f\circ\gamma(0)+(f\circ\gamma)^+(0)\cdot\ell+\lambda\cdot \tfrac{\ell^2}2.\]
It follows that 
\[\dd_x(\dir xy)=(f\circ\gamma)^+(0)>s,\]
and by \ref{prop:grad-exist}, $|\nabla_x f|>s$.

\parbf{\ref{ex:elf-contracting}.}
Note that $f\circ\alpha$ is a nondecreasing function.
Apply \ref{ex:d(distfun):<} and the definition of gradient to show that
\[
-\dd_{\alpha(t)}\distfun_{\alpha(t_3)}(\nabla_{\alpha(t)}f)
\ge
\langle \nabla_{\alpha(t)},\dir{\alpha(t)}{\alpha(t_3)}\rangle
\ge
\dd_{\alpha(t)}(\dir{\alpha(t)}{\alpha(t_3)})
\ge0
\]
for any $t<t_3$.
Conclude that the function 
$t\mapsto \distfun_{\alpha(t_3)}\circ\alpha(t)$ is noncreasing for $t\le t_3$.

\parbf{\ref{ex:mayer}.}
For any $s>s_0$,
\begin{align*}
(f\circ\hat\alpha)^+(s_0)&=|\nabla_{\hat\alpha(s_0)}f|
\ge
\\
&\ge
(d_{\hat\alpha(s_0)}f)(\dir{\hat\alpha(s_0)}{\hat\alpha(s)})
\ge
\\
&\ge
\frac{f\circ\hat\alpha(s)-f\circ\hat\alpha(s_0)}{\dist{\hat\alpha(s)}{\hat\alpha(s_0)}{}}.
\end{align*} 
Since $s-s_0\ge\dist{\hat\alpha(s)}{\hat\alpha(s_0)}{}$, for any $s>s_0$ we have 
\[(f\circ\hat\alpha)^+(s_0)\ge
\frac{f\circ\hat\alpha(s)-f\circ\hat\alpha(s_0)}{s-s_0}.\]

\parbf{\ref{lem:fg-dist-est}.}
Fix $t$, and let $p=\alpha(t)$ and $q=\beta(t)$.
Apply \ref{eq:fist-var-inq+} to get
\begin{align*}
 \ell^+
&\le -\<\dir{p}{q},\nabla_{p}f\>
-\<\dir{q}{p},\nabla_{q}g\>
\le
\\
&\le -{\left({f(q)}-{f(p)}-\lambda\cdot\tfrac{\ell^2}2\right)}/{\ell}
-{\left({g(p)}-{g(q)}-\lambda\cdot\tfrac{\ell^2}2\right)}/{\ell}\le
\\
&\le \lambda\cdot\ell+\tfrac{2\cdot\eps}{\ell}.
\end{align*}
Integrating this inequality, we get the second statement.

\parbf{\ref{ex:busemann-CBB}.} Apply \ref{ex:distfun-semiconcave}.

\parbf{\ref{ex:bus+bus}.} By the triangle inequality, 
\[\dist{\gamma(-t)}{x}{}+\dist{\gamma(t)}{x}{}-2\cdot t\ge 0\]
for any $t\ge 0$.
Passing to the limit as $t\to\infty$, we get the result.

\parbf{\ref{ex:cone-CBB}.}
Suppose $\Cone\spc{X}$ is $\Alex0$.
Observe that two half-lines in $\Cone\spc{X}$ that start from the origin and go into directions $x$ and $y\in\spc{X}$ form a line if and only if $\dist{x}{y}{\spc{X}}\ge \pi$.
Apply the splitting theorem to show that for any $x\in \spc{X}$ there is at most one point $y$ such that $\dist{x}{y}{\spc{X}}\ge \pi$ and in this case we have equality.
Conclude that $\diam \spc{X}\z\le \pi$.

Now choose a quadruple of points $p,x_1,x_2,x_3\in \spc{X}$;
we will identify $\spc{X}$ with the unit sphere in $\Cone\spc{X}$.
Suppose $\dist{p}{x_i}{}<\tfrac\pi2$ for any $i$.
Consider the following points in the cone: $y_i=\tfrac1{\cos \dist{p}{x_i}{\spc{X}}}\cdot x_i$, and $q=p$.
Show that $\EE^2$-comparison for $q,y_1,y_2,y_3$ in $\Cone\spc{X}$ implies $\SSS^2$-comparsion for $p,x_1,x_2,x_3$ in $\spc{X}$.
Conclude that $\spc{X}$ is locally $\Alex1$. 
Apply the globalization theorem (\ref{thm:globalization+}).

Now assume $\spc{X}$ is $\Alex1$ and $\diam\spc{X}\le \pi$.
By \ref{ex:perim-k>0}, the perimeter of any triangle in $\spc{X}$ is at most $2\cdot\pi$.
We need to check $\EE^2$-comparison for a given quadruple of points $q,y_1,y_2,y_3$ in $\Cone\spc{X}$.
We can assume that none of these points is the origin; otherwise perturb them a bit.

Set $x_i=y_i/|y_i|$ for each $i$ and $p=q/|q|$; we can assume that $p,x_1,x_2,x_3$ are distinct in $\spc{X}$, which is the unit sphere in $\Cone\spc{X}$.

Assume the model triangles $\modtrig(px_1x_2)$, $\modtrig(px_2x_3)$, and $\modtrig(px_3x_1)$ are defined;
that is, perimeters triangles $[px_1x_2]$, $[px_2x_3]$, and $[px_3x_1]$ are strictly less than $2\cdot\pi$. 
Note that $\EE^3\iso\Cone\SSS^2$.
Use this together with the $\SSS^2$-comparison for $p,x_1,x_2,x_3$ in $\spc{X}$ to show that $\EE^2$-comparison holds for $q,y_1,y_2,y_3$ in $\Cone\spc{X}$.

Finally, if some of the model triangles are not defined, consider rescaling of $\spc{X}$ with a coefficient $\lambda$ slightly smaller than 1.
Apply the argument above to show that the comparison holds for the corresponding points in $\Cone(\lambda\cdot\spc{X})$ and pass to the limit as $\lambda\to 1$.

\parit{Comment.}
The last part of the proof is close to the argument in \ref{thm:CBB-closed}.

\parbf{\ref{ex:|antisum|}.}
Observe that
\begin{align*}
\langle u,u\rangle+\langle v,u\rangle+\langle w,u\rangle &\ge 0,
\\
\langle u,v\rangle+\langle v,v\rangle+\langle w,v\rangle &\ge 0,
\\
\langle u,w\rangle+\langle v,w\rangle+\langle w,w\rangle &= 0.
\end{align*}
Add the first two inequalities and subtract the last identity.

\parbf{\ref{prop:two-opp}.}
Apply \ref{prop:opposite} to show that 
$\langle v,v\rangle =\langle v,w\rangle=\langle w,w\rangle$,
and use it.

\parbf{\ref{ex:3<,>=0}.} Show and use that
\[\langle u,x\rangle +\langle v,x\rangle +\langle w,x\rangle \ge 0\]
and
\[\langle u,-x\rangle +\langle v,-x\rangle +\langle w,-x\rangle \ge 0.\]

\parbf{\ref{ex:-u}.} Part $\Rightarrow$ is evident.
To prove part $\Leftarrow$, observe that 
\[\langle u^*,u^*\rangle =-\langle u,u^*\rangle\le \langle u,u\rangle\]
and since $|u|=|u^*|$, we have equality.

\parbf{\ref{ex:grad-dist}.}
Apply \ref{ex:-u}.

\parbf{\ref{ex:tangent=Em}.}
By \ref{ex:diam-compact:proper}, $\spc{A}$ is \emph{separable}; that is, it contains a countable dense set of points.
Apply \ref{cor:euclid-subcone} to this set.

\parbf{\ref{ex:dim=1}.} Argue as in \ref{ex:RisCBB(1)}.

\parbf{\ref{ex:resporka}.} The only-if part is trivial.
Suppose the configuration $p$, $a_0,\z\dots, a_{m}\in \spc{A}$ meets the condition.
By \ref{ex:grad-dist} the directions $\dir q{a_0},\z\dots,\dir q{a_m}\in \Lin_q$ for G-delta dense set of points $q\in \spc{A}$.
If $q$ is sufficiently close to $p$, then $\angk q{a_i}{a_j}>\tfrac\pi2$,
and therefore, $\mangle\hinge q{a_i}{a_j}>\tfrac\pi2$ for $i\ne j$.
Conclude that $\dim\Lin_q\ge m$ in this case.

\parbf{\ref{ex:finite-tan}}; 
\ref{SHORT.ex:finite-tan:tan}. Apply \ref{ex:geodesic-cone}, \ref{prop:Tan-is-CBB(0)}, and \ref{thm:finite-space-of-directions}.

\parit{\ref{SHORT.ex:finite-space-of-directions-dim}.}
Apply \ref{ex:resporka} to show that $\LinDim\T_p=\LinDim\spc{A}$ (argue as in \ref{prop:Tan-is-CBB(0)}).

\parit{\ref{SHORT.ex:finite-tan:sigma}.}
By \ref{thm:finite-space-of-directions} for any two points $\xi,\zeta\in\Sigma_p$ such that $\dist{\xi}{\zeta}{\Sigma_p}<\pi$ there is a geodesic $[\xi\zeta]_{\Sigma_p}$.
Suppose $\dist{\xi}{\zeta}{\Sigma_p}\ge\pi$, then $\T_p$ contains a line thru the origin in the directions $\xi$ and $\zeta$.
By \ref{SHORT.ex:finite-tan:tan} we can apply the splitting theorem (\ref{thm:splitting}) to $\T_p$.
We get that $\Sigma_p$ is a spherical suspension with poles $\xi$ and $\zeta$.
Hence, $\dist\xi\zeta{}=\pi$ and there is a geodesic $[\xi\zeta]$.


\parbf{\ref{ex:proof-right-inverse}}; \ref{SHORT.ex:proof-right-inverse:grad}.
By \ref{ex:distfun-semiconcave}, each function $\distfun_{a_i}$ is semiconcave in a small neighborhood of $p$.
Therefore we can choose $\lambda$ and $r>0$ so that $f_{\bm{y}}$ is $\lambda$-concave in $\oBall(p,r)$; further we will assume that $r$ is sufficiently small.
Choose $\alpha>0$ such that $\angk{x}{a_i}{a_j}>\tfrac\pi2+\alpha$ for all $i\ne j$;
we may assume that $\alpha<\tfrac{1}{10}$;
in particular,
\[(\dd_x\distfun_{a_j}{}{})(\dir{x}{a_i})
\ge
-\cos\angk{x}{a_i}{a_j}
>
\tfrac\alpha2\eqlbl{inq-a_j}\]
for $j\ne i$.

By the definition of gradient and \ref{ex:d(distfun):<}, we have
\begin{align*}
-(\dd_x\distfun_{a_i}{}{})(\nabla_x f_{\bm{y}})
&\ge
\<\dir x{a_i},\nabla_x f_{\bm{y}}\>
\ge
\\
&\ge
(\dd_xf_{\bm{y}})(\dir x{a_i}).
\end{align*}
If $\dist{a_i}{x}{}>y_i$, then 
\[\dd_xf_{\bm{y}}=\sigma+\eps\cdot \dd_x\distfun_{a_0},\]
where $\sigma$ is a minimum of a subset of the following functions
$0$, and $\dd_x\distfun_{a_j}$ for $0\ne j\ne i$.
By \ref{inq-a_j}, 
\[(\dd_x\distfun_{a_i}{}{})(\nabla_x f_{\bm{y}})< -\tfrac\alpha2\cdot\eps.\]
Hence (\ref{111}) holds for all sufficiently small $\eps>0$.

Now assume that $\dist{a_i}{x}{}-y_i=\min_j\{\dist{a_j}{x}{}\z-y_j\}<0$.
Then
\begin{align*}
\dd_x f_{\bm{y}}
&=
\min_{i\in S} \{\,\dd_x\distfun_{a_j}\,\}+\eps\cdot \dd_x\distfun_{a_0}
\le
\\
&\le
\dd_x \distfun_{a_i}{}{}+\eps\cdot(\dd_p\distfun_{a_0}{}{}),
\end{align*}
where $j\in S$ if and only if $\dist{a_i}{x}{}-y_i=\dist{a_j}{x}{}-y_j$.
Applying \ref{inq-a_j}, we get
\begin{align*}
(\dd_x \distfun_{a_i}{}{})(\nabla_x f_{\bm{y}})
&\ge 
\dd_xf_{\bm{y}}(\nabla_x f_{\bm{y}}) -\eps\cdot(\dd_x \distfun_{a_0}{}{})(\nabla_x f_{\bm{y}}) 
\ge 
\\
&\ge
\left[(\dd_xf_{\bm{y}})(\dir x{a_0})\right]^2-2\cdot \eps
\ge
\\
&\ge
\left[\tfrac\alpha2-\eps\right]^2-2\cdot \eps.
\end{align*}
Thus, (\ref{222}) holds for all sufficiently small $\eps>0$. 

\parit{\ref{SHORT.ex:proof-right-inverse:alpha}}
Consider the following real-to-real functions:
\[\begin{aligned}
\phi(t)
&\df
\max_{i}\{\dist{a_i}{\alpha_{\bm{y}}(t)}{}-y_i\},
\\
\psi(t)
&\df
\min_{i}\{\dist{a_i}{\alpha_{\bm{y}}(t)}{}-y_i\}.
\end{aligned}\eqlbl{eq:xy-def}\]
Use \ref{SHORT.ex:proof-right-inverse:grad}, to show that for $t\in[0,t_0]$, we have $\phi^+(t)<-\tfrac{1}{10}\cdot\eps^2$ if $\phi(t)>0$
and $\psi^+(t)>\tfrac{1}{10}\cdot\eps^2$ if $\psi(t)<0$.
Conclude that $\phi(t_0)=\psi(t_0)=0$; hence the result.


\parit{\ref{SHORT.ex:proof-right-inverse:end}}
A straightforward application of \ref{lem:fg-dist-est} and a reformulation of \ref{SHORT.ex:proof-right-inverse:alpha}.

\parbf{\ref{ex:proof-dist-chart}.}
Apply the (\textit{n}+1)-comparison (\ref{thm:n+1}) to show that at least one of the inequalities
\[
\mangle\hinge xy{a_0}<\tfrac\pi2-\eps,\ \dots,\  \mangle\hinge xy{a_m}<\tfrac\pi2-\eps,
\]
holds.
Similarty, we get that at least one of the inequalities
\[
\mangle\hinge yx{a_0}<\tfrac\pi2-\eps,\ \dots,\  \mangle\hinge yx{a_m}<\tfrac\pi2-\eps,
\]
holds.

Suppose our statement does not hold for $x$ and $y$ in a sufficiently small neighborhood of $p$.
It follows that 
\[\mangle\hinge yx{a_0}<\tfrac\pi2-\eps
\quad\text{and}\quad
\mangle\hinge yx{a_0}<\tfrac\pi2-\eps.
\eqlbl{eq:a0}
\]
Note that $\dist{x}{y}{}$ is small compared to $\dist{a_0}{x}{}$ and $\dist{a_0}{y}{}$.
Therefore, the comparison contradicts \ref{eq:a0}. 

By the construction, $f$ is Lipschitz.
From above, we can choose $i>0$ so that $\mangle\hinge xy{a_i}<\tfrac\pi2-\eps$ (if $\mangle\hinge yx{a_i}<\tfrac\pi2-\eps$, then swap $x$ and $y$).
By comparison, there is $c>0$ such that $\dist{a_i}{y}{}\le \dist{a_i}{x}{}+c\cdot \dist{x}{y}{}$.
Hence $f$ is bi-Lipschitz, and now \ref{thm:right-inverse} implies \ref{thm:dist-chart}.


 
\parbf{\ref{ex:diam-compact:proper}.}
Reuse the argument from  the first part of the proof of Bishop--Gromov inequality.

\parbf{\ref{ex:BG}.} 
You should follow the proof Bishop--Gromov inequality, plus prove the following two inequalities 
\begin{align*}
\sinh r_2\cdot \dist{\log_p x}{\log_p y}{\T_p} &\ge\dist{x}{y}{\spc{A}}
\\
\sinh r_2\cdot\dist{w(x)}{w(y)}{\spc{A}} &\ge \sinh r_1\cdot\dist{x}{y}{\spc{A}}
\end{align*}
for any $x,y\in\oBall(p,r)$.

\parbf{\ref{ex:dim=dim}.} 
Suppose $K$ is a compact set in $\spc{A}$ such that $\HausDim K\ge m$.
Use the map $w$ from the proof of the Bishop--Gromov inequality (\ref{inq:BG} and \ref{ex:BG}) to show that any open ball in $\spc{A}$ contains a compact set $K'$ such that $\HausDim K'\ge m$.

Use this in addition to the arguments in \ref{thm:dim=dim}. 

\parbf{\ref{ex:dim-lim}.}
Apply \ref{ex:resporka}.

\parbf{\ref{ex:net}};
\ref{SHORT.ex:net:finite}.
Suppose $X$ is compact.
Then for any $\eps>0$ any cover of $X$ by open $\eps$-balls have a finite subcover.
Note that the centers of these balls is an $\eps$-net of $X$.

Suppose $X$ has a finite $\eps$-net.
Show that any sequence $x_n$ of points in $X$ has a subsequence such that all of its points lie in one $\eps$-ball.
Apply this statement for $\eps=\tfrac1n$ together with the diagonal procedure.

\parit{\ref{SHORT.ex:net:compact}.}
Let $Z$ be a compact $\eps$-net of $X$.
By \ref{SHORT.ex:net:finite}, $Z$ admits a finite $\eps$-net $F$.
Note that $F$ is a $2\cdot\eps$-net of $X$.
Since $\eps>0$ is arbitrary, we get the result.


\parbf{\ref{ex:pack-net}.} If $x_1,\dots,x_n$ is not an $\eps$-net, then there is a point $y$ such that $\dist{x_i}{y}{}\ge\eps$ for any $i$.
Therefore $x_1,\dots,x_n$ is not a maximal packing --- a contradiction.

\parbf{\ref{ex:pack-vol}}; \ref{SHORT.ex:pack-vol:pack}
Apply the Bishop--Gromov inequality (\ref{inq:BG}).

\parit{\ref{SHORT.ex:pack-vol:dim}}
By \ref{ex:dim-lim}, $\dim\spc{A}_\infty\le m$.
To show that $\dim\spc{A}_\infty\ge m$,
apply \ref{cor:euclid-subcone} to a maximal packing and use the estimate in \ref{SHORT.ex:pack-vol:pack}.

\parit{Comment.}
A stronger statement holds 
\[\vol_m\spc{A}_\infty=\lim_{n\to\infty} \vol_m\spc{A}_n;\]
in other words, if $\bm{K}\subset \GH$ denotes the set of isometry classes of all compact $\Alex\kappa$ spaces with dimension $\le m$, then the function
$\vol_m\:\bm{K}\to \RR$ is continuous.


\parbf{\ref{ex:diam-compact:GH}.}
Argue as in \ref{thm:gromov-compactness} to construct a Gromov--Hausdorff convergence of $\cBall(p_n,R)_{\spc{A}_n}$ for given $R>0$, then apply the diagonal procedure to construct the needed convergence.

\parbf{\ref{ex:no-conc}.}
Consider the infinite product $\SSS^1\times ({\tfrac 12}\cdot \SSS^1)\times ({\tfrac 14}\cdot \SSS^1)\times\dots$

\parbf{\ref{ex:conic}.}
Let $V$ and $W$ be two conic neighborhoods of a point~$p$.
Without loss of generality, we may assume that $V\Subset W$;
that is, the closure of $V$ lies in $W$.

Construct a sequence of embeddings $f_n\:V\to W$
such that 
\begin{itemize}
\item 
For any compact set $K\subset V$ 
there is a positive integer $n=n_K$ such that 
$f_n(k)=f_m(k)$ for any $k\in K$ and $m, n \ge n_K$.
\item For any point $w\in W$ there is a point $v\in V$ such that $f_n(v)=w$ for all large $n$.
\end{itemize}

Note that once such a sequence is constructed, $f\:V\to W$ defined by $f(v)=f_n(v)$ for all large values of $n$ gives the needed homeomorphism.

The sequence $f_n$ can be constructed recursively
\[f_{n+1}=\Psi_n\circ f_n\circ \Phi_n,\]
where $\Phi_n\:V\to V$ 
and $\Psi_n\:W\to W$ 
are homeomorphisms
of the form 
\[\Phi_n(x)=\phi_n(x)\ast x\quad \text{and}\quad \Phi_n(x)=\psi_n(x)\star x,\]
where $\phi_n\:V\to \RR_{\ge 0}$, $\psi_n\:W\to \RR_{\ge 0}$ are suitable continuous functions;
``$\ast$'' and ``$\star$'' denote the multiplications in the cone structures of $V$ and $W$ respectively.

\parit{Comment.} If it is hard to follow, read the original proof by Kyung Whan Kwun \cite{kwun1964}.

\parbf{\ref{ex:conic-tangent}}; \ref{SHORT.ex:conic-tangen:tangent}. Apply \ref{thm:spherical-nbhd} and \ref{lem:kwun}.

\parit{\ref{SHORT.ex:conic-tangen:dir}.} Apply \ref{SHORT.ex:conic-tangen:tangent}.

\parit{\ref{SHORT.ex:conic-tangen:example}.} Recall that the Poincaré homology sphere can be obtained as a quotient space $\Sigma=\SSS^3/\Gamma$ by an isometric action of a finite group $\Gamma$  --- the so-called binary icosahedral group.
By the double suspension theorem,  $\Susp^2\Sigma\cong\SSS^5$.
Note that $\Susp^2\Sigma$ is an Alexandrov space and it has a point with space of directions isometric to $\Susp\Sigma$.
Observe that $\Susp\Sigma$ is not a manifold; in particular $\Susp\Sigma\ncong\SSS^4$.
Therefore the pair $\Susp^2\Sigma$ and $\SSS^5$ provides the needed example.

\parbf{\ref{ex:bry2bry}.} Apply \ref{thm:spherical-nbhd}, \ref{lem:kwun}, and \ref{thm:top-bry}.

\parbf{\ref{ex:bry-closed}.}
Let $\spc{A}$ be a finite-dimensional Alexandrov space.
Choose $x\in\spc{A}$.
By \ref{thm:spherical-nbhd}, a neighborhood $U\ni x$ is homeomorphic to $\T_x$.
Therefore \ref{ex:bry2bry}, implies that $U\cap\partial\spc{A}=\emptyset$ if and only if $x\notin \partial\spc{A}$;
that is, the complement $\spc{A}\setminus\partial\spc{A}$ is open, and therefore, $\spc{A}$ is closed.

\parbf{\ref{ex:pz<ypz}.}
Consider the model triangle $[\tilde x\tilde y\tilde z']=\modtrig(xyz)$.
\begin{figure}[ht!]
\vskip-0mm
\centering
\includegraphics{mppics/pic-1015}
\end{figure}

Show that 
\[\dist{\tilde p}{\tilde z}{}\le \dist{\tilde p}{\tilde z'}{}\le\side\hinge yp{z}.\]


\parbf{\ref{ex:bry-connected}.}
Assume $\spc{A}$ has at least two boundary components, say $A$ and $B$.
Denote by $\gamma$ a geodesic that minimizes the distance from $A$ to $B$.

Let 
\[\dots,\spc{A}_{-1},\spc{A}_{0},\spc{A}_{1},\dots\]
be a two-sided infinite sequence of copies on $\partial\spc{A}$.
Let us glue $\spc{A}_{i}$ to $\spc{A}_{i+1}$ along $A$ if $i$ is even and along $B$ if $i$ is odd.

By the doubling theorem, every point in the obtained space $\spc{N}$ has a neighborhood that is isometric to a neighborhood of the corresponding point in $\spc{A}$ or its doubling.
By the globalization theorem, $\spc{N}$ is $\Alex1$.

Note that the copies of $\gamma$ in $\spc{A}_{i}$ form a line in $\spc{N}$.
By the splitting theorem, $\spc{N}$ is isometric to a product $\spc{N}'\oplus \RR$.
Since $\dim\spc{N}>1$, Exercise~\ref{ex:dim=1} implies that $\diam\spc{N}\le \pi$ --- a contradiction.

\parbf{\ref{ex:dist-to-bry}.} Choose $x$ on $\gamma$;
we can assume that $x=\gamma(0)$.
Let $y\in \partial \spc{A}$ be a closest point to $x$.
Let $\alpha=\mangle(\dir xy,\gamma^+(0)$.

Suppose $x\notin \partial \spc{A}$.
Show that $\T_y=\RR_{\ge0}\times\T_y\partial \spc{A}$
and $\dir yx\perp \T_y\partial \spc{A}$.

Given a vector $v\in \T_y$, denote by $\bar v$ its projection to $\T_y\partial \spc{A}$.
Apply the comparison and \ref{prop:gexp} to show that 
\[\dist{\gamma(t)}{\gexp_y(\overline{\log_x\gamma(t)})}{}\le \dist{x}{y}{}+t\cdot\cos\alpha.\]
Conclude that $\gamma''(0)\le 0$ in the barrier sense.


\parbf{\ref{ex:liberman}.}
Suppose $\gamma$ is defined on the interval $[0,\ell]$.
Assume that the function $\rho\:t\mapsto \tfrac12\cdot\distfun_p^2\circ\gamma(t)$ is not $1$-concave.
Let $\bar\rho\:[0,\ell]\to\RR$ be the minimal $1$-concave function such that $\bar\rho\ge \rho$.
Note that $\bar\rho=\rho$ at the ends of $[0,\ell]$.

Consider the curve $\bar\gamma(t)\df \GF_f^{s(t)}\gamma(t)$;
where $f=\tfrac12\cdot\distfun_p^2$ and $s(t)\z=\ln\circ\bar\rho(t)-\ln\circ\rho(t)$.
Use the first distance estimate to show that $\length\bar\gamma<\length\gamma$ and arrive at a contradiction.

\parit{Comment.}
The statement was proved by Grigory Perelman and the second author \cite{perelman-petrunin};
it generalizes a theorem of Joseph Liberman \cite{liberman} about geodesics on convex surfaces.
The original Liberman's version of the following geometric statement.
\textit{Suppose that $C$ is the cone over $\gamma$ with the vertex at $p$,
where $\gamma$ is a geodesic on a convex surface and $p$ is a point in the convex body bounded by the surface.
Then after unfolding $C$ into plane, $\gamma$ becomes a locally convex curve.}
It is instructive to check that this formulation is equivalent to ours for convex bodies.

\parbf{\ref{ex:native}.}
Choose a geodesic $\gamma$ in $\spc{W}$.
Arguing as in the proof of \ref{thm:doubling:doubling}, we get 
that $\gamma$ can cross the common boundary of two halves $\spc{A}_0$ and $\spc{A}_1$ of $\spc{W}$ at most once, or it lies in the common boundary.

In the later case $\lambda$-concavity of $f\circ\proj\circ\gamma$ follows from $\lambda$-concavity of $f$.
In the former case the convexity has to be checked only at the point of crossing;
we may assume that it happens at $x=\gamma(0)$.
Since $\nabla_xf\in\partial\T_x$ for any $x\in\partial\spc{A}$ the $f$-gradient flows agree on $\spc{A}_0$ and $\spc{A}_1$.

Assume $f\circ\proj\circ\gamma$ is not $\lambda$-concavity at $0$.
Apply $f$-gradinent flow to shorten $\gamma$ keeping its ends as in the proof of \ref{ex:liberman},
and arrive at a contradiction.

\parbf{\ref{ex:Hilbert/G}.} Read \cite[Section 4]{terng-thorbergsson} and/or the solution for ``Quotient of the Hilbert space'' in \cite{petrunin2020}.

\parbf{\ref{ex:sumbetries(S^2)}}; \ref{SHORT.ex:sumbetries(S^2):1}.
Choose an isometric $\SSS^1$-action on $\SSS^2$ that fixes the poles of the sphere.
Consider the projection to the quotient space $\sigma_1\:\SSS^2\z\to \SSS^2/\SSS^1=[0,\pi]$.

\parit{\ref{SHORT.ex:sumbetries(S^2):2}.}
Take a half-circle $\gamma$ on $\SSS^2$ and define 
$\sigma_2(x)\df\distfun_\gamma(x)_{\SSS^2}$.

\parit{\ref{SHORT.ex:sumbetries(S^2):n}.}
Consider the subdivision of $\SSS^2$ into $\SSS^1$-orbits of the action from~\ref{SHORT.ex:sumbetries(S^2):1}.
Cut $\SSS^2$ into two hemispheres by meridians rotate one hemisphere by an angle $\alpha=\pi/n$ and glue it back.
Observe that there is a submetry $\sigma_n$ such that the inverse image $\sigma_n^{-1}\{y\}$ is a union of the arcs from the original $\SSS^1$-orbits.

Note that for $n=2$ we get the solution in \ref{SHORT.ex:sumbetries(S^2):2}.

\parbf{\ref{ex:sumbetries(E^2)}.}
Show that for any $x\in\EE^2$ there is a half-line $H\ni x$ such that 
the restriction $\sigma|_H$ is an isometry.
Suppose such a half-line $H$ starts at $p$ and passes thru $q$.
Show that $\langle x-p,q-p \rangle\le 0$ for any $x\in \sigma^{-1}\{0\}$.
Conclude that $\sigma^{-1}\{0\}$ is a convex closed set.
Finally use the definition of submetry to show that  $\sigma^{-1}\{0\}$ has no interior points. 

\parbf{\ref{ex:S^3/S^1}};
\ref{SHORT.ex:S^3/S^1:pq}.
Our $\SSS^1$ is a commutative subgroup of $\SO(3)$.
Therefore it is a subgroup of a maximal torus in $\SO(3)$.
Show that the described torus action is induced by a maximal torus in $\SO(3)$.
Use that maximal tori in $\SO(3)$ are conjugate.

\parit{\ref{SHORT.ex:S^3/S^1:sphere}.}
Cut $\SSS^3$ into two solid tori the Clifford torus $\tfrac1{\sqrt2}\cdot \SSS^1\times \SSS^1$.
Observe that the quotient of each solid torus is a disc;
conclude that $\Sigma_{p,q}$ is a sphere.
The torus action on $\SSS^3$ induce the needed $\SSS^1$-cation on $\Sigma_{p,q}$.

\parit{\ref{SHORT.ex:S^3/S^1:a}+\ref{SHORT.ex:S^3/S^1:b}+\ref{SHORT.ex:S^3/S^1:c}.} Straightforward calculations.

\parit{\ref{SHORT.ex:S^3/S^1:cc}.}
Consider the map $\Sigma_{p,q}\to\Sigma_{1,1}$ that sends poles to poles,
preserve the distance to the poles and respects the $\SSS^1$ action.

\parbf{\ref{ex:number(m-1)}};
\ref{SHORT.ex:number(m-1):2}.
Suppose $\#_{m-1}(\Gamma)\ge 3$;
that is $\spc{A}=\EE^m/\Gamma$ has at least 3 boundary components.
Follow Case~3 in the proof \ref{thm:hsiang-kleiner} to glue a train-space from copies of $\spc{A}$ using two of these components.
Show that the obtained space splits and arrive at a contradiction.

(Alternatively, apply a similar construction to all components of the boundary.
Show that the obtained space has {}\emph{exponential volume growth};
that is, there is $a>1$ such that $\vol \oBall(p,r)>a^r$ for all large~$r$.
Arrive at a contradiction with the Bishop--Gromov inequality.)

\parit{\ref{SHORT.ex:number(m-1):1}.}
Apply the doubling theorem as in Case~2 in the proof \ref{thm:hsiang-kleiner}.

\parbf{\ref{ex:S1actsS3}.}
Show that the quotient space $\Delta=\spc{A}/\mathbb{S}^1$ is an $\Alex1$ disc and $\gamma$ projects isometrically to $\partial\Delta$.
It remains to show that the perimeter of $\Delta$ cannot exceed $2\cdot\pi$.
The latter follows from \cite[3.3.5]{petrunin:survey};
it states that if $\Delta$ as an $m$-dimensional $\Alex1$ space, then $\vol_{m-1}\partial \Delta\le \vol_{m-1}\partial \mathbb{S}^{m-1}$.

\parbf{\ref{ex:surf-S2}.}
We can assume that the origin lies in the interior of the convex body.
Consider the central projection from its surface, say $\Sigma$, to the sphere $\SSS^2$ centered at the origin.
Show that this projection $\Sigma\to \SSS^2$ is a homeomorphism.

\parbf{\ref{ex:vertex-essential-vertex}.}
Follow the argument in \ref{clm:total-angle}.
Show that the inequality is strict if and only if $F$ has opposite points.


\parbf{\ref{ex:geodesic-vertex}.}
Suppose a geodesic $\gamma$ passes thru a vertex $v$.
Denote by $\alpha$ and $\beta$ the angles that $\gamma$ cuts at $v$.
Since $v$ is essential, $\alpha+\beta<2\cdot\pi$.
Therefore $\alpha<\pi$ or $\beta<\pi$.
Arrive at a contradiction by showing that $\gamma$ is not length-minimizing.

\parbf{\ref{pr:tetrahedron}}; \ref{SHORT.pr:tetrahedron:=}.
Cut the surface of $T$ along three edges coming from one vertex $v_1$ and unfold the obtained surface onto the plane.
Show that this way we get a triangle, the three vertices correspond to $v_1$ and the midpoints of sides correspond to the remaining three vertices.
Make a conclusion.

\parit{\ref{SHORT.pr:tetrahedron:perp}}.
Suppose that $0,v_1,v_2,v_3\in\RR^3$ are the vertices of $T$.
From \ref{SHORT.pr:tetrahedron:=}, we have that 
\[|v_1|=|v_2-v_3|,\quad |v_2|=|v_3-v_1|,\quad|v_3|=|v_1-v_1|.\]
Use it to show that $\langle v_1,v_2+v_3-v_1\rangle=0$.
Make a conclusion.

\parbf{\ref{ex:poly-CBB}.}
We need to show that if a polyhedral surface is $\Alex0$, then the total angle $\theta$ at every vertex $p$ it at most $2\cdot\pi$.

Assume that $\theta>2\cdot\pi$,
let $\phi=\max\{\,\pi,\tfrac13\cdot\theta\,\}$.
Note that we can choose three points $x_1$, $x_2$, and $x_3$ close to $p$ such that 
$\mangle \hinge p{x_i}{x_j}=\phi$ for $i\ne j$.
Since the points $x_i$ are close to $p$, we have $\mangle \hinge p{x_i}{x_j}=\angk p{x_i}{x_j}$.
The latter contradicts $\EE^2$-comparison. 

\parbf{\ref{ex:surface-covergence}.}
We will use that the closest-point projection from the Euclidean space to a convex body is \index{short map}\emph{short};
that is, distance-nonexpanding \cite[13.3]{petrunin-zamora}.

Assume $K_\infty$ is nondegenerate.
Without loss of generality, we may assume that 
\[\cBall(0,r)\subset K_\infty\subset\cBall(0,1)\]
for some $r>0$.
Note that there is a sequence $\eps_n\to 0$ such that 
\[ K_n\subset(1+\eps_n)\cdot K_\infty
\quad\text{and}\quad
K_\infty\subset(1+\eps_n)\cdot K_n\]
for each large $n$.

Given $x\in K_n$, denote by $g_n(x)$ the closest-point projection of $(1+\eps_n)\cdot x$ to $K_\infty$.
Similarly, given $x\in K_\infty$, denote by $h_n(x)$ the closest point projection of $(1+\eps_n)\cdot x$ to $K_n$.
Note that 
\begin{align*}
\dist{g_n(x)}{g_n(y)}{}&\le (1+\eps_n)\cdot\dist{x}{y}{}
\intertext{and}
\dist{h_n(x)}{h_n(y)}{}&\le (1+\eps_n)\cdot\dist{x}{y}{}.
\end{align*}

Denote by $\Sigma_\infty$ and $\Sigma_n$ the surface of $K_\infty$ and $K_n$ respectively. 
The above inequalities imply 
\begin{align*}
\dist{g_n(x)}{g_n(y)}{\Sigma_\infty}&\le (1+\eps_n)\cdot\dist{x}{y}{\Sigma_n}
\intertext{for any $x,y\in \Sigma_n$, and}
\dist{h_n(x)}{h_n(y)}{\Sigma_n}&\le (1+\eps_n)\cdot\dist{x}{y}{\Sigma_\infty}.
\end{align*}
for any $x,y\in \Sigma_\infty$.

Note that the maps $g_n$ and $h_n$ are onto.
Apply \ref{ex:GH-po} to finish the proof.

Alternatively, since the closest-point projection cannot increase the length of curve, we also get
\begin{align*}
\dist{x}{h_n\circ g_n(x)}{\Sigma_\infty}&\le 10\cdot \eps_n
\\
\dist{y}{g_n\circ h_n(y)}{\Sigma_n}&\le 10\cdot \eps_n.
\end{align*}
for all large $n$.
Therefore, $g_n$ is a $\delta_n$-isometry $\Sigma_n\to\Sigma_\infty$ for a sequence $\delta_n\to 0$.

\parit{Comments.}
More generally, if a sequence of $m$-dimensional $\Alex\kappa$ spaces $\spc{A}_1,\spc{A}_2,\dots$ converges to $\spc{A}_\infty$ and $\dim \spc{A}_\infty=m<\infty$,
then $\partial \spc{A}_n$ equipped with the induced length metrics converge to  $\partial \spc{A}_\infty$.
This statement is a partial case of the theorem about extremal subsets proved by the second author \cite[1.2]{petrunin1997}.

\parbf{\ref{ex:liberman+milka}}; \ref{SHORT.ex:liberman+milka:liberman}.
By \ref{ex:liberman}, the function $f_p\:t\mapsto \distfun_p\circ\gamma(t)$ is semiconcave for any $p\in K$.
In particular, one-sided derivatives $f_p^+(t)$ are defined for every $t$.

Given $x=\gamma(t)$, choose three points $p_1,p_2,p_3\in K$ in general position;
that is, the four points $x,p_1,p_2,p_3$ do not lie in one plane.
Observe that the distance functions $\distfun_{p_i}$ give smooth coordinates in a neighborhood of $x$.
From above the functions $f_{p_i}$ have one-sided derivatives at $t$.
Since the coordinates are smooth we get that $\gamma^+(t)$ is defined as well.

\parit{\ref{SHORT.ex:liberman+milka:milka}.}
If the plane $py_1y_2$ supports $K$, then 
$\mangle\hinge p{y_1}{y_2}_{\EE^3}=\mangle\hinge p{x_1}{x_2}_S$.
In this case, the statement follows from \ref{prop:conv-surf-CBB(0)}.

Now suppose that the line segment $[y_1y_2]_{\EE^3}$ intersects $K$.
Choose a geodesic $[y_1y_2]_W$;
note that it contains a point of $K$, say $z$.
Now consider a one-parameter family of points 
$y_i(t)\df \gamma(t)+\gamma^+(t)\z\cdot (1-t)\z\cdot \dist{p}{x_i}{S}$.
Note that this family is not continuous.

Show that for any point $p\in K$, the function $t\mapsto \dist{p}{\gamma_i(t)}{\EE^3}$ is nonincreasing.
Conclude that the function $t\mapsto \dist{p}{\gamma_i(t)}{W}$ is nonincreasing for any $p\in S$.
Therefore, 
\begin{align*}
\dist{y_1}{y_2}{W}
&=\dist{y_1(0)}{y_2(0)}{W}=
\\
&=\dist{y_1(0)}{z}{W}+\dist{y_2(0)}{z}{W}\ge
\\
&\ge\dist{y_1(1)}{z}{W}+\dist{y_2(1)}{z}{W}\ge 
\\
&\ge\dist{x_1}{x_2}{S}.
\end{align*}
The last inequality follows since the closest point projection $W\to S$ is short.

It remains to consider the case when the plane $py_1y_2$ does not support $K$,
and $[y_1y_2]_{\EE^3}$ does not intersect $K$.
In this case the plane $py_1y_2$ intersects $K$ along a convex figure $F$ that lies in the solid triangle 
$py_1y_2$ and contains its vertex $p$.

Choose points $y_1'\in [py_1]_{\EE^3}$ and $y_2'\in [py_2]_{\EE^3}$ such that $[y_1'y_2']$ touches $F$.
Denote by $x_1'\in [px_1]_{S}$ and $x_2'\in [px_2]_{S}$ the corresponding points;
that is, $\dist{p}{y_1'}{\EE^3}=\dist{p}{x_1'}S$ and $\dist{p}{y_2'}{\EE^3}=\dist{p}{x_2'}S$.
From the above, we have that $\dist{y_1'}{y_2'}{\EE^3}\ge\dist{x_1'}{x_2'}S$;
in other words, 
\[\angk p{y_1'}{y_2'}\ge \angk p{x_1'}{x_2'};\]
here we think that $[p{y_1'}{y_2'}]$ is a triangle in $\EE^3$, but $[p{x_1'}{x_2'}]$ is a triangle in $S$.
Note that 
\[\angk p{y_1'}{y_2'}=\angk p{y_1}{y_2}
\quad\text{and}\quad
\angk p{x_1}{x_2}\le \angk p{x_1'}{x_2'};
\]
the second inequality follows from \ref{ex:noncreasing}.
Hence the remaining case follows.

\parit{Comments.}
Part~\ref{SHORT.ex:liberman+milka:liberman} is the so-called Liberman lemma --- the main tools in studying geodesics on convex surfaces.
It was originally proved by Joseph Liberman \cite{liberman}; the proof of \ref{ex:liberman} is its generalization. 

Part~\ref{SHORT.ex:liberman+milka:milka} is the result of Anatolii Milka \cite[Theorem 2]{milka1982}.

%%%%%%%%%%%%%%%%

{
\documentclass[twoside]{book}

%\newcommand{\spell}[2]{#1} %spell
\newcommand{\spell}[2]{#2} %notes


\def\thetitle{A journey into Alexandrov geometry:\\
curvature bounded below}
\def\theauthors{Vitali Kapovitch and Anton Petrunin}

\usepackage{lectures}
\usepackage[colorlinks=true,
citecolor=black,
linkcolor=black,
anchorcolor=black,
filecolor=black,
menucolor=black,
urlcolor=black,
pdftitle={\thetitle},
pdfsubject={Geometry},
pdfauthor={\theauthors}
]{hyperref}
\makeindex

%\usepackage[x-1a]{pdfx}

%\overfullrule=100mm
\def\red{\color{red}}
\begin{document}

\spell{\pagestyle{empty}\renewcommand\includegraphics[2][{}]{}\def\emph{\textit}\renewcommand\footnote[1]{\ (#1)}\renewcommand\z{}\renewcommand\section[1]{SECTION. {#1} SECTION.}}{}

\frontmatter
\title{\thetitle}
\author{\theauthors}
\date{}
\maketitle
\thispagestyle{empty}

\mainmatter
\newpage
\tableofcontents

\chapter*{Preface}

As in our previous invitation \cite{alexander-kapovitch-petrunin-2019},
we try to demonstrate the beauty and power of Alexandrov geometry by reaching interesting applications and theorems with a minimum of preparation.
This time we do spaces with curvature bounded below in the sense of Alexandrov.

This subject is more technical, this time we jumped over proofs of couple of technical results,
namely existence part in generalized Picard's theorem (\ref{thm:glob-exist-grad-curv})
and Perelman's theorem about conic neighborhoods (\ref{thm:spherical-nbhd}).
The rest of our presentation is nearly rigorous.

\medskip 

In Lecture~\ref{chap:prelim}, we discuss necessary preliminaries and fix notations.

Lecture~\ref{chap:defs} introduces the main object of our study --- spaces with curvature bounded below in the sense of Alexandrov.

In Lecture~\ref{chap:globalization} we formulate and prove the globalization theorem --- local Alexandrov condition implies global.
To simplify the presentation we consider only compact case, but this case is leading.

In Lecture~\ref{chap:derivative} we do beginning of calculus --- tangent space and space of directions, differential, and gradient.

Lecture~\ref{chap:GF} introduces gradient flow --- this is the main technical tool in the theory.

Lecture~\ref{chap:splitting} proves the line splitting theorem.
It provides the first application of gradient flow.

In Lecture~\ref{chap:dim} we introduce and discuss dimension of Alexandrov spaces,
introduce volume,
and prove the Bishop--Gromov inequality.

Lecture~\ref{chap:lim} shows that lower curvature bound survives in the Gromov--Hausdorff limit and proves Gromov's selection theorem.
Further we do Perelman's construction of strictly concave functions and apply it with Gromov's selection theorem to prove the homotopy finiteness theorem.
This proof illustrates the main source of applications of Alexandrov geometry.

In Lecture~\ref{chap:bry} we introduce boundary of finite-dimensioanal Alexandrov space and prove the doubling theorem.

Lecture~\ref{chap:L/G} we show that quotient Alexandrov space by isometric group action is an Alexandrov space and give several applications of this statement.
These proofs illustrate another source of applications of Alexandrov geometry.

Lecture~\ref{chap:convex-body} brings us back to the original object of study of Alexandrov.
We show that surface of a convex body in Euclidean space is an Alexandrov space.
This is historically the first serious application of Alexandrov geometry.

Finally, Appendix~\ref{chap:embedding} sketches Alexandrov embedding theorem of convex polyhedra.
Historically, this theorem is the first remarkable result in Alexandrov geometry that dates back to 1941.
The proof is very well written by Alexandrov, but we decided to include its sketch here due to its beauty and importance.
This appendix was written by Nina Lebedeva and the second author for a book about .

Let us give a list of available texts on Alexandrov spaces with curvature bounded below: 
\begin{itemize}
\item The first introduction to Alexandrov geometry is given in the original paper of Yuriy Burago, Michael Gromov, and Grigory Perelman \cite{burago-gromov-perelman} 
and its extension \cite{perelman1991} written by Perelman.
\item A brief and reader-friendly introduction was written by Katsuhiro Shiohama \cite[Sections 1--8]{shiohama}.
\item Another reader-friendly introduction, written by Dmiti Burago, Yuriy
Burago, and Sergei Ivanov \cite[Chapter 10]{burago-burago-ivanov}.
\item Survey by Conrad Plaut \cite{plaut:survey}.
\item Survey by the second author \cite{petrunin:survey}.
\end{itemize}

\parbf{Acknowledgments.}
Our notes were shaped in a number of lectures given by the authors
at different occasions in Penn State, including the MASS program,
at the Summer School ``Algebra and Geometry'' in Yaroslavl,
at SPbSU,
and University of Toronto.
We want to thank these institution for hospitality and support.

We were partially supported by the following grants:
Vitali Kapovitch ---   NSERC Discovery grants;
Anton Petrunin --- 
NSF grant DMS-2005279. %??? check!!!




%%!TEX root = the-prelim.tex

\chapter{Preliminaries}\label{chap:prelim}

\section{Prerequisites}

We assume that the reader is familiar with the following topics in metric geometry:
\begin{itemize}
\item Compactness and proper metric spaces;
recall that a metric space is \index{proper space}\emph{proper} if all its closed balls (with finite radius) are compact.
\item Complete metric spaces and completion.
\item Curves, semicontinuity of length and rectifiability.
\item Hausdorff and Gromov--Hausdorff convergence.
These are discussed briefly in \ref{sec:Hausdorff convergence}--\ref{sec:Gromov--Hausdorff-metric}.
The definitions are there, but it would be hard to follow without prior experience.
\end{itemize}
These topics are treated in \cite{burago-burago-ivanov} and \cite{petrunin2023pure}.
Occasionally, we use the Baire category theorem and Rademacher's theorem, but these could be used as black boxes.

We use some topology. 
Most of the time, any introductory text in algebraic topology should be sufficient.
For some examples, we use more advanced results, but these could also be used as black boxes.

Since most of the applications come from Riemannian geometry, it is better to be familiar with the Toponogov comparison theorem and related topics.
The classical book by Jeff Cheeger and David Ebin \cite{cheeger-ebin} contains more than you will need.

\section{Notations}

The distance between two points $x$ and $y$ in a metric space $\spc{X}$ will be denoted by \index{$\dist{x}{y}{}=\dist{x}{y}{\spc{X}}$ (distance)}$\dist{x}{y}{}$ or $\dist{x}{y}{\spc{X}}$.\label{page:|x-y|X}
The latter notation is used if we need to emphasize 
that the distance is taken in the space~${\spc{X}}$.

Given radius $r\in[0,\infty]$ and center $x\in \spc{X}$, the sets
\begin{align*}
\oBall(x,r)&=\set{y\in \spc{X}}{\dist{x}{y}{}<r},
\\
\cBall[x,r]&=\set{y\in \spc{X}}{\dist{x}{y}{}\le r}
\end{align*}
are called, respectively, the \index{open ball}\emph{open} and  the \index{closed ball}\emph{closed  balls}.
The notations $\oBall(x,r)_{\spc{X}}$ and $\cBall[x,r]_{\spc{X}}$
might be used if we need to emphasize that these balls are taken in the metric space $\spc{X}$.

We will denote by \index{$\SSS^n$, $\EE^n$, $\HH^n$, and $\MM^n(\kappa)$}$\SSS^n$, $\EE^n$, and $\HH^n$ the $n$-dimensional sphere (with angle metric), 
Euclidean space, and Lobachevsky space respectively.
More generally, $\MM^n(\kappa)$ will denote the \index{model!space}\emph{model $n$-space} of curvature $\kappa$;
that is,
\begin{itemize}
\item if $\kappa>0$, then $\MM^n(\kappa)$ is the $n$-sphere of radius $\tfrac{1}{\sqrt{\kappa}}$, so $\SSS^n=\MM^n(1)$
\item $\MM^n(0)=\EE^n$,
\item if $\kappa<0$, then $\MM^n(\kappa)$ is the Lobachevsky $n$-space $\HH^n$ rescaled by factor $\tfrac{1}{\sqrt{-\kappa}}$;
in particular $\MM^n(-1)=\HH^n$.
\end{itemize}

\section{Length spaces}\label{sec:length}

Let $\spc{X}$ be a metric space.
If for any $\eps>0$ and any pair of points $x,y\in\spc{X}$, there is a path $\alpha$ connecting $x$ to $y$ such that
\[\length\alpha< \dist{x}{y}{}+\eps,\]
then $\spc{X}$ is called a \index{length space}\emph{length space} and the metric on $\spc{X}$ is called a \index{length metric}\emph{length metric}.\label{page:length metric}

\begin{thm}{Exercise}\label{ex:compact+connceted}
Let $\spc{X}$ be a complete length space.
Show that for any compact subset $K\subset\spc{X}$
there is a compact path-connected subset $K'\subset\spc{X}$ that contains $K$.  
\end{thm}

\parbf{Induced length metric.}
Directly from the definition, it follows that if $\alpha\:[0,1]\to\spc{X}$ is a path from $x$ to $y$ 
(that is, $\alpha(0)=x$ and $\alpha(1)=y$), then 
\[\length\alpha\ge \dist{x}{y}{}.\]
Set 
\[\yetdist{x}{y}{}=\inf\{\,\length\alpha\,\}\]
where the greatest lower bound is taken for all paths from $x$ to~$y$.
It is straightforward to check that $(x,y)\mapsto \yetdist{x}{y}{}$ is an \emph{$\infty$-metric};
that is, $(x,y)\mapsto \yetdist{x}{y}{}$ is a metric in the extended positive reals $[0,\infty]$. 
The metric $\yetdist{*}{*}{}$ is called the \index{induced length metric}\emph{induced length metric}.

\begin{thm}{Exercise}\label{ex:compact=>complete}
Suppose $(\spc{X},\dist{*}{*}{})$ is a complete metric space.
Show that $(\spc{X},\yetdist{*}{*}{})$ is complete;
that is, any Cauchy sequence of points in $(\spc{X},\yetdist{*}{*}{})$ converges in $(\spc{X},\yetdist{*}{*}{})$.
\end{thm}

Let $A$ be a subset of a metric space $\spc{X}$.
Given two points $x,y\in A$,
consider the value
\[\dist{x}{y}{A}=\inf_{\alpha}\{\,\length\alpha\,\},\]
where the greatest lower bound is taken for all paths $\alpha$ from $x$ to $y$ in~$A$.
In other words, $\dist{*}{*}{A}$ denotes the induced length metric on the subspace $A$.
(The notation $\dist{*}{*}{A}$ conflicts with the previously defined notation for distance $\dist{x}{y}{\spc{X}}$ in a metric space $\spc{X}$.
However, most of the time we will work with ambient length spaces where the meaning will be unambiguous.)

\section{Geodesics}

Let $\spc{X}$ be a metric space 
and $\II$\index{$\II$ (real interval)} a real interval. 
A distance-preserving map $\gamma\:\II\to \spc{X}$ is called a \index{geodesic}\emph{geodesic}%
\footnote{Others call it differently: \textit{shortest path}, \textit{minimizing geodesic}.
Also, note that the meaning of the term \textit{geodesic} is different from what is used in Riemannian geometry, altho they are closely related.}; 
in other words, $\gamma\:\II\z\to \spc{X}$ is a geodesic if 
\[\dist{\gamma(s)}{\gamma(t)}{}=|s-t|\]
for any pair $s,t\in \II$.

If $\gamma\:[a,b]\to \spc{X}$ is a geodesic such that $p=\gamma(a)$, $q=\gamma(b)$, then we say that $\gamma$ is a geodesic from $p$ to $q$.
In this case, the image of $\gamma$ is denoted by $[p q]$\index{$[pq]$ (geodesic)}, and, with abuse of notations, we also call it a \index{geodesic}\emph{geodesic}.
We may write $[p q]_{\spc{X}}$ 
to emphasize that the geodesic $[p q]$ is in the space  ${\spc{X}}$.

In general, a geodesic from $p$ to $q$ need not exist and if it exists, it need not  be unique;
for example, any meridian is a geodesic between poles on the sphere.
However, once we write $[p q]$ we assume that we have chosen such a geodesic.

A \index{geodesic!path}\emph{geodesic path} is a geodesic with constant-speed parameterization by the unit interval $[0,1]$.

A metric space is called \index{geodesic!space}\emph{geodesic} if any pair of its points can be joined by a geodesic.

Evidently, any geodesic space is a length space.

\begin{thm}{Exercise}\label{ex:compact-length}
Show that any proper length space is geodesic.
\end{thm}

\section{Menger's lemma}

\begin{thm}{Lemma}\label{lem:mid>geod}
Let $\spc{X}$ be a complete metric space.
Assume that for any pair of points $x,y\in \spc{X}$, 
there is a midpoint~$z$.
Then $\spc{X}$ is a geodesic space.

\end{thm}

This lemma is due to Karl Menger \cite[Section 6]{menger}.

%???+PIC!!!

\parit{Proof.}
Choose $x,y\in \spc{X}$;
set $\gamma(0)=x$, and $\gamma(1)=y$.

\begin{figure}[ht!]
\vskip-0mm
\centering
\includegraphics{mppics/pic-104}
\end{figure}

Let $\gamma(\tfrac12)$ be a midpoint between $\gamma(0)$ and $\gamma(1)$.
Further, let $\gamma(\frac14)$ 
and $\gamma(\frac34)$ be midpoints between the pairs $(\gamma(0),\gamma(\tfrac12))$ 
and $(\gamma(\tfrac12),\gamma(1))$ respectively.
Applying the above procedure recursively,
on the $n$-th step we define $\gamma(\tfrac{k}{2^n})$,
for every odd integer $k$ such that $0<\tfrac k{2^n}<1$, 
as a midpoint of the already defined
$\gamma(\tfrac{k-1}{2^n})$ and $\gamma(\tfrac{k+1}{2^n})$.

This way we define $\gamma(t)$ for all dyadic rationals $t$ in $[0,1]$.
Moreover, $\gamma$ has Lipschitz constant $\dist{x}{y}{}$.
Since $\spc{X}$ is complete, the map $\gamma$ can be extended continuously to $[0,1]$.
Moreover,
\[
\length\gamma\le \dist{x}{y}{}.
\]
Therefore $\gamma$ is a geodesic path from $x$ to $y$.
\qedsf

\begin{thm}{Exercise}\label{ex:menger}
Let $\spc{X}$ be a complete metric space.
Assume that for any pair of points $x,y\in \spc{X}$, 
there is an \index{almost midpoint}\emph{almost midpoint};
that is, given $\eps>0$, there is a point $z$ such that 
\[\dist{x}{z}{}<\tfrac12\cdot\dist{x}{y}{}+\eps 
\quad\text{and}\quad
\dist{y}{z}{}<\tfrac12\cdot\dist{x}{y}{}+\eps.\]
Show that $\spc{X}$ is a length space.
\end{thm}


\section{Triangles and model tangles}

\parbf{Triangles.}
Given a triple of distinct points $p,q,r$ in a metric space $\spc{X}$, a choice of geodesics $([q r], [r p], [p q])$ will be called a \index{triangle}\emph{triangle}; we will use the short notation 
$\trig p q r=\trig p q r_{\spc{X}}=([q r], [r p], [p q])$\index{$\trig p q r=\trig p q r_{\spc{X}}$ (triangle)}.

Given a triple $p,q,r\in \spc{X}$ there may be no triangle 
$\trig p q r$ simply because one of the pairs of these points cannot be joined by a geodesic.
Also, many different triangles with these vertices may exist, any of which can be denoted by $\trig p q r$.
If we write $\trig p q r$, it means that we have chosen such a triangle.


\parbf{Model triangles.}
Given three points $p,q,r$ in a metric space $\spc{X}$,
let us define its \index{model!triangle}\emph{model triangle} $\trig{\tilde p}{\tilde q}{\tilde r}$ 
(briefly, 
$\trig{\tilde p}{\tilde q}{\tilde r}=\modtrig(p q r)_{\EE^2}$%
\index{$\modtrig$ (model triangle)}) to be a triangle in the Euclidean plane $\EE^2$ such that
\begin{align*}\dist{\tilde p}{\tilde q}{\EE^2}&=\dist{p}{q}{\spc{X}},
&
\quad\dist{\tilde q}{\tilde r}{\EE^2}&=\dist{q}{r}{\spc{X}},
&
\quad\dist{\tilde r}{\tilde p}{\EE^2}&=\dist{r}{p}{\spc{X}}.
\end{align*}

In the same way, we can define the \index{hyperbolic model triangle}\emph{hyperbolic} and the \index{spherical model triangles}\emph{spherical model triangles} $\modtrig(p q r)_{\HH^2}$, $\modtrig(p q r)_{\SSS^2}$
in the Lobachevsky plane $\HH^2$ and the unit sphere~$\SSS^2$.
In the latter case, the model triangle is said to be defined if in addition
\[\dist{p}{q}{}+\dist{q}{r}{}+\dist{r}{p}{}< 2\cdot\pi.\]
In this case, the model triangle again exists and is unique up to an isometry of~$\SSS^2$.

\parbf{Model angles.}
If 
$\trig{\tilde p}{\tilde q}{\tilde r}=\modtrig(p q r)_{\EE^2}$ 
and $\dist{p}{q}{},\dist{p}{r}{}>0$, 
the angle measure of 
$\trig{\tilde p}{\tilde q}{\tilde r}$ at $\tilde p$ 
will be called the \index{model!angle}\emph{model angle} of the triple $p$, $q$, $r$ and will be denoted by
$\angk p q r_{\EE^2}$%
\index{$\angk{p}{q}{r}$ (model angle)}.\label{page:model-angle}

For example, if $\dist{p}{q}{}=\dist{q}{r}{}=\dist{r}{p}{}$, then $\angk p q r_{\EE^2}=\tfrac\pi3$ regardless of existence and relative position of geodesics $[pq]$ and $[pr]$.

The same way we define $\angk p q r_{\MM^2(\kappa)}$;
in particular, $\angk p q r_{\HH^2}$ and $\angk p q r_{\SSS^2}$.
We may use the notation $\angk p q r$ if it is evident which of the model spaces is meant.

\begin{thm}{Exercise}\label{ex:k-><mono}
Show that for any triple of point $p$, $q$, and $r$,
the function
\[\kappa\mapsto \angk p q r_{\MM^2(\kappa)}\]
is nondecreasing in its domain of definition.
\end{thm}


\section{Hinges and their angle measure}\label{sec:angles}

\parbf{Hinges.} Let $p,x,y\in \spc{X}$ be a triple of points such that $p$ is distinct from $x$ and~$y$.
A pair of geodesics $([p x],[p y])$ will be called  a \index{hinge}\emph{hinge} and will be denoted by 
$\hinge p x y=([p x],[p y])$\index{$\hinge p x y$ (hinge)}.

\parbf{Angles.}
The angle measure of a hinge $\hinge p x y$ is defined as the following limit
\[\mangle\hinge p x y=\lim_{\bar x,\bar y\to p} \angk p{\bar x}{\bar y},\]
where $\bar x\in\left]p x\right]$ and $\bar y\in\left]p y\right]$.

Note that if $\mangle\hinge p x y$ is defined, then
\[0\le \mangle\hinge p x y\le \pi.\]

\begin{thm}{Exercise}\label{ex:angkK}
Suppose that in the above definition, one uses spherical or hyperbolic model angles instead of Euclidean.
Show that it does not change the value $\mangle\hinge p x y$.
\end{thm}


\begin{thm}{Exercise}\label{ex:undefined-angle}
Give an example of a hinge $\hinge p x y$ in a metric space with an undefined angle measure $\mangle\hinge p x y$.
\end{thm}

\section{Triangle inequality for angles}

\begin{thm}{Proposition}\label{claim:angle-3angle-inq}
Let  $[px_1]$, $[px_2]$, and $[px_3]$ be three geodesics in a metric space.
Suppose all the angle measures $\alpha_{i j}=\mangle\hinge p {x_i}{x_j}$ are defined.
Then 
\[\alpha_{13}\le \alpha_{12}+\alpha_{23}.\]

\end{thm}



\parit{Proof.}
Since $\alpha_{13}\le\pi$, we can assume that $\alpha_{12}+\alpha_{23}< \pi$.
Denote by $\gamma_i$ the unit-speed parametrization of $[px_i]$ from $p$ to $x_i$.
Given any $\eps>0$, for all sufficiently small $t,\tau,s\in\RR_{\ge0}$ we have
\begin{align*}
\dist{\gamma_1(t)}{\gamma_3(\tau)}{}
&\le 
\dist{\gamma_1(t)}{\gamma_2(s)}{}+\dist{\gamma_2(s)}{\gamma_3(\tau)}{}<\\
&<
\sqrt{t^2+s^2-2\cdot t\cdot  s\cdot \cos(\alpha_{12}+\eps)} +
\\
&\quad+\sqrt{s^2+\tau^2-2\cdot s\cdot \tau\cdot \cos(\alpha_{23}+\eps)}\le
\end{align*}

\begin{wrapfigure}{o}{30 mm}
\vskip-6mm
\centering
\includegraphics{mppics/pic-615}
\vskip6mm
\end{wrapfigure}

Below we define 
$s(t,\tau)$ so that for 
$s=s(t,\tau)$, this chain of inequalities can be continued as follows:
\[\le
\sqrt{t^2+\tau^2-2\cdot t\cdot \tau\cdot \cos(\alpha_{12}+\alpha_{23}+2\cdot \eps)}.
\]

Thus for any $\eps>0$, 
\[\alpha_{13}\le \alpha_{12}+\alpha_{23}+2\cdot \eps.\]
Hence the result follows.

To define $s(t,\tau)$, consider three half-lines $\tilde \gamma_1$, $\tilde \gamma_2$, $\tilde \gamma_3$ on a Euclidean plane starting at one point, such that
$\mangle(\tilde \gamma_1,\tilde \gamma_2)\z=\alpha_{12}+\eps$,
$\mangle(\tilde \gamma_2,\tilde \gamma_3)\z=\alpha_{23}+\eps$,
and $\mangle(\tilde \gamma_1,\tilde \gamma_3)\z=\alpha_{12}\z+\alpha_{23}\z+2\cdot \eps$.
We parametrize each half-line by the distance from the starting point.
Given two positive numbers $t,\tau\in\RR_{\ge0}$, let $s=s(t,\tau)$ be 
the number such that 
$\tilde \gamma_2(s)\in[\tilde \gamma_1(t)\ \tilde \gamma_3(\tau)]$. 
Clearly, $s\le\max\{t,\tau\}$, so $t,\tau,s$ may be taken sufficiently small.
\qeds 

\begin{thm}{Exercise}\label{ex:adjacent-angles}
Prove that the sum of adjacent angles is at least $\pi$.

More precisely: suppose two hinges $\hinge pxz$ and $\hinge pyz$ are \index{adjacent hinges}\emph{adjacent};
that is, they share side $[pz]$, and the union of two sides $[px]$ and $[py]$ form a geodesic $[xy]$.
Show that
\[\mangle\hinge pxz+\mangle\hinge pyz\ge \pi\]
whenever  each angle on the left-hand side is defined.

Give an example showing that the inequality can be strict.
\end{thm}

\begin{thm}{Exercise}\label{ex:first-var}
Assume that the angle measure of $\hinge q p x$ is defined.
Let $\gamma$ be the unit speed parametrization of $[qx]$ from $q$ to $x$.
Show that
\[\dist{p}{\gamma(t)}{}
\le
\dist{q}{p}{}-t\cdot \cos(\mangle\hinge q p x)+o(t).\]

\end{thm}

\section{Hausdorff convergence}\label{sec:Hausdorff convergence}

\begin{thm}{Definition}\label{def:gen-Haus-conv}
Let $A_1,A_2,\dots$ be a sequence of closed sets in a metric space $\spc{X}$.
We say that the sequence $A_n$ \index{Hausdorff!limit}\emph{converges} to a closed set $A_\infty$ in the {}\emph{sense of Hausdorff} if, for any $x\in\spc{X}$, we have
$\distfun_{A_n}(x)\z\to \distfun_{A_\infty}(x)$ as $n\to\infty$.
\end{thm}

For example, suppose $\spc{X}$ is the Euclidean plane and $A_n$ is the circle with radius $n$ and center at the point $(0,n)$; it converges to the $x$-axis.

\begin{figure}[ht!]
\vskip-0mm
\centering
\includegraphics{mppics/pic-415}
\end{figure}

Further, consider the sequence of one-point sets $B_n=\{(n,0)\}$ in the Euclidean plane.
It converges to the empty set;
indeed, for any point $x$ we have $\distfun_{B_n}(x)\to\infty$ as $n\to \infty$ and $\distfun_{\emptyset}(x)= \infty$ for any~$x$.

The following exercise is an extension of the so-called Blaschke selection theorem to our version of Hausdorff convergence.

\begin{thm}{Exercise}\label{ex:generalized-selection}
Show that any sequence of closed sets in a proper metric space has a convergent subsequence in the sense of Hausdorff.
\end{thm}

\section{Hausdorff metric}

\begin{thm}{Definition}\label{def:hausdorff-convergence}
Let $A$ and $B$ be two non-empty compact subsets of a metric space $\spc{X}$.
Then the \index{Hausdorff!distance}\emph{Hausdorff distance} between $A$ and $B$ is defined as 
$$|A-B|_{\Haus\spc{X}}
\df
\sup_{x\in \spc{X}}\{\,|\distfun_A(x)-\distfun_B(x)|\,\}.
$$

\end{thm}

The following observation gives a useful reformulation of the definition:

\begin{thm}{Observation}\label{obs:Haus-nbhds}
Suppose $A$ and $B$ be two compact subsets of a metric space $\spc{X}$.
Then $|A-B|_{\Haus\spc{X}}< R$ if and only if and only if 
$B$ lies in an $R$-neighborhood of $A$, 
and 
$A$ lies in an $R$-neighborhood of~$B$.
\end{thm}

The following exercise implies that Hausdorff convergence of compact subsets is the convergence in Hausdorff metric.

\begin{thm}{Exercise}\label{ex:Haus-conv}
Let $A_1,A_2,\dots,$ and $A_\infty$ be compact non-empty sets in a metric space $\spc{X}$.
Show that $\dist{A_n}{A_\infty}{\Haus\spc{X}}\to 0$ as $n\to\infty$
if and only if $A_n\to A_\infty$ in the sense of Hausdorff.
\end{thm}

\section{Gromov--Hausdorff convergence}\label{sec:Gromov--Hausdorff}

Let $\spc{X}_1,\spc{X}_2,\dots,$ and $\spc{X}_\infty$ be a sequence of complete metric spaces.
Suppose that there is a metric on the disjoint union 
\[\bm{X}=\bigsqcup_{n\in \NN\cup\{\infty\}} \spc{X}_n\] 
that satisfies the following property:

\begin{thm}{Property}\label{propery:GH}
The restriction of the metric on each $\spc{X}_n$ and $\spc{X}_\infty$ coincides with its original metric, 
and $\spc{X}_n\to \spc{X}_\infty$ as subsets in $\bm{X}$ in the sense of Hausdorff.
\end{thm}

In this case we say that the metric on $\bm{X}$ \textit{defines} a \index{Gromov--Hausdorff limit}\emph{convergence} $\spc{X}_n\z\to \spc{X}_\infty$ in the {}\emph{sense of Gromov--Hausdorff}.
The metric on  $\bigsqcup \spc{X}_n$ makes it possible to talk about limits of sequences $x_n\in \spc{X}_n$ as $n\to\infty$, as well as weak limits of a sequence of Borel measures $\mu_n$ on $\spc{X}_n$ and so on.

The limit space is not uniquely defined by the sequence.
For example, if each space $\spc{X}_n$ in the sequence is isometric to the half-line, then its limit might be isometric to the half-line or the whole line.
The first convergence is evident and the second could be guessed from the diagram.

\begin{figure}[ht!]
\vskip-0mm
\centering
\includegraphics{mppics/pic-500}
\end{figure}

Note that any sequence of spaces has an empty space as its limit in some  Gromov--Hausdorff convergence.
Exercise \ref{ex:compact-GH} states that if the limit is non-empty and compact, then it is unique up to isometry. 

\begin{thm}{Exercise}\label{ex:geod-closed}
Let $\spc{X}_1,\spc{X}_2,\dots$ be a sequence of geodesic metric spaces.
Suppose $\spc{X}_n\to \spc{X}_\infty$ is a convergence in the sense of Gromov--Hausdorff.
Assume $\spc{X}_\infty$ is proper, show that it is geodesic.
\end{thm}

\parbf{Pointed convergence.}
Often the isometry class of the limit can be fixed by marking a point $p_n$ in each space $\spc{X}_n$.
We say that $(\spc{X}_n,p_n)$ converges to $(\spc{X}_\infty,p_\infty)$ if there is a metric on $\bm{X}$ as in \ref{propery:GH} such that $p_n\to p_\infty$.
This is called \index{pointed convergence}\emph{pointed Gromov--Hausdorff convergence}.
For example, the sequence $(\spc{X}_n,p_n)=(\RR_{\ge0},0)$ converges to $(\RR_{\ge0},0)$, while $(\spc{X}_n,p_n)=(\RR_{\ge0},n)$ converges to $(\RR,0)$ as $n\to \infty$.

\section{Gromov--Hausdorff metric}\label{sec:Gromov--Hausdorff-metric}

In this section we cook up a metric space out of all compact non-empty metric spaces
that defines Gromov--Hausdorff convergence.
We want to define the metric on the set of \textit{isometry classes} of compact metric spaces.
Further, the term \textit{metric space} might also stand for its \textit{isometry class}.

The obtained metric is called the Gromov--Hausdorff metric;
the corresponding metric space will be denoted by $\GH$.
This distance is defined as the maximal metric such that \textit{the distance between subspaces in a metric space is not greater than the Hausdorff distance between them}.
Here is a formal definition.

\begin{thm}{Definition}\label{def:GH}
The \index{Gromov--Hausdorff distance}\emph{Gromov--Hausdorff distance} $|\spc{X}-\spc{Y}|_{\GH}$ between compact metric spaces $\spc{X}$ and $\spc{Y}$
is defined by the following
relation.
 
Given  $r > 0$, we have $|\spc{X}-\spc{Y}|_{\GH} < r$ if and only if there exists a metric
space $\spc{W}$ and subspaces $\spc{X}'$ and $\spc{Y}'$ in $\spc{W}$ that are isometric to $\spc{X}$ and $\spc{Y}$,
respectively, such that $|\spc{X}'-\spc{Y}'|_{\Haus\spc{W}} < r$. 
(Here $|\spc{X}'-\spc{Y}'|_{\Haus\spc{W}}$ denotes the Hausdorff distance between sets $\spc{X}'$ and $\spc{Y}'$ in $\spc{W}$.)
\end{thm}

For the proof of the following statement we refer to \cite{burago-burago-ivanov} and \cite{petrunin2023pure}.

\begin{thm}{Proposition}\label{prop:complete}
$\GH$ is a complete metric space.
\end{thm}

Note that this means in particular that if $X,Y$ are compact and $|\spc{X}-\spc{Y}|_{\GH}=0$ then $X$ and $Y$ are isometric.

Gromov--Hausdorff convergence of compact spaces has particularly nice properties.
From the technical point of view, they follow from the next statement, which we formulate as an exercise.

\begin{thm}{Exercise}\label{ex:non-contracting-map}
Let $f$ be a distance noncontracting map from 
a compact metric space $\spc{K}$ to itself.
Show that $f$ is an isometry; that is, it is a distance-preserving bijection.
\end{thm}

For two metric spaces $\spc{X}$ and $\spc{Y}$,
we write $\spc{X}\le \spc{Y}+\eps$ if
there is a map $f\:\spc{X}\to \spc{Y}$ such that 
\[\dist{x}{x'}{\spc{X}}\le \dist{f(x)}{f(x')}{\spc{Y}}+\eps\]
for any $x,x'\in \spc{X}$.

\begin{thm}{Exercise}\label{ex:GH-po}
Let $\spc{X}_1,\spc{X}_2,\dots,$ and $\spc{X}_\infty$ are compact metric spaces.
Show that there is a Gromov--Hausdorff convergence $\spc{X}_n\to\spc{X}_\infty$ if and only if for some sequence $\eps_n\to 0$,
we have 
\[\spc{X}_\infty\le \spc{X}_n+\eps_n\quad\text{and}\quad \spc{X}_n\le \spc{X}_\infty+\eps_n.\]
\end{thm}

\begin{thm}{Exercise}\label{ex:compact-GH}
Let $\spc{X}_1,\spc{X}_2,\dots$ be a sequence of metric spaces.
Suppose $\spc{X}_\infty$ and $\spc{X}_\infty'$ are non-empty limit spaces for some Gromov--Hausdorff convergences of $\spc{X}_n$.
Assume $\spc{X}_\infty$ is compact, show that it is isometric to~$\spc{X}_\infty'$.
\end{thm}

\section{Almost isometries}

\begin{thm}{Definition}
Let $\spc{X}$ and $\spc{Y}$ be metric spaces.
A map $f\:\spc{X}\to\spc{Y}$
is called an \index{isometry!$\eps$-isometry}\emph{$\eps$-isometry}
if the following two conditions hold:

\begin{subthm}{}
$f(\spc{X})$ is an \index{$\eps$-net}\emph{$\eps$-net} in $\spc{Y}$; that is, for any $y\in \spc{Y}$ there is $x\in \spc{X}$ such that $\dist{f(x)}{y}{\spc{Y}}<\eps$.
\end{subthm}

\begin{subthm}{}
$\bigl|\dist{f(x)}{f(x')}{\spc{Y}}-\dist{x}{x'}{\spc{X}}\bigr|\le \eps$ for any $x,x'\in\spc{X}$.
\end{subthm}

\end{thm}

When dealing with Gromov--Hausdorff convergence the following lemma is often useful as it allows to bypass constructing explicit metrics on the disjoint unions of $\spc{X}_1,\spc{X}_2,\dots$, and $\spc{X}_\infty$

\begin{thm}{Lemma}\label{lem:almost-isom}
Let $\spc{X}_1,\spc{X}_2,\dots$, and $\spc{X}_\infty$ be complete metric spaces,
and let $\eps_n\to\0+$ as $n\to\infty$.
Suppose that either 
\begin{subthm}{lem:almost-isom-a}
for each $n$ there is an $\eps_n$-isometry $f_n\:\spc{X}_n\to\spc{X}_\infty$, or
\end{subthm}
\begin{subthm}{lem:almost-isom-b}
for each $n$ there is an $\eps_n$-isometry $h_n\:\spc{X}_\infty\to\spc{X}_n$.
\end{subthm}
Then there is a Gromov--Hausdorff convergence $\spc{X}_n\to \spc{X}_\infty$.

Furthermore, a partial converse also holds.

\begin{subthm}{lem:almost-isom-c}
Suppose we have a Gromov--Hausdorff convergence $\spc{X}_n\to \spc{X}_\infty$ and $\spc{X}_\infty$ is compact. Then there exist $\eps_n\to\0+$ as $n\to\infty$ and  $\eps_n$-isometris $f_n\:\spc{X}_n\to\spc{X}_\infty$ (and $h_n\:\spc{X}_\infty\to\spc{X}_n$)
such that $x_n\in \spc{X}_n$ converges to $x_\infty \in  \spc{X}_\infty$ with respect to the  convergence $\spc{X}_n\z\to \spc{X}_\infty$ if and only if $f_n(x_n)\to x_\infty$ (respectively, $\dist{h_n(x_\infty) }{x_n}{\spc{X}_n}\to 0$) as $n\to\infty$.
\end{subthm}
\end{thm}


\parit{Proof.}
To prove part \ref{SHORT.lem:almost-isom-a} let us construct a common space $\bm{X}$ for the spaces $\spc{X}_1,\spc{X}_2,\dots$, and $\spc{X}_\infty$
by taking the metric $\rho$ on the disjoint union $\spc{X}_\infty\sqcup\spc{X}_1\sqcup\spc{X}_2\sqcup\dots$ that is defined the following way:
\begin{align*}
\dist{x_n}{y_n}{\bm{X}}&=\dist{x_n}{y_n}{\spc{X}_n},
\\
\dist{x_\infty}{y_\infty}{\bm{X}}&=\dist{x_\infty}{y_\infty}{\spc{X}_\infty},
\\
\dist{x_n}{x_\infty}{\bm{X}}&=\inf\set{\dist{x_n}{y_n}{\spc{X}_n}+\eps_n+\dist{x_\infty}{f(y_n)}{\spc{X}_\infty}}{{y_n}\in \spc{X}_n},
\\
\dist{x_n}{x_m}{\bm{X}}&=\inf\set{\dist{x_n}{y_\infty}{\bm{X}}+\dist{x_m}{y_\infty}{\bm{X}}}{y_\infty\in\spc{X}_\infty},
\end{align*}
where we assume that $x_\infty,y_\infty\in \spc{X}_\infty$, and $x_n,y_n\in \spc{X}_n$ for each $n$. 
It remains to observe that this indeed defines a metric on $\bm{X}$, and $\spc{X}_n\to \spc{X}_\infty$ in the sense of Hausdorff.

The proof of the second part is analogous; one only needs to change one line in the definition of the metric to the following:
\[\dist{x_n}{x_\infty}{\bm{X}}=\inf\set{\dist{x_n}{h(y_\infty)}{\spc{X}_n}+\eps_n+\dist{x_\infty}{y_\infty}{\spc{X}_\infty}}{{y_\infty}\in \spc{X}_\infty}.\]

We leave part \ref{SHORT.lem:almost-isom-c} as an exercise.
\qedsf

Lemma~\ref{lem:almost-isom} has a natural analogue for pointed convergence.
For simplicity we only state part \ref{SHORT.lem:almost-isom-a} of the lemma.
Parts \ref{SHORT.lem:almost-isom-b} and \ref{SHORT.lem:almost-isom-c} can be rephrased similarly.



\begin{thm}{Lemma}\label{lem:almost-isom-pointed}
Let $(\spc{X}_1, p_1),(\spc{X}_2,p_2) ,\dots$, let $(\spc{X}_\infty, p_\infty)$ be pointed metric spaces, and let $\eps(n,R)\to\0+$ as $n\to\infty$ for any fixed $R>0$.
Suppose that for each $n$ there is a map $f_n\:\spc{X}_n\to\spc{X}_\infty$ such that


\begin{subthm}{}
$f_n(p_n)\to p_\infty$
\end{subthm}

\begin{subthm}{}
$\bigl|\dist{f_n(x)}{f_n(x')}{\spc{X}_\infty }-\dist{x}{x'}{\spc{X}_n}\bigr|\le \eps(n,R)$ for any $x,x'\z\in \oBall(p_n,R)$.
\end{subthm}

\begin{subthm}{}
For any $x \in \oBall(p_\infty,R)$ there is $x_n\in \oBall(p_n,R)$ such that $\dist{x}{f_n(x_n)}{}\le \eps(n,R)$
\end{subthm}

Then there is a pointed  Gromov--Hausdorff convergence $(\spc{X}_n,p_n)\z\to (\spc{X}_\infty,p_\infty)$.
\end{thm}

The proofs of \ref{lem:almost-isom-pointed} and \ref{lem:almost-isom} are analogous;
we leave the former to the reader.



\section{Comments}

In principle, our definition of Gromov--Hausdorff distance works for complete metric spaces that are not necessarily compact.
However, according to the following exercise, it only defines a \emph{semimetric}; that is, zero Gromov--Hausdorff distance does not imply that the spaces are isometric.
For that reason it is not in use.

\begin{thm}{Exercise}\label{ex:GH-noncompact}
Construct two nonisometric proper (noncompact) metric spaces with vanishing Gromov--Hausdorff distance.
\end{thm}


%%%%%%%%%%%%%%%%%%%%%%%%%%%%
%\chapter{Definitions}

The first synthetic description of curvature is due to Abraham Wald \cite{wald} published in 1936;
it was his student work, written under the supervision of Karl Menger. 
This publication was not noticed for about 50 years \cite{berestovskii}.
In 1941, similar definitions were rediscovered by Alexandr Alexandrov \cite{alexandrov:def}.



\section{Wald's approach}

Abraham Wald noticed that given a \textit{typical} metric on the quadruple of points $\spc{X}\z=\{x_1,x_2,x_3,x_4\}$ there is a closed interval,
say 
\[[\kappa_{\min}(x_1,x_2,x_3,x_4),\kappa_{\max}(x_1,x_2,x_3,x_4)]\subset \RR\]
such that there is a \textit{model configuration} in $\MM^3(\kappa)$;
that is, $\tilde x_1$, $\tilde x_2$, $\tilde x_3$, $\tilde x_4\in\MM^3(\kappa)$ such that
\[\dist{\tilde x_i}{\tilde x_j}{\MM^3(\kappa)}=\dist{x_i}{x_j}{\spc{X}}\]
for all $i$ and $j$.


\begin{wrapfigure}{r}{33mm}
\vskip-2mm
\centering
\includegraphics{mppics/pic-710}
\end{wrapfigure}

In $\MM^3(\kappa_{\min})$ and $\MM^3(\kappa_{\max})$, the points $\tilde x_1,\tilde x_2,\tilde x_3,\tilde x_4$ form degenerate tetrahedrons shown on the diagram (for $\kappa_{\min}$ it is a convex quadrangle and for $\kappa_{\max}$ --- a triangle with a point inside).
In the interior of the interval, the tetrahedron is nondegenerate.

Moreover, one can use $[-\infty,\infty)$ instead of $\RR$ 
and let
\[\kappa_{\min}(x_1,x_2,x_3,x_4)=-\infty\]
if there is \textit{almost} model quadruple in
$\MM^3(\kappa)$ for $\kappa\to -\infty$;
that is, for any $\eps>0$ there is a quadruple
$\tilde x_1,\tilde x_2,\tilde x_3,\tilde x_4\in\MM^3(\kappa)$
such that $\kappa\le -\tfrac1\eps$, and
\[\dist{\tilde x_i}{\tilde x_j}{\MM^3(\kappa)}\lege\dist{x_i}{x_j}{\spc{X}}\pm\eps\]
for all $i$ and $j$.
In this case the interval 
\[[\kappa_{\min}(x_1,x_2,x_3,x_4),\kappa_{\max}(x_1,x_2,x_3,x_4)]\subset [-\infty,\infty)\]
is defined for \textit{any} quadruple.

\begin{thm}{Exercise}
Let $x_1,x_2,x_3,x_4$ be a quadruple in a metric space such that $\kappa_{\min}(x_1,x_2,x_3,x_4)=-\infty$.
Show that two maximal numbers from the following three are equal to each other.
\begin{align*}
a&=\dist{x_1}{x_2}{}+\dist{x_3}{x_4}{},
\\
b&=\dist{x_1}{x_3}{}+\dist{x_2}{x_4}{},
\\
c&=\dist{x_1}{x_4}{}+\dist{x_2}{x_3}{}.
\end{align*}


\end{thm}


\begin{thm}{Exercise}
Suppose that $x_1,x_2,x_3,x_4$ in a metric space
such that
\begin{align*}
\dist{x_1}{x_2}{}=\dist{x_1}{x_3}{}=\dist{x_1}{x_4}{}&=1,
\\
\dist{x_2}{x_3}{}=\dist{x_3}{x_4}{}=\dist{x_4}{x_1}{}&=2.
\end{align*}
Show that 
\[\kappa_{\min}(x_1,x_2,x_3,x_4)=\kappa_{\max}(x_1,x_2,x_3,x_4)=-\infty.\]
\end{thm}

\begin{thm}{Exercise}
Let $x_1,x_2,x_3,x_4$ be a quadruple in $\EE^2$.
Suppose that $x_3$ lie on the line thru $x_1$ and $x_2$,
but $x_4$ does not.
Show that 
\[\kappa_{\min}(x_1,x_2,x_3,x_4)=\kappa_{\max}(x_1,x_2,x_3,x_4)=0.\]
\end{thm}

\begin{thm}{Wald-style definition}
Let $\kappa\in \RR$.
A metric space $\spc{X}$ has curvature $\ge\kappa$ (or $\le\kappa$) 
if for any quadruple $x_1,x_2,x_3,x_4\in \spc{X}$ we have 
$\kappa_{\max}(x_1,x_2,x_3,x_4)\ge \kappa$ (or $\kappa_{\min}(x_1,x_2,x_3,x_4)\le \kappa$ respectively). 
\end{thm}

This definition is given for its historical value.
It will not be used further in the sequel.
We will use another definition that is very close, but not equivalent.

\section{Substance}\label{sec:manifesto}

Consider the space $\mathcal{M}_4$ of all isometry classes of 4-point metric spaces.
Each element in $\mathcal{M}_4$ can be described by 6 numbers 
 --- the distances between all 6 pairs of its points, say $\ell_{i,j}$ for $1\le i< j\le 4$ modulo permutations of the index set $(1,2,3,4)$.
These 6 numbers are subject to 12 triangle inequalities; that is,
\[\ell_{i,j}+\ell_{j,k}\ge \ell_{i,k}\]
holds for all $i$, $j$ and $k$, where we assume that $\ell_{j,i}=\ell_{i,j}$, and $\ell_{i,i}=0$.

{

\begin{wrapfigure}{o}{33mm}
\vskip-3mm
\centering
\includegraphics{mppics/pic-700}
\end{wrapfigure}

The space $\mathcal{M}_4$ comes with topology.
It can be defined as a quotient topology of the cone in $\RR^6$ by permutations of the 4 points of the space.

Consider the subset $\mathcal{E}_4\subset \mathcal{M}_4$ of all isometry classes of 4-point metric spaces that admit isometric embeddings into Euclidean space.

}

\begin{thm}{Claim}\label{clm:two-components-of-M4}
The complement $\mathcal{M}_4\setminus \mathcal{E}_4$ has two connected components.
\end{thm}

\begin{thm}{Exercise}
Spend 10 minutes trying to prove the claim.
\end{thm}


The definition of Alexandrov spaces is based on the claim above.
Let us denote one of the components by $\mathcal{P}_4$ and the other by~$\mathcal{N}_4$.
Here $\mathcal{P}$ and $\mathcal{N}$ stand for {}\emph{positive} 
and {}\emph{negative curvature} because spheres have no quadruples of type $\mathcal{N}_4$ and 
hyperbolic space
has no quadruples of type~$\mathcal{P}_4$.

A metric space that has no quadruples of points of type $\mathcal{P}_4$ or $\mathcal{N}_4$
respectively 
is called an Alexandrov space with non-positive or non-negative curvature (briefly.

\begin{wrapfigure}{r}{33mm}
\vskip-0mm
\centering
\includegraphics{mppics/pic-710}
\end{wrapfigure}

Let us describe the subdivision into  $\mathcal{P}_4$, $\mathcal{E}_4$, and $\mathcal{N}_4$ intuitively.
Imagine that you move out of $\mathcal{E}_4$ --- your path is a one-parameter family of 4-point metric spaces.
The last thing you see in $\mathcal{E}_4$ is one of the two plane configurations shown on the diagram.
If you see the right configuration then you move into $\mathcal{N}_4$;
if it is the one on the left, then you move into $\mathcal{P}_4$.
More degenerate pictures can be avoided; for example, a triangle with a point on a side.
From such a configuration one may move in $\mathcal{N}_4$ and $\mathcal{P}_4$ (as well as come back to $\mathcal{E}_4$).

Here is an exercise, solving which would force you to rebuild a considerable part of Alexandrov geometry.
It is wise to spend some time thinking about this it before proceeding.

\begin{thm}{Advanced exercise}\label{ex:convex-set}
Assume $\spc{X}$ is a complete metric space with length metric (see Section~\ref{sec:length}), 
containing only quadruples of type~$\mathcal{E}_4$.
Show that $\spc{X}$ is isometric to a convex set in a Hilbert space.
\end{thm}

If in the definition above, we take $\MM^3(\kappa)$ instead of $\EE^3$.
Then we will arrive at Wald's definition of curvature bounded below and above by $\kappa$.
The parameter $\kappa$ has three interesting choices $-1$, $0$, and $1$;
the rest can be obtained from these three applying rescaling.

Again, the definition that we are going to use is not equivalent.


\section{Embedding theorem}

The following theorem is historically the first remarkable result in Alexandrov geometry.
The main part of the following theorem is due to Alexandr Alexandrov~\cite{alexandrov-1948}.
The last part is very difficult; it was proved by Aleksei Pogorelov~\cite{pogorelov}.

\begin{thm}{Theorem}\label{thm:alexandrov+pogorelov}
A metric space $\spc{X}$ is isometric to the surface of a convex body in the Euclidean space if and only if $\spc{X}$ is an $\Alex0$ space that is homeomorphic to $\SSS^2$.

Moreover, $\spc{X}$ determines the convex body up to congruence.
\end{thm}

The convex body above is a compact convex subset in $\EE^3$;
we assume that it does not lie in a line but might degenerate to a plane figure, say $F$.
In the latter case, its surface is defined as two copies of $F$ glued along the boundary.
For nondegenerate convex body $B$, its surface is its boundary $\partial B$ equipped with the induced length metric. 

The only-if part of the theorem is the simplest; we will give a complete proof of it eventually.
The if part will be sketched.
We will not touch the last part.

%%!TEX root = the-definitions.tex
\chapter{Definitions}\label{chap:defs}

In this lecture we prove equivalence of several definitions of Alexandrov space.


\section{Four-point comparison}\label{sec:4-point}

Recall that $\angk  pxy$ denotes the model angle; see page \pageref{page:model-angle}.

Let $p,x,y,z$ be a quadruple of points in a metric space.
If the inequality 
\[\angk  pxy_{\EE^2}+\angk pyz_{\EE^2}+\angk pzx_{\EE^2}
\le 
2\cdot\pi
\eqlbl{eq:CBB-comparison}\]
holds, then we say that the quadruple meets \index{comparison}\emph{$\EE^2$-comparison}.
If the left-hand side is undefined, then we assume that the comparison holds.

\begin{thm}{Exercise}\label{ex:CBB+-}
Suppose $\EE^2$-comparison holds for quadruple $p,x_1,x_2,x_3$.
Show that $\EE^2$-comparison holds for quadruple $q,y_1,y_2,y_3$ if
\[\dist{q}{y_i}{}\ge\dist{p}{x_i}{}\qquad\text{and}\qquad\dist{y_i}{y_j}{}\le\dist{x_i}{x_j}{}\]
for all $i$ and $j$.
\end{thm}

Instead of $\EE^2$, we can use $\SSS^2$ or $\HH^2$.
This way we get the definition of $\SSS^2$- or $\HH^2$-comparisons.
Recall that $\angk  pxy_{\EE^2}$ and $\angk  pxy_{\HH^2}$ are defined if $p\ne x$, $p\ne y$,
but for $\angk  pxy_{\SSS^2}$ we require in addition that
\[\dist{p}{x}{}+\dist{p}{y}{}+\dist{x}{y}{}<2\cdot\pi;\]
if this does not hold for one of the angles, then we assume that $\SSS^2$-comparison holds for this quadruple.

More generally, one may apply this definition to $\MM^2(\kappa)$ and  define $\MM^2(\kappa)$-comparison for any real $\kappa$.
However, if you see $\MM^2(\kappa)$-comparison, it is safe to assume that $\kappa=-1$, $0$, or $1$;
applying rescaling, the $\MM^2(\kappa)$-comparison can be reduced to these three cases.

\begin{thm}{Definition}\label{def:CBB}
A metric space $\spc{X}$ has {}\emph{curvature $\ge\kappa$} in the sense of Alexandrov
if $\MM^2(\kappa)$-comparison
holds for any quadruple of points in $\spc{X}$.
\end{thm}

While this definition can be applied to any metric space,
we will use it mostly for geodesic spaces that are complete (and often compact or proper). 
If a complete geodesic space has curvature $\ge\kappa$ in the sense of Alexandrov, 
then it will be called an $\Alex\kappa$ space; here $\Alex\kappa$ is an adjective.
An $\spc{X}$ is $\Alex\kappa$ for some $\kappa$, then we say that $\spc{X}$ is an \index{Alexandrov space}\emph{Alexandrov space}.

It is common practice in Alexandrov geometry to write proofs for nonnegative curvature and 
leave the general curvature bound as an exercise. These generalizations are usually straightforward. We will add notes when they are not.
We will also often formulate statements just for $\kappa=0$ even when they admit straightforward generalizations to arbitrary curvature bounds;
see \cite{alexander-kapovitch-petrunin2024} for a more formal tratment.


\begin{thm}{Exercise}\label{ex:Euclid-is-CBB}
Show that $\EE^n$ is $\Alex0$.
\end{thm}

\begin{thm}{Exercise}\label{ex:(3+1)-expanding}
Show that a metric space $\spc{X}$ has nonnegative curvature in the sense of Alexandrov
if and only if for any quadruple of points $p,x_1,x_2,x_3\in \spc{X}$ 
there is a quadruple of points $q,y_1,y_2,y_3\in\EE^3$
such that 
\[\dist{p}{x_i}{\spc{X}}\ge\dist{q}{y_i}{\EE^2} 
\quad \text{and}\quad
\dist{x_i}{x_j}{\spc{X}}\le\dist{y_i}{y_j}{\EE^2}\] 
for all $i$ and $j$.
\end{thm}

\section{Alexandrov's lemma}

Recall that $[xy]$ denotes a geodesic from $x$ to $y$;
set  
\index{10@$\left]x y\right]$, $\left[x y\right[$, $\left]x y\right[$}
\[
\left]x y\right]=[xy]\setminus\{x\},
\quad
\left[x y\right[=[xy]\setminus\{y\},
\quad
\left]x y\right[=[xy]\setminus\{x,y\}.\]

\begin{thm}{Lemma}
\index{Alexandrov's lemma}
\label{lem:alex}  
Let $p,x,y,z$ be distinct points in a metric space such that $z\in \left]x y\right[$.
Then 
the following expressions have the same sign:

\begin{subthm}{lem-alex-difference}
$\angk x p y
-\angk x p z$,
\end{subthm} 

\begin{subthm}{lem-alex-angle}
$\angk z p x
+\angk z p y -\pi$.
\end{subthm}

\begin{wrapfigure}{r}{25mm}
\vskip-6mm
\centering
\includegraphics{mppics/pic-730}
\end{wrapfigure}

The same holds for the hyperbolic and spherical model angles, 
but in the latter case, one has to assume in addition that
\[\dist{p}{x}{}+\dist{p}{y}{}+\dist{x}{y}{}< 2\cdot\pi.\]

\end{thm}

In the proof we will apply the following statement from elementary geometry.

\begin{thm}{Angle monotonicity}\label{angle-monotonicity}
Increasing the opposite side in a plane triangle increases the corresponding angle, and the other way around.

Moreover, the same statement holds for spherical and hyperbolic triangles.
\end{thm}


\parit{Proof.} 
Consider the model triangle $\trig{\tilde x}{\tilde p}{\tilde z}=\modtrig(x p z)$.
Take 
a point $\tilde y$ on the extension of 
$[\tilde x \tilde z]$ beyond $\tilde z$ so that $\dist{\tilde x}{\tilde y}{}=\dist{x}{y}{}$ (and therefore $\dist{\tilde x}{\tilde z}{}=\dist{x}{z}{}$). 

\begin{wrapfigure}{r}{33mm}
\vskip-0mm
\centering
\includegraphics{mppics/pic-740}
\end{wrapfigure}

By the angle monotonicity (\ref{angle-monotonicity}),
the following expressions have the same sign:
\begin{enumerate}[(i)]
\item $\mangle\hinge{\tilde x}{\tilde p}{\tilde y}-\angk{x}{p}{y}$,
\item $\dist{\tilde p}{\tilde y}{}-\dist{p}{y}{}$,
\item $\mangle\hinge{\tilde z}{\tilde p}{\tilde y}-\angk{z}{p}{y}$.
\end{enumerate}
Since 
\[\mangle\hinge{\tilde x}{\tilde p}{\tilde y}=\mangle\hinge{\tilde x}{\tilde p}{\tilde z}=\angk{x}{p}{z}\]
and
\[ \mangle\hinge{\tilde z}{\tilde p}{\tilde y}
=\pi-\mangle\hinge{\tilde z}{\tilde x}{\tilde p}
=\pi-\angk{z}{x}{p},\]
the statement follows.


The spherical and hyperbolic cases can be proved along the same lines.
\qeds

\begin{thm}{Exercise}\label{ex:alex-lemma-cat}
Assume $p,x,y,z$ are as in Alexandrov's lemma (\ref{lem:alex}).
Show that
\[\angk p x y
\ge
\angk p x z + \angk p z y,\]
with equality if and only if the expressions in \ref{SHORT.lem-alex-difference} and \ref{SHORT.lem-alex-angle} in Alexandrov's lemma vanish.
\end{thm}

Note that 
\[p\in\left]x y\right[
\quad\Longrightarrow\quad
\angk pxy=\pi.
\]
Applying it with Alexandrov's lemma and $\EE^2$-comparison, we get the following.

\begin{thm}{Claim}\label{clm:angle-mono}
If $p,x,y,z$ are points in an $\Alex0$ space.
Suppose $p\in\left]x y\right[$, then 
\[\angk xyz\le \angk xpz.\]
\end{thm}

\begin{wrapfigure}{r}{25mm}
\vskip-0mm
\centering
\includegraphics{mppics/pic-750}
\end{wrapfigure}

\begin{thm}{Exercise}\label{ex:noncreasing}
Let $\hinge p x y$ be a hinge in an $\Alex0$ space.
Consider the function
\[f\:(\dist{p}{\bar x}{},\dist{p}{\bar y}{})\mapsto \angk p{\bar x}{\bar y},\]
where $\bar x\in\left]p x\right]$ and $\bar y\in\left]p y\right]$.
Show that $f$ is nonincreasing in each argument.
\end{thm}

This exercise implies the following.

\begin{thm}{Claim}\label{clm:angle-defined}
The angle measure of any hinge in an $\Alex0$ 
space is defined and  is at least as large as the corresponding model angle;
that is,
\[\mangle\hinge p x y\ge \angk p x y\]
for any hinge $\hinge p x y$ in an $\Alex0$.

\end{thm}

\begin{thm}{Exercise}\label{ex:0-angle}
Let $\hinge p x y$ be a hinge in an $\Alex0$ space.
Suppose $\mangle\hinge p x y=0$; show that $[px]\subset [py]$ or $[py]\subset [px]$.

Conclude that geodesics in $\Alex0$ space cannot \emph{bifurcate};
that is, if two geodesics $[px]$ and $[py]$ share a nontrivial arc with an end at $p$, then $[px]\subset [py]$ or $[py]\subset [px]$.
\end{thm}

\begin{thm}{Exercise}\label{ex:pi-angle}
Let $[xy]$ be a geodesic in an $\Alex0$ space.
Suppose $z\in \left]xy\right[$. Show that there is a unique geodesic $[xz]$ and $[xz]\subset [xy]$.
\end{thm}

Recall that adjacent hinges are defined in \ref{ex:adjacent-angles}.

\begin{thm}{Exercise}\label{ex:adjacent-CBB}
Let $\hinge pxz$ and $\hinge pyz$ be adjacent hinges in an $\Alex0$ 
space.
Show that
\[\mangle\hinge pxz+\mangle\hinge pyz= \pi.\]
\end{thm}


\begin{thm}{Exercise}\label{ex:pxyvw}
Let $\spc{A}$ be an $\Alex0$ 
space.
Show that  
\[
\angk xyp=\angk xvp
\quad\Longleftrightarrow\quad
\angk xyp=\angk xwp
\]
for any points
$p,x,y,v,w$ in $\spc{A}$ such that $v,w\in \left]xy\right[$.
\end{thm}

\begin{thm}{Exercise}\label{ex:angle-lim}
Let $\spc{A}$ be an $\Alex0$ space.
Suppose hinges $\hinge {x_n}{y_n}{z_n}$ in $\spc{A}$ converge to a hinge $\hinge {x_\infty}{y_\infty}{z_\infty}$;
that is, geodesics $[x_ny_n]$ and $[x_nz_n]$ converge to the geodesics $[x_\infty y_\infty]$ and $[x_\infty z_\infty]$ in the sense of Hausdorff.
Show that 
\[\liminf_{n\to\infty}\mangle \hinge {x_n}{y_n}{z_n}\ge \mangle \hinge {x_\infty}{y_\infty}{z_\infty}.\]
\end{thm}

The last inequality might be strict;
for example, on the surface of convex polyhedron, which is a $\Alex0$ space by \ref{prop:conv-surf-CBB(0)}.

\section{Hinge comparison}

Let $\hinge pxy$ be a hinge in an $\Alex0$ space $\spc{A}$.
By \ref{ex:noncreasing}, the angle measure $\mangle\hinge pxy$ is defined and
\[\mangle\hinge pxy\ge \angk pxy.\]
Further, according to \ref{ex:adjacent-CBB}, we have 
\[\mangle\hinge pxz+\mangle\hinge pyz=\pi\]
for adjacent hinges $\hinge pxz$ and $\hinge pyz$ in $\spc{A}$.

The following theorem provides a converse.

\begin{thm}{Theorem}\label{thm:angle-cbb}
A complete geodesic space $\spc{A}$ is $\Alex0$ if the following conditions hold.

\begin{subthm}{angle-a}
For any hinge $\hinge x p y$ in $\spc{A}$, the angle 
$\mangle\hinge x p y$ is defined and 
\[\mangle\hinge x p y\ge\angk x p y.\]
\end{subthm}

\begin{subthm}{angle-b}
For any two adjacent hinges $\hinge pxz$ and $\hinge pyz$ in $\spc{A}$, we have
\[\mangle\hinge pxz+\mangle\hinge pyz\le\pi.\]
\end{subthm}

\end{thm}

\parit{Proof.}
Consider a point  $w\in \mathopen{]} p z \mathclose{[}$ close to $p$.
From \ref{SHORT.angle-b}, it follows that 
\[\mangle\hinge w x z+ \mangle\hinge w x{p}\le\pi\quad \text{and}\quad \mangle\hinge w y z + \mangle\hinge w y{p}\le\pi.\]

\begin{wrapfigure}{o}{30 mm}
\vskip-0mm
\centering
\includegraphics{mppics/pic-805}
\vskip4mm
\end{wrapfigure}

Since $\mangle\hinge w x y\le \mangle\hinge w x p +\mangle\hinge w y{p}$ (see \ref{claim:angle-3angle-inq}), we get 
\[\mangle\hinge w x z+ \mangle\hinge w y z +\mangle\hinge w x y
\le
2\cdot\pi.\]
Applying \ref{SHORT.angle-a}, 
\[\angk w x z
+ \angk w y z 
+\angk w x y
\le
2\cdot\pi.\]
Passing to the limits as $w\to p$, we have
\[\angk p x z 
+ \angk p y z 
+\angk p x y
\le
2\cdot\pi.\]
\qedsf

\section{Equivalent conditions}

The following theorem summarizes \ref{clm:angle-mono}, \ref{clm:angle-defined}, \ref{ex:adjacent-CBB}, and \ref{thm:angle-cbb}.

\begin{thm}{Theorem}\label{thm:defs_of_alex} 
Let $\spc{A}$ be a complete geodesic space.
Then the following conditions are equivalent.

\begin{subthm}{cbb}
$\spc{A}$ is $\Alex0$.
\end{subthm}
 

\begin{subthm}{2-sum} 
(adjacent angle comparison\index{comparison!adjacent angle comparison})
\[\angk z p x
+\angk z p y\le \pi\]
for any geodesic $[x y]$ and point $z\in \mathopen{]}x y\mathclose{[}$, $z\ne p$ in $\spc{A}$.
\end{subthm}

\begin{subthm}{point-on-side}
(\index{comparison!point-on-side comparison}point-on-side comparison)
\[\angk x p y\le\angk x p z\]
for any geodesic $[x y]$ and $z\in \mathopen{]}x y\mathclose{[}$ in $\spc{A}$.
\end{subthm}

\begin{subthm}{angle}(hinge comparison\index{comparison!hinge comparison})
\index{hinge!comparison}
the angle $\mangle\hinge x p y$ is defined for any hinge $\hinge x p y$ in $\spc{A}$.
Moreover, 
\[\mangle\hinge x p y\ge\angk x p y\]
for any hinge $\hinge x p y$, and
\[\mangle\hinge z p y + \mangle\hinge z p x\le\pi\]
for any adjacent hinges $\hinge z p y$ and $\hinge z p x$.
\end{subthm}

Moreover, the implications \ref{SHORT.cbb}$\Rightarrow$\ref{SHORT.2-sum}$\Rightarrow$\ref{SHORT.point-on-side}$\Rightarrow$\ref{SHORT.angle} hold in any space, not necessarily a geodesic one.
\end{thm}

\begin{thm}{Advanced Exercise}\label{ex:urysohn}
Construct a complete geodesic space $\spc{X}$ that is not $\Alex0$, but satisfies the following weaker version of the adjacent angle comparison \ref{2-sum}.

For any three points $p,x,y\in \spc{X}$ there is a geodesic $[x y]$ such that for any $z\in \left]x y\right[$
\[\angk{z}{p}{x}+\angk{z}{p}{y}
\le
\pi.\]
\end{thm}

\begin{thm}{Exercise}\label{ex:normCBB}
Let $\spc{W}$ be $\RR^n$ with the metric induced by a norm.
Show that if $\spc{W}$ is $\Alex0$, then $\spc{W}$ is isometric to the Euclidean space~$\EE^n$.
\end{thm}

\section{Function comparison}\label{Function comparison}

\parbf{Real-to-real functions.}
Choose $\lambda\in \RR$.
Let $s\:\II\to\RR$ be a locally Lipschitz function defined on an interval $\II$.
The following statement are equivalent;
if one (and therefore any) of them holds for $s$, then we say that $s$ is \index{91@$\lambda$-concave function}\emph{$\lambda$-concave}.
\begin{itemize}
\item We have inequality $s''\le \lambda$, where the second derivative $s''$ is understood in the sense of distributions.
\item The function $t\mapsto s(t)-\lambda\cdot\tfrac{t^2}2$ is concave.
\item The \index{Jensen inequality}\emph{Jensen inequality}
\[s(a\cdot t_0+(1-a)\cdot t_1)\ge a\cdot s(t_0)+(1-a)\cdot s(t_1)+\tfrac\lambda2\cdot a\cdot(1-a)\cdot(t_1-t_0)^2 \]
holds for any $t_0,t_1\in \II$ and $a\in[0,1]$.
\item for any $t_0\in \II$ there is a quadratic polynomial $\ell=\tfrac\lambda2\cdot t^2+a\cdot t+b$ (it is called a \index{barrier}\emph{barrier}) that supports (locally) $s$ at $t_0$ from above;
that is, $\ell(t_0)\z= s(t_0)$ and $\ell(t)\ge s(t)$ for any $t$ (in a neighborhood of $t_0$)
\end{itemize}

To prove equivalence, approximate $f$ by smooth functions taking a convolutions $f_n=f*k_n$ for a suitable sequence of kernels $k_n$.
Note that all the conditions are equivalent for $f_n$;
passing to the limit we get the same for $f$.

\begin{thm}{Exercise}\label{ex:concave'}
Show that $\lambda$-concave functions are one-sided differentiable.
\end{thm}

The following exercise implies that if the function defined on an open interval, then the Lipschitz condition can be dropped from the definition of $\lambda$-concavity.

\begin{thm}{Exercise}\label{ex:concave-open}
Suppose a real-to-real function $f$ is defined on an open inerval and satisfies one the Jensen inequality stated above.
Show that $f$ is locally Lipscitz.
\end{thm}

\parbf{Functions on metric spaces.}
A function on a metric space $\spc{A}$ will usually mean a \textit{locally Lipschitz real-valued function defined on an open subset of $\spc{A}$}.
The domain of a function $f$ will be denoted by $\Dom f$.

We say that $f$ is \index{91@$\lambda$-concave function}\emph{$\lambda$-concave} (briefly $f''\le \lambda$) if
for any unit-speed geodesic $\gamma\:\II\z\to \Dom f$
the real-to-real function $t\mapsto f\circ\gamma(t)$ is $\lambda$-concave.

The following proposition is simple but conceptual ---
it reduces a global comparison to an infinitesimal condition on distance functions.

\begin{thm}{Proposition}\label{comp-kappa}
A complete geodesic space $\spc{A}$ is $\Alex0$ if and only if $f''\le 1$ for any function $f$ of the form
\[f\:x\mapsto \tfrac12\cdot\dist[2]{p}{x}{}.\] 
\end{thm} 

\parit{Proof.}
Choose a unit-speed geodesic $\gamma$ in $\spc{A}$ and two points $x=\gamma(t_0)$, $y=\gamma(t_1)$ for some $t_0<t_1$.
Consider the model triangle $\trig{\tilde p}{\tilde x}{\tilde y}\z=\modtrig(p x y)$.
Let $\tilde \gamma\:[t_0,t_1]\to\EE^2$ be the unit-speed parametrization of $[\tilde x \tilde y]$ from $\tilde x$ to $\tilde y$.

Set
\begin{align*} 
\tilde r(t)&\df\dist{\tilde p}{\tilde\gamma(t)}{},
& 
r(t)&\df\dist{p}{\gamma(t)}{}.
\end{align*}
Clearly, $\tilde r(t_0)=r(t_0)$ and $\tilde r(t_1)=r(t_1)$.
Note that the point-on-side comparison (\ref{point-on-side}) says that the implication
\[t_0\le t\le t_1
\qquad\Longrightarrow\qquad
\tilde r(t)\le r(t)
\eqlbl{eq:r=<r}\]
holds for any $\gamma$ and $t_0<t_1$.

Jensen's inequality for the function $h$ is equivalent to \ref{eq:r=<r}.
Hence the proposition follows.
\qeds

\section{Semiconcave functions}\label{sec:Semiconcave functions}

Recall that $\lambda$-concave functions were defined in Section \ref{Function comparison},
and when we say \textit{function} we usually mean a \textit{locally Lipschitz function defined on an open domain}.

Let $f$ be a locally Lipschitz real-valued function defined in an open subset $\Dom f$ of an Alexandrov space $\spc{A}$.
Suppose $\phi$ is a continuous function defined in $\Dom f$.
We will write $f''\le \phi$ if for any point $x\in \Dom f$ and any $\eps>0$ there is a neighborhood $U\ni x$ such that
the restriction $f|_U$ is $(\phi(x)+\eps)$-concave.

If $f''\le \phi$ for some continuous function $\phi$, then $f$ is called  \index{semiconcave function}\emph{semiconcave}.

\begin{thm}{Exercise}\label{ex:distfun-semiconcave}
Let $f$ be a \emph{distance function} on an $\Alex0$ space $\spc{A}$;
that is, $f(x)\equiv\dist{p}{x}{}$ for some $p\in \spc{A}$.
Show that $f''\le \tfrac1f$.
In particular, $f$ is semiconcave in $\spc{A}\setminus\{p\}$.
\end{thm}

Proposition~\ref{comp-kappa} admits the following generalization.
The is nearly the same, but the formulas are getting more complicated.

\begin{thm}{Proposition}
A complete geodesic space $\spc{A}$ is $\Alex{\mp1}$
if $f''\z\le \pm f$ for any function of the type $f=\cosh\circ\distfun_p$ (respectively, $f=-\cos\circ\distfun_p|_{\oBall(p,\pi)}$).
\end{thm}

The geometric meaning of these inequalities remains the same:
\textit{distance functions are more concave than distance functions in $\MM^2(\kappa)$}.

\section{Remarks}

Note that Alexandrov's lemma is a result in neutral geometry;
it has the following useful variation; see \cite[10.2]{alexander-kapovitch-petrunin2024} or \cite[3.3]{alexander-kapovitch-kirszbraun}.

\begin{thm}{Overlap lemma}\label{lem:extend-overlap}
Let $\tilde x^1$, $\tilde x^2$, $\tilde x^3$, $\tilde p^1$, $\tilde p^2$, ans $\tilde p^3$ be points in $\EE^2$, $\SSS^2$, or $\HH^2$.
Assume that, for any permutation $\{i,j,k\}$ of $\{1,2,3\}$, we have
\begin{enumerate}[(i)]

\item
\label{no-overlap:px=px}
$\dist{\tilde p^i}{\tilde x^\kay}{}=\dist{\tilde p^j}{\tilde x^\kay}{}$,
%$\dist{\tilde p^i}{\tilde x^\kay}{}=\dist{\tilde p^j}{\tilde x^\kay}{}$,

\item
\label{no-overlap:orient-1}
$\tilde p^i$ and $\tilde x^i$ lie in the same closed half-space determined by $[\tilde x^j\tilde x^\kay]$,
\end{enumerate}

If no pair of triangles $\trig{\tilde p^i}{\tilde x^j}{\tilde x^\kay}$ overlap,
then
\[\mangle{\tilde p^1} +\mangle {\tilde p^2}+\mangle{\tilde p^3}> 2\cdot\pi,\]
where $\mangle\tilde p^i\df\mangle\hinge{\tilde p^i}{\tilde x^\kay}{\tilde x^j}$
for a permutation $\{i,j,k\}$ of $\{1,2,3\}$.
\end{thm}

The condition \ref{SHORT.angle-b} in \ref{thm:angle-cbb} might be superfluous.
This is a long-standing open problem possibly dating back to Alexandrov \cite[footnote in 4.1.5]{burago-burago-ivanov}.
Let us state it formally.

\begin{thm}{Open question}\label{open:hinge-}
Let $\spc{A}$ be a complete geodesic space (you can also assume that $\spc{A}$ is homeomorphic to $\mathbb{S}^2$ or $\RR^2$)
such that for any hinge $\hinge x p y$ in $\spc{A}$,
the angle $\mangle\hinge x p y$ is defined and
\[\mangle\hinge x p y\ge\angk x p y.\]
Is it true that $\spc{A}$ is an Alexandrov space?
\end{thm}

Our 4-point comparison in Section~\ref{sec:4-point} is closely related to the so-called $\CAT$ comparison, which defines an \textit{upper} curvature bound in the sense of Alexandrov;
this is the subject of our previous  book  \cite{alexander-kapovitch-petrunin-2019}.

In both comparisons we check certain conditions on the 6 distances between pairs of points in a 4-point set.
Michael Gromov \cite[Section 1.19$_+$]{gromov1999} suggested considering other conditions of that type for $n$-point subsets;
see \cite{toyoda,lebedeva-petrunin-zolotov,lebedeva2019,petrunin2017,lebedeva-petrunin2024,lebedeva-petrunin2023,lebedeva-petrunin2021,lebedeva-petrunin2025,eskenazis-mendel-naor,gromov2001} for the development of this idea.

One coul define Alxandrov space as a complete \textit{length} space with curvature $\ge \kappa$.
This condition is more natural and general, but many statements can be reduced to the geodesic case.
In particular, suppose $\spc{A}$ is a complete length space with curvature $\ge \kappa$,
then 
\textit{$\spc{A}$ can be isometrically embedded into an $\Alex\kappa$ space} --- the ultrapower of $\spc{A}$; see \cite[4.11+8.4]{alexander-kapovitch-petrunin2024}.
Also, by Plaut's theorem, any point $p$ in $\spc{A}$ can be connected by geodesics to \textit{most} of points in $\spc{A}$
\cite[8.11]{alexander-kapovitch-petrunin2024}; compare to \ref{ex:grad-dist:geod}.

%%!TEX root = the-globalization.tex
\chapter{Globalization}\label{chap:globalization}

The globalization theorem states that a locally Alexandrov space is globally Alexandrov.
We start with the simplest meaningful case of this theorem and indicate a way to extend.

\section{Globalization}

A complete geodesic metric space $\spc{A}$ is \index{locally $\Alex0$}\emph{locally $\Alex0$} if any point $p\in\spc{A}$ admits a neighborhood $U\ni p$ such that the $\EE^2$-comparison holds for any quadruple of points in $U$.

\begin{thm}{Globalization theorem}\label{thm:glob} 
Any compact locally $\Alex0$ space is $\Alex0$.
\end{thm}

\parit{Proof modulo the key lemma.}
Note that condition \ref{angle-b} holds in $\spc{A}$ (the proof is the same).
It remains to check \ref{angle-a};
that is,
\[\mangle\hinge x p y\ge\angk x p y
\eqlbl{eq:mod-angle-CBB-comp-glob}\]
for any hinge $\hinge x p y$ in $\spc{A}$.

First note that \ref{eq:mod-angle-CBB-comp-glob} holds for hinges in a small neighborhood of any point;
this can be proved the same way as \ref{clm:angle-defined} and \ref{ex:adjacent-CBB}, applying the local version of the $\EE^2$-comparison.
Since $\spc{A}$ is compact, there is $\eps>0$ such that \ref{eq:mod-angle-CBB-comp-glob} holds if $\dist{x}{p}{}+\dist{p}{y}{}<\eps$.
Applying the key lemma several times we get that \ref{eq:mod-angle-CBB-comp-glob} holds for any given hinge.
\qeds

\begin{thm}{Key lemma}\label{key-lem:globalization} 
Let $\spc{A}$ be locally $\Alex0$. 
Assume that the comparison
\[\mangle\hinge x p q
\ge\angk x p q\]
holds for any hinge $\hinge x p q$ with 
$\dist{x}{y}{}+\dist{x}{q}{}
<
\frac{2}{3}\cdot\ell$.
Then the comparison
\[\mangle\hinge x p q
\ge\angk x p q\] 
holds for any hinge $\hinge x p q$ with $\dist{x}{ p}{}+\dist{x}{q}{}<\ell$.
\end{thm}

Let $\hinge x p q$ be a hinge in $\spc{A}$.
Denote by $\side \hinge x p q$ its \index{$\side \hinge x p q$ (model side)}\index{model!side}\emph{model side};
this is the opposite side in a flat triangle with the same angle and two adjacent sides as in $\hinge x p q$.

\begin{wrapfigure}{r}{44mm}
\centering
\includegraphics{mppics/pic-105}
\end{wrapfigure}

More precisely,
consider the model hinge $\hinge {\tilde x} {\tilde p} {\tilde q}$ in $\EE^2$ that is defined by 
\begin{align*}
\mangle\hinge {\tilde x} {\tilde p} {\tilde q}_{\EE^2}&=\mangle\hinge x p q_{\spc{A}},
\\
\dist{\tilde x} {\tilde p}{\EE^2}&=\dist{x} {p}{\spc{A}},
\\
\dist{\tilde x} {\tilde q}{\EE^2}&=\dist{x} {q}{\spc{A}};
\intertext{then}
\side \hinge x p q_{\spc{A}}
&\df
\dist{\tilde p}{\tilde q}{\EE^2}.
\end{align*}

Note that 
\[\side \hinge x p q \ge\dist{p}{q}{}
\quad\Longleftrightarrow\quad
\mangle\hinge x p q\ge \angk x p q.
\]
We will use it in the following proof.

\parit{Proof.} 
It is sufficient to prove the inequality
\[\side \hinge x p q
\ge\dist{p}{q}{}\eqlbl{eq:thm:=def-loc*}\] 
for any hinge $\hinge x p q$ with $\dist{x}{p}{}+\dist{x}{q}{}<\ell$.

Consider a hinge $\hinge x p q$ such that 
\[\tfrac{2}{3}\cdot\ell \le\dist{p}{x}{}\z+\dist{x}{q}{}< \ell.\]
First, let us construct a new hinge $\hinge{x'}p q$ with
\[
\dist{p}{x}{}+\dist{x}{q}{}\ge\dist{p}{x'}{}+\dist{x'}{q}{},
\eqlbl{eq:thm:=def-loc-fourstar}\]
such that 
\[\side \hinge x p q
\ge\side \hinge{x'}p q.
\eqlbl{eq:thm:=def-loc-fivestar}\]

\parit{Construction.}
Assume $\dist{x}{q}{}\ge\dist{x}{p}{}$; otherwise, switch the roles of $p$ and $q$.
Take $x'\in [x q]$ such that 
\[\dist{p}{x}{}+3\cdot\dist[{{}}]{x}{x'}{}
=\tfrac{2}{3}\cdot\ell. \eqlbl{3|xx'|}\]
Choose a geodesic $[x' p]$ and consider the  hinge $\hinge{x'}p q$ formed by $[x'p]$ and $[x' q]\subset [x q]$.
The triangle inequality implies \ref{eq:thm:=def-loc-fourstar}.
Further, note that 
\begin{align*}
\dist{p}{x}{}\z+\dist{x}{x'}{}&<\tfrac{2}{3}\cdot\ell,
&
\dist{p}{x'}{}\z+\dist{x'}{x}{}&<\tfrac{2}{3}\cdot\ell.
\end{align*}
In particular, 
\[\mangle\hinge x p{x'}
\ge\angk x p{x'}
\quad \text{and}\quad 
\mangle\hinge {x'}p x
\ge\angk {x'}p x.
\eqlbl{eq:thm:=def-loc-threestar}\]

{

\begin{wrapfigure}{r}{30 mm}
\vskip-0mm
\centering
\includegraphics{mppics/pic-820}
\vskip-4mm
\end{wrapfigure}

Now, let 
$\trig{\tilde x}{\tilde x'}{\tilde p}=\modtrig(x x' p)$.
Take $\tilde  q$ on the extension of $[\tilde  x\tilde  x']$ beyond $x'$ such that $\dist{\tilde x}{\tilde q}{}\z=\dist{x}{q}{}$ (and therefore $\dist{\tilde x'}{\tilde q}{}=\dist{x'}{q}{}$).
By~\ref{eq:thm:=def-loc-threestar},
\[\mangle\hinge x p q
=\mangle\hinge  x p{x'}\ge\angk x p{x'}\quad \Rightarrow\quad 
\side \hinge x q p\ge\dist{\tilde p}{\tilde q}{}.\]
Hence
\begin{align*}
\mangle\hinge{\tilde x'}{\tilde p}{\tilde q}&= 
\pi
-\angk{x'}p x
\ge
\\
&\ge
\pi-\mangle\hinge{x'}p x
=
\\
&=
\mangle\hinge{x'}p q,
\end{align*}
and \ref{eq:thm:=def-loc-fivestar} follows.

}

\medskip

Let us continue the proof.
Set $x_0=x$.
Let us apply inductively the above construction to get a sequence of hinges  $\hinge{x_n}p q$ with $x_{n+1}=x_n'$.
From \ref{eq:thm:=def-loc-fivestar}, we have that the sequence  $s_n\z=\side \hinge{x_n}p q$ is nonincreasing.
\begin{figure}[ht!]
\centering
\includegraphics{mppics/pic-825}
\end{figure}

The sequence might terminate at some $n$ only if $\dist{p}{x_n}{}+\dist{x_n}{q}{}\z< \tfrac{2}{3}\cdot\ell $.
In this case, by the assumptions of the lemma, $\side \hinge{x_n}p q\ge\dist{p}{q}{}$.
Since the sequence $s_n$ is nonincreasing, inequality \ref{eq:thm:=def-loc*} follows.

Otherwise, the sequence $r_n=\dist{p}{x_n}{}+\dist{x_n}{q}{}$ is nonincreasing, and $r_n\ge\tfrac{2}{3}\cdot\ell$ for all $n$.
Note that by construction, the distances
$\dist{x_n}{x_{n+1}}{}$, $\dist{x_n}{p}{}$, and $\dist{x_n}{q}{}$ are bounded away from zero for all large $n$.
Indeed, since on each step, we move $x_n$ toward to the point $p$ or $q$ that is further away, the distances $\dist{x_n}{p}{}$ and $\dist{x_n}{q}{}$ become about the same.
Namely, by \ref{3|xx'|}, we have that $\dist{p}{x_n}{}-\dist{x_n}{q}{}\le \tfrac29\cdot\ell$ for all large $n$.
Since $\dist{p}{x_n}{}+\dist{x_n}{q}{}\ge \tfrac23\cdot\ell$, we have $\dist{x_n}{p}{}\ge \tfrac\ell{100}$ and $\dist{x_n}{q}{}\ge \tfrac\ell{100}$.
Further, since $r_n\ge\tfrac{2}{3}\cdot\ell$, \ref{3|xx'|} implies that $\dist{x_n}{x_{n+1}}{}>\tfrac\ell{100}$.


Since the sequence $r_n$ is nonincreasing, it converges.
In particular, $r_n-r_{n+1}\to 0$ as $n\to\infty$.
It follows that $\angk{x_n}{p_n}{x_{n+1}}\to \pi$,
where $p_n=p$ if $x_{n+1}\in [x_nq]$, and otherwise $p_n=q$.
Since $\mangle\hinge{x_n}{p_n}{x_{n+1}}\ge\angk{x_n}{p_n}{x_{n+1}}$, we have
$\mangle\hinge{x_n}{p_n}{x_{n+1}}\to \pi$  as $n\to\infty$.

It follows that
\[r_n-s_n=\dist{p}{x_n}{}+\dist{x_n}{q}{}-\side \hinge{x_n}p q\to 0.\] 
Together with the triangle inequality
\[
\dist{p}{x_n}{}+\dist{x_n}{q}{}\ge\dist{p}{q}{}
\]
this yields
\[\lim_{n\to\infty}\side \hinge{x_n}p q\ge \dist{p}{q}{}.\]
Finally, the monotonicity of the sequence $s_n=\side \hinge{x_n}p q$ implies \ref{eq:thm:=def-loc*}.
\qeds

\section{General case}

The globalization theorem  can be generalized to any curvature bound $\kappa$.
The case $\kappa\le 0$ is proved in the same way, but the case $\kappa>0$ requires modifications.

The compactness condition in our version of the theorem can be traded for completeness.
The proof uses the following statement where $r(x)$ measures the size of a neighborhood of $x$ where the comparison holds.

\begin{thm}{Exercise}\label{ex:alm-min}
Let $\spc{X}$ be a complete metric space.
Suppose $r\:\spc{X}\to \RR$ is a positive continuous function.
Show that for any $\eps>0$ there is a point $p\in \spc{X}$ such that 
\[r(x)> (1-\eps)\cdot r(p)\] 
for any $x\in \cBall[p,\tfrac{1}{\eps}\cdot r(p)]$.

\end{thm}

This implies the following general version of the globalization theorem.

\begin{thm}{Theorem}\label{thm:globalization+}
Any locally $\Alex\kappa$ length space is $\Alex\kappa$.
\end{thm}

By \ref{ex:k-><mono}, we have
\[\angk x y z_{\MM^2(\kappa)}\le \angk x y z_{\MM^2(\Kappa)}\]
if $\kappa\le \Kappa$ and the right-hand side is defined.
It follows that a $\Alex\Kappa$ space is \textit{locally} $\Alex\kappa$.
Therefore, the globalization theorem implies the following.

\begin{thm}{Claim}\label{clm:K>k}
If $\Kappa>\kappa$, then any $\Alex\Kappa$ space is $\Alex\kappa$.
\end{thm}

In other words the expression \textit{curvature bounded below by $\kappa$} makes sense for geodesic spaces.
However, by the following exercise, it does not make much sense in general.

\begin{thm}{Exercise}\label{ex:CBB(1)notitCBB(0)}
Let $\spc{X}$ be the set $\{p,x_1,x_2,x_3\}$ with the metric defined by
\[\dist{p}{x_i}{}=\pi,\quad\dist{x_i}{x_j}{}=2\cdot\pi\]
for all $i\ne j$.
Show that $\spc{X}$ has curvature $\ge 1$, but does not have curvature $\ge 0$.
\end{thm}

\begin{thm}{Exercise}\label{ex:RisCBB(1)}
Let $p$ and $q$ be points in an $\Alex1$ space $\spc{A}$.
Suppose $\dist{p}{q}{}>\pi$.
Denote by $m$ the midpoint of $[pq]$.
Show that for any hinge $\hinge mxp$ we have
either $\mangle\hinge mxp=0$ or $\mangle\hinge mxp=\pi$.

Conclude that $\spc{A}$ is isometric to a line interval or a circle.

\end{thm}

\begin{thm}{Exercise}\label{ex:perim-k>0}
Suppose  
$\spc{A}$ is an $\Alex1$
and $\diam \spc{A}\le \pi$.
Show that 
\[\dist{x}{y}{}+\dist{y}{z}{}+\dist{z}{x}{}\le 2\cdot\pi\]
for any triple of points $x,y,z\in \spc{A}$.
\end{thm}


\section{Remarks}

The following question about \ref{angle-a} was stated in \cite[footnote in 4.1.5]{burago-burago-ivanov} but this is a long-standing open problem (possibly dating back to Alexandrov).

\begin{thm}{Open question}\label{open:hinge-}
Let $\spc{A}$ be a complete geodesic space (you can also assume that $\spc{A}$ is homeomorphic to $\mathbb{S}^2$ or $\RR^2$) 
such that for any hinge $\hinge x p y$ in $\spc{A}$, 
the angle $\mangle\hinge x p y$ is defined and 
\[\mangle\hinge x p y\ge\angk x p y.\]
Is it true that $\spc{A}$ is an Alexandrov space?
\end{thm}

The globalization theorem is also known as the \textit{generalized Toponogov theorem}.
Its two-dimensional case was proved by Paolo Pizzetti \cite{pizzetti};
later it was reproved independently by Alexandr Alexandrov \cite{alexandrov:devel}. %is it right ref?? 
Victor Toponogov \cite{toponogov-globalization+splitting} proved it for Riemannian manifolds of all dimensions.
For Alexandrov spaces of all dimensions, the theorem first appears in the paper of Michael Gromov, Yuriy Burago, and Grigory Perelman \cite{burago-gromov-perelman}.
Their statement is slightly more general than \ref{thm:globalization+}; it is for complete length spaces.
Another version for noncomplete, but geodesic spaces was proved by the second author \cite{petrunin:globalization}.


We took the proof from our book \cite{alexander-kapovitch-petrunin2024}, but reduced generality to compact nonnegatively curved spaces.
This proof is based on simplifications obtained by Conrad Plaut \cite{plaut:dimension} and Dmitry Burago, Yuriy Burago, and Sergei Ivanov \cite{burago-burago-ivanov}.
The same proof was rediscovered independently by Urs Lang and Viktor Schroeder \cite{lang-schroeder:globalization}.
Another simplified argument was found by Katsuhiro Shiohama \cite{shiohama}.





%%!TEX root = the-calculus.tex

\chapter{Calculus}\label{chap:derivative}

This lecture defines several notions related to the first-order derivatives in Alexandrov spaces;
this includes space of directions, tangent space, differential, and gradient.

\section{Space of directions} 
\label{sec:space+directions}

Let $\spc{A}$ be an Alexandrov space.
By \ref{ex:noncreasing}, the angle measure of any hinge in is defined.
Given $p\in \spc{A}$, consider the set $\mathfrak{S}_p$ of all nontrivial geodesics starting at $p$.
By \ref{claim:angle-3angle-inq}, the triangle inequality holds for $\mangle$ on $\mathfrak{S}_p$,
that is, $(\mathfrak{S}_p,\mangle)$ 
forms a \index{semimetric}\emph{semimetric} space;
that is, $\mangle$ behaves like a metric, but might vanish for distinct directions. 

The metric space corresponding to  $(\mathfrak{S}_p,\mangle)$ is called the \index{70@$\Sigma_p'$ (geodesic directions)}\index{space of geodesic directions}\emph{space of geodesic directions} at $p$, denoted by $\Sigma'_p$ or $\Sigma'_p\spc{A}$.
The elements of $\Sigma'_p$ are called \index{geodesic!direction}\emph{geodesic directions} at $p$.
Each geodesic direction is formed by an equivalence class of geodesics starting from $p$ 
for the equivalence relation 
\[[px]\sim[py]\quad \iff\quad \mangle\hinge pxy=0;\]
the direction of $[px]$ is denoted by $\dir px $.\index{40@$\dir{p}{q}$ (direction)}
By \ref{ex:0-angle}, 
\[[px]\sim[py]
\quad\iff\quad
[px]\subset [py]
\quad\text{or}\quad
[px]\supset[py].
\]
 
The completion of $\Sigma'_p$ is called the \index{space of directions}\emph{space of directions} at $p$ and is denoted by \index{70@$\Sigma_p$ (space of directions)}$\Sigma_p$ or $\Sigma_p\spc{A}$.
The elements of $\Sigma_p$ are called \index{direction}\emph{directions} at $p$.

\begin{thm}{Exercise}\label{ex:dir-compact}
Let $\spc{A}$ be an Alexandrov space.
Assume that a sequence of geodesics $[px_n]$ converge to a geodesic $[px_\infty]$ in the sense of Hausdorff,
and $x_\infty\ne p$.
Suppose $\Sigma_p$ is compact.
Show that $\dir p{x_n}\z\to\dir p{x_\infty}$ as $n\to\infty$.

\end{thm}


\section{Tangent space}\label{sec: tangent space}

The \index{65@$\Cone$}\index{cone}\emph{Euclidean cone} $\spc{V}=\Cone\spc{X}$
over a metric space $\spc{X}$
is defined as the metric space whose underlying set consists of
equivalence classes in
$[0,\infty)\times \spc{X}$ with the equivalence relation ``$\sim$'' given by $(0,p)\sim (0,q)$ for any points $p,q\in\spc{X}$,
and whose metric is given by the cosine rule
\[
\dist{(s,p)}{(t,q)}{\spc{V}} 
=
\sqrt{s^2+t^2-2\cdot s\cdot t\cdot \cos\theta},
\]
where $\theta= \min\{\pi, \dist{p}{q}{\spc{X}}\}$.

The leading example is
\[\Cone\SSS^n\iso\EE^{n+1};\]
here ``$\iso$'' stands for ``isometric to''. 
Now let us extend several notions from Euclidean space to Euclidean cones. 

The point in $\spc{V}$ that corresponds $(t,x)\z\in[0,\infty)\times \spc{X}$ will be denoted by $t\cdot x$.
The point in $\spc{V}$ formed by the equivalence class of $\{\0\}\times\spc{X}$ is called the \index{origin}\emph{origin} of the cone and is denoted by $\0$ or $\0_{\spc{V}}$.
For $v\in\spc{V}$ the distance $\dist{\0}{v}{\spc{V}}$ is called the \index{norm}\emph{norm} of $v$ and is denoted by $|v|$ or $|v|_{\spc{V}}$.
The \index{scalar product}\emph{scalar product} $\<v,w\>$
of $v=s\cdot p$ and $w=t\cdot q$
is defined by 
\[\<v,w\>
\df |v|\cdot|w|\cdot\cos\theta
\]
where $\theta= \min\{\pi, \dist{p}{q}{\spc{X}}\}$.
The value $\theta$ is undefined if $v=\0$ or $w=\0$;
in these cases we set $\<v,w\>\df0$.

\begin{thm}{Exercise}\label{ex:geodesic-cone}
Show that $\Cone\spc{X}$ is geodesic if and only if $\spc{X}$ is \index{91@$\ell$-geodesic space}\emph{$\pi$-geodesic};
that is, any two points $x,y\in \spc{X}$ such that $\dist{x}{y}{\spc{X}}<\pi$ can be joined by a geodesic in $\spc{X}$.
\end{thm}

\parbf{Tangent space.}
The Euclidean cone $\Cone\Sigma_p$ over the space of directions $\Sigma_p$ is called the \index{tangent space}\emph{tangent space} at $p$ and is denoted by \index{70@$\T_p$ (tangent space)}$\T_p$ or $\T_p\spc{A}$.
The elements of $\T_p\spc{A}$ will be called \index{tangent vector}\emph{tangent vectors} at $p$
(despite that $\T_p$ is only a cone --- not a vector space).
The space of directions $\Sigma_p$ can be (and will be) identified with the unit sphere in~$\T_p$;
that is, with the set $\set{v\in\T_p}{|v|=1}$.

\begin{thm}{Proposition}\label{prop:Tan-is-CBB(0)}
Any tangent space to an Alexandrov space has nonnegative curvature in the sense of Alexandrov.
\end{thm}

Halbeisen's example \cite{alexander-kapovitch-petrunin2024} shows that the tangent space $\T_p$ at some point of Alexandrov space might fail to be geodesic;
in this case $\T_p$ is \textit{not} $\Alex0$.

\parit{Proof.}
Consider the tangent space $\T_p=\Cone \Sigma_p$ of an Alexandrov space $\spc{A}$ at a point $p$.
We need to show that the $\EE^2$-comparison holds for a given quadruple $v_0$, $v_1$, $v_2$, $v_3\in \T_p$.

Recall that the space of geodesic directions $\Sigma_p'$ is dense in $\Sigma_p$.
It follows that the subcone $\T'_p=\Cone\Sigma_p'$ is dense in $\T_p$.
Therefore, it is sufficient to consider the case $v_0$, $v_1$, $v_2$, $v_3\in \T'_p$.

For each $i$, choose a geodesic $\gamma_i$ from $p$ in the direction of $v_i$;
reparametrize each $\gamma_i$ so that it has speed $|v_i|$.
Since the angles are defined, we have
\[\dist{\gamma_i(\eps)}{\gamma_j(\eps)}{\spc{A}}=\eps\cdot\dist{v_i}{v_j}{\T_p}+o(\eps)
\eqlbl{eq:gamma-v}\]
for $\eps>0$.
The quadruple $\gamma_0(\eps)$, $\gamma_1(\eps)$, $\gamma_2(\eps)$, $\gamma_3(\eps)$ meets the $\MM^2(\kappa)$-comparison.
After rescaling all the distances by $\tfrac1\eps$, it becomes the $\MM^2(\eps^2\cdot\kappa)$-comparison.
Passing to the limit as $\eps\to 0$ and applying \ref{eq:gamma-v}, we get the $\EE^2$-comparison for $v_0$, $v_1$, $v_2$, $v_3$.
\qeds


\begin{thm}{Exercise}\label{ex:GHto-tangent}
Let $p$ be a point in an Alexandrov space $\spc{A}$,
and let $\lambda_n\to\infty$.
Suppose $\Sigma_p$ is compact.
Show that there is a pointed Gromov--Hausdorff convergence $(\lambda_n\cdot \spc{A},p)\z\to (\T_p,0)$.
Moreover, for any geodesic $\gamma$ that starts at $p$, we have
\[\iota_n\circ\gamma(t/\lambda_n)\to t\cdot \gamma^+(0),\]
where $\iota_n$ sends a point in $\spc{A}$ to the corresponding point in $\lambda_n\cdot\spc{A}$.
\end{thm}

\section{Semiconcave functions}\label{sec:Semiconcave functions}

Recall that $\lambda$-concave functions were defined in Section \ref{Function comparison},
and when we say \textit{function} we usually mean a \textit{locally Lipschitz function defined on an open domain}.

Let $f$ be a locally Lipschitz real-valued function defined in an open subset $\Dom f$ of an Alexandrov space $\spc{A}$.
Suppose $\phi$ is a continuous function defined in $\Dom f$.
We will write $f''\le \phi$ if for any point $x\in \Dom f$ and any $\eps>0$ there is a neighborhood $U\ni x$ such that 
the restriction $f|_U$ is $(\phi(x)+\eps)$-concave.


If $f''\le \phi$ for some continuous function $\phi$, then $f$ is called  \index{semiconcave function}\emph{semiconcave}.


\begin{thm}{Exercise}\label{ex:distfun-semiconcave}
Let $f$ be a \emph{distance function} on an $\Alex0$ space $\spc{A}$;
that is, $f(x)\equiv\dist{p}{x}{}$ for some $p\in \spc{A}$.
Show that $f''\le \tfrac1f$.
In particular, $f$ is semiconcave in $\spc{A}\setminus\{p\}$.
\end{thm}


\section{Differential}\label{sec:differential}
\index{differential of a function}

Let $f$ be a semiconcave function on an Alexandrov space $\spc{A}$, and $p\z\in \Dom f$.
Choose a unit-speed geodesic $\gamma$ that starts at $p$;
let $\xi\in\Sigma_p$ be its direction.
Define 
\[(\dd_pf)(\xi)\df(f\circ\gamma)^+(0),\]
here $(f\circ\gamma)^+$ denotes the \index{right derivative}\emph{right derivative} of $(f\circ\gamma)$;
it is defined since $f$ is semiconcave.

By the following exercise, the value $(\dd_pf)(\xi)$ is defined; that is, it does not depend on the choice of $\gamma$.
Moreover, $\dd_pf$ is a Lipschitz function on $\Sigma'_p$.
It follows that the function $\dd_pf\:\Sigma_p'\to\RR$ can be uniquely extended to a Lipschitz function $\dd_pf\:\Sigma_p\to\RR$.
Further, we can extend it to the tangent space by setting 
\[(\dd_pf)(r\cdot \xi)
\df
r\cdot (\dd_pf)(\xi)\]
for any $r\ge 0$ and $\xi\in\Sigma_p$.
The obtained function $\dd_pf\:\T_p\to\RR$ is Lipschitz;
it is called the \index{differential}\emph{differential} of $f$ at $p$.

\begin{thm}{Exercise}\label{ex:df(xi)}
Let $f$ be a semiconcave function on an Alexandrov space.
Suppose $\gamma_1$ and $\gamma_2$ are geodesics that start at $p\z\in \Dom f$;
denote by $\theta$ the angle between $\gamma_1$ and $\gamma_2$ at $p$.
Show that 
\[|(f\circ\gamma_1)^+(0)-(f\circ\gamma_2)^+(0)|\le L\cdot \theta,\]
where $L$ is the Lipschitz constant of $f$ in a neighborhood of $p$.
\end{thm}

\begin{thm}{Exercise (First variation formula)} \label{ex:d(distfun)}
Let $p$ and $q$ be distinct points in an Alexandrov space~$\spc{A}$.
Show the following.

\begin{subthm}{ex:d(distfun):<}
$\dd_p\distfun_q(v)\le -\langle\dir pq,v\rangle$
for any $v\in\T_p$.
\end{subthm}

\begin{subthm}{ex:d(distfun):=}
Suppose $\spc{A}$ is proper.
Let $\Uparrow_p^q$ be the set of all direction of  geodesics from $p$ to $q$.
Then
\[\dd_p\distfun_q(v)=-\max_{\xi\in\Uparrow_p^q}\langle\xi,v\rangle\]
for any $v\in\T_p$.
\end{subthm}

\end{thm}

\section{Gradient}\label{sec:grad-def}

The following definition generalizes the gradient to semiconcave functions on Alexandrov space.
This generalization is not trivial even for concave functions on Euclidean space;
we suggest keeping this example in mind while reading further.

\begin{thm}{Definition}\label{def:grad} 
Let $f$ be a semiconcave function on an Alexandrov space.
A tangent vector $g\in \T_p$ is called a 
\index{gradient}\emph{gradient} of $f$ at $p$ 
(briefly,  $g\z=\nabla_p f$\index{19@$\nabla$ (gradient)}) if
\begin{subthm}{}
$(\dd_p f)(w)\le \<g,w\>$ for any $w\in \T_p$, and
\end{subthm}

\begin{subthm}{}
$(\dd_p f)(g) = \<g,g\>.$
\end{subthm}
\end{thm}

The following exercise provides a property of gradients that will play a key role in the proof of the first distance estimate (\ref{thm:dist-est}).

\begin{thm}{Exercise}\label{ex:monotonicity}
Let $f$ be a $\lambda$-concave function on an Alexandrov space.
Suppose that gradients $\nabla_xf$ and $\nabla_yf$ are defined.
Show that 
\[\<\dir{x}{y},\nabla_{x}f\>
+
\<\dir{y}{x},\nabla_{y}f\>
+
\lambda\cdot\dist{x}{y}{}\ge 0.\]
\end{thm}

\begin{figure}[ht!]
\centering
\includegraphics{mppics/pic-409}
\end{figure}

\begin{thm}{Proposition}\label{prop:grad-exist}
Suppose that a semiconcave function $f$ is defined in a neighborhood of a point $p$ in an Alexandrov space.
Then the gradient $\nabla_pf$ is uniquely defined.

Moreover, if $\dd_pf\le 0$, then we have $\nabla_pf=0$;
otherwise, $\nabla_pf\z=s\cdot \overline{\xi}$, where 
$s= \dd_pf(\overline{\xi})$
and
$\overline{\xi}\in \Sigma_p$ is the direction that maximize the value $\dd_pf(\xi)$ for $\xi\in \Sigma_p$.
\end{thm}


\begin{thm}{Key lemma}\label{lem:ohta} 
Let $f$ be a semiconcave function that is defined in a neighborhood of a point $p$
in an Alexandrov space $\spc{A}$. 
Then for any $u,v\in \T_p$, we have
\[s\cdot \sqrt{|u|^2+2\cdot\<u,v\> +|v|^2}
\ge 
(\dd_p f)(u)+(\dd_p f)(v),\]
where
\[s=\sup\set{(\dd_p f)(\xi)}{\xi\in\Sigma_p}.\]

\end{thm}

If $\T_p\iso\EE^m$ and $\dd_p f$ is a concave function,
then
\[2\cdot(\dd_p f)(\tfrac{u+v}2)\ge(\dd_p f)(u)+(\dd_p f)(v).\]
The latter implies the statement since $|u+v|=\sqrt{|u|^2+2\cdot\<u,v\> +|v|^2}$.
In general, $\T_p$ is not geodesic (and not even a length space), so concavity of $\dd_p f$ does not make  sense.
The key lemma however says  that in a certain sense $\dd_p f$ behaves as a concave function.

Solving the following exercise should help to find an approach to the key lemma.

\begin{thm}{Exercise}\label{ex:d(distfun):==}
Let $p$ and $q$ be distinct points in an Alexandrov space $\spc{A}$.
Suppose the geodesic $[pq]$ can be extended beyond $q$.

Show that
\[\dd_p\distfun_q(v)= -\langle\dir pq,v\rangle\]
for any $v\in\T_p$.
\end{thm}

\parit{Proof of \ref{lem:ohta}.}
We will assume that $\spc{A}$ is $\Alex0$ and $f$ is concave;
the general case requires only minor modifications.
We can assume that $v\ne 0$, $w\ne 0$, and $\alpha=\mangle(u,v)>0$; otherwise, the statement is trivial.

{

\begin{wrapfigure}{r}{34 mm}
\vskip-4mm
\centering
\includegraphics{mppics/pic-1205}
\vskip0mm
\end{wrapfigure}

Consider a model configuration of five points: $\tilde p$, $\tilde u$, $\tilde v$, $\tilde q$, $\tilde w\in\EE^2$ such that
\begin{itemize}
\item $\mangle\hinge{\tilde p}{\tilde u}{\tilde v}=\alpha$, 
\item $\dist{\tilde p}{\tilde u}{}=|u|$, 
\item $\dist{\tilde p}{\tilde v}{}=|v|$,
\end{itemize}
}
\begin{itemize}
\item $\tilde q$ lies on an extension of $[\tilde p\tilde v]$ so that $\tilde v$ is the midpoint of $[\tilde p\tilde q]$, 
\item $\tilde w$ is the midpoint between $\tilde u$ and ${\tilde v}$.
\end{itemize}
Note that 
\[\dist{\tilde p}{\tilde w}{}
=
\tfrac{1}{2}\cdot\sqrt{|u|^2+2\cdot\<u,v\>+|v|^2}.\eqlbl{eq:|p-w|=}\]

Since the geodesic space of directions $\Sigma'_p$ is dense in $\Sigma_p$,
we can assume that there are geodesics in the directions of $u$ and $v$.
Choose such geodesics $\gamma_u$ and $\gamma_v$ and assume that they are parametrized with speed $|u|$ and $|v|$ respectively.
For all small $t>0$, consider points $u_t,v_t,q_t,w_t\in \spc{A}$ such that
\begin{itemize}
\item $v_t=\gamma_v(t)$,\quad  $q_t=\gamma_v(2\cdot t)$
\item $u_t=\gamma_u(t)$.
\item $w_t$ is the midpoint of $[u_t v_t]$.
\end{itemize}
Clearly 
\[\dist{p}{u_t}{}=t\cdot |u|,\qquad \dist{p}{v_t}{}=t\cdot|v|,\qquad \dist{p}{q_t}{}=2\cdot t\cdot|v|.\] 
Since $\mangle(u,v)$ is defined, 
we have 
\[\dist{u_t}{v_t}{}=t\cdot\dist{\tilde u}{\tilde v}{}+o(t),
\qquad
\dist{u_t}{q_t}{}=t\cdot\dist{\tilde u}{\tilde q}{}+o(t).\]

From the point-on-side and hinge comparisons (\ref{point-on-side}$+$\ref{angle}), we have
\[\angk{v_t}p{w_t}
\ge
\angk{v_t}p{u_t}
\ge
\mangle\hinge{\tilde v}{\tilde p}{\tilde u}+\tfrac{o(t)}t\]
and
\[\angk{v_t}{q_t}{w_t}
\ge
\angk{v_t}{q_t}{u_t}
\ge
\mangle\hinge{\tilde v}{\tilde q}{\tilde u}+\tfrac{o(t)}t.\]
Clearly, 
$\mangle\hinge{\tilde v}{\tilde p}{\tilde u}+\mangle\hinge{\tilde v}{\tilde q}{\tilde u}=\pi$. 
From the adjacent angle comparison (\ref{2-sum}), 
$\angk{v_t}p{u_t}\z+\angk{v_t}{u_t}{q_t}\le \pi$.
Hence
$\angk{v_t}p{w_t}
\to
\mangle\hinge{\tilde v}{\tilde p}{\tilde w}$ as $t\to0+$
and thus 
\[\dist{p}{w_t}{}=t\cdot\dist{\tilde p}{\tilde w}{}+o(t).\]

Without loss of generality, we can assume that $f(p)=0$.
Since $f$ is concave, we have 
\begin{align*}
2\cdot f(w_t)&\ge f(u_t)+f(v_t)=
\\
&=t\cdot [(\dd_p f)(u)+(\dd_p f)(v)]+o(t).
\end{align*}
 
Applying concavity of $f$, we have
\begin{align*}
(\dd_p f)(\dir p{w_t})
&\ge 
\frac{f(w_t)}{\dist{p}{w_t}{}}
\ge 
\\
&\ge
\frac{t\cdot[(\dd_p f)(u)+(\dd_p f)(v)]+o(t)}{2\cdot t\cdot\dist{\tilde p}{\tilde w}{}+o(t)}.
\end{align*}
By \ref{eq:|p-w|=}, the key lemma follows.
\qeds

\parit{Proof of \ref{prop:grad-exist}; uniqueness.} 
If $g,g'\in \T_p$ are two gradients of $f$,
then 
\begin{align*}
\<g,g\>
&=(\dd_p f)(g)\le \<g,g'\>,
&
\<g',g'\>
&=(\dd_p f)(g')\le \<g,g'\>.
\end{align*}
Therefore,
\[\dist[2]{g}{g'}{}=\<g,g\>-2\cdot\<g,g'\>+\<g',g'\>\le0.\] 
It follows that $g=g'$.

\parit{Existence.} 
If $\dd_p f\le 0$, then one can take $\nabla_p f=\0$.

Suppose $s=\sup\set{(\dd_p f)(\xi)}{\xi\in\Sigma_p}>0$, 
it is sufficient to show that there is  $\overline{\xi}\in \Sigma_p$ such that 
\[
(\dd_p f)\left(\overline{\xi}\right)=s.
\eqlbl{overlinexi}
\]
Indeed, suppose $\overline{\xi}$ exists.
Applying \ref{lem:ohta} for $u=\overline{\xi}$, $v=\eps\cdot w$ with $\eps\to0+$, 
we get
\[(\dd_p f)(w)\le \<w,s\cdot\overline{\xi}\>\] 
for any $w\in\T_p$;
that is, $s\cdot\overline{\xi}$ is the gradient at $p$.

Take a sequence of directions $\xi_n\in \Sigma_p$, such that $(\dd_p f)(\xi_n)\to s$.
Applying \ref{lem:ohta} for $u=\xi_n$ and $v=\xi_m$, we get
\[s
\ge
\frac{(\dd_p f)(\xi_n)+(\dd_p f)(\xi_m)}{\sqrt{2+2\cdot\cos\mangle(\xi_n,\xi_m)}}.\]
Therefore $\mangle(\xi_n,\xi_m)\to0$ as $n,m\to\infty$;
that is, $\xi_1,\xi_2,\dots$ is a Cauchy sequence.
Clearly, $\overline{\xi}=\lim_n\xi_n$ meets \ref{overlinexi}.
\qeds

\begin{thm}{Exercise}\label{ex:convergence-grad}
Let $f$ and $g$ be locally Lipschitz semiconcave functions defined in a neighborhood of a point $p$ in an Alexandrov space.
Show that 
\[\dist[2]{\nabla_p f}{\nabla_p g}{\T_p}
\le 
s\cdot(|\nabla_p f|+|\nabla_p g|),\]
where
\[s=\sup\set{|(\dd_p f)(\xi)-(\dd_p g)(\xi)|}{\xi\in\Sigma_p}.\]

Conclude that if the sequence of restrictions $\dd_p f_n|_{\Sigma_p}$ converges uniformly, then $\nabla_pf_n$ converges as $n\to\infty$.
Here we assume that all functions $f_1$, $f_2,\dots$ are semiconcave and locally Lipschitz. 
\end{thm}

\begin{thm}{Exercise}\label{ex:semicontinuous-grad}
Let $f$ be a locally Lipschitz $\lambda$-concave function on an Alexandrov space $\spc{A}$.

\begin{subthm}{ex:semicontinuous-grad:>s}
Suppose $s\ge 0$.
Show that $|\nabla_xf|> s$ if and only if for some point $y$ we have
\[f(y)-f(x)>s\cdot \ell+\lambda\cdot \tfrac{\ell^2}2,\]
were $\ell=\dist{x}{y}{}$.
\end{subthm}

\begin{subthm}{ex:semicontinuous-grad:lim} Show that $x\mapsto|\nabla_xf|$ is lower semicontinuous;
that is,
\[|\nabla_{x_\infty}f|\le \liminf_{x_n\to x_\infty} |\nabla_{x_n}f|.\]

\end{subthm}

\end{thm}

%%!TEX root = the-gradient-flow.tex

\chapter{Gradient flow}\label{chap:GF}

\section{Velocity of curve}

Let $\alpha$ be a curve in an Alexandrov space $\spc{L}$.
If for any choice of 
geodesics $[p\,\alpha(t_0+\eps)]$ the vectors 
\[\tfrac{1}{\eps}\cdot\dist{p}{\alpha(t_0+\eps)}{}\cdot\dir p{\alpha(t_0+\eps)}\]
converge as $\eps\to 0+$, then their limit in $\T_p$ is called the \index{right derivative}\emph{right derivative} of $\alpha$ at $t_0$; it will be denoted by $\alpha^+(t_0)$.
In addition, $\alpha^+(t_0)\df0$
if $\tfrac{1}{\eps}\cdot\dist{p}{\alpha(t_0+\eps)}{}\to 0$ as $\eps\to 0+$.

The tangent vector $v=\dist px{}\cdot\dir px$ can be called \index{logarithm}\emph{logarithm} of $x$ at $p$ (briefly, \index{$v=\log_p x$ (logarithm)}$\log_p x$);
it is a tangent vector at $p$ of a geodesic path from $p$ to $x$.\label{page:log}


\begin{thm}{Claim}\label{clm:fa'=dfa'}
Let $\alpha$ be a curve in an Alexandrov space $\spc{L}$.
Suppose $f$ a semiconcave Lipschitz function
defined in a neighborhood of $p\z=\alpha(0)$,
and $\alpha^+(0)$ is defined.
Then 
\[(f\circ\alpha)^+(0)
=
(\dd_pf)(\alpha^+(0)).\]
\end{thm}

\parit{Proof.}
Without loss of generality, we can assume that $f(p)=0$.
Suppose $f$ and therefore $\dd_pf$ are $L$-Lipschitz.

Choose a constant-speed geodesic $\gamma$ that starts from $p$,
such that the distance
$s=\dist{\alpha^+(0)}{\gamma^+(0)}{\T_p}$
is small.
By the definition of differential,
\[(f\circ\gamma)^+(0)=\dd_pf(\gamma^+(0)).\]

By comparison and the definition of $\alpha^+$,
\[\dist{\alpha(\eps)}{\gamma(\eps)}{\spc{L}}\le s\cdot\eps+o(\eps)\]
for $\eps>0$.
Therefore,
\[|f\circ\alpha(\eps)-f\circ\gamma(\eps)|\le L\cdot s\cdot\eps+o(\eps).\]

Suppose $(f\circ\alpha)^+(0)$ is defined.
Then
\[|(f\circ\alpha)^+(0)-(f\circ\gamma)^+(0)|\le L\cdot s.\]
Since $\dd_pf$ is $L$-Lipschitz, we also get 
\[|\dd_pf(\alpha^+(0))-\dd_pf(\gamma^+(0))|\le L\cdot s.\]
It follows that the needed identity holds up to error $2\cdot L\cdot s$.
The statement follows since $s>0$ can be chosen arbitrarily.

The same argument is applicable if in the place of $(f\circ\alpha)^+(0)$
we use any limit of $\tfrac1{\eps_n}\cdot [f\circ\alpha(\eps_n)-f(p)]$ for a sequence $\eps_n\to 0+$.
It proves that all such limits coincide; in particular, $(f\circ\alpha)^+(0)$ is defined and equals to $(\dd_pf)(\alpha^+(0))$.
\qeds


\section{Gradient curves}

\begin{thm}{Definition}\label{def:grad-curve}
Let $f\:\spc{L}\subto\RR$ be a locally Lipschitz and semiconcave function on an Alexandrov space
$\spc{L}$.

A locally Lipschitz curve $\alpha\:[t_{\min},t_{\max})\to\Dom f$ will be called an \index{gradient curve}\emph{$f$-gradient curve} if
\[\alpha^+=\nabla_{\alpha} f;\]
that is, for any $t\in[t_{\min},t_{\max})$, $\alpha^+(t)$ is defined and 
$\alpha^+(t)=\nabla_{\alpha(t)} f$.
\end{thm}

A complete proof of the following theorem is given in \cite{alexander-kapovitch-petrunin2024}; 
it mimics the proof of the standard Picard theorem on the existence  and uniqueness of solutions of ordinary differential equations.
We omit the proof of existence as it is rather lengthy;
the uniqueness will be proved in the next section.


\begin{thm}{Picard theorem}\label{thm:glob-exist-grad-curv}
Let $f\:\spc{L}\subto \RR$ be a locally Lipschitz and $\lambda$-concave function on an Alexandrov space $\spc{L}$.
Then for any $p\in \Dom f$, there are unique $t_{\max}\in(0,\infty]$ and $f$-gradient curve $\alpha\:[0,t_{\max})\to \spc{L}$ with $\alpha(0)=p$ such that any sequence $t_n\to t_{\max}-$, the sequence $\alpha(t_n)$ does not have a limit point in $\Dom f$.
\end{thm}

Note that the theorem says that the future of a gradient curve is determined by its present, but it says nothing about its past.

Here is an example showing that the past is not determined by the present.
Consider the function $f\:x\mapsto -|x|$ on the real line $\RR$.
The tangent space $\T_x\RR$ can be identified with $\RR$.
Note that 
\[\nabla_xf=
\begin{cases}
1&\text{if}\quad x<0,
\\
0&\text{if}\quad x=0,
\\
-1&\text{if}\quad x>0.
\end{cases}
\]
So, the $f$-gradient curves go to the origin with unit speed and then stand there forever.
In particular, if $\alpha$ is an $f$-gradient curve that starts at $x$,
then $\alpha(t)=0$ for any $t\ge |x|$.

Here is a slightly more interesting example;
it shows that gradient curves can merge even in the region where $|\nabla f|\z\ne 0$. 


\begin{wrapfigure}[8]{r}{34 mm}
\vskip-0mm
\centering
\includegraphics{mppics/pic-1215}
\vskip0mm
\end{wrapfigure}

\begin{thm}{Example}
Consider the function $f\:(x,y)\mapsto-|x|-|y|$ on the $(x,y)$-plane.
Note that $f$ is concave;
its gradient field is sketched on the figure.

Let $\alpha$ be an $f$-gradient curve that starts at $(x,y)$ for $x>y>0$.
Then 
\[\alpha(t)=
\begin{cases}
(x-t,y-t) &\text{for}\quad 0\le t\le  x-y,
\\
(x-t,0) &\text{for}\quad x-y\le t\le  x,
\\
(0,0) &\text{for}\quad x\le t.
\end{cases}
\]

\end{thm}


\section{Distance estimates}

\begin{thm}{Observation}\label{eq:fist-var-inq+}
Let $\alpha$ be a gradient curve of a $\lambda$-concave function $f$ 
defined on an Alexandrov space.
Choose a point $p$; let $\ell(t)\df\distfun_p\circ\alpha(t)$ and $q=\alpha(t_0)$.
Then 
\[
\ell^+(t_0)\le -\left({f(p)}-{f(q)}-\tfrac\lambda2\cdot\ell^2(t_0)\right)/\ell(t_0)
\]
\end{thm}

\parit{Proof.}
Let $\gamma$ be the unit-speed parametrization of $[qp]$ from $q$ to $p$, so $q=\gamma(0)$.
Then 
\begin{align*}
\ell^+(t_0)&=(\dd_q\distfun_p)(\nabla_qf)\le
\\
&\le -\langle\dir qp,\nabla_qf\rangle \le
\\
&\le -\dd_qf(\dir qp)=
\\
&=-(f\circ\gamma)^+(0)\le
\\
&\le -\left({f(p)}-{f(q)}-\tfrac\lambda2\cdot\ell^2(t_0)\right)/\ell(t_0)
\end{align*}
In the above calculations we consequently applied
\ref{clm:fa'=dfa'},
\ref{ex:d(distfun)},
the definition of gradient,
the definition of differential,
and concavity of $t\z\mapsto f\circ\gamma(t)-\tfrac \lambda2\cdot {t^2}$.
\qeds

Note that the following estimate implies uniqueness in the Picard theorem (\ref{thm:glob-exist-grad-curv}).

\begin{thm}{First distance estimate}\label{thm:dist-est}
Let $f$ be a $\lambda$-concave locally Lipschitz function on an Alexandrov space $\spc{L}$.
Then
\[\dist{\alpha(t)}{\beta(t)}{}
\le 
e^{\lambda\cdot t}\cdot\dist[{{}}]{\alpha(0)}{\beta(0)}{}\]
for any $t\ge 0$ and any two $f$-gradient curves $\alpha$ and $\beta$.

Moreover, the statement holds for a locally Lipschitz $\lambda$-concave function defined in an open domain if there is a geodesic $[\alpha(t)\,\beta(t)]$ in $\Dom f$ for any~$t$.
\end{thm}

\parit{Proof.} 
Fix a choice of geodesic $[\alpha(t)\,\beta(t)]$ for each $t$.
Let $\ell(t)=\dist{\alpha(t)}{\beta(t)}{}$. 
Note that
\[\ell^+(t)
\le-
\<\dir{\alpha(t)}{\beta(t)},\nabla_{\alpha(t)}f\>-\<\dir{\beta(t)}{\alpha(t)},\nabla_{\beta(t)}f\>
\le
\lambda\cdot\ell(t).\]
Here one has to apply \ref{eq:fist-var-inq+} for distance to the midpoint $m$ of $[\alpha(t)\,\beta(t)]$, and then apply the triangle inequality.
Hence the result. 
\qeds



The following exercise describes a global geometric property of a gradient curve without direct reference to its function.
It uses the notion of \textit{self-contracting curves} introduced by Aris Daniilidis, Olivier Ley, and St\'ephane Sabourau \cite{daniilidis-ley-sabourau}.

\begin{thm}{Exercise}\label{ex:elf-contracting}
Let $f\:\spc{L}\subto\RR$ be a locally Lipschitz and concave function on an Alexandrov space
$\spc{L}$.
Then 
\[\dist{\alpha(t_1)}{\alpha(t_3)}{\spc{L}}\ge \dist{\alpha(t_2)}{\alpha(t_3)}{\spc{L}}.\]
for any $f$-gradient curve $\alpha$ and $t_1\le t_2\le t_3$.
\end{thm}

\begin{thm}{Exercise}\label{ex:mayer}
Let $f$ be a locally Lipschitz concave function defined on an Alexandrov space $\spc{L}$.
Suppose $\hat\alpha\:[0,\ell]\to\spc{L}$ is an arc-length reparametrization of an $f$-gradient curve.
Show that $(f\circ\hat\alpha)$ is concave.
\end{thm}




The following exercise implies that gradient curves for a uniformly converging sequence of $\lambda$-concave functions converge to the gradient curves of the limit function.

\begin{thm}{Exercise}\label{lem:fg-dist-est}
Let $f$ and $g$ be $\lambda$-concave locally Lipschitz functions on an Alexandrov space $\spc{L}$.
Suppose
$\alpha,\beta\:[0,t_{\max})\to \spc{L}$ are respectively $f$- and $g$-gradient curves.
Assume $|f-g|<\eps$; let $\ell\:t\mapsto\dist{\alpha(t)}{\beta(t)}{}$.
Show that
\[\ell^+\le \lambda\cdot\ell+\tfrac{2\cdot\eps}{\ell}.\]

Conclude that if $\alpha(0)=\beta(0)$ and $t_{\max}<\infty$, then
\[\dist{\alpha(t)}{\beta(t)}{}
\le
\Const\cdot\sqrt{\eps\cdot t}\]
for some constant $\Const=\Const(t_{\max},\lambda)$.
\end{thm}

\section{Gradient flow}

Let $\spc{L}$ be an Alexandrov space 
and $f$ be a locally Lipschitz semiconcave function defined on an open subset of $\spc{L}$.
If there is an $f$-gradient curve $\alpha$ such that $\alpha(0)=x$ and $\alpha(t)=y$,
then we will write 
\[\GF^t_f(x)=y.\]
The partially defined map $\GF^t_f$ from $\spc{L}$ to itself is called the \index{gradient flow}\emph{$f$-gradient flow} for time $t$.
Note that
\[\GF^{t_1+t_2}_f=\GF_f^{t_1}\circ\GF_f^{t_2}.\]
In other words, one may think that gradient flow is an action of the \textit{semigroup} $(\RR_{\ge0},+)$ on the space.
 
From the first distance estimate \ref{thm:dist-est}, 
it follows that for any $t\ge 0$, the domain of definition of $\GF^t_f$ is an open subset of $\spc{L}$.
In some cases, it is globally defined.
For example, if $f$ is a $\lambda$-concave function defined on the whole space $\spc{L}$, then $\GF^t_f(x)$ is defined for all $x\in \spc{L}$ and $t\ge0$;
see \cite[16.19]{alexander-kapovitch-petrunin2024}.

Now let us reformulate the statements about gradient curves obtained earlier using this new terminology.
From the first distance estimate, we have the following.

\begin{thm}{Proposition}\label{prop:GF-is-lip}
Let $\spc{L}$ be an Alexandrov space 
and $f\:\spc{L}\to \RR$ be a semiconcave function.
Then the map $x\mapsto\GF^t_f(x)$ is locally Lipschitz.

Moreover, if $f$ is $\lambda$-concave, then $\GF^t_f$ is $e^{\lambda\cdot t}$-Lipschitz.
\end{thm}

The next proposition follows from \ref{lem:fg-dist-est}.

\begin{thm}{Proposition}\label{grad-curve-conv}
Let $\spc{L}$ be an Alexandrov space.
Suppose $f_n\:\spc{L}\to\RR$ is a sequence of
$\lambda$-concave functions 
that converges to $f_\infty\:\spc{L}\to \RR$. 
Then for any $x\in \spc{L}$ and $t\ge 0$, we have
\[\GF_{f_n}^t(x)\to \GF_{f_\infty}^t(x)\]
as $n\to \infty$.
\end{thm}

%??? do we need GH-limit version???

\section{Gradient exponent}\label{gexp}

One of the technical difficulties in Alexandrov's geometry comes from
nonextendability of geodesics. 
In particular, the exponential map, $\exp_p\:\T_p\to \spc{L}$, if defined in the usual way, can
be undefined in an arbitrary small neighborhood of the origin. 

We construct its analog, the \index{gradient exponential map}\emph{gradient exponential map} 
\[\gexp_p\:\T_p\to\spc{L},\]
which essentially solves this problem. 
It shares many properties with the ordinary exponential map, and better in certain respects,
even in the Riemannian universe.

Let $\spc{L}$ be Alexandrov's space and $p\in \spc{L}$, consider the function $f\z=\distfun_p^2/2$.
Suppose $i_{\lambda}\:\lambda\cdot \spc{L}\to \spc{L}$ sends a point in the rescaled copy $\lambda\cdot\spc{L}$ to the corresponding point in $\spc{L}$.
Consider the one parameter family of maps
$$\Phi^t_{f}\circ i_{e^t}\:e^t{\cdot} \spc{L}\to \spc{L}$$
where $\Phi^t_{f}$ denotes gradient flow. 
Note that $(e^t{\cdot} \spc{L},p)\GHto (\T_p,o_p)$ as $t\to\infty$.
Let us define the \textit{gradient exponential map} as the limit
\[\gexp_p=\lim_{t\to\infty}\Phi^t_{f}\circ i_{e^t}.\]

\begin{thm}{Proposition}\label{prop:gexp}
Let $\spc{L}$ be a proper $\Alex0$ space.
Then for any $p\in \spc{L}$ the gradient exponent $\gexp_p\:\T_p\to\spc{L}$ is defined.
Moreover, $\gexp_p$ is a short map and 
\[\gexp_p(\gamma^+(0))=\gamma(1)\]
for any geodesic path $\gamma$ that starts at $p$.
\end{thm}

The last statement in the proposition says that it is appropriate to use term \textit{exponent} for $\gexp$.


\parit{Proof.} 
Note that $f''\le 1$.
By the first distance estimate, we have that $\Phi^t_{f}$ is an $e^t$-Lipschitz.
Therefore, the compositions $\Phi^t_{f}\circ i_{e^t}\:e^t{\cdot} \spc{L}\to \spc{L}$ are short. 
Hence a partial limit $\gexp_p\:\T_p
\spc{L}\to \spc{L}$ exists, and it is a short map.

Clearly for any partial limit we have
\[\Phi^t_f\circ\gexp_p(v)=\gexp_p(e^t\cdot v).\]
Since $\Phi^t$ is $e^t$-Lipschitz, it follows that $\gexp_p$ is uniquely
defined.
\qeds

\section{Remarks}

??? gradient exponent for $\kappa\ne 0$
and for nonproper.

The gradient exponential map $\gexp_p$  for a point $p$ a Riemannian manifold $(M,g)$ coincides with the Riemannian exponential map inside the cut locus of $p$ but \emph{is different } from the  Riemannian exponential outside it.

quasigeodesics

%%!TEX root = the-splitting.tex
\chapter{Line splitting}\label{chap:splitting}

\section{Busemann function}

A \index{half-line}\emph{half-line}
\footnote{\red V: we used half-line in the other book but I would still like to change this to "ray" which is the established term in literature. A: Both terms are used, since we use line is bit more  natuaral to say half-line --- but will agree to change it if you want it.}
is a distance-preserving map
from $\RR_{\ge0}=[0,\infty)$ 
to a metric space.
In other words, a half-line is a geodesic defined on the real half-line $\RR_{\ge0}$.

If $\gamma\:[0,\infty)\to \spc{X}$ is a half-line,
then the limit 
\[\bus_\gamma(x)=\lim_{t\to\infty}\dist{\gamma(t)}{x}{}- t\eqlbl{eq:def:busemann*}\]
is called the \index{Busemann function}\emph{Busemann function} of $\gamma$.

The Busemann function $\bus_\gamma$ mimics behavior of the distance function from the ideal point of $\gamma$.

\begin{thm}{Proposition}\label{prop:busemann}
For any half-line $\gamma$ in a metric space $\spc{X}$,
its Busemann function $\bus_\gamma\:\spc{X}\to \RR$ 
is defined.
Moreover, $\bus_\gamma$ is $1$-Lipschitz and $\bus_\gamma (\gamma(t))=-t$ for any $t$.

\end{thm}

\parit{Proof.}
Since $t=\dist{\gamma(0)}{\gamma(t)}{}$, the triangle inequality implies that, the function
\[t\mapsto\dist{\gamma(t)}{x}{}- t\] 
is nonincreasing, and 
\[\dist{\gamma(t)}{x}{}- t\ge-\dist{\gamma(0)}{x}{}\]
for any $x\in \spc{X}$.
Therefore, the limit in \ref{eq:def:busemann*} is defined,
and it is 1-Lipschitz as a limit of 1-Lipschitz functions.
The last statement follows since 
$\dist{\gamma(t)}{\gamma(t_0)}{}\z=t-t_0$ for all large~$t$.
\qeds

\begin{thm}{Exercise}\label{ex:busemann-CBB}
Any Busemann function on an $\Alex0$ space is concave.
\end{thm}

\section{Splitting theorem}

A \index{line}\emph{line} is a distance-preserving map
from $\RR$ to a metric space.
In other words, a line is a geodesic defined on the real line $\RR$.

\begin{thm}{Exercise}\label{ex:bus+bus}
Let $\gamma$ be a line in a metric space $\spc{X}$.
Show that for any point $x$ we have
\[\bus_+(x)+\bus_-(x)\ge 0\]
where, $\bus_+$ and $\bus_-$, are the Busemann functions asociated with half-lines $\gamma:[0,\infty)\to \spc{L}$ and $\gamma:(-\infty,0]\to \spc{L}$ respectively.
\end{thm}


Let $\spc{X}$ be a metric space and $A,B\subset \spc{X}$.
We will write 
\[\spc{X}=A\oplus B\]\index{$A\oplus B$}
if there are projections $\proj_A\:\spc{X}\to A$ 
and 
$\proj_B\:\spc{X}\to B$
such that 
\[\dist[2]{x}{y}{}=\dist[2]{\proj_A(x)}{\proj_A(y)}{}+\dist[2]{\proj_B(x)}{\proj_B(y)}{}\]
for any two points $x,y\in \spc{X}$.

Note that if 
\[\spc{X}=A\oplus B\]
then 
\begin{itemize}
\item $A$ intersects $B$ at a single point,
\item both sets $A$ and $B$ are \index{convex set}\emph{convex sets} in $\spc{X}$;
the latter means that any geodesic with the ends in $A$ (or $B$) lies in $A$ (or $B$). 
\end{itemize}

\begin{thm}{Line splitting theorem}\label{thm:splitting}
Let $\gamma$ be a line in a $\Alex0$ space~$\spc{L}$. 
Then 
\[\spc{L}=\spc{L}'\oplus \gamma(\RR)\]
for some subset $\spc{L}'\subset \spc{L}$.
\end{thm}

\begin{thm}{Corollary}\label{cor:splitting}
Any $\Alex0$ space $\spc{L}$ splits isometrically as
\[
\spc{L}=\spc{L}'\oplus H
\]
where $H\subset \spc{L}$ is a subset isometric to a Hilbert space, and $\spc{L}'\subset \spc{L}$ is a convex subset that contains no lines. 
\end{thm}

The following lemma is closely related to the first distance estimate (\ref{thm:dist-est});
it is also a limit case of \ref{prop:gexp}.
The proof goes along the same lines.

\begin{thm}{Lemma}\label{lem:dist-estimate}
Suppose $f\:\spc{L}\to\RR$ be a concave 1-Lipschitz function on an $\Alex0$ space $\spc{L}$.
Consider two $f$-gradient curves $\alpha$ and~$\beta$.
Then for any $t, s\ge 0$ we have
\begin{align*}
&\dist[2]{\alpha(s)}{\beta(t)}{}
\le 
\dist[2]{p}{q}{}+
2\cdot(f(p)-f(q))\cdot(s-t)+ (s-t)^2,
\end{align*}
where $p=\alpha(0)$ and $q=\beta(0)$.
\end{thm}

\parit{Proof.}
Since $f$ is 1-Lipschitz, $|\nabla f|\le1$.
Therefore 
\[f\circ\beta(t)\le f(q)+t\]
for any $t\ge0$.

Set $\ell(t)=\dist{p}{\beta(t)}{}$.
Applying \ref{eq:fist-var-inq+}, we get
\begin{align*}
(\ell^2)^+(t)
&\le 2\cdot \left(f\circ\beta(t)-f(p)\right)\le 
\\
&\le2\cdot\left(f(q)+t-f(p)\right).
\end{align*}
Therefore 
\[\ell^2(t)-\ell^2(0)\le 2\cdot\left(f(q)-f(p)\right)\cdot t + t^2.\]
It proves the needed inequality in case $s=0$.
Combining it with the first distance estimate (\ref{thm:dist-est}), we get the result in case $s\le t$.
The case $s\ge t$ follows by switching the roles of $s$ and $t$.
\qeds


\parit{Proof of \ref{thm:splitting}.} Consider two Busemann functions, $\bus_+$ and $\bus_-$, asociated with half-lines $\gamma:[0,\infty)\to \spc{L}$ and $\gamma:(-\infty,0]\to \spc{L}$ respectively; that is,
\[
\bus_\pm(x)
\df
\lim_{t\to\infty}\dist{\gamma(\pm t)}{x}{}- t.
\]
According to \ref{ex:busemann-CBB}, 
both $\bus_+$ and $\bus_-$ are concave.

By \ref{ex:bus+bus}, $\bus_+(x)+\bus_-(x)\ge0$ for any $x\in \spc{L}$.
On the other hand, by \ref{comp-kappa}, 
$f(t)=\distfun_x^2\circ\gamma(t)$ 
is $2$-concave.
In particular, $f(t)\le t^2+at+b$ for some constants $a,b\in\RR$.  Therefore, for all large $t$
\[
\dist{\gamma( t)}{x}{}- t +\dist{\gamma(- t)}{x}{}- t\le \sqrt{ t^2+at+b}-t+\sqrt{ t^2-at+b}-t
\]

Passing to the limit as $t\to\infty$, we get that  $\bus_+(x)+\bus_-(x)\le 0$.
Hence
\[
\bus_+(x)+\bus_-(x)= 0
\]
for any $x\in \spc{L}$.
In particular, the functions $\bus_+$ and $\bus_-$ are \index{affine function}\emph{affine};
that is, they are convex and concave at the same time.

Note that for any $x$,
\begin{align*}
|\nabla_x \bus_\pm|
&=\sup\set{\dd_x\bus_\pm(\xi)}{\xi\in\Sigma_x}=
\\
&=\sup\set{-\dd_x\bus_\mp(\xi)}{\xi\in\Sigma_x}\equiv
\\
&\equiv1.
\end{align*}

Observe that $\alpha$ is a $\bus_\pm$-gradient curve
if and only if $\alpha$ is a geodesic such that $(\bus_\pm\circ\alpha)^+=1$.
Indeed, if $\alpha$ is a geodesic, then $(\bus_\pm\circ\alpha)^+\le 1$ and the equality holds only if $\nabla_\alpha\bus_\pm=\alpha^+$.
Now suppose $\nabla_\alpha\bus_\pm=\alpha^+$.
Then $|\alpha^+|\le 1$ and $(\bus_\pm\circ\alpha)^+=1$; therefore 
\begin{align*}
|t_0-t_1|&\ge \dist{\alpha(t_0)}{\alpha(t_1)}{}\ge
\\
&\ge|\bus_\pm\circ\alpha(t_0)-\bus_\pm\circ\alpha(t_1)=
\\
&=|t_0-t_1|.
\end{align*}

It follows that for any $t>0$, the $\bus_\pm$-gradient flows commute;
that is, 
\[\GF_{\bus_+}^t\circ\GF_{\bus_-}^t=\id_\spc{L}.\]
Setting
\[\GF^t=\left[\begin{matrix}
\GF_{\bus_+}^t&\hbox{if}\ t\ge0\\
\GF_{\bus_-}^{-t}&\hbox{if}\ t\le0
               \end{matrix}\right.\]
defines an $\RR$-action on $\spc{L}$.

Consider the level set $\spc{L}'=\bus_+^{-1}(0)=\bus_-^{-1}(0)$;
it is a closed convex subset of $\spc{L}$, and therefore forms an Alexandrov space.
Consider the map $h\:\spc{L}'\times \RR\to \spc{L}$ defined by $h\:(x,t)\mapsto \GF^t(x)$.
Note that $h$ is onto.
Applying \ref{lem:dist-estimate} for $\GF_{\bus_+}^t$ and $\GF_{\bus_-}^t$ shows that $h$ is distance non-expanding and non-contracting at the same time; that is, $h$ is an isometry.
\qeds

Recall that according our definition the real line $\RR$ is $\Alex1$.
However, most of $\Alex1$ spaces have diameter at most $\pi$;
see \ref{ex:RisCBB(1)}.

\begin{thm}{Exercise}\label{ex:cone-CBB}
Suppose $\spc{X}$ is a complete geodesic space.
Show that $\Cone\spc{X}$ is $\Alex0$ if and only if $\spc{X}$ is $\Alex1$ and $\diam\spc{X}\le \pi$.
\end{thm}

\section{Anti-sum}

Here we give a corollary of \ref{ex:convergence-grad}.
It will be used to prove basic properties of the tangent space.


\begin{thm}{Anti-sum lemma}\label{lem:minus-sum} 
Let $\spc{L}$ be an Alexandrov space and $p\in \spc{L}$.

Given two vectors $u,v\in \T_p$, there is a unique vector $w\in \T_p$ such that
\[\langle u,x\rangle +\langle v,x\rangle +\langle w,x\rangle \ge 0\]
for any $x\in \T_p$, and
\[\langle u,w\rangle +\langle v,w\rangle +\langle w,w\rangle =0.\]

\end{thm}

\begin{thm}{Exercise}\label{ex:|antisum|}
Suppose $u,v, w\in \T_p$ are as in \ref{lem:minus-sum}.
Show that 
\[|w|^2\le |u|^2+|v|^2+2\cdot\langle u,v\rangle.\]

\end{thm}

If $\T_p$ were geodesic, then the lemma would follow from the existence  of the gradient, applied to the function $\T_p\to \RR$ defined by $x\mapsto -(\langle u,x\rangle +\langle v,x\rangle )$ which is concave.
However, the tangent space $\T_p$ might fail to be geodesic; see  Halbeisen's example \cite{alexander-kapovitch-petrunin2024}.

Applying the above lemma for $u=v$, we have the following statement.

\begin{thm}{Existence of polar vector}\label{cor:polar}
Let $\spc{L}$ be an Alexandrov space 
and $p\in \spc{L}$. 
Given a vector $u\in \T_p$,  there is a unique vector $u^*\in\T_p$ such that $\langle u^*,u^*\rangle +\langle u,u^*\rangle = 0$ and
$u^*$ is \index{polar vectors}\emph{polar} to $u$;
that is,
\[\langle u^*,x\rangle +\langle u,x\rangle \ge 0\]
for any $x\in \T_p$.
\end{thm}

\parit{Proof of \ref{lem:minus-sum}.}
By \ref{ex:d(distfun):==}, we can choose two sequences of points $a_n,b_n$ such that 
\begin{align*}
\dd_p\distfun_{a_n}(w)&=-\langle\dir{p}{a_n},w\rangle
\\
\dd_p\distfun_{b_n}(w)&=-\langle\dir{p}{b_n},w\rangle
\end{align*}
for any $w\in\T_p$ and $\dir{p}{a_n}\to u/|u|$, $\dir{p}{b_n}\to v/|v|$ as $n\to \infty$

Consider a sequence of functions 
\[f_n=|u|\cdot\distfun_{a_n}+|v|\cdot\distfun_{b_n}.\]
Note that 
\[(\dd_pf_n)(x)=-|u|\cdot\langle \dir{p}{a_n},x\rangle -|v|\cdot\langle \dir{p}{b_n},x\rangle .\]
Thus we have the following uniform convergence for $x\in\Sigma_p$:
\[(\dd_pf_n)(x)\to-\langle u,x\rangle -\langle v,x\rangle \]
as $n\to\infty$,
According to \ref{ex:convergence-grad}, 
the sequence $\nabla_pf_n$ converges.
Let 
\[w=\lim_{n\to\infty}\nabla_pf_n.\]
By the definition of gradient,
\[\begin{aligned}
\langle w,w\rangle &=\lim_{n\to\infty}\langle \nabla_pf_n,\nabla_pf_n\rangle =
&&&%right side
\langle w,x\rangle &=\lim_{n\to\infty}\langle \nabla_pf_n,x\rangle \ge
\\%second line
&=\lim_{n\to\infty}(\dd_p f_n)(\nabla_p f_n)
=
&&&%second line right side
&\ge
\lim_{n\to\infty}(\dd_pf_n)(x)
=
\\%line 3
&=-\langle u,w\rangle -\langle v,w\rangle ,
&&&%line 3 right side
&=-\langle u,x\rangle -\langle v,x\rangle .
\end{aligned}\]
\qedsf

\section{Linear subspace}

\begin{thm}{Definition}\label{def:opp+Lin}
Let $\spc{L}$ be an Alexandrov space, $p\in \spc{L}$ and $u,v\in\T_p$.
We say that vectors $u$ and $v$ are \index{opposite vectors}\emph{opposite}\label{def:opposite:page} to each other, (briefly, $u+v=0$) if $|u|=|v|=0$ or $\mangle(u,v)=\pi$ and $|u|=|v|$.

The subcone
\[\Lin_p=\set{v\in\T_p}{\exists\ w\in\T_p\quad \text{such that}\quad w+v=0}\]
will be called the \index{linear subspace}\emph{linear subspace} of $\T_p$.
\end{thm}

Soon we will introduce a natural linear structure on $\Lin_p$.

\begin{thm}{Proposition}\label{prop:opposite}
Let $\spc{L}$ be an Alexandrov space and $p\in \spc{L}$.
Given two vectors $u,v\in\T_p$, the following statements are equivalent:
\begin{subthm}{opposite} $u+v=0$;
\end{subthm}
\begin{subthm}{<x,u>} $\langle u,x\rangle +\langle v,x\rangle =0$ for any $x\in\T_p$;
\end{subthm}
\begin{subthm}{<xi,u>} $\langle u,\xi\rangle +\langle v,\xi\rangle =0$ for any $\xi\in\Sigma_p$.
\end{subthm}
\end{thm}

\parit{Proof.}
The equivalence  \ref{SHORT.<x,u>}$\Leftrightarrow$\ref{SHORT.<xi,u>} is trivial.

The condition $u+v=0$ is equivalent to 
$\langle u,u\rangle =-\langle u,v\rangle =\langle v,v\rangle$;
thus,
\ref{SHORT.<x,u>}$\Rightarrow$\ref{SHORT.opposite}.

Recall that $\T_p$ has nonnegative curvature.
Note that the hinges $\hinge 0ux$ and $\hinge 0vx$ are adjacent.
By \ref{ex:adjacent-CBB}, $\mangle\hinge 0ux+\mangle\hinge 0vx=\pi$;
hence \ref{SHORT.opposite}$\Rightarrow$\ref{SHORT.<x,u>}.
\qeds

\begin{thm}{Exercise}\label{prop:two-opp}
Let $\spc{L}$  be an Alexandrov space and $p\in \spc{L}$.
Then for any three vectors $u,v,w\in\T_p$, if $u+v=0$ and $u+ w=0$ then $v=w$.
\end{thm}

Let $u\in \Lin_p$; that is, $u+v=0$ for some $v\in\T_p$.
Given $s<0$, let 
\[s\cdot u\df (-s)\cdot v.\]
So we can multiply any vector in $\Lin_p$ by any real number (positive and negative).
By \ref{prop:two-opp}, this multiplication is uniquely defined;
by \ref{prop:opposite}; we have identity
\[\langle -v,x\rangle=-\langle v,x\rangle.\]


\begin{thm}{Exercise}\label{ex:3<,>=0}
Suppose $u,v,w\in\T_p$ are as in \ref{lem:minus-sum}.
Show that
\[\langle u,x\rangle +\langle v,x\rangle +\langle w,x\rangle = 0\]
for any $x\in \Lin_p$.
\end{thm}

\begin{thm}{Exercise}\label{ex:-u}
Let $\spc{L}$ be an Alexandrov space,
$p\in \spc{L}$ and $u\in \T_p$.
Suppose $u^*\in \T_p$ is provided by \ref{cor:polar};
that is, 
\[\langle u^*,u^*\rangle +\langle u,u^*\rangle = 0
\quad\text{and}\quad
\langle u^*,x\rangle +\langle u,x\rangle \ge 0
\]
for any $x\in \T_p$.
Show that 
\[u=-u^*\quad\Longleftrightarrow\quad|u|=|u^*|.\]
\end{thm}

\begin{thm}{Theorem}\label{thm:lin-subcone}
Let $p$ be a point in an Alexandrov space. 
Then $\Lin_p$ is isometric to a Hilbert space.
\end{thm}

\parit{Proof.}
Note that $\Lin_p$ is a closed subset of $\T_p$;
in particular, it is complete.

If any two vectors in $\Lin_p$ can be connected by a geodesic in $\Lin_p$,
then the statement follows from the splitting theorem (\ref{thm:splitting}).
By Menger's lemma (\ref{lem:mid>geod}), it is sufficient to show that for any two vectors $x,y\in\Lin_p$
there is a midpoint $w\in \Lin_p$.

Choose $w\in \T_p$ to be the anti-sum of $u=-\tfrac{1}{2}\cdot x$ and $v=-\tfrac{1}{2}\cdot y$;
see \ref{lem:minus-sum}.
By \ref{ex:|antisum|} and \ref{ex:3<,>=0},
\begin{align*}
|w|^2&\le \tfrac14\cdot |x|^2+\tfrac14\cdot|y|^2+\tfrac12\cdot\langle x,y\rangle,
\\
\langle w,x\rangle&= \tfrac12\cdot|x|^2+\tfrac12\cdot\langle x,y\rangle,
\\
\langle w,y\rangle&= \tfrac12\cdot|y|^2+\tfrac12\cdot\langle x,y\rangle,
\end{align*}
It follows that 
\begin{align*}
|x-w|^2
&= |x|^2+|w|^2-2\cdot\langle w,x\rangle\le
\\
&\le \tfrac14\cdot |x|^2+\tfrac14\cdot|y|^2-\tfrac12\cdot\langle x,y\rangle=
\\
&=\tfrac14\cdot|x-y|^2.
\end{align*}
That is, $|x-w|\le \tfrac12\cdot|x-y|$.
Similarly, we get $|y-w|\le \tfrac12\cdot|x-y|$.
Therefore $w$ is a midpoint of $x$ and $y$.
In addition we get equality 
\[|w|^2= \tfrac14\cdot |x|^2+\tfrac14\cdot|y|^2+\tfrac12\cdot\langle x,y\rangle.\]

It remains to show that $w\in\Lin_p$.
Let $w^*$ be the polar vector provided by \ref{cor:polar}.
Note that 
\[|w^*|\le |w|,
\quad
\langle w^*,x\rangle+\langle w,x\rangle=0,
\quad
\langle w^*,y\rangle+\langle w,y\rangle=0.
\]
The same calculation as above shows that $w^*$ is a midpoint of $-x$ and $-y$ and 
\[|w^*|^2= \tfrac14\cdot |x|^2+\tfrac14\cdot|y|^2+\tfrac12\cdot\langle x,y\rangle=|w|^2.\]
By \ref{ex:-u}, $w=-w^*$;
hence $w\in\Lin_p$.
\qeds

\begin{thm}{Lemma}\label{ex:grad-dist:G-delta}
Given a point $p$ in an Alexandrov space $\spc{L}$,
let $f\z=\distfun_p$, and let $S$ be the subset of points $x\in\spc{L}$ such that $|\nabla_xf|=1$.
Then $S$ is a dense G-delta set.

\end{thm}

\parit{Proof.}
Let $S_n\subset \spc{L}$ be defined by inequality $|\nabla_xf|>1-\tfrac1n$.
By \ref{ex:semicontinuous-grad:>s}, $S_n$ is open.

Choose a point $q\ne p$.
Observe that $|\nabla_xf|=1$ for any point $x\in\left]pq\right[$.
It follows that $S_n$ is dense in $\spc{L}$.

Since $S=\bigcap_nS_n$, the lemma follows.
\qeds


\begin{thm}{Exercise}\label{ex:grad-dist}
Let $p$, $f$, and $S$ be as in \ref{ex:grad-dist:G-delta}.

\begin{subthm}{ex:grad-dist:lin}
Show that 
\[\nabla_xf+\dir xp=0\]
for any 
$x\in S$;
in particular, $\dir xp\in \Lin_x$.
\end{subthm}

\begin{subthm}{ex:grad-dist:|grad|=1}
Show that if $|\nabla_xf|=1$, then $\dd_xf(w)= \langle\nabla_xf,w\rangle$ for any $w\in \T_x$.
\end{subthm}
\end{thm}

Note that \ref{ex:grad-dist} implies the following.

\begin{thm}{Corollary}\label{cor:euclid-subcone}
Given a countable set of points $X$ in an Alexandrov space $\spc{L}$
there is a G-delta dense set $S\subset\spc{L}$
such that 
$\dir sx\in \Lin_s$
for any $s\in S$ and $x\in X$.
\end{thm}

\section{Comments}

The splitting theorem has an interesting history that starts with Stefan Cohn-Vossen \cite{cohn-vossen_line};
who proved its $2$-dimensional case.
For Riemannian manifolds of higher dimensions 
it was proved by Victor Toponogov \cite{toponogov-globalization+splitting}.
Then it was generalized by Anatoliy Milka \cite{milka-line}
to Alexandrov spaces;
historically, it was the first result about Alexandrov spaces of dimension higher than 2.
Nearly the same proof is used in \cite[1.5]{burago-burago-ivanov}.

Further generalizations of the splitting theorem for Riemannian manifolds with nonnegative Ricci curvature were obtained by Jeff Cheeger and Detlef Gromoll \cite{cheeger-gromoll-split}.
This was further generalized by Jeff Cheeger and Toby Colding for limits of Riemannian manifolds with almost nonnegative Ricci curvature \cite{cheeger-colding-alm-rigidity} and to their synthetic generalizations, so-called {}\emph{RCD spaces}, by Nicola Gigli \cite{gigli2013splitting, gigli-splitting-overview}.
Jost-Hinrich Eschenburg obtained an analogous result for  Lorentzian manifolds \cite{eshenburg-split}, that is, pseudo-Riemannian manifolds of signature $(1,n)$.

The presented proof is close in spirit to the proof given by Cheeger and Gromoll \cite{cheeger-gromoll-split};
it is taken from our book \cite{alexander-kapovitch-petrunin2024}.

\begin{thm}{Open question}
Let $p$ be a point in an Alexandrov space $\spc{L}$.
Suppose that $0\ne v\in \Lin_p$.
Is it true that the tangent space $\T_p$ splits in the direction of $v$?
\end{thm}

Halbeisen's example \cite{alexander-kapovitch-petrunin2024,halbeisen} shows that compactness of space of directions is essential in the proof that space of directions is $\pi$-geodesic (see \ref{thm:finite-space-of-directions}).

\begin{thm}{Open question}\label{open:Halb-proper}
Let $\spc{L}$ be a proper Alexandrov space.
Is it true that for any $p\in \spc{L}$, the tangent space $\T_p$ is a length space?
\end{thm}

%%%%%%%%%%%%%%%%%%%%%%%%%%%%%%%%%%%%%%%%%%%%%%%%%%

\chapter{Dimension and volume}\label{chap:dim}

\section{Linear dimension}

Let $\spc{L}$ be an Alexandrov space.
Let us define its \index{linear dimension}\emph{linear dimension} \index{$\LinDim \spc{L}$}$\LinDim \spc{L}$ as the least upper bound on integers $m$ such that 
the Euclidean space $\EE^m$ is isometric to a subspace of the tangent space $\T_p\spc{L}$ at some point $p\in \spc{L}$.
If not stated otherwise, dimension of an Alexandrov space is its linear dimension.

If not stated otherwise, dimension will mean linear dimension.
In Section~\ref{sec:all-dim}, we will show that linear dimension of Alexandrov space coincides with all reasonable dimensions;
after that we will use \index{$\dim \spc{L}$}$\dim$ for $\LinDim$.

\begin{thm}{(\textit{n}+1)-comparison}
Let $\spc{L}$ be an $\Alex0$ space.
Then for any finite set of points $p,x_1,\dots,x_n\in \spc{L}$, there is a model configuration 
$\tilde p,\tilde x_1,\dots,\tilde x_n\in \EE^m$ such that 
\[|\tilde p-\tilde x_i|_{\EE^m}=| p- x_i|_{\spc{L}}
\quad\text{and}\quad
|\tilde x_i-\tilde x_j|_{\EE^m}\ge |x_i- x_j|_{\spc{L}}\]
for any $i$ and $j$.
Moreover, we can assume that $m\le \LinDim\spc{L}$. 
\end{thm}

\parit{Proof.}
By \ref{cor:euclid-subcone}, we can choose a point $p'$ arbitrarily close to $p$ so that 
$\Lin_{p'}\ni \dir{p'}{x_i}$ for any $i$.
Let us identify $\EE^m$ with a subspace of $\Lin_{p'}$ spanned by $\dir{p'}{x_1},\dots,\dir{p'}{x_n}$.
Note that $m\le \LinDim\spc{L}$.

Set $\tilde p'=0\in \EE^m$ and $\tilde x_i=\dist{p'}{x_n}{}\cdot\dir{p'}{x_n}\in \EE^m$ for every $i$.
Note that 
\[|\tilde p'-\tilde x_i|_{\EE^m}=| p'- x_i|_{\spc{L}}\]
for every $i$.
Applying the comparison $\mangle\hinge {p'}{x_i}{x_j}\ge \angk {p'}{x_i}{x_j}$, we get
\[|\tilde x_i-\tilde x_j|_{\EE^m}\ge |x_i- x_j|_{\spc{L}}\]
for any $i$ and $j$.
Passing to a limit configuration as $p'\to p$ we get the result.
\qeds

\begin{thm}{Exercise}\label{ex:tangent=Em}
Let $\spc{L}$ be an $\Alex0$ space.
Suppose $\LinDim\spc{L}\z=m<\infty$.
Show that $\T_p\spc{L}\iso \EE^m$ for a G-delta dense set of points $p\in\spc{L}$.
\end{thm}

\begin{thm}{Exercise}\label{ex:dim=1}
Show that a 1-dimensional Alexandrov space is homeomorphic to a 1-dimensional manifold, possibly with nonempty boundary.
\end{thm}


\begin{thm}{Exercise}\label{ex:resporka}
Let $\spc{L}$ be an $\Alex0$ space.

Show that $\LinDim \spc{L}\ge m$ if and only if for some $m+2$ points $p$, $a_0,\z\dots, a_{m}\in \spc{L}$
we have
\[\angk p{a_i}{a_j}>\tfrac\pi2\]
for any pair $i\ne j$.
\end{thm}

\section{Space of directions}

A metric space $\spc{X}$ will be called $\ell$-geodesic 
if any two points $x,y\in\spc{X}$ such that $\dist{x}{y}{}<\ell$ can be connected by a geodesic.
For instance, any geodesic space is $\infty$-geodesic.

\begin{thm}{Theorem}\label{thm:finite-space-of-directions}
Let $\spc{L}$ be a finite-dimensional Alexandrov space.
Then for any point $p\in \spc{L}$, its space of directions $\Sigma_p$ is a compact $\pi$-geodesic space.
\end{thm}


\begin{thm}{Exercise}\label{ex:finite-tan}
Let $p$ be a point in a finite-dimensional Alexandrov space $\spc{L}$.
Prove the following.
\begin{subthm}{ex:finite-tan:tan}
The tangent space $\T_p$ is a proper $\Alex0$ space.
\end{subthm}

\begin{subthm}{ex:finite-space-of-directions-dim}
$\LinDim\Sigma_p=\LinDim\spc{L}-1$.
\end{subthm}

\begin{subthm}{ex:finite-tan:sigma}
If $\LinDim \spc{L}>1$, then $\Sigma_p$ is geodesic.
\end{subthm}


\end{thm}

Using \ref{ex:finite-space-of-directions-dim}, one can prove results for all finite dimensional Alexandrov spaces via induction on  dimension.
Such proofs will be indicated below.

\parit{Proof of \ref{thm:finite-space-of-directions}.}
Choose $\eps>0$; suppose $\spc{L}$ is $m$-dimensional.
Assume can choose $n$ directions $\xi_1,\dots, \xi_n\in \Sigma_p$ such that $\mangle(\xi_i,\xi_j)\z>\eps$ for any $i\ne j$.
Without loss of generality, we may assume that each direction is geodesic;
that is, there is a point $x_i\in \spc{L}$ such that $\xi_i=\dir p{x_i}$.

Choose $y_i\in [px_i]$ such that $\dist{p}{y_i}{}=r$ for each $i$ and small fixed $r>0$.
Since $r$ is small, we can assume that $\angk p{y_i}{y_j}>\eps$ for any $i\ne j$.
By \ref{cor:euclid-subcone}, we can choose $p'$ arbitrarily close to $p$ such that $\dir{p'}{y_i}\in \Lin_{p'}$ for any $i$.
Since  $\dist{p'}{p}{}$ is small, $\angk {p'}{y_i}{y_j}>\eps$ for any $i\ne j$.
By comparison, 
\[\mangle \hinge{p'}{y_i}{y_j}>\eps.\]
Therefore $n\le \pack_\eps\SSS^{m-1}$,
where \index{$\pack_\eps\spc{X}$}$\pack_\eps\spc{X}$ is the exact upper bound on the number of points $x_1,\z\dots,x_k\in \spc{X}$ such that $\dist{x_i}{x_j}{}\ge\eps$ if $i\ne j$.

Since $\SSS^{m-1}$ is compact, $\pack_\eps\SSS^{m-1}<\infty$.
By the definition, the space of directions $\Sigma_p$ is complete. 
Applying \ref{ex:pack-net}, we get that  $\Sigma_p$ is compact.

It remains to prove the following claim.

\begin{clm}{}
If $\Sigma_p$ is compact, then it is $\pi$-geodesic
\end{clm}

Choose two geodesic directions $\xi=\dir px$ and $\zeta=\dir py$;
let 
\[\alpha\z=\tfrac12\cdot \mangle \hinge pxy=\tfrac12\cdot \dist{\xi}{\zeta}{\Sigma_p}.\]

Suppose $\alpha<\pi/2$.
Let us show that it is sufficient to construct an \index{almost midpoint}\emph{almost midpoint} $\mu\z=\dir pz$ of $\xi$ and $\zeta$ in $\Sigma_p$;
that is, we need to show that for any $\eps>0$ there is a geodesic $[pz]$ such that
\[\mangle\hinge pxz\le \alpha+\eps
\quad\text{and}\quad
\mangle\hinge pyz\le \alpha+\eps.\]
Indeed, once it is done, the compactness of $\Sigma_p$ can be used to get an actual midpoint for any two directions in $\Sigma_p$.
After that Menger's lemma (\ref{lem:mid>geod}) will finish the proof.

Choose a sequence of small positive numbers $r_n\to0$
Consider sequnces $x_n\z\in [px]$ and $y_n\z\in [py]$ such that $\dist{p}{x_n}{}=\dist{p}{y_n}{}=r_n$.
Let $m_n$ be a midpoint of $[x_n\,y_n]$.
%??? we use here that the directions $\xi=\dir px$ and $\zeta=\dir py$ are not opposite???

Since $\Sigma_p$ is compact, we can pass to a sequence of $r_n$ such that 
$\dir{p}{m_n}$ converges;
denote its limit by $\mu$.
Choose a geodesic $[pz]$ that runs at small angle from $\mu$.
Let us show that $\dir pz$ is the needed almost midpoint.

Evidently, $\angk p{x_n}{m_n}=\angk p{y_n}{m_n}$.
By \ref{ex:alex-lemma-cat}, we have
\[\angk p{x_n}{m_n}+\angk p{y_n}{m_n}\le \angk p{x_n}{y_n}.\]

Let $z_n\in [pz]$ be the point such that $\dist{p}{z_n}{}=\dist{p}{m_n}{}$.
By construction, for all large $n$, we have $\mangle\hinge pz{m_n}\approx0$  with arbitrary small given error.
By comparison, the value $\frac{\dist{z_n}{m_n}{}}{\dist{p}{z_n}{}}$ can be assumed to be arbitrary small for all large $n$.
Applying this observation and the definition of angle measure, we also have the following approximations
\begin{align*}
\angk p{x_n}{y_n}&\approx \mangle\hinge p{x_n}{y_n},
\\
\angk p{x_n}{m_n}\approx\angk p{x_n}{z_n}&\approx\mangle\hinge p{x_n}{z_n},
\\
\angk p{m_n}{y_n}\approx\angk p{z_n}{y_n}&\approx\mangle\hinge p{z_n}{y_n},
\end{align*}
again, with arbitrary given error and all large $n$.
It follows that $\dir pz$ is an almost midpoint of $\dir px$ and $\dir py$, as required.
\qeds

In the above proof, the angles $\mangle\hinge pxz$ and $\mangle\hinge pyz$ have lower bounds by 
the comparison, but we needed upper bounds that were extracted from the definition of angle measure and compactness of space of directions.

\section{Right-inverse theorem}

\begin{thm}{Theorem}\label{thm:right-inverse}
Suppose $p,a_0,\dots,a_m$ be points in an Alexandrov space $\spc{L}$ such
\[\angk p{a_i}{a_j}>\tfrac\pi2\]
for any $i\ne j$.
Then the map $f\:\spc{L}\to\RR^m$ defined by
\[f\:x\mapsto (\dist{a_1}{x}{},\dots,\dist{a_m}{x}{})\]
has a left inverse defined in a neighborhood of $f(p)$.
\end{thm}

In the proof we construct a local right inverse $\map$ of $f$ around $f(p)$.
The construction uses gradient flow for suitably chosen family of functions.
The structure of the proof can be seen in the following exercise,
more details are given in the hints.

\begin{thm}{Exercise}\label{ex:proof-right-inverse}
Suppose $p,a_0,\dots,a_m\in\spc{L}$ and $f\:\spc{L}\to\RR$ are as in \ref{thm:right-inverse}.
Assume $\eps>0$ is sufficiently small.
Given $\bm{y}=(y_1,y_2,\dots,y_m)\in \RR^m$, 
consider the function on $\spc{L}$ defined by
\[f_{\bm{y}}(x)=\min\{\,0, \dist{a_1}{x}{}-y_1,\dots,\dist{a_m}{x}{}-y_m\,\}+\eps\cdot\dist{a_0}{x}{}.\]

\begin{subthm}{ex:proof-right-inverse:grad}
There is $r>0$ such that 
Show that $f_{\bm{y}}$ is $\lambda$-concave in $\oBall(p,r)$ for some $\lambda$ and
\begin{enumerate}[(i)]
\item\label{111} $(\dd_x\distfun_{a_i})(\nabla_x f_{\bm{y}})<-\tfrac{1}{10}\cdot\eps^2$ if $\dist{a_i}{x}{}>y_i$ and
\item\label{222} $(\dd_x\distfun_{a_i})(\nabla_x f_{\bm{y}})>\tfrac{1}{10}\cdot\eps^2$ if 
\[\dist{a_i}{x}{}-y_i=\min_j\{\dist{a_j}{x}{}\z-y_j\}<0.\]
\end{enumerate}
for any $x\in \oBall(p,r)$.

\end{subthm}

\begin{subthm}{ex:proof-right-inverse:alpha}
Let $\alpha_{\bm{y}}$ be $f_{\bm{y}}$-gradient curve that starts at $p$.
Use \ref{SHORT.ex:proof-right-inverse:grad} to show that 
if for some $\bm{y}\in\RR^m$ and $t_0\le\tfrac{r}{2}$ we have
$|\distfun_{\bm{a}}{p}-\bm{y}|
\le
\tfrac{\eps^2}{10}\cdot t_0$, then 
$
\distfun_{\bm{a}}{[\alpha_{\bm{y}}(t_0)]}
= 
\bm{y}$.
\end{subthm}

\begin{subthm}{ex:proof-right-inverse:end}
Let $t_0(\bm{y})=\tfrac{10}{\eps^2}\cdot|\dist{\bm{a}}{p}{}-\bm{y}|$.
Use \ref{lem:fg-dist-est} to show that the map
\[\map\:{\bm{y}}\mapsto \alpha_{\bm{y}}\circ t_0(\bm{y})\]
continuous in $\Omega=\oBall(\dist{\bm{a}}{p}{},\tfrac{\eps^2\cdot r}{20} )\subset\RR^m$
and $f\circ \Phi(\bm{y})=\bm{y}$ for any $\bm{y}\in \Omega$.
This finishes the proof of \ref{thm:right-inverse}.
\end{subthm}

\end{thm}

%??? I think that since this is used later it should be proved and not left as a reference A: If we add a solution, then that is OK, is not it? in any case, the idea is more tranparent in the exercise and if needed one can read the solution. But lets do the real solution, not just a hint.

\section{Distance chart}

\begin{thm}{Theorem}\label{thm:dist-chart}
Suppose $p,a_0,\dots,a_m$ be points in an $m$-dimensional Alexandrov space $\spc{L}$ such
\[\angk p{a_i}{a_j}>\tfrac\pi2\]
for any $i\ne j$.
Then the map $f\:\spc{L}\to\RR^m$ defined by
\[f\:x\mapsto (\dist{a_1}{x}{},\dots,\dist{a_m}{x}{})\]
gives a bi-Lipschitz embedding of a neighborhood $\Omega$ of $p$;
the restriction $f|_\Omega$ is called \emph{distance chart} at $p$.
\end{thm}

The following exercise guides you to prove the theorem.

\begin{thm}{Exercise}\label{ex:proof-dist-chart}
Suppose $p,a_0,\dots,a_m\in\spc{L}$ and $f\:\spc{L}\to\RR$ are as in \ref{thm:right-inverse}.
Show that there is $\eps>0$ such that one of the following $m$ inequalities hold
\begin{align*}
\mangle\hinge xy{a_1}&<\tfrac\pi2-\eps,\ \dots,\  \mangle\hinge xy{a_m}<\tfrac\pi2-\eps,
\\
\mangle\hinge yx{a_1}&<\tfrac\pi2-\eps,\ \dots,\ \mangle\hinge yx{a_m}<\tfrac\pi2-\eps
\end{align*}
for any two points $x,y$ in a sufficiently small neighborhood of $p$.
Use it to prove \ref{thm:dist-chart}.
\end{thm}

\section{Volume}

Fix a positive integer $m$.
The $m$-dimensional Hausdorff measure of a Borel set $B$ in a metric space will be called its \index{volume}\emph{$m$-volume}; it will be denoted by $\vol_m B$.
We assume that the Hausdorff measure is calibrated so that the unit cube in $\EE^m$ has unit volume.

This definition will be applied mostly to subsets in $m$-dimensional Alexandrov spaces.
In this case, we may write $\vol B$ instead of $\vol_m B$.


\begin{thm}{Bishop--Gromov inequality}\label{inq:BG}
Let $\spc{L}$ be an $\Alex0$ space.
Suppose $\dim \spc{L}=m<\infty$.
Then 
\[\vol \oBall(p,r)\le \omega_m\cdot r^m,\]
where $\omega_m$ denotes the volume of the unit ball in $\EE^m$.
Moreover, the function 
\[r\mapsto \frac{\vol B(p,r)}{r^m}\]
is nonincreasing.
\end{thm}

\parit{Proof.}
Given $x\in\spc{L}$ choose a geodesic path $\gamma_x$ from $p$ to $x$.
Recall that $\log_p\:\spc{L}\to \T_p$ can be defined by $\log_p\:x\mapsto \gamma_x^+(0)$.
By comparison, $\log_p$ is distance-noncontracting.
Note that $\log_p$ maps $\oBall(p,r)_{\spc{L}}$ to $\oBall(0,r)_{\T_p}$.

\begin{wrapfigure}{r}{44 mm}
\vskip-0mm
\centering
\includegraphics{mppics/pic-803}
\vskip1mm
\end{wrapfigure}

If $\T_p\iso \EE^m$, then $\vol\oBall(0,r)_{\T_p}\z=\omega_m\cdot r^m$,
and the first statement follows.

If $\T_p$ is not isometric to $\EE^m$, then by \ref{ex:tangent=Em}, we can find a point $p'$ arbitrarily close to $p$ such that $\T_{p'}\iso \EE^m$.
If $\eps>\dist{p}{p'}{}$, then $\oBall(p,r)\subset \oBall(p',r+\eps)$.
Therefore,
\[\vol \oBall(p,r)\le \omega_m\cdot (r+\eps)^m\]
for any $\eps>0$.
Hence the first statement follows.

Now, suppose $0<r_1<r_2$.
Consider the map $w\: \spc{L}\to \spc{L}$ defined by $w\:x\mapsto \gamma_x(\tfrac {r_1}{r_2})$.
(The map $w$ mimics the dilation with center at $p$ and coefficient $\tfrac {r_1}{r_2}$.)
By comparison,
\[\dist{w(x)}{w(y)}{}\ge \tfrac {r_1}{r_2}\cdot \dist{x}{y}{}.\]
Observe that $\oBall(p,r_1) \supset w[\oBall(p,r_2)]$.
Therefore, 
\[\vol \oBall(p,r_1)\ge (\tfrac {r_1}{r_2})^m\cdot\vol \oBall(p,r_2).\]
\qedsf

The following exercise generalizes the Bishop--Gromov inequality to $\Alex{-1}$ case. 
It is sufficient for most applications, but a more exact statement will be given in \ref{inq:BG+} which also includes the case of  $\Alex{1}$ spaces.

\begin{thm}{Exersice}\label{ex:BG}
Let $\spc{L}$ be an $\Alex{-1}$ space.
Suppose $\spc{L}=m<\infty$.
Show that
\[\vol \oBall(p,r)\le \omega_m\cdot(\sinh r)^m,\]
where $\omega_m$ denotes the volume of the unit ball in $\EE^m$.
Moreover, the function 
\[r\mapsto \frac{\vol B(p,r)}{(\sinh r)^m}\]
is nonincreasing.
\end{thm}

\section{Other dimensions}\label{sec:all-dim}

Next we want to show that \textit{all reasonable definitions of dimension give the same result for Alexandrov spaces}.
More precisely, we have the following theorem; compare to \cite[15.16]{alexander-kapovitch-petrunin2024}.
We refer to \cite{hurewicz-wallman} for definitions of \index{Lebesgue coverning dimension}\emph{Lebesgue coverning dimension} \index{$\TopDim$ (topological dimension)}$\TopDim$ and \index{Hausdorff dimension}\emph{Hausdorff dimension} \index{$\HausDim$ (Hausdorff dimension)}$\HausDim$.

\begin{thm}{Theorem}\label{thm:dim=dim}
For any Alexandrov space $\spc{L}$, we have
\[\LinDim \spc{L}=\TopDim \spc{L}=\HausDim \spc{L}.\]
\end{thm}

\parit{Proof.}
Suppose $\LinDim\spc{L}\ge m$.
The right inverse theorem implies that $\spc{L}$ contains a subset homeomorphic to an open ball in $\EE^m$.
It follows that
\[\TopDim\spc{L}\ge \LinDim\spc{L}.\]

By Szpilrajn's theorem \cite[theorems V 8 and VII 2]{hurewicz-wallman}, 
\[\HausDim\spc{L}\ge\TopDim\spc{L}.\]

Finally, by the Bishop--Gromov inequality (\ref{inq:BG} and \ref{ex:BG}), we get that 
\[\LinDim \spc{L}\ge \HausDim\spc{L}.\]
\qedsf

\begin{thm}{Exercise}\label{ex:dim=dim}
Let $\Omega$ be an open subset of Alexandrov space $\spc{L}$.
Show that 
\[\LinDim \spc{L}=\LinDim \Omega=\TopDim \Omega=\HausDim \Omega.\]
\end{thm}

\section{Comments}

Let us state a version of Bishop--Gromov inequality for $\Alex\kappa$ spaces.
Its proof requires additionally the so-called \textit{coarea formula} for Alexandrov spaces. 
The weaker inequality from \ref{ex:BG} is sufficient for the sequel.

\begin{thm}{Bishop--Gromov inequality}\label{inq:BG+}
Let $p$ be a point in an $m$-dimensional $\Alex\kappa$ space.
Consider the function $v(r)\z=\vol_m\oBall(p,r)$;
denote by $\tilde v(r)$ the volume of $r$ ball in $\MM^m(\kappa)$.
Then 
\[v(r)\le \tilde v(r)\]
for $r>0$ and the function 
\[r\mapsto \frac{v(r)}{\tilde v(r)}\] is nonincreasing.
If $\kappa>0$, then one has to assume that $r<\tfrac\pi{\sqrt\kappa}$.
\end{thm}

This inequality was originally proved for Riemannian manifolds with lower Ricci curvature.
The first part is also called \emph{Bishop's inequality}.
It is due to Richard Bishop; see \cite{bishop1964} and \cite[Corollary 4, p. 256]{bishop-crittenden}.
The second part is due to Michael Gromov \cite{gromov1981}.

Theorem~\ref{thm:dim=dim}, was ssentially proved by Conrad Plaut \cite{plaut:dimension}.
At that time, it was not known whether
\[\LinDim\spc{L}=\infty\quad \Rightarrow\quad \TopDim\spc{L}=\infty\]
for any Alexandrov space $\spc{L}$.
The latter implication was proved by Grigory Perelman and the second author \cite{perelman-petrunin:qg}.


%%!TEX root = the-volume.tex

\chapter{Limit spaces}\label{chap:lim}\label{chap:stability}


Here we will show that lower curvature bound in the sense of Alexandrov survives under Gromov--Hausdorff limit,
present Perelman's construction of strictly concave functions, and
prove Gromov's selection theorem.

The suvival of curvature bound provides the main source of applications of Alexandrov geometry;
as an illustration we prove the homotopy stability theorem (\ref{thm:h-stability}) and deduce the homotopy finiteness theorem (\ref{thm:h-finiteness}) from it.



\section{Survival of curvature bounds}

\begin{thm}{Theorem}\label{thm:CBB-closed}
Let $\spc{X}_n\z\to \spc{X}_\infty$ be a convergence in the sense of Gromov--Hausdorff.
Suppose that for each $n$, the space $\spc{X}_n$ has curvature $\ge\kappa$ in the sense of Alexandrov.
Then the same holds for~$\spc{X}_\infty$.
\end{thm}

\parit{Proof}.
Choose a quadruple of points $p_\infty, x_\infty,y_\infty,z_\infty\in \spc{X}_\infty$.

By the definition of Gromov--Hausdorff convergence, we can choose points $p_n$,  $x_n$, $y_n$, $z_n\in \spc{X}_n$ for each $n$
that converge to $p_\infty$, $x_\infty$, $y_\infty$, $z_\infty\in \spc{X}_\infty$, respectively.
In particular, each of the 6 distances between pairs of $p_n$, $x_n$, $y_n$, $z_n$ converge to the distance between the corresponding pairs of $p_\infty, x_\infty,y_\infty,z_\infty$.

Since $\MM^2(\kappa)$-comparison holds for $p_n$, $x_n$, $y_n$, $z_n\z\in \spc{X}_n$,
passing to the limit, we get the $\MM^2(\kappa)$-comparison for $p_\infty$,  $x_\infty$, $y_\infty$, $z_\infty$.
\qeds

\begin{thm}[!]{Exercise}\label{ex:dim-lim}
Suppose that a sequence $\spc{A}_1,\spc{A}_2,\dots$ of $\Alex\kappa$ spaces converges to $\spc{A}_\infty$ in the sense of Gromov--Hausdorff.
Show that $\spc{A}_\infty$ is $\Alex\kappa$ and
\[\dim \spc{A}_\infty\le \liminf_{n\to\infty} \dim \spc{A}_n.\]
\end{thm}

\section{Gromov's selection theorem}

\begin{thm}{Gromov's selection theorem}\label{thm:gromov-compactness}
Let $m$ be a positive integer, and let $D,\kappa\in\RR$.
Then any sequence of $m$-dimensional $\Alex\kappa$ spaces with diameters at most $D$
has a converging subsequence in the sense of Gromov--Hausdorff.
\end{thm}

\parit{Proof of \ref{thm:gromov-compactness}.}
Denote by $\bm{K}$ the set of all isometry classes of $\Alex0$ spaces with dimension $\le m$ and diameter $\le D$.
By \ref{ex:dim-lim}, $\bm{K}$ is a closed subset of $\GH$.

Choose a space $\spc{A}\in \bm{K}$;
suppose $x_1,\dots,x_n\in \spc{A}$ is a collection of points such that $\dist{x_i}{x_j}{}> \eps$ for all $i\ne j$.
Note that the balls $B_i=\oBall(x_i,\tfrac\eps2)$ do not overlap.

By \ref{thm:right-inverse}, $\vol \spc{A}>0$.
By Bishop--Gromov inequality, $\vol \spc{A}<\infty$,
and if $\eps<D$, then 
\[\vol B_i\ge (\tfrac\eps{2\cdot D})^m\cdot\vol \spc{A}\]
for any $i$.
It follows that $n\le (\tfrac{2\cdot D}\eps)^m$;
that is, 
\[\pack_\eps\spc{A}\le  N(\eps)\df(\tfrac{2\cdot D}\eps)^m\]
for all small $\eps>0$.

Choose a maximal $\eps$-packing $x_1,\z\dots,x_n\in \spc{A}$.
By \ref{ex:pack-net}, $\spc{F}_\eps\z\df\{x_1,\z\dots,x_n\}$ is an $\eps$-net of $\spc{A}$.
Observe that $\dist{\spc{F}_\eps}{\spc{A}}{\GH}\le \eps$.
Further, note that the set $\bm{F}_\eps$ of finite metric spaces with diameter $\le D$ and at most $N(\eps)$ points forms a compact subset in $\GH$.

Summarizing, for any $\eps>0$ we can find a compact $\eps$-net $\bm{F}_\eps\subset \GH$ of $\bm{K}$.
Since $\GH$ is complete (\ref{prop:complete}), it remains to apply \ref{ex:net:compact}.

We finished the proof of the case $\kappa=0$.
In the general case, applying rescaling, we can assume that $\kappa=-1$ and then argue as before, using \ref{ex:BG} instead of \ref{inq:BG}.
\qeds

\begin{thm}[!]{Exercise}\label{ex:pack-vol}

\begin{subthm}{ex:pack-vol:pack}
Let $\spc{A}$ be an $m$-dimensional $\Alex0$ space with diameter $\le D$.
Suppose $\vol\spc{A}\ge v_0>0$.
Show that 
\[\pack_\eps\spc{A}\ge \frac\Const{\eps^m}\]
for some constant $\Const=\Const(m,D,v_0)>0$.
\end{subthm}


\begin{subthm}{ex:pack-vol:dim}
Conclude that if $\spc{A}_n$ is a sequence of $m$-dimensional $\Alex0$ spaces with diameter $\le D$, and volume $\ge v_0$, then its Gromov--Hausdorff limit $\spc{A}_\infty$ (if it exists) has dimension~$m$.
\end{subthm}
\end{thm}

\begin{thm}{Exercise}\label{ex:diam-compact:GH}
Show that any sequence of $m$-dimensional $\Alex\kappa$ spaces with marked points contains a subsequence pointed-converging in the sense of Gromov--Hausdorff (see Section~\ref{sec:Gromov--Hausdorff}).

\end{thm}

%%!TEX root = the-homot-finite.tex
\section{Controlled concavity}

Alexandrov spaces have plenty of semiconcave functions;
for instance, square of distance function. 
The following theorem provides a source of strictly concave functions  defined in a small open sets of finite-dimensional Alexandrov spaces. 

\begin{thm}{Theorem}
\label{thm:strictly-concave}
Let $\spc{L}$ be a complete finite-dimensional Alexandrov  space.
Then for any point $p\in \spc{L}$, there is  a strictly concave function $f$ defined in an
open neighborhood of $p$.

Moreover, given $0\ne v\in T_p$, the differential, $\dd_p f$, can be chosen
arbitrarily close to $x\mapsto -\<v,x\>$.
\end{thm}

\parit{Proof.} 
Fix small $r>0$ and large $c$;
consider the real-to-real function 
$$\phi_{r,c}(x)=(x-r)- c\cdot(x-r)^2/r,$$
so we have 
$\phi_{r,c}(r)=0$,
$\phi_{r,c}'(r)=1$,
and $\phi_{r,c}''(r)=- {2c}/{r}$. 

\begin{wrapfigure}{o}{44 mm}
\vskip-0mm
\centering
\includegraphics{mppics/pic-901}
\vskip1mm
\end{wrapfigure}

Let $\gamma$ be a unit-speed geodesic, fix a point $q$ and let 
$$\alpha(t)=\mangle(\gamma^+(t),\dir{\gamma(t)}{q}).$$
Recall that $r$ is small.
If $\dist q{\gamma(t)}{}$ is sufficiently close to
$r$, then direct calculations show that
$$(\phi_{r,c}\circ\distfun_q\circ\gamma)''(t)
\le 
\frac{3-c\cdot \cos^2[\alpha(t)]}{r}.$$
(Since $c$ is large, this inequality implies that $\phi_{r,c}\circ\distfun_q\circ\gamma$ is strictly concave at $t$ unless $\alpha(t)\approx\tfrac\pi2$.) 

Now, assume $\{q_1,\dots, q_N\}$ is a finite set of points such that $\dist p{q_i}{}=r$ for any $i$. 
For a geodesic $\gamma$, set $\alpha_i(t)=\mangle(\gamma^+(t),\dir {\gamma(t)}{q_i})$. 
Assume we have a collection $\{q_i\}$ such
that 
\[\max_i\{|\alpha_i(t)-\tfrac\pi2|\}\ge\eps>0\]
for any geodesic $\gamma$ in $\oBall(p,\eps)$. 
We can assume that $c>3N/\cos^2\eps$;
then the inequality above implies that the function
$$f=\sum_i \phi_{r,c}\circ\distfun_{q_i}$$
is strictly concave in $\oBall(p,\eps')$ for some positive $\eps'<\eps$.

The same argument as in \ref{ex:pack-vol} shows that for small $r>0$, one can
choose $N\ge \Const/\delta^{m-1}$ points $\{q_i\}$ such that $\dist{p}{q_i}{}=r$
and $\angk p{q_j}{q_i}>\delta$ (here $\Const=\Const(\Sigma_p)>0$).
On the other hand, suppose $\gamma$ runs from $x$ to~$y$.
If $|\alpha_i(t)- \tfrac\pi2|<\eps\ll\delta$, then applying the ($n$+1)-point comparison to $\gamma(t)$, $x$, $y$ and all $\{q_i\}$ we get that
$N\le \Const(m)/\delta^{m-2}$. 
Therefore, for small $\delta>0$ and yet smaller $\eps>0$, the set $\{q_i\}$ forms the needed collection.

If $r$ is small, then points $q_i$ can be chosen so that all directions
$\dir p {q_i}$ will be $\eps$-close to a given direction $\xi$ and
therefore the second property follows.
\qeds

The function $f$ in \ref{thm:strictly-concave} can be chosen to have maximum value $0$ at $p$,
$f(p)=0$ and with $\dd_p f(x)\approx-|x|$.
It can be constructed by taking the minimum of the functions in the theorem.
Then the set $K=\set{x\in\spc{L}}{f(x)\ge -\eps}$ forms a closed convex neighborhood of $p$ for any small $\eps>0$, so we get the following.


\begin{thm}{Corollary}\label{cor:convex-nbhd}
Any point $p$ of a finite-dimensional Alexandrov space admits an arbitrary small convex closed neighborhood $K$ and a strictly concave function $f$ defined in a neighborhood of $K$ such that $p$ is the maximum point of $f$
and $f|_{\partial K}=0$.
\end{thm}

\section{Liftings}

Suppose that the Gromov--Haudorff distance $\dist{\spc{L}}{\spc{L}'}{\GH}$ is sufficienlty small, so we may think that both spaces $\spc{L}$ and $\spc{L}'$ lie at small Hausdorff distance in an ambient metric space $\spc{W}$.
In particular, we may choose a small $\eps>0$, so that for any point $p\in \spc{L}$, there is a point $p'\in \spc{L}'$ such that $\dist{p}{p'}{\spc{W}}<\eps$;
the point $p'$ will be called a \index{lifting}\emph{lifting} (or \emph{$\eps$-lifting}) of $p$ in $\spc{L}'$.
We may choose a lifting $p'\in\spc{L}'$ for every point $p\in\spc{L}$, 
in this case the map $p\mapsto p'$ is called a {}\emph{($\eps$-)lifting map}.

Note that the lifting is not uniquely defined.
The lifting maps is not assumed to be continuous.
When we talk about liftings, we assume that $\eps>0$, the inclusions $\spc{L},\spc{L}'\hookrightarrow\spc{W}$,
as well as $\spc{W}$ are chosen.

Let $\spc{L}$ be  a compact $m$-dimensional Alexandrov space.
Suppose $\spc{L}'$ is another compact $m$-dimensional Alexandrov space such that $\dist{\spc{L}}{\spc{L}'}{\GH}$ is sufficiently small --- smaller than some $\eps=\eps(\spc{L})>0$.
Then the construction in $\spc{L}$ from the previous section  
can be repeated in $\spc{L}'$ for the liftings of all points and the same function $\phi$.
It produces a strictly concave function defined in a controlled neighborhood of the lifting $p'$ of $p$.

The result of this and related constructions will be called \index{lifting}\emph{liftings},
say we can talk about a lifting from $\spc{L}$ to $\spc{L}'$ of a function provided by \ref{thm:strictly-concave} (if the Gromov--Hausdorff distance $\dist{\spc{L}}{\spc{L}'}{\GH}$ is small, then these liftings are stricly concave)
and a lifting of a convex neighborhood from \ref{cor:convex-nbhd}.
Here one cannot use \ref{thm:strictly-concave} and \ref{cor:convex-nbhd} as black boxes --- one has to understand the construction, but it is straightforward.

\section{Nerves}

Let $\{\Omega_1,\dots,\Omega_k\}$ be a finite open cover of a compact metric space $\spc{X}$.
Consider an abstract simplicial complex $\spc{N}$, with one vertex $v_i$ for each set $\Omega_i$ such that a simplex with vertices $v_{i_1},\dots, v_{i_m}$ is included in $\spc{N}$ if 
the intersection $\Omega_{i_1}\cap\dots\cap \Omega_{i_m}$ is nonempty.
\begin{figure}[ht!]
\vskip-0mm
\centering
\includegraphics{mppics/pic-1402}
\end{figure}
The obtained simplicial complex $\spc{N}$ is called the \index{nerve}\emph{nerve} of the covering $\{\Omega_i\}$.
Evidently $\spc{N}$ is a finite simplicial complex ---
it is a subcomplex of a simplex with the vertices $\{v_1,\dots,v_k\}$.
Recall that $\Star_{v_i}$ denotes the union of all simplexes in $\spc{N}$ that shares vertex $v_i$.

The next statement follows from \cite[4G.3]{hatcher}.


\begin{thm}{Nerve theorem}\label{thm:nerve}
Let $\{\Omega_1,\dots,\Omega_k\}$ be an open cover of a compact metric space $\spc{X}$
and let $\spc{N}$ be the corresponging nerve with vertices $\{v_1,\dots,v_k\}$.
Suppose that every nonempty finite intersection $\Omega_{\alpha_1}\cap\z\dots\cap\Omega_{\alpha_k}$ is contractible.
Then $\spc{X}$ is homotopy equivalent to the nerve $\spc{N}$ of the cover.

Moreover homotopy equivalences  $a\:\spc{X}\to \spc{N}$ and $b\:\spc{N}\to\spc{X}$ can be chosen so that 
if $x\in \Omega_i$, then $a(x)\in \Star_{v_i}$,
and if $y\in\spc{N}$ lies in the simplex with vertices $v_{i_1},\dots, v_{i_m}$, then $b(y)\in \Omega_{i_1}\cup\dots\cup \Omega_{i_m}$.
\end{thm}

%???Вить, посмотри на это утверждение --- оно мне не сильно нравится.


\section{Homotopy stability}

\begin{thm}{Theorem}\label{thm:h-stability}
Let $\spc{L}_1,\spc{L}_2,\dots$, and $\spc{L}_\infty$ be $m$-dimensional $\Alex\kappa$ spaces, and $m<\infty$.
Suppose $\spc{L}_n\z\GHto \spc{L}_\infty$ as $n\to \infty$.
Then $\spc{L}_\infty$ is homotopically equivalent to $\spc{L}_n$ for all large $n$.

Moreover, given $\eps>0$ there are maps $h_n\:\spc{L}_\infty\to \spc{L}_n$ that are homotopy equivalences and $\eps$-liftings for all large $n$.
\end{thm}

Applying this theorem with the Gromov's selection theorem (\ref{thm:gromov-compactness}) and Exercise \ref{ex:pack-vol}, we get the following.


\begin{thm}{Theorem}\label{thm:h-finiteness}
There are only finitely many homotopy types of $m$-dimensional $\Alex\kappa$ spaces with diameter $\le D$, and volume $\ge v_0$;
here we assume that an integer $m$, and $v_0>0$ and $D>0$ are given.
\end{thm}

\parit{Proof of \ref{thm:h-finiteness} modulo \ref{thm:h-stability}.}
Assume the contrary, then we can choose a sequence of spaces $\spc{L}_1,\spc{L}_2,\dots$ that have different homotopy types and satisfy the assumptions of the theorem.
By Gromov's compactness theorem, we can assume that $\spc{L}_n$ converges to say $\spc{L}_\infty$ in the sense of Gromov--Hausdorff.

By \ref{ex:pack-vol}, $\dim \spc{L}_\infty=m$.
It remains to apply \ref{thm:h-stability}.
\qeds

\parit{Proof of \ref{thm:h-stability}.}
Since $\spc{L}_\infty$ is compact, applying \ref{cor:convex-nbhd}, we can find a finite open cover of $\spc{L}_\infty$ by convex open sets $\Omega_1,\dots, \Omega_k$ such that 
for each $\Omega_i$ there is a strictly concave function $f_i$ that is defined in a neighborhood of $\bar \Omega_i$ and such that $f_i|_{\partial \Omega_i}=0$.

Subtracting from functions $f_i$ some small value $\eps>0$,
we can ensure that $\bigcap_{i\in S}\Omega_{i}\ne \emptyset$ if and only if $\bigcap_{i\in S}\bar\Omega_{i}\ne \emptyset$.

Suppose that $W=\bigcap_{i\in S}\Omega_{i}\ne \emptyset$.
Then $W$ is contractible.
Indeed the function 
\[f_S\df\min_{i\in S} f_i\]
is strictly concave and it vanished on the boundary of $W$.
The $f_S$-gradient flow $(t,x)\mapsto \GF_{f_S}^t(x)$ defines a homotopy
$[0,\infty)\times W\to W$.
By the first distance estimate (\ref{thm:dist-est}), $\GF_{f_S}^t(x)$ converges to the (necessarily unique) maximum point of $f_S$ as $t\to\infty$.
Therefore, in the obtained homotoly we can parametrize $[0,\infty)$ by $[0,1)$ and extend the homotopy by continiously to $[0,1]$;
thus we get that $W$ is contractible.
In other words, the cover $\{\Omega_1,\dots, \Omega_k\}$ meets the assumptions of the nerve theorem (\ref{thm:nerve}).

The functions $f_i$ and sets $\Omega_i$ can be lifted to $\spc{L}_n$ keeping their properties for all large $n$. 
More precisely, there are liftings $f_{i,n}$ of all $f_i$ to $\spc{L}_n$ which are strictly concave for all large $n$ and such that $\bar\Omega_{i,n}=\set{x\in \spc{L}_n}{f_{i,n}(x)\ge 0}$ is a compact convex set and $\Omega_{i,n}\z=\set{x\in \spc{L}_n}{f_{i,n}(x)> 0}$ is an open convex set for each $i$.

Notice that $\{\Omega_{1,n},\dots,\Omega_{k,n}\}$ is an open cover of $\spc{L}_n$ for all large~$n$.
Indeed suppose we have $p_n\in \spc{L}_n\setminus(\Omega_{1,n}\cup\dots\cup\Omega_{k,n})$ for arbitrary large $n$.
Since $\spc{L}_\infty$ is compact, there is a limit point $p_\infty\in \spc{L}_\infty$ for a subsequnce of $p_n$.
But $p_\infty\in\Omega_i$ for some $i$ and therefore $p_n\in \Omega_{i,n}$ for arbitrary large $n$ --- a contradiction.

In a similar fashion, we can show that if $n$ is large, then any collection $\{\Omega_{i,n}\}_{i\in S}$ has a common point in $\spc{L}_n$ 
if and only if $\{\Omega_{i}\}_{i\in S}$ has a common point in $\spc{L}_\infty$.
Here we have to use that $\bigcap_{i\in S}\Omega_{i}\ne \emptyset$ if and only if $\bigcap_{i\in S}\bar\Omega_{i}\ne \emptyset$.

It follows that for any large $n$ the covers 
\begin{itemize}
\item $\{\Omega_{1},\dots,\Omega_{k}\}$ of $\spc{L}_\infty$ and 
\item $\{\Omega_{1,n},\dots,\Omega_{k,n}\}$ of $\spc{L}_n$.
\end{itemize}
have the same nerve.
By the nerve theorem (\ref{thm:nerve}), $\spc{L}_n$ and $\spc{L}_\infty$ are homotopically equivalent for all large $n$ --- a contradiction.
\qeds

\section{Comments}

Gromov's selection theorem provides the main source of applications of Alexandrov spaces to Riemannian geometry.
The homotopy-type finiteness theorem (\ref{thm:h-finiteness})  illustrates this technique.

Originally, Gromov's selection theorem was proved for Riemannian manifolds with a lower bound on Ricci curvature \cite{gromov1981}.
It motivates the study of the so-called $\mathrm{CD}(K,m)$ spaces; $\mathrm{CD}$ stands for curvature-dimension condition.
This theory has serious applications in Alexandrov geometry;
in particular, it provides a version of Liouville theorem about phase-space volume of geodesic flow in Alexandrov space \cite{brue-mondino-semola}.

The construction of strictly concave function is due to Grigory Perelman \cite{perelman1993,perelman-petrunin}.

Let us list some results that can be proved by applying Gromov's selection theorem
in the same fashion as in the proof of homotopy-type finiteness theorem (\ref{thm:h-finiteness}).

\begin{thm}{Betti-number theorem}
There is a constant $\Const=\Const(m,D,\kappa)$ such that 
\[\beta_0(M)+\beta_1(M)+\dots+\beta_m(M)\le \Const\]
for any closed $m$-dimensional Riemannian manifold $M$ with sectional curvature $\ge \kappa$ and diameter $\le D$.
Here $\beta_i(M)$ denotes $i^\text{th}$ Betti number of $M$.
\end{thm}

Gromov's original proof \cite{gromov-1981} of the Betti-number theorem did not use Alexandrov geometry directly;
but it is quite natural to prove it via Gromov's selection theorem.
The following result proved the second author \cite{petrunin2008}, and it uses the same technique.

\begin{thm}{Scalar curvature bound}
There is a constant $\Const=\Const(m,D,\kappa)$ such that 
\[\int_M\Sc\le \Const\]
for any closed $m$-dimensional Riemannian manifold $M$ with sectional curvature $\ge \kappa$ and diameter $\le D$.
Here $\Sc$ denotes the scalar curvature.
\end{thm}

The following theorem is a more exact version of \ref{thm:h-stability}.
Its close relative (\ref{thm:spherical-nbhd}) will play an important role in the following lecture.

\begin{thm}{Stability theorem}\label{thm:stability}
Let $\spc{L}_1,\spc{L}_2,\dots$, and $\spc{L}_\infty$ be  $m$-dimensional $\Alex\kappa$ spaces, and $m<\infty$.
Suppose $\spc{L}_n\GHto \spc{L}_\infty$ as $n\to \infty$.
Then $\spc{L}_\infty$ is homeomorphic to $\spc{L}_n$ for all large $n$.

Moreover, given $\eps>0$ there are maps $h_n\:\spc{L}_\infty\to \spc{L}_n$ that are homeomorphisms and $\eps$-liftings for all large $n$.
\end{thm}

This theorem was proved by Grigory Perelman \cite{perelman1991};
the proof was rewritten with more details by the first author \cite{kapovitch}.
Perelman have made an informal annoncemnt that the homeomorphisms in the theorem can be assumed to be bi-Lipschitz with constants that depend on $\spc{L}_\infty$;
he refused to write the proof, and so it save to consider it as a conjecture.

The last statement in the theorem implies the following finiteness result.

\begin{thm}{Homeomorphism-type finiteness}
There are only finitely many homeomorphism types of closed $m$-dimensional manifolds that admit a Riemannian metric with sectional curvature $\ge \kappa$, and diameter $\le D$.
\end{thm}

Applying several results in differential topology, this statement can be improved to diffeomorphism-type finiteness in all dimensions $m$ except $m=4$; see \cite{kirby-siebenmann} and  \cite{moise,thurston} for cases $m\ge 5$ and $m\le 3$, respectively.



%%!TEX root = the-boundary.tex
\chapter{Boundary}\label{chap:bry}

This lecture defines the boundary of a finite-dimensional Alexandrov space.
After discussing its properties, we prove the doubling theorem (\ref{thm:doubling:doubling}).

\section{Definition}

Let us give an inductive definition of the boundary of finite-dimensional Alexandrov spaces.

Suppose $\spc{A}$ is a 1-dimensional Alexandrov space.
By Exercise~\ref{ex:dim=1},
$\spc{A}$ is homeomorphic to a 1-dimensional manifold (possibly with non-empty boundary).
This  allows us to define the boundary $\partial\spc{A}\subset \spc{A}$ as the boundary of the manifold.

Now assume that the notion of boundary is defined in dimensions $1,\dots,m-1$.
Suppose  $\spc{A}$ is $m$-dimensional Alexandrov space.
We say that $p\in \spc{A}$ belongs to the boundary (briefly $p\in \partial \spc{A}$) if 
$\partial\Sigma_p\ne\emptyset$.
By \ref{thm:finite-space-of-directions} and \ref{ex:finite-space-of-directions-dim}, $\Sigma_p$ is an $(m-1)$-dimensional Alexandrov space;
therefore its boundary is already defined and hence this inductive definition makes sense.

It is instructive to check the following statements.
\begin{itemize}
\item For a closed convex set $K\subset \EE^m$ with non-empty interior, the topological boundary of $K$ as a subset of $\EE^m$ coincides with the boundary $K$ described above.
\item If $\spc{A}\iso\spc{A}_1\times\spc{A}_2$ is a finite-dimensional Alexandrov space,
then
\[\partial \spc{A}=(\partial\spc{A}_1\times\spc{A}_2)\,\cup\,(\spc{A}_1\times\partial\spc{A}_2)\]
\item If $\Cone\Sigma$ is an $\Alex0$ space of dimensions $\ge 2$  (this necessarily implies that   $\Cone\Sigma$  is  $\Alex1 $ then
\[\partial \Cone\Sigma=\Cone\partial\Sigma,\]
where $\Cone\partial\Sigma=\set{s\cdot \xi\in\Cone\Sigma }{\xi\in \partial\Sigma}$.
\end{itemize}


\section{Conic neighborhoods}

The following statement \cite{perelman1993} is a close relative of Perelman's stability theorem \ref{thm:stability}.
% but its proof is  simpler .
We are going to use this result without proof.

Recall that the logarithm $\log_px\:\spc{A}\to \T_p$ is defined on page \pageref{page:log}.

\begin{thm}{Theorem}\label{thm:spherical-nbhd}
For any point $p$ in a finite-dimensional Alexandrov space $\spc{A}$
and all sufficiently small $\eps>0$
there is a homeomorphism $h_\eps\:\oBall(p,\eps)_{\spc{A}}\to \oBall(0,\eps)_{\T_p}$ such that $0=h_\eps(p)$.

Moreover, we may assume that
\[
\sup_{x\in \oBall(p,\eps)}\{\,\tfrac1\eps\cdot\dist{\log_px}{h_\eps(x)}{\T_p}\,\}\to 0
\quad\text{as}\quad
\eps\to 0.\]
\end{thm}
Note that the last condition automatically implies that  $h_\eps$ as an $o(\eps)$ G-H approximation.

The above theorem is often used together with the \textit{uniqueness of conic neighborhoods} stated below.

Suppose that an open  neighborhood $U$ of a point $x$ in a metric space $\spc{X}$
% openness is very important. it's false otherwise
admits a homeomorphism to $\Cone\Sigma$ such that $x$ is mapped to the origin of the cone.
In this case, we say that $U$ has a \index{conic neighborhood}\emph{conic neighborhood} of~$x$.

\begin{thm}{Uniqueness of conic neighborhoods}\label{lem:kwun}
Any two conic neighborhoods of a given point in a metric space are \index{pointed homeomorphic}\emph{pointed homeomorphic}; that is, there is a homeomorphism between neighborhoods that maps the origin of one cone to the origin of the other.
\end{thm}

\begin{thm}{Advanced exercise}\label{ex:conic}
Prove \ref{lem:kwun} or read the proof in \cite{kwun1964}.
\end{thm}


\begin{thm}{Exercise}\label{ex:conic-tangent}
Suppose $x\mapsto x'$ is a homeomorphism between finite-dimensional Alexandrov spaces $\spc{A}$ and $\spc{A}'$. Show that 

\begin{subthm}{ex:conic-tangen:tangent}
$\T_x\cong \T_{x'}$ (here and below $\cong $ means homeomorphic) 
\end{subthm}

\begin{subthm}{ex:conic-tangen:dir}
$\Susp\Sigma_x\cong \Susp\Sigma_{x'}$.
\end{subthm}

\begin{subthm}{ex:conic-tangen:example}
but in general $\Sigma_x\ncong\Sigma_{x'}$.
\end{subthm}

\end{thm}



\section{Topology}

The following theorem states that boundary is a topological invariant, despite our definition having used geometry.

\begin{thm}{Theorem}\label{thm:top-bry}
Let $\spc{A}$ and $\spc{A}'$ be homeomorphic finite-dimensional Alexandrov spaces.
Then $\dim \spc{A}=\dim\spc{A}'$ and
\[\partial\spc{A}\ne \emptyset
\quad\iff\quad
\partial\spc{A}'\ne \emptyset
\]
\end{thm}

While working on the proof, keep in mind that there are pairs of spaces $\spc{K}_1$ and $\spc{K}_2$ such that $\spc{K}_1\ncong \spc{K}_2$, but $\RR\times \spc{K}_1\cong \RR\times \spc{K}_2$.
Suspension over the Poincaré homology sphere with $\SSS^4$ is one of the examples; compare to \ref{ex:conic-tangen:example}.

Let $\spc{A}$ be an $m$-dimensional Alexandrov space and $m<\infty$.
Define \index{rank}\emph{rank} of $\spc{A}$ (briefly, \index{$\rank\spc{A}$}$\rank\spc{A}$) as the minimal value $k$ such that $\spc{A}$ splits isometrically as $\RR^{m-k}\times \spc{K}$;
here $\spc{K}$ is a $k$-dimensional Alexandrov space.

In the following proof we will apply induction on the rank of $\spc{A}$.


\parit{Proof.}
The first statement follows from \ref{thm:dim=dim}.

Suppose we have a counterexample, say $\partial \spc{A}\ne \emptyset$, but $\partial \spc{A}'=\emptyset$.
Let $k\df\rank \spc{A}$ and $k'\df\rank \spc{A}'$.
We can assume that the pair $(k,k')$ is minimal in lexicographic order;
in particular, $k$ is minimal.
Let $x\mapsto x'$ be a homeomorphism from $\spc{A}$ to $\spc{A}'$.

Choose $x\in \partial \spc{A}$.
Since $\partial \spc{A}'=\emptyset$, we have $x'\notin \partial \spc{A}'$.
Note that 
\[\rank \T_x\le k
\quad\text{and}\quad
\rank \T_{x'}\le k',
\]
By \ref{ex:conic-tangen:tangent}, $\T_x\cong\T_{x'}$.
Note that $\partial \T_x\ne\emptyset$ and $\partial \T_{x'}=\emptyset$.
Therefore, we may assume that $\spc{A}$ and $\spc{A}'$ are Euclidean cones
and the homeomorphism sends the origin to the origin.
The remaining part of the proof is divided into three cases.

\parit{Case 1.}
Suppose $k>1$.
Let $\spc{A}\iso \RR^{m-k}\times \spc{C}$, where $\spc{C}$ a $k$-dimensional $\Alex0$ cone.
Observe that $\rank\T_y\le\rank\spc{A}$ for any $y\in\spc{A}$ and the equality holds only if $y$ projects to the origin of $\spc{C}$.

Since $k>1$ we can find $z\in\partial\spc{C}$ such that $z\ne 0$.
Choose $y$ that projects to $z$;
in particular, $\rank\T_y<\rank\spc{A}$.
By \ref{ex:conic-tangen:tangent}, $\T_y\cong\T_{y'}$,
$\partial  \T_y\ne\emptyset$ and $\partial \T_{y'}=\emptyset$.
The latter contradicts the minimality of $k$.

\parit{Case 2.} Suppose $k\le1$ and $k'>1$.
Since $\partial \spc{A}\ne \emptyset$, we get that $k=1$;
therefore, $\spc{A}=\RR^{m-1}\times\RR_{\ge0}$.

Let $\spc{A}'\iso \RR^{m-k'}\times \spc{C}'$, where $\spc{C}'$ a $k'$-dimensional $\Alex0$ cone.
Since $\partial\spc{A}\cong\RR^{m-1}$,
the image of $\partial\spc{A}$ in $\spc{A}'$ does not lie in $\RR^{m-k'}\z\times\{0\}$.
In other words, we can choose $y\in \partial \spc{A}$ such that its image $y'\in \spc{A}'$ has a nonzero projection in $\spc{C}'$.
Observe that $\T_y\cong\T_{y'}$,
\[
\rank\T_y\le k=1,
\quad
\rank\T_{y'}< k',
\quad
\partial \T_y=\emptyset,
\quad\text{and}\quad
\partial \T_{y'}\ne \emptyset\]
--- a contradiction.

\parit{Case 3.}
Suppose $k\le 1$ and $k'\le 1$.
Since $\partial \spc{A}\ne \emptyset$, $k=1$.
By \ref{ex:dim=1}, $\spc{A}\z\cong \RR^{m-1}\times\RR_{\ge0}$.
Therefore, $\spc{A}'\cong\RR^m$, and $\spc{A}\ncong\spc{A}'$ --- a contradiction.
\qeds

\begin{thm}{Exercise}\label{ex:bry2bry}
Let $x\mapsto x'$ be a homeomorphism $\Omega\to\Omega'$
between open subsets in finite-dimensional Alexandrov spaces $\spc{A}$ and $\spc{A}'$.
Show that $x\in \partial \spc{A}$ if and only if $x'\in \partial \spc{A}'$.

\end{thm}

\begin{thm}{Exercise}\label{ex:bry-closed}
Show that boundary of a finite-dimensional Alexandrov space is a closed subset.
\end{thm}

\section{Tangent space}

Spaces of directions and tangent spaces of an Alexandrov space have already been defined in \ref{sec:space+directions} and \ref{sec: tangent space}.
Let us extend these definitions to subsets of an Alexandrov space.

Let $X$ be a subset in a finite-dimensional Alexandrov space $\spc{A}$.
Choose $p\in \spc{A}$ and $\xi\in \Sigma_p$.
Suppose $\xi$ is a limit of directions $\dir{p}{x_n}$ for a sequence $x_1,x_2,\dots{}\in X$ that converges to $p$.
Then we say that $\xi$ is in the \index{space of directions}\emph{space of directions} from $p$ to $X$;
briefly \index{$\Sigma_p$ (space of directions)}$\xi\in\Sigma_pX$.

Further, $\Cone(\Sigma_pX)$ will be called the \index{tangent space}\emph{tangent space} to $X$ at $p$;
it will be denoted by \index{$\T_p$ (tangent space)}$\T_pX$.

Note that $\Sigma_pX$ is a subset of $\Sigma_p$ and $\T_pX$ is a subcone in $\T_p$

\begin{thm}{Theorem}\label{thm:partial-Sigma}
For any finite-dimensional Alexandrov space $\spc{A}$, we have
\[\partial (\Sigma_p\spc{A})=\Sigma_p(\partial\spc{A})
\quad\text{and}\quad
\partial(\T_p\spc{A})=\T_p(\partial\spc{A}).\]
\end{thm}

\parit{Proof.}
Choose a sequence $x_n\in \partial \spc{A}$ such that $x_n\to p$ and $\dir p{x_n}\to\xi$.

Let $\eps_n=2\cdot \dist{p}{x_n}{}$,
and let $h_{\eps_n}\:\oBall(p,\eps_n)_{\spc{A}}\to \oBall(0,\eps_n)_{\T_p}$ be the homeomorphisms provided by \ref{thm:spherical-nbhd};
in particular, $\tfrac2{\eps_n}\cdot h_{\eps_n}(x_n)\to \xi$ as $n\to\infty$.
By \ref{ex:bry2bry}, $h_{\eps_n}(x_n)\in \partial \T_p$.
By \ref{ex:bry-closed}, $\xi\in \partial \T_p$.
Therefore,
\[\partial (\Sigma_p\spc{A})\supset\Sigma_p(\partial\spc{A})
\quad\text{and}\quad
\partial(\T_p\spc{A})\supset\T_p(\partial\spc{A}).\]

Similarly, choose $\xi\in\partial\Sigma_p$.
Let $h_{\eps_n}\:\oBall(p,\eps_n)_{\spc{A}}\to \oBall(0,\eps_n)_{\T_p}$ be the homeomorphisms provided by \ref{thm:spherical-nbhd} for a sequence $\eps_n\to 0$ as $n\to\infty$.
By \ref{ex:bry2bry}, $x_n=h_{\eps_n}^{-1}(\tfrac{\eps_n}2\cdot\xi)\in \partial \spc{A}$.
By \ref{thm:spherical-nbhd}, $\dir p{x_n}\to \xi$.
Hence
\[\partial (\Sigma_p\spc{A})\subset\Sigma_p(\partial\spc{A})
\quad\text{and}\quad\partial(\T_p\spc{A})\subset\T_p(\partial\spc{A}).\]
\qedsf

\section{Doubling}

Let $A$ be a closed subset in a metric space $\spc{X}$.
The \index{doubling}\emph{doubling} $\spc{W}$ of $\spc{X}$ across $A$ is two copies of $\spc{X}$ glued along $A$;
more precisely, the underlying set of $\spc{W}$ is the quotient $\spc{X}\times\{0,1\}/\sim$, where $(a,0)\sim (a,1)$ for any $a\in A$ and $\spc{W}$ is equipped with the minimal metric such that both maps $\spc{X}\to \spc{W}$ defined by $x\mapsto (x,0)$ and $x\mapsto (x,1)$ are distance-preserving.

Alternatively, one may say that $\spc{W}$ is equipped with the maximal metric such that the projection $\proj\:\spc{W}\to\spc{A}$ defined by $(x,i)\mapsto x$ is a short map. 
The metric on $\spc{W}$ can also be defined explicitly as
\[\dist{(x,i)}{(y,j)}{\spc{W}}=
\begin{cases}
\dist{x}{y}{\spc{X}}&\text{if}\quad i= j.
\\
\inf\set{\dist{x}{a}{\spc{X}}+\dist{y}{a}{\spc{X}}}{a\in A}&\text{if}\quad i\ne j.
\end{cases}
\]

\begin{thm}{Theorem}\label{thm:doubling}
Let $\spc{A}$ be a finite-dimensional Alexandrov space with non-empty boundary.
Suppose $f\z=\tfrac12\cdot\distfun_p^2$ for some $p\in \spc{A}$.
Then

\begin{subthm}{thm:doubling:concave}
If $\dim \spc{A}\ge 2$, then
$\distfun_{\partial \Sigma_x}(\xi)\le \tfrac\pi2$ for any $x\in\partial \spc{A}$ and $\xi\in \Sigma_x$.
Moreover, if $\distfun_{\partial \Sigma_x}(\xi)= \tfrac\pi2$, then $\mangle(\xi,\zeta)\le\tfrac\pi2$ for any $\zeta\in \Sigma_x$. 
\end{subthm}

\begin{subthm}{thm:partial-grad:grad}
$\nabla_xf\in \partial\T_x$ for any $x\in\partial \spc{A}$.
\end{subthm}

\begin{subthm}{thm:partial-grad:flow}
If $\alpha$ is an $f$-gradient curve that starts at $x\in \partial \spc{A}$, then $\alpha(t)\in \partial \spc{A}$ for any $t$.
Moreover, if $p\in \partial \spc{A}$, then $\gexp_p(v)\in \partial \spc{A}$ for any $v\in\partial\T_p$.
\end{subthm}

\begin{subthm}{thm:doubling:doubling}
The doubling $\spc{W}$ of $\spc{A}$ across $\partial \spc{A}$ is an Alexandrov space with the same curvature bound.
\end{subthm}

\end{thm}

Part \ref{SHORT.thm:doubling:doubling} is called the \index{doubling theorem}\emph{doubling theorem}.

\parit{Proof.}
We will denote by 
\ref{SHORT.thm:doubling:concave}$_m,\dots,$\ref{SHORT.thm:doubling:doubling}$_m$ the corresponding statement assuming $m=\dim\spc{A}$.

The proof goes by induction on $m$.
Statement \ref{SHORT.thm:doubling:doubling}$_1$ follows from \ref{ex:dim=1} --- this is the base.
The induction step is a combination of the implications below.

\parit{\ref{SHORT.thm:doubling:doubling}$_{m-1}\Rightarrow$\ref{SHORT.thm:doubling:concave}$_m$.}
Suppose $m=2$, then $\dim\Sigma_x=1$; see \ref{ex:finite-space-of-directions-dim}.
By \ref{ex:dim=1}, $\Sigma_x$ isometric to a line segment $[0,\ell]$;
we need to show that $\ell\le\pi$.

Assume $\ell>\pi$, then the tangent space $\T_x=\Cone\Sigma_x$ has several different lines thru the origin.
Recall that $\T_x$ is an Alexandrov space; see \ref{ex:finite-tan}.
By \ref{cor:splitting}, $\T_x$ is isometric to the Euclidean plane;
the latter contradicts that $\Sigma_x$ is a line segment.

Now suppose $m>2$, so $\dim \Sigma_x>1$.
Assume $\distfun_{\partial \Sigma_x}(\xi)> \tfrac\pi2$ for some $\xi$.
By \ref{SHORT.thm:doubling:doubling}$_{m-1}$, the doubling $\Xi$ of $\Sigma_x$ is $\Alex1$.
Denote by $\xi_0$ and $\xi_1$ the points in $\Xi$ that correspond to $\xi$.
Observe that $\dist{\xi_0}{\xi_1}{\Xi}>\pi$.
The latter contradicts \ref{ex:RisCBB(1)}.

Finally, if $\distfun_{\partial \Sigma_x}(\xi)= \tfrac\pi2$, then $\dist{\xi_0}{\xi_1}{\Xi}=\pi$.
Therefore, $\Cone \Xi$ contains a line in the directions of $\xi_0$ and $\xi_1$;
in other words, $\Xi$ is a spherical suspension with poles $\xi_0$ and $\xi_1$.
In particular, every point of $\Xi$ lies on distance at most $\tfrac\pi2$ from $\xi_0$ or $\xi_1$.
The natural projection $\Xi\to \Sigma_x$ does not increase distances and sends both  $\xi_0$ and $\xi_1$ to $\xi$.
Therefore, the second statement of \ref{SHORT.thm:doubling:concave}$_m$ follows.

\parit{\ref{SHORT.thm:doubling:doubling}$_{m-1}+$\ref{SHORT.thm:doubling:concave}$_{m-1}+$\ref{SHORT.thm:doubling:concave}$_m\Rightarrow$\ref{SHORT.thm:partial-grad:grad}$_m$.}
We can assume that $s=\nabla_xf\ne 0$.
By \ref{prop:grad-exist}, $\nabla_xf\z=s\cdot \overline{\xi}$, where $s=\dd_xf(\overline{\xi})>0$ and $\overline{\xi}\in\Sigma_p$ is the direction that maximizes $\dd_xf(\overline{\xi})$.

Let $\zeta\in \partial\Sigma_x$ be a direction that minimizes the angle $\mangle(\overline{\xi},\zeta)$.
It is sufficient to show that $\zeta=\overline{\xi}$.

Assume $\zeta\ne \overline{\xi}$;
let $\eta=\dir[\Sigma_x]\zeta{\overline{\xi}}$.
By \ref{SHORT.thm:doubling:concave}$_m$, $\mangle(\overline{\xi},\zeta)\le \tfrac\pi2$ and
\ref{SHORT.thm:doubling:concave}$_{m-1}$ implies that 
\[\mangle(\eta,\nu)\le \tfrac\pi2\eqlbl{eq:<pi/2}\]
for any $\nu\in \Sigma_\zeta\Sigma_x$ (if $m=2$, then the last statement is evident). 

Let $\phi\:\Sigma_x\to\RR$ be restriction of $\dd_xf$ to $\Sigma_x$.
Applying \ref{ex:d(distfun):<} and \ref{eq:<pi/2}, we get that $\dd_{\bar \xi}\phi(\eta)\le 0$.
Since $\dd_xf$ is concave, we have that $\phi''+\phi\le 0$.
If $\phi(\zeta)\le 0$, then it implies that $\phi(\overline{\xi})\le 0$ --- a contradiction to the fact that $s>0$.
If $\phi(\zeta)> 0$, then $\phi(\overline{\xi})<\phi(\zeta)$ --- a contradiction again.

\parit{\ref{SHORT.thm:partial-grad:grad}$_m\Rightarrow$\ref{SHORT.thm:partial-grad:flow}$_m$.}
Let $\alpha$ be an $f$-gradient curve and $\ell(t)=\distfun_{\partial \spc{A}}\alpha(t)$.

Choose $t$;
let $x=\alpha(t)$ and $y\in \partial\spc{A}$ be a closest point to $x$.
By \ref{SHORT.thm:partial-grad:grad}$_m$, we have that $\nabla_y f\in\partial \T_y$.
Since the distance $\dist{x}{y}{}$ is minimal, 
we get $\langle \dir yx,v\rangle\le 0$ for any $v\in \partial \T_y$.
In particular,
\[\langle \dir yx,\nabla_y f\rangle\le 0\]
Applying Exercise~\ref{ex:monotonicity} to $x$ and $y$, 
we get
\[\ell'(t)\le \ell(t)\]
if the left-hand side is defined.
Since $\ell$ is Lipschitz, $\ell'$ is defined almost everywhere.
Integrating the inequality, we get 
\[\ell(t)\le e^t\cdot\ell(0)\]
for any $t\ge 0$.
In particular, if $\ell(0)=0$, then $\ell(t)=0$ for any $t\ge 0$.
Since $\partial\spc{A}$ is closed (\ref{ex:bry-closed}), the statement follows.

\parit{\ref{SHORT.thm:partial-grad:flow}$_{m}+$\ref{SHORT.thm:doubling:doubling}$_{m-1}\Rightarrow$\ref{SHORT.thm:doubling:doubling}$_m$.}
We will consider the case $\kappa=0$;
other cases can be done in the same way, but formulas get more complicated.

Denote by $\spc{A}_0$ and $\spc{A}_1$ the two copies of $\spc{A}$ in $\spc{W}$;
let us keep the notation $\partial \spc{A}$ for the common boundary of $\spc{A}_0$ and $\spc{A}_1$.

\begin{clm}{}
Let $\gamma$ be a geodesic in $\spc{W}$.
Then either $\gamma$ has at most one interior point in $\partial \spc{A}$ or
$\gamma\subset \partial \spc{A}$.
\end{clm}

\begin{wrapfigure}{r}{45mm}
\vskip-2mm
\centering
\includegraphics{mppics/pic-1315}
\end{wrapfigure}

Indeed, assume $\gamma$ shares at least two points with $\partial \spc{A}$, say $x=\gamma(t_1)$ and $y=\gamma(t_2)$ and these are not endpoints of $\gamma$.
Remove from $\gamma$ the set $\gamma\cap \spc{A}_1$
and exchange it to its reflection across $\partial\spc{A}$;
denote the obtained curve by $\hat\gamma$.

Any arc of $\hat\gamma$ with one endpoint in $\partial \spc{A}$
is a geodesic in $\spc{A}_0$.
Since $x,y\in \partial \spc{A}$, the arc of $\hat\gamma$ behind $y$ lies in the image of map $t\mapsto \GF^t_{f_x}(y)$, where $f_x=\tfrac12\cdot\distfun^2_x$.
By \ref{SHORT.thm:partial-grad:flow}, this arc lies in $\partial\spc{A}$.

Now choose a point $z$ on this arc, so $z\in \partial\spc{A}$.
Applying the same argument, we get that the arc of $\hat\gamma$ before $y$ lies in $\partial\spc{A}$.
Hence the claim follows.\claimqeds

Choose a point $p$ in $\spc{W}$;
let $f\df\tfrac12\cdot\distfun_p^2$.
It is sufficient to show that $(f\circ\gamma)''\le 1$ for any $t$.
If $p\in \partial\spc{A}$, then the statement follows from function comparison in $\spc{A}_0$ and $\spc{A}_1$.
So, we can assume that $p\in \spc{A}_0\setminus \partial\spc{A}$.

If $\gamma$ lies in $\partial \spc{A}$, then this inequality follows from the comparison in~$\spc{A}_0$.

\begin{wrapfigure}{r}{55mm}
\vskip-2mm
\centering
\includegraphics{mppics/pic-1325}
\end{wrapfigure}

Choose $y=\gamma(t_0)$; without loss of generality we can assume that $t_0=0$.

If $y\z\in \spc{A}_0\setminus\partial\spc{A}$, then $(f\z\circ\gamma)''(0)\le 1$ in the barrier sense;
it follows from the comparison in $\spc{A}_0$.

Assume $y\in \spc{A}_1\setminus\partial\spc{A}$.
Suppose $[py]$ crosses $\partial\spc{A}$ at $x$.
Let $\Sigma_x$ be the space of directions of $\spc{A}$ at $x$,
and let $\Xi$ be its doubling.
As before, we denote by $\Sigma_0$ and $\Sigma_1$ two copies of $\Sigma_x$ in  $\Xi$
and keep notation $\partial\Sigma_x$ for their common boundary.
By \ref{SHORT.thm:doubling:doubling}$_{m-1}$, $\Xi$ is $\Alex1$.

The directions $\dir x{y}$ and $\dir xp$ lie on opposite sides from $\Xi$ and
\[\dist{\dir x{y}}{\dir xp}\Xi\ge \pi.\]
Otherwise, we could choose a direction $\xi\in\partial\Sigma$ such that
\[\dist{\dir x{y}}{\xi}\Xi+\dist{\xi}{\dir xp}\Xi<\pi.\]
Furthermore, we could consider the radial curve $\alpha(t)=\gexp_x(t\cdot \xi)$.
By \ref{SHORT.thm:partial-grad:flow}$_m$, $\alpha$ lies in $\partial \spc{A}$.
By \ref{prop:gexp}
\[\dist{p}{\alpha(s)}{\spc{A}_0}
+\dist{y}{\alpha(s)}{\spc{A}_1}
<\dist{p}{y}{\spc{W}}\]
for small values $s>0$
--- a contradition.

$\Cone \Xi$ contains a line with directions $\dir x{y}$ and $\dir xp$.
By the splitting theorem, $\Cone \Xi$ split in these directions;
in particular, 
\[\dist{\dir x{y}}{\xi}{}+\dist{\xi}{\dir xp}{}=\pi.\]
for any $\xi\in\Xi$.
It follows that for any $\xi\in\Xi$ there is $\xi'\in\partial\Sigma_x$ such that 
$\xi$ and $\xi'$ lie on some geodesic $[\dir x{y} \dir xp]_\Xi$.

Fix $t\approx 0$ such that $t\ne 0$; let $z=\gamma(t)$.
Choose such $\xi'$ for $\xi=\dir xz$.
Consider the radial curve $\alpha(s)\df\gexp_x(s\cdot\xi')$.
Let us show that 
\[
\begin{aligned}
\dist{p}{z}{\spc{W}}
&\le \dist{p}{\alpha(s)}{\spc{A}_0}+ \dist{\alpha(s)}{z}{\spc{A}_1}\le
\\
&\le\side\hinge yp{z}.
\end{aligned}
\eqlbl{eq:gamma''}
\]
for suitable value $s$.

The first inequality in \ref{eq:gamma''} is evident.
Set $\phi=\mangle\hinge{x}{y}{z}$ and $\psi\z=\mangle(\dir xp,\xi')$.
The choice of $s$ comes from the model configuration $\tilde p$, $\tilde x$, $\tilde y$, $\tilde w$, $\tilde z\in \EE^2$ such that
\begin{align*}
\tilde x&\in [\tilde p\tilde y],
&
\dist{\tilde p}{\tilde x}{}&=\dist{ p}{x}{},
&
\dist{\tilde p}{\tilde y}{}&=\dist{p}{y}{},
&
\dist{\tilde x}{\tilde z}{}&=\dist{x}{z}{},
\\
\tilde w&\in [\tilde p\tilde z],
&
\mangle\hinge{\tilde x}{\tilde y}{\tilde z}&=\phi,
&
\mangle\hinge{\tilde x}{\tilde p}{\tilde w}&=\psi, 
&
s&=\dist{\tilde x}{\tilde w}{}.
\end{align*}
\begin{figure}[ht!]
\vskip-0mm
\centering
\includegraphics{mppics/pic-1014}
\end{figure}

\noindent
By \ref{prop:gexp}, we get 
\begin{align*}
\dist{p}{\alpha(s)}{\spc{A}_0}&\le \dist{\tilde p}{\tilde w}{},
\\
\dist{\alpha(s)}{z}{\spc{A}_1}&\le\dist{\tilde w}{\tilde z}{};
\end{align*}
by the comparison, 
\[\dist{\tilde p}{\tilde z}{}\le \side\hinge ypz.\]

\begin{thm}{Exercise}\label{ex:pz<ypz}
Prove the last inequality.
\end{thm}

Hence we get $(f\circ\gamma)''(0)\le 1$ in the barrier sense.

Finally if $\gamma(0)\in\partial\spc{A}$, then splitting argument shows that 
\[(f\circ\gamma)^+(0)+(f\circ\gamma)^-(0)\le 0.\]

Summarizing, we get that $(f\circ\gamma)''\le 1$ on every arc of $\gamma$ that lies entirely in $\spc{A}_0$ or $\spc{A}_1$.
If $\gamma$ crosses $\partial \spc{A}$, then we know that it happens only once and at the crossing moment $t_0$ 
we have $f\circ\gamma^+(t_0)\z+f\circ\gamma^-(t_0)\z\le 0$.
All this implies that $(f\circ\gamma)''\le 1$.
\qeds

\begin{thm}{Exercise}\label{ex:bry-connected}
Let $\spc{A}$ be a finite-dimensional $\Alex1$ space of dimension $\ge 2$ with non-empty boundary $\partial\spc{A}$.
Show that $\partial\spc{A}$ is connected.
\end{thm}


\begin{thm}{Exercise}\label{ex:dist-to-bry}
Let $\spc{A}$ be an $m$-dimensional $\Alex0$ space with non-empty boundary $\partial\spc{A}$
for $2\le m<\infty$.
Show that the distance function to the boundary
\[\distfun_{\partial\spc{A}}\:\spc{A}\to\RR\]
is concave.
\end{thm}

\begin{thm}{Exercise}\label{ex:liberman}
Let $\spc{A}$ be a finite-dimensional $\Alex0$ space with non-empty boundary $\partial\spc{A}$.
Suppose $\gamma$ is a geodesic in $\partial\spc{A}$ with the induced length metric.
Show that the function $t\mapsto \tfrac12\cdot\distfun_p^2\circ\gamma(t)$ is 1-concave for any point $p$. 
\end{thm}

\begin{thm}{Exercise}\label{ex:native}
Let $\spc{W}$ be a doubling of finite-dimensional Alexandrov space $\spc{A}$ across its boundary,
and let $proj\:\spc{W}\to\spc{A}$ be the natural projection.
Suppose $f\:\spc{A}\to\RR$ is a $\lambda$-concave function.
Show that $f\circ\proj\:\spc{W}\to\RR$ is $\lambda$-concave if and only if $\nabla_xf\in \partial \T_x$ 
for any $x\in\partial \spc{A}$.
\end{thm}



\section{Remarks}

It easily follows by induction on dimension  that the doubling of a finite-dimensional Alexandrov space across its boundary results in an Alexandrov space without boundary.
This observation can often be used to reduce a statement about general finite-dimensional Alexandrov spaces to  Alexandrov spaces without boundary.

For spaces without boundary the following tools become available.

\begin{thm}{Fundamental-class lemma}\label{lem:fund-class}
Any compact finite-dimensional Alexandrov space $\spc{A}$ without boundary has a fundamental class with $\ZZ/2$ coefficients;
that is, if $\spc{A}$ is $m$-dimensional, then
\[H^m(\spc{A},\ZZ/2)=\ZZ/2.\]

\end{thm}

This lemma was proved by Karsten Grove and Peter Petersen \cite{grove-petersen1993}.
Originally it was stated for Alexander--Spanier cohomology. We do not make this distinction  because for compact Alexandrov spaces it is the same as singular cohomology. Indeed,  both cohomology theories are homotopy invariant \cite[Chapter 6]{Spanier}, compact Alexandrov spaces are homotopy equivalent to finite simplicial complexes \ref{thm:finite-dim-hom-simplicial} and  for paracompact  CW complexes  Alexander--Spanier cohomology is isomorphic to \v{C}ech  and singular cohomolgy \cite[Chapter 6]{Spanier}.

This lemma implies, for example, that on finite-dimensional Alexandrov spaces without boundary 
the gradient flow for a $\lambda$-concave function is an onto map;
in other words, gradient curves can be extended into the past.
It is also used in the proof of the following version of the domain invariance theorem \cite[Theorem 3.2]{kapovitch-zhu}.

\begin{thm}{Domain invariance}\label{thm-inv-domain}
Let $\spc{A}_1$ and $\spc{A}_2$ be two $m$-dimensional Alexandrov spaces with empty boundary; $m$ is finite.
Suppose $\Omega_1$ is an open subset in $\spc{A}_1$ and $f\:\Omega_1\to \spc{A}_2$ is an injective continuous map.
Then $f(\Omega_1)$ is open in $\spc{A}_2$.
\end{thm}

Theorem~\ref{thm:spherical-nbhd} can be used to prove the following. 

\begin{thm}{Topological stratification}\label{thm:top-stratification}
Any $m$-dimensional Alexandrov space with $m<\infty$ can be subdivided into topological manifolds $S_0,\z\dots,S_m$ such that for every $i$ we have $\dim S_i=i$ or $S_i=\emptyset$.
Moreover,
\begin{subthm}{}
the closure of $S_{m-1}$ is the boundary of the space, and
\end{subthm}

\begin{subthm}{}
$S_{m-2}=\emptyset$.
\end{subthm}

\end{thm}

Let us mention that this statement implies that a compact finite-dimensional Alexandrov space has the homotopy type of a finite CW complex,
but it seems to be unknown if it has to be homeomorphic to a CW complex.

The stratification theorem~\ref{thm:top-stratification} can be sharpened as follows.

\begin{thm}{Boundary characterization}
Let $\spc{A}$ be an $m$-dimensional Alexandrov space with $m<\infty$.
Then the following statements are equivalent.

\begin{subthm}{item-boundary} $p\in \partial \spc{A}$;
\end{subthm}

\begin{subthm}{item-contractible} $\Sigma_p$ is contractible;
\end{subthm}

\begin{subthm}{item-space-dir-homology} $\tilde H_{m-1}(\Sigma_p,\ZZ/2)= 0$;
\end{subthm}

\begin{subthm}{item-local-homology} $H_m(\spc{A},\spc{A}\setminus \{p\},\ZZ/2)= 0$;
\end{subthm}

\end{thm}

Let $f$ be a semiconcave function.
A point $p\in \Dom f$ is called \index{critical point}\emph{critical} point of $f$ if $\dd_pf\le 0$; 
otherwise it is called \index{regular point}\emph{regular}.

The following statement \footnote{\red add reference? А: Добрый Витя обещал найти :)} plays a technical role in the proof of stability theorem,
but it is also a useful technical tool on its own.

\begin{thm}{Morse lemma}
Let $f$ be a semiconcave function on a finite-dimensional Alexandrov space without boundary.
Suppose $K$ is a compact set of regular points of $f$ in its level set $f=a$.
Then an open neighborhood $\Omega$ of $K$ admits a homeomorphism $x\mapsto (h(x),f(x))$ to a product space $\Lambda\times (a-\eps,a+\eps)$.
\end{thm}

Subsets in Alexandrov spaces that satisfy the condition in \ref{thm:partial-grad:flow} are called extremal.
More precisely, a subset $E$ is \index{extremal set}\emph{extremal} if for any $x\in E$
and $f$-gradient curve that starts in $E$ remains in $E$;
here $f$ is arbitrary function of the form $\tfrac12\cdot \distfun_p^2$. %{\red V: should we add the condition that $E$ is closed?} A: No, it follows.}

Extremal subsets were introduced by Grigory Perelman and the second author \cite{perelman-petrunin}.
They will pop up in the next lecture.

The following conjecture is one of the oldest questions in Alexandrov geometry that remains open.

\begin{thm}{Conjecture}
Let $S$ be a component of the boundary of a finite-dimensional Alexandrov space.
Then $S$ equipped with the induced length metric is an Alexandrov space with the same curvature bound.
\end{thm}

The doubling theorem has several generalizations \cite{petrunin1997,ge-li} that allow to glue nonidentical spaces.


%%!TEX root = the-quotients.tex
\chapter{Quotients}\label{chap:L/G}

This lecture gives several applications of Alexandrov geometry to isometric group actions.

\section{Quotient space}

Suppose that a group $G$ acts isometrically on a metric space $\spc{X}$.
Note that
\[\dist{G\cdot x}{G\cdot y}{\spc{X}/G}
\df
\inf
\set{\dist{x}{g\cdot y}{\spc{X}}}{g\in G}\]
defines a semimetric on the orbit space $\spc{X}/G$.
Moreover, if the orbits of the action are closed,
then it is a genuine metric.

\begin{thm}{Theorem}\label{thm:CBB/G}
Suppose that a group $G$ acts isometrically on a proper $\Alex0$ space $\spc{A}$, and $G$ has closed orbits.
Then the quotient space $\spc{A}/G$ is $\Alex0$.

\end{thm}

A more general formulation will be given in \ref{thm:submetry-CBB-1}.

\parit{Proof.}
Denote by $\sigma\:\spc{A}\to \spc{A}/G$ the quotient map.

Fix a quadruple of points $p,x_1,x_2,x_3\in \spc{A}/G$.
Choose $\hat p\in \spc{A}$ such that $\sigma(\hat{p})=p$.
Since $\spc{A}$ is proper, we can choose  points $\hat{x}_i\in \spc{A}$ such that $\sigma(\hat x_i)=x_i$ and
\[\dist{p}{x_i}{\spc{A}/G}
=
\dist{\hat{p}}{\hat{x}_i}{\spc{A}}\]
for all $i$.

Note that 
\[\dist{x_i}{x_j}{\spc{A}/G}
\le 
\dist{\hat{x}_i}{\hat{x}_j}{\spc{A}}
\]
for all $i$ and $j$.
Therefore 
\[\angk p{x_i}{x_j}
\le
\angk {\hat{p}}{\hat{x}_i}{\hat{x}_j}
\eqlbl{eq:angles-M-L}\]
for all $i$ and $j$.

By $\EE^2$-comparison in $\spc{A}$,
we have
\[\angk {\hat{p}}{\hat{x}_1}{\hat{x}_2}
+\angk {\hat{p}}{\hat{x}_2}{\hat{x}_3}
+\angk {\hat{p}}{\hat{x}_3}{\hat{x}_1}
\le 
2\cdot\pi.\]
Applying  \ref{eq:angles-M-L}, 
we get 
\[\angk p{x_1}{x_2}
+\angk p{x_2}{x_3}
+\angk p{x_3}{x_1}\le 2\cdot\pi;\]
that is,
the $\EE^2$-comparison holds for any quadruple in $\spc{A}/G$.
\qeds

\begin{thm}{Very advanced exercise}\label{ex:Hilbert/G}
Let $G$ be a compact Lie group with a bi-invariant Riemannian metric.
Show that $G$ is isometric to a quotient of a Hilbert space by an isometric group action.

Conclude that $G$ is $\Alex0$.
\end{thm}

\section{Submetries}

A map $\sigma\:\spc{X}\to\spc{Y}$ between metric spaces $\spc{X}$ and $\spc{Y}$
is called a \index{submetry}\emph{submetry} if 
\[\sigma(\oBall(p,r)_\spc{X})=\oBall(\sigma(p),r)_{\spc{Y}}\]
for any $p\in \spc{X}$ and $r\ge 0$.

Suppose $G$ and $\spc{A}$ are as in \ref{thm:CBB/G}.
Observe that the quotient map $\sigma\:\spc{A}\to \spc{A}/G$ is a submetry.
The following two exercises show that this is not the only source of submetries. 

\begin{thm}{Exercise}\label{ex:sumbetries(S^2)}
Construct submetries
\begin{subthm}{ex:sumbetries(S^2):1}
$\sigma_1\:\mathbb{S}^2\to[0,\pi]$,
\end{subthm}
\begin{subthm}{ex:sumbetries(S^2):2}
$\sigma_2\:\mathbb{S}^2\to[0,\tfrac\pi2]$,
\end{subthm}
\begin{subthm}{ex:sumbetries(S^2):n}
$\sigma_n\:\mathbb{S}^2\to[0,\tfrac\pi n]$ (for integer $n\ge 1$)
\end{subthm}
such that the fibers $\sigma_n^{-1}\{x\}$ are connected for any $x$.
\end{thm}

\begin{thm}{Exercise}\label{ex:sumbetries(E^2)}
Let $\sigma\:\EE^2\to [0,\infty)$ be a submetry.
Show that $K\z=\sigma^{-1}\{0\}$ is a closed convex set without interior points and $\sigma(x)\z=\distfun_Kx$.
\end{thm}

The proof of \ref{thm:CBB/G} works for submetries;
that is, \textit{if $\sigma\:\spc{A}\to\spc{B}$ is a submetry and $\spc{A}$ is a proper $\Alex0$ space, then so is $\spc{B}$}.
Theorem \ref{thm:CBB/G} admits a straightforward generalization to $\Alex{-1}$ case.

In the $\Alex1$ case, the proof produces a slightly weaker statement ---  \textit{$\SSS^2$-comparison holds for a quartuple $p,x_1,x_2,x_3$ in the quotient of $\Alex1$ if $\dist{p}{x_i}{}<\tfrac\pi 2$ for each $i$}.
In particular, the quotient space is \textit{locally} $\Alex1$.
But since $\Alex1$ space is geodesic, then so is its quotient.
Therefore, the globalization theorem implies that it is globally $\Alex1$.
The same holds for the targets of submetries from an  $\Alex1$ space.
With a bit of extra work, one can extend the statement to nonproper spaces \cite[8.34]{alexander-kapovitch-petrunin2024}.
Thus, we have the following.

\begin{thm}{Theorem}\label{thm:submetry-CBB-1}
Let $\sigma\:\spc{A}\to\spc{B}$ be a submetry.
If $\spc{A}$ is $\Alex\kappa$ space, then so is $\spc{B}$.

In particular, if $G$ acts isometrically on an $\Alex\kappa$ space $\spc{A}$, and $G$ has closed orbits.
Then the quotient space $\spc{A}/G$ is $\Alex\kappa$.
\end{thm}

\section{Hopf's conjecture}

\textit{Does $\mathbb{S}^2\times\mathbb{S}^2$ admit a Riemannian metric with positive sectional curvature?} \index{Hopf's conjecture}\emph{Hopf's conjecture} says that the answer should be negative.
Let us take a close look at the following partial result obtained by Wu-Yi Hsiang and Bruce Kleiner \cite{hsiang-kleiner}.

\begin{thm}{Theorem}\label{thm:hsiang-kleiner}
There is no Riemannian metric on $\SSS^2\times\SSS^2$ with sectional curvature $\ge 1$ and a nontrivial isometric $\SSS^1$-action.
\end{thm}

Reacall that a group action $G\acts\spc{X}$ is called \index{effective action}\emph{effective} if for any $g\in G$ there is $x\in\spc{X}$ such that $g\cdot x\ne x$.

\begin{thm}{Key lemma}\label{lem:S^3/S^1}
Suppose $\SSS^1\acts\SSS^3$ is an effective isometric action without fixed points
and $\Sigma=\SSS^3/\SSS^1$ is its quotient space.
Then there is a distance noncontracting map $\Sigma\to \tfrac12\cdot \SSS^2$, where $\tfrac12\cdot \SSS^2$ is the standard 2-sphere rescaled with a factor $\tfrac12$.
\end{thm}

The proof of the lemma is guided by the following exercise.

\begin{thm}{Exercise}\label{ex:S^3/S^1}
Suppose $\SSS^1\acts\SSS^3$ is an effective isometric action without fixed points.
Let us think   of $\SSS^3$ as the unit sphere in $\RR^4$.

\begin{subthm}{ex:S^3/S^1:pq}
Show that one can identify $\RR^4$ with $\CC^2$ so that the action
is given by matrix multiplication
\[\left(\begin{matrix}
u^p&0\\
0& u^q
\end{matrix}
\right),\]
where $(p,q)$ is a pair of relatively prime positive integers and $u\in \SSS^1=\set{z\in\CC}{|z|=1}$.
In particular, our $\SSS^1$ is a subgroup of the torus that acts by
matrix multiplication
\[\left(\begin{matrix}
v&0\\
0& w
\end{matrix}
\right),\]
where  $v,w\in \SSS^1$.
\end{subthm}

\smallskip

\noindent Fix $p$ and $q$ as above.
Let $\Sigma_{p,q}=\SSS^3/\SSS^1$ be the quotient space.

\smallskip

\begin{subthm}{ex:S^3/S^1:sphere}
Show that the $\Sigma_{p,q}=\SSS^3/\SSS^1$ is a topological sphere with $\SSS^1$-symmetry.
This symmetry has two fixed points, north pole and south pole, that correspond to the orbits of $(1,0)$ and $(0,1)$ in $\SSS^3$.
\end{subthm}

\smallskip

\noindent Denote by $S(r)$ the circle of radius $r$ with the center at the north pole of $\Sigma_{p,q}$.

\begin{subthm}{ex:S^3/S^1:a}
Denote by $T(r)$ the inverse image $T(r)$ in $\SSS^3$, and let $a(r)$ be its area.
Show that $T(r)$ is an orbit of the torus action and
\[a(r)=\pi^2\cdot\sin r\cdot \cos r.\]

\end{subthm}

\smallskip

\begin{subthm}{ex:S^3/S^1:b}
Let $b_{p,q}(r)$ be the length of the $\SSS^1$-orbit in $\SSS^3$ that corresponds to a point on $S(r)$. 
Show that
\[b_{p,q}=\pi\cdot\sqrt{(p\cdot \sin r)^2+(q\cdot \cos r)^2}.\]
\end{subthm}

\smallskip

\begin{subthm}{ex:S^3/S^1:c}
Let $c_{p,q}(r)$ be the length of $S(r)$.
Show that $a(r)=c_{p,q}(r)\cdot b_{p,q}(r)$.
\end{subthm}

\smallskip

\begin{subthm}{ex:S^3/S^1:cc}
Show that $c_{p,q}(r)\le c_{1,1}(r)$ for any pair $(p,q)$ of relatively prime positive integers.
Use it to construct a distance noncontracting map $\Sigma_{p,q}\to \tfrac12\cdot \SSS^2\iso\Sigma_{1,1}$.
\end{subthm}

\end{thm}

\parit{Proof of \ref{thm:hsiang-kleiner}.}
Assume $\spc{B}=(\SSS^2\times\SSS^2,g)$ is a counterexample.
By the Toponogov theorem, $\spc{B}$ is $\Alex1$.
By \ref{thm:CBB/G}, the quotient space $\spc{A}\z=\spc{B}/\SSS^1$ is $\Alex1$;
evidently, $\spc{A}$ is 3-dimensional.

Denote by $F\subset \spc{B}$ the fixed point set of the $\SSS^1$-action.
Then $\chi(\spc{B})\z=\chi(F)$.
Each connected component of $F$ is either an isolated point or a 2-dimensional geodesic submanifold in $\spc{B}$;
the latter has to have positive curvature, and therefore it is homeomorphic to $\SSS^2$ or $\RP^2$.
Notice that 
\begin{itemize}
 \item each isolated point contributes 1 to the Euler characteristic of~$\spc{B}$,
 \item each sphere contributes 2 to the Euler characteristic of $\spc{B}$, and
 \item each projective plane contributes 1 to the Euler characteristic of~$\spc{B}$.
\end{itemize}
Since $\chi(\spc{B})=4$, we are in one of the following three cases:
\begin{enumerate}
 \item\label{case1} $F$ has exactly 4 isolated points,
 \item\label{case2} $F$ has one 2-dimensional submanifold and at least 2 isolated points,
 \item\label{case3} $F$ has at least two 2-dimensional submanifolds.
\end{enumerate}
In each case we will arrive at a contradiction.

\parit{Case \ref{case1}.}
Suppose $F$ has exactly 4 isolated points $x_1$, $x_2$, $x_3$, and $x_4$.
Denote by $y_1$, $y_2$, $y_3$, and $y_4$ the corresponding points in $\spc{A}$.
Note that $\Sigma_{y_i}\spc{A}$ is isometric to a quotient of $\SSS^3$ by an isometric $\SSS^1$-action without fixed points.

By \ref{ex:S^3/S^1}, each angle $\mangle\hinge{y_i}{y_j}{y_k}\le \tfrac\pi2$ for any three distinct points 
$y_i$, $y_j$, $y_k$.
In particular, all four triangles $[y_1y_2y_3]$, $[y_1y_2y_4]$, $[y_1y_3y_4]$, and $[y_2y_3y_4]$ are nondegenerate.
By the comparison, the sum of angles in each triangle is strictly greater than $\pi$.

Denote by $\omega$ the sum of all 12 angles in the 4 triangles $[y_1y_2y_3]$, $[y_1y_2y_4]$, $[y_1y_3y_4]$, and $[y_2y_3y_4]$.
From above,
\[\omega>4\cdot\pi.\]

On the other hand, by \ref{ex:S^3/S^1} any triangle in $\Sigma_{y_1}\spc{A}$ has perimeter at most $\pi$.
In particular, 
\[\mangle\hinge{y_1}{y_2}{y_3}+\mangle\hinge{y_1}{y_3}{y_4}+\mangle\hinge{y_1}{y_4}{y_2}\le \pi.\]
Apply the same argument in $\Sigma_{y_2}\spc{A}$, $\Sigma_{y_3}\spc{A}$, and $\Sigma_{y_4}\spc{A}$;
adding the results, we get 
\[\omega\le 4\cdot\pi\]
--- a contradiction.

\parit{Case \ref{case2}.}
Suppose $F$ contains one surface $S$.
Then the projection of $S$ to $\spc{A}$ forms its boundary $\partial \spc{A}$.
The doubling $\spc{W}$ of $\spc{A}$ across its boundary has at least 4 singular points --- each singular point of $\spc{A}$ corresponds to two singular points of $\spc{W}$.

By the doubling theorem, $\spc{W}$ is a $\Alex1$ space.
Therefore we arrive at a contradiction in the same way as in the first case.

\parit{Case \ref{case3}.} Impossible by \ref{ex:bry-connected}.
\qeds

\section{Erdős' problem rediscovered}

A point $p$ in an Alexandrov space is called \index{extremal point}\emph{extremal} if $\mangle\hinge pxy\le \tfrac\pi2$ for any hinge $\hinge pxy$ with the vertex at $p$; equivalently, $\diam \Sigma_p\le \pi/2$.

\begin{thm}{Theorem}\label{thm:extr-point}
Let $\spc{A}$ be a compact $m$-dimensional $\Alex0$ space.
Then it has at most $2^m$ extremal points.
\end{thm}

\parit{Proof of \ref{thm:extr-point}.}
Let $\{p_1,\dots,p_N\}$ be extremal points in $\spc{A}$.
For each $p_i$ consider its open \index{Voronoi domain}\emph{Voronoi domain} $V_i$; that is, 
\[V_i=\set{x\in \spc{A}}{\dist{p_i}{x}{}<\dist{p_j}{x}{}\ \text{for any}\ j\not=i}.\]
Clearly $V_i\cap V_j=\emptyset$ if $i\not=j$.

Suppose  $0<\alpha\le 1$.
Given a point $x\in\spc{A}$, choose a geodesic $[p_ix]$ and denote by $x_i$ the point on $[p_ix]$ such that $\dist{p_i}{x_i}{}=\alpha\cdot\dist{p_i}{x}{}$;
let $\map_i\:x\to x_i$ be the corresponding map.
By the comparison, 
\[\dist{x_i}{y_i}{}\ge\alpha\cdot \dist{x}{y}{}\]
for any $x$, $y$, and $i$.
Therefore 
\[\vol(\map_i \spc{A})\ge\alpha^m\cdot\vol \spc{A}.\]

Suppose $\alpha<\tfrac12$.
Then $x_i\in V_i$ for any $x\in \spc{A}$.
Indeed, assume $x_i\notin V_i$,
then there is $p_j$ such that $\dist{p_i}{x_i}{}\ge\dist{p_j}{x_i}{}$.
Then by comparison, we have $\angk{p_j}{p_i}{x}_{\EE^2}>\tfrac\pi2$;
that is, $p_j$ is not an extremal point.

It follows that $\vol V_i\ge\alpha^m\cdot\vol \spc{A}$
for any $0<\alpha<\tfrac12$; hence 
\[\vol V_i\ge\tfrac1{2^m}\cdot\vol \spc{A}.\]
Since $V_1,\dots,V_N$ are disjoint subsets of $\spc{A}$, we have $N\le 2^m$.
\qeds


\section{Crystallographic actions}

An isometric action $\Gamma\acts \EE^m$ is called \index{crystallographic action}\emph{crystallographic} if it is 
\index{properly discontinuous}\emph{properly discontinuous} (that is, for any compact set $K\subset \EE^m$ and $x\z\in \EE^m$ there are only finitely many elements $g\in \Gamma$ such that $g\cdot x\in K$) and \emph{cocompact} (that is, the quotient space $\spc{A}=\EE^m/\Gamma$ is compact).

Let $F$ be a maximal finite subgroup of $\Gamma$;
that is, if $F<H<\Gamma$ for a finite group $H$, then $F=H$.
Denote by $\mathfrak{M}(\Gamma)$ the number of maximal finite subgroups of $\Gamma$ up to conjugation.

\begin{thm}{Open question}
Let $\Gamma\acts \EE^m$ be a crystallographic action.
Is it true that $\mathfrak{M}(\Gamma)\le 2^m$?
\end{thm}

Note that any finite subgroup $F$ of $\Gamma$ fixes an affine subspace $A_F$ in $\EE^m$.
If $F$ is maximal, then $A_F$ completely describes $F$.
Indeed, since the action is properly discontinuous, the subgroup of $\Gamma$ that fix $A_F$ has to be finite.
This subgroup must contain $F$, but since $F$ is maximal, it must coinside with $F$. 

Denote by $\mathfrak{M}_k(\Gamma)$ the number of maximal finite subgroups $F<\Gamma$ (up to conjugation) such that $\dim A_F=k$.

Choose a finite subgroup $F<\Gamma$; consider a conjugate subgroup $F'=g \cdot F \cdot g^{-1}$.
Note that $A_{F'}=g\cdot A_F$.
In particular, the subspaces $A_F$ and $A_{F'}$ have the same image in the quotient space $\spc{A}=\EE^m/\Gamma$.
Therefore, to count subgroups up to conjugation, we need to count the images of their fixed sets.
By the lemma below (\ref{lem:extr/G}), $\mathfrak{M}_0(\Gamma)$ cannot exceed the number of extremal points in $\spc{A}=\EE^m/\Gamma$.
Combining this observation with \ref{thm:extr-point}, we get the following.

\begin{thm}{Proposition}\label{prop:2m}
Let $\Gamma\acts \EE^m$ be a crystallographic action.
Then $\mathfrak{M}_0(\Gamma)\le 2^m$.
\end{thm}

\begin{thm}{Lemma}\label{lem:extr/G}
Let $\Gamma\acts \EE^m$ be a crystallographic action and $F$ be a maximal finite subgroup of $\Gamma$ that fixes an isolated point $p$.
Then the image of $p$ in the quotient space $\spc{A}=\EE^m/\Gamma$ is an extremal point.
\end{thm}

\parit{Proof.}
Let $q$ be the image of $p$.
Suppose $q$ is not extremal;
that is, $\mangle \hinge q{y_1}{y_2}>\tfrac\pi2$ for some hinge $\hinge q{y_1}{y_2}$ in $\spc{A}$.

Choose the inverse images $x_1,x_2\in \EE^m$ of $y_1,y_2\in \spc{A}$ such that $\dist{p}{x_i}{\EE^m}=\dist{q}{y_i}{\spc{A}}$.
Note that $\mangle \hinge p{x_1}{x_2}\ge \mangle \hinge q{y_1}{y_2}>\tfrac\pi2$.
Moreover, since $p$ is fixed by $F$, we have
\[\mangle \hinge p{x_1}{g\cdot x_2}>\tfrac\pi2
\eqlbl{eq:>pi/2}\]
for any $g\in F$.

Denote by $z$ the barycenter of the orbit $F\cdot x_2$.
Note that $z$ is a fixed point of $F$.
By \ref{eq:>pi/2}, $z\ne p$;
so $F$ must fix the line $pz$.
But $p$ is an isolated fixed point of $F$ --- a contradiction.
\qeds

\begin{thm}{Exercise}\label{ex:number(m-1)}
Let $\Gamma\acts \EE^m$ be a crystallographic action.
Show that
\begin{subthm}{ex:number(m-1):2}
$\mathfrak{M}_{m-1}(\Gamma)\le 2$, and
\end{subthm}

\begin{subthm}{ex:number(m-1):1}
if $\mathfrak{M}_{m-1}(\Gamma)=1$, then $\mathfrak{M}_0(\Gamma)\le 2^{m-1}$.
\end{subthm}

Construct  crystallographic actions with equalities in \ref{SHORT.ex:number(m-1):2} and \ref{SHORT.ex:number(m-1):1}.
\end{thm}

\section{Remarks}

Submetries were introduced by Valerii Berestovskii \cite{berestovskii1987} and have attracted attention in various contexts of differential and metric geometry.



A more general form of Theorem \ref{thm:hsiang-kleiner} was found by Karsten Grove and Burkhard Wilking \cite{grove-wilking};
it classifies isometric $\SSS^1$ actions on  4-dimensional manifolds with nonnegative sectional curvature.
This proof is as beautiful as the original work of Wu-Yi Hsiang and Bruce Kleiner.

It is expected that \textit{no $\Alex1$ space with a nontrivial isometric $\SSS^1$-action can be homeomorphic to $\SSS^2\times\SSS^2$};
so \ref{thm:hsiang-kleiner} holds for general $\Alex1$ space.
The proof of \ref{thm:hsiang-kleiner} would work if we had the following generalization of \ref{lem:S^3/S^1};
see \cite{harvey-searle}.

\begin{thm}{Open question}
Let $\Sigma$ be an $\Alex1$ space homeomorphic to $\SSS^3$.
Suppose $\SSS^1$ acts on $\Sigma$ isometrically and without fixed points.
Is it true that any triangle in $\Sigma/\SSS^1$ has perimeter at most $\pi$?

And if the answer is, is there a distance-noncontracting map
\[\Sigma/\SSS^1\z\to \tfrac12\cdot\SSS^2?\]
\end{thm}


\begin{thm}{Advanced exercise}\label{ex:S1actsS3}
Suppose $\SSS^1$ acts isometrically on an $\Alex1$ space $\spc{A}$ that is homeomorphic to $\SSS^3$.
Assume its fixed-point set is a closed local geodesic $\gamma$.
Show that
\[\length\gamma\le2\cdot\pi.\]
\end{thm}

An analogous question for a $\ZZ_2$-action is open \cite{petrunin-involution}.

Theorem \ref{thm:extr-point} is a translation of the following classical problem in discrete geometry to Alexandrov's language.

\begin{thm}{Problem}\label{erdos-problem}
Let $F$ be a set of points in $\EE^m$ such that any triangle formed by three distinct points in $F$ has no obtuse angles.
Then  $|F|\le2^m$.
Moreover, if $|F|=2^m$, then $F$ consists of the vertices of an $m$-dimensional rectangle.
\end{thm}

This problem was posed by Paul Erdős \cite{erdos} and solved by Ludwig Danzer and Branko Gr\"unbaum \cite{danzer-gruenbaum}.
Grigory Perelman noticed that, after proper definitions, the same proof works in Alexandrov spaces \cite{perelman-Erdos}; thus, it proves \ref{thm:extr-point}.
Applying the our argument to the convex hull of $F$ in \ref{erdos-problem} proves that $|F|\le 2^m$;
the case of equality requires more work.

Compact $m$-dimensional $\Alex0$ spaces with the maximal number of extremal points include $m$-dimensional rectangles and the quotients of flat tori by reflections across a point.
(This action has $2^m$ isolated fixed points; each corresponds to an extremal point in the quotient space $\spc{A}=\TT^m/\ZZ_2$.)
Nina Lebedeva has proved \cite{lebedeva2015} that \textit{every $m$-dimensional $\Alex0$ space with $2^m$ extremal points is a quotient of Euclidean space by a crystallographic action}.

The extremal subsets of Alexandrov space were brifly discussed in \ref{sec:bry-remarks}.
The following definition is more relevant to isometric group actions.

A closed subset $E$ in a finite-dimensional Alexandrov space is called
\index{extremal set}\emph{extremal} if $\mangle\hinge pxy\z\le \tfrac\pi2$ for any $x\notin E$ and $p\in E$ such that $\dist{x}{p}{}$ takes a minimal value.
An extremal set is called \index{minimal extremal set}\emph{minimal} if it contains no proper extremal subsets.

For example, the whole space and the empty set are extremal.
Also, every vertex, edge, or face (as well as their unions) of the cube is an extremal subset of the cube.
Vertices of the cube are its only minimal extremal subsets.

Counting maximal finite subgroups in a crystallographic group $\Gamma$ (up to conjugation) is equivalent to counting the minimal extremal subsets in the quotient space $\spc{A}=\EE^m/\Gamma$.
So, \ref{prop:2m} would follow from the next conjecture.

\begin{thm}{Conjecture}
Any $m$-dimensional compact $\Alex0$ space has at most $2^m$ minimal extremal subset.
\end{thm}

Let us mention another related conjecture.
An extremal set is called \index{primitive extremal set}\emph{primitive} if it contains no proper extremal subsets with nonempty relative interior.
For example, each face of $m$-dimensional cube is its primitive extremal subset;
therefore the cube has exactly $3^m$ primitive extremal subset, including the empty set and the whole cube.

\begin{thm}{Conjecture}
Any $m$-dimensional compact $\Alex0$ space has at most $3^m$ minimal extremal subset.
\end{thm}

Some crude estimates on number of extremal subsets follow from the idea in Gromov's Betti number theorem \ref{thm:betti}.


%\chapter{CBB: definition}

\section{Distances and geodesics}

\parbf{Distances.}
The distance between two points $x$ and $y$ in a metric space $\spc{X}$ will be denoted by $\dist{x}{y}{}$ or $\dist{x}{y}{\spc{X}}$.
The latter notation is used if we need to emphasize 
that the distance is taken in the space~${\spc{X}}$.
The function $(x,y)\mapsto \dist{x}{y}{\spc{X}}$ is called \index{metric}\emph{metric};
it has to meet the following conditions for any three points $x,y,z\in \spc{X}$:

\begin{subthm}{metric>=0}
$\dist{x}{y}{\spc{X}}\ge 0$,
\end{subthm}

\begin{subthm}{metric=0} $\dist{x}{y}{\spc{X}}= 0$ $\iff$ $x=y$,
\end{subthm}

\begin{subthm}{metric:sym} $\dist{x}{y}{\spc{X}}=\dist{y}{x}{\spc{X}}$,
\end{subthm}

\begin{subthm}{metric:triangle} $\dist{x}{y}{\spc{X}}+\dist{y}{z}{\spc{X}}\ge\dist{x}{z}{\spc{X}}$.
\end{subthm}

\parbf{Geodesics.}
Let $\II$\index{$\II$} be a real interval. 
A distance-preserving map $\gamma$ from $\II$ to a metric space $\spc{X}$ is called a \index{geodesic}\emph{geodesic}%
\footnote{Others call it differently: \textit{shortest path}, \textit{minimizing geodesic}.
Also, note that the meaning of the term \textit{geodesic} is different from what is used in Riemannian geometry, altho they are closely related.}; 
in other words, $\gamma\:\II\to \spc{X}$ is a geodesic if 
\[\dist{\gamma(s)}{\gamma(t)}{\spc{X}}=|s-t|\]
for any pair $s,t\in \II$.

If $\gamma\:[a,b]\to \spc{X}$ is a geodesic such that $p=\gamma(a)$, $q=\gamma(b)$, then we say that $\gamma$ is a geodesic from $p$ to $q$.
In this case, the image of $\gamma$ is denoted by $[p q]$\index{$[{*}{*}]$}, and, with abuse of notations, we also call it a \index{geodesic}\emph{geodesic}.
We may write $[p q]_{\spc{X}}$ 
to emphasize that the geodesic $[p q]$ is in the space  ${\spc{X}}$.

In general, a geodesic from $p$ to $q$ need not exist and if it exists, it need not  be unique.  
However, once we write $[p q]$ we assume that we have chosen such geodesic.

\parbf{Geodesic path.}
A \index{geodesic path}\emph{geodesic path} is a geodesic with constant-speed parameterization by the unit interval $[0,1]$.

\parbf{Geodesic space.}
A metric space is called \index{geodesic space}\emph{geodesic} if any pair of its points can be joined by a geodesic.

\section{Baby Toponogov}

Recall that \index{polyhedral space}\emph{polyhedral space} is a geodesic space that admits a finite triangulation such that each simplex is isometric to a simplex in a Euclidean space.
If, in addition, it is homeomorphic to a surface (without boundary), then it is called a \index{polyhedral surface}\emph{polyhedral surface}.
A point on a polyhedral surface with nonzero curvature is called an \index{essential vertex}\emph{essential vertex}.
Any other point on the surface will be called \index{regular point}\emph{regular}.
Note that \textit{any regular point has a neighborhood that is isometric to an open set in the Euclidean plane}.

\begin{thm}{Exercise}\label{ex:poly+geod}
Let $P$ be a non-negatively curved polyhedral surface.

\begin{subthm}{}
Show that a geodesic in $P$ cannot pass thru an essential vertex.
\end{subthm}

\begin{subthm}{}
Show that if two geodesics in $P$ intersect at two points, 
then these are the endpoints for both geodesics.
\end{subthm}

\end{thm}

The next theorem gives a global geometric property of non-negatively curved polyhedral surfaces.

Given a hinge $\hinge pxy$ in a non-negatively curved polyhedral surface $P$, denote by $\mangle\hinge pxy$ the minimal angle that the hinge cuts from $P$ at~$p$.
(Soon we will give a more general definition of $\mangle\hinge pxy$; see \ref{sec:angles}.)

\begin{thm}{Theorem}\label{thm:poly-cbb}
Let $P$ be a polyhedral surface.
Assume $P$ has non-negative curvature at each point (see \ref{sec:Alexandrov-existence}).
Then 
\[\mangle\hinge pxy\ge\angk pxy\]
for any hinge $\hinge pxy$ in $P$.
\end{thm}

The following exercise will be used in the proof.

\begin{thm}{Exercise}\label{ex:concave-loc}
Let $f\:[0,\ell]\to\RR$ be a continuous function such that for any $t\in \left]0,\ell\right[$ there is a linear function $h$ that locally supports $f$ from above;
that is, $h(t_0)=f(t_0)$, and there is $\eps>0$ such that $h(t)\ge f(t)$ if $|t-t_0|<\eps$.
Show that $f$ is concave.
\end{thm}


\parit{Proof.}
Let $[pxy]$ be a triangle in $P$ and let $[\tilde p\tilde x\tilde y]$ be the model triangle of $[pxy]$.
Set $\ell=|x-y|_P=|\tilde x-\tilde y|_{\EE^2}$.

Denote by $\gamma(t)$ and $\tilde \gamma(t)$ the geodesics $[xy]$ and $[\tilde x\tilde y]$ parametrized by length starting from $x$ and $\tilde x$, respectively.
Observe that it is sufficient to show that 
$$| p- \gamma(t)|\le|\tilde p-\tilde \gamma(t)| 
\eqlbl{eq:comp-gamma}$$
for any $t$ in $[0,\ell]$.

We may assume that $p$ is a regular point;
otherwise, move it slightly and apply approximation.


From the cosine law, we get that the function 
$$\tilde f(t)=|\tilde p-\tilde \gamma(t)|^2-t^2$$
is linear.
Consider the function
$$f(t)=|p- \gamma(t)|^2-t^2.$$
Note that $f(0)=\tilde f(0)$, $f(\ell)=\tilde f(\ell)$, and the inequality~\ref{eq:comp-gamma} is equivalent to
$$f(t)\ge \tilde f(t).
\eqlbl{eq:comp-f}$$
By Jensen's inequality, \ref{eq:comp-f} holds if $f$ is concave.

By \ref{ex:poly+geod}, 
$\gamma(t_0)$ is regular.
Since $p$ is regular,
a geodesic $[p\gamma(t)]$ contains only regular points.
Therefore for small $\eps>0$,
 the $\eps$-neighborhood of $[p\gamma(t)]$, say $\Omega$, contains only regular points. 
We may assume that $\Omega$ is homeomorphic to a disc;
in this case, there is a locally distance-preserving embedding $\iota\:\Omega\to\EE^2$.
Note the image $\iota[p\gamma(t)]$ is a line segment that 
and $\iota(\Omega)$ is the $\eps$-neighborhood of $\iota[p\gamma(t)]$ in $\EE^2$;
in particular, $\iota(\Omega)$ is convex.
Thus $\iota(\Omega)$ contains a triangle with  base $\iota[\gamma(t_0-\eps)\ \gamma(t_0+\eps)]$  and vertex $\iota(p)$.

Clearly, for any $t\in[t_0-\eps,t_0+\eps]$ 
we have 
$$|\iota(p)-\iota(\gamma(t))|\ge|p-\gamma(t)|.$$
Note that
the function
$$h(t)= |\iota(p)-\iota(\gamma(t))|^2-t^2$$
is linear.
From above, $h$ supports $f$ locally  at $t_0$.
It remains to apply~\ref{ex:concave-loc}.
\qeds

\section{Definition}

\begin{thm}{Definition}\label{def:CBB}
A metric space $\spc{X}$ has {}\emph{nonnegative curvature} in the sense of Alexandrov if the inequality 
\[\angk  pxy_{\EE^2}+\angk pyz_{\EE^2}+\angk pzx_{\EE^2}
\le 
2\cdot\pi
\eqlbl{eq:CBB-comparison}\]
holds for any quadruple $p,x,y,z\in\spc{X}$ such that each model angle in \ref{eq:CBB-comparison} is defined. 

The inequality \ref{eq:CBB-comparison} is called \index{4-point comparison}\emph{4-point comparison} for the quadruple $p,x,y,z$.
If instead of $\EE^2$, we use $\SSS^2$ or $\HH^2$, then we get the definition of
$\CBB(1)$ and $\CBB(-1)$ comparisons.
(Note that $\angk  pxy_{\EE^2}$ and $\angk  pxy_{\HH^2}$ are defined if $p\ne x$, $p\ne y$,
but for $\angk  pxy_{\SSS^2}$ we need in addition, $\dist{p}{x}{}+\dist{p}{y}{}+\dist{x}{y}{}<2\cdot\pi$.)

More generally, one may apply this definition to $\MM^2(\kappa)$ --- the model plane of curvature $\kappa$, defined as follows:
$\MM^2(0)=\EE^2$,
if $\kappa>0$, then $\MM^2(\kappa)$ is the sphere of radius $\tfrac{1}{\sqrt{\kappa}}$ and if $\kappa<0$, then it is Lobachevsky plane rescaled by factor $\tfrac{1}{\sqrt{-\kappa}}$.
This way we define $\CBB(\kappa)$ comparison for any real $\kappa$.
\end{thm}

While this definition can be applied to any metric space,
it is usually applied to geodesic spaces (or, at least, length spaces that will be defined later).

\begin{thm}{Exercise}
Show that Euclidean space $\EE^n$ is $\CBB(0)$.
\end{thm}


\begin{thm}{Exercise}\label{ex:polyCBB}
Show that a polyhedral surface is $\CBB(0)$ if and only if it has nonnegative curvature in the sense of \ref{sec:Alexandrov-existence}. 
\end{thm}





\section{Comments}

The first synthetic description of curvature is due to Abraham Wald \cite{wald}; 
it was given in a lone publication on a ``coordinateless description of Gauss surfaces'' published in 1936.
In 1941, similar definitions were rediscovered by Alexandr Alexandrov \cite{alexandrov:def}.

In Alexandrov's work, the first applications of this approach were given.
Mainly: the main part of \ref{thm:alexandrov+pogorelov} \cite{alexandrov-1941,alexandrov-1941convex}
and the {}\emph{gluing theorem} \cite{alexandrov-1946}, which gave a flexible tool to modify non-negatively curved metrics on a sphere.
These two results together formed the foundation of the branch of geometry now called {}\emph{Alexandrov geometry};
they gave  a very intuitive geometric tool to study embeddings and bending of surfaces in Euclidean space and changed the subject dramatically.

In particular, the existence of bending of a large spherical dome (sphere with a small disc removed) easily follows from these two theorems; moreover, it provides an intuitive description of such bending that can be extended to a closed convex surface.






%%%!TEX root = invitation-CBB.tex
\chapter{Polyhedral surfaces}\label{chap:alex-embedding}

In this lecture we discuss intrisic geometry of surfaces of convex polyhedra and convex bodies.
Furhter, we prove the Cauchy theorem, and then modify the proof to get the Alexandrov uniqueness theorem.

\section{Surface of convex polyhedron}

Let us define a \index{convex body}\emph{convex body} as a compact convex subset in $\EE^3$ with nonempty interior.
The \index{surface}\emph{surface} of a convex body is defined as its boundary equipped with the induced length metric.

\begin{thm}{Exercise}\label{ex:surf-S2}
Show that the surface of a convex body is homeomorphic to the 2-dimensional sphere.
\end{thm}

A \index{convex polyhedron}\emph{convex polyhedron} is a convex body with a finite number of extremal points, called its \index{vertex}\emph{vertices}.

The surface, say $P$, of a convex polyhedron $K$ admits a finite triangulation such that each triangle is isometric to a plane triangle.
In other words, $P$ is a closed \index{polyhedral surface}\emph{polyhedral surface};
that is, it is a 2-dimensional manifold with a length metric that admits a finite triangulation such that each triangle is isometric to a solid plane triangle.
A \index{triangulation}\emph{triangulation} of a polyhedral surface will always be assumed to satisfy this condition.

The total angle around a vertex $v$ in $P$ is defined as the sum of angles at $v$ of all triangles in the triangulation that contain $v$.

If a point $p\in P$ is not a vertex of $K$,
then
\begin{itemize}
\item $p$ lies in the interior of a face of $K$, and its neighborhood in $P$ is a piece of plane, or
\item $p$ lies on an edge, and its neighborhood is two half-planes glued along the boundary.
\end{itemize}
In both cases, a neighborhood of $p$ in $P$ (with the induced length metric) is isometric to an open domain of the plane.
In this case, the total angle around $p$ will be defined to be $2\cdot\pi$.

\begin{thm}{Claim}\label{clm:total-angle}
Let $P$ be the surface of a convex polyhedron $K$.
Then, the total angle around any point $p\in P$ cannot exceed $2\cdot\pi$.
\end{thm}

The proof relies on the triangle inequality for angles (or the spherical triangle inequality).
It follows from \ref{claim:angle-3angle-inq}, but our proof of this statement is a straightforward generalization of the argument in the classical geometry textbook \cite[§ 47]{kiselev-stereo-en} that proves the following statement.

\begin{thm}{Spherical triangle inequality}\label{ex:angle-triangle}
Let $w_1,w_2,w_3$ be unit vectors in $\EE^3$.
Denote by $\alpha_{i,j}$ the angle between the vectors $v_i$ and $v_j$.
Then
$$\alpha_{1,3}\le \alpha_{1,2}+\alpha_{2,3}.$$
Moreover, in the case of equality, the three solid angles spanned by $w_1$, $w_2$, and $w_3$ form a plane.
\end{thm}

\parit{Proof of \ref{clm:total-angle}.}
Consider the intersection of $K$ with a small sphere centered at~$p$;
it is a convex spherical polygon, say $F$.
Applying rescaling we may assume that the sphere has unit radius.
Then we need to show that the perimeter of $F$ does not exceed $2\cdot\pi$.

\begin{wrapfigure}{o}{22mm}
\vskip-4mm
\centering
\includegraphics{mppics/pic-1103}
\end{wrapfigure}

Note that $F$ lies in a hemisphere, say $H$.
Moreover, there is a decreasing sequence of convex spherical polygons
\[H=H_0\supset\dots\supset H_n=F,\]
such that $H_{i+1}$ is obtained from $H_{i}$ by cutting along a chord.

By the spherical triangle inequality (\ref{ex:angle-triangle}), we have
\[
2\cdot\pi=\perim H=\perim H_0\ge\dots\ge\perim H_n=\perim F
\]
--- hence the result.
\qeds

\section{Curvature}

Let $p$ be a point on a polyhedral surface, and $\theta_p$ is the total angle around $p$.
The value $2\cdot \pi -\theta_p$ is called the \index{curvature}\emph{curvature} of the polyhedral surface at $p$.

Note that if $p$ is not a vertex in a triangulation of $P$, then its curvature is zero.
A vertex of a triangulation of a polyhedral surface is called \index{essential vertex}\emph{essential} if its curvature is not $0$.

\begin{thm}{Exercise}\label{ex:vertex-essential-vertex}
Let $v$ be a point on the surface $P$ of a convex polyhedron $K$.
Show that $v$ is a vertex of $K$ if and only if
$v$ is an essential vertex of $P$.
\end{thm}


\begin{thm}{Exercise}\label{ex:geodesic-vertex}
Show that geodesics on a closed polyhedral surface with nonnegative curvature may have essential vertices only at their ends.
\end{thm}

\begin{thm}{Exercise}\label{pr:tetrahedron}
Assume that the surface of a nondegenerate tetrahedron $T$ has curvature $\pi$ at each of its vertices.
Show that

\begin{subthm}{pr:tetrahedron:=}
all faces of $T$ are congruent;
\end{subthm}

\begin{subthm}{pr:tetrahedron:perp} the line containing midpoints of opposite edges of $T$ intersects these edges at right angles.
\end{subthm}

\end{thm}

\begin{thm}{Exercise}\label{ex:gauss-bonnet}
Show that sum of curvatures of a closed polyhedral surface $P$ equals to $2\cdot\pi\cdot\chi(P)$,
where $\chi(P)$ denotes the Euler characteristic of $P$.
\end{thm}


Claim~\ref{clm:total-angle} says that \textit{surfaces of convex polyhedra have nonnegative curvature} in the sense of the above definition.
Now we show that this definition agrees with the 4-point comparison.

\begin{thm}{Proposition}\label{prop:poly-CBB}
A polyhedral surface with nonnegative curvature at each vertex is $\Alex0$.
\end{thm}

\parit{Proof.}
Denote the surface by $P$.
By \ref{comp-kappa}, it is sufficient to check that
$\distfun_p^2\circ\gamma$ is 1-concave for any geodesic $\gamma$ and any point $p$ in $P$.

We can assume that $p$ is not a vertex;
the vertex case can be done by approximation.
By \ref{ex:geodesic-vertex}, $\gamma$ does not contain vertices.

Given a point $x=\gamma(t_0)$, choose a geodesic $[px]$.
Again, by \ref{ex:geodesic-vertex}, $[px]$ does not contain vertices.
Therefore a small neighborhood of $U\supset [px]$ can be unfolded on a plane;
that is, there is an injective length-preserving map $z\mapsto \tilde z$
of $U$ into the Euclidean plane.
This way we map part of $\gamma$ in $U$ to a line segment $\tilde\gamma$.
Let
\[\tilde f(t)\df\tfrac12\cdot\distfun_{\tilde p}^2\circ\tilde \gamma(t).\]
Since the geodesic $[px]$ maps to a line segment, we have $\tilde f(t_0)= f(t_0)$.
Furthermore, since the unfolding $z\mapsto \tilde z$ preserves lengths of curves, we get
$\tilde f(t)\ge f(t)$ if $t$ if the left-hand side is defined.
That is, $\tilde f$ is a local upper barrier of $f$ at $t_0$.
Evidently, $\tilde f''\equiv 1$; therefore $f''\le 1$.
It remains to apply \ref{comp-kappa}.
\qeds

\begin{thm}{Exercise}\label{ex:poly-CBB}
Prove the converse to the proposition;
that is, show that if a poyhedral surface is $\Alex0$, then it has nonnegative curvature in the sense defined in this section.
\end{thm}

\section{Surface of convex body}

\begin{thm}{Advanced exercise}\label{ex:surface-covergence}
Let $K_1,K_2,\dots,$ and $K_\infty$ be convex bodies in $\EE^m$.
Denote by $P_n$ the surface of $K_n$ with induced length metric.
Suppose $K_n\z\to K_\infty$ in the sense of Hausdorff.
Show that $P_n\to P_\infty$ in the sense of Gromov--Hausdorff.
\end{thm}

Any convex body is a Hausdorff limit of a sequence of convex polyhedra.
Therefore, the next proposition follows from \ref{prop:poly-CBB}, \ref{ex:surface-covergence}, and \ref{thm:CBB-closed}.

\begin{thm}{Proposition}\label{prop:conv-surf-CBB(0)}
The surface of a convex body in $\EE^3$ is $\Alex0$.
\end{thm}

\begin{thm}{Very advanced exercise}\label{ex:liberman+milka}
Let $P$ be the surface of a nondegenerate convex body $K\subset\EE^3$;
we assume that $P$ is equipped with the induced length metric.

\begin{subthm}{ex:liberman+milka:liberman}
Show that any geodesic $\gamma$ in $P$ is one-sided differentiable as a curve in $\EE^3$.
\end{subthm}

\begin{subthm}{ex:liberman+milka:convex}
Suppose a plane $\Pi$ cuts from $P$ a disc $\Delta$, and the reflection of $\Delta$ across $\Pi$ lies in $K$.
Show that $\Delta$ is a convex subset of $P$;
that is, if a geodesic has ends in $\Delta$, then it completely lies in $\Delta$.
\end{subthm}


\begin{subthm}{ex:liberman+milka:milka}
Let $\gamma_1$ and $\gamma_2$ be geodesic paths in $P$ that start at one point $p\z=\gamma_1(0)\z=\gamma_2(0)$.
Suppose $x_i=\gamma_i(1)$, and $y_i\z=p+\gamma_i^+(0)$.
Show that
\[\dist{x_1}{x_2}{P}\le \dist{y_1}{y_2}{W},\]
where $W$ is the complement to the interior of $K$.
\end{subthm}

\end{thm}


\section{Cauchy theorem}

Recall that \textit{surfaces} of convex polyhedrons are considered with the induced length metric.
 
\begin{thm}{Theorem}\label{thm:cauchy}
Let $K$ and $K'$ be convex polyhedrons in $\EE^3$;
denote their surfaces 
by $P$ and $P'$.
Suppose there is an isometry $P\to P'$ that sends each face of $K$ to a face of $K'$.
Then $K$ is congruent to $K'$; moreover the isometry $P\to P'$ can be extended to a motion of $\EE^3$ that maps $K$  to $K'$.
\end{thm}

\parit{Proof modulo two lemmas.}
Consider the graph $\Gamma$ formed by the edges of $K$;
the edges of $K'$ form a naturally isomorphic graph.
 
For an edge $e$ in $\Gamma$, denote by $\alpha_e$ and $\alpha'_e$ the dihedral angles in $K$ and $K'$, respectively.
Mark $e$ by plus if $\alpha_e < \alpha'_e$ and by minus if $\alpha_e > \alpha'_e$.

Let us remove from $\Gamma$ everything that is not marked;
that is, leave only the edges marked by $(+)$ or $(-)$ and their endpoints.
If $\Gamma$ is an empty graph, then the theorem follows.
Let us assume the contrary.

The graph $\Gamma$ is embedded into $P$, which is homeomorphic to the sphere.
In particular, the edges coming from one vertex have a natural cyclic order. 
Given a vertex $v$ of $\Gamma$, count the \textit{number of sign changes} around $v$;
that is, the number of consequent pairs edges with different signs. 

\begin{thm}{Local lemma}\label{lem:local}
For any vertex of  $\Gamma$ the number of sign changes is at least $4$.
\end{thm}

In other words, at each vertex of $\Gamma$, one can choose 4 edges marked by $(+)$, $(-)$, $(+)$, $(-)$ in the same cyclical order.
Note that the local lemma contradicts the following.

\begin{thm}{Global lemma}\label{lem:global}
Let $\Gamma$ be a nonempty planar graph.
Then it is impossible to mark all of the edges of $\Gamma$ by $(+)$ or $(-)$
in such a way  that the number of sign changes around each vertex of $\Gamma$ is at least $4$.
\end{thm}

It remains to prove these two lemmas.
\qeds


\section{Arm lemma}

\begin{thm}{Lemma}\label{lem:arm}
Assume that $A=[a_0 a_1\dots a_n]$ is a convex polygon in $\EE^2$
and $A'=[a'_0 a'_1\dots a'_n]$ be a polygonal line in $\EE^3$
such that 
$|a_i-a_{i+1}|=|a'_i-a'_{i+1}|$ for any $i\in\{0,\dots,n-1\}$
and 
$\measuredangle a_i\le \measuredangle a'_i$
for each $i\in\{1,\dots,n-1\}$.
Then 
$$|a_0-a_n|\le |a'_0-a'_n|$$
and equality holds if and only if $A$ is congruent to $A'$.
\end{thm}

One may view the polygonal lines $[a_0a_1\dots a_n]$ and $[a'_0a'_1\dots a'_n]$ as a robot's arm in two positions.
Informally speaking, the arm lemma says that when the arm opens,
the distance between the shoulder and tip of a finger increases;
assuming that starting position a convex plane polygon.

\begin{thm}{Exercise}\label{ex:arm-nonconvex}
Show that the arm lemma does not hold if 
instead of the convexity,
one only the local convexity;
that is, if we assume only that the polygonal line $a_0 a_1\dots a_n$ turns only left.
\end{thm}

\begin{thm}{Exercise}\label{ex:cauchy}
Suppose $A=[a_1\dots a_n]$ and $A'=[a'_1\dots a'_n]$ be noncongruent convex plane polygons with equal corresponding sides.
Mark each vertex $a_i$ with plus (minus) if the interior angle of $A$ at $a_i$ is smaller (respectively bigger) than the interior angle of $A'$ at $a_i'$.
Show that there are at least 4 sign changes around $A$. %+PIC

Give an example showing the statement does not hold without assuming convexity.

\end{thm}

\parit{Proof.}
We will view $\EE^2$ as the $xy$-plane in~$\EE^3$; 
so both $A$ and $A'$ lie in~$\EE^3$.

Let $a_m$ be the vertex of $A$ that lies on the maximal distance to the line $(a_0a_n)$.
Let us shift indexes of $a_i$ and $a'_i$ down by $m$,
so that 
\begin{align*}
a_{-m}&:=a_0,
&&\dots
&
a_{0}&:=a_m,
&&\dots
&
a_k&:=a_n,
\\
a'_{-m}&:=a'_0,
&&\dots
&
a'_{0}&:=a'_m,
&&\dots
&
a'_k&:=a'_n,
\end{align*}
where $k=n-m$.
(Here the symbol ``$:=$'' means an assignment as in programming.)

Without loss of generality, we may assume that
\begin{itemize}
\item $a_0=a'_0$ and they both coincide with the origin in $\EE^3$;
\item all $a_i$ lie in the $xy$-plane and the $x$-axis is parallel to the line $a_{-m}a_k$;
\item the angle $\measuredangle a'_0$ lies in $xy$-plane and contains the angle $\measuredangle a_0$ inside so that the directions to $a'_{-1}$,$a_{-1}$, $a_{1}$ and $a'_{1}$ from $a_0$ appear in the same cyclic order.
\end{itemize}

Denote by $x_i$ and $x'_i$ the projections of $a_i$ and $a'_i$ to the $x$-axis.
We can assume in addition that $x_k\ge x_{-m}$.
In this case,
$$|a_k-a_{-m}|=x_k-x_{-m}.$$
Since the projection is a distance non-expanding, we also have
$$|a'_k-a'_{-m}|\ge x'_k-x'_{-m}.$$ 

\begin{wrapfigure}{r}{60mm}
\vskip-5mm
\centering
\includegraphics{mppics/pic-30}
\vskip3mm
\end{wrapfigure}

Therefore it is sufficient to show
that 
$$x'_k-x'_{-m}\ge x_k-x_{-m}.$$
The latter holds if
$$x'_i-x'_{i-1}\ge x_i-x_{i-1}.\eqlbl{eq:|bb|=<|aa|}$$
for each $i$.
It remains to prove \ref{eq:|bb|=<|aa|}.

Let us assume that $i>0$; 
the case $i\le 0$ is similar.
Denote by $\sigma_i$ ($\sigma'_i$) the angle between the vector $w_i=a_{i}-a_{i-1}$ (respectively $w_i'=a'_{i}-a'_{i-1}$) and the $x$-axis.
Note that
$$\begin{aligned}
x_i-x_{i-1}&=|a_i-a_{i-1}|\cdot\cos\sigma_i,
\\
x'_i-x'_{i-1}&=|a_i-a_{i-1}|\cdot\cos\sigma'_i
\end{aligned}
\eqlbl{eq:proj}$$
for each $i>0$.
By construction $\sigma_1\ge \sigma'_1$.
Note that $\measuredangle (w_{i-1},w_i)\z=\pi -\measuredangle a_i$.
From convexity of $[a_1 a_1\dots a_i]$, we have
$$\sigma_i=\sigma_1+(\pi-\measuredangle a_1)+\dots+(\pi-\measuredangle a_i)$$
 for any $i>0$.
Since $\measuredangle (w'_{i-1},w'_i)=\pi -\measuredangle a'_i$,
applying the triangle inequality for angles (\ref{ex:angle-triangle}) several times,
we get
$$\sigma'_i\le\sigma'_1+(\pi-\measuredangle a'_1)+\dots+(\pi-\measuredangle a'_i).$$
Since $\measuredangle a'_j\ge \measuredangle a_j$ for each $j$, we get
$\sigma'_i\le \sigma_i$, and therefore
\[\cos \sigma'_i\ge \cos\sigma_i\]
Applying \ref{eq:proj}, we get \ref{eq:|bb|=<|aa|}.

In the case of equality, we have $\sigma_i=\sigma'_i$,
which implies $\measuredangle a_i=\measuredangle a'_i$ for each $i$.
This also implies that all $A'$ is a convex polygonal line in the $xy$-plane.
The latter easily follows from the equality case in \ref{ex:angle-triangle}.
\qeds

\begin{thm}{Advanced exercise}\label{ex:arm'}
Let $A$ and $A'$ be as in the arm lemma (\ref{lem:arm}).

\begin{subthm}{ex:bow'+}
Suppose that $\measuredangle \hinge{a_n}{a_{n-1}}{a_0}\le\tfrac\pi2$.
Show that $\measuredangle \hinge{a_0}{a_1}{a_n}\ge \measuredangle \hinge{a_0'}{a_1'}{a_n'}$.
\end{subthm}

\begin{subthm}{ex:bow'-} Show that the inequality $\measuredangle \hinge{a_0}{a_1}{a_n}\ge \measuredangle \hinge{a_0'}{a_1'}{a_n'}$ does not hold in general.
\end{subthm}

\end{thm}

\section{Proof of local lemma}
 
\parit{Proof of the local lemma (\ref{lem:local}).}
Assume that the local lemma does not hold at the vertex $v$ of $\Gamma$.
Choose a plane that cuts from $P$ a small pyramid $\Delta$ with the vertex~$v$.
One can choose two points $a$ and $b$ on the base of $\Delta$
so that on one side of the segments $[va]$ and $[vb]$ we have only pluses
and on the other side only minuses.

The base of $\Delta$ has two polygonal lines with ends at $a$ and $b$.
Choose the one that has only pluses;
denote it by $a_0 a_1 \dots a_n$;
so $a=a_0$ and $b=a_n$.
Denote by $a'_0 a'_1 \dots a'_n$
the corresponding line in $P'$;
let $a'=a'_0$ and $b'=a'_n$.

{

\begin{wrapfigure}{r}{40mm}
\vskip-0mm
\centering
\includegraphics{mppics/pic-40}
\vskip-0mm
\end{wrapfigure}

Since each marked edge passing thru $a_i$ has a $(+)$ on it or nothing, 
we have 
$$\measuredangle a_i\le\measuredangle a'_i$$
for each $i$.

}

\begin{thm}{Exercise}\label{ex:a<a}
Prove the last statement. 
\end{thm}

By the construction we have $|a_i-a_{i-1}|=|a'_i-a'_{i-1}|$ for all $i$.
By the arm lemma (\ref{lem:arm}), 
we get 
\[|a-b|\le |a'-b'|.
\eqlbl{clm:ab<ab}\]

Swap $K$ and $K'$ and repeat the same construction for a plane passing thru $a'$ and $b'$.
We get
\[|a-b|\ge |a'-b'|.
\eqlbl{clm:ab>ab}\]

The inequalities
\ref{clm:ab<ab} and \ref{clm:ab>ab} 
together imply $|a-b|=|a'-b'|$.
The equality case in the arm lemma implies that no edge at $v$ is marked;
that is, $v$ is not a vertex of $\Gamma$
--- a contradiction.
\qeds

From the proof, it follows that the local lemma is indeed local --- it works for two nonconguent convex polyhedral angles with equal corresponding faces.
Use this observation in the following exercise.

\begin{wrapfigure}{r}{20mm}
\vskip-0mm
\centering
\includegraphics{mppics/pic-10}
\bigskip
\includegraphics{mppics/pic-20}
\vskip-0mm
\end{wrapfigure}

\begin{thm}{Exercise}\label{ex:disc-bend}
Consider two polyhedral discs in $\EE^3$ glued from regular polygons by the rule on the diagrams.
Assume that each disc is part of a surface of a convex polyhedron.

\begin{subthm}{}
The first configuration is rigid; that is, one can not fix the position of the pentagon and continuously move the remaining 5 vertices in a new position so that each triangle moves by a one-parameter family of isometries of $\EE^3$.
\end{subthm}

\begin{subthm}{}
Show that the second configuration has a rotational symmetry with the axis passing thru the midpoint of the marked edge.
\end{subthm}

\end{thm}

\section{Proof of global lemma}

It is instructive to do the next exercise before diving into the proof.

\begin{thm}{Exercise}\label{ex:octahedron}
Try to mark the edges of an octahedron
by pluses and minuses
such that there would be 4 sign changes at each vertex.

Show that this is impossible.
\end{thm}

The proof of the global lemma is based on counting the sign changes
in two ways;
first while walking around each vertex of $\Gamma$
and second while moving around each of the regions separated by $\Gamma$
on the surface~$P$.
If two edges are adjacent at a vertex,
then they are also adjacent in a region.
The converse is true as well.
Therefore, both countings give the same number.

\parit{Proof of \ref{lem:global}.}
We can assume that $\Gamma$ is connected;
that is, one can get from any vertex to any other vertex by walking along edges.
(If not, pass to a connected component of $\Gamma$.)

We can assume that $\Gamma$ is embedded in the sphere.
Denote by $k$ and $l$ the number of vertices and edges in $\Gamma$.
Denote by $m$ the number of \textit{regions} that $\Gamma$ cuts from $P$.
Since $\Gamma$ is connected, each region is homeomorphic to an open disc.

\begin{thm}{Exercise}\label{ex:disc}
Prove the last statement.
\end{thm}

Now we can apply Euler's formula
$$k-l+m=2.
\eqlbl{eq:cauchy:euler}$$

Denote by $s$ the total number of sign changes in $\Gamma$ for all vertices. 
By the local lemma (\ref{lem:local}), we have 
$$ 4\cdot k\le s.\eqlbl{eq:S>=4k}$$

Let us get an upper bound on $s$ by counting the number of sign changes when one walks around
each region. 
Denote by $m_n$ the number of regions bounded by $n$ edges;
if an edge appears twice when it is counted twice.
Note that each region is bounded by at least $3$ edges;
therefore
$$m=m_3+m_4+m_5+\dots\eqlbl{eq:3-4-5}$$
Since edge belongs to exactly two regions, we get
$$2\cdot l=3\cdot m_3+ 4\cdot m_4+5\cdot m_5+\dots$$
Combining this with Euler's formula \ref{eq:cauchy:euler}, we get
$$4\cdot k=8+2\cdot m_3+4\cdot m_4+6\cdot m_5+8\cdot m_6+\dots
\eqlbl{eq:k=2+}$$
Observe that the number of sign changes in $n$-gon regions has to be even and $\le n$.
Therefore
$$s \le 2\cdot m_3 + 4\cdot m_4 + 4\cdot m_5 + 6\cdot m_6+\dots
\eqlbl{eq:23-44-45}$$
Clearly, \ref{eq:S>=4k} and \ref{eq:23-44-45} contradict \ref{eq:k=2+}.
\qeds


\section{Alexandrov uniqueness theorem}

Alexandrov's uniqueness theorem states that the conclusion of the Cauchy theorem (\ref{thm:cauchy}) still holds without the face-to-face assumption.

\begin{thm}{Theorem}\label{thm:alexandrov-uni'}
Any two convex polyhedrons in $\EE^3$ with isometric surfaces are congruent.

Moreover, any isometry between surfaces of convex polyhedrons can be extended to an isometry of the whole $\EE^3$.
\end{thm}

Instead of proof we list the modifications needed in the proof of Cauchy's theorem.

\parit{List of modifications in the proof of \ref{thm:cauchy}.}
Suppose $\iota\:P\z\to P'$  is an isometry between surfaces of $K$ and $K'$.
Mark in $P$ all the edges of $K$ and all the inverse images of edges in $K'$.
It might happen that an edge in $K'$ does not correspond to an edge in $K$;
it this case its inverse image in $K$ will be called a \index{fake edge}\emph{fake edge} of $K$.

The marked lines divide $P$ into convex polygons, and the restriction of $\iota$ to each polygon is a rigid motion.
These polygons should be used instead of faces in the proof of \ref{thm:cauchy}.

A vertex of the obtained graph can be a vertex of $K$, or it can be a fake vertex;
that is, it might be an intersection of an edge and a fake edge.

\begin{figure}[ht!]
\vskip-0mm
\centering
\includegraphics{mppics/pic-50}
\vskip-0mm
\end{figure}

For the first type of vertex, the local lemma can be proved the same way.
For a fake vertex $v$, it is easy to see that both parts of the edge coming thru $v$ are marked with minus
while both of the fake edges at $v$ are marked with plus.
Therefore, the local lemma holds for the fake vertices as well.

The remaining parts of the proof need no modifications.
\qeds

\begin{thm}{Exercise}\label{pr:K-P-simmetry}
Let $K$ be a convex polyhedron in $\EE^3$;
denote by $P$ its surface.
Show that each isometry $\iota\:P\z\to P$,
can be extended to an isometry of $\EE^3$.
\end{thm}


\section{Remarks}

This lecture contains selected material from Alexandrov's book~\cite{alexandrov}.

In Euclid's Elements, 
solids were claimed to be equal if the same holds for their faces, but no proof was given.
Adrien-Marie Legendre became interested in this problem towards the end of the 18th century.
He discussed it with his colleague Joseph-Louis Lagrange, who suggested this problem to Augustin-Louis Cauchy in 1813; soon he proved it \cite{cauchy}.
This theorem is included in many popular books \cite{aigner-zigler,dolbilin,tabacnikov-fuks}.
The key observation that the face-to-face condition can be removed was made by
Alexandr Alexandrov \cite{alexandrov-1941}.

\parit{Arm lemma.}
Cauchy's proof \cite{cauchy}
also used a version of the arm lemma, but its proof contained a small mistake that was corrected in a century \cite{sabitov}.

Several proofs of the arm lemma can be found in the letters between Isaac Schoenberg and Stanisław Zaremba \cite{schoenberg-zaremba}.

The following variation of the arm lemma makes sense for nonconvex spherical polygons.
It is due to Viktor Zalgaller \cite{zalgaller}.
It can be used instead of the standard arm lemma.

\begin{thm}{Another arm lemma}
Let $A=[a_1\dots a_n]$ and $A'\z=[a'_1\z\dots a'_n]$ be two spherical $n$-gons (not necessarily convex).
Assume that $A$ lies in a half-sphere,
the corresponding sides of $A$ and $A'$ are equal
and each angle of $A$ is at least the corresponding angle in $A'$.
Then $A$ is congruent to~$A'$. 
\end{thm}

Another close relative of the arm lemma is Reshetnyak's majorization theorem \cite{reshetnyak}.

\parit{Global lemma.}
A more visual proof of the global lemma is given in \cite[II \S 1.3]{alexandrov}.
This argument reused by Anton Klyachko \cite{klyachko} in his \index{car-crash lemma}\emph{car-crash lemma}.

\parit{Approximation.}
Proposition \ref{prop:conv-surf-CBB(0)} generalizes to boundaries of convex bodies  in $\EE^m$ for any $m\ge 2$.
This could be considered as a partial case of the conjecture about boundary of Alexandrov space; see \ref{conj:bry}.
Another partial case is proved by the authors and Stephanie Alexander \cite{alexander-kapovitch-petrunin-2008}.

\parit{Existance theorem.}
\ref{ex:surf-S2} and \ref{prop:poly-CBB} imply that the surface of a convex body is a sphere with nonnegative curvature in the sense of Alexandrov.
The celebrated theorem of Alexandrov states that the converse also holds if we allow degeneration of convex bodies to plane figures;
the surface of a plane figure is defined as its doubling across the boundary.
In other words, any $\Alex0$ metric on the two-sphere is isometric to a surface of a (possibly degenerate) convex body.
Moreover this convex body is unique up to congruence.
The last part is due to Alexei Pogorelov \cite{pogorelov}.

Originally, Alexandrov proved the statement for polyhedral metrics on the sphere; this proof is sketched in the appendix.
Then he used approximation to extend the result to  arbitrary $\Alex0$ metrics on the sphere.



%\chapter{Misc}

\section{Existence}\label{sec:Alexandrov-existence}

\begin{thm}{Theorem}\label{thm:exist}
A polyhedral metric on the sphere is isometric to the surface of a convex polyhedron (possibly degenerate to a flat polygon) if and only if it has nonnegative curvature at each point.
\end{thm}

\begin{wrapfigure}{r}{30mm}
\vskip-5mm
\centering
\includegraphics{mppics/pic-1010}
\vskip-0mm
\end{wrapfigure}

By \ref{thm:alexandrov-uni'}, a convex polyhedron is completely defined by the intrinsic metric of its surface.
By \ref{thm:exist}, it follows that knowing the metric we could find the position of the edges.
However, in practice, it is not easy to do.

For example, the surface glued from a rectangle as shown on the diagram defines a tetrahedron.
Some of the glued lines appear inside facets of the tetrahedron and some edges (dashed lines) do not follow the sides of the rectangle.

\paragraph{Space of polyhedrons.}
Let us denote by $\mathbf{K}$ the space of all convex polyhedrons in the Euclidean space,
including polyhedrons that degenerate to a plane polygon.
Polyhedra in $\mathbf{K}$ will be considered up to a motion of the space,
and the whole space $\mathbf{K}$ will be considered with Hausdorff distance up to a motion of the space;
that is, the distance between $K$ and $K'$ is the exact lower bound on Hausdorff distance from $\iota(K)$ to $K'$, where $\iota$ is arbitrary motion of $\EE^3$.

Further, denote by $\mathbf{K}_n$ the polyhedrons in $\mathbf{K}$ with exactly $n$ vertices.
Since any polyhedron has at least 3 vertices, the space $\mathbf{K}$ admits a subdivision into a countable number of subsets $\mathbf{K}_3,\mathbf{K}_4,\dots$

\paragraph{Space of polyhedral metrics.}
The space of polyhedral metrics on the sphere with nonnegative curvature will be denoted by $\mathbf{P}$.
The metrics in $\mathbf{P}$ will be considered up to an isometry, and the whole space $\mathbf{P}$ will be equipped with the topology induced by the Gromov--Hausdorff metric.

The subset of $\mathbf{P}$ of all metrics with exactly $n$ essential vertices will be denoted by $\mathbf{P}_n$.
It is easy to see that any metric in $\mathbf{P}$ has at least 3 essential vertices.
Therefore $\mathbf{P}$ is subdivided into countably many subsets
 $\mathbf{P}_3,\mathbf{P}_4,\dots$

\paragraph{From a polyhedron to its surface.}

By \ref{prop:poly-CBB}, passing from a polyhedron to its surface defines a map
\[\iota\:\mathbf{K}\to \mathbf{P}.\]

By \ref{ex:vertex-essential-vertex}, the number of vertices of a polyhedron is equal to the number of essential vertices on its surface.
In other words, $\iota(\mathbf{K}_n)\subset \mathbf{P}_n$ for any $n\ge 3$.

Using the introduced notation, we can unite \ref{thm:alexandrov-uni'} and \ref{thm:exist} in the following more exact statement.

\begin{thm}{Reformulation}
For any integer $n\ge 3$,
the map $\iota$ induces a bijection between $\mathbf{K}_n$ and~$\mathbf{P}_n$.
\end{thm}

The proof is based on a construction of a one-parameter family of polyhedrons that starts at an arbitrary polyhedron
and ends at a polyhedron with its surface isometric to the given one.
This type of argument is called the \textit{continuity method}; it is often used in the theory of differential equations.


\parit{Sketch.}
By \ref{thm:alexandrov-uni'}, the map $\iota\:\mathbf{K}_n\to\mathbf{P}_n$ is injective.
Let us prove that it is surjective.

\begin{thm}{Lemma}
For any integer $n\ge 3$, the space $\mathbf{P}_n$ is connected.
\end{thm}

The proof of this lemma is not complicated, but it requires ingenuity;
it can be done by the direct construction of a one-parameter family of metrics in $\mathbf{P}_n$ that connects two given metrics.
Such a family can be obtained by а sequential application of the following construction and its inverse.

Let $P\in\mathbf{P}_n$.
Suppose $v$ and $w$ are essential vertices in $P$.
Let us cut $P$ along a geodesic from $v$ to $w$.
Note that the geodesic cannot pass thru an essential vertex of $P$.
Further, note that there is a three-parameter family of patches that can be used to patch the cut so that the obtained metric remains in $\mathbf{P}_n$;
in particular, the obtained metric has exactly $n$ essential vertices (after the patching, the vertices $v$ and $w$ may become inessential).


\begin{thm}{Lemma}
The map $\iota\:\mathbf{K}_n\to\mathbf{P}_n$ is open,
that is, it maps any open set in $\mathbf{K}_n$ to an open set in $\mathbf{P}_n$.

In particular, for any $n\ge 3$, the image $\iota(\mathbf{K}_n)$ is open in~$\mathbf{P}_n$.
\end{thm}

This statement is very close to the so-called \textit{invariance of domain theorem};
the latter states that a continuous injective map between manifolds of the same dimension is open.

Recall that $\iota$ is injective.
The proof of the invariance of domain theorem can be adapted to our case since both spaces $\mathbf{K}_n$ and $\mathbf{P}_n$ are $(3\cdot n-6)$-dimensional and both look like manifolds, altho, formally speaking, they are \textit{not} manifolds.
In a more technical language, $\mathbf{K}_n$ and $\mathbf{P}_n$ have the natural structure of $(3\cdot n-6)$-dimensional \textit{orbifolds},
and the map $\iota$ respects the \textit{orbifold structure}.

We will only show that both spaces $\mathbf{K}_n$ and $\mathbf{P}_n$ are $(3\cdot n-6)$-dimensional.

Choose  $K\in\mathbf{K}_n$.
Note that $K$ is uniquely determined by the $3\cdot n$ coordinates of its $n$ vertices.
We can assume that the first vertex is the origin, the second has two vanishing coordinates and the third has one vanishing coordinate; therefore, all polyhedrons in $\mathbf{K}_n$ that lie sufficiently close to $K$ can be described by $3\cdot n-6$ parameters.
If $K$ has no symmetries, then this description can be made one-to-one;
in this case, a neighborhood of $K$ in $\mathbf{K}_n$ is a $(3\cdot n-6)$-dimensional manifold.
If $K$ has a nontrivial symmetry group, then this description is not one-to-one but it does not have an impact on the dimension of~$\mathbf{K}_n$.

The case of polyhedral metrics is analogous.
We need to construct a subdivision of the sphere into plane triangles using only essential vertices.
By Euler's formula, there are exactly $3\cdot n-6$ edges in this subdivision.
Note that the lengths of edges completely describe the metric, and slight changes in these lengths produce a metric with the same property.
Again, if $P$ has no symmetries, then this description is one-to-one.

\begin{thm}{Lemma}
The map $\iota\:\mathbf{K}_n\to\mathbf{P}_n$ is closed;
that is, the image of a closed set in $\mathbf{K}_n$ is closed in $\mathbf{P}_n$.

In particular, for any $n\ge 3$, the set $\iota(\mathbf{K}_n)$ is closed in~$\mathbf{P}_n$.
\end{thm}

Choose a closed set $Z$ in $\mathbf{K}_n$.
Denote by $\bar Z$ the closure of $Z$ in $\mathbf{K}$; note that $Z=\mathbf{K}_n\cap \bar Z$.
Assume $K_1,K_2,\dots\in Z$ is a sequence of polyhedrons that converges to a polyhedron $K_\infty\in\bar Z$.
By \ref{lem:H>GH}, $\iota(K_n)$ converges to $\iota(K_\infty)$ in $\mathbf{P}$.
In particular, $\iota(\bar Z)$ is closed in $\mathbf{P}$.

Since $\iota(\mathbf{K}_n)\subset \mathbf{P}_n$ for any $n\ge 3$, we have $\iota (Z)=\iota(\bar Z)\cap \mathbf{P}_n$;
that is, $\iota (Z)$ is closed in $\mathbf{P}_n$.

\medskip

Summarizing, $\iota(\mathbf{K}_n)$ is a nonempty closed and open set in $\mathbf{P}_n$, and $\mathbf{P}_n$ is connected for any $n\ge 3$.
Therefore, $\iota(\mathbf{K}_n)=\mathbf{P}_n$; that is, $\iota\:\mathbf{K}_n\z\to\mathbf{P}_n$ is surjective.
\qeds

\section{Approximation}

By now, the embedding theorem is proved for polyhedral metrics on the sphere.
The general case is done by approximation, using the following statement.

\begin{thm}{Proposition}\label{prop:H>GH}
Let $K_1,K_2,\dots$ be a sequence of convex bodies that converge to $K_\infty$ in the sense of Hausdorff.
Then the surface of $K_n$ converges to the surface of $K_\infty$ in the sense of Gromov--Hausdorff.
\end{thm}

If $K_\infty$ is nondegenerate, then the statement follows from \ref{lem:H>GH}.
The degenerate case is left as an exercise.

Let $\spc{X}_\infty$ be an $\Alex0$ space that is homeomorphic to $\SSS^2$.
Suppose that $\spc{X}_\infty$ is a Gromov--Hausdorff limit of a sequence of spheres with polyhedral metrics $\spc{X}_1,\spc{X}_2,\dots$
By \ref{thm:exist}, there is a sequence of convex polyhedra $K_1,K_2,\dots$ with surfaces isometric to $\spc{X}_1,\spc{X}_2,\dots$, respectively.
Note that  $\diam K_n\le \diam \spc{X}_n$ for any $n$.
Therefore we can assume that all polyhedra $K_1,K_2,\dots$ lie in a closed ball of sufficiently large radius.

Applying Blaschke selection theorem, we can pass to a subsequence of $K_1,K_2,\dots$ that converges in the sense of Hausdorff; denote its limit by $K_\infty$.
By \ref{prop:H>GH} the surface of $K_\infty$ is isometric to $\spc{X}_\infty$.

Therefore it remains to prove the following lemma.

\begin{thm}{Lemma}\label{lem:GH-approximation}
Let $\spc{X}$ be an $\Alex0$ space that is homeomorphic to $\SSS^2$.
Then there is a sphere with polyhedral metrics $\spc{X}'$
that is arbitrarily close to $\spc{X}$ in the sense of Gromov--Hausdorff.
\end{thm}

\parit{Proof with two cheatings.}
Suppose we can triangulate $\spc{X}_\infty$ by small geodesic triangles;
that is, we can choose a finite set of points $p_1,\dots,p_n\z\in \spc{X}_\infty$ and some geodesics $[p_ip_j]$ that cut $\spc{X}_\infty$ into regions of small diameter bounded by geodesic triangles $[p_ip_jp_k]$.
(This is the first chating, the actual proof uses a triangulation with a weaker property.)

Observe that total angle around each $p_i$ cannot exceed $2\cdot \pi$.
That is, suppose $p_{j_1},\dots,p_{j_k}$ are points connected to $p_i$ by geodesics.
Assume that they are ordered in the natural cyclic order.
Then
\[\mangle\hinge{p_i}{p_{j_1}}{p_{j_2}}+\dots+\mangle\hinge{p_i}{p_{j_{k-1}}}{p_{j_k}}+\mangle\hinge{p_i}{p_{j_{k}}}{p_{j_1}}\le 2\cdot\pi.\]
By comparison, we get
\[\angk{p_i}{p_{j_1}}{p_{j_2}}+\dots+
\angk{p_i}{p_{j_{k-1}}}{p_{j_k}}+\angk{p_i}{p_{j_{k}}}{p_{j_1}}\le 2\cdot\pi.\eqlbl{eq:sum<=<2pi}\]

Now let us exchange each triangle by its model triangle.
That is, consider a model triangle for each region in the subdivision of $\spc{X}$ and glue them together by the same rule.
By \ref{eq:sum<=<2pi}, the obtained polyhedral surface $\spc{X}'$ has nonnegative curvature.
It remains to show that this way we can produce $\spc{X}'$ arbitrarily close to $\spc{X}$.

Denote by $p_i\to p_i'$ the natural map; it takes $p_i$ in $\spc{X}$ and returns the corresponding point in $\spc{X}'$.
Observe that
\[\dist{p_i'}{p_j'}{\spc{X}'}
\le
\dist{p_i}{p_j}{\spc{X}}.\eqlbl{eq:|pp|}\]
Indeed, choose a geodesic $\gamma$ from $p_i$ to $p_j$.
Let $p_i=x_0,x_1,\dots,x_n=p_j$ be the points of intersections of $\gamma$ with the edges of the triangulation listed as they appear on $\gamma$.
For each $i$, denote by $x_i'$ the corresponding point in $\spc{X}'$.
By comparison, we get
\[\dist{x_k'}{x_{k-1}'}{\spc{X}'}
\le
\dist{x_k}{x_{k-1}}{\spc{X}}.\]
for each $k$.
Therefore, \ref{eq:|pp|} follows.

Suppose $\eps>0$ is small, the points $p_1,\dots,p_n$ form an $\eps$-net in $\spc{X}$, all edges of the triangulation are smaller than $\eps$ and
\[\dist{p_i'}{p_j'}{\spc{X}'}
\ge
\dist{p_i}{p_j}{\spc{X}} -100\cdot \eps.\eqlbl{eq:|pp|>=}\]
Then, together with the inequality above it proves the lemma.

Now let us assume that the sides of model triangles in $\spc{X}'$ are geodesics.
(This is the second cheating; the sides of the model triangles are local geodesics in $\spc{X}'$,
but not necessarily geodesic; that is, they do not have to be length-minimizing.
The actual proof does not use this assumption.)

Choose a geodesic $\gamma'$ from $p_i'$ to $p_j'$ in $\spc{X}'$.
Note that $\gamma'$ visits each triangle in the triangulation of $\spc{X}'$ at most once.

Let $p_i'=x_0',x_1',\dots,x_n'\z=p_j'$ be the points of intersections of $\gamma'$ with the edges of the triangulation listed from $p_i'$ to $p_j'$.
For each $i$, denote by $x_i$ the corresponding point in $\spc{X}$.
Let $\Delta_k'$ be the triangle that contains arc $[x'_{k-1}x'_k]$ of $\gamma'$ and $\Delta_k$ the corresponding triangle in~$\spc{X}$.
Note that
\[\dist{x_k'}{x_{k-1}'}{\spc{X}'}
\ge
\dist{x_k}{x_{k-1}}{\spc{X}} -\eps\cdot K(\Delta_k),
\eqlbl{eq:|xx|<}\]
where $K(\Delta_k)$ denotes the access of $\Delta_k$;
that is, the sum of its internal angles minus $\pi$.

Euler's formula and \ref{eq:sum<=<2pi} imply that the sum of all accesses is at most $4\cdot\pi$.
Therefore, summing up \ref{eq:|xx|<}, we get
\[\dist{p_i'}{p_j'}{\spc{X}'}
\ge
\dist{p_i}{p_j}{\spc{X}}-4\cdot \pi\cdot \eps.\]
Whence \ref{eq:|pp|>=} follows.
\qeds

\section{Comments}

\parit{Existence theorem.}
This theorem was proved by Alexandr Alexandrov~\cite{alexandrov-1941}.
Our sketch is taken from \cite{lebedeva-petrunin};
a complete proof is nicely written in~\cite{alexandrov}.
In the original proof, the spaces $\mathbf{K}_n$ and $\mathbf{P}_n$ were modified so the they become $(3\cdot n-6)$-dimensional manifolds.
It was done by introducing extra structure (for $\mathbf{K}_n$ it is orientation + a marked vertex and an edge) that \textit{brakes symmetries} of the spaces.
After that one could apply the domain invariance theorem directly.
Alternatively, one may first remove from $\mathbf{K}_n$ and $\mathbf{P}_n$ elements (polyhedron or surface)with nontrivial symmetries (after that the spaces become manifolds) and show that any symmetric polyhedron (or surface) can be approximated by a non-symmetric polyhedron (or surface).

A very different proof was found by Yuri Volkov in his thesis \cite{volkov};
it uses a deformation of three-dimensional polyhedral space.

%P and \Sigma???



\chapter{Surface theory}\label{chap:surfaces}

This lecture is less rigorous;
it aims to demonstrate beauty of geometry of convex surfaces, which is the precursor of modern Alexandrov geometry.
For a deeper dive into this theory, we recommend turning to the classic and brilliantly written books by Alexandr Alexandrov \cite{alexandrov,alexandrov-1948}.
Also, the book by Alexey Pogorelov \cite{pogorelov1969} is very recommended, despite being a challenge to read.


\section{Polyhedral surfaces}

A \index{polyhedral surface}\emph{polyhedral surface} is defined as a 2-dimensional manifold (possibly with boundary) with a length metric that admits a finite triangulation such that each triangle is isometric to a solid plane triangle.
A \index{triangulation}\emph{triangulation} of a polyhedral surface will always be assumed to satisfy this condition.

Note that according to our definition, any polyhedral surface is compact.

Choose a point $p$ on a polyhedral surface $\spc{P}$.
We can assume that $p$ is a vertex of a triangulation $\spc{P}$;
it can be achieved by subdividing the triangulation.
Denote by $\theta_p$ the \emph{total angle} around $p$;
that is, the sum of all angles at $p$ in all the triangles that have $p$ as a vertex.

Note that $\theta_p$ does not depend on the choice of triangulation.
If $p$ is an interior point, then the value $2\cdot\pi-\theta_p$ is called \emph{curvature} at $p$.
If $p$ lies on the boundary of $\spc{P}$, then the value $\pi-\theta_p$ is called \emph{inner turn} at $p$.

A point with nonvanishing curvature or inner turn will be called an \emph{essential vertex} of the surface.
Observe that an essential vertex is a vertex in any triangulation.

\begin{thm}{Exercise}\label{ex:geodesic-vertex}
Show that geodesics on a polyhedral surface with nonnegative curvatures and nonnegative inner turns may have essential vertices only at their endpoints.
\end{thm}

The following statement is an analog of the Gauss--Bonnet formula.

\begin{thm}{Exercise}\label{ex:gauss-bonnet}
Let $K(\spc{P})$ and $T(\spc{P})$ denote the sum of curvatures of all interior points
and the sum of all inner turns of the boundary points a polyhedral surface $\spc{P}$.
Show that
\[K(\spc{P})+T(\spc{P})=2\cdot\pi\cdot\chi(\spc{P}),\]
where $\chi(\spc{P})$ denotes the Euler characteristic of $\spc{P}$.
\end{thm}

The following proposition states that this new definition of curvature agrees with the $\Alex0$ comparison.

\begin{thm}{Proposition}\label{prop:poly-CBB}
A polyhedral surface is $\Alex0$ if and only if it has nonnegative curvature at every inner point and and nonnegative inner turn at each boundary point.
\end{thm}

\parit{Proof.}
By \ref{comp-kappa}, it is sufficient to check that
$f=\tfrac12\cdot\distfun_p^2\circ\gamma$ is 1-concave for any geodesic $\gamma$ and any point $p$.

We can assume that $p$ is not a vertex and the endpoints of $\gamma$ are not vertices;
the vertex case can be done by approximation.
By \ref{ex:geodesic-vertex}, $\gamma$ does not contain vertices.

Given a point $x=\gamma(t_0)$, choose a geodesic $[px]$.
Again, by \ref{ex:geodesic-vertex}, $[px]$ does not contain vertices.
Therefore, a neighborhood $U\supset [px]$ can be unfolded on a plane;
that is, there is an injective length-preserving map $z\mapsto \tilde z$
of $U$ into the Euclidean plane.
This way we map the part of $\gamma$ in $U$ to a line segment $\tilde\gamma$.
Let
\[\tilde f(t)\df\tfrac12\cdot\distfun_{\tilde p}^2\circ\tilde \gamma(t).\]
Since the geodesic $[px]$ maps to a line segment, we have $\tilde f(t_0)= f(t_0)$.
Furthermore, since the unfolding $z\mapsto \tilde z$ preserves lengths of curves, we get
$\tilde f(t)\ge f(t)$ if $t$ is close to $t_0$.
That is, $\tilde f$ is a local upper barrier of $f$ at $t_0$; see \ref{Function comparison}.
Evidently, $\tilde f''(t)\equiv 1$.
Therefore, $f$ is 1-concave.

\begin{thm}{Exercise}\label{ex:poly-CBB}
The converse is left to the reader.\qeds
\end{thm}

\section{Approximation}

The following theorem is the main extra tool available in Alexandrov geometry of surfaces.
We will use this statement in the proof of \ref{cor:Alex0-convex} to reduce questions about $\Alex0$ surfaces to polyhedral surfaces with nonnegative curvature.

\begin{thm}{Theorem}\label{thm:approximation}
Any closed $\Alex0$ surface is a Gromov--Hausdorff limit of homeomorphic polyhedral surfaces with nonnegative curvature.
\end{thm}

The construction of polyhedral approximations is based on the following exercise.

\begin{thm}{Exercise}\label{ex:construction}
Let $\spc{P}$ be a closed $\Alex0$ surface.

\begin{subthm}{ex:approximation:nbhd}
Show that any point $p$ admits an arbitrary small closed convex polygonal neighborhood $N$;
that is, $N$ is convex and bounded by a broken geodesic.

\end{subthm}

\begin{subthm}{ex:approximation:triangulation}
Given $\delta>0$, show that $\spc{P}$ admits a triangulation $\tau$ by convex triangles
with positive inner turn at each vertex and diameter smaller than $\delta$.
\end{subthm}

\begin{subthm}{ex:approximation:poly}
Suppose that $v$ is a vertex of a triangulation $\tau$ of $\spc{P}$ by convex triangles.
Let $\theta_v$ be the sum of angles at $v$ in all the triangles of~$\tau$.
Show that $\theta_v\le 2\cdot\pi$.
\end{subthm}

\end{thm}

\parit{Construction.}
Let $\spc{P}$ be a closed $\Alex0$ surface.
By part \ref{SHORT.ex:approximation:triangulation}, we can triangulate $\spc{P}$ by small convex triangles, say diameter of each triangle is less than given $\delta>0$.
Let us exchange each triangle of the triangulation by its model solid triangle; denote by $\tilde {\spc{P}}_\delta$ the obtained polyhedral surface.
Note that $\tilde {\spc{P}}_\delta$ is homeomorphic to $\spc{P}$;
moreover, there is a homeomorphism $\spc{P}\to \tilde {\spc{P}}_\delta$ that sends a point $x\in \spc{P}$ to a point $\tilde x\in \tilde {\spc{P}}_\delta$ in the corresponding model triangle.

By the angle comparison (\ref{angle-a}) and part \ref{SHORT.ex:approximation:poly} of the exercise, the total angle around each vertex in $\tilde {\spc{P}}_\delta$ cannot exceed $2\cdot\pi$.
That is, the obtained polyhedral space $\tilde {\spc{P}}_\delta$ has nonnegative curvature.
\qeds

Observe that Theorem \ref{thm:approximation} follows from the following.

\begin{thm}{Claim}\label{clm:approximation}
If $\tilde {\spc{P}}_\delta$ is provides by the construction, then $\tilde {\spc{P}}_\delta\to\spc{P}$ as $\delta\to 0$ in the sense of Gromov--Hausdorff.
\end{thm}

This claim seems to be self-evident, but it is not;
a very smart proof was given by Alexandrov \cite[VII §~6]{alexandrov-1948}.
We will indicate an alternative proof based on the following exercise and two theorems which will be stated without a proof.
The first theorem is due to Yuri Burago, Mikhael Gromov, and Grigori Perelman \cite[10.8]{burago-gromov-perelman};
it is a generalization of Alexandrov's theorem for surfaces \cite[X §~2]{alexandrov-1948}.
The second theorem is due to Nan Li \cite[Corollary 0.1]{li}.

\begin{thm}{Theorem}\label{thm:cont-vol}
Let $\spc{X}_1, \spc{X}_2$ be a sequence of $n$-dimensional $\Alex\kappa$ spaces that converges to $\spc{X}_\infty$ in the sense of Gromov--Hausdorff.
Then the $n$-volume on $\spc{X}_i$ weakly converges to the $n$-volume on $\spc{X}_\infty$.
\end{thm}

\begin{thm}{Theorem}\label{thm:vol-short}
Let $\spc{X}$ be an Alexandrov space without boundary, and let $\spc{Y}$ be an arbitrary Alexandrov space.
Then any short volume-preserving map $\spc{X}\to\spc{Y}$ is an isometry.
\end{thm}

Suppose a convex solid triangle $\Delta$ in an $\Alex0$ surface has angles $\alpha$, $\beta$ and $\gamma$.
Let us define its excess by
\[\excess\Delta=\alpha+\beta+\gamma-\pi.\]
Since the angles of a model triangle sum up to $\pi$, by the angle comparison (\ref{angle-a}),
the excess is nonnegative.

\begin{thm}{Exercise}\label{ex:approximation}
Let $\tau$ be a triangulation of a closed $\Alex0$ surface $\spc{P}$ by convex triangles $\Delta_1,\dots,\Delta_n$.


%\begin{subthm}{ex:approximation:diangle}
%Let $\Upsilon$ be a topological disc in $\tilde{\spc{P}}$ bounded by a geodesic $\gamma_0$ and local geodesic $\gamma_1$ with common endpoints.
%Let $\omega$ be the sum of the curvatures of points in $\Upsilon$.
%Show that
%\[\cos \omega\cdot \length \gamma_1\le \length \gamma_0\]
%if $\omega<\pi$.
%\end{subthm}


\begin{subthm}{ex:approximation:excess}
Show that
\[\excess\Delta_1+\dots+\excess\Delta_n \le 2\cdot\pi\cdot \chi(\spc{P}),\]
where $\chi(\spc{P})$ denotes the Euler characteristic of $\spc{P}$.
\end{subthm}


\begin{subthm}{ex:approximation:length}
Let $x$ and $y$ be points on the sides of a triangle $\Delta_i$, and let $\tilde x$ and $\tilde y$ be the corresponding points in the corresponding triangle $\tilde \Delta$ in $\tilde{\spc{P}}$.
Show that
\[\dist{\tilde x}{\tilde y}{\tilde \Delta}\le\dist{x}{y}{\Delta}\le \dist{\tilde x}{\tilde y}{\tilde \Delta}+\excess\Delta\cdot \diam \Delta.\]

\end{subthm}

\begin{subthm}{ex:approximation:area}
Let $\Delta$ be a solid triangle in the triangulation $\tau$ of $\spc{P}$, and $\tilde \Delta$ --- the corresponding triangle in $\tilde{\spc{P}}$.
Show that
\[\area \tilde \Delta\le \area \Delta\le \area \tilde \Delta+\tfrac12\cdot\excess\Delta\cdot (\diam \Delta)^2.\]

\end{subthm}

\end{thm}

Note that part \ref{SHORT.ex:approximation:excess} implies that $\chi(\spc{P})\ge 0$.
Therefore, $\spc{P}$ is homeomorphic to a sphere, projctive plane, torus, or Klein bottle.
In the latter two cases, the construction produces a flat surface $\tilde{\spc{P}}_\delta$, which has to be isometric to $\spc{P}$.
Therefore the cases of sphere, projctive plane are more interesting.

\parit{Proof of \ref{clm:approximation}.}
Choose a sequence of positive numbers $\delta_n\to0$;
let $\tilde{\spc{P}}_{\delta_n}$ be polyhedral spaces provided by the construction and let $\tau_n$ be the corresponding triangulation.

According to part \ref{SHORT.ex:approximation:length} of the exercise, the spaces $\tilde{\spc{P}}_{\delta_n}$ have bounded diameter.
Therefore by Gromov's selection theorem, we can pass to a converging sequence of $\tilde{\spc{P}}_{\delta_n}$;
denote its Gromov--Hausdorff limit by $\tilde{\spc{P}}$.
Note that if $\tilde{\spc{P}}_{\delta}$ does not converge to $\spc{P}$, then we can assume that $\tilde{\spc{P}}$ is not isometric to $\spc{P}$.

Choose two points $x,y\in \spc{P}$, and connect them by a geodesic.
Denote by $s_1,\dots,s_m$ the points of the geodesic on the sides of the triangulation $\tau_n$;
we assume that these points appear in the same order on the geodesic.
Denote by $\tilde x$, $\tilde y$, and $\tilde s_1,\dots,\tilde s_m$ the corresponding points in $\tilde{\spc{P}}_{\delta_n}$.
By part \ref{SHORT.ex:approximation:length} of the exercise,
\[\dist{\tilde s_{i-1}}{\tilde s_i}{\tilde{\spc{P}}_{\delta_n}}\le\dist{s_{i-1}}{s_i}{\spc{P}}.\]
Note also that
\[\dist{\tilde x}{\tilde s_1}{\tilde{\spc{P}}_{\delta_n}}\le \delta_n
\quad\text{and}\quad
\dist{\tilde s_m}{\tilde y}{\tilde{\spc{P}}_{\delta_n}}\le \delta_n
\]
Therefore
\[\dist{\tilde x}{\tilde y}{\tilde{\spc{P}}_{\delta_n}}\le\dist{x}{y}{\spc{P}}+2\cdot\delta_n.\]

Passing to the limit, we get a short onto map $\spc{P}\to \tilde{\spc{P}}$.
On the other hand, applying parts \ref{SHORT.ex:approximation:excess} and \ref{SHORT.ex:approximation:area}, we get that
\[
\area \spc{P} -\pi\cdot\chi(\spc{P})\cdot \delta_n^2
\le
\area \tilde{\spc{P}}_{\delta_n}
\le
\area \spc{P}
\]
By Theorem \ref{thm:cont-vol}, $\area \spc{P}= \area \tilde{\spc{P}}$.
Applying Theorem \ref{thm:vol-short}, we get that the short map $\spc{P}\to \tilde{\spc{P}}$ is an isometry --- a contradiction.
\qeds

\parit{Remark.}
The main difficulty in the proof comes from nonconvexity of triangles in the triangulation of $\tilde{\spc{P}}_\delta$.
If these triangles would be convex, then the first estimate in parts \ref{SHORT.ex:approximation:length} and \ref{SHORT.ex:approximation:excess} would imply that $\tilde{\spc{P}}_\delta$ is close to $\spc{P}$ in the sense of Gromov--Hausdorff.













\section{Surface of polyhedrons and bodies}

Let us define a \index{convex body}\emph{convex body} as a compact convex subset in $\EE^3$ with a non-empty interior.
The \index{surface}\emph{surface} of a convex body is defined as its boundary equipped with the induced length metric.

\begin{thm}{Exercise}\label{ex:surf-S2}
Show that the surface of a convex body is homeomorphic to the 2-dimensional sphere.
\end{thm}

A \index{convex polyhedron}\emph{convex polyhedron} is a convex body with a finite number of extremal points, called its \index{vertex}\emph{vertices}.

Note that the surface of a convex polyhedron $K$ is a closed polyhedral surface.

\begin{thm}{Exercise}\label{pr:tetrahedron}
Assume that the surface of a nondegenerate tetrahedron $T$ has curvature $\pi$ at each of its vertices.
Show that

\begin{subthm}{pr:tetrahedron:=}
all faces of $T$ are congruent;
\end{subthm}

\begin{subthm}{pr:tetrahedron:perp} the line containing the midpoints of opposite edges of $T$ intersects these edges at right angles.
\end{subthm}

\end{thm}

\begin{thm}{Claim}\label{clm:total-angle}
The surface $\spc{P}$ of any convex polyhedron $K$ has nonnegative curvature.
Moreover, a point $v$ is a vertex of $K$ if and only if
$v$ is an essential vertex of $\spc{P}$.
\end{thm}

A proof is given in Kiselyov's school textbook \cite[§ 48]{kiselev-stereo-en};
one can also deduce it from \ref{claim:angle-3angle-inq}.

\begin{thm}{Exercise}\label{ex:surface-covergence}
Let $K_1,K_2,\dots,$ and $K_\infty$ be convex bodies in $\EE^m$.
Denote by $\spc{P}_n$ the surface of $K_n$.
Suppose $K_n\z\to K_\infty$ in the sense of Hausdorff.
Show that $\spc{P}_n\to \spc{P}_\infty$ in the sense of Gromov--Hausdorff.
\end{thm}

Since any convex body is a Hausdorff limit of a sequence of convex polyhedrons, the next proposition follows from \ref{prop:poly-CBB}, \ref{ex:surface-covergence}, and \ref{thm:CBB-closed}.

\begin{thm}{Proposition}\label{prop:conv-surf-CBB(0)}
The surface of a convex body in $\EE^3$ is $\Alex0$.
\end{thm}

\section{Uniqueness theorem}

\begin{thm}{Theorem}\label{thm:alexandrov-uni'}
Any two convex polyhedrons in $\EE^3$ with isometric surfaces are congruent.

Moreover, any isometry between the surfaces of convex polyhedrons can be extended to an isometry of the whole $\EE^3$.
\end{thm}

If one assumes that the isometry between the surfaces is face-to-face,
then we get an equivalent reformulation of Cauchy's theorem.
Cauchy's argument, with a small addition, proves \ref{thm:alexandrov-uni'}.

First, let us remind Cauchy's proof, assuming the reader knows it.
If not, then read it in one of the classical texts \cite{aigner-zigler,dolbilin,tabacnikov-fuks}.

\parit{Sketch of Cauchy's proof.}
Suppose $K$ and $K'$ are convex polyhedrons in $\EE^3$;
denote their surfaces
by $\spc{P}$ and $\spc{P}'$.
Suppose there is an isometry $\iota\:\spc{P}\to \spc{P}'$ that sends each face of $K$ to a face of $K'$.

Let us mark an edge of $K$ with ``$+$'' (or ``$-$'') if the dihedral angle at this edge in $K$ is smaller (respectively, bigger) than the corresponding angle in $K'$.
Further, we consider the  graph $\Gamma$ that is formed by all marked edges.
If $\Gamma$ is empty, then Cauchy's theorem follows; assume the contrary.

The graph $\Gamma$ is embedded into $\spc{P}$, which is homeomorphic to the sphere.
In particular, the edges coming from one vertex have a natural cyclic order.
Given a vertex $v$ of $\Gamma$, we can count the \textit{number of sign changes} around $v$;
that is, the number of consequent pairs of edges with different signs.

We need to show two statements:

\begin{thm}{Local lemma}
At any vertex of $\Gamma$, the number of sign changes is at least $4$.
\end{thm}

\begin{thm}{Global lemma}
No (nonempty) planar graph meets the condition of the local lemma.
\end{thm}

Once the lemmas are proved, Cauchy's theorem follows.
\qeds

Once more, the argument above is  written only to make sure we are on the same page;
it will not work without reading the actual proof.

\parit{Alexandrov's addition.}
We need to remove the assumption that the isometry $\iota\:\spc{P}\z\to \spc{P}'$ is face-to-face.
Mark in $\spc{P}$ all the edges of $K$ as we did above.
In addition, if an edge in $K'$ does not correspond to an edge of $K$, then mark its inverse image in $K$   with ``$-$''; these lines on $K$ will be referred to as \index{fake edges and vertices}\emph{fake edges}.

The marked lines divide $\spc{P}$ into convex polygons, and the restriction of $\iota$ to each polygon is a rigid motion.
These polygons should be used instead of faces in the Cauchy's argument.

A vertex of the obtained graph can be a vertex of $K$, or it can be a {}\emph{fake vertex};
that is, it might be an intersection of an edge and a fake edge.

\begin{figure}[ht!]
\vskip-0mm
\centering
\includegraphics{mppics/pic-50}
\vskip-0mm
\end{figure}

For a usual vertex, the local lemma can be proved the same way.
For a fake vertex $v$, it is easy to see that both parts of the edge coming thru $v$ are marked with minus
while both of the fake edges at $v$ are marked with plus.
Therefore, we still have at least four sign changes at $v$.
The remaining argument works as before.
\qeds

Let us also state the following result of Alexey Pogorelov \cite[chapter III]{pogorelov};
an alternative proof was found by Yurii Volkov \cite{volkov1968}.

\begin{thm}{Theorem}
Any two convex bodies in $\EE^3$ with isometric surfaces are congruent.

Moreover, any isometry between surfaces of convex bodies can be extended to an isometry of the whole $\EE^3$.
\end{thm}

At first glance, this theorem might look like a small improvement of Alexandrov's uniqueness,
but this improvement is huge.
The proof is quite hard.
Let us just mention that it would follow if any two polyhedra $K$ and $K'$  with close surfaces in the sense of Gromov--Hausdorff would be almost congruent;
that is, there is a motion $\mu$ of $\EE^3$ such that the Hausdorff distance from $K$ to $\mu(K')$ is small.


\section{Existence theorem}

By \ref{prop:poly-CBB}, \ref{clm:total-angle}, and \ref{ex:surf-S2}, the surface of a convex polyhedron is an $\Alex0$ and homeomorphic to the sphere.
Alexandrov's theorem states that the converse holds if one includes in the consideration \textit{twice covered polygons}.
In other words, we have to consider a plane polygon as a degenerate polyhedron;
in this case, its surface is defined as the doubling of the polygon across its boundary.

From now on, we assume that a polyhedron can degenerate to a plane polygon.

\begin{thm}{Theorem}\label{thm:alexandrov-first}
A polyhedral metric on the two-sphere is isometric to the surface of a convex polyhedron (possibly degenerate) if and only if it has nonnegative curvature.

\end{thm}

Applying the approximation theorem (\ref{thm:approximation}) and \ref{ex:surface-covergence}, we get the following statement.
Here we again assume that a convex body can degenerate to a convex plane figure,
and, in this case, its surface is defined as the doubling of the figure across its boundary.

\begin{thm}{Corollary}\label{cor:Alex0-convex}
A metric on the two-sphere is $\Alex0$ if and only if it is isometric to the surface of a convex body (possibly degenerate).

\end{thm}

The proof of the existence theorem will be discussed in the following two sections.
It is instructive to solve the following exercise before going further.

\begin{thm}{Exercise}\label{ex:alexandrov=<4}
Let $\spc{P}$ be the 2-sphere equipped with a polyhedral metric with nonnegative curvature.

\begin{subthm}{ex:alexandrov=<4:>=3}
Prove that $\spc{P}$ has at least 3 essential vertices.
\end{subthm}

\begin{subthm}{ex:alexandrov=<4:=3}
If $\spc{P}$ has exactly 3 essential vertices $u$, $v$, and $w$, then it is isometric to the doubling of the solid model triangle $\modtrig(uvw)$.
\end{subthm}

\begin{subthm}{ex:alexandrov=<4:4}
If $\spc{P}$ has exactly 4 essential vertices, then it is isometric to the surface of a tetrahedron (possibly degenerate to a quadrangle).
\end{subthm}

\end{thm}

\section{Reformulation}

In this section, we introduce several notions and use them to reformulate the existence theorem (\ref{thm:reformulation}).

\paragraph{Space of polyhedrons.}
Let us denote by $\bm{K}$ the space of all convex polyhedrons in the Euclidean space,
including polyhedrons that degenerate to a plane polygon.
Polyhedrons in $\bm{K}$ will be considered up to a motion of the space; we will not distinguish between a convex polyhedron and its congruence class.

The space $\bm{K}$ will be considered with the topology induced by the {}\emph{Hausdorff metric up to a motion};
that is, the distance between (equivalence classes of) polyhedrons $K$ and $L$ is defined by
\[\dist{K}{L}{}\df \inf_\mu \{\dist{K}{\mu(L)}{\Haus}\},\]
where $\mu$ runs among all motions of $\EE^3$.

We say that a polyhedron $K$ in $\bm{K}$ has \emph{no symmetries} if  $K\z\ne \mu(K)$ for any nontrivial motion $\mu$ of $\EE^3$.
The set of all polyhedrons without symmetry in $\bm{K}$ will be denoted by $\bm{K}^\circ$.
Observe that $\bm{K}^\circ$ is an open set in $\bm{K}$.

Further, denote by $\bm{K}_n$ the polyhedrons in $\bm{K}$ with exactly $n$ vertices, and let $\bm{K}_n^\circ=\bm{K}_n\cap \bm{K}^\circ$.
Since any polyhedron has at least 3 vertices, the space $\bm{K}$ admits a subdivision into a countable number of subsets $\bm{K}_3,\bm{K}_4,\dots$

\paragraph{Space of surfaces.}
The space of polyhedral surfaces with nonnegative curvature that are homeomorphic to the 2-sphere will be denoted by $\bm{P}$.
The surfaces in $\bm{P}$ will be considered up to an isometry, and the whole space $\bm{P}$ will be equipped with the natural topology induced by the Gromov--Hausdorff metric.

We say that a surface $\spc{P}$ in $\bm{P}$ has \emph{no symmetries} if there is no nontrivial isometry
$\mu\:\spc{P}\to \spc{P}$.
The set of all surfaces without symmetry in $\bm{P}$ will be denoted by $\bm{P}^\circ$.
Observe that $\bm{P}^\circ$ is an open set in $\bm{P}$.

The subset of $\bm{P}$ of all surfaces with exactly $n$ essential vertices will be denoted by $\bm{P}_n$; let $\bm{P}_n^\circ=\bm{P}_n\cap \bm{P}^\circ$.
By \ref{ex:alexandrov=<4:>=3}, any surface in $\bm{P}$ has at least 3 essential vertices.
Therefore $\bm{P}$ is subdivided into countably many subsets
 $\bm{P}_3,\bm{P}_4,\dots$

\paragraph{From a polyhedron to its surface.}
Recall that the surface of a convex polyhedron is a sphere with nonnegative curvature.
Therefore, passing from a polyhedron to its surface defines a map
\[\iota\:\bm{K}\to \bm{P}.\]

Note that the existence theorem (\ref{thm:alexandrov-first}) follows from the next statement.

\begin{thm}{Theorem}\label{thm:reformulation}
For any integer $n\ge 3$,
the map $\iota$ is a bijection from $\bm{K}_n$ to~$\bm{P}_n$.
\end{thm}

\section{About the proof of existence}

By \ref{ex:surface-covergence}, the map $\iota\:\bm{K}\to\bm{P}$ is continuous.
Combining \ref{clm:total-angle} with the uniqueness theorem (\ref{thm:alexandrov-uni'}), we get that $\iota(\bm{K}_n)\subset \bm{P}_n$ and the map $\iota\:\bm{K}_n\to\bm{P}_n$ is injective.
It remains to prove the following.

\begin{thm}{Claim}\label{clm:surjective}
For any $n\ge 3$, the map $\iota\:\bm{K}_n\to\bm{P}_n$ is surjective.
\end{thm}

The proof is based on the construction of a one-parameter family of polyhedrons that starts at an arbitrary polyhedron
and ends at a polyhedron with its surface isometric to the given surface $\spc{P}$.
This type of argument is called the \index{continuity method}\emph{continuity method}; it is often used in the theory of differential equations.

\medskip

Now let us get into details.
First, observe that the second part of the uniqueness theorem (\ref{thm:alexandrov-uni'}) implies that $\iota(\bm{K}_n^\circ)\subset \bm{P}_n^\circ$.

\begin{thm}{Lemma}\label{lem:connected}
For any integer $n\ge 4$, the space $\bm{P}_n^\circ$ is connected and dense in $\bm{P}_n$.
\end{thm}

Note that $\bm{P}_3^\circ=\emptyset$;
indeed the surface of a triangle admits a reflection symmetry.
The case $n=4$ can be deduced from \ref{ex:alexandrov=<4:4}; thus, we can assume that $n\ge 5$.

The second statement is proved by a general-position-type argument.

The proof of the first statement is not complicated, but it requires ingenuity;
it can be done by the direct construction of a one-parameter family of surfaces in $\bm{P}_n^\circ$ that connects two given surfaces.
Such a family can be obtained as a sequence of the following deformations (direct or reversed).

Start with a surface $\spc{P}$ from $\bm{P}_n^\circ$.
Suppose $v$ and $w$ are essential vertices in $\spc{P}$.
Let us cut $\spc{P}$ along a shortest path from $v$ to~$w$.
This way we obtain a sphere with a hole.
The hole can be patched by a disc so that the obtained surface remains in $\bm{P}_n$.
In particular, the obtained surface has exactly $n$ essential vertices;
note that after the patching, the vertices $v$ and $w$ may become inessential.
(There is a three-parameter family of such patches, so we have something to choose from.)
Choosing a one-parameter family of such patches, we can get a deformation of~$\spc{P}$.

Again, applying a general-position-type argument to the above construction, we get a path in $\bm{P}_n^\circ$, assuming that the starting and ending surfaces are in $\bm{P}_n^\circ$.

\begin{thm}{Lemma}\label{lem:open}
The map $\iota\:\bm{K}_n^\circ\to\bm{P}_n^\circ$ is open,
that is, it maps any open set in $\bm{K}_n^\circ$ to an open set in $\bm{P}_n^\circ$.

In particular, for any $n\ge 3$, the image $\iota(\bm{K}_n^\circ)$ is open in~$\bm{P}_n^\circ$.
\end{thm}

This statement follows from the so-called \index{invariance of domain}\emph{invariance of domain theorem},
which states that a \textit{continuous injective map between manifolds of the same dimension is open}.

Recall that $\iota$ defines a continuous and injective $\bm{K}_n^\circ\to\bm{P}_n^\circ$.
It remains to check that both spaces $\bm{K}_n^\circ$ and $\bm{P}_n^\circ$ are $(3\cdot n-6)$-dimensional manifolds.

Choose a polyhedron $K$ in $\bm{K}_n$.
It is uniquely determined by the $3\cdot n$ coordinates of its $n$ vertices.
We can assume that the first vertex is at the origin,
the second has a positive $x$-coordinate
and the remaining two coordinates vanish,
and the third has a vanishing $z$-coordinate and a positive $y$-coordinate.
Therefore, all polyhedrons in $\bm{K}_n$ that lie sufficiently close to $K$ can be described by $3\cdot n-6$ parameters.
If $K$ has no symmetries, then this description is one-to-one;
in this case, a neighborhood of $K$ in $\bm{K}_n$ admits a parametrization by an open set in $\RR^{3\cdot n-6}$.

The case of surfaces is analogous.
We need to construct a subdivision of the sphere into plane triangles using only essential vertices.
By Euler's formula, there are exactly $3\cdot n-6$ edges in this subdivision.
The lengths of the edges completely describe the surface $\spc{P}$ and any surface near by.
If the surface has no symmetries, then this description is one-to-one, and a neighborhood of $\spc{P}$ in $\bm{P}_n$ admits a parametrization by an open set in  $\RR^{3\cdot n-6}$.

\begin{thm}{Lemma}\label{lem:closed}
The map $\iota\:\bm{K}_n\to\bm{P}_n$ is closed;
that is, the image of a closed set in $\bm{K}_n$ is closed in $\bm{P}_n$.

In particular, for any $n\ge 3$, the set $\iota(\bm{K}_n)$ is closed in~$\bm{P}_n$.
\end{thm}

Choose a sequence of polyhedrons $K_1,K_2,\ldots$ in $\bm{K}_n$.
Assume that the sequence $\spc{P}_i=\iota(K_n)$ converges in $\bm{P}_n$ as $i\to \infty$;
denote its limit by $\spc{P}_\infty$.
We need to construct a polyhedron $K_\infty\in \bm{K}_n$ such that $\iota(K_\infty)=\spc{P}_\infty$;
let us do it.

Passing to a subsequence, we can assume that $K_i$ converges in $\bm{K}$;
denote the limit polyhedron by $K_\infty$.
Since $\iota$ is continuous, $\iota(K_i)$ converges to $\iota(K_\infty)$ in~$\bm{P}$; so, $\iota(K_\infty)=\spc{P}_\infty$.
Recall that $\iota(\bm{K}_m)\subset\bm{P}_m$ for each $m$; therefore, $K_\infty\in \bm{K}_n$.


\parit{Proof of \ref{clm:surjective}.}
The case $n\le 4$ is already solved in \ref{ex:alexandrov=<4}; so we assume that $n\ge 5$.
By \ref{lem:closed} and \ref{lem:open},
$\iota(\bm{K}_n^\circ)$ is a non-empty closed and open set in $\bm{P}_n^\circ$, and $\bm{P}_n^\circ$ is connected.
Therefore, $\iota(\bm{K}_n^\circ)=\bm{P}_n^\circ$.

By \ref{lem:closed}, $\iota(\bm{K}_n)$ is closed in $\bm{P}_n$.
By \ref{lem:connected}, $\bm{P}_n^\circ$ is dense in $\bm{P}_n$.
Since $\iota(\bm{K}_n^\circ)=\bm{P}_n^\circ$, we have $\bm{P}_n^\circ\subset \iota(\bm{K}_n)$;
therefore, $\iota(\bm{K}_n)=\bm{P}_n$;
that is, $\iota\:\bm{K}_n\z\to\bm{P}_n$ is surjective.
\qeds

\section{Ambient space}

On one hand the Alexandrov surface theory is simpler since it has extra tools,
On the other hand, this tool comes with extra structure, which makes the theory more complicated.
The following result of Joseph Liberman \cite{liberman} gives an example.

\begin{thm}{Theorem}
Any geodesic in the surface of a convex body is one-sided differentiable as a curve in $\EE^3$.
\end{thm}

\parit{Proof.}
Let $\gamma$ be a geodesic on the surface of a convex body $K$.
Choose $p\in K$.
By \ref{ex:liberman}, the function $f_p\:t\mapsto \distfun_p\circ\gamma(t)$ is semiconcave for any $p\in K$.
In particular, one-sided derivatives $f_p^+(t)$ are defined for every $t$.

Given $x=\gamma(t)$, choose three points $p_1,p_2,p_3\in K$ in general position;
that is, the four points $x,p_1,p_2,p_3$ do not lie in one plane.
Observe that the distance functions $\distfun_{p_i}$ give smooth coordinates in a neighborhood of $x$.
From above the functions $f_{p_i}$ have one-sided derivatives at $t$.
Since the coordinates are smooth, we get that $\gamma^+(t)$ is defined as well.
\qeds




\begin{thm}{Exercise}\label{ex:convex}
Suppose a plane $\Pi$ cuts from the surface of a convex body $K$ a disc $\Delta$, and the reflection of $\Delta$ across $\Pi$ lies in $K$.
Show that $\Delta$ is a convex subset of the surface;
that is, if a geodesic has endpoints in $\Delta$, then it completely lies in $\Delta$.
\end{thm}

The following exercise gives a more exact version of comparison for convex surfaces;
it is due to Anatolii Milka \cite[Theorem 2]{milka1982}.

\begin{thm}{Very advanced exercise}\label{ex:milka}
Let $\spc{P}$ be the surface of a nondegenerate convex body $K\subset\EE^3$,
and let $\gamma_1$ and $\gamma_2$ be geodesic paths in $\spc{P}$ that start at one point $p\z=\gamma_1(0)\z=\gamma_2(0)$.
Suppose $x_i=\gamma_i(1)$, and $y_i\z=p+\gamma_i^+(0)$.
Show that
\[\dist{x_1}{x_2}{\spc{P}}\le \dist{y_1}{y_2}{W},\]
where $W$ is the complement to the interior of $K$.

\end{thm}



\section{Remarks}


The statement of Cauchy's theorem was conjectured by Adrien-Marie Legendre at the end of the 18$^\text{th}$ century;
a formulation was given in the first edition of his geometry textbook \cite{legendre}.
It was motivated by a vague definition in Euclid's Elements, which could be interpreted as
\textit{polyhedrons are equal if the same holds for their faces}.

The local lemma was already known to Legendre.
Legendre discussed this question with his colleague Joseph-Louis Lagrange, who suggested this problem to Augustin-Louis Cauchy in 1813; soon he solved it \cite{cauchy}.

The key observation that the face-to-face condition can be removed was made by
Alexandr Alexandrov in 1941; in the same paper he proved the uniqueness theorem \cite{alexandrov-1941}.
A quite different proof was found by Yurii Volkov in his thesis \cite{volkov}; it uses a deformation of three-dimensional polyhedral space.
(Be aware that the proof of this theorem given in the book by Igor Pak contains an essential mistake \cite{petrunin-2023}.)

In Cauchy's proof \cite{cauchy}, it was deducted from an analog of the following lemma.
Cauchy made a small mistake in its proof that was fixed in a century \cite{sabitov}.
Several proofs of the arm lemma can be found in the letters between Isaac Schoenberg and Stanisław Zaremba \cite{schoenberg-zaremba}.

\begin{thm}{Arm lemma}\label{lem:arm}
Assume that $A=[a_0 a_1\dots a_n]$ is a convex polygon in $\EE^2$
and $A'=[a'_0 a'_1\dots a'_n]$ is a polygonal line in $\EE^3$
such that
$|a_i-a_{i+1}|=|a'_i-a'_{i+1}|$ for any $i\in\{0,\dots,n-1\}$
and
$\measuredangle a_i\le \measuredangle a'_i$
for each $i\in\{1,\dots,n-1\}$.
Then
$$|a_0-a_n|\le |a'_0-a'_n|$$
and equality holds if and only if $A$ is congruent to $A'$.
\end{thm}

The following variation of the arm lemma makes sense for nonconvex spherical polygons.
It is due to Viktor Zalgaller \cite{zalgaller}.
It can be used instead of the standard arm lemma.

\begin{thm}{Another arm lemma}
Let $A=[a_1\dots a_n]$ and $A'\z=[a'_1\z\dots a'_n]$ be two spherical $n$-gons (not necessarily convex).
Assume that $A$ lies in a half-sphere,
the corresponding sides of $A$ and $A'$ are equal,
and each angle of $A$ is at least the corresponding angle in $A'$.
Then $A$ is congruent to~$A'$.
\end{thm}

Another close relative of the arm lemma is Reshetnyak's majorization theorem \cite{reshetnyak}.

Alexandrov gave two proofs of the global lemma \cite[2.1.2 and 2.1.3]{alexandrov}.
The first is combinatorial, and the second is more visual.
The argument in the second proof was reused by Anton Klyachko \cite{klyachko} in his \index{car-crash lemma}\emph{car-crash lemma}.

Proposition \ref{prop:conv-surf-CBB(0)} generalizes to the boundaries of convex bodies  in $\EE^m$ for any $m\ge 2$.
It could be considered as a partial case of the conjecture about the boundary of Alexandrov space; see \ref{conj:bry}.
Another partial case, for Riemannian manifolds with boundary, is proved by the authors and Stephanie Alexander \cite{alexander-kapovitch-petrunin-2008}.


\begin{wrapfigure}{r}{30mm}
\vskip-3mm
\centering
\includegraphics{mppics/pic-15}
\vskip-0mm
\end{wrapfigure}

According to the uniqueness theorem, a convex polyhedron is completely defined by the intrinsic metric of its surface.
In particular, knowing the metric, we could find the position of the edges.
However, in practice, it is not easy to do.
For example, the surface glued from a rectangle, as shown in the picture, defines a tetrahedron.
Some of the glued lines appear inside the facets of the tetrahedron, and some edges (dashed lines) do not follow the sides of the rectangle.


%\chapter[Alexandrov's embedding theorem]{Alexandrov's embedding theorem\\ \textsc{\normalsize by Nina Lebedeva and Anton Petrunin}}\label{chap:embedding}

\section{Introduction}

Intrinsic distance between two points on the surface of a convex polyhedron is defined as the length of a shortest curve on the surface between these points.

Recall that the sum of angles at the tip of a convex polyhedral angle is less than $2\cdot\pi$;
this statement can be found in a school textbook \cite[§~48]{kiselev-stereo-en}.

It is easy to see that the surface of a convex polyhedron is homeomorphic to the sphere.
Therefore the statements above imply that the surface of a convex polyhedron equipped with its intrinsic metric is an example of a \textit{polyhedral metric on the sphere with the sum of angles around each vertex at most $2\cdot\pi$};
a metric is called \emph{polyhedral} if the sphere admits a triangulation such that every triangle is congruent to a plane triangle.

Alexandrov's theorem states that the converse holds if one includes in the consideration \textit{twice covered polygons}.
In other words, we assume that a polyhedron can degenerate to a plane polygon;
in this case, its surface is defined as two copies of the polygon glued along their boundary.

Further, we assume that a polyhedron can degenerate to a plane polygon.

\pagebreak

\begin{thm}{Alexandrov's theorem}
\begin{enumerate}[I.]
\item\label{thm:exist}
A polyhedral metric on the sphere is isometric to the surface of a convex polyhedron if and only if the sum of angles around each of its vertex is not greater than $2\cdot\pi$.

\item\label{thm:unique} 
Moreover, a convex polyhedron is defined up to congruence by the intrinsic metric on its surface.
\end{enumerate}

\end{thm}

A. D. Alexandrov has many remarkable theorems, but in our opinion, this theorem is the most remarkable.
At the same time, its proof is elementary;
it could be explained to anyone familiar with basic topology.

This theorem has many applications.
In particular, it is used in the proof of its generalization \cite{alexandrov-1948} that gives a complete description of intrinsic metrics on the sphere that are isometric to convex surfaces in the Euclidean space.
The latter statement is fundamental in a branch of modern mathematics --- the so-called \emph{Alexandrov geometry}.

The first part is central; it is called the \emph{existence theorem}.
The second part is called the \emph{uniqueness theorem}; it is a slight variation of Cauchy's theorem about polyhedrons.
(There is another uniqueness theorem of Alexandrov that generalizes Minkowski's theorem about  polyhedrons.)

According to the theorem, a convex polyhedron is completely defined by the intrinsic metric of its surface.
In particular, knowing the metric we could find the position of the edges.
However, in practice, it is not easy to do.
For example, the surface glued from a rectangle as shown on the diagram defines a tetrahedron.
Some of the glued lines appear inside facets of the tetrahedron and some edges (dashed lines) do not follow the sides of the rectangle.

{

\begin{wrapfigure}{r}{30mm}
\vskip-3mm
\centering
\includegraphics{mppics/pic-15}
\vskip-0mm
\end{wrapfigure}

The theorem was proved by A. D. Alexandrov in 1941 \cite{alexandrov-1941};
we will present a sketch of his proof.
A complete proof is nicely written by A. D. Alexandrov in his book~\cite{alexandrov}.
Yet another proof was found by Yu.~A.~Volkov in his thesis \cite{volkov};
it uses a deformation of three-dimensional polyhedral space.

}

\section{Space of polyhedrons and metrics}

\paragraph{Space of polyhedrons.}
Let us denote by $\Phi$ the space of all convex polyhedrons in the Euclidean space,
including polyhedrons that degenerate to a plane polygon.
Polyhedra in $\Phi$ will be considered up to a motion of the space, 
and the whole space $\Phi$ will be considered with the natural topology (an intuitive meaning of closeness of two polyhedrons should be sufficient).  

Further, denote by $\Phi_n$ the polyhedrons in $\Phi$ with exactly $n$ vertices.
Since any polyhedron has at least 3 vertices, the space $\Phi$ admits a subdivision into a countable number of subsets $\Phi_3,\Phi_4,\dots$

\paragraph{Space of polyhedral metrics.}
The space of polyhedral metrics on the sphere with the sum of angles around each point at most $2\cdot\pi$ will be denoted by $\Psi$.
The metrics in $\Psi$ will be considered up to an isometry, and the whole space $\Psi$ will be equipped with the natural topology (again, an intuitive meaning of closeness of two metrics is sufficient).

A point on the sphere with the sum of angles strictly less than $2\cdot\pi$ will be called an \emph{essential vertex}.
The subset of $\Psi$ of all metrics with exactly $n$ essential vertices will be denoted by $\Psi_n$.
It is easy to see that any metric in $\Psi$ has at least 3 essential vertices.
Therefore $\Psi$ is subdivided into countably many subsets
 $\Psi_3,\Psi_4,\dots$

\paragraph{From a polyhedron to its surface.}

Recall that the surface of a convex polyhedron is a sphere with a polyhedral metric such that the sum of angles around each point is at most $2\cdot\pi$.
Therefore passing from a polyhedron to its surface defines a map
\[\iota\:\Phi\to \Psi.\]

Note that the number of vertices of a polyhedron is equal to the number of essential vertices of its surface.
In other words, $\iota(\Phi_n)\subset \Psi_n$ for any $n\ge 3$.

\section{About the proof}

Using the notation introduced in the previous section, we can give the following more exact formulation of Alexandrov's theorem: 

\begin{thm}{Reformulation}
For any integer $n\ge 3$,
the map $\iota$ is a bijection from $\Phi_n$ to~$\Psi_n$.
\end{thm}

We sketch the original proof of A. D. Alexandrov.
It is based on the  construction of a one-parameter family of polyhedrons that starts at arbitrary polyhedron
and ends at a polyhedron with its surface isometric to the given one.
This type of argument is called the \emph{continuity method}; it is often used in the theory of differential equations.

\medskip

The two parts of the first formulation will be proved separately.

\parit{Part \ref{thm:unique}.} Let us show that the map $\iota\:\Phi_n\to\Psi_n$ is injective;
in other words, a convex polyhedron is defined by the intrinsic metric on its surface up to a motion of the space.

The last statement is analogous to the Cauchy theorem about polyhedrons,
and the proof goes along the same lines. 

The Cauchy theorem states that facets of a polyhedron together with the gluing rule completely describe a convex polyhedron;
its proof is given in many classical popular texts \cite{aigner-zigler,dolbilin,tabacnikov-fuks}.

\medskip

\parit{Part \ref{thm:exist}.}
Let us prove that $\iota\:\Phi_n\to\Psi_n$ is surjective.
This part of the proof is subdivided into the following lemmas:

\begin{thm}{Lemma}
For any integer $n\ge 3$, the space $\Psi_n$ is connected.
\end{thm}

The proof of this lemma is not complicated, but it requires ingenuity;
it can be done by the direct construction of a one-parameter family of metrics in $\Psi_n$ that connects two given metrics.
Such a family can be obtained by а sequential application of the following construction and its inverse.

Let $M$ be a sphere with metric from $\Psi_n$.
Suppose $v$ and $w$ are essential vertices in $M$.
Let us cut $M$ along a shortest line from $v$ to~$w$.
Note that the shortest line cannot pass thru an essential vertex of $M$.
Further, note that there is a three-parameter family of patches that can be used to patch the cut so that the obtained metric remains in $\Psi_n$;
in particular, the obtained metric has exactly $n$ essential vertices (after the patching, the vertices $v$ and $w$ may become inessential).


\begin{thm}{Lemma}
The map $\iota\:\Phi_n\to\Psi_n$ is open, 
that is, it maps any open set in $\Phi_n$ to an open set in $\Psi_n$.

In particular, for any $n\ge 3$, the image $\iota(\Phi_n)$ is open in~$\Psi_n$.
\end{thm}

This statement is very close to the so-called \emph{invariance of domain theorem};
the latter states that a continuous injective map between manifolds of the same dimension is open.

According to part \ref{thm:unique}, $\iota$ is injective.
The proof of the invariance of domain theorem can be adapted to our case since both spaces $\Phi_n$ and $\Psi_n$ are $(3\cdot n-6)$-dimensional and both look like manifolds, altho, formally speaking, they are \emph{not} manifolds.
In a more technical language, $\Phi_n$ and $\Psi_n$ have the natural structure of $(3\cdot n-6)$-dimensional \emph{orbifolds},
and the map $\iota$ respects the \emph{orbifold structure}.

We will only show that both spaces $\Phi_n$ and $\Psi_n$ are $(3\cdot n-6)$-dimensional.

Choose a polyhedron $P$ in $\Phi_n$.
Note that $P$ is uniquely determined by the $3\cdot n$ coordinates of its $n$ vertices.
We can assume that the first vertex is the origin, the second has two vanishing coordinates and the third has one vanishing coordinate; therefore, all polyhedrons in $\Phi_n$ that lie sufficiently close to $P$ can be described by $3\cdot n-6$ parameters.
If $P$ has no symmetries then this description can be made one-to-one;
in this case, a neighborhood of $P$ in $\Phi_n$ is a $(3\cdot n-6)$-dimensional manifold.
If $P$ has a nontrivial symmetry group, then this description is not one-to-one but it does not have an impact on the dimension of $\Phi_n$.

The case of polyhedral metrics is analogous.
We need to construct a subdivision of the sphere into plane triangles using only essential vertices.
By Euler's formula, there are exactly $3\cdot n-6$ edges in this subdivision.
Note that the lengths of edges completely describe the metric, and slight changes of these lengths produce a metric with the same property.

\begin{thm}{Lemma}
The map $\iota\:\Phi_n\to\Psi_n$ is closed;
that is, the image of a closed set in $\Phi_n$ is closed in $\Psi_n$.

In particular, for any $n\ge 3$, the set $\iota(\Phi_n)$ is closed in~$\Psi_n$.
\end{thm}

Choose a closed set $Z$ in $\Phi_n$.
Denote by $\bar Z$ the closure of $Z$ in $\Phi$; note that $Z=\Phi_n\cap \bar Z$.
Assume $P_1,P_2,\dots\in Z$ is a sequence of polyhedrons that converges to a polyhedron $P_\infty\in\bar Z$.
Note that $\iota(P_n)$ converges to $\iota(P_\infty)$ in $\Psi$.
In particular, $\iota(\bar Z)$ is closed in $\Psi$.

Since $\iota(\Phi_n)\subset \Psi_n$ for any $n\ge 3$, we have  $\iota (Z)=\iota(\bar Z)\cap \Psi_n$;
that is, $\iota (Z)$ is closed in $\Psi_n$. 

\medskip

Summarizing, $\iota(\Phi_n)$ is a nonempty closed and open set in $\Psi_n$, and $\Psi_n$ is connected for any $n\ge 3$.
Therefore, $\iota(\Phi_n)=\Psi_n$; that is, $\iota\:\Phi_n\z\to\Psi_n$ is surjective.
\qeds

\parbf{Acknowledgments.} We want to thank Stephanie Alexander, Yuri Burago, and Jules %Kiyoshi
Tsukahara for help. 
The authors were partially supported by RFBR grant 20-01-00070 and NSF grant DMS-2005279.


\backmatter

%%!TEX root = the-sols.tex

\chapter{Semisolutions}

\parbf{\ref{ex:compact+connceted}.}
Choose a sequence of positive numbers $\varepsilon_n\to 0$ and a finite $\varepsilon_n$-net $N_n$ of $K$ for each $n$.
%???eps-net are not defined!!!
We can assume that $\eps_0>\diam K$, and $N_0$ is a one-point set.
If $\dist{x}{y}{}<\eps_k$ for some $x\in N_{k+1}$ and $y\in N_{k}$, then connect them by a curve of length at most $\eps_k$.

Let $K'$ be the union of all these curves and $K$.
Show that $K'$ is compact and path-connected.

\parit{Source:} This problem is due to Eugene Bilokopytov \cite{bilokopytov}.

\parbf{\ref{ex:compact=>complete}.}
Choose a Cauchy sequence $x_n$ in $(\spc{X},\|*\z-*\|)$; it is sufficient to show that a subsequence of $x_n$ converges.

Observe that the sequence $x_n$ is Cauchy in $(\spc{X},|*-*|)$;
denote its limit by $x_\infty$.

Passing to a subsequence, we can assume that $\|x_n-x_{n+1}\|\z<\tfrac1{2^n}$.
It follows that there is a 1-Lipschitz path $\gamma$ in $(\spc{X},\|*-*\|)$ such that $x_n=\gamma(\tfrac1{2^n})$ for each $n$ and $x_\infty=\gamma(0)$.
Therefore,
\begin{align*}
\|x_\infty-x_n\|&\le \length\gamma|_{[0,\frac1{2^n}]}\le \tfrac1{2^n}.
\end{align*}
In particular, $x_n$ converges to $x_\infty$ in $(\spc{X},\|*\z-*\|)$.

\parit{Source:} \cite[Corollary]{hu-kirk}; see also \cite[Lemma 2.3]{petrunin-stadler}.

\parbf{\ref{ex:compact-length}.}
Given a pair of points $p$ and $q$, choose a sequence of paths $\gamma_n$ from $p$ to $q$ such that
\[\length\gamma_n\to \dist pq{}
\quad\text{as}\quad
n\to\infty;\]
these paths exist since we are in a length space.
Note that we can assume that each $\gamma_n$ is parametrized proportionally to the arc length;
in particular, $\gamma_n$ are equicontinuous.
Show that paths $\gamma_n$ lie in a closed ball, say $\cBall[p,r]$ of some radius $r<\infty$.
Since the space is proper, $\cBall[p,r]$ is compact.
By the Arzelà--Ascoli theorem, we can pass to a converging subsequence of $\gamma_n$.
Show that its limit is a geodesic path from $p$ to $q$.

\parbf{\ref{ex:menger}.}
Choose a sequence $\eps_n>0$ that converges to zero very fast, say such that $\sum_n10^n\cdot \eps_n$ is small.
Follow the argument in the proof of Menger's lemma, taking $\eps_n$-midpoints at the $n^{\text{th}}$ stage.

\parbf{\ref{ex:k-><mono}.}
Let us write the Riemannian metric on $\MM^2(\kappa)$ in polar coordinates $(\theta,r)$;
it has the form 
$(\begin{smallmatrix}
h^2&0
\\
0&1
\end{smallmatrix})$, where $h=h(\kappa,r)\ge 0$.
Calculate $h(\kappa,r)$.
Show that for fixed $r$, the function $r\mapsto h(\kappa,r)$ is nonincreasing in the domain of definition.
Suppose $\kappa<\Kappa$, consider the partially defined map $\MM^2(\kappa)\to\MM^2(\Kappa)$ that sends a point to the point with the same polar coordinates.
Show that this map is short in the domain of definition.
Use it to prove the statement in the exercise.


\parbf{\ref{ex:angkK}.} Show and use that 
$\angk p{x}{y}_{\SSS^2}-\angk p{x}{y}_{\EE^2}=O(\dist[2]{p}{x}{}+\dist[2]{p}{y}{})$
and 
$\angk p{x}{y}_{\EE^2}-\angk p{x}{y}_{\HH^2}=O(\dist[2]{p}{x}{}+\dist[2]{p}{y}{})$.

\parbf{\ref{ex:undefined-angle}.}
Consider a hinge in the plane $\RR^2$ with a metric defined by norm, say by the $\ell^\infty$-norm.

\parbf{\ref{ex:adjacent-angles}.}
Assume $\mangle\hinge pxz+\mangle\hinge pyz<\pi$.
By \ref{claim:angle-3angle-inq}, $\mangle\hinge pxy<\pi$.
Therefore,
$\angk p{\bar x}{\bar y}<\pi$
for some $\bar x\in \left]px\right]$ and $\bar y\in \left]py\right]$.
Hence 
\[\dist p{\bar x}{}+\dist {\bar y}p{}<\dist {\bar x}{\bar y}{}\]
--- a contradiction.

\parbf{\ref{ex:first-var}.}
Denote by $\alpha$ the arc-length parametrization of $[qp]$ from $q$ to $p$.
Choose $\eps>0$.
Observe that 
\[\dist[2]{\gamma(t)}{\alpha(\tfrac1\eps\cdot t)}{}\le t^2\cdot(1-\tfrac2\eps\cdot\cos\phi+\tfrac1{\eps^2})+o(t^2),\]
where $\phi=\mangle\hinge q p x$.
By the triangle  inequality
\[\dist{p}{\gamma(t)}{}\le \dist{\gamma(t)}{\alpha(\tfrac1\eps\cdot t)}{}+\dist{q}{p}{}-\tfrac1\eps\cdot t.\]
Conclude that
\[\dist{p}{\gamma(t)}{}
\le
\dist{q}{p}{}-t\cdot \cos \phi+\delta(\eps)\cdot t+o(t),\]
where $\delta(\eps)\to 0$ as $\eps\to0$.
The statement follows since $\eps>0$ is arbitrary.

\parbf{\ref{ex:generalized-selection}.}
Since the space is proper, it is separable; 
that is, we can choose an countable everywhere dense set $\{x_1,x_2,\dots\}$.

Let $A_1,A_2,\dots$ be a sequence of closed sets.
Applying the diagonal procedure, we can pass to a subsequence such that for each $i$ the sequence $\distfun_{A_n}x_i$ converges as $n\to\infty$;
denote its limit by $f(x_i)$.

Since $\distfun_{A_n}$ is $1$-Lipschitz for any $n$, we have 
\[|f(x_i)-f(x_j)|\le \dist{x_i}{x_j}{}\]
for all $i$ and $j$.
Suppose $f(x_i)<\infty$ for some $i$; note that the same holds for any $i$.
Therefore, the function $f$ can be extended to a continuous function defined on the whole ambient space.
Show that $A_\infty=f^{-1}\{0\}$ is the limit of $A_n$ in the sense of Hausdorff.

If $f(x_i)=\infty$ for some $i$, then the same holds for any $i$.
Show that in this case $A_n\to\emptyset$ in the sense of Hausdorff.

\parbf{\ref{ex:Haus-conv}.}
Apply the definition of Hausdorff distance (\ref{def:hausdorff-convergence}).

\parbf{\ref{ex:geod-closed}.}
Given $x_\infty,y_\infty\in\spc{X}_\infty$, choose $x_n,y_n\in \spc{X}_n$ such that $x_n\to x_\infty$ and $y_n\to y_\infty$.
Let $z_n$ be the midpoint of $[x_ny_n]$.
Since $\spc{X}_\infty$ is proper, we can choose a subsequence of $z_m$ that converges to a point, say $z_\infty\in \spc{X}_\infty$.
Note that $z_\infty$ is a midpoint of $x_\infty$ and $y_\infty$, then apply Menger's lemma (\ref{lem:mid>geod}).

\parbf{\ref{ex:non-contracting-map}.}
Given a pair of points $x_0,y_0\in \spc{K}$, 
consider two sequences $x_0,x_1,\dots$ and $y_0,y_1,\dots$
such that $x_{n+1}=f(x_n)$ and $y_{n+1}\z=f(y_n)$ for each $n$.

Since $\spc{K}$ is compact, 
we can choose an increasing sequence of integers $n_k$
such that both sequences $(x_{n_i})_{i=1}^\infty$ and $(y_{n_i})_{i=1}^\infty$
converge.
In particular, both are Cauchy;
that is,
\[
|x_{n_i}-x_{n_j}|_{\spc{K}}\to 0 
\quad\text{and}\quad
|y_{n_i}-y_{n_j}|_{\spc{K}}\to 0
\]
as $\min\{i,j\}\to\infty$.

Since $f$ is distance-noncontracting, 
\[
|x_0-x_{|n_i-n_j|}|
\le 
|x_{n_i}-x_{n_j}|
\]
for any $i$ and $j$.
Therefore, there is a sequence $m_i\to\infty$ such that
\[
x_{m_i}\to x_0\quad\text{and}\quad y_{m_i}\to y_0
\leqno({*})\]
as $i\to\infty$.

Since $f$ is distance-noncontracting, the sequence $\ell_n=|x_n-y_n|_{\spc{K}}$ is nondecreasing.
By $({*})$,  $\ell_{m_i}\to\ell_0$ as $m_i\to\infty$.
It follows that 
\[\ell_0=\ell_1=\dots\]
In particular, 
\[|x_0-y_0|_{\spc{K}}=\ell_0=\ell_1=|f(x_0)-f(y_0)|_{\spc{K}}\]
for any pair of points $(x_0,y_0)$ in $\spc{K}$.
That is, the map $f$ is distance-preserving; hence $f$ is injective.
From $({*})$, we also get that $f(\spc{K})$ is everywhere dense.
Since $\spc{K}$ is compact $f\:\spc{K}\to \spc{K}$ is surjective --- hence the result.

\parit{Remarks.}
This is a basic lemma in the introduction to Gromov--Hausdorff distance \cite[see 7.3.30 in][]{burago-burago-ivanov}.
The presented proof is not quite standard;
I learned it from Travis Morrison, 
a student in my MASS class at Penn State, Fall 2011.

Note that this exercise implies that \textit{any surjective non-expanding map from a compact metric space to itself is an isometry}.

\parbf{\ref{ex:GH-po}.}
The only-if part is trivial. 
Let us prove the if part.

If $\dist{\spc{X}_n}{\spc{X}_\infty}{\GH}\not\to 0$, then we can pass to a subsequence such that $\dist{\spc{X}_n}{\spc{X}_\infty}{\GH}\ge\eps$ for some $\eps>0$.
Show that we can pass to a subsequence again, so that $\spc{X}_n$ converges in the sense of Gromov--Hausdorff, say to $\spc{Y}$.
Observe that $\spc{Y}\le \spc{X}_\infty$ and $\spc{X}_\infty\le\spc{Y}$.
By \ref{ex:non-contracting-map}, $\spc{Y}\iso \spc{X}_\infty$ --- a contradiction.

\parbf{\ref{ex:compact-GH}.} Show and use that $\dist{\spc{X}_\infty}{\spc{X}_\infty'}{\GH}<\eps$ for any $\eps>0$.



\parbf{\ref{ex:GH-noncompact}.  } \parit{\ref{SHORT.ex:GH-noncompact:proper}}
 Consider the graphs of the following functions with the induced metric from $\RR^2$.
\[
x\mapsto \cos x+\cos \tfrac x\pi
\quad\text{and}\quad
x\mapsto \cos x+\sin \tfrac x\pi.
\]


\parit{\ref{SHORT.ex:GH-noncompact:bounded}}
For every rational number  $q\in[1,2]$ consider an interval of length $q$. Let $\spc{X}$ be obtained by identifying all  initial points of  the intervals to one point and all  end points to another.

Let $\spc{Y}$ be constructed in the same way but skipping the interval of length $1.5$.



\parbf{\ref{ex:Euclid-is-CBB}.}
The 4-point comparison (\ref{def:CBB}) reduces our question to the following.
\textit{Any spherical triangle has perimeter at most $2\cdot\pi$.}
Choose a spherical triangle $[xyz]$.
Let $x'$ be the antipode of $x$; that is $x'=-x$.
The spherical triangle inequality (\ref{claim:angle-3angle-inq} or \ref{ex:angle-triangle}) implies that
\[\dist{x}{z}{\mathbb{S}^2}\le \dist{y}{x'}{\mathbb{S}^2}+\dist{x'}{z}{\mathbb{S}^2}.\]
Observe that 
\[
\dist{x}{y}{\mathbb{S}^2}+\dist{y}{x'}{\mathbb{S}^2}=\pi,
\quad\text{and}\quad
\dist{x}{z}{\mathbb{S}^2}+\dist{z}{x'}{\mathbb{S}^2}=\pi.
\]
Hence
\[\dist{x}{y}{\mathbb{S}^2}+\dist{x}{z}{\mathbb{S}^2}+\dist{y}{z}{\mathbb{S}^2}\le2\cdot \pi.\]

\parbf{\ref{ex:(3+1)-expanding}.} For the only-if part consider the following two cases.

If $\angk p{x_1}{x_2}+\angk p{x_2}{x_3}\ge \pi$, then choose two model triangles $[qy_1y_2]\z=\modtrig(px_1x_2)$ and $[qy_2y_3]=\modtrig(px_2x_y)$ that lie on the opposite sides of $[qy_2]$.
By the comparison, $\dist{y_1}{y_3}{}\ge \dist{x_1}{x_3}{}$.
Therefore the obtained configuration meets all the conditions.

If $\angk p{x_1}{x_2}+\angk p{x_2}{x_3}\ge \pi$, then choose two model triangles $[qy_1y_2]\z=\modtrig(px_1x_2)$
and take $y_3$ on the extension of $[y_1q]$ behind $q$ such that $\dist{q}{y_3}{}=\dist{p}{x_3}{}$.
Then $\mangle \hinge q{y_2}{y_3}\ge \angk p{x_2}{x_3}$, therefore $\dist{y_2}{y_3}{}\ge \dist{x_2}{x_3}{}$.
Further, $\dist{y_2}{y_3}{}=\dist{x_2}{p}{}+\dist{p}{x_3}{} \ge \dist{x_2}{x_3}{}$,
and again, the obtained configuration meets all the conditions.

To prove the if part, choose a configuration $q,y_1,y_2,y_3$ that meets all the conditions and maximize the sum
\[\dist{y_1}{y_2}{}+\dist{y_2}{y_3}{}+\dist{y_3}{y_1}{}.\]
Show that $q$ lies in the solid triangle $y_1y_2y_3$;
in particular 
\[\mangle \hinge q{y_1}{y_2}+\mangle \hinge q{y_2}{y_3}+ \mangle \hinge q{y_3}{y_1}=2\cdot\pi.\]
Moreover, $\dist{q}{y_i}{}=\dist{p}{x_i}{}$ for each $i$.
Applying that increasing the opposite side in a plane triangle increases the corresponding angle, we get 
\[\angk  p{x_1}{x_2}+\angk p{x_2}{x_3}+\angk p{x_3}{x_1}
\le 
2\cdot\pi.
\]

\parbf{\ref{ex:alex-lemma-cat}.}
Consider model triangles $[\tilde p\tilde x\tilde z]=\modtrig(pxz)$ and $[\tilde p\tilde y\tilde z]=\modtrig(pyz)$
that share side $[\tilde p\tilde z]$ and lie on its opposite sides.
Note that 
\begin{align*}
\dist{\tilde x}{\tilde y}{\EE^2}
&\ge \dist{\tilde x}{\tilde y}{\EE^2}+\dist{\tilde x}{\tilde y}{\EE^2}=
\\
&=\dist{x}{z}{\spc{X}}+\dist{z}{y}{\spc{X}}=
\\
&=\dist{x}{y}{\spc{X}},
\end{align*}
where $\spc{X}$ is our metric space.
It remains to apply the monotonicity of angle in a triangle with respect to its opposite side. 


\parbf{\ref{ex:noncreasing}.}
Apply \ref{clm:angle-mono}.

\parbf{\ref{ex:0-angle}.}
Without loss of generality, we can assume that $\dist{p}{x}{}\le \dist{p}{y}{}$.
Choose $\bar x\in [px]$;
let $\bar y\in [px]$ be such that $\dist{p}{\bar x}{}=\dist{p}{\bar y}{}$.
Apply \ref{clm:angle-mono} to show that $\bar x=\bar y$.
Conclude that $[px]\subset [py]$.

\parbf{\ref{ex:pi-angle}.}
Assume that there are two distinct geodesics from $z$ to $x$.
Then we can choose distinct points $p$ and $q$ on these geodesics such that $\dist{z}{p}{}=\dist{z}{q}{}$.
Observe that $\angk zpq>0$.
By the triangle inequality, we get 
\[\dist{x}{p}{}+\dist{p}{y}{}\le \dist{x}{p}{}+\dist{p}{z}{}+\dist{z}{y}{}=\dist{x}{z}{}+\dist{z}{y}{}\]
Observe that $\angk zxy=\pi$.
Therefore $\mangle\hinge zxy=\pi$ for any geodesic $[zx]$.

\parbf{\ref{ex:adjacent-CBB}.}
By \ref{ex:adjacent-angles}, we have
\[\mangle\hinge pxz+\mangle\hinge pyz\ge \pi.\]
Since $z\in \left]xy\right[$ we have 
\[\angk z{\bar x}{\bar y}=\pi\]
for any $\bar x\in \left[xz\right[$ and $\bar y\in \left]zy\right]$.
By comparison, we have that 
\[\angk z{\bar x}{\bar p}+\angk z{\bar p}{\bar y}\le\pi\]
for any $\bar p\in \left]zp\right]$.
Passing to the limit as
$\dist{z}{\bar x}{}\to 0$,
$\dist{z}{\bar y}{}\to 0$, and
$\dist{z}{\bar p}{}\to 0$,
we get the statement.

\parbf{\ref{ex:pxyvw}.} 
Without loss of generality, we can assume that $x$, $v$, $w$, and $y$ appear on 
$[xy]$ in this order.
By \ref{clm:angle-mono},
\[
\angk xyp\ge \angk xwp \ge\angk xvp.
\]
Hence, $\Rightarrow$ follows.

By Alexandrov's lemma,
\begin{align*}
\angk xyp=\angk xvp
\quad&\Longleftrightarrow\quad
\angk yxp=\angk yvp,
\\
\angk xyp=\angk xwp
\quad&\Longleftrightarrow\quad
\angk yxp=\angk ywp.
\end{align*}
Whence, $\Leftarrow$ follows.

\parbf{\ref{ex:angle-lim}.} Suppose $\mangle \hinge {x_\infty}{y_\infty}{z_\infty}>\alpha$.
Then we can choose $\bar y_\infty\in\left]x_\infty y_\infty\right]$
and $\bar z_\infty\in\left]x_\infty z_\infty\right]$ such that 
$\angk{x_\infty}{\bar y_\infty}{\bar z_\infty}>\alpha$.
Now choose $\bar y_n\in\left]x_n y_n\right]$ and $\bar y_n\in\left]x_n z_n\right]$ such that $\bar y_n\to \bar y_\infty$ and $\bar z_n\to \bar z_\infty$.
Observe that 
\[\liminf_{n\to\infty}\mangle \hinge {x_n}{y_n}{z_n}\ge\liminf_{n\to\infty}\angk{x_n}{\bar y_n}{\bar z_n} \ge \alpha,\]
hence the result.

\parbf{\ref{ex:urysohn}.}
The Urysohn space provides an example;
see for example \cite[Lecture 2]{petrunin2023pure}.

\parbf{\ref{ex:normCBB}.}
Choose a triangle $[0vw]$.
Note that $m=\tfrac12(v+w)$ is the midpoint of $[vw]$.

Use comparison, to show that
\[2\cdot |\tfrac12(v+w)|^2+2\cdot |\tfrac12(v-w)|^2\ge |v|^2+|w|^2.\]

Note this inequality implies the opposite one;
it follows if we rewrite it via $x=\tfrac12(v+w)$ and $y=\tfrac12(v-w)$.
Hence we have 
\[2\cdot |\tfrac12(v+w)|^2+2\cdot |\tfrac12(v-w)|^2= |v|^2+|w|^2\]
for any $v,w$.
That is, the norm is quadratic and the statement follows.

\parbf{\ref{ex:alm-min}.}
Suppose such a point does not exist;
that is, for any $p\in \spc{X}$ there is a point $p'$ such that $r(p')\le  (1-\eps)\cdot r(p)$ and $\dist p{p'}{}<\tfrac{1}{\eps}\cdot r(p)$.
Construct a sequence of points $p_0,p_1,\dots$ such that $p_n=p_{n-1}'$ for any~$n$.
Show that this sequence is Cauchy; denote its limit by $p_\infty$.
Arrive at a contradiction by showing that $r(p_\infty)\le0$.

\parbf{\ref{ex:CBB(1)notitCBB(0)}.}
Note that $\spc{X}$ has no defined sphericlal model angles;
therefore it has curvature $\ge 1$.

However, $\spc{X}$ does not have curvature $\ge 0$ since
\[\angk  p{x_1}{x_2}_{\EE^2}=\angk  p{x_2}{x_3}_{\EE^2}=\angk  p{x_1}{x_3}_{\EE^2}=\pi.\]

\parbf{\ref{ex:RisCBB(1)}.}
Suppose $\mangle\hinge mxp\ne 0$ and $\mangle\hinge mxp\ne\pi$, or equivalently $\mangle\hinge mxq\ne0$.

We can assume that $\dist pq{}$ only slightly exceeds $\pi$,
so $\dist pm{}<\pi$ and $\dist qm{}<\pi$.
We can also assume that $\dist xm{}<\pi$.
Use the comparison to show that 
\[\dist px{}+\dist qx{} < \dist pq,\]
and arrive at a contradiction with the triangle inequality.

Extend $[pq]$ to a maximal local geodesic $\gamma$.
It might be a closed or a line segment.
Argue as above to show that any point lies on $\gamma$ and make a conclusion.

\parbf{\ref{ex:perim-k>0}.}
Arguing by contradiction, suppose 
\[\dist{p}{q}{}+\dist{q}{r}{}+\dist{r}{p}{}> 2\cdot\pi\eqlbl{eq:perimeter-of-triange<2pi}\] 
for $p,q,r\in \spc{A}$. 
Rescaling the space slightly, we can assume that $\diam\spc{A}<\pi$,
but the inequality \ref{eq:perimeter-of-triange<2pi} still holds.
By \ref{clm:K>k},
after rescaling $\spc{A}$ is still $\Alex1$.

Take $z_0\in [q r]$ on maximal distance from $p$.
Consider the following model configuration:
two geodesics $[\tilde p\tilde z_0]$, $[\tilde q\tilde r]$ in $\mathbb{S}^2$ such that 
\begin{align*}
\dist{\tilde p}{\tilde z_0}{}&=\dist{p}{z_0}{},
&  
\dist{\tilde q}{\tilde r}{}&=\dist{q}{r}{},
\\ 
\dist{\tilde z_0}{\tilde q}{}&=\dist{z_0}{q}{},
&  
\dist{\tilde z_0}{\tilde r}{}&=\dist{z_0}{q}{},
\end{align*}
and 
\[\mangle\hinge{\tilde z_0}{\tilde q}{\tilde p}
=\mangle\hinge{\tilde z_0}{\tilde r}{\tilde p}
=\tfrac\pi2.\]

Let $\tilde z\in [\tilde q\tilde r]$,
and let $z\in [q r]$ be the corresponding point.
By comparison, $\dist pz{}\le\dist {\tilde p}{\tilde z}{}$ for points $z$ near $z_0$.
Moreover, this inequality holds as far as 
\[\dist{\tilde p}{\tilde z_0}{}+\dist{\tilde z_0}{\tilde z}{}+\dist{\tilde p}{\tilde z}{}<2\cdot\pi.\]
But this inequality holds for all $\tilde z$ since  $\dist{\tilde p}{\tilde z_0}{}<\pi$, $\dist{\tilde z_0}{\tilde q}{}<\pi$, and $\dist{\tilde z_0}{\tilde r}{}<\pi$.
Hence we get $\dist pq{}\le\dist {\tilde p}{\tilde q}{}$ and $\dist pr{}\le\dist {\tilde p}{\tilde r}{}$.
The latter contradicts \ref{eq:perimeter-of-triange<2pi}.

\parbf{\ref{ex:dir-compact}.}
Suppose $\dir p{x_n}\not\to\dir p{x_\infty}$.
Since $\Sigma_p$ is compact, we may pass to a converging subsequence of $\dir p{x_n}$;
denote by $\xi$ its limit.
We may assume that $\mangle (\dir p{x_\infty},\xi)>0$.

Denote by $\gamma_n$ and $\gamma_\infty$ the arc-length parametrization of $[px_n]$ and $[px_\infty]$ from $p$.
Choose a geodesic $\alpha$ that starts from $p$ and goes in a direction sufficiently close to $\xi$.
By comparison we can choose $\alpha$ so that
\[\dist{\alpha(t)}{\gamma_n(t)}{}<\eps\cdot t\]
for all large $n$ and all sufficiently small $t$.
Moreover, we can assume that
\[\dist{\alpha(t)}{\gamma_\infty(t)}{}>a\cdot t\]
for some fixed $a>0$ and all small $t$.
These two inequalities imply 
that 
\[\dist{\gamma_n(t)}{\gamma_\infty(t)}{}>\tfrac a2\cdot t\]
for all small $t$ and all large $n$.
On the other hand, by assumption, $\dist{\gamma_n(t)}{\gamma_\infty(t)}{}\to0$ as $n\to\infty$ --- a contradiction.

\parit{Comments.}
The compactness of $\Sigma_p$ is necessary.
An example can be built using iterated warped product of line segments and applying \cite[Theorem 1.2]{alexander-bishop2004}.
The space $\spc{A}$ can be assumed to be compact.


\parbf{\ref{ex:geodesic-cone}.}
Note that any point of $\Cone \spc{X}$ can be connected to the origin by a geodesic.
Given a nonzero element $v\in\Cone \spc{X}$, denote by $v'$ its projection in $\spc{X}$.

Suppose $\spc{X}$ is $\pi$-geodesic.
Choose two nonzero elements $v,w\in\Cone \spc{X}$; let $\alpha=\mangle(v,w)=\dist{v'}{w'}{\spc{X}}$.
If $\alpha\ge \pi$, then the product of geodesics $[v0]\cup [0w]$ forms a geodesic $[vw]$.
If $\alpha<\pi$, there is a geodesic $\gamma\:[0,\alpha]\to \spc{X}$ from $v'$ to $w'$.
Consider hinge $\hinge {\tilde o}{\tilde v}{\tilde w}$ in the plane 
such that $\mangle\hinge {\tilde o}{\tilde v}{\tilde w}=\alpha$, $\dist{\tilde o}{\tilde v}{}=|v|$, and $\dist{\tilde o}{\tilde w}{}=|w|$.
Let $t\mapsto (\phi(t),r(t))$ be geodesic $[\tilde v\tilde w]$ written in polar coordinates with origin $\tilde o$, so that $\phi(0)=0$.
Show that $t\mapsto r(t)\cdot\gamma\circ\phi(t)$ is a geodesic from $v$ to $w$;
here we identify $\spc{X}$ with the unit sphere in $\Cone \spc{X}$.

To prove the converse, try to reverse the steps in the argument above.

\parbf{\ref{ex:GHto-tangent}.}
Let  $\spc{A}_n=\lambda_n\cdot\spc{A}$.
Note that for any $n$ the space $\Sigma_p\spc{A}$ is identical to $\Sigma_{\iota_n(p)} \spc{A}_n$.
In particular, we can identify isometrically $\T_p\spc{A}$ with $\T_{\iota_n(p)}(\lambda\cdot \spc{A})$.
So for any geodesic $\gamma$ that starts at $p$, the vector $\gamma^+(0)$ corresponds to $\frac{1}{\lambda}\cdot(\iota_n\circ\gamma)^+(0))$.

Consider the logarithm maps $f_n=\log_{\iota_n(p)}\:\spc{A}_n\to T_p\spc{A}$.
We claim that this sequence of maps satisfies the assumptions of Lemma~\ref{lem:almost-isom-pointed};
the condition in \ref{SHORT.lem:almost-isom-pointed-basepoint} is evident.  

Note that it is sufficient to check the conditions in \ref{SHORT.lem:almost-isom-pointed-b} and \ref{SHORT.lem:almost-isom-pointed-c} only for $R=1$. 

Choose $\eps>0$.
By compactness of $\Sigma_p$ we can find a finite $\eps$-net $\xi_1,\dots,\xi_N$ in $\Sigma_p$. Moreover, without loss of generality we can assume that these directions are geodesic;
that is, there exist geodesics $\gamma_1,\ldots, \gamma_N$ starting at $p$ such that $\xi_i=\gamma_i^+(0)$ for each $i$.

Choose $T>0$ such that all $\gamma_i$ are defined on $[0,T]$.
Apply the comparison to show that for any $\lambda_n>\frac{1}{T}$ the image under $f_n$ of the union $\bigcup_N\gamma_i([0,T])$ is an $\eps$-net in $\oBall(0,1)_{T_p}$.
This proves \ref{SHORT.lem:almost-isom-pointed-c}.

By comparison, we have that
\[\dist{\xi_i}{\xi_j}{\Sigma_p}\ge \angk p{\gamma_i(t_i)}{\gamma_j(t_j)}<\eps\]
for all $i\ne j$ and any $t_i,t_j\in (0,T]$.
By the definition of an angle, we can assume that $T$ have been chosen so that in addition 
\[\dist{\xi_i}{\xi_j}{\Sigma_p}\le \angk p{\gamma_i(t)}{\gamma_j(t)}+\eps\]
for all $i\ne j$ and any $t\in (0,T]$.

By construction of the map $f_n$ this implies that 
\[|\dist{x}{x'}{\spc{A}_n}-\dist{f_n(x)}{f_n(x')}{T_p}|<\eps\]
for all $\lambda_n>\frac{1}{T}$ and all points $x,x'$ in $\bigcup_N\gamma_i([0,\frac{1}{\lambda_n}])\subset \oBall(p,1)_{\spc{A}_n}$.
  
Now hinge comparison and the triangle inequality imply that the same  holds for arbitrary points $x,x'$  in  $\oBall(p,1)_{\spc{A}_n}$ with $\eps$ replaced by $3\eps$.
This verifies \ref{SHORT.lem:almost-isom-pointed-b}.

\parbf{\ref{ex:distfun-semiconcave}.} From \ref{comp-kappa}, this inequality follows in the sense of distributions, and hence in any other sense.

\parbf{\ref{ex:df(xi)}.}
Since angles are defined, it follows that 
\[\dist{\gamma_1(t)}{\gamma_2(t)}{}\le \theta\cdot t\]
for all small $t>0$.     
Since $f$ is $L$-Lipschitz, we get 
\[|f(\gamma_1(t))-f(\gamma_2(t))|\le L\cdot \theta\cdot t,\]
hence the statement.

\parbf{\ref{ex:d(distfun)}}; \ref{SHORT.ex:d(distfun):<}
Note that we can assume there is a geodesic in the direction of $v$, and apply \ref{ex:first-var}.

\parit{\ref{SHORT.ex:d(distfun):=}.}
By \ref{SHORT.ex:d(distfun):<}, $\dd_p\distfun_q(v)\le-\max_{\xi\in\Uparrow_p^q}\langle\xi,v\rangle$.
Suppose this inequality is strict for some $v$.
We can assume that $|v|=1$ and there is a geodesic, say $\gamma$ in the direction of $v$.
Let $\dd_p\distfun_q(v)=-\cos\alpha_0$ for some $\alpha\in [0,\pi]$.
Note that any geodesic from $p$ to $q$ makes angle bigger than $\alpha_0$ with $\gamma$.


The function $f=\distfun_q\circ\gamma$ is Lipschitz.
By Rademacher's theorem it is differentiable almost everywhere;
moreover, 
\[f(t)-f(0)=\int_0^t f'(t)\cdot dt.\]
Suppose $f'(t)$ is defined.
Use \ref{SHORT.ex:d(distfun):<} to show that 
$f'(t)=-\cos\alpha(t)$, where $\alpha(t)$ is the angle between $\gamma$ and any geodesic from $\gamma(t)$ to $q$.
Note that we can choose a sequence $t_n\to 0$ such that 
\[\lim_{n\to\infty}\alpha(t_n) \le \alpha_0.\]
Consider a sequence of geodsics $[p\,\gamma(t_n)]$.
Since the space is proper, we can pass to its convergent subsequence.
Its limit is a geodesic from $p$ to $q$, denote it by $[pq]$.

Use \ref{ex:angle-lim} to show that $[pq]$ makes an angle at most $\alpha_0$ with $\gamma$ --- a contradiction.
 
\parbf{\ref{ex:monotonicity}.}
Let $\gamma\:[0,\ell]\to \spc{A}$ be the geodesic $[xy]$ parametrized from $x$ to $y$,
and let $\phi=f\circ\gamma$.
Observe that 
\[\phi'(0)=\dd_xf(\dir xy)\le \<\dir{x}{y},\nabla_{x}f\>.\]
The same way we get $-\phi'(\ell)\le \<\dir{y}{x},\nabla_{y}f\>$.
Since $f$ is $\lambda$-concave, we have
\begin{align*}
f(y)&\le f(x)+\phi'(0)\cdot \ell+\tfrac\lambda2\cdot\ell^2,
\\
f(x)&\le f(y)-\phi'(\ell)\cdot \ell+\tfrac\lambda2\cdot\ell^2.
\end{align*}
Hence the statement follows.

\parbf{\ref{ex:d(distfun):==}.}
If the space is proper, then the statement follows from \ref{SHORT.ex:d(distfun):=} and \ref{ex:pi-angle}.

To do the general case argue by contradiction.
Let $z$ be a point on the extension of $[pq]$ behind $q$;
it exists by the assumption.
Note that we can assume that $|v|=1$ and it is a direction of a geodesic, say $[px]$.

Show that for there is a sequence $x_n\in \left]px\right]$ such that $\dist{p}{x_n}{}\to0$ ad
$\mangle \hinge q{x_n}p>\eps$ for each $n$ and some fixed $\eps>0$.
Observe that $\mangle\hinge q{x_n}z\z<\pi-\eps$; therefore
\[\dist{z}{x_n}{}<\dist{x_n}{q}{}+\dist{q}z{}-\delta\]
for each $n$ and some fixed $\delta>0$.
Pass to the limit as $x_n\to p$ and arrive at a contradiction.

\parbf{\ref{ex:convergence-grad}.}
Note that
$|(\dd_p f)(v)-(\dd_p g)(v)|\le s\cdot|v|$
for any $v\in \T_p$.
From the definition of gradient (\ref{def:grad}) we have:
\begin{align*}
&(\dd_p f)(\nabla_p g)\le\<\nabla_p f,\nabla_p g\>,
&&(\dd_p g)(\nabla_p f)\le\<\nabla_p f,\nabla_p g\>,
\\
&(\dd_p f)(\nabla_p f)=\<\nabla_p f,\nabla_p f\>,
&&(\dd_p g)(\nabla_p g)=\<\nabla_p g,\nabla_p g\>.
\end{align*}
Therefore,
\begin{align*}
&\dist[2]{\nabla_pf}{\nabla_pg}{}
=\<\nabla_p f,\nabla_p f\>+\<\nabla_p g,\nabla_p g\>-2\cdot\<\nabla_p f,\nabla_p g\>
\le
\\
&\le (\dd_p f)(\nabla_p f)+(\dd_p g)(\nabla_p g)-
(\dd_p f)(\nabla_p g)-(\dd_p g)(\nabla_p f)
\le
\\
&\le s\cdot(|\nabla_p f|+|\nabla_p g|).
\end{align*}

\parbf{\ref{ex:semicontinuous-grad}.}
Suppose $|\nabla_xf|> s$.
Then we can choose a geodesic $\gamma$ that starts at $x$ such that 
$(f\circ\gamma)^+(0)>s$.
In particular, there is $\eps>0$ such that
\[f\circ\gamma(t)>(s+\eps)\cdot t+o(t),\]
hence the only-if part follows.

Now suppose $f(y)-f(x)>s\cdot \ell+\lambda\cdot \tfrac{\ell^2}2$,
were $\ell=\dist{x}{y}{}$.
Let $\gamma\:[0,\ell]\to \spc{A}$ be a geodesic from $x$ to $y$.
Since $f\circ\gamma$ is $\lambda$-concave, we have
\[f\circ\gamma(\ell)\le f\circ\gamma(0)+(f\circ\gamma)^+(0)\cdot\ell+\lambda\cdot \tfrac{\ell^2}2.\]
It follows that 
\[\dd_x(\dir xy)=(f\circ\gamma)^+(0)>s,\]
and by \ref{prop:grad-exist}, $|\nabla_x f|>s$.

\parbf{\ref{ex:elf-contracting}.}
Note that $f\circ\alpha$ is a nondecreasing function.
Apply \ref{ex:d(distfun):<} and the definition of gradient to show that
\[
-\dd_{\alpha(t)}\distfun_{\alpha(t_3)}(\nabla_{\alpha(t)}f)
\ge
\langle \nabla_{\alpha(t)},\dir{\alpha(t)}{\alpha(t_3)}\rangle
\ge
\dd_{\alpha(t)}(\dir{\alpha(t)}{\alpha(t_3)})
\ge0
\]
for any $t<t_3$.
Conclude that the function 
$t\mapsto \distfun_{\alpha(t_3)}\circ\alpha(t)$ is noncreasing for $t\le t_3$.

\parbf{\ref{ex:mayer}.}
For any $s>s_0$,
\begin{align*}
(f\circ\hat\alpha)^+(s_0)&=|\nabla_{\hat\alpha(s_0)}f|
\ge
\\
&\ge
(d_{\hat\alpha(s_0)}f)(\dir{\hat\alpha(s_0)}{\hat\alpha(s)})
\ge
\\
&\ge
\frac{f\circ\hat\alpha(s)-f\circ\hat\alpha(s_0)}{\dist{\hat\alpha(s)}{\hat\alpha(s_0)}{}}.
\end{align*} 
Since $s-s_0\ge\dist{\hat\alpha(s)}{\hat\alpha(s_0)}{}$, for any $s>s_0$ we have 
\[(f\circ\hat\alpha)^+(s_0)\ge
\frac{f\circ\hat\alpha(s)-f\circ\hat\alpha(s_0)}{s-s_0}.\]

\parbf{\ref{lem:fg-dist-est}.}
Fix $t$, and let $p=\alpha(t)$ and $q=\beta(t)$.
Apply \ref{eq:fist-var-inq+} to get
\begin{align*}
 \ell^+
&\le -\<\dir{p}{q},\nabla_{p}f\>
-\<\dir{q}{p},\nabla_{q}g\>
\le
\\
&\le -{\left({f(q)}-{f(p)}-\lambda\cdot\tfrac{\ell^2}2\right)}/{\ell}
-{\left({g(p)}-{g(q)}-\lambda\cdot\tfrac{\ell^2}2\right)}/{\ell}\le
\\
&\le \lambda\cdot\ell+\tfrac{2\cdot\eps}{\ell}.
\end{align*}
Integrating this inequality, we get the second statement.

\parbf{\ref{ex:busemann-CBB}.} Apply \ref{ex:distfun-semiconcave}.

\parbf{\ref{ex:bus+bus}.} By the triangle inequality, 
\[\dist{\gamma(-t)}{x}{}+\dist{\gamma(t)}{x}{}-2\cdot t\ge 0\]
for any $t\ge 0$.
Passing to the limit as $t\to\infty$, we get the result.

\parbf{\ref{ex:cone-CBB}.}
Suppose $\Cone\spc{X}$ is $\Alex0$.
Observe that two half-lines in $\Cone\spc{X}$ that start from the origin and go into directions $x$ and $y\in\spc{X}$ form a line if and only if $\dist{x}{y}{\spc{X}}\ge \pi$.
Apply the splitting theorem to show that for any $x\in \spc{X}$ there is at most one point $y$ such that $\dist{x}{y}{\spc{X}}\ge \pi$ and in this case we have equality.
Conclude that $\diam \spc{X}\z\le \pi$.

Now choose a quadruple of points $p,x_1,x_2,x_3\in \spc{X}$;
we will identify $\spc{X}$ with the unit sphere in $\Cone\spc{X}$.
Suppose $\dist{p}{x_i}{}<\tfrac\pi2$ for any $i$.
Consider the following points in the cone: $y_i=\tfrac1{\cos \dist{p}{x_i}{\spc{X}}}\cdot x_i$, and $q=p$.
Show that $\EE^2$-comparison for $q,y_1,y_2,y_3$ in $\Cone\spc{X}$ implies $\SSS^2$-comparsion for $p,x_1,x_2,x_3$ in $\spc{X}$.
Conclude that $\spc{X}$ is locally $\Alex1$. 
Apply the globalization theorem (\ref{thm:globalization+}).

Now assume $\spc{X}$ is $\Alex1$ and $\diam\spc{X}\le \pi$.
By \ref{ex:perim-k>0}, the perimeter of any triangle in $\spc{X}$ is at most $2\cdot\pi$.
We need to check $\EE^2$-comparison for a given quadruple of points $q,y_1,y_2,y_3$ in $\Cone\spc{X}$.
We can assume that none of these points is the origin; otherwise perturb them a bit.

Set $x_i=y_i/|y_i|$ for each $i$ and $p=q/|q|$; we can assume that $p,x_1,x_2,x_3$ are distinct in $\spc{X}$, which is the unit sphere in $\Cone\spc{X}$.

Assume the model triangles $\modtrig(px_1x_2)$, $\modtrig(px_2x_3)$, and $\modtrig(px_3x_1)$ are defined;
that is, perimeters triangles $[px_1x_2]$, $[px_2x_3]$, and $[px_3x_1]$ are strictly less than $2\cdot\pi$. 
Note that $\EE^3\iso\Cone\SSS^2$.
Use this together with the $\SSS^2$-comparison for $p,x_1,x_2,x_3$ in $\spc{X}$ to show that $\EE^2$-comparison holds for $q,y_1,y_2,y_3$ in $\Cone\spc{X}$.

Finally, if some of the model triangles are not defined, consider rescaling of $\spc{X}$ with a coefficient $\lambda$ slightly smaller than 1.
Apply the argument above to show that the comparison holds for the corresponding points in $\Cone(\lambda\cdot\spc{X})$ and pass to the limit as $\lambda\to 1$.

\parit{Comment.}
The last part of the proof is close to the argument in \ref{thm:CBB-closed}.

\parbf{\ref{ex:|antisum|}.}
Observe that
\begin{align*}
\langle u,u\rangle+\langle v,u\rangle+\langle w,u\rangle &\ge 0,
\\
\langle u,v\rangle+\langle v,v\rangle+\langle w,v\rangle &\ge 0,
\\
\langle u,w\rangle+\langle v,w\rangle+\langle w,w\rangle &= 0.
\end{align*}
Add the first two inequalities and subtract the last identity.

\parbf{\ref{prop:two-opp}.}
Apply \ref{prop:opposite} to show that 
$\langle v,v\rangle =\langle v,w\rangle=\langle w,w\rangle$,
and use it.

\parbf{\ref{ex:3<,>=0}.} Show and use that
\[\langle u,x\rangle +\langle v,x\rangle +\langle w,x\rangle \ge 0\]
and
\[\langle u,-x\rangle +\langle v,-x\rangle +\langle w,-x\rangle \ge 0.\]

\parbf{\ref{ex:-u}.} Part $\Rightarrow$ is evident.
To prove part $\Leftarrow$, observe that 
\[\langle u^*,u^*\rangle =-\langle u,u^*\rangle\le \langle u,u\rangle\]
and since $|u|=|u^*|$, we have equality.

\parbf{\ref{ex:grad-dist}.}
Apply \ref{ex:-u}.

\parbf{\ref{ex:tangent=Em}.}
By \ref{ex:diam-compact:proper}, $\spc{A}$ is \emph{separable}; that is, it contains a countable dense set of points.
Apply \ref{cor:euclid-subcone} to this set.

\parbf{\ref{ex:dim=1}.} Argue as in \ref{ex:RisCBB(1)}.

\parbf{\ref{ex:resporka}.} The only-if part is trivial.
Suppose the configuration $p$, $a_0,\z\dots, a_{m}\in \spc{A}$ meets the condition.
By \ref{ex:grad-dist} the directions $\dir q{a_0},\z\dots,\dir q{a_m}\in \Lin_q$ for G-delta dense set of points $q\in \spc{A}$.
If $q$ is sufficiently close to $p$, then $\angk q{a_i}{a_j}>\tfrac\pi2$,
and therefore, $\mangle\hinge q{a_i}{a_j}>\tfrac\pi2$ for $i\ne j$.
Conclude that $\dim\Lin_q\ge m$ in this case.

\parbf{\ref{ex:finite-tan}}; 
\ref{SHORT.ex:finite-tan:tan}. Apply \ref{ex:geodesic-cone}, \ref{prop:Tan-is-CBB(0)}, and \ref{thm:finite-space-of-directions}.

\parit{\ref{SHORT.ex:finite-space-of-directions-dim}.}
Apply \ref{ex:resporka} to show that $\LinDim\T_p=\LinDim\spc{A}$ (argue as in \ref{prop:Tan-is-CBB(0)}).

\parit{\ref{SHORT.ex:finite-tan:sigma}.}
By \ref{thm:finite-space-of-directions} for any two points $\xi,\zeta\in\Sigma_p$ such that $\dist{\xi}{\zeta}{\Sigma_p}<\pi$ there is a geodesic $[\xi\zeta]_{\Sigma_p}$.
Suppose $\dist{\xi}{\zeta}{\Sigma_p}\ge\pi$, then $\T_p$ contains a line thru the origin in the directions $\xi$ and $\zeta$.
By \ref{SHORT.ex:finite-tan:tan} we can apply the splitting theorem (\ref{thm:splitting}) to $\T_p$.
We get that $\Sigma_p$ is a spherical suspension with poles $\xi$ and $\zeta$.
Hence, $\dist\xi\zeta{}=\pi$ and there is a geodesic $[\xi\zeta]$.


\parbf{\ref{ex:proof-right-inverse}}; \ref{SHORT.ex:proof-right-inverse:grad}.
By \ref{ex:distfun-semiconcave}, each function $\distfun_{a_i}$ is semiconcave in a small neighborhood of $p$.
Therefore we can choose $\lambda$ and $r>0$ so that $f_{\bm{y}}$ is $\lambda$-concave in $\oBall(p,r)$; further we will assume that $r$ is sufficiently small.
Choose $\alpha>0$ such that $\angk{x}{a_i}{a_j}>\tfrac\pi2+\alpha$ for all $i\ne j$;
we may assume that $\alpha<\tfrac{1}{10}$;
in particular,
\[(\dd_x\distfun_{a_j}{}{})(\dir{x}{a_i})
\ge
-\cos\angk{x}{a_i}{a_j}
>
\tfrac\alpha2\eqlbl{inq-a_j}\]
for $j\ne i$.

By the definition of gradient and \ref{ex:d(distfun):<}, we have
\begin{align*}
-(\dd_x\distfun_{a_i}{}{})(\nabla_x f_{\bm{y}})
&\ge
\<\dir x{a_i},\nabla_x f_{\bm{y}}\>
\ge
\\
&\ge
(\dd_xf_{\bm{y}})(\dir x{a_i}).
\end{align*}
If $\dist{a_i}{x}{}>y_i$, then 
\[\dd_xf_{\bm{y}}=\sigma+\eps\cdot \dd_x\distfun_{a_0},\]
where $\sigma$ is a minimum of a subset of the following functions
$0$, and $\dd_x\distfun_{a_j}$ for $0\ne j\ne i$.
By \ref{inq-a_j}, 
\[(\dd_x\distfun_{a_i}{}{})(\nabla_x f_{\bm{y}})< -\tfrac\alpha2\cdot\eps.\]
Hence (\ref{111}) holds for all sufficiently small $\eps>0$.

Now assume that $\dist{a_i}{x}{}-y_i=\min_j\{\dist{a_j}{x}{}\z-y_j\}<0$.
Then
\begin{align*}
\dd_x f_{\bm{y}}
&=
\min_{i\in S} \{\,\dd_x\distfun_{a_j}\,\}+\eps\cdot \dd_x\distfun_{a_0}
\le
\\
&\le
\dd_x \distfun_{a_i}{}{}+\eps\cdot(\dd_p\distfun_{a_0}{}{}),
\end{align*}
where $j\in S$ if and only if $\dist{a_i}{x}{}-y_i=\dist{a_j}{x}{}-y_j$.
Applying \ref{inq-a_j}, we get
\begin{align*}
(\dd_x \distfun_{a_i}{}{})(\nabla_x f_{\bm{y}})
&\ge 
\dd_xf_{\bm{y}}(\nabla_x f_{\bm{y}}) -\eps\cdot(\dd_x \distfun_{a_0}{}{})(\nabla_x f_{\bm{y}}) 
\ge 
\\
&\ge
\left[(\dd_xf_{\bm{y}})(\dir x{a_0})\right]^2-2\cdot \eps
\ge
\\
&\ge
\left[\tfrac\alpha2-\eps\right]^2-2\cdot \eps.
\end{align*}
Thus, (\ref{222}) holds for all sufficiently small $\eps>0$. 

\parit{\ref{SHORT.ex:proof-right-inverse:alpha}}
Consider the following real-to-real functions:
\[\begin{aligned}
\phi(t)
&\df
\max_{i}\{\dist{a_i}{\alpha_{\bm{y}}(t)}{}-y_i\},
\\
\psi(t)
&\df
\min_{i}\{\dist{a_i}{\alpha_{\bm{y}}(t)}{}-y_i\}.
\end{aligned}\eqlbl{eq:xy-def}\]
Use \ref{SHORT.ex:proof-right-inverse:grad}, to show that for $t\in[0,t_0]$, we have $\phi^+(t)<-\tfrac{1}{10}\cdot\eps^2$ if $\phi(t)>0$
and $\psi^+(t)>\tfrac{1}{10}\cdot\eps^2$ if $\psi(t)<0$.
Conclude that $\phi(t_0)=\psi(t_0)=0$; hence the result.


\parit{\ref{SHORT.ex:proof-right-inverse:end}}
A straightforward application of \ref{lem:fg-dist-est} and a reformulation of \ref{SHORT.ex:proof-right-inverse:alpha}.

\parbf{\ref{ex:proof-dist-chart}.}
Apply the (\textit{n}+1)-comparison (\ref{thm:n+1}) to show that at least one of the inequalities
\[
\mangle\hinge xy{a_0}<\tfrac\pi2-\eps,\ \dots,\  \mangle\hinge xy{a_m}<\tfrac\pi2-\eps,
\]
holds.
Similarty, we get that at least one of the inequalities
\[
\mangle\hinge yx{a_0}<\tfrac\pi2-\eps,\ \dots,\  \mangle\hinge yx{a_m}<\tfrac\pi2-\eps,
\]
holds.

Suppose our statement does not hold for $x$ and $y$ in a sufficiently small neighborhood of $p$.
It follows that 
\[\mangle\hinge yx{a_0}<\tfrac\pi2-\eps
\quad\text{and}\quad
\mangle\hinge yx{a_0}<\tfrac\pi2-\eps.
\eqlbl{eq:a0}
\]
Note that $\dist{x}{y}{}$ is small compared to $\dist{a_0}{x}{}$ and $\dist{a_0}{y}{}$.
Therefore, the comparison contradicts \ref{eq:a0}. 

By the construction, $f$ is Lipschitz.
From above, we can choose $i>0$ so that $\mangle\hinge xy{a_i}<\tfrac\pi2-\eps$ (if $\mangle\hinge yx{a_i}<\tfrac\pi2-\eps$, then swap $x$ and $y$).
By comparison, there is $c>0$ such that $\dist{a_i}{y}{}\le \dist{a_i}{x}{}+c\cdot \dist{x}{y}{}$.
Hence $f$ is bi-Lipschitz, and now \ref{thm:right-inverse} implies \ref{thm:dist-chart}.


 
\parbf{\ref{ex:diam-compact:proper}.}
Reuse the argument from  the first part of the proof of Bishop--Gromov inequality.

\parbf{\ref{ex:BG}.} 
You should follow the proof Bishop--Gromov inequality, plus prove the following two inequalities 
\begin{align*}
\sinh r_2\cdot \dist{\log_p x}{\log_p y}{\T_p} &\ge\dist{x}{y}{\spc{A}}
\\
\sinh r_2\cdot\dist{w(x)}{w(y)}{\spc{A}} &\ge \sinh r_1\cdot\dist{x}{y}{\spc{A}}
\end{align*}
for any $x,y\in\oBall(p,r)$.

\parbf{\ref{ex:dim=dim}.} 
Suppose $K$ is a compact set in $\spc{A}$ such that $\HausDim K\ge m$.
Use the map $w$ from the proof of the Bishop--Gromov inequality (\ref{inq:BG} and \ref{ex:BG}) to show that any open ball in $\spc{A}$ contains a compact set $K'$ such that $\HausDim K'\ge m$.

Use this in addition to the arguments in \ref{thm:dim=dim}. 

\parbf{\ref{ex:dim-lim}.}
Apply \ref{ex:resporka}.

\parbf{\ref{ex:net}};
\ref{SHORT.ex:net:finite}.
Suppose $X$ is compact.
Then for any $\eps>0$ any cover of $X$ by open $\eps$-balls have a finite subcover.
Note that the centers of these balls is an $\eps$-net of $X$.

Suppose $X$ has a finite $\eps$-net.
Show that any sequence $x_n$ of points in $X$ has a subsequence such that all of its points lie in one $\eps$-ball.
Apply this statement for $\eps=\tfrac1n$ together with the diagonal procedure.

\parit{\ref{SHORT.ex:net:compact}.}
Let $Z$ be a compact $\eps$-net of $X$.
By \ref{SHORT.ex:net:finite}, $Z$ admits a finite $\eps$-net $F$.
Note that $F$ is a $2\cdot\eps$-net of $X$.
Since $\eps>0$ is arbitrary, we get the result.


\parbf{\ref{ex:pack-net}.} If $x_1,\dots,x_n$ is not an $\eps$-net, then there is a point $y$ such that $\dist{x_i}{y}{}\ge\eps$ for any $i$.
Therefore $x_1,\dots,x_n$ is not a maximal packing --- a contradiction.

\parbf{\ref{ex:pack-vol}}; \ref{SHORT.ex:pack-vol:pack}
Apply the Bishop--Gromov inequality (\ref{inq:BG}).

\parit{\ref{SHORT.ex:pack-vol:dim}}
By \ref{ex:dim-lim}, $\dim\spc{A}_\infty\le m$.
To show that $\dim\spc{A}_\infty\ge m$,
apply \ref{cor:euclid-subcone} to a maximal packing and use the estimate in \ref{SHORT.ex:pack-vol:pack}.

\parit{Comment.}
A stronger statement holds 
\[\vol_m\spc{A}_\infty=\lim_{n\to\infty} \vol_m\spc{A}_n;\]
in other words, if $\bm{K}\subset \GH$ denotes the set of isometry classes of all compact $\Alex\kappa$ spaces with dimension $\le m$, then the function
$\vol_m\:\bm{K}\to \RR$ is continuous.


\parbf{\ref{ex:diam-compact:GH}.}
Argue as in \ref{thm:gromov-compactness} to construct a Gromov--Hausdorff convergence of $\cBall(p_n,R)_{\spc{A}_n}$ for given $R>0$, then apply the diagonal procedure to construct the needed convergence.

\parbf{\ref{ex:no-conc}.}
Consider the infinite product $\SSS^1\times ({\tfrac 12}\cdot \SSS^1)\times ({\tfrac 14}\cdot \SSS^1)\times\dots$

\parbf{\ref{ex:conic}.}
Let $V$ and $W$ be two conic neighborhoods of a point~$p$.
Without loss of generality, we may assume that $V\Subset W$;
that is, the closure of $V$ lies in $W$.

Construct a sequence of embeddings $f_n\:V\to W$
such that 
\begin{itemize}
\item 
For any compact set $K\subset V$ 
there is a positive integer $n=n_K$ such that 
$f_n(k)=f_m(k)$ for any $k\in K$ and $m, n \ge n_K$.
\item For any point $w\in W$ there is a point $v\in V$ such that $f_n(v)=w$ for all large $n$.
\end{itemize}

Note that once such a sequence is constructed, $f\:V\to W$ defined by $f(v)=f_n(v)$ for all large values of $n$ gives the needed homeomorphism.

The sequence $f_n$ can be constructed recursively
\[f_{n+1}=\Psi_n\circ f_n\circ \Phi_n,\]
where $\Phi_n\:V\to V$ 
and $\Psi_n\:W\to W$ 
are homeomorphisms
of the form 
\[\Phi_n(x)=\phi_n(x)\ast x\quad \text{and}\quad \Phi_n(x)=\psi_n(x)\star x,\]
where $\phi_n\:V\to \RR_{\ge 0}$, $\psi_n\:W\to \RR_{\ge 0}$ are suitable continuous functions;
``$\ast$'' and ``$\star$'' denote the multiplications in the cone structures of $V$ and $W$ respectively.

\parit{Comment.} If it is hard to follow, read the original proof by Kyung Whan Kwun \cite{kwun1964}.

\parbf{\ref{ex:conic-tangent}}; \ref{SHORT.ex:conic-tangen:tangent}. Apply \ref{thm:spherical-nbhd} and \ref{lem:kwun}.

\parit{\ref{SHORT.ex:conic-tangen:dir}.} Apply \ref{SHORT.ex:conic-tangen:tangent}.

\parit{\ref{SHORT.ex:conic-tangen:example}.} Recall that the Poincaré homology sphere can be obtained as a quotient space $\Sigma=\SSS^3/\Gamma$ by an isometric action of a finite group $\Gamma$  --- the so-called binary icosahedral group.
By the double suspension theorem,  $\Susp^2\Sigma\cong\SSS^5$.
Note that $\Susp^2\Sigma$ is an Alexandrov space and it has a point with space of directions isometric to $\Susp\Sigma$.
Observe that $\Susp\Sigma$ is not a manifold; in particular $\Susp\Sigma\ncong\SSS^4$.
Therefore the pair $\Susp^2\Sigma$ and $\SSS^5$ provides the needed example.

\parbf{\ref{ex:bry2bry}.} Apply \ref{thm:spherical-nbhd}, \ref{lem:kwun}, and \ref{thm:top-bry}.

\parbf{\ref{ex:bry-closed}.}
Let $\spc{A}$ be a finite-dimensional Alexandrov space.
Choose $x\in\spc{A}$.
By \ref{thm:spherical-nbhd}, a neighborhood $U\ni x$ is homeomorphic to $\T_x$.
Therefore \ref{ex:bry2bry}, implies that $U\cap\partial\spc{A}=\emptyset$ if and only if $x\notin \partial\spc{A}$;
that is, the complement $\spc{A}\setminus\partial\spc{A}$ is open, and therefore, $\spc{A}$ is closed.

\parbf{\ref{ex:pz<ypz}.}
Consider the model triangle $[\tilde x\tilde y\tilde z']=\modtrig(xyz)$.
\begin{figure}[ht!]
\vskip-0mm
\centering
\includegraphics{mppics/pic-1015}
\end{figure}

Show that 
\[\dist{\tilde p}{\tilde z}{}\le \dist{\tilde p}{\tilde z'}{}\le\side\hinge yp{z}.\]


\parbf{\ref{ex:bry-connected}.}
Assume $\spc{A}$ has at least two boundary components, say $A$ and $B$.
Denote by $\gamma$ a geodesic that minimizes the distance from $A$ to $B$.

Let 
\[\dots,\spc{A}_{-1},\spc{A}_{0},\spc{A}_{1},\dots\]
be a two-sided infinite sequence of copies on $\partial\spc{A}$.
Let us glue $\spc{A}_{i}$ to $\spc{A}_{i+1}$ along $A$ if $i$ is even and along $B$ if $i$ is odd.

By the doubling theorem, every point in the obtained space $\spc{N}$ has a neighborhood that is isometric to a neighborhood of the corresponding point in $\spc{A}$ or its doubling.
By the globalization theorem, $\spc{N}$ is $\Alex1$.

Note that the copies of $\gamma$ in $\spc{A}_{i}$ form a line in $\spc{N}$.
By the splitting theorem, $\spc{N}$ is isometric to a product $\spc{N}'\oplus \RR$.
Since $\dim\spc{N}>1$, Exercise~\ref{ex:dim=1} implies that $\diam\spc{N}\le \pi$ --- a contradiction.

\parbf{\ref{ex:dist-to-bry}.} Choose $x$ on $\gamma$;
we can assume that $x=\gamma(0)$.
Let $y\in \partial \spc{A}$ be a closest point to $x$.
Let $\alpha=\mangle(\dir xy,\gamma^+(0)$.

Suppose $x\notin \partial \spc{A}$.
Show that $\T_y=\RR_{\ge0}\times\T_y\partial \spc{A}$
and $\dir yx\perp \T_y\partial \spc{A}$.

Given a vector $v\in \T_y$, denote by $\bar v$ its projection to $\T_y\partial \spc{A}$.
Apply the comparison and \ref{prop:gexp} to show that 
\[\dist{\gamma(t)}{\gexp_y(\overline{\log_x\gamma(t)})}{}\le \dist{x}{y}{}+t\cdot\cos\alpha.\]
Conclude that $\gamma''(0)\le 0$ in the barrier sense.


\parbf{\ref{ex:liberman}.}
Suppose $\gamma$ is defined on the interval $[0,\ell]$.
Assume that the function $\rho\:t\mapsto \tfrac12\cdot\distfun_p^2\circ\gamma(t)$ is not $1$-concave.
Let $\bar\rho\:[0,\ell]\to\RR$ be the minimal $1$-concave function such that $\bar\rho\ge \rho$.
Note that $\bar\rho=\rho$ at the ends of $[0,\ell]$.

Consider the curve $\bar\gamma(t)\df \GF_f^{s(t)}\gamma(t)$;
where $f=\tfrac12\cdot\distfun_p^2$ and $s(t)\z=\ln\circ\bar\rho(t)-\ln\circ\rho(t)$.
Use the first distance estimate to show that $\length\bar\gamma<\length\gamma$ and arrive at a contradiction.

\parit{Comment.}
The statement was proved by Grigory Perelman and the second author \cite{perelman-petrunin};
it generalizes a theorem of Joseph Liberman \cite{liberman} about geodesics on convex surfaces.
The original Liberman's version of the following geometric statement.
\textit{Suppose that $C$ is the cone over $\gamma$ with the vertex at $p$,
where $\gamma$ is a geodesic on a convex surface and $p$ is a point in the convex body bounded by the surface.
Then after unfolding $C$ into plane, $\gamma$ becomes a locally convex curve.}
It is instructive to check that this formulation is equivalent to ours for convex bodies.

\parbf{\ref{ex:native}.}
Choose a geodesic $\gamma$ in $\spc{W}$.
Arguing as in the proof of \ref{thm:doubling:doubling}, we get 
that $\gamma$ can cross the common boundary of two halves $\spc{A}_0$ and $\spc{A}_1$ of $\spc{W}$ at most once, or it lies in the common boundary.

In the later case $\lambda$-concavity of $f\circ\proj\circ\gamma$ follows from $\lambda$-concavity of $f$.
In the former case the convexity has to be checked only at the point of crossing;
we may assume that it happens at $x=\gamma(0)$.
Since $\nabla_xf\in\partial\T_x$ for any $x\in\partial\spc{A}$ the $f$-gradient flows agree on $\spc{A}_0$ and $\spc{A}_1$.

Assume $f\circ\proj\circ\gamma$ is not $\lambda$-concavity at $0$.
Apply $f$-gradinent flow to shorten $\gamma$ keeping its ends as in the proof of \ref{ex:liberman},
and arrive at a contradiction.

\parbf{\ref{ex:Hilbert/G}.} Read \cite[Section 4]{terng-thorbergsson} and/or the solution for ``Quotient of the Hilbert space'' in \cite{petrunin2020}.

\parbf{\ref{ex:sumbetries(S^2)}}; \ref{SHORT.ex:sumbetries(S^2):1}.
Choose an isometric $\SSS^1$-action on $\SSS^2$ that fixes the poles of the sphere.
Consider the projection to the quotient space $\sigma_1\:\SSS^2\z\to \SSS^2/\SSS^1=[0,\pi]$.

\parit{\ref{SHORT.ex:sumbetries(S^2):2}.}
Take a half-circle $\gamma$ on $\SSS^2$ and define 
$\sigma_2(x)\df\distfun_\gamma(x)_{\SSS^2}$.

\parit{\ref{SHORT.ex:sumbetries(S^2):n}.}
Consider the subdivision of $\SSS^2$ into $\SSS^1$-orbits of the action from~\ref{SHORT.ex:sumbetries(S^2):1}.
Cut $\SSS^2$ into two hemispheres by meridians rotate one hemisphere by an angle $\alpha=\pi/n$ and glue it back.
Observe that there is a submetry $\sigma_n$ such that the inverse image $\sigma_n^{-1}\{y\}$ is a union of the arcs from the original $\SSS^1$-orbits.

Note that for $n=2$ we get the solution in \ref{SHORT.ex:sumbetries(S^2):2}.

\parbf{\ref{ex:sumbetries(E^2)}.}
Show that for any $x\in\EE^2$ there is a half-line $H\ni x$ such that 
the restriction $\sigma|_H$ is an isometry.
Suppose such a half-line $H$ starts at $p$ and passes thru $q$.
Show that $\langle x-p,q-p \rangle\le 0$ for any $x\in \sigma^{-1}\{0\}$.
Conclude that $\sigma^{-1}\{0\}$ is a convex closed set.
Finally use the definition of submetry to show that  $\sigma^{-1}\{0\}$ has no interior points. 

\parbf{\ref{ex:S^3/S^1}};
\ref{SHORT.ex:S^3/S^1:pq}.
Our $\SSS^1$ is a commutative subgroup of $\SO(3)$.
Therefore it is a subgroup of a maximal torus in $\SO(3)$.
Show that the described torus action is induced by a maximal torus in $\SO(3)$.
Use that maximal tori in $\SO(3)$ are conjugate.

\parit{\ref{SHORT.ex:S^3/S^1:sphere}.}
Cut $\SSS^3$ into two solid tori the Clifford torus $\tfrac1{\sqrt2}\cdot \SSS^1\times \SSS^1$.
Observe that the quotient of each solid torus is a disc;
conclude that $\Sigma_{p,q}$ is a sphere.
The torus action on $\SSS^3$ induce the needed $\SSS^1$-cation on $\Sigma_{p,q}$.

\parit{\ref{SHORT.ex:S^3/S^1:a}+\ref{SHORT.ex:S^3/S^1:b}+\ref{SHORT.ex:S^3/S^1:c}.} Straightforward calculations.

\parit{\ref{SHORT.ex:S^3/S^1:cc}.}
Consider the map $\Sigma_{p,q}\to\Sigma_{1,1}$ that sends poles to poles,
preserve the distance to the poles and respects the $\SSS^1$ action.

\parbf{\ref{ex:number(m-1)}};
\ref{SHORT.ex:number(m-1):2}.
Suppose $\#_{m-1}(\Gamma)\ge 3$;
that is $\spc{A}=\EE^m/\Gamma$ has at least 3 boundary components.
Follow Case~3 in the proof \ref{thm:hsiang-kleiner} to glue a train-space from copies of $\spc{A}$ using two of these components.
Show that the obtained space splits and arrive at a contradiction.

(Alternatively, apply a similar construction to all components of the boundary.
Show that the obtained space has {}\emph{exponential volume growth};
that is, there is $a>1$ such that $\vol \oBall(p,r)>a^r$ for all large~$r$.
Arrive at a contradiction with the Bishop--Gromov inequality.)

\parit{\ref{SHORT.ex:number(m-1):1}.}
Apply the doubling theorem as in Case~2 in the proof \ref{thm:hsiang-kleiner}.

\parbf{\ref{ex:S1actsS3}.}
Show that the quotient space $\Delta=\spc{A}/\mathbb{S}^1$ is an $\Alex1$ disc and $\gamma$ projects isometrically to $\partial\Delta$.
It remains to show that the perimeter of $\Delta$ cannot exceed $2\cdot\pi$.
The latter follows from \cite[3.3.5]{petrunin:survey};
it states that if $\Delta$ as an $m$-dimensional $\Alex1$ space, then $\vol_{m-1}\partial \Delta\le \vol_{m-1}\partial \mathbb{S}^{m-1}$.

\parbf{\ref{ex:surf-S2}.}
We can assume that the origin lies in the interior of the convex body.
Consider the central projection from its surface, say $\Sigma$, to the sphere $\SSS^2$ centered at the origin.
Show that this projection $\Sigma\to \SSS^2$ is a homeomorphism.

\parbf{\ref{ex:vertex-essential-vertex}.}
Follow the argument in \ref{clm:total-angle}.
Show that the inequality is strict if and only if $F$ has opposite points.


\parbf{\ref{ex:geodesic-vertex}.}
Suppose a geodesic $\gamma$ passes thru a vertex $v$.
Denote by $\alpha$ and $\beta$ the angles that $\gamma$ cuts at $v$.
Since $v$ is essential, $\alpha+\beta<2\cdot\pi$.
Therefore $\alpha<\pi$ or $\beta<\pi$.
Arrive at a contradiction by showing that $\gamma$ is not length-minimizing.

\parbf{\ref{pr:tetrahedron}}; \ref{SHORT.pr:tetrahedron:=}.
Cut the surface of $T$ along three edges coming from one vertex $v_1$ and unfold the obtained surface onto the plane.
Show that this way we get a triangle, the three vertices correspond to $v_1$ and the midpoints of sides correspond to the remaining three vertices.
Make a conclusion.

\parit{\ref{SHORT.pr:tetrahedron:perp}}.
Suppose that $0,v_1,v_2,v_3\in\RR^3$ are the vertices of $T$.
From \ref{SHORT.pr:tetrahedron:=}, we have that 
\[|v_1|=|v_2-v_3|,\quad |v_2|=|v_3-v_1|,\quad|v_3|=|v_1-v_1|.\]
Use it to show that $\langle v_1,v_2+v_3-v_1\rangle=0$.
Make a conclusion.

\parbf{\ref{ex:poly-CBB}.}
We need to show that if a polyhedral surface is $\Alex0$, then the total angle $\theta$ at every vertex $p$ it at most $2\cdot\pi$.

Assume that $\theta>2\cdot\pi$,
let $\phi=\max\{\,\pi,\tfrac13\cdot\theta\,\}$.
Note that we can choose three points $x_1$, $x_2$, and $x_3$ close to $p$ such that 
$\mangle \hinge p{x_i}{x_j}=\phi$ for $i\ne j$.
Since the points $x_i$ are close to $p$, we have $\mangle \hinge p{x_i}{x_j}=\angk p{x_i}{x_j}$.
The latter contradicts $\EE^2$-comparison. 

\parbf{\ref{ex:surface-covergence}.}
We will use that the closest-point projection from the Euclidean space to a convex body is \index{short map}\emph{short};
that is, distance-nonexpanding \cite[13.3]{petrunin-zamora}.

Assume $K_\infty$ is nondegenerate.
Without loss of generality, we may assume that 
\[\cBall(0,r)\subset K_\infty\subset\cBall(0,1)\]
for some $r>0$.
Note that there is a sequence $\eps_n\to 0$ such that 
\[ K_n\subset(1+\eps_n)\cdot K_\infty
\quad\text{and}\quad
K_\infty\subset(1+\eps_n)\cdot K_n\]
for each large $n$.

Given $x\in K_n$, denote by $g_n(x)$ the closest-point projection of $(1+\eps_n)\cdot x$ to $K_\infty$.
Similarly, given $x\in K_\infty$, denote by $h_n(x)$ the closest point projection of $(1+\eps_n)\cdot x$ to $K_n$.
Note that 
\begin{align*}
\dist{g_n(x)}{g_n(y)}{}&\le (1+\eps_n)\cdot\dist{x}{y}{}
\intertext{and}
\dist{h_n(x)}{h_n(y)}{}&\le (1+\eps_n)\cdot\dist{x}{y}{}.
\end{align*}

Denote by $\Sigma_\infty$ and $\Sigma_n$ the surface of $K_\infty$ and $K_n$ respectively. 
The above inequalities imply 
\begin{align*}
\dist{g_n(x)}{g_n(y)}{\Sigma_\infty}&\le (1+\eps_n)\cdot\dist{x}{y}{\Sigma_n}
\intertext{for any $x,y\in \Sigma_n$, and}
\dist{h_n(x)}{h_n(y)}{\Sigma_n}&\le (1+\eps_n)\cdot\dist{x}{y}{\Sigma_\infty}.
\end{align*}
for any $x,y\in \Sigma_\infty$.

Note that the maps $g_n$ and $h_n$ are onto.
Apply \ref{ex:GH-po} to finish the proof.

Alternatively, since the closest-point projection cannot increase the length of curve, we also get
\begin{align*}
\dist{x}{h_n\circ g_n(x)}{\Sigma_\infty}&\le 10\cdot \eps_n
\\
\dist{y}{g_n\circ h_n(y)}{\Sigma_n}&\le 10\cdot \eps_n.
\end{align*}
for all large $n$.
Therefore, $g_n$ is a $\delta_n$-isometry $\Sigma_n\to\Sigma_\infty$ for a sequence $\delta_n\to 0$.

\parit{Comments.}
More generally, if a sequence of $m$-dimensional $\Alex\kappa$ spaces $\spc{A}_1,\spc{A}_2,\dots$ converges to $\spc{A}_\infty$ and $\dim \spc{A}_\infty=m<\infty$,
then $\partial \spc{A}_n$ equipped with the induced length metrics converge to  $\partial \spc{A}_\infty$.
This statement is a partial case of the theorem about extremal subsets proved by the second author \cite[1.2]{petrunin1997}.

\parbf{\ref{ex:liberman+milka}}; \ref{SHORT.ex:liberman+milka:liberman}.
By \ref{ex:liberman}, the function $f_p\:t\mapsto \distfun_p\circ\gamma(t)$ is semiconcave for any $p\in K$.
In particular, one-sided derivatives $f_p^+(t)$ are defined for every $t$.

Given $x=\gamma(t)$, choose three points $p_1,p_2,p_3\in K$ in general position;
that is, the four points $x,p_1,p_2,p_3$ do not lie in one plane.
Observe that the distance functions $\distfun_{p_i}$ give smooth coordinates in a neighborhood of $x$.
From above the functions $f_{p_i}$ have one-sided derivatives at $t$.
Since the coordinates are smooth we get that $\gamma^+(t)$ is defined as well.

\parit{\ref{SHORT.ex:liberman+milka:milka}.}
If the plane $py_1y_2$ supports $K$, then 
$\mangle\hinge p{y_1}{y_2}_{\EE^3}=\mangle\hinge p{x_1}{x_2}_S$.
In this case, the statement follows from \ref{prop:conv-surf-CBB(0)}.

Now suppose that the line segment $[y_1y_2]_{\EE^3}$ intersects $K$.
Choose a geodesic $[y_1y_2]_W$;
note that it contains a point of $K$, say $z$.
Now consider a one-parameter family of points 
$y_i(t)\df \gamma(t)+\gamma^+(t)\z\cdot (1-t)\z\cdot \dist{p}{x_i}{S}$.
Note that this family is not continuous.

Show that for any point $p\in K$, the function $t\mapsto \dist{p}{\gamma_i(t)}{\EE^3}$ is nonincreasing.
Conclude that the function $t\mapsto \dist{p}{\gamma_i(t)}{W}$ is nonincreasing for any $p\in S$.
Therefore, 
\begin{align*}
\dist{y_1}{y_2}{W}
&=\dist{y_1(0)}{y_2(0)}{W}=
\\
&=\dist{y_1(0)}{z}{W}+\dist{y_2(0)}{z}{W}\ge
\\
&\ge\dist{y_1(1)}{z}{W}+\dist{y_2(1)}{z}{W}\ge 
\\
&\ge\dist{x_1}{x_2}{S}.
\end{align*}
The last inequality follows since the closest point projection $W\to S$ is short.

It remains to consider the case when the plane $py_1y_2$ does not support $K$,
and $[y_1y_2]_{\EE^3}$ does not intersect $K$.
In this case the plane $py_1y_2$ intersects $K$ along a convex figure $F$ that lies in the solid triangle 
$py_1y_2$ and contains its vertex $p$.

Choose points $y_1'\in [py_1]_{\EE^3}$ and $y_2'\in [py_2]_{\EE^3}$ such that $[y_1'y_2']$ touches $F$.
Denote by $x_1'\in [px_1]_{S}$ and $x_2'\in [px_2]_{S}$ the corresponding points;
that is, $\dist{p}{y_1'}{\EE^3}=\dist{p}{x_1'}S$ and $\dist{p}{y_2'}{\EE^3}=\dist{p}{x_2'}S$.
From the above, we have that $\dist{y_1'}{y_2'}{\EE^3}\ge\dist{x_1'}{x_2'}S$;
in other words, 
\[\angk p{y_1'}{y_2'}\ge \angk p{x_1'}{x_2'};\]
here we think that $[p{y_1'}{y_2'}]$ is a triangle in $\EE^3$, but $[p{x_1'}{x_2'}]$ is a triangle in $S$.
Note that 
\[\angk p{y_1'}{y_2'}=\angk p{y_1}{y_2}
\quad\text{and}\quad
\angk p{x_1}{x_2}\le \angk p{x_1'}{x_2'};
\]
the second inequality follows from \ref{ex:noncreasing}.
Hence the remaining case follows.

\parit{Comments.}
Part~\ref{SHORT.ex:liberman+milka:liberman} is the so-called Liberman lemma --- the main tools in studying geodesics on convex surfaces.
It was originally proved by Joseph Liberman \cite{liberman}; the proof of \ref{ex:liberman} is its generalization. 

Part~\ref{SHORT.ex:liberman+milka:milka} is the result of Anatolii Milka \cite[Theorem 2]{milka1982}.

%%%%%%%%%%%%%%%%

{
\documentclass[twoside]{book}

%\newcommand{\spell}[2]{#1} %spell
\newcommand{\spell}[2]{#2} %notes


\def\thetitle{A journey into Alexandrov geometry:\\
curvature bounded below}
\def\theauthors{Vitali Kapovitch and Anton Petrunin}

\usepackage{lectures}
\usepackage[colorlinks=true,
citecolor=black,
linkcolor=black,
anchorcolor=black,
filecolor=black,
menucolor=black,
urlcolor=black,
pdftitle={\thetitle},
pdfsubject={Geometry},
pdfauthor={\theauthors}
]{hyperref}
\makeindex

%\usepackage[x-1a]{pdfx}

%\overfullrule=100mm
\def\red{\color{red}}
\begin{document}

\spell{\pagestyle{empty}\renewcommand\includegraphics[2][{}]{}\def\emph{\textit}\renewcommand\footnote[1]{\ (#1)}\renewcommand\z{}\renewcommand\section[1]{SECTION. {#1} SECTION.}}{}

\frontmatter
\title{\thetitle}
\author{\theauthors}
\date{}
\maketitle
\thispagestyle{empty}

\mainmatter
\newpage
\tableofcontents

\include{preface}

\include{prelim}

%%%%%%%%%%%%%%%%%%%%%%%%%%%%
%\include{wald}
\include{definitions}
\include{globalization}
\include{calculus}
\include{gradient-flow}
\include{splitting}
\input{volume.tex}
\input{homot-finite.tex}
\include{boundary}

\include{CBB-quotients}

%\include{CBB-def}

%\include{embedding-theorem}
%\include{misc}


\include{overview}
%\include{alexandrov-theorem-en}

\backmatter

\include{sols}%%%%%%%%%%%%%%%%

{
\input{invitation-CBB.ind}

\def\emph{\textit}

\printbibliography[heading=bibintoc]
\fussy
}


\end{document}


\def\emph{\textit}

\printbibliography[heading=bibintoc]
\fussy
}


\end{document}


\def\emph{\textit}

\printbibliography[heading=bibintoc]
\fussy
}


\end{document}


\def\emph{\textit}

\printbibliography[heading=bibintoc]
\fussy
}


\end{document}
