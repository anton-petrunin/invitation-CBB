%%!TEX root = the-quotients.tex
\chapter{Quotients}\label{chap:L/G}

This lecture gives several applications of Alexandrov geometry to isometric group actions.

\section{Quotient space}

Suppose that a group $G$ acts isometrically on a metric space $\spc{X}$.
Note that
\[\dist{G\cdot x}{G\cdot y}{\spc{X}/G}
\df
\inf
\set{\dist{x}{g\cdot y}{\spc{X}}}{g\in G}\]
defines a semimetric on the orbit space $\spc{X}/G$.
Moreover, if the orbits of the action are closed,
then it is a genuine metric.

\begin{thm}{Theorem}\label{thm:CBB/G}
Suppose that a group $G$ acts isometrically on a proper $\Alex0$ space $\spc{A}$, and $G$ has closed orbits.
Then the quotient space $\spc{A}/G$ is $\Alex0$.

\end{thm}

A more general formulation will be given in \ref{thm:submetry-CBB-1}.

\parit{Proof.}
Denote by $\sigma\:\spc{A}\to \spc{A}/G$ the quotient map.

Fix a quadruple of points $p,x_1,x_2,x_3\in \spc{A}/G$.
Choose $\hat p\in \spc{A}$ such that $\sigma(\hat{p})=p$.
Since $\spc{A}$ is proper, we can choose  points $\hat{x}_i\in \spc{A}$ such that $\sigma(\hat x_i)=x_i$ and
\[\dist{p}{x_i}{\spc{A}/G}
=
\dist{\hat{p}}{\hat{x}_i}{\spc{A}}\]
for all $i$.

Note that 
\[\dist{x_i}{x_j}{\spc{A}/G}
\le 
\dist{\hat{x}_i}{\hat{x}_j}{\spc{A}}
\]
for all $i$ and $j$.
Therefore 
\[\angk p{x_i}{x_j}
\le
\angk {\hat{p}}{\hat{x}_i}{\hat{x}_j}
\eqlbl{eq:angles-M-L}\]
for all $i$ and $j$.

By $\EE^2$-comparison in $\spc{A}$,
we have
\[\angk {\hat{p}}{\hat{x}_1}{\hat{x}_2}
+\angk {\hat{p}}{\hat{x}_2}{\hat{x}_3}
+\angk {\hat{p}}{\hat{x}_3}{\hat{x}_1}
\le 
2\cdot\pi.\]
Applying  \ref{eq:angles-M-L}, 
we get 
\[\angk p{x_1}{x_2}
+\angk p{x_2}{x_3}
+\angk p{x_3}{x_1}\le 2\cdot\pi;\]
that is,
the $\EE^2$-comparison holds for any quadruple in $\spc{A}/G$.
\qeds

\begin{thm}{Very advanced exercise}\label{ex:Hilbert/G}
Let $G$ be a compact Lie group with a bi-invariant Riemannian metric.
Show that $G$ is isometric to a quotient of a Hilbert space by an isometric group action.

Conclude that $G$ is $\Alex0$.
\end{thm}

\section{Submetries}

A map $\sigma\:\spc{X}\to\spc{Y}$ between metric spaces $\spc{X}$ and $\spc{Y}$
is called a \index{submetry}\emph{submetry} if 
\[\sigma(\oBall(p,r)_\spc{X})=\oBall(\sigma(p),r)_{\spc{Y}}\]
for any $p\in \spc{X}$ and $r\ge 0$.

Suppose $G$ and $\spc{A}$ are as in \ref{thm:CBB/G}.
Observe that the quotient map $\sigma\:\spc{A}\to \spc{A}/G$ is a submetry.
The following two exercises show that this is not the only source of submetries. 

\begin{thm}{Exercise}\label{ex:sumbetries(S^2)}
Construct submetries
\begin{subthm}{ex:sumbetries(S^2):1}
$\sigma_1\:\mathbb{S}^2\to[0,\pi]$,
\end{subthm}
\begin{subthm}{ex:sumbetries(S^2):2}
$\sigma_2\:\mathbb{S}^2\to[0,\tfrac\pi2]$,
\end{subthm}
\begin{subthm}{ex:sumbetries(S^2):n}
$\sigma_n\:\mathbb{S}^2\to[0,\tfrac\pi n]$ (for integer $n\ge 1$)
\end{subthm}
such that the fibers $\sigma_n^{-1}\{x\}$ are connected for any $x$.
\end{thm}

\begin{thm}{Exercise}\label{ex:sumbetries(E^2)}
Let $\sigma\:\EE^2\to [0,\infty)$ be a submetry.
Show that $K\z=\sigma^{-1}\{0\}$ is a closed convex set without interior points and $\sigma(x)\z=\distfun_Kx$.
\end{thm}

The proof of \ref{thm:CBB/G} works for submetries;
that is, \textit{if $\sigma\:\spc{A}\to\spc{B}$ is a submetry and $\spc{A}$ is a proper $\Alex0$ space, then so is $\spc{B}$}.
Theorem \ref{thm:CBB/G} admits a straightforward generalization to $\Alex{-1}$ case.

In the $\Alex1$ case, the proof produces a slightly weaker statement ---  \textit{$\SSS^2$-comparison holds for a quartuple $p,x_1,x_2,x_3$ in the quotient of $\Alex1$ if $\dist{p}{x_i}{}<\tfrac\pi 2$ for each $i$}.
In particular, the quotient space is \textit{locally} $\Alex1$.
But since $\Alex1$ space is geodesic, then so is its quotient.
Therefore, the globalization theorem implies that it is globally $\Alex1$.
The same holds for the targets of submetries from an  $\Alex1$ space.
With a bit of extra work, one can extend the statement to nonproper spaces \cite[8.34]{alexander-kapovitch-petrunin2024}.
Thus, we have the following.

\begin{thm}{Theorem}\label{thm:submetry-CBB-1}
Let $\sigma\:\spc{A}\to\spc{B}$ be a submetry.
If $\spc{A}$ is $\Alex\kappa$ space, then so is $\spc{B}$.

In particular, if $G$ acts isometrically on an $\Alex\kappa$ space $\spc{A}$, and $G$ has closed orbits.
Then the quotient space $\spc{A}/G$ is $\Alex\kappa$.
\end{thm}

\section{Hopf's conjecture}

\textit{Does $\mathbb{S}^2\times\mathbb{S}^2$ admit a Riemannian metric with positive sectional curvature?} \index{Hopf's conjecture}\emph{Hopf's conjecture} says that the answer should be negative.
Let us take a close look at the following partial result obtained by Wu-Yi Hsiang and Bruce Kleiner \cite{hsiang-kleiner}.

\begin{thm}{Theorem}\label{thm:hsiang-kleiner}
There is no Riemannian metric on $\SSS^2\times\SSS^2$ with sectional curvature $\ge 1$ and a nontrivial isometric $\SSS^1$-action.
\end{thm}

Reacall that a group action $G\acts\spc{X}$ is called \index{effective action}\emph{effective} if for any $g\in G$ there is $x\in\spc{X}$ such that $g\cdot x\ne x$.

\begin{thm}{Key lemma}\label{lem:S^3/S^1}
Suppose $\SSS^1\acts\SSS^3$ is an effective isometric action without fixed points
and $\Sigma=\SSS^3/\SSS^1$ is its quotient space.
Then there is a distance noncontracting map $\Sigma\to \tfrac12\cdot \SSS^2$, where $\tfrac12\cdot \SSS^2$ is the standard 2-sphere rescaled with a factor $\tfrac12$.
\end{thm}

The proof of the lemma is guided by the following exercise.

\begin{thm}{Exercise}\label{ex:S^3/S^1}
Suppose $\SSS^1\acts\SSS^3$ is an effective isometric action without fixed points.
Let us think   of $\SSS^3$ as the unit sphere in $\RR^4$.

\begin{subthm}{ex:S^3/S^1:pq}
Show that one can identify $\RR^4$ with $\CC^2$ so that the action
is given by matrix multiplication
\[\left(\begin{matrix}
u^p&0\\
0& u^q
\end{matrix}
\right),\]
where $(p,q)$ is a pair of relatively prime positive integers and $u\in \SSS^1=\set{z\in\CC}{|z|=1}$.
In particular, our $\SSS^1$ is a subgroup of the torus that acts by
matrix multiplication
\[\left(\begin{matrix}
v&0\\
0& w
\end{matrix}
\right),\]
where  $v,w\in \SSS^1$.
\end{subthm}

\smallskip

\noindent Fix $p$ and $q$ as above.
Let $\Sigma_{p,q}=\SSS^3/\SSS^1$ be the quotient space.

\smallskip

\begin{subthm}{ex:S^3/S^1:sphere}
Show that the $\Sigma_{p,q}=\SSS^3/\SSS^1$ is a topological sphere with $\SSS^1$-symmetry.
This symmetry has two fixed points, north pole and south pole, that correspond to the orbits of $(1,0)$ and $(0,1)$ in $\SSS^3$.
\end{subthm}

\smallskip

\noindent Denote by $S(r)$ the circle of radius $r$ with the center at the north pole of $\Sigma_{p,q}$.

\begin{subthm}{ex:S^3/S^1:a}
Denote by $T(r)$ the inverse image $T(r)$ in $\SSS^3$, and let $a(r)$ be its area.
Show that $T(r)$ is an orbit of the torus action and
\[a(r)=\pi^2\cdot\sin r\cdot \cos r.\]

\end{subthm}

\smallskip

\begin{subthm}{ex:S^3/S^1:b}
Let $b_{p,q}(r)$ be the length of the $\SSS^1$-orbit in $\SSS^3$ that corresponds to a point on $S(r)$. 
Show that
\[b_{p,q}=\pi\cdot\sqrt{(p\cdot \sin r)^2+(q\cdot \cos r)^2}.\]
\end{subthm}

\smallskip

\begin{subthm}{ex:S^3/S^1:c}
Let $c_{p,q}(r)$ be the length of $S(r)$.
Show that $a(r)=c_{p,q}(r)\cdot b_{p,q}(r)$.
\end{subthm}

\smallskip

\begin{subthm}{ex:S^3/S^1:cc}
Show that $c_{p,q}(r)\le c_{1,1}(r)$ for any pair $(p,q)$ of relatively prime positive integers.
Use it to construct a distance noncontracting map $\Sigma_{p,q}\to \tfrac12\cdot \SSS^2\iso\Sigma_{1,1}$.
\end{subthm}

\end{thm}

\parit{Proof of \ref{thm:hsiang-kleiner}.}
Assume $\spc{B}=(\SSS^2\times\SSS^2,g)$ is a counterexample.
By the Toponogov theorem, $\spc{B}$ is $\Alex1$.
By \ref{thm:CBB/G}, the quotient space $\spc{A}\z=\spc{B}/\SSS^1$ is $\Alex1$;
evidently, $\spc{A}$ is 3-dimensional.

Denote by $F\subset \spc{B}$ the fixed point set of the $\SSS^1$-action.
Then $\chi(\spc{B})\z=\chi(F)$.
Each connected component of $F$ is either an isolated point or a 2-dimensional geodesic submanifold in $\spc{B}$;
the latter has to have positive curvature, and therefore it is homeomorphic to $\SSS^2$ or $\RP^2$.
Notice that 
\begin{itemize}
 \item each isolated point contributes 1 to the Euler characteristic of~$\spc{B}$,
 \item each sphere contributes 2 to the Euler characteristic of $\spc{B}$, and
 \item each projective plane contributes 1 to the Euler characteristic of~$\spc{B}$.
\end{itemize}
Since $\chi(\spc{B})=4$, we are in one of the following three cases:
\begin{enumerate}
 \item\label{case1} $F$ has exactly 4 isolated points,
 \item\label{case2} $F$ has one 2-dimensional submanifold and at least 2 isolated points,
 \item\label{case3} $F$ has at least two 2-dimensional submanifolds.
\end{enumerate}
In each case we will arrive at a contradiction.

\parit{Case \ref{case1}.}
Suppose $F$ has exactly 4 isolated points $x_1$, $x_2$, $x_3$, and $x_4$.
Denote by $y_1$, $y_2$, $y_3$, and $y_4$ the corresponding points in $\spc{A}$.
Note that $\Sigma_{y_i}\spc{A}$ is isometric to a quotient of $\SSS^3$ by an isometric $\SSS^1$-action without fixed points.

By \ref{ex:S^3/S^1}, each angle $\mangle\hinge{y_i}{y_j}{y_k}\le \tfrac\pi2$ for any three distinct points 
$y_i$, $y_j$, $y_k$.
In particular, all four triangles $[y_1y_2y_3]$, $[y_1y_2y_4]$, $[y_1y_3y_4]$, and $[y_2y_3y_4]$ are nondegenerate.
By the comparison, the sum of angles in each triangle is strictly greater than $\pi$.

Denote by $\omega$ the sum of all 12 angles in the 4 triangles $[y_1y_2y_3]$, $[y_1y_2y_4]$, $[y_1y_3y_4]$, and $[y_2y_3y_4]$.
From above,
\[\omega>4\cdot\pi.\]

On the other hand, by \ref{ex:S^3/S^1} any triangle in $\Sigma_{y_1}\spc{A}$ has perimeter at most $\pi$.
In particular, 
\[\mangle\hinge{y_1}{y_2}{y_3}+\mangle\hinge{y_1}{y_3}{y_4}+\mangle\hinge{y_1}{y_4}{y_2}\le \pi.\]
Apply the same argument in $\Sigma_{y_2}\spc{A}$, $\Sigma_{y_3}\spc{A}$, and $\Sigma_{y_4}\spc{A}$;
adding the results, we get 
\[\omega\le 4\cdot\pi\]
--- a contradiction.

\parit{Case \ref{case2}.}
Suppose $F$ contains one surface $S$.
Then the projection of $S$ to $\spc{A}$ forms its boundary $\partial \spc{A}$.
The doubling $\spc{W}$ of $\spc{A}$ across its boundary has at least 4 singular points --- each singular point of $\spc{A}$ corresponds to two singular points of $\spc{W}$.

By the doubling theorem, $\spc{W}$ is a $\Alex1$ space.
Therefore we arrive at a contradiction in the same way as in the first case.

\parit{Case \ref{case3}.} Impossible by \ref{ex:bry-connected}.
\qeds

\section{Erdős' problem rediscovered}

A point $p$ in an Alexandrov space is called \index{extremal point}\emph{extremal} if $\mangle\hinge pxy\le \tfrac\pi2$ for any hinge $\hinge pxy$ with the vertex at $p$; equivalently, $\diam \Sigma_p\le \pi/2$.

\begin{thm}{Theorem}\label{thm:extr-point}
Let $\spc{A}$ be a compact $m$-dimensional $\Alex0$ space.
Then it has at most $2^m$ extremal points.
\end{thm}

\parit{Proof of \ref{thm:extr-point}.}
Let $\{p_1,\dots,p_N\}$ be extremal points in $\spc{A}$.
For each $p_i$ consider its open \index{Voronoi domain}\emph{Voronoi domain} $V_i$; that is, 
\[V_i=\set{x\in \spc{A}}{\dist{p_i}{x}{}<\dist{p_j}{x}{}\ \text{for any}\ j\not=i}.\]
Clearly $V_i\cap V_j=\emptyset$ if $i\not=j$.

Suppose  $0<\alpha\le 1$.
Given a point $x\in\spc{A}$, choose a geodesic $[p_ix]$ and denote by $x_i$ the point on $[p_ix]$ such that $\dist{p_i}{x_i}{}=\alpha\cdot\dist{p_i}{x}{}$;
let $\map_i\:x\to x_i$ be the corresponding map.
By the comparison, 
\[\dist{x_i}{y_i}{}\ge\alpha\cdot \dist{x}{y}{}\]
for any $x$, $y$, and $i$.
Therefore 
\[\vol(\map_i \spc{A})\ge\alpha^m\cdot\vol \spc{A}.\]

Suppose $\alpha<\tfrac12$.
Then $x_i\in V_i$ for any $x\in \spc{A}$.
Indeed, assume $x_i\notin V_i$,
then there is $p_j$ such that $\dist{p_i}{x_i}{}\ge\dist{p_j}{x_i}{}$.
Then by comparison, we have $\angk{p_j}{p_i}{x}_{\EE^2}>\tfrac\pi2$;
that is, $p_j$ is not an extremal point.

It follows that $\vol V_i\ge\alpha^m\cdot\vol \spc{A}$
for any $0<\alpha<\tfrac12$; hence 
\[\vol V_i\ge\tfrac1{2^m}\cdot\vol \spc{A}.\]
Since $V_1,\dots,V_N$ are disjoint subsets of $\spc{A}$, we have $N\le 2^m$.
\qeds


\section{Crystallographic actions}

An isometric action $\Gamma\acts \EE^m$ is called \index{crystallographic action}\emph{crystallographic} if it is 
\index{properly discontinuous}\emph{properly discontinuous} (that is, for any compact set $K\subset \EE^m$ and $x\z\in \EE^m$ there are only finitely many elements $g\in \Gamma$ such that $g\cdot x\in K$) and \emph{cocompact} (that is, the quotient space $\spc{A}=\EE^m/\Gamma$ is compact).

Let $F$ be a maximal finite subgroup of $\Gamma$;
that is, if $F<H<\Gamma$ for a finite group $H$, then $F=H$.
Denote by $\mathfrak{M}(\Gamma)$ the number of maximal finite subgroups of $\Gamma$ up to conjugation.

\begin{thm}{Open question}
Let $\Gamma\acts \EE^m$ be a crystallographic action.
Is it true that $\mathfrak{M}(\Gamma)\le 2^m$?
\end{thm}

Note that any finite subgroup $F$ of $\Gamma$ fixes an affine subspace $A_F$ in $\EE^m$.
If $F$ is maximal, then $A_F$ completely describes $F$.
Indeed, since the action is properly discontinuous, the subgroup of $\Gamma$ that fix $A_F$ has to be finite.
This subgroup must contain $F$, but since $F$ is maximal, it must coinside with $F$. 

Denote by $\mathfrak{M}_k(\Gamma)$ the number of maximal finite subgroups $F<\Gamma$ (up to conjugation) such that $\dim A_F=k$.

Choose a finite subgroup $F<\Gamma$; consider a conjugate subgroup $F'=g \cdot F \cdot g^{-1}$.
Note that $A_{F'}=g\cdot A_F$.
In particular, the subspaces $A_F$ and $A_{F'}$ have the same image in the quotient space $\spc{A}=\EE^m/\Gamma$.
Therefore, to count subgroups up to conjugation, we need to count the images of their fixed sets.
By the lemma below (\ref{lem:extr/G}), $\mathfrak{M}_0(\Gamma)$ cannot exceed the number of extremal points in $\spc{A}=\EE^m/\Gamma$.
Combining this observation with \ref{thm:extr-point}, we get the following.

\begin{thm}{Proposition}\label{prop:2m}
Let $\Gamma\acts \EE^m$ be a crystallographic action.
Then $\mathfrak{M}_0(\Gamma)\le 2^m$.
\end{thm}

\begin{thm}{Lemma}\label{lem:extr/G}
Let $\Gamma\acts \EE^m$ be a crystallographic action and $F$ be a maximal finite subgroup of $\Gamma$ that fixes an isolated point $p$.
Then the image of $p$ in the quotient space $\spc{A}=\EE^m/\Gamma$ is an extremal point.
\end{thm}

\parit{Proof.}
Let $q$ be the image of $p$.
Suppose $q$ is not extremal;
that is, $\mangle \hinge q{y_1}{y_2}>\tfrac\pi2$ for some hinge $\hinge q{y_1}{y_2}$ in $\spc{A}$.

Choose the inverse images $x_1,x_2\in \EE^m$ of $y_1,y_2\in \spc{A}$ such that $\dist{p}{x_i}{\EE^m}=\dist{q}{y_i}{\spc{A}}$.
Note that $\mangle \hinge p{x_1}{x_2}\ge \mangle \hinge q{y_1}{y_2}>\tfrac\pi2$.
Moreover, since $p$ is fixed by $F$, we have
\[\mangle \hinge p{x_1}{g\cdot x_2}>\tfrac\pi2
\eqlbl{eq:>pi/2}\]
for any $g\in F$.

Denote by $z$ the barycenter of the orbit $F\cdot x_2$.
Note that $z$ is a fixed point of $F$.
By \ref{eq:>pi/2}, $z\ne p$;
so $F$ must fix the line $pz$.
But $p$ is an isolated fixed point of $F$ --- a contradiction.
\qeds

\begin{thm}{Exercise}\label{ex:number(m-1)}
Let $\Gamma\acts \EE^m$ be a crystallographic action.
Show that
\begin{subthm}{ex:number(m-1):2}
$\mathfrak{M}_{m-1}(\Gamma)\le 2$, and
\end{subthm}

\begin{subthm}{ex:number(m-1):1}
if $\mathfrak{M}_{m-1}(\Gamma)=1$, then $\mathfrak{M}_0(\Gamma)\le 2^{m-1}$.
\end{subthm}

Construct  crystallographic actions with equalities in \ref{SHORT.ex:number(m-1):2} and \ref{SHORT.ex:number(m-1):1}.
\end{thm}

\section{Remarks}

Submetries were introduced by Valerii Berestovskii \cite{berestovskii1987} and have attracted attention in various contexts of differential and metric geometry.



A more general form of Theorem \ref{thm:hsiang-kleiner} was found by Karsten Grove and Burkhard Wilking \cite{grove-wilking};
it classifies isometric $\SSS^1$ actions on  4-dimensional manifolds with nonnegative sectional curvature.
This proof is as beautiful as the original work of Wu-Yi Hsiang and Bruce Kleiner.

It is expected that \textit{no $\Alex1$ space with a nontrivial isometric $\SSS^1$-action can be homeomorphic to $\SSS^2\times\SSS^2$};
so \ref{thm:hsiang-kleiner} holds for general $\Alex1$ space.
The proof of \ref{thm:hsiang-kleiner} would work if we had the following generalization of \ref{lem:S^3/S^1};
see \cite{harvey-searle}.

\begin{thm}{Open question}
Let $\Sigma$ be an $\Alex1$ space homeomorphic to $\SSS^3$.
Suppose $\SSS^1$ acts on $\Sigma$ isometrically and without fixed points.
Is it true that any triangle in $\Sigma/\SSS^1$ has perimeter at most $\pi$?

And if the answer is, is there a distance-noncontracting map
\[\Sigma/\SSS^1\z\to \tfrac12\cdot\SSS^2?\]
\end{thm}


\begin{thm}{Advanced exercise}\label{ex:S1actsS3}
Suppose $\SSS^1$ acts isometrically on an $\Alex1$ space $\spc{A}$ that is homeomorphic to $\SSS^3$.
Assume its fixed-point set is a closed local geodesic $\gamma$.
Show that
\[\length\gamma\le2\cdot\pi.\]
\end{thm}

An analogous question for a $\ZZ_2$-action is open \cite{petrunin-involution}.

Theorem \ref{thm:extr-point} is a translation of the following classical problem in discrete geometry to Alexandrov's language.

\begin{thm}{Problem}\label{erdos-problem}
Let $F$ be a set of points in $\EE^m$ such that any triangle formed by three distinct points in $F$ has no obtuse angles.
Then  $|F|\le2^m$.
Moreover, if $|F|=2^m$, then $F$ consists of the vertices of an $m$-dimensional rectangle.
\end{thm}

This problem was posed by Paul Erdős \cite{erdos} and solved by Ludwig Danzer and Branko Gr\"unbaum \cite{danzer-gruenbaum}.
Grigory Perelman noticed that, after proper definitions, the same proof works in Alexandrov spaces \cite{perelman-Erdos}; thus, it proves \ref{thm:extr-point}.
Applying the our argument to the convex hull of $F$ in \ref{erdos-problem} proves that $|F|\le 2^m$;
the case of equality requires more work.

Compact $m$-dimensional $\Alex0$ spaces with the maximal number of extremal points include $m$-dimensional rectangles and the quotients of flat tori by reflections across a point.
(This action has $2^m$ isolated fixed points; each corresponds to an extremal point in the quotient space $\spc{A}=\TT^m/\ZZ_2$.)
Nina Lebedeva has proved \cite{lebedeva2015} that \textit{every $m$-dimensional $\Alex0$ space with $2^m$ extremal points is a quotient of Euclidean space by a crystallographic action}.

The extremal subsets of Alexandrov space were brifly discussed in \ref{sec:bry-remarks}.
The following definition is more relevant to isometric group actions.

A closed subset $E$ in a finite-dimensional Alexandrov space is called
\index{extremal set}\emph{extremal} if $\mangle\hinge pxy\z\le \tfrac\pi2$ for any $x\notin E$ and $p\in E$ such that $\dist{x}{p}{}$ takes a minimal value.
An extremal set is called \index{minimal extremal set}\emph{minimal} if it contains no proper extremal subsets.

For example, the whole space and the empty set are extremal.
Also, every vertex, edge, or face (as well as their unions) of the cube is an extremal subset of the cube.
Vertices of the cube are its only minimal extremal subsets.

Counting maximal finite subgroups in a crystallographic group $\Gamma$ (up to conjugation) is equivalent to counting the minimal extremal subsets in the quotient space $\spc{A}=\EE^m/\Gamma$.
So, \ref{prop:2m} would follow from the next conjecture.

\begin{thm}{Conjecture}
Any $m$-dimensional compact $\Alex0$ space has at most $2^m$ minimal extremal subset.
\end{thm}

Let us mention another related conjecture.
An extremal set is called \index{primitive extremal set}\emph{primitive} if it contains no proper extremal subsets with nonempty relative interior.
For example, each face of $m$-dimensional cube is its primitive extremal subset;
therefore the cube has exactly $3^m$ primitive extremal subset, including the empty set and the whole cube.

\begin{thm}{Conjecture}
Any $m$-dimensional compact $\Alex0$ space has at most $3^m$ minimal extremal subset.
\end{thm}

Some crude estimates on number of extremal subsets follow from the idea in Gromov's Betti number theorem \ref{thm:betti}.
