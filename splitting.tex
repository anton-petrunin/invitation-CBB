\chapter{Line splitting}\label{chap:splitting}

\section{Busemann function}

A \index{half-line}\emph{half-line} is a distance-preserving map
from $\RR_{\ge0}=[0,\infty)$ 
to a metric space.
In other words, a half-line is a geodesic defined on the real half-line $\RR_{\ge0}$.
If $\gamma\:[0,\infty)\to \spc{X}$ is a half-line,
then the limit 
\[\bus_\gamma(x)=\lim_{t\to\infty}\dist{\gamma(t)}{x}{}- t\eqlbl{eq:def:busemann*}\]
is called the \index{Busemann function}\emph{Busemann function} of $\gamma$.

The Busemann function $\bus_\gamma$ mimics behavior of the distance function from the ideal point of $\gamma$.

\begin{thm}{Proposition}\label{prop:busemann}
For any half-line $\gamma$ in a metric space $\spc{X}$,
its Busemann function $\bus_\gamma\:\spc{X}\to \RR$ 
is defined.
Moreover, $\bus_\gamma$ is $1$-Lipschitz and $\bus_\gamma\circ\gamma(t)+t= 0$ for any $t$.

\end{thm}

\parit{Proof.}
By the triangle inequality, the function
\[t\mapsto\dist{\gamma(t)}{x}{}- t\] 
is nonincreasing for any fixed $x$.  

Since $t=\dist{\gamma(0)}{\gamma(t)}{}$, the triangle inequality implies that
\[\dist{\gamma(t)}{x}{}- t\ge-\dist{\gamma(0)}{x}{}.\]
Thus the limit in \ref{eq:def:busemann*} is defined,
and it is 1-Lipschitz as a limit of 1-Lipschitz functions.
The last statement follows since 
$\dist{\gamma(t)}{\gamma(t_0)}{}\z=t-t_0$ for all large~$t$.
\qeds

\begin{thm}{Exercise}\label{ex:busemann-CBB}
Any Busemann function on a geodesic $\Alex{0}$ space is concave.
\end{thm}

\section{Splitting theorem}

A \index{line}\emph{line} is a distance-preserving map
from $\RR$ to a metric space.
In other words, a line is a geodesic defined on the real line $\RR$.

\begin{thm}{Exercise}\label{ex:bus+bus}
Let $\gamma$ be a line in a metric space $\spc{X}$.
Show that for any point $x$ we have
\[\bus_+(x)+\bus_-(x)\ge 0\]
where, $\bus_+$ and $\bus_-$, are the Busemann functions asociated with half-lines $\gamma:[0,\infty)\to \spc{L}$ and $\gamma:(-\infty,0]\to \spc{L}$ respectively.
\end{thm}


Let $\spc{X}$ be a metric space and $A,B\subset \spc{X}$.
We will write 
\[\spc{X}=A\oplus B\]\index{$A\oplus B$}
if there are projections $\proj_A\:\spc{X}\to A$ 
and 
$\proj_B\:\spc{X}\to B$
such that 
\[\dist[2]{x}{y}{}=\dist[2]{\proj_A(x)}{\proj_A(y)}{}+\dist[2]{\proj_B(x)}{\proj_B(y)}{}\]
for any two points $x,y\in \spc{X}$.

Note that if 
\[\spc{X}=A\oplus B\]
then 
\begin{itemize}
\item $A$ intersects $B$ at a single point,
\item both sets $A$ and $B$ are convex sets in $\spc{X}$;
the latter means that any geodesic with the ends in $A$ (or $B$) lies in $A$ (or $B$). 
\end{itemize}

\begin{thm}{Line splitting theorem}\label{thm:splitting}
Let $\spc{L}$  be a complete geodesic $\Alex{0}$ space
and $\gamma$ be a line in $\spc{L}$. 
Then 
\[\spc{L}=\spc{L}'\oplus \gamma(\RR)\]
for some subset $\spc{L}'\subset \spc{L}$.
\end{thm}

Before going into the proof, let us state a corollary of the theorem.

\begin{thm}{Corollary}\label{cor:splitting}
Let $\spc{L}$ be a complete geodesic $\Alex{0}$ space. 
Then there is an isometric splitting
\[
\spc{L}=\spc{L}'\oplus H
\]
where $H\subset \spc{L}$ is a subset isometric to a Hilbert space, and $\spc{L}'\subset \spc{L}$ is a convex subset that contains no line. 
\end{thm}

The following lemma is closely relevant to the first distance estimate (\ref{thm:dist-est}); its proof goes along the same lines.

\begin{thm}{Lemma}\label{lem:dist-estimate}
Suppose $f\:\spc{L}\to\RR$ be a concave 1-Lipschitz function on a geodesic $\CBB(0)$ space $\spc{L}$.
Consider two $f$-gradient curves $\alpha$ and~$\beta$.
Then for any $t, s\ge 0$ we have
\begin{align*}
&\dist[2]{\alpha(s)}{\beta(t)}{}
\le 
\dist[2]{p}{q}{}+
2\cdot(f(p)-f(q))\cdot(s-t)+ (s-t)^2,
\end{align*}
where $p=\alpha(0)$ and $q=\beta(0)$.
\end{thm}

\parit{Proof.}
Since $f$ is 1-Lipschitz, $|\nabla f|\le1$.
Therefore 
\[f\circ\beta(t)\le f(q)+t\]
for any $t\ge0$.

Set $\ell(t)=\dist{p}{\beta(t)}{}$.
Applying \ref{eq:fist-var-inq+}, we get
\begin{align*}
(\ell^2)^+(t)
&\le 2\cdot \left(f\circ\beta(t)-f(p)\right)\le 
\\
&\le2\cdot\left(f(q)+t-f(p)\right).
\end{align*}
Therefore 
\[\ell^2(t)-\ell^2(0)\le 2\cdot\left(f(q)-f(p)\right)\cdot t + t^2.\]
It proves the needed inequality in case $s=0$.
Combining it with the first distance estimate (\ref{thm:dist-est}), we get the result in case $s\le t$.
The case $s\ge t$ follows by switching the roles of $s$ and $t$.
\qeds


\parit{Proof of \ref{thm:splitting}.} Consider two Busemann functions, $\bus_+$ and $\bus_-$, asociated with half-lines $\gamma:[0,\infty)\to \spc{L}$ and $\gamma:(-\infty,0]\to \spc{L}$ respectively; that is,
\[
\bus_\pm(x)
\df
\lim_{t\to\infty}\dist{\gamma(\pm t)}{x}{}- t.
\]
According to \ref{ex:busemann-CBB}, 
both functions $\bus_\pm$ are concave.

By \ref{ex:bus+bus}, $\bus_+(x)+\bus_-(x)\ge0$ for any $x\in \spc{L}$.
On the other hand, by \ref{comp-kappa}, 
$f(t)=\distfun_x^2\circ\gamma(t)$ 
is $2$-concave.
In particular, $f(t)\le t^2+at+b$ for some constants $a,b\in\RR$. 
Passing to the limit as $t\to\pm\infty$, we have $\bus_+(x)+\bus_-(x)\le0$.
Hence
\[
\bus_+(x)+\bus_-(x)= 0
\]
for any $x\in \spc{L}$.
In particular, the functions $\bus_\pm$ are \index{affine function}\emph{affine};
that is, they are convex and concave at the same time.

Note that for any $x$,
\begin{align*}
|\nabla_x \bus_\pm|
&=\sup\set{\dd_x\bus_\pm(\xi)}{\xi\in\Sigma_x}=
\\
&=\sup\set{-\dd_x\bus_\mp(\xi)}{\xi\in\Sigma_x}\equiv
\\
&\equiv1.
\end{align*}

Observe that $\alpha$ is a $\bus_\pm$-gradient curve
if and only if $\alpha$ is a geodesic such that $(\bus_\pm\circ\alpha)^+=1$.
Indeed, if $\alpha$ is a geodesic, then $(\bus_\pm\circ\alpha)^+\le 1$ and the equality holds only if $\nabla_\alpha\bus_\pm=\alpha^+$.
Now suppose $\nabla_\alpha\bus_\pm=\alpha^+$.
Then $|\alpha^+|\le 1$ and $(\bus_\pm\circ\alpha)^+=1$; therefore 
\begin{align*}
|t_0-t_1|&\ge \dist{\alpha(t_0)}{\alpha(t_1)}{}\ge
\\
&\ge|\bus_\pm\circ\alpha(t_0)-\bus_\pm\circ\alpha(t_1)=
\\
&=|t_0-t_1|.
\end{align*}

It follows that for any $t>0$, the $\bus_\pm$-gradient flows commute;
that is, 
\[\GF_{\bus_+}^t\circ\GF_{\bus_-}^t=\id_\spc{L}.\]
Setting
\[\GF^t=\left[\begin{matrix}
\GF_{\bus_+}^t&\hbox{if}\ t\ge0\\
\GF_{\bus_-}^t&\hbox{if}\ t\le0
               \end{matrix}\right.\]
defines an $\RR$-action on $\spc{L}$.

Consider the level set $\spc{L}'=\bus_+^{-1}(0)=\bus_-^{-1}(0)$;
it is a closed convex subset of $\spc{L}$, and therefore forms an Alexandrov space.
Consider the map $h\:\spc{L}'\times \RR\to \spc{L}$ defined by $h\:(x,t)\mapsto \GF^t(x)$.
Note that $h$ is onto.
Applying Lemma \ref{lem:dist-estimate} for $\GF_{\bus_+}^t$ and $\GF_{\bus_-}^t$ shows that $h$ is short and non-contracting at the same time; that is, $h$ is an isometry.
\qeds

Recall that according our definition the real line $\RR$ is a $\CBB(1)$ space.
However most of geodesic $\CBB(1)$ spaces have diameter at most $\pi$;
see \ref{ex:RisCBB(1)}.

\begin{thm}{Exercise}\label{ex:cone-CBB}
Suppose $\spc{X}$ is a complete geodesic space.
Show that $\Cone\spc{X}$ is $\CBB(0)$ if and only if $\spc{X}\iso\Susp^k\spc{Y}$
for some integer $k\ge 0$ and a geodesic $\CBB(1)$ space $\spc{Y}$ such that $\dist{x}{y}{}<\pi$ for any $x,y\in\spc{Y}$.

???

(For example, if $\spc{Y}=\emptyset$, then $\spc{X}\iso \mathbb{S}^k$;
if $\spc{Y}$ is a one-point space, then $\spc{X}$ is a hemisphere.)
\end{thm}

\section{Comments}

The splitting theorem has an interesting history that starts with Stefan Cohn-Vossen \cite{cohn-vossen_line}.
Our proof is based on the idea of Jeff Cheeger and Detlef Gromoll \cite{cheeger-gromoll-split}.

\chapter{Dimension}\label{chap:dim}

\section{Polar vectors}

Here we give a corollary of \ref{ex:convergence-grad}.
It will be used to prove basic properties of the tangent space.


\begin{thm}{Anti-sum lemma}\label{lem:minus-sum} 
Let $\spc{L}$ be a complete geodesic $\CBB$ space and $p\in \spc{L}$.

Given two vectors $u,v\in \T_p$, there is a unique vector $w\in \T_p$ such that
\[\langle u,x\rangle +\langle v,x\rangle +\langle w,x\rangle \ge 0\]
for any $x\in \T_p$, and
\[\langle u,w\rangle +\langle v,w\rangle +\langle w,w\rangle =0.\]

\end{thm}

\begin{thm}{Exercise}\label{ex:|antisum|}
Suppose $u,v, w\in \T_p$ are as in \ref{lem:minus-sum}.
Show that 
\[|w|^2\le |u|^2+|v|^2+2\cdot\langle u,v\rangle.\]

\end{thm}


If $\T_p$ were geodesic, then the lemma would follow from the existence  of the gradient, applied to the function $\T_p\to \RR$ defined by $x\mapsto -(\langle u,x\rangle +\langle v,x\rangle )$ which is concave.
However, the tangent space $\T_p$ might fail to be geodesic; see  Halbeisen's example \cite{alexander-kapovitch-petrunin2024}.


Applying the above lemma for $u=v$, we have the following statement.

\begin{thm}{Existence of polar vector}\label{cor:polar}
Let $\spc{L}$ be a complete geodesic $\CBB$ space 
and $p\in \spc{L}$. 
Given a vector $u\in \T_p$,  there is a unique vector $u^*\in\T_p$ such that $\langle u^*,u^*\rangle +\langle u,u^*\rangle = 0$ and
$u^*$ is \index{polar vectors}\emph{polar} to $u$;
that is,
 $\langle u^*,x\rangle +\langle u,x\rangle \ge 0$ for any $x\in \T_p$.

In particular, for any vector $u\in \T_p$ there is a polar vector $u^*\in\T_p$ such that
$|u^*|\le |u|$.
\end{thm}

\begin{thm}{Example}
Let $\spc{L}$ be the upper half plane in $\EE^2$;
that is, $\spc{L}\z=\{(x,y)\in \EE^2\mid y\ge 0\}$.
It is a complete geodesic $\Alex{0}$ space.
For $p=0$, the tangent space $\T_p$ can be canonically identified with $\spc{L}$.
If $y>0$, then $u=(x,y)\in \T_p$ has many polar vectors;
it includes $u^*=(-x,0)$ which is provided by \ref{cor:polar},
but the vector $w=(-x,y)$ is polar as well.
\end{thm}

\parit{Proof of \ref{lem:minus-sum}.}
By \ref{ex:first-var-CBB}, we can choose two sequences of points $a_n,b_n$ such that 
\begin{align*}
\dd_p\distfun_{a_n}(w)&=-\langle\dir{p}{a_n},w\rangle
\\
\dd_p\distfun_{b_n}(w)&=-\langle\dir{p}{b_n},w\rangle
\end{align*}
for any $w\in\T_p$ and $\dir{p}{a_n}\to u/|u|$, $\dir{p}{b_n}\to v/|v|$ as $n\to \infty$

Consider a sequence of functions 
\[f_n=|u|\cdot\distfun_{a_n}+|v|\cdot\distfun_{b_n}.\]
Note that 
\[(\dd_pf_n)(x)=-|u|\cdot\langle \dir{p}{a_n},x\rangle -|v|\cdot\langle \dir{p}{b_n},x\rangle .\]
Thus we have the following uniform convergence for $x\in\Sigma_p$:
\[(\dd_pf_n)(x)\to-\langle u,x\rangle -\langle v,x\rangle \]
as $n\to\infty$,
According to \ref{ex:convergence-grad}, 
the sequence $\nabla_pf_n$ converges.
Let 
\[w=\lim_{n\to\infty}\nabla_pf_n.\]
By the definition of gradient,
\[\begin{aligned}
\langle w,w\rangle &=\lim_{n\to\infty}\langle \nabla_pf_n,\nabla_pf_n\rangle =
&&&%right side
\langle w,x\rangle &=\lim_{n\to\infty}\langle \nabla_pf_n,x\rangle \ge
\\%second line
&=\lim_{n\to\infty}(\dd_p f_n)(\nabla_p f_n)
=
&&&%second line right side
&\ge
\lim_{n\to\infty}(\dd_pf_n)(x)
=
\\%line 3
&=-\langle u,w\rangle -\langle v,w\rangle ,
&&&%line 3 right side
&=-\langle u,x\rangle -\langle v,x\rangle .
\end{aligned}\]
\qedsf












\section{Linear subspace of tangent space}

\begin{thm}{Definition}\label{def:opp+Lin}
Let $\spc{L}$ be a complete geodesic $\Alex{\kappa}$ space, $p\in \spc{L}$ and $u,v\in\T_p$.
We say that vectors $u$ and $v$ are \index{opposite vectors}\emph{opposite}\label{def:opposite:page} to each other, (briefly, $u+v=0$) if $|u|=|v|=0$ or $\mangle(u,v)=\pi$ and $|u|=|v|$.

The subcone
\[\Lin_p=\set{v\in\T_p}{\exists\ w\in\T_p\quad \text{such that}\quad w+v=0}\]
will be called the \index{linear subspace}\emph{linear subcone} of $\T_p$.
\end{thm}

\begin{thm}{Proposition}\label{prop:opposite}
Let $\spc{L}$ be a complete geodesic $\CBB$ space and $p\in \spc{L}$.
Given two vectors $u,v\in\T_p$, the following statements are equivalent:
\begin{subthm}{opposite} $u+v=0$;
\end{subthm}
\begin{subthm}{<x,u>} $\langle u,x\rangle +\langle v,x\rangle =0$ for any $x\in\T_p$;
\end{subthm}
\begin{subthm}{<xi,u>} $\langle u,\xi\rangle +\langle v,\xi\rangle =0$ for any $\xi\in\Sigma_p$.
\end{subthm}
\end{thm}

\parit{Proof.}
The equivalence  \ref{SHORT.<x,u>}$\Leftrightarrow$\ref{SHORT.<xi,u>} is trivial.

The condition $u+v=0$ is equivalent to 
\[\langle u,u\rangle =-\langle u,v\rangle =\langle v,v\rangle ;\]
thus 
\ref{SHORT.<x,u>}$\Rightarrow$\ref{SHORT.opposite}.

Recall that $\T_p$ is $\CBB(0)$.
Note that the hinges $\hinge 0ux$ and $\hinge 0vx$ are adjacent.
By \ref{ex:adjacent-CBB}, $\mangle\hinge 0ux+\mangle\hinge 0vx=0$;
hence \ref{SHORT.opposite}$\Rightarrow$\ref{SHORT.<x,u>}.
\qeds

\begin{thm}{Exercise}\label{prop:two-opp}
Let $\spc{L}$  be a complete geodesic $\CBB$ space and $p\in \spc{L}$.
Then for any three vectors $u,v,w\in\T_p$, if $u+v=0$ and $u+ w=0$ then $v=w$.
\end{thm}

Let $u\in \Lin_p$; that is, $u+v=0$ for some $v\in\T_p$.
Given $s<0$, let 
\[s\cdot u\df (-s)\cdot v.\]
So we can multiply any vector in $\Lin_p$ by any real number (positive and negative).
By \ref{prop:two-opp}, this multiplication is uniquely defined;
by \ref{prop:opposite}; we have identity
\[\langle -v,x\rangle=-\langle v,x\rangle;\]
later we will see that it extends to a linear structure on $\Lin_p$.

\begin{thm}{Exercise}\label{ex:3<,>=0}
Suppose $u,v,w\in\T_p$ are as in \ref{lem:minus-sum}.
Show that
\[\langle u,x\rangle +\langle v,x\rangle +\langle w,x\rangle = 0\]
for any $x\in \Lin_p$.
\end{thm}

\begin{thm}{Exercise}\label{ex:-u}
Let $\spc{L}$ be a complete geodesic $\CBB$ space,
$p\in \spc{L}$ and $u\in \T_p$.
Suppose $u^*\in \T_p$ is provided by \ref{cor:polar};
that is, 
\[\langle u^*,u^*\rangle +\langle u,u^*\rangle = 0
\quad\text{and}\quad
\langle u^*,x\rangle +\langle u,x\rangle \ge 0
\]
for any $x\in \T_p$.
Show that 
\[u=-u^*\quad\Longleftrightarrow\quad|u|=|u^*|.\]
\end{thm}


\begin{thm}{Theorem}\label{thm:lin-subcone}
Let $p$ be a point in a complete geodesic $\Alex{\kappa}$. 
Then $\Lin_p$ is isometric to a Hilbert space.
\end{thm}

\parit{Proof.}
Note that $\Lin_p$ is a closed subset of $\T_p$;
in particular, it is complete.

If any two vectors in $\Lin_p$ can be connected by a geodesic in $\Lin_p$,
then the statement follows from the splitting theorem (\ref{thm:splitting}).
By Menger's lemma (\ref{lem:mid>geod}), it is sufficient to show that for any two vectors $x,y\in\Lin_p$
there is a midpoint $w\in \Lin_p$.

Choose $w\in \T_p$ to be the anti-sum of $u=-\tfrac{1}{2}\cdot x$ and $v=-\tfrac{1}{2}\cdot y$;
see \ref{lem:minus-sum}.
By \ref{ex:|antisum|} and \ref{ex:3<,>=0},
\begin{align*}
|w|^2&\le \tfrac14\cdot |x|^2+\tfrac14\cdot|y|^2+\tfrac12\cdot\langle x,y\rangle,
\\
\langle w,x\rangle&= \tfrac12\cdot|x|^2+\tfrac12\cdot\langle x,y\rangle,
\\
\langle w,y\rangle&= \tfrac12\cdot|y|^2+\tfrac12\cdot\langle x,y\rangle,
\end{align*}
It follows that 
\begin{align*}
|x-w|^2
&= |x|^2+|w|^2-2\cdot\langle w,x\rangle\le
\\
&\le \tfrac14\cdot |x|^2+\tfrac14\cdot|y|^2-\tfrac12\cdot\langle x,y\rangle=
\\
&=\tfrac14\cdot|x-y|^2.
\end{align*}
That is, $|x-w|\le \tfrac12\cdot|x-y|$, and similarly $|y-w|\le \tfrac12\cdot|x-y|$.
Therefore $w$ is a midpoint of $x$ and $y$.
In addition we get equality 
\[|w|^2= \tfrac14\cdot |x|^2+\tfrac14\cdot|y|^2+\tfrac12\cdot\langle x,y\rangle.\]

It remains to show that $w\in\Lin_p$.
Let $w^*$ be the polar vector provided by \ref{cor:polar}.
Note that 
\[|w^*|\le |w|,
\quad
\langle w^*,x\rangle+\langle w,x\rangle=0,
\quad
\langle w^*,y\rangle+\langle w,y\rangle=0.
\]
The same calculation as above shows that $w^*$ is a midpoint of $-x$ and $-y$ and 
\[|w^*|^2= \tfrac14\cdot |x|^2+\tfrac14\cdot|y|^2+\tfrac12\cdot\langle x,y\rangle=|w|^2.\]
By \ref{ex:-u}, $w=-w^*$;
hence $w\in\Lin_p$.
\qeds



\begin{thm}{Exercise}\label{ex:grad-dist}
Let $p$ be a point in a complete geodesic $\CBB(\kappa)$ space $\spc{L}$ and $f=\distfun_p$.
Denote by $S$ the subset of points $x\in\spc{L}$ such that $|\nabla_xf|=1$.

\begin{subthm}{ex:grad-dist:G-delta}
Show that $S$ is a dense G-delta set.
\end{subthm}

\begin{subthm}{ex:grad-dist:lin}
Show that 
\[\nabla_xf+\dir xp=0\]
for any 
$x\in S$;
in particular, $\dir xp\in \Lin_x$.
\end{subthm}

\begin{subthm}{ex:grad-dist:|grad|=1}
Show that if $|\nabla_xf|=1$, then $\dd_xf(w)= \langle\nabla_xf,w\rangle$ for any $w\in \T_x$.
\end{subthm}


\end{thm}

Note that \ref{ex:grad-dist:lin} implies the following.

\begin{thm}{Corollary}\label{cor:euclid-subcone}
Given a countable set of points $X$ in a complete geodesic $\Alex{\kappa}$ space $\spc{L}$
there is a G-delta dense set $S\subset\spc{L}$
such that 
$\dir sx\in \Lin_s$
for any $s\in S$ and $x\in X$.
\end{thm}

\section{Linear dimension}


Suppose $\spc{L}$ is a complete geodesic $\CBB(\kappa)$ space.
Let us define its \emph{linear dimension} $\LinDim \spc{L}$ as the least upper bound on integers $m$ such that 
the Euclidean space $\EE^m$ is isometric to a subspace of the tangent space $\T_p\spc{L}$ at some point $p\in \spc{L}$.
If not stated otherwise, dimension of a $\CBB$ space is its linear dimension.

\begin{thm}{(\textit{n}+1)-comparison}
Let $\spc{L}$ be a complete geodesic $\CBB(0)$ space.
Then for any finite set of points $p,x_1,\dots,x_n\in \spc{L}$, there is a model configuration 
$\tilde p,\tilde x_1,\dots,\tilde x_n\in \EE^m$ such that 
\[|\tilde p-\tilde x_i|_{\EE^m}=| p- x_i|_{\spc{L}}
\quad\text{and}\quad
|\tilde x_i-\tilde x_j|_{\EE^m}\ge |x_i- x_j|_{\spc{L}}\]
for any $i$ and $j$.
Moreover, we can assume that $m\le \LinDim\spc{L}$. 
\end{thm}

\parit{Proof.}
By \ref{cor:euclid-subcone}, we can choose a point $p'$ arbitrarily close to $p$ so that 
$\Lin_{p'}\ni \dir{p'}{x_i}$ for any $i$.
Let us identify $\EE^m$ with a subspace of $\Lin_{p'}$ spanned by $\dir{p'}{x_1},\dots,\dir{p'}{x_n}$.
Note that $m\le \LinDim\spc{L}$.

Set $\tilde p'=0\in \EE^m$ and $\tilde x_i=\dist{p'}{x_n}{}\cdot\dir{p'}{x_n}\in \EE^m$ for every $i$.
Note that 
\[|\tilde p'-\tilde x_i|_{\EE^m}=| p'- x_i|_{\spc{L}}\]
for every $i$.
Applying the comparison $\mangle\hinge {p'}{x_i}{x_j}\ge \angk {p'}{x_i}{x_j}$, we get
\[|\tilde x_i-\tilde x_j|_{\EE^m}\ge |x_i- x_j|_{\spc{L}}\]
for any $i$ and $j$.
Passing to a limit configuration as $p'\to p$ we get the result.
\qeds

\begin{thm}{Exercise}\label{ex:tangent=Em}
Let $\spc{L}$ is a complete geodesic $\CBB(0)$ space.
Suppose $\LinDim\spc{L}=m<\infty$.
Show that $\T_p\spc{L}\iso \EE^m$ for a G-delta dense set of points $p\in\spc{L}$.
\end{thm}

\begin{thm}{Exercise}\label{ex:dim=1}
Show that a 1-dimensional complete geodesic $\CBB(\kappa)$ space is homeomorphic to a 1-dimensional manifold, possibly with nonempty boundary.
\end{thm}


\begin{thm}{Exercise}\label{ex:resporka}
Let $\spc{L}$ be a complete geodesic $\CBB(0)$ space.

Show that $\LinDim \spc{L}\ge m$ if and only if for some $m+2$ points $p$, $x_0,\z\dots, x_{m}\in \spc{L}$
we have
\[\angk p{x_i}{x_j}>\tfrac\pi2\]
for any pair $i\ne j$.%
\footnote{\textit{If $m=\LinDim \spc{L}$ then the map $q\mapsto (\dist{x_1}{q}{},\dots,\dist{x_m}{q}{})$ induces a bi-Lipschitz embedding of a neighborhood of $p$ to $\EE^m$.}
(We mention it without proof, altho it is not hard to prove.)}
\end{thm}

\section{Space of directions}

A metric space $\spc{X}$ will be called $\ell$-geodesic 
if any two points $x,y\in\spc{X}$ such that $\dist{x}{y}{}<\ell$ can be connected by a geodesic.
Note that geodesic spaces are $\infty$-geodesic.

\begin{thm}{Theorem}\label{thm:finite-space-of-directions}
Let $\spc{L}$ be a finite-dimensional complete geodesic $\CBB(\kappa)$ space.
Then for any point $p\in \spc{L}$, its space of directions $\Sigma_p$ is a compact $\pi$-geodesic $\CBB(1)$ space.
\end{thm}


\begin{thm}{Exercise}\label{ex:finite-tan}
Prove the following in the assumptions of \ref{thm:finite-space-of-directions}.
\begin{subthm}{}
The tangent space $\T_p$ is a proper geodesic space.
\end{subthm}

\begin{subthm}{ex:finite-space-of-directions-dim}
$\dim\Sigma_p=\dim\spc{L}-1$.
\end{subthm}

\begin{subthm}{}
If $\dim \spc{L}>1$, then $\Sigma_p$ is geodesic.
\end{subthm}


\end{thm}

Using \ref{ex:finite-space-of-directions-dim}, one can prove results for all finite-dimensional complete geodesic $\CBB(\kappa)$ spaces via induction on its dimension.
Such proofs will be indicated below.

\parit{Proof.}
Note that \ref{prop:Tan-is-CBB(0)} and \ref{ex:cone+CBB} imply that $\Sigma_p$ is $\CBB(1)$.

\parit{Compactness.}
Choose $\eps>0$; suppose $\spc{L}$ is $m$-dimensional.
Assume can choose $n$ directions $\xi_1,\dots, \xi_n\in \Sigma_p$ such that $\mangle(\xi_i,\xi_j)\z>\eps$ for any $i\ne j$.
Without loss of generality, we may assume that each direction is geodesic;
that is, there is a point $x_i\in \spc{L}$ such that $\xi_i=\dir p{x_i}$.

Choose $y_i\in [px_i]$ such that $\dist{p}{y_i}{}=r$ for each $i$ and small fixed $r>0$.
Since $r$ is small, we can assume that $\angk p{y_i}{y_j}>\eps$ for any $i\ne j$.
By \ref{cor:euclid-subcone}, we can choose $p'$ arbitrarily close to $p$ such that $\dir{p'}{y_i}\in \Lin{p'}$ for any $i$.
Since  $\dist{p'}{p}{}$ is small, $\angk {p'}{y_i}{y_j}>\eps$ for any $i\ne j$.
By comparison, 
\[\mangle \hinge{p'}{y_i}{y_j}>\eps.\]
Therefore $n\le \pack_\eps\SSS^{m-1}$.

Since $\SSS^{m-1}$ is compact, $\pack_\eps\SSS^{m-1}<\infty$.
By the definition, the space of directions $\Sigma_p$ is complete. 
Applying \ref{ex:pack-net}, we get that  $\Sigma_p$ is compact.

\parit{Geodesicness.}
Now we will show that \textit{if $\Sigma_p$ is compact, then it is $\pi$-geodesic};
we will not use the finiteness of dimension directly.

Choose two geodesic directions $\xi=\dir px$ and $\zeta=\dir py$;
let 
\[\alpha\z=\tfrac12\cdot \mangle \hinge pxy=\tfrac12\cdot \dist{\xi}{\zeta}{\Sigma_p}.\]

Suppose $\alpha<\pi$.
Let us show that it is sufficient to construct an \emph{almost midpoint} $\mu\z=\dir pz$ of $\xi$ and $\zeta$ in $\Sigma_p$;
that is, we need to show that for any $\eps>0$ there is a geodesic $[pz]$ such that
\[\mangle\hinge pxz\le \alpha+\eps
\quad\text{and}\quad
\mangle\hinge pyz\le \alpha+\eps.\]
Indeed, once it is done, the compactness of $\Sigma_p$ can be used to get an actual midpoint for any two directions in $\Sigma_p$.
After that Menger's lemma (\ref{lem:mid>geod}) will finish the proof.

Choose a sequence of small positive numbers $r_n\to0$
Consider sequnces $x_n\z\in [px]$ and $y_n\z\in [py]$ such that $\dist{p}{x_n}{}=\dist{p}{y_n}{}=r_n$.
Let $m_n$ be a midpoint of $[x_n\,y_n]$.
%??? we use here that the directions $\xi=\dir px$ and $\zeta=\dir py$ are not opposite???

Since $\Sigma_p$ is compact, we can pass to a sequence of $r_n$ such that 
$\dir{p}{m_n}$ converges;
denote its limit by $\mu$.
Choose a geodesic $[pz]$ that runs at small angle from $\mu$.
Let us show that $\dir pz$ is the needed almost midpoint.

Evidently, $\angk p{x_n}{m_n}=\angk p{y_n}{m_n}$.
By \ref{ex:alex-lemma-cat}, we have
\[\angk p{x_n}{m_n}+\angk p{y_n}{m_n}\le \angk p{x_n}{y_n}.\]

Let $z_n\in [pz]$ be the point such that $\dist{p}{z_n}{}=\dist{p}{m_n}{}$.
By construction, for all large $n$, we have $\mangle\hinge pz{m_n}\approx0$  with arbitrary small given error.
By comparison, the value $\frac{\dist{z_n}{m_n}{}}{\dist{p}{z_n}{}}$ can be assumed to be arbitrary small for all large $n$.
Applying this observation and the definition of angle measure, we also have the following approximations
\begin{align*}
\angk p{x_n}{y_n}&\approx \mangle\hinge p{x_n}{y_n},
\\
\angk p{x_n}{m_n}\approx\angk p{x_n}{z_n}&\approx\mangle\hinge p{x_n}{z_n},
\\
\angk p{m_n}{y_n}\approx\angk p{z_n}{y_n}&\approx\mangle\hinge p{z_n}{y_n},
\end{align*}
again, with arbitrary given error and all large $n$.
It follows that $\dir pz$ is an almost midpoint of $\dir px$ and $\dir py$, as required.
\qeds

Notice that $\CBB$ comparison gives lower bounds on $\mangle\hinge pxz$ and $\mangle\hinge pyz$, but in the proof we needed upper bounds which were obtained by from the definition of angle measure and compactness of space of directions.

\begin{thm}{Exercise}\label{ex:GHto-tangent}
Let $p$ be a point in a finite-dimensional complete geodesic $\CBB(\kappa)$ space $\spc{L}$.
Show that $(\lambda\cdot \spc{L},p)\to (\T_p,0)$ in the sense of pointed Gromov--Hausdorff convergence.
\end{thm}

\section{Comments}

Corollary \ref{cor:euclid-subcone} is the key to the proof that \textit{all reasonable definitions of dimension give the same result on complete geodesic $\CBB$ spaces}.
More precisely, we have the following theorem \cite[15.16]{alexander-kapovitch-petrunin2024}.

\begin{thm}{Theorem}\label{thm:dim=dim}
For any complete length $\CBB(\kappa)$ space $\spc{L}$, we have
\[\LinDim \spc{L}=\TopDim \spc{L}=\HausDim \spc{L},\]
where $\TopDim$ and $\HausDim$ stands for \emph{Lebesgue coverning dimension} and \emph{Hausdorff dimension}, respectively.
\end{thm}

\begin{thm}{Open question}
Let $p$ be a point in a geodesic $\CBB$ space $\spc{L}$.
Suppose that $0\ne v\in \Lin_p$
Is it true that the tangent space $\T_p$ splits in the direction of $v$?
\end{thm}

Halbeisen's example \cite{alexander-kapovitch-petrunin2024,halbeisen} shows that compactness of space of directions is essential in the proof that space of directions is $\pi$-geodesic (see \ref{thm:finite-space-of-directions}).

\begin{thm}{Open question}\label{open:Halb-proper}
Let $\spc{L}$ be a proper length $\Alex\kappa$ space.
Is it true that for any $p\in \spc{L}$, the tangent space $\T_p$ is a length space?
\end{thm}
