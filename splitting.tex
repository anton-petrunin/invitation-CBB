%%!TEX root = the-splitting.tex
\chapter{Line splitting}\label{chap:splitting}

In this lecture, we prove the line splitting theorem and apply it to study tangent spaces of an Alexandrov space.

\section{Busemann function}

A \index{half-line}\emph{half-line}
%??? Вить, можешь поменять везде half-line на ray --- я ни за ни против.
is a distance-preserving map
from $\RR_{\ge0}=[0,\infty)$ 
to a metric space.
In other words, a half-line is a geodesic defined on the real half-line $\RR_{\ge0}$.

If $\gamma\:[0,\infty)\to \spc{X}$ is a half-line,
then the limit 
\[\bus_\gamma(x)=\lim_{t\to\infty}\dist{\gamma(t)}{x}{}- t\eqlbl{eq:def:busemann*}\]
is called the \index{Busemann function}\emph{Busemann function} of $\gamma$.

The Busemann function $\bus_\gamma$ mimics behavior of the distance function from the ideal point of $\gamma$.

\begin{thm}{Proposition}\label{prop:busemann}
For any half-line $\gamma$ in a metric space $\spc{X}$,
its Busemann function $\bus_\gamma\:\spc{X}\to \RR$ 
is defined.
Moreover, $\bus_\gamma$ is $1$-Lipschitz and $\bus_\gamma (\gamma(t))=-t$ for any $t$.

\end{thm}

\parit{Proof.}
Since $t=\dist{\gamma(0)}{\gamma(t)}{}$, the triangle inequality implies that, the function
\[t\mapsto\dist{\gamma(t)}{x}{}- t\] 
is nonincreasing, and 
\[\dist{\gamma(t)}{x}{}- t\ge-\dist{\gamma(0)}{x}{}\]
for any $x\in \spc{X}$.
Therefore, the limit in \ref{eq:def:busemann*} is defined,
and it is 1-Lipschitz as a limit of 1-Lipschitz functions.
The last statement follows since 
$\dist{\gamma(t)}{\gamma(t_0)}{}\z=t-t_0$ for all large~$t$.
\qeds

\begin{thm}{Exercise}\label{ex:busemann-CBB}
Any Busemann function on an $\Alex0$ space is concave.
\end{thm}

\section{Splitting theorem}

A \index{line}\emph{line} is a distance-preserving map
from $\RR$ to a metric space.
In other words, a line is a geodesic defined on the real line $\RR$.

\begin{thm}{Exercise}\label{ex:bus+bus}
Let $\gamma$ be a line in a metric space $\spc{X}$.
Show that for any point $x$ we have
\[\bus_+(x)+\bus_-(x)\ge 0\]
where, $\bus_+$ and $\bus_-$, are the Busemann functions asociated with half-lines $\gamma:[0,\infty)\to \spc{A}$ and $\gamma:(-\infty,0]\to \spc{A}$ respectively.
\end{thm}


Let $\spc{X}$ be a metric space and $A,B\subset \spc{X}$.
A metric space $\spc{X}$ is a \index{direct sum}\emph{direct sum} of its two $A$ and $B$,
or briefly,
\[\spc{X}=A\oplus B\]\index{$A\oplus B$ (direct sum)}
if there are projections $\proj_A\:\spc{X}\to A$ 
and 
$\proj_B\:\spc{X}\to B$
such that 
\[\dist[2]{x}{y}{}=\dist[2]{\proj_A(x)}{\proj_A(y)}{}+\dist[2]{\proj_B(x)}{\proj_B(y)}{}\]
for any two points $x,y\in \spc{X}$.

If $\spc{X}=A\oplus B$, then 
\begin{itemize}
\item $A$ intersects $B$ at a single point,
\item both sets $A$ and $B$ are \index{convex set}\emph{convex sets} in $\spc{X}$;
the latter means that any geodesic with the ends in $A$ (or $B$) lies in $A$ (or $B$). 
\end{itemize}

\begin{thm}{Line splitting theorem}\label{thm:splitting}
Let $\gamma$ be a line in a $\Alex0$ space~$\spc{A}$. 
Then 
\[\spc{A}=\spc{A}'\oplus \gamma(\RR)\]
for some subset $\spc{A}'\subset \spc{A}$.
\end{thm}

\begin{thm}{Corollary}\label{cor:splitting}
Any $\Alex0$ space $\spc{A}$ splits isometrically as
\[
\spc{A}=\spc{A}'\oplus H
\]
where $H\subset \spc{A}$ is a subset isometric to a Hilbert space, and $\spc{A}'\subset \spc{A}$ is a convex subset that contains no lines. 
\end{thm}

The following lemma is closely related to the first distance estimate (\ref{thm:dist-est});
it is also a limit case of \ref{prop:gexp}.
The proof follows similar  lines.

\begin{thm}{Lemma}\label{lem:dist-estimate}
Suppose $f\:\spc{A}\to\RR$ be a concave 1-Lipschitz function on an $\Alex0$ space $\spc{A}$.
Consider two $f$-gradient curves $\alpha$ and~$\beta$.
Then for any $t, s\ge 0$ we have
\begin{align*}
&\dist[2]{\alpha(s)}{\beta(t)}{}
\le 
\dist[2]{p}{q}{}+
2\cdot(f(p)-f(q))\cdot(s-t)+ (s-t)^2,
\end{align*}
where $p=\alpha(0)$ and $q=\beta(0)$.
\end{thm}

\parit{Proof.}
Since $f$ is 1-Lipschitz, $|\nabla f|\le1$.
Therefore 
\[f\circ\beta(t)\le f(q)+t\]
for any $t\ge0$.

Set $\ell(t)=\dist{p}{\beta(t)}{}$.
Applying \ref{eq:fist-var-inq+}, we get
\begin{align*}
(\ell^2)^+(t)
&\le 2\cdot \left(f\circ\beta(t)-f(p)\right)\le 
\\
&\le2\cdot\left(f(q)+t-f(p)\right).
\end{align*}
Therefore 
\[\ell^2(t)-\ell^2(0)\le 2\cdot\left(f(q)-f(p)\right)\cdot t + t^2.\]
It proves the needed inequality in case $s=0$.
Combining it with the first distance estimate (\ref{thm:dist-est}), we get the result in case $s\le t$.
The case $s\ge t$ follows by switching the roles of $s$ and $t$.
\qeds


\parit{Proof of \ref{thm:splitting}.} Consider two Busemann functions, $\bus_+$ and $\bus_-$, associated with half-lines $\gamma:[0,\infty)\to \spc{A}$ and $\gamma:(-\infty,0]\to \spc{A}$ respectively; that is,
\[
\bus_\pm(x)
\df
\lim_{t\to\infty}\dist{\gamma(\pm t)}{x}{}- t.
\]
According to \ref{ex:busemann-CBB}, 
both $\bus_+$ and $\bus_-$ are concave.

By \ref{ex:bus+bus}, $\bus_+(x)+\bus_-(x)\ge0$ for any $x\in \spc{A}$.
On the other hand, by \ref{comp-kappa}, 
$f(t)=\distfun_x^2\circ\gamma(t)$ 
is $2$-concave.
In particular, $f(t)\le t^2+at+b$ for some constants $a,b\in\RR$.  Therefore, for all large $t$
\[
\dist{\gamma( t)}{x}{}- t +\dist{\gamma(- t)}{x}{}- t\le \sqrt{ t^2+at+b}-t+\sqrt{ t^2-at+b}-t
\]

Passing to the limit as $t\to\infty$, we get that  $\bus_+(x)+\bus_-(x)\le 0$.
Hence
\[
\bus_+(x)+\bus_-(x)= 0
\]
for any $x\in \spc{A}$.
In particular, the functions $\bus_+$ and $\bus_-$ are \index{affine function}\emph{affine};
that is, they are convex and concave at the same time.

For any $x$,
\begin{align*}
|\nabla_x \bus_\pm|
&=\sup\set{\dd_x\bus_\pm(\xi)}{\xi\in\Sigma_x}=
\\
&=\sup\set{-\dd_x\bus_\mp(\xi)}{\xi\in\Sigma_x}\equiv
\\
&\equiv1.
\end{align*}

A curve $\alpha$ is a $\bus_\pm$-gradient curve
if and only if $\alpha$ is a geodesic such that $(\bus_\pm\circ\alpha)^+=1$.
Indeed, if $\alpha$ is a geodesic, then $(\bus_\pm\circ\alpha)^+\le 1$ and the equality holds only if $\nabla_\alpha\bus_\pm=\alpha^+$.
Now suppose $\nabla_\alpha\bus_\pm=\alpha^+$.
Then $|\alpha^+|\le 1$ and $(\bus_\pm\circ\alpha)^+=1$; therefore 
\begin{align*}
|t_0-t_1|&\ge \dist{\alpha(t_0)}{\alpha(t_1)}{}\ge
\\
&\ge|\bus_\pm\circ\alpha(t_0)-\bus_\pm\circ\alpha(t_1)=
\\
&=|t_0-t_1|.
\end{align*}

It follows that for any $t>0$, the $\bus_\pm$-gradient flows commute;
that is, 
\[\GF_{\bus_+}^t\circ\GF_{\bus_-}^t=\id_\spc{A}.\]
Setting
\[\GF^t=\left[\begin{matrix}
\GF_{\bus_+}^t&\hbox{if}\ t\ge0\\
\GF_{\bus_-}^{-t}&\hbox{if}\ t\le0
               \end{matrix}\right.\]
defines an $\RR$-action on $\spc{A}$.

Consider the level set $\spc{A}'=\bus_+^{-1}(0)=\bus_-^{-1}(0)$;
it is a closed convex subset of $\spc{A}$, and therefore forms an Alexandrov space.
Consider the map $h\:\spc{A}'\times \RR\to \spc{A}$ defined by $h\:(x,t)\mapsto \GF^t(x)$.
Note that $h$ is onto.
Applying \ref{lem:dist-estimate} for $\GF_{\bus_+}^t$ and $\GF_{\bus_-}^t$ shows that $h$ is distance non-expanding and non-contracting at the same time; that is, $h$ is an isometry.
\qeds

Recall that according our definition the real line $\RR$ is $\Alex1$.
However, most of $\Alex1$ spaces have diameter at most $\pi$;
see \ref{ex:RisCBB(1)}.

\begin{thm}{Exercise}\label{ex:cone-CBB}
Suppose $\spc{X}$ is a complete geodesic space.
Show that $\Cone\spc{X}$ is $\Alex0$ if and only if $\spc{X}$ is $\Alex1$ and $\diam\spc{X}\le \pi$.
\end{thm}

\section{Anti-sum}

Here we give a corollary of \ref{ex:convergence-grad}.
It will be used to prove basic properties of the tangent space.


\begin{thm}{Anti-sum lemma}\label{lem:minus-sum} 
Let $\spc{A}$ be an Alexandrov space and $p\in \spc{A}$.

Given two vectors $u,v\in \T_p$, there is a unique vector $w\in \T_p$ such that
\[\langle u,x\rangle +\langle v,x\rangle +\langle w,x\rangle \ge 0\]
for any $x\in \T_p$, and
\[\langle u,w\rangle +\langle v,w\rangle +\langle w,w\rangle =0.\]

\end{thm}

\begin{thm}{Exercise}\label{ex:|antisum|}
Suppose $u,v, w\in \T_p$ are as in \ref{lem:minus-sum}.
Show that 
\[|w|^2\le |u|^2+|v|^2+2\cdot\langle u,v\rangle.\]

\end{thm}

If $\T_p$ were geodesic, then the lemma would follow from the existence  of the gradient, applied to the function $\T_p\to \RR$ defined by $x\mapsto -(\langle u,x\rangle +\langle v,x\rangle )$ which is concave.
However, the tangent space $\T_p$ might fail to be geodesic; see  Halbeisen's example \cite{alexander-kapovitch-petrunin2024}.

Applying the above lemma for $u=v$, we have the following statement.

\begin{thm}{Existence of polar vector}\label{cor:polar}
Let $\spc{A}$ be an Alexandrov space 
and $p\in \spc{A}$. 
Given a vector $u\in \T_p$,  there is a unique vector $u^*\in\T_p$ such that $\langle u^*,u^*\rangle +\langle u,u^*\rangle = 0$ and
$u^*$ is \index{polar vectors}\emph{polar} to $u$;
that is,
\[\langle u^*,x\rangle +\langle u,x\rangle \ge 0\]
for any $x\in \T_p$.
\end{thm}

\parit{Proof of \ref{lem:minus-sum}.}
By \ref{ex:d(distfun):==}, we can choose two sequences of points $a_n,b_n$ such that 
\begin{align*}
\dd_p\distfun_{a_n}(w)&=-\langle\dir{p}{a_n},w\rangle
\\
\dd_p\distfun_{b_n}(w)&=-\langle\dir{p}{b_n},w\rangle
\end{align*}
for any $w\in\T_p$ and $\dir{p}{a_n}\to u/|u|$, $\dir{p}{b_n}\to v/|v|$ as $n\to \infty$

Consider a sequence of functions 
\[f_n=|u|\cdot\distfun_{a_n}+|v|\cdot\distfun_{b_n}.\]
Note that 
\[(\dd_pf_n)(x)=-|u|\cdot\langle \dir{p}{a_n},x\rangle -|v|\cdot\langle \dir{p}{b_n},x\rangle .\]
Thus we have the following uniform convergence for $x\in\Sigma_p$:
\[(\dd_pf_n)(x)\to-\langle u,x\rangle -\langle v,x\rangle \]
as $n\to\infty$,
According to \ref{ex:convergence-grad}, 
the sequence $\nabla_pf_n$ converges.
Let 
\[w=\lim_{n\to\infty}\nabla_pf_n.\]
By the definition of gradient,
\[\begin{aligned}
\langle w,w\rangle &=\lim_{n\to\infty}\langle \nabla_pf_n,\nabla_pf_n\rangle =
&&&%right side
\langle w,x\rangle &=\lim_{n\to\infty}\langle \nabla_pf_n,x\rangle \ge
\\%second line
&=\lim_{n\to\infty}(\dd_p f_n)(\nabla_p f_n)
=
&&&%second line right side
&\ge
\lim_{n\to\infty}(\dd_pf_n)(x)
=
\\%line 3
&=-\langle u,w\rangle -\langle v,w\rangle ,
&&&%line 3 right side
&=-\langle u,x\rangle -\langle v,x\rangle .
\end{aligned}\]
\qedsf

\section{Linear subspace}

\begin{thm}{Definition}\label{def:opp+Lin}
Let $\spc{A}$ be an Alexandrov space, $p\in \spc{A}$ and $u,v\z\in\T_p$.
We say that vectors $u$ and $v$ are \index{opposite vectors}\emph{opposite}\label{def:opposite:page} to each other, (briefly, $u+v=0$) if $|u|=|v|=0$ or $\mangle(u,v)=\pi$ and $|u|=|v|$.

The subcone
\[\Lin_p=\set{v\in\T_p}{\exists\ w\in\T_p\quad \text{such that}\quad w+v=0}\]
will be called the \index{linear subspace}\emph{linear subspace} of $\T_p$.
\end{thm}

Soon we will introduce a natural linear structure on $\Lin_p$.

\begin{thm}{Proposition}\label{prop:opposite}
Let $\spc{A}$ be an Alexandrov space and $p\in \spc{A}$.
Given two vectors $u,v\in\T_p$, the following statements are equivalent:
\begin{subthm}{opposite} $u+v=0$;
\end{subthm}
\begin{subthm}{<x,u>} $\langle u,x\rangle +\langle v,x\rangle =0$ for any $x\in\T_p$;
\end{subthm}
\begin{subthm}{<xi,u>} $\langle u,\xi\rangle +\langle v,\xi\rangle =0$ for any $\xi\in\Sigma_p$.
\end{subthm}
\end{thm}

\parit{Proof.}
The equivalence  \ref{SHORT.<x,u>}$\Leftrightarrow$\ref{SHORT.<xi,u>} is trivial.

The condition $u+v=0$ is equivalent to 
$\langle u,u\rangle =-\langle u,v\rangle =\langle v,v\rangle$;
thus,
\ref{SHORT.<x,u>}$\Rightarrow$\ref{SHORT.opposite}.

Recall that $\T_p$ has nonnegative curvature.
The hinges $\hinge 0ux$ and $\hinge 0vx$ are adjacent.
By \ref{ex:adjacent-CBB}, $\mangle\hinge 0ux+\mangle\hinge 0vx=\pi$;
hence \ref{SHORT.opposite}$\Rightarrow$\ref{SHORT.<x,u>}.
\qeds

\begin{thm}{Exercise}\label{prop:two-opp}
Let $\spc{A}$  be an Alexandrov space and $p\in \spc{A}$.
Then for any three vectors $u,v,w\in\T_p$, if $u+v=0$ and $u+ w=0$ then $v=w$.
\end{thm}

Let $u\in \Lin_p$; that is, $u+v=0$ for some $v\in\T_p$.
Given $s<0$, let 
\[s\cdot u\df (-s)\cdot v.\]
So we can multiply any vector in $\Lin_p$ by any real number (positive and negative).
By \ref{prop:two-opp}, this multiplication is uniquely defined.
By \ref{prop:opposite}, we have identity
\[\langle -v,x\rangle=-\langle v,x\rangle.\]


\begin{thm}{Exercise}\label{ex:3<,>=0}
Suppose $u,v,w\in\T_p$ are as in \ref{lem:minus-sum}.
Show that
\[\langle u,x\rangle +\langle v,x\rangle +\langle w,x\rangle = 0\]
for any $x\in \Lin_p$.
\end{thm}

\begin{thm}{Exercise}\label{ex:-u}
Let $\spc{A}$ be an Alexandrov space,
$p\in \spc{A}$ and $u\in \T_p$.
Suppose $u^*\in \T_p$ is provided by \ref{cor:polar};
that is, 
\[\langle u^*,u^*\rangle +\langle u,u^*\rangle = 0
\quad\text{and}\quad
\langle u^*,x\rangle +\langle u,x\rangle \ge 0
\]
for any $x\in \T_p$.
Show that 
\[u=-u^*\quad\Longleftrightarrow\quad|u|=|u^*|.\]
\end{thm}

\begin{thm}{Theorem}\label{thm:lin-subcone}
Let $p$ be a point in an Alexandrov space. 
Then $\Lin_p$ is isometric to a Hilbert space.
\end{thm}

\parit{Proof.}
$\Lin_p$ is a closed subset of $\T_p$;
in particular, it is complete.

If any two vectors in $\Lin_p$ can be connected by a geodesic in $\Lin_p$,
then the statement follows from the splitting theorem (\ref{thm:splitting}).
By Menger's lemma (\ref{lem:mid>geod}), it is sufficient to show that for any two vectors $x,y\in\Lin_p$
there is a midpoint $w\in \Lin_p$.

Choose $w\in \T_p$ to be the anti-sum of $u=-\tfrac{1}{2}\cdot x$ and $v=-\tfrac{1}{2}\cdot y$;
see \ref{lem:minus-sum}.
By \ref{ex:|antisum|} and \ref{ex:3<,>=0},
\begin{align*}
|w|^2&\le \tfrac14\cdot |x|^2+\tfrac14\cdot|y|^2+\tfrac12\cdot\langle x,y\rangle,
\\
\langle w,x\rangle&= \tfrac12\cdot|x|^2+\tfrac12\cdot\langle x,y\rangle,
\\
\langle w,y\rangle&= \tfrac12\cdot|y|^2+\tfrac12\cdot\langle x,y\rangle,
\end{align*}
It follows that 
\begin{align*}
|x-w|^2
&= |x|^2+|w|^2-2\cdot\langle w,x\rangle\le
\\
&\le \tfrac14\cdot |x|^2+\tfrac14\cdot|y|^2-\tfrac12\cdot\langle x,y\rangle=
\\
&=\tfrac14\cdot|x-y|^2.
\end{align*}
That is, $|x-w|\le \tfrac12\cdot|x-y|$.
Similarly, we get $|y-w|\le \tfrac12\cdot|x-y|$.
Therefore $w$ is a midpoint of $x$ and $y$.
In addition, we get the equality 
\[|w|^2= \tfrac14\cdot |x|^2+\tfrac14\cdot|y|^2+\tfrac12\cdot\langle x,y\rangle.\]

It remains to show that $w\in\Lin_p$.
Let $w^*$ be the polar vector provided by \ref{cor:polar}.
Note that 
\[|w^*|\le |w|,
\quad
\langle w^*,x\rangle+\langle w,x\rangle=0,
\quad
\langle w^*,y\rangle+\langle w,y\rangle=0.
\]
The same calculation as above shows that $w^*$ is a midpoint of $-x$ and $-y$ and 
\[|w^*|^2= \tfrac14\cdot |x|^2+\tfrac14\cdot|y|^2+\tfrac12\cdot\langle x,y\rangle=|w|^2.\]
By \ref{ex:-u}, $w=-w^*$;
hence $w\in\Lin_p$.
\qeds

\begin{thm}{Lemma}\label{ex:grad-dist:G-delta}
Given a point $p$ in an Alexandrov space $\spc{A}$,
let $f\z=\distfun_p$, and let $S$ be the subset of points $x\in\spc{A}$ such that $|\nabla_xf|=1$.
Then $S$ is a dense G-delta set.

\end{thm}

\parit{Proof.}
Let $S_n\subset \spc{A}$ be defined by inequality $|\nabla_xf|>1-\tfrac1n$.
By \ref{ex:semicontinuous-grad:>s}, $S_n$ is open.

Choose a point $q\ne p$.
Observe that $|\nabla_xf|=1$ for any point $x\in\left]pq\right[$.
It follows that $S_n$ is dense in $\spc{A}$.

Since $S=\bigcap_nS_n$, the lemma follows.
\qeds


\begin{thm}{Exercise}\label{ex:grad-dist}
Let $p$, $f$, and $S$ be as in \ref{ex:grad-dist:G-delta}.

\begin{subthm}{ex:grad-dist:lin}
Show that 
\[\nabla_xf+\dir xp=0\]
for any 
$x\in S$;
in particular, $\dir xp\in \Lin_x$.
\end{subthm}

\begin{subthm}{ex:grad-dist:|grad|=1}
Show that if $|\nabla_xf|=1$, then $\dd_xf(w)= \langle\nabla_xf,w\rangle$ for any $w\in \T_x$.
\end{subthm}

\begin{subthm}{ex:grad-dist:geod}
Show that for any $x\in S$ there is a unique geodesic $[px]$.
\end{subthm}

\end{thm}

This exercise implies the following.

\begin{thm}{Corollary}\label{cor:euclid-subcone}
Given a countable set of points $X$ in an Alexandrov space $\spc{A}$
there is a G-delta dense set $S\subset\spc{A}$
such that 
$\dir sx\in \Lin_s$
for any $s\in S$ and $x\in X$.
\end{thm}

\section{Remarks}

The splitting theorem has an interesting history that starts with Stefan Cohn-Vossen \cite{cohn-vossen_line};
who proved its $2$-dimensional case.
For Riemannian manifolds of higher dimensions 
it was proved by Victor Toponogov \cite{toponogov-globalization+splitting}.
Then it was generalized by Anatoliy Milka \cite{milka-line}
to Alexandrov spaces;
historically, it was the first result about Alexandrov spaces of dimension higher than 2.
Nearly the same proof is used in \cite[1.5]{burago-burago-ivanov}.

Further generalizations of the splitting theorem for Riemannian manifolds with nonnegative Ricci curvature were obtained by Jeff Cheeger and Detlef Gromoll \cite{cheeger-gromoll-split}.
This was further generalized by Jeff Cheeger and Toby Colding for limits of Riemannian manifolds with almost nonnegative Ricci curvature \cite{cheeger-colding-alm-rigidity} and to their synthetic generalizations, so-called {}\emph{RCD spaces}, by Nicola Gigli \cite{gigli2013splitting, gigli-splitting-overview}.
Jost-Hinrich Eschenburg obtained an analogous result for Lorentzian manifolds \cite{eshenburg-split}, that is, pseudo-Riemannian manifolds of signature $(1,n)$.

The presented proof is close in spirit to the proof given by Cheeger and Gromoll \cite{cheeger-gromoll-split};
it is taken from our book \cite{alexander-kapovitch-petrunin2024}.

\begin{thm}{Open question}
Let $p$ be a point in an Alexandrov space $\spc{A}$.
Suppose that $0\ne v\in \Lin_p$.
Is it true that the tangent space $\T_p$ splits in the direction of $v$?
\end{thm}

Halbeisen's example \cite{alexander-kapovitch-petrunin2024,halbeisen} shows that compactness of space of directions is essential in the proof that space of directions is $\pi$-geodesic (see \ref{thm:finite-space-of-directions}).

\begin{thm}{Open question}\label{open:Halb-proper}
Let $\spc{A}$ be a proper Alexandrov space.
Is it true that for any $p\in \spc{A}$, the tangent space $\T_p$ is a length space?
\end{thm}

%%%%%%%%%%%%%%%%%%%%%%%%%%%%%%%%%%%%%%%%%%%%%%%%%%

\chapter{Dimension and volume}\label{chap:dim}

This lecture shows that several different notions of dimension are the same for Alexandrov spaces.
Also, we introduce volume and prove the Bishop--Gromov inequality, the right-inverse theorem and introduce the distance chart in finite-dimensional Alexandrov space.

\section{Linear dimension}

Let $\spc{A}$ be an Alexandrov space.
We define its \index{linear dimension}\emph{linear dimension} \index{$\LinDim$}$\LinDim \spc{A}$ as the least upper bound on integers $m$ such that 
the Euclidean space $\EE^m$ is isometric to a subspace of the tangent space $\T_p\spc{A}$ at some point $p\in \spc{A}$.
If not stated otherwise, dimension of an Alexandrov space is its linear dimension.

If not stated otherwise, dimension will mean linear dimension.
In Section~\ref{sec:all-dim}, we will show that linear dimension of Alexandrov space coincides with all reasonable dimensions;
after that, we will use \index{$\dim$}$\dim\spc{A}$ for $\LinDim\spc{A}$.

\begin{thm}{(\textit{n}+1)-comparison}\label{thm:n+1}
Let $\spc{A}$ be an $\Alex0$ space.
Then for any finite set of points $p,x_1,\dots,x_n\in \spc{A}$, there exists a model configuration 
$\tilde p,\tilde x_1,\dots,\tilde x_n\in \EE^m$ such that 
\[|\tilde p-\tilde x_i|_{\EE^m}=| p- x_i|_{\spc{A}}
\quad\text{and}\quad
|\tilde x_i-\tilde x_j|_{\EE^m}\ge |x_i- x_j|_{\spc{A}}\]
for any $i$ and $j$.
Moreover, we can assume that $m\le \LinDim\spc{A}$. 
\end{thm}

\parit{Proof.}
By \ref{cor:euclid-subcone}, we can choose a point $p'$ arbitrarily close to $p$ so that 
$\Lin_{p'}\ni \dir{p'}{x_i}$ for any $i$.
Let us identify $\EE^m$ with a subspace of $\Lin_{p'}$ spanned by $\dir{p'}{x_1},\dots,\dir{p'}{x_n}$.
Note that $m\le \LinDim\spc{A}$.

Set $\tilde p'=0\in \EE^m$ and $\tilde x_i=\dist{p'}{x_n}{}\cdot\dir{p'}{x_n}\in \EE^m$ for every $i$.
Note that 
\[|\tilde p'-\tilde x_i|_{\EE^m}=| p'- x_i|_{\spc{A}}\]
for every $i$.
Applying the comparison $\mangle\hinge {p'}{x_i}{x_j}\ge \angk {p'}{x_i}{x_j}$, we get
\[|\tilde x_i-\tilde x_j|_{\EE^m}\ge |x_i- x_j|_{\spc{A}}\]
for any $i$ and $j$.
Passing to a limit configuration as $p'\to p$ we get the result.
\qeds

\begin{thm}{Exercise}\label{ex:tangent=Em}
Let $\spc{A}$ be an $\Alex0$ space.
Suppose $\LinDim\spc{A}\z=m<\infty$.
Show that $\T_p\spc{A}\iso \EE^m$ for a G-delta dense set of points $p\in\spc{A}$.
\end{thm}

\begin{thm}{Exercise}\label{ex:dim=1}
Show that a 1-dimensional Alexandrov space is homeomorphic to a 1-dimensional manifold, possibly with non-empty boundary.
\end{thm}


\begin{thm}{Exercise}\label{ex:resporka}
Let $\spc{A}$ be an $\Alex0$ space.

Show that $\LinDim \spc{A}\ge m$ if and only if for some $m+2$ points $p$, $a_0,\z\dots, a_{m}\in \spc{A}$
we have
\[\angk p{a_i}{a_j}>\tfrac\pi2\]
for any pair $i\ne j$.
\end{thm}

\section{Space of directions}

A metric space $\spc{X}$ will be called $\ell$-geodesic 
if any two points $x,y\in\spc{X}$ such that $\dist{x}{y}{}<\ell$ can be connected by a geodesic.
For instance, any geodesic space is $\infty$-geodesic.

\begin{thm}{Theorem}\label{thm:finite-space-of-directions}
Let $\spc{A}$ be a finite-dimensional Alexandrov space.
Then for any point $p\in \spc{A}$, its space of directions $\Sigma_p$ is a compact $\pi$-geodesic space.
\end{thm}


By  \ref{ex:GHto-tangent} this immediately gives 

\begin{thm}{Corollary}\label{ex:GHto-tangent-finite-dim}
Let $p$ be a point in a finite dimensional Alexandrov space $\spc{A}$,
and let $\lambda_n\to\infty$.
Then there is a pointed Gromov--Hausdorff convergence $(\lambda_n\cdot \spc{A},p)\z\to (\T_p,0)$.
%Moreover, for any geodesic $\gamma$ that starts at $p$, we have
%\[\iota_n\circ\gamma(t/\lambda_n)\to t\cdot \gamma^+(0),\]
%where $\iota_n$ sends a point in $\spc{A}$ to the corresponding point in $\lambda_n\cdot\spc{A}$.
\end{thm}


\begin{thm}{Exercise}\label{ex:finite-tan}
Let $p$ be a point in a finite-dimensional Alexandrov space $\spc{A}$.
Prove the following.
\begin{subthm}{ex:finite-tan:tan}
The tangent space $\T_p$ is a proper $\Alex0$ space.
\end{subthm}

\begin{subthm}{ex:finite-space-of-directions-dim}
$\LinDim\Sigma_p=\LinDim\spc{A}-1$.
\end{subthm}

\begin{subthm}{ex:finite-tan:sigma}
If $\LinDim \spc{A}>1$, then $\Sigma_p$ is geodesic.
\end{subthm}


\end{thm}

Using \ref{ex:finite-space-of-directions-dim}, one can prove results for all finite-dimensional Alexandrov spaces via induction on  dimension.
Such proofs will be indicated below.

\parit{Proof of \ref{thm:finite-space-of-directions}.}
Choose $\eps>0$; suppose $\spc{A}$ is $m$-dimensional.
Assume we can choose $n$ directions $\xi_1,\dots, \xi_n\in \Sigma_p$ such that $\mangle(\xi_i,\xi_j)\z>\eps$ for any $i\ne j$.
Without loss of generality, we may assume that each direction is geodesic;
that is, there is a point $x_i\in \spc{A}$ such that $\xi_i=\dir p{x_i}$.

Choose $y_i\in [px_i]$ such that $\dist{p}{y_i}{}=r$ for each $i$ and small fixed $r>0$.
Since $r$ is small, we can assume that $\angk p{y_i}{y_j}>\eps$ for any $i\ne j$.
By \ref{cor:euclid-subcone}, we can choose $p'$ arbitrarily close to $p$ such that $\dir{p'}{y_i}\in \Lin_{p'}$ for any $i$.
Since  $\dist{p'}{p}{}$ is small, $\angk {p'}{y_i}{y_j}>\eps$ for any $i\ne j$.
By comparison, 
\[\mangle \hinge{p'}{y_i}{y_j}>\eps.\]
Therefore, $n\le \pack_\eps\SSS^{m-1}$,
where \index{$\pack_\eps\spc{X}$}$\pack_\eps\spc{X}$ is the exact upper bound on the number of points $x_1,\z\dots,x_k\in \spc{X}$ such that $\dist{x_i}{x_j}{}\ge\eps$ if $i\ne j$.

Since $\SSS^{m-1}$ is compact, $\pack_\eps\SSS^{m-1}<\infty$.
By the definition, the space of directions $\Sigma_p$ is complete. 
Applying \ref{ex:pack-net}, we get that  $\Sigma_p$ is compact.

It remains to prove the following claim.

\begin{clm}{}
If $\Sigma_p$ is compact, then it is $\pi$-geodesic
\end{clm}

Choose two geodesic directions $\xi=\dir px$ and $\zeta=\dir py$;
let 
\[\alpha\z=\tfrac12\cdot \mangle \hinge pxy=\tfrac12\cdot \dist{\xi}{\zeta}{\Sigma_p}.\]

Suppose $\alpha<\pi/2$.
Let us show that it is sufficient to construct an \index{almost midpoint}\emph{almost midpoint} $\mu\z=\dir pz$ of $\xi$ and $\zeta$ in $\Sigma_p$;
that is, we need to show that for any $\eps>0$ there is a geodesic $[pz]$ such that
\[\mangle\hinge pxz\le \alpha+\eps
\quad\text{and}\quad
\mangle\hinge pyz\le \alpha+\eps.\]
Indeed, once this is done, the compactness of $\Sigma_p$ can be used to get an actual midpoint for any two directions in $\Sigma_p$.
After that, Menger's lemma (\ref{lem:mid>geod}) will finish the proof.

Choose a sequence of small positive numbers $r_n\to0$
Consider sequences $x_n\z\in [px]$ and $y_n\z\in [py]$ such that $\dist{p}{x_n}{}=\dist{p}{y_n}{}=r_n$.
Let $m_n$ be a midpoint of $[x_n\,y_n]$.
%??? we use here that the directions $\xi=\dir px$ and $\zeta=\dir py$ are not opposite???

Since $\Sigma_p$ is compact, we can pass to a sequence of $r_n$ such that 
$\dir{p}{m_n}$ converges;
denote its limit by $\mu$.
Choose a geodesic $[pz]$ that runs at a small angle from $\mu$.
Let us show that $\dir pz$ is the needed almost midpoint.

Evidently, $\angk p{x_n}{m_n}=\angk p{y_n}{m_n}$.
By \ref{ex:alex-lemma-cat}, we have
\[\angk p{x_n}{m_n}+\angk p{y_n}{m_n}\le \angk p{x_n}{y_n}.\]

Let $z_n\in [pz]$ be the point such that $\dist{p}{z_n}{}=\dist{p}{m_n}{}$.
By construction, for all large $n$, we have $\mangle\hinge pz{m_n}\approx0$  with arbitrary small given error.
By comparison, the value $\frac{\dist{z_n}{m_n}{}}{\dist{p}{z_n}{}}$ can be assumed to be arbitrarily small for all large $n$.
Applying this observation and the definition of angle measure, we also have the following approximations
\begin{align*}
\angk p{x_n}{y_n}&\approx \mangle\hinge p{x_n}{y_n},
\\
\angk p{x_n}{m_n}\approx\angk p{x_n}{z_n}&\approx\mangle\hinge p{x_n}{z_n},
\\
\angk p{m_n}{y_n}\approx\angk p{z_n}{y_n}&\approx\mangle\hinge p{z_n}{y_n},
\end{align*}
again, with arbitrary given error for all large $n$.
It follows that $\dir pz$ is an almost midpoint of $\dir px$ and $\dir py$, as required.
\qeds

In the above proof, the angles $\mangle\hinge pxz$ and $\mangle\hinge pyz$ have lower bounds by 
the comparison, but we needed upper bounds that were extracted from the definition of angle measure and the compactness of space of directions.

\section{Right-inverse theorem}

\begin{thm}{Theorem}\label{thm:right-inverse}
Suppose $p,a_0,\dots,a_m$ be points in an Alexandrov space $\spc{A}$ such
\[\angk p{a_i}{a_j}>\tfrac\pi2\]
for any $i\ne j$.
Then the map $f\:\spc{A}\to\RR^m$ defined by
\[f\:x\mapsto (\dist{a_1}{x}{},\dots,\dist{a_m}{x}{})\]
has a right inverse defined in a neighborhood of $f(p)$.
\end{thm}

In the proof we construct a local right inverse $\map$ of $f$ around $f(p)$.
The construction uses gradient flow for a suitably chosen family of functions.
The structure of the proof can be seen in the following exercise;
more details are given in the hints.

\begin{thm}{Exercise}\label{ex:proof-right-inverse}
Suppose $p,a_0,\dots,a_m\in\spc{A}$ and $f\:\spc{A}\to\RR$ are as in \ref{thm:right-inverse}.
Assume $\eps>0$ is sufficiently small.
Given $\bm{y}\z=(y_1,\z\dots,y_m)\in \RR^m$, 
consider the function on $\spc{A}$ defined by
\[f_{\bm{y}}(x)=\min\{\,0, \dist{a_1}{x}{}-y_1,\dots,\dist{a_m}{x}{}-y_m\,\}+\eps\cdot\dist{a_0}{x}{}.\]


\begin{subthm}{ex:proof-right-inverse:grad}
Show that for some fixed $r>0$ and $\lambda$, the function $f_{\bm{y}}$ is $\lambda$-concave in $\oBall(p,r)$,
\begin{enumerate}[(i)]
\item\label{111} $(\dd_x\distfun_{a_i})(\nabla_x f_{\bm{y}})<-\eps^2$ if $\dist{a_i}{x}{}>y_i$ and
\item\label{222} $(\dd_x\distfun_{a_i})(\nabla_x f_{\bm{y}})>\eps^2$ if 
\[\dist{a_i}{x}{}-y_i=\min_j\{\dist{a_j}{x}{}\z-y_j\}<0.\]
\end{enumerate}
at any $x\in \oBall(p,r)$.

\end{subthm}

\begin{subthm}{ex:proof-right-inverse:alpha}
Let $\alpha_{\bm{y}}$ be $f_{\bm{y}}$-gradient curve that starts at $p$.
Use \ref{SHORT.ex:proof-right-inverse:grad} to show that 
\[
\distfun_{\bm{a}}{[\alpha_{\bm{y}}(t_0)]}
= 
\bm{y}\]
if 
$\tfrac1{\eps^2}\cdot|\distfun_{\bm{a}}{p}-\bm{y}|
\le
t_0
\le
\tfrac{r}{2}$.

\end{subthm}

\begin{subthm}{ex:proof-right-inverse:end}
Let $t_0(\bm{y})=\tfrac{1}{\eps^2}\cdot|\dist{\bm{a}}{p}{}-\bm{y}|$.
Use \ref{lem:fg-dist-est} to show that the map
\[\map\:{\bm{y}}\mapsto \alpha_{\bm{y}}\circ t_0(\bm{y})\]
continuous in $\Omega=\oBall(\dist{\bm{a}}{p}{},\tfrac{\eps^2}{2}\cdot r )\subset\RR^m$
and $f\circ \Phi(\bm{y})=\bm{y}$ for any $\bm{y}\in \Omega$.
\end{subthm}

This finishes the proof of \ref{thm:right-inverse}.
\end{thm}

%??? I think that since this is used later it should be proved and not left as a reference A: If we add a solution, then that is OK, is not it? in any case, the idea is more tranparent in the exercise and if needed one can read the solution. But lets do the real solution, not just a hint.

\section{Distance chart}

\begin{thm}{Theorem}\label{thm:dist-chart}
Suppose $p,a_0,\dots,a_m$ be points in an $m$-dimensional Alexandrov space $\spc{A}$ such
\[\angk p{a_i}{a_j}>\tfrac\pi2\]
for any $i\ne j$.
Then the map $f\:\spc{A}\to\RR^m$ defined by
\[f\:x\mapsto (\dist{a_1}{x}{},\dots,\dist{a_m}{x}{})\]
gives a bi-Lipschitz embedding of a neighborhood $\Omega$ of $p$;
the restriction $f|_\Omega$ is called \emph{distance chart} at $p$.
\end{thm}

The following exercise guides you to prove the theorem.

\begin{thm}{Exercise}\label{ex:proof-dist-chart}
Suppose $p,a_0,\dots,a_m\in\spc{A}$ and $f\:\spc{A}\to\RR$ are as in \ref{thm:right-inverse}.
Show that there is $\eps>0$ such that one of the following $m$ inequalities hold
\begin{align*}
\mangle\hinge xy{a_1}&<\tfrac\pi2-\eps,\ \dots,\  \mangle\hinge xy{a_m}<\tfrac\pi2-\eps,
\\
\mangle\hinge yx{a_1}&<\tfrac\pi2-\eps,\ \dots,\ \mangle\hinge yx{a_m}<\tfrac\pi2-\eps
\end{align*}
for any two points $x,y$ in a sufficiently small neighborhood of $p$.

Use this together with the right-inverse theorem (\ref{thm:right-inverse}) to prove \ref{thm:dist-chart}.
\end{thm}

\section{Volume}

Fix a positive integer $m$.
The $m$-dimensional Hausdorff measure of a Borel set $B$ in a metric space will be called its \index{volume}\emph{$m$-volume}; it will be denoted by $\vol_m B$.
We assume that the Hausdorff measure is calibrated so that the unit cube in $\EE^m$ has unit volume.

This definition will be applied mostly to subsets in $m$-dimensional Alexandrov spaces.
In this case, we may write $\vol B$ instead of $\vol_m B$.


\begin{thm}{Bishop--Gromov inequality}\label{inq:BG}
Let $\spc{A}$ be an $\Alex0$ space.
Suppose $\dim \spc{A}=m<\infty$.
Then 
\[\vol \oBall(p,r)\le \omega_m\cdot r^m,\]
where $\omega_m$ denotes the volume of the unit ball in $\EE^m$.
Moreover, the function 
\[r\mapsto \frac{\vol B(p,r)}{r^m}\]
is nonincreasing.
\end{thm}

\parit{Proof.}
Given $x\in\spc{A}$ choose a geodesic path $\gamma_x$ from $p$ to $x$.
Recall that $\log_p\:\spc{A}\to \T_p$ can be defined by $\log_p\:x\mapsto \gamma_x^+(0)$.
By comparison, $\log_p$ is distance-noncontracting.
Note that $\log_p$ maps $\oBall(p,r)_{\spc{A}}$ to $\oBall(0,r)_{\T_p}$.

\begin{wrapfigure}{r}{44 mm}
\vskip-0mm
\centering
\includegraphics{mppics/pic-803}
\vskip1mm
\end{wrapfigure}

If $\T_p\iso \EE^m$, then $\vol\oBall(0,r)_{\T_p}\z=\omega_m\cdot r^m$,
and the first statement follows.

If $\T_p$ is not isometric to $\EE^m$, then by \ref{ex:tangent=Em}, we can find a point $p'$ arbitrarily close to $p$ such that $\T_{p'}\iso \EE^m$.
If $\eps>\dist{p}{p'}{}$, then $\oBall(p,r)\subset \oBall(p',r+\eps)$.
Therefore,
\[\vol \oBall(p,r)\le \omega_m\cdot (r+\eps)^m\]
for any $\eps>0$.
Hence the first statement follows.

Now, suppose $0<r_1<r_2$.
Consider the map $w\: \spc{A}\to \spc{A}$ defined by $w\:x\mapsto \gamma_x(\tfrac {r_1}{r_2})$.
(The map $w$ mimics the dilation with center at $p$ and coefficient $\tfrac {r_1}{r_2}$.)
By comparison,
\[\dist{w(x)}{w(y)}{}\ge \tfrac {r_1}{r_2}\cdot \dist{x}{y}{}.\]
Observe that $\oBall(p,r_1) \supset w[\oBall(p,r_2)]$.
Therefore, 
\[\vol \oBall(p,r_1)\ge (\tfrac {r_1}{r_2})^m\cdot\vol \oBall(p,r_2).\]
\qedsf

The following exercise generalizes the Bishop--Gromov inequality to $\Alex{-1}$ case. 
It is sufficient for most applications, but a more exact statement will be given in \ref{inq:BG+} which also includes the case of  $\Alex{1}$ spaces.

\begin{thm}{Exercise}\label{ex:diam-compact:proper}
Show that any finite-dimensional Alexandrov space is proper.

\end{thm}

\begin{thm}{Exercise}\label{ex:BG}
Let $\spc{A}$ be an $\Alex{-1}$ space.
Suppose $\spc{A}=m<\infty$.
Show that
\[\vol \oBall(p,r)\le \omega_m\cdot(\sinh r)^m,\]
where $\omega_m$ denotes the volume of the unit ball in $\EE^m$.
Moreover, the function 
\[r\mapsto \frac{\vol B(p,r)}{(\sinh r)^m}\]
is nonincreasing.
\end{thm}

\section{Other dimensions}\label{sec:all-dim}

Next we want to show that \textit{all reasonable definitions of dimension give the same result for Alexandrov spaces}.
More precisely, we have the following theorem; compare to \cite[15.16]{alexander-kapovitch-petrunin2024}.
We refer to \cite{hurewicz-wallman} for definitions of \index{Lebesgue covering dimension}\emph{Lebesgue covering dimension} \index{$\TopDim$}$\TopDim$ and \index{Hausdorff!dimension}\emph{Hausdorff dimension} \index{$\HausDim$}$\HausDim$.

\begin{thm}{Theorem}\label{thm:dim=dim}
For any Alexandrov space $\spc{A}$, we have
\[\LinDim \spc{A}=\TopDim \spc{A}=\HausDim \spc{A}.\]
\end{thm}

\parit{Proof.}
Suppose $\LinDim \spc{A}=\infty$.
By the right-inverse theorem (\ref{thm:right-inverse}), $\spc{A}$ contains a compact subset $K$ with an arbitrarily large $\TopDim K$.
In particular,
\[\TopDim\spc{A}=\infty.\] 
By Szpilrajn's theorem,
$\HausDim K\ge \TopDim K$.
Thus we also have 
\[\HausDim\spc{A}=\infty.\]

Now suppose $\LinDim \spc{A}=m<\infty$.
By the Bishop--Gromov inequality (\ref{inq:BG} and \ref{ex:BG}), 
\[\HausDim\spc{A}\le m.\]

Since $\spc{A}$ is proper (\ref{ex:diam-compact:proper}),
Szpilrajn's theorem, implies that
\[\TopDim\spc{A}\le \HausDim\spc{A}\le m.\]
Finally, the right-inverse theorem (\ref{thm:right-inverse}) implies that $m\le\TopDim\spc{A}$.
\qeds

\begin{thm}{Exercise}\label{ex:dim=dim}
Let $\Omega$ be an open subset of Alexandrov space $\spc{A}$.
Show that 
\[\LinDim \spc{A}=\LinDim \Omega=\TopDim \Omega=\HausDim \Omega.\]
\end{thm}

\section{Remarks}

Let us state a version of the Bishop--Gromov inequality for $\Alex\kappa$ spaces.
Its proof requires additionally the so-called \textit{coarea formula} for Alexandrov spaces. 
The weaker inequality from \ref{ex:BG} is sufficient for the sequel.

\begin{thm}{Bishop--Gromov inequality}\label{inq:BG+}
Given a point $p$ in an $m$-dimensional $\Alex\kappa$ space,
consider the function $v(r)\z=\vol_m\oBall(p,r)$;
denote by $\tilde v(r)$ the volume of $r$ ball in $\MM^m(\kappa)$.
Then 
\[v(r)\le \tilde v(r)\]
for $r>0$ and the function 
\[r\mapsto \frac{v(r)}{\tilde v(r)}\] is nonincreasing.
If $\kappa>0$, then one has to assume that $r<\tfrac\pi{\sqrt\kappa}$.
\end{thm}

This inequality was originally proved for Riemannian manifolds with lower Ricci curvature.
The first part is also called \emph{Bishop's inequality}.
It is due to Richard Bishop; see \cite{bishop1964} and \cite[Corollary 4, p. 256]{bishop-crittenden}.
The second part is due to Michael Gromov \cite{gromov1981}.

Theorem~\ref{thm:dim=dim}, was essentially proved by Conrad Plaut \cite{plaut:dimension}.
At that time, it was not known whether
\[\LinDim\spc{A}=\infty\quad \Rightarrow\quad \TopDim\spc{A}=\infty\]
for any Alexandrov space $\spc{A}$.
The latter implication was proved by Grigory Perelman and the second author \cite{perelman-petrunin:qg}.

