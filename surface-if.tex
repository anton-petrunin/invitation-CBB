%%!TEX root = the-surface-if.tex
\chapter{Surface of convex body}\label{chap:convex-body}

In this lecture, we discuss surfaces of convex bodies;
this is historically the first application of Alexandrov geometry.

\section{Definitions}

Let us define a \index{convex body}\emph{convex body} as a compact convex subset in $\EE^m$ with non-empty interior.

In This chapter we will only deal with convex bodies in  $\EE^3$. Therefore, unless indicated otherwise,  we will assume that $m=3$ from now on.

The \index{surface}\emph{surface} of a convex body is defined as its boundary equipped with the induced length metric.

\begin{thm}{Exercise}\label{ex:surf-S2}
Show that the surface of a convex body is homeomorphic to $\SSS^2$.
\end{thm}

In this lecture, we will prove that \textit{surface of a convex body is $\Alex0$.}


\section{Surface of convex polyhedra}

A \index{convex polyhedron}\emph{convex polyhedron} is a convex body with a finite number of extremal points, called its \index{vertex}\emph{vertices}.

Observe that the  surface, say $\Sigma$, of a convex polyhedron $P$ admits a triangulation such that each triangle is isometric to a plane triangle.
In other words, $\Sigma$ is a \index{polyhedral surface}\emph{polyhedral surface};
that is, it is a 2-dimensional manifold with a length metric that admits a triangulation such that each triangle is isometric to a solid plane triangle.
A \index{triangulation}\emph{triangulation} of a polyhedral surface will always be assumed to satisfy this condition.

The total angle around a vertex $v$ in $\Sigma$ is defined as the sum of angles at $v$ of all triangles in the triangulation that contain $v$.

If a point $p\in \Sigma$ is not a vertex of $P$,
then
\begin{itemize}
\item $p$ lies in the interior of a face of $P$, and its neighborhood in $\Sigma$ is a piece of plane, or
\item $p$ lies on an edge, and its neighborhood is two half-planes glued along the boundary.
\end{itemize}
In both cases, a neighborhood of $p$ in $\Sigma$ (with the induced length metric) is isometric to an open domain of the plane.
Therefore, the total angle around $p$ should be defined to be $2\cdot\pi$.

\begin{thm}{Claim}\label{clm:total-angle}
Let $\Sigma$ be the surface of a convex polyhedron $P$.
Then, the total angle around any point in $\Sigma$ cannot exceed $2\cdot\pi$.
\end{thm} 

In the proof, we will need the triangle inequality for angles (or the spherical triangle inequality).
A proof of this statement is given in the classical geometry textbook by Andrei Kiselyov \cite[§ 47]{kiselev-stereo-en};
it also follows from \ref{claim:angle-3angle-inq}.
(In fact our proof of \ref{claim:angle-3angle-inq} is a straightforward generalization of the argument in \cite[§ 47]{kiselev-stereo-en}.)

\begin{thm}{Spherical triangle inequality}\label{ex:angle-triangle}
Let $w_1,w_2,w_3$ be unit vectors in $\EE^3$.
Denote by $\theta_{i,j}$ the angle between the vectors $v_i$ and $v_j$.
Then
$$\theta_{1,3}\le \theta_{1,2}+\theta_{2,3}$$
and in case of equality, the vectors $w_1,w_2,w_3$ lie in a plane.
\end{thm}

\parit{Proof of \ref{clm:total-angle}.}
Consider the intersection of $P$ with a small sphere centered at~$p$;
it is a convex spherical polygon, say $F$.
Applying rescaling we may assume that the sphere has unit radius.
Then we need to show that the perimeter of $F$ does not exceed $2\cdot\pi$.

\begin{wrapfigure}{o}{22mm}
\vskip-2mm
\centering
\includegraphics{mppics/pic-1103}
\end{wrapfigure}

Note that $F$ lies in a hemisphere, say $H$.
Moreover, there is a decreasing sequence of convex spherical polygons 
\[H=H_0\supset\dots\supset H_n=F,\]
such that $H_{i+1}$ is obtained from $H_{i}$ by cutting along a chord.

By the spherical triangle inequality (\ref{ex:angle-triangle}), we have
\[
2\cdot\pi=\perim H=\perim H_0\ge\dots\ge\perim H_n=\perim F
\]
--- hence the result.
\qedsf

A vertex of a triangulation of a polyhedral surface is called \index{essential vertex}\emph{essential} if the total angle around it is not $2\cdot\pi$.

\begin{thm}{Exercise}\label{ex:vertex-essential-vertex}
Let $v$ be a point on the surface $\Sigma$ of a convex polyhedron $P$.
Show that $v$ is a vertex of $P$ if and only if 
$v$ is an essential vertex of $\Sigma$.
\end{thm}


\begin{thm}{Exercise}\label{ex:geodesic-vertex}
Show that geodesics on the surface of a convex polyhedron do not pass thru its essential vertices.
\end{thm}

\section{Curvature}

Let $p$ be a point on the surface of a polyhedron, and $\theta_p$ is the total angle around $p$.
The value $2\cdot \pi -\theta_p$ is called the \index{curvature}\emph{curvature} of the polyhedral surface at $p$.
If $p$ is not a vertex, then its curvature is zero.

\begin{thm}{Exercise}\label{pr:tetrahedron} 
Assume that the surface of a nondegenerate tetrahedron $T$ has curvature $\pi$ at each of its vertices.
Show that 

\begin{subthm}{pr:tetrahedron:=}
all faces of $T$ are congruent; 
\end{subthm}

\begin{subthm}{pr:tetrahedron:perp} the line passing thru midpoints of opposite edges of $T$ intersects these edges at right angles.
\end{subthm}
 
\end{thm}

Claim~\ref{clm:total-angle} says that \textit{surfaces of convex polyhedra have nonnegative curvature} in the sense of the above definition.
Now we show that this definition agrees with the 4-point comparison.

\begin{thm}{Proposition}\label{prop:poly-CBB}
A polyhedral surface with nonnegative curvature at each vertex is $\Alex0$.
\end{thm}

\parit{Proof.}
Denote the surface by $\Sigma$.
By \ref{comp-kappa}, it is sufficient to check that 
$\distfun_p^2\circ\gamma$ is 1-concave for any geodesic $\gamma$ and a point $p$ in $\Sigma$.

We can assume that $p$ is not a vertex;
the vertex case can be done by approximation.
Further, by \ref{ex:geodesic-vertex}, we may assume that $\gamma$ does not contain vertices.

Given a point $x=\gamma(t_0)$, choose a geodesic $[px]$.
Again, by \ref{ex:geodesic-vertex}, $[px]$ does not contain vertices.
Therefore a small neighborhood of $U\supset [px]$ can be unfolded on a plane;
that is, there is an injective length-preserving map $z\mapsto \tilde z$
of $U$ into the Euclidean plane.
This way we map part of $\gamma$ in $U$ to a line segment $\tilde\gamma$.
Let 
\[\tilde f(t)\df\tfrac12\cdot\distfun_{\tilde p}^2\circ\tilde \gamma(t).\]
Since the geodesic $[px]$ maps to a line segment, we have $\tilde f(t_0)= f(t_0)$.
Furthermore, since the unfolding $z\mapsto \tilde z$ preserves lengths of curves, we get 
$\tilde f(t)\ge f(t)$ if $t$ is sufficiently close to $t_0$.
That is, $\tilde f$ is a local upper support of $f$ at $t_0$.
Evidently, $\tilde f''\equiv 1$; therefore $f''\le 1$.
It remains to apply \ref{comp-kappa}.
\qeds

\begin{thm}{Exercise}\label{ex:poly-CBB}
Prove the converse to the proposition;
that is, show that if a poyhedral surface is $\Alex0$, then it has nonnegative curvature in the sense defined in this section.
\end{thm}

\section{Surface of convex body}

\begin{thm}{Advanced exercise}\label{ex:surface-covergence}
Let $K_1,K_2,\dots,$ and $K_\infty$ be convex bodies in $\EE^m$.
Denote by $S_n$ the surface of $K_n$ with induced length metric.
Suppose $K_n\z\to K_\infty$ in the sense of Hausdorff.
Show that $S_n\to S_\infty$ in the sense of Gromov--Hausdorff.
\end{thm}

Any convex body is a Hausdorff limit of a sequence of convex polyhedra.
Therefore, the next proposition follows from \ref{prop:poly-CBB}, \ref{ex:surface-covergence}, and \ref{thm:CBB-closed}.

\begin{thm}{Proposition}\label{prop:conv-surf-CBB(0)}
The surface of a convex body is $\Alex0$.
\end{thm}


\section{Remarks}

\ref{ex:surf-S2} and \ref{prop:poly-CBB} imply that the surface of a convex body is a sphere with nonnegative curvature in the sense of Alexandrov.
The celebrated theorem of Alexandrov states that the converse also holds if we allow degeneration of convex bodies to plane figures;
the surface of a plane figure is defined as its doubling across the boundary.
In other words, any $\Alex0$ metric on the sphere is isometric to a surface of (possibly degenerate) convex body.
Moreover this convex body is unique up to congruence.
The last result is due to Alexei Pogorelov \cite{pogorelov}.

Originally, Alexandrov proved the statement for polyhedral metrics on the sphere; this proof is sketched in the appendix.
Then he used \ref{ex:surface-covergence} to extend the result to an arbitrary $\Alex0$ metric on the sphere.

\begin{thm}{Advanced exercise}\label{ex:liberman+milka}
Let $S$ be the surface of a nondegenerate convex body $K\subset\EE^3$;
we assume that $S$ is equipped with its induced length metric.

\begin{subthm}{ex:liberman+milka:liberman}
Show that any geodesic $\gamma$ in $S$ is one-sided differentiable as a curve in $\EE^3$ 
\end{subthm}

\begin{subthm}{ex:liberman+milka:milka}
Let $\gamma_1$ and $\gamma_2$ be geodesic paths in $S$ that start at one point $p\z=\gamma_1(0)\z=\gamma_2(0)$.
Suppose $x_i=\gamma_i(1)$, and $y_i\z=p+\gamma_i^+(0)$.
Show that 
\[\dist{x_1}{x_2}{S}\le \dist{y_1}{y_2}{W},\]
where $W$ is the complement to the interior of $K$.
\end{subthm}

\end{thm}

