\chapter{Surface of convex body}\label{chap:convex-body}

Let us define a \emph{convex body} as a compact convex subset in $\EE^3$;
we assume that it does not lie in a line but it might degenerate to a plane figure.

Suppose $B$ is a nondegenerate convex body; that is, it has nonempty interior.
Then the \emph{surface} of $B$ is defined as its boundary $\partial B$ equipped with the induced length metric.

If a convex body degenerates to a plane convex figure, say $F$, then its surface is defined as a \emph{doubling} of $F$ along its boundary;
that is, two copies of $F$ glued along the boundary $\partial F$.
Intuitively, one can regard these copies as different sides of $F$ --- we live on $F$ and to get from one side to the other one has to cross the boundary.

\begin{thm}{Exercise}\label{ex:surf-S2}
Show that surface of a convex body is homeomorphic to $\SSS^2$.
\end{thm}

In this lecture, we will prove that \textit{surface of a convex body is $\CBB(0)$.}



\section{Surface of convex polyhedra}

Recall that a \emph{convex polyhedron} $P$ is a convex hull of a finite set of points
that called its \emph{vertices}.
We always assume that the vertices are extremal points of polyhedron;
in other words the set of vertices of $P$ is the minimal set of points such that its convex hull is $P$.
As for convex bodies, our convex polyhedra might degenerate to a plane polygon, but we assume that it does not belong to a line.

Observe that a surface, say $\Sigma$, of a convex polyhedron $P$ admits a triangulation such that each triangle is isometric to a plane triangle.
In other words, $\Sigma$ is a \index{polyhedral surfaces}\emph{polyhedral surfaces};
that is, it is a 2-dimensional manifold with length metric that admits a triangulation such that each triangle is isometric to a solid plane triangle.
A \index{triangulation}\emph{triangulation} of polyhedral surface will be assumed to satisfy this condition.

The total angle around a vertex $p$ in $\Sigma$ is defined as the sum of angles at $p$ of all triangles in the triangulation that contain $p$.

Note that if a point $p$ is not a vertex of $P$,
then
\begin{itemize}
\item $p$ lies in the interior of a face of $P$, and its neighborhood in $\Sigma$ is a piece of plane, or
\item $p$ lies on an edge, and its neighborhood is two half-planes glued along the boundary.
\end{itemize}
In both cases, a neighborhood of $p$ in $\Sigma$ (with the induced length metric) is isometric to an open domain of the plane.

\begin{thm}{Claim}\label{clm:total-angle}
Let $\Sigma$ be the surface of a convex polyhedron $P$.
Then, the total angle around a vertex in $\Sigma$ cannot exceed $2\cdot\pi$.
\end{thm} 


In the proof, we will need the triangle inequality for angles (or the spherical triangle inequality).
A proof of this statement is given in the classical geometry textbook by Andrei Kiselyov \cite[§ 47]{kiselev-stereo-en},
it also follows from \ref{claim:angle-3angle-inq}.
(In fact our proof of \ref{claim:angle-3angle-inq} is a straightforward generalization of the argument in \cite[§ 47]{kiselev-stereo-en}.)

\begin{thm}{Spherical triangle inequality}\label{ex:angle-triangle}
Let $w_1,w_2,w_3$ be unit vectors in $\EE^3$.
Denote by $\theta_{i,j}$ the angle between the vectors $v_i$ and $v_j$.
Show that
$$\theta_{1,3}\le \theta_{1,2}+\theta_{2,3}$$
and in case of equality, the vectors $w_1,w_2,w_3$ lie in a plane.
\end{thm}

\parit{Proof of \ref{clm:total-angle}.}
Consider the intersection of $P$ with a small sphere centered at~$p$;
it is a convex spherical polygon, say $F$.
Applying rescaling we may assume that the sphere has unit radius.
Then we need to show that the perimeter of $F$ does not exceed $2\cdot\pi$.

Note that $F$ lies in a hemisphere, say $H$.
Moreover, there is a decreasing sequence 
\[H=H_0\supset H_1\supset\dots\supset H_n=F,\]
such that $H_{i+1}$ is obtained from $H_{i}$ by cutting along a chord.

By \ref{ex:angle-triangle}, we have
\begin{align*}
2\cdot\pi=\perim H=\perim H_0&\ge\perim H_1\ge\dots\ge\perim H_n=\perim F
\end{align*}
--- hence the result.
\qedsf

A vertex of a triangulation of a polyhedral surface is called \index{essential vertex}\emph{essential} if the total angle around it is not $2\cdot\pi$.

\begin{thm}{Exercise}\label{ex:vertex-essential-vertex}
Show that any vertex of a polyhedron is an essential vertex of its surface;
that is, the inequality in the claim is strict.
\end{thm}


\begin{thm}{Exercise}\label{ex:geodesic-vertex}
Show that geodesics on a surface of convex polyhedron do not pass thru its essential vertices.
\end{thm}

\section{Curvature}

Let $p$ be a vertex of a polyhedron.
If $\theta_p$ is the total angle around $p$, then the value $2\cdot \pi -\theta_p$ is called the \emph{curvature} of the polyhedral surface at $p$;
if $p$ is not a vertex, then its curvature is defined to be zero.

\begin{thm}{Exercise}\label{pr:tetrahedron} 
Assume that the surface of a nondegenerate tetrahedron $T$ has curvature $\pi$ at each of its vertices.
Show that 

\begin{subthm}{}
all faces of $T$ are congruent; 
\end{subthm}

\begin{subthm}{} the line passing thru midpoints of opposite edges of $T$ intersects these edges at right angles.
\end{subthm}
 
\end{thm}

Note that the claim above says that \textit{surface of a convex polyhedron has nondegenerate curvature}.
However this definition works only for polyhedral surfaces.
Now we show that it agrees with the $\CBB(0)$ definition.

\begin{thm}{Proposition}\label{prop:poly-CBB}
A polyhedral surface with nonnegative curvature at each vertex is $\CBB(0)$.
\end{thm}

\parit{Proof.}
Denote the surface by $\Sigma$.
By \ref{comp-kappa}, it is sufficient to check that 
$\distfun_p^2\circ\gamma$ is 1-concave for any geodesic $\gamma$ and a point $p$ in $\Sigma$.

We can assume that $p$ is not a vertex;
the vertex case can be done by approximation.
Further, by \ref{ex:geodesic-vertex}, we may assume that $\gamma$ does not contain vertices.

Given a point $x=\gamma(t_0)$, choose a geodesic $[px]$.
Again, by \ref{ex:geodesic-vertex}, $[px]$ does not contain vertices.
Therefore a small neighborhood of $U\supset [px]$ can be unfolded on a plane;
that is, there is an injective length-preserving map $z\mapsto \tilde z$
of $U$ into the Euclidean plane.
Note that this way we map part of $\gamma$ in $U$ to a line segment.
Let 
\[\tilde f(t)\df\tfrac12\cdot\distfun_{\tilde p}^2\circ\tilde \gamma(t).\]
Note that $\tilde f(t_0)\ge f(t_0)$.
Further, since the unfolding $z\mapsto \tilde z$ preserves lengths of curves, we get 
$\tilde f(t)\ge f(t)$ if $t$ is sufficiently close to $t_0$.
That is, $\tilde f$ is a local upper support of $f$ at $t_0$.
Evidently, $\tilde f''\equiv 1$; therefore $f''\le 1$.
It remains to apply \ref{comp-kappa}.
\qeds

\begin{thm}{Exercise}\label{ex:poly-CBB}
Prove the converse to the proposition;
that is, show that if a poyhedral surface is $\CBB(0)$, then it has nonnegative curvature at each vertex.
\end{thm}

\section{Surface of convex body}

\begin{thm}{Advanced exercise}\label{ex:surface-covergence}
Let $K_\infty,K_1,K_2,\dots$ be convex bodies in $\EE^3$.
Denote by $S_n$ the surface of $K_n$ with induced length metric.
Suppose $K_n\z\to K_\infty$ in the sense of Hausdorff.
Show that $S_n\to S_\infty$ in the sense of Gromov--Hausdorff.
\end{thm}

\begin{thm}{Proposition}\label{prop:conv-surf-CBB(0)}
The surface of a nondegenerate convex body is $\CBB(0)$.
\end{thm}

Note that any convex body is a Hausdorff limit of a sequence of convex polyhedra.
Therefore, the proposition follows from \ref{prop:poly-CBB}, \ref{ex:surface-covergence}, and \ref{thm:CBB-closed}.

\section{Comments}

Note that \ref{ex:surf-S2} and \ref{prop:poly-CBB} imply that surface of convex body is a sphere with nonnegative curvature in the sense of Alexandrov.
The selbrated theorem of Alexandrov states that the converse also holds;
that is any geodesic $\CBB(0)$ metric on the sphere is isometric to a surface of convex body.
Moreover this convex body is unique up to congruence.
The last result is due to Alexei Pogorelov \cite{pogorelov}.

Originally, Alexandrov proved the statement for polyhedral metrics on the sphere
and then used \ref{ex:surface-covergence} to extend the result to arbitrary geodesic $\CBB(0)$ metric on the sphere.


\begin{thm}{Advanced exercise}\label{ex:liberman+milka}
Let $\Sigma$ be the surface of a nondegenerate convex body $K\subset\EE^3$;
we assume that $\Sigma$ is equipped with its induced length metric.

\begin{subthm}{ex:liberman+milka:liberman}
Show that any geodesic $\gamma$ in $\Sigma$ is one-side differentiable as a curve in $\EE^3$ 
\end{subthm}

\begin{subthm}{ex:liberman+milka:milka}
Let $\gamma_1$ and $\gamma_2$ be geodesic paths in $\Sigma$ that start at one point $p=\gamma_1(0)=\gamma_2(0)$.
Suppose $x_1=\gamma_1(1)$, $x_2=\gamma_2(1)$, $y_1=p+\gamma_1^+(0)$, and $y_2=p+\gamma_2^+(0)$.
Show that 
\[\dist{x_1}{x_2}{\Sigma}\le \dist{y_1}{y_2}{W},\]
where $W$ is the complement to the interior of $K$.
\end{subthm}

\end{thm}

