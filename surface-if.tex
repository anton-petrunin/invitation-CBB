%%!TEX root = the-surface-if.tex
\chapter{Surface of convex body}\label{chap:convex-body}

In this lecture, we will prove that \textit{surface of a convex body is $\Alex0$},
which is historically the first application of Alexandrov geometry.



\section{Uniqueness}

\begin{thm}{Theorem}

\end{thm}




\section{Surface of convex body}

\begin{thm}{Advanced exercise}\label{ex:surface-covergence}
Let $K_1,K_2,\dots,$ and $K_\infty$ be convex bodies in $\EE^m$.
Denote by $S_n$ the surface of $K_n$ with induced length metric.
Suppose $K_n\z\to K_\infty$ in the sense of Hausdorff.
Show that $S_n\to S_\infty$ in the sense of Gromov--Hausdorff.
\end{thm}

Any convex body is a Hausdorff limit of a sequence of convex polyhedra.
Therefore, the next proposition follows from \ref{prop:poly-CBB}, \ref{ex:surface-covergence}, and \ref{thm:CBB-closed}.

\begin{thm}{Proposition}\label{prop:conv-surf-CBB(0)}
The surface of a convex body in $\EE^3$ is $\Alex0$.
\end{thm}


\section{Remarks}

\ref{ex:surf-S2} and \ref{prop:poly-CBB} imply that the surface of a convex body is a sphere with nonnegative curvature in the sense of Alexandrov.
A celebrated theorem of Alexandrov states that the converse also holds if we allow degeneration of convex bodies to plane figures;
the surface of a plane figure is defined as its doubling across the boundary.
In other words, any $\Alex0$ metric on the two-sphere is isometric to a surface of a (possibly degenerate) convex body.
Moreover this convex body is unique up to congruence.
The last result is due to Alexei Pogorelov \cite{pogorelov}.

Originally, Alexandrov proved the statement for polyhedral metrics on the sphere; this proof is sketched in the appendix.
Then he used \ref{ex:surface-covergence} to extend the result to  arbitrary $\Alex0$ metrics on the sphere.

Proposition \ref{prop:conv-surf-CBB(0)} generalizes to boundaries of convex bodies  in $\EE^m$ for any $m\ge 2$.
This can be proved using \ref{ex:surface-covergence} by approximating a convex body by smooth ones.
Boundaries of smooth convex bodies have nonnegative sectional curvature by the Gauss formula and hence are $\Alex0$.


\begin{thm}{Advanced exercise}\label{ex:liberman+milka}
Let $S$ be the surface of a nondegenerate convex body $K\subset\EE^3$;
we assume that $S$ is equipped with the induced length metric.

\begin{subthm}{ex:liberman+milka:liberman}
Show that any geodesic $\gamma$ in $S$ is one-sided differentiable as a curve in $\EE^3$ 
\end{subthm}

\begin{subthm}{ex:liberman+milka:milka}
Let $\gamma_1$ and $\gamma_2$ be geodesic paths in $S$ that start at one point $p\z=\gamma_1(0)\z=\gamma_2(0)$.
Suppose $x_i=\gamma_i(1)$, and $y_i\z=p+\gamma_i^+(0)$.
Show that 
\[\dist{x_1}{x_2}{S}\le \dist{y_1}{y_2}{W},\]
where $W$ is the complement to the interior of $K$.
\end{subthm}

\end{thm}

