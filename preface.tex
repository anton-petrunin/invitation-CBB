\chapter*{Preface}

As in our previous booklet \cite{alexander-kapovitch-petrunin-2019},
we try to demonstrate the beauty and power of Alexandrov geometry by reaching interesting applications and theorems with a minimum of preparation.
This time we do spaces with curvature bounded below in the sense of Alexandrov.

This subject is more technical, this time we jumped over proofs of couple of technical results,
namely existence part in generalized Picard's theorem (\ref{thm:glob-exist-grad-curv})
and Perelman's theorem about conic neighborhoods (\ref{thm:spherical-nbhd}).
The rest of our presentation is nearly rigorous.

\medskip 

In Lecture~\ref{chap:prelim}, we discuss necessary preliminaries and fix notations.

Lecture~\ref{chap:defs} introduces the main object of our study --- spaces with curvature bounded below in the sense of Alexandrov.

In Lecture~\ref{chap:globalization} we formulate and prove the globalization theorem --- local Alexandrov condition implies global.
To simplify the presentation we consider only compact case, but this case is leading.

In Lecture~\ref{chap:derivative} we do beginning of calculus --- tangent space and space of directions, differential, and gradient.

Lecture~\ref{chap:GF} introduces gradient flow --- this is the main technical tool in the theory.

Lecture~\ref{chap:splitting} proves the line splitting theorem.
It provides the first application of gradient flow.

In Lecture~\ref{chap:dim} we introduce and discuss dimension of Alexandrov spaces.

Lecture~\ref{chap:lim} shows that lower curvature bound survives in the Gromov--Hausdorff limit.
Further, we introduce volume, prove the Bishop--Gromov inequality and use it to prove Gromov's selection theorem.

Lecture~\ref{chap:stability} starts with Perelman's construction of concave functions.
Further, we apply it with Gromov's selection theorem to prove the homotopy finiteness theorem.
This proof illustrates the main source of applications of Alexandrov geometry.

In Lecture~\ref{chap:bry} we introduce boundary of finite-dimensioanal Alexandrov space and prove the doubling theorem.

Lecture~\ref{chap:L/G} we show that quotient Alexandrov space by isometric group action is an Alexandrov space and give several applications of this statement.
These proofs illustrate another source of applications of Alexandrov geometry.

In Lecture~\ref{chap:convex-body} we show that surface of a convex body in Euclidean space is an Alexandrov space. This is historically the first serious application of Alexandrov geometry.

Finally, Appendix~\ref{chap:embedding} sketches Alexandrov embedding theorem of convex polyhedra.
Historically, this theorem is the first remarkable result in Alexandrov geometry that dates back to ???.
The proof is very well written by Alexandrov, but we decided to include its sketch here.
This appendix was written by Nina Lebedeva and the second author.

Let us give a list of available texts on Alexandrov spaces with curvature bounded below: 
\begin{itemize}
\item The first introduction to Alexandrov geometry is given in the original paper of Yuriy Burago, Michael Gromov, and Grigory Perelman \cite{burago-gromov-perelman} 
and its extension \cite{perelman1991} written by Perelman.
\item A brief and reader-friendly introduction was written by Katsuhiro Shiohama \cite[Sections 1--8]{shiohama}.
\item Another reader-friendly introduction, written by Dmiti Burago, Yuriy
Burago, and Sergei Ivanov \cite[Chapter 10]{burago-burago-ivanov}.
\item Survey by Conrad Plaut \cite{plaut:survey}.
\item Survey by the second author \cite{petrunin:survey}.
\end{itemize}

\parbf{Acknowledgments.}
Our notes were shaped in a number of lectures given by the authors
at different occasions in Penn State, including the MASS program,
at the Summer School ``Algebra and Geometry'' in Yaroslavl,
at SPbSU,
and University of Toronto.
We want to thank these institution for hospitality and support.

We were partially supported by the following grants:
Vitali Kapovitch ---   NSERC Discovery grants;
Anton Petrunin --- 
NSF grant DMS-2005279. %??? check!!!


