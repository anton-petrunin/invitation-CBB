\chapter*{Preface}
\addcontentsline{toc}{chapter}{Preface}

This book is similar to our ``Invitation to Alexandrov geometry'' written jointly with Stephanie Alexander \cite{alexander-kapovitch-petrunin-2019}.
We try to demonstrate the beauty and power of Alexandrov geometry by reaching interesting applications and theorems with minimal preparation.
This time we do spaces with curvature bounded below in the sense of Alexandrov.

This subject is more technical; it takes more preparation, and we had to jump over several proofs.
Namely, we do not prove the existence part in generalized Picard's theorem (\ref{thm:glob-exist-grad-curv})
and Perelman's theorem about conic neighborhoods (\ref{thm:spherical-nbhd});
up to the last lecture, the rest is nearly rigorous.
Several times, proofs of important statements are left as exercises,
but all of them are solved at the end of the book;
in all these cases, the statement is more important than its proof.


\medskip

In Lecture~\ref{chap:prelim}, we discuss necessary preliminaries and fix notations.

Lecture~\ref{chap:defs} introduces the main object of our study --- spaces with curvature bounded below in the sense of Alexandrov.

In Lecture~\ref{chap:globalization}, we formulate and prove the globalization theorem that local Alexandrov condition implies global.
To simplify the presentation we consider only the compact case.
This case is leading, it gives the main ideas of the proof but is less technical.

In Lecture~\ref{chap:derivative}, we develop calculus --- tangent space and space of directions, differential, and gradient.

Lecture~\ref{chap:GF} introduces gradient flow, which will be further used as the main technical tool.

Lecture~\ref{chap:splitting} proves the line splitting theorem.
Furthermore, we introduce and study  linear subspaces of tangent spaces.

In Lecture~\ref{chap:dim}, we introduce linear dimension and volume.
Further, we prove the Bishop--Gromov inequality and the right-inverse theorem,
introduce  distance charts, and show that all reasonable types of dimension are the 
same for Alexandrov spaces.

In Lecture~\ref{chap:lim} we show that a lower curvature bound survives under Gromov--Hausdorff limits and prove Gromov's selection theorem.
Further, we present Perelman's construction of strictly concave functions and apply it with Gromov's selection theorem to prove the homotopy finiteness theorem.
This proof illustrates one of the main sources of applications of Alexandrov geometry.

In Lecture~\ref{chap:bry}, we introduce the boundary of finite-dimensional Alexandrov spaces and prove the doubling theorem.

In Lecture~\ref{chap:L/G}, we show that quotients of Alexandrov spaces by isometric group action are Alexandrov spaces.
This gives another source of applications of Alexandrov geometry, several examples are given.


Finally, in Lecture~\ref{chap:surfaces} we briefly discuss convex surfaces in Euclidean space; this subject is the main precursor to the modern Alexandrov geometry.

Exercises that are used later in the sequel are marked with an exclamation point: \textbf{Exercise!}

Let us list available texts on Alexandrov spaces with curvature bounded below:
\begin{itemize}
\item The first introduction to Alexandrov geometry of all dimensions is given in the original paper by Yuriy Burago, Michael Gromov, and Grigory Perelman \cite{burago-gromov-perelman} 
and its extension \cite{perelman1991} written by Perelman.
\item A brief and reader-friendly introduction was written by Katsuhiro Shiohama \cite[Sections 1--8]{shiohama}.
\item Another reader-friendly introduction, written by Dmitri Burago, Yuriy
Burago, and Sergei Ivanov \cite[Chapter 10]{burago-burago-ivanov}.
\item Survey by Conrad Plaut \cite{plaut:survey}.
\item Survey by the third author \cite{petrunin:survey}.
\item Our book written jointly with Stephanie Alexander \cite{alexander-kapovitch-petrunin2024}.
\end{itemize}

\parbf{Acknowledgments.}
The idea for this book emerged in collaboration with Stephanie Alexander, and it drew extensively from another book written with her \cite{alexander-kapovitch-petrunin2024}.
Our notes were shaped in a series of lectures given by the authors on various occasions: at Penn State, including the MASS program; at the Summer School ``Algebra and Geometry'' in Yaroslavl; at SPbSU; and at the University of Toronto.
We are grateful to these institutions for their hospitality and support.

We were partially supported by the following grants:
Vitali Kapovitch ---   NSERC Discovery grants;
Anton Petrunin --- 
NSF grant DMS-2005279. %??? check!!!


