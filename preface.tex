\chapter*{Preface}

As in our previous invitation \cite{alexander-kapovitch-petrunin-2019} written jointly with Stephanie Alexander,
we try to demonstrate the beauty and power of Alexandrov geometry by reaching interesting applications and theorems with minimal preparation.
This time we do spaces with curvature bounded below in the sense of Alexandrov.
We have extensively used another book of us with Stephanie Alexander \cite{alexander-kapovitch-petrunin2024}.

This subject is more technical; it takes more preparation, and we had to jump over some proofs.
Namely, we skip the proof of existence part in generalized Picard's theorem (\ref{thm:glob-exist-grad-curv})
and Perelman's theorem about conic neighborhoods (\ref{thm:spherical-nbhd});
the rest is nearly rigorous.
Some important statements stated as exercises, but they are nearly solved in hints at the end of the book.

\medskip 

In Lecture~\ref{chap:prelim}, we discuss necessary preliminaries and fix notations.

Lecture~\ref{chap:defs} introduces the main object of our study --- spaces with curvature bounded below in the sense of Alexandrov.

In Lecture~\ref{chap:globalization}, we formulate and prove the globalization theorem --- local Alexandrov condition implies global.
To simplify the presentation, we consider only the compact case, but this case is leading.

In Lecture~\ref{chap:derivative}, we do beginning of calculus --- tangent space and space of directions, differential, and gradient.

Lecture~\ref{chap:GF} introduces gradient flow, which will be further used as the main technical tool.

Lecture~\ref{chap:splitting} proves the line splitting theorem, providing the first application of gradient flow.
Furthermore, we introduce and study the linear subspace of tangent space.

In Lecture~\ref{chap:dim}, we introduce linear dimension and volume.
Further, we prove the Bishop--Gromov inequality and the right-inverse theorem,
introduce the distance chart, and show that all reasonable types of dimension are the 
same for Alexandrov spaces.

Lecture~\ref{chap:lim} shows that a lower curvature bound survives in the Gromov--Hausdorff limit and proves Gromov's selection theorem.
Further, we present Perelman's construction of strictly concave functions and apply it with Gromov's selection theorem to prove the homotopy finiteness theorem.
This proof illustrates the main source of applications of Alexandrov geometry.

In Lecture~\ref{chap:bry}, we introduce the boundary of finite-dimensional Alexandrov spaces and prove the doubling theorem.

In Lecture~\ref{chap:L/G}, we show that quotients of Alexandrov spaces by isometric group action are Alexandrov spaces and give several applications of this statement.
This is another source of applications of Alexandrov geometry.

Lecture~\ref{chap:convex-body} brings us back to the original object of study of Alexandrov.
We show that the surface of a convex body in Euclidean space is an Alexandrov space.
This is historically the first source of applications of Alexandrov geometry.

Finally, Appendix~\ref{chap:embedding} sketches Alexandrov's embedding theorem of convex polyhedra.
Historically, this theorem is the first remarkable result in Alexandrov geometry, dating back to 1941.
The proof is very well written by Alexandrov, but we decided to include its sketch here due to its beauty and importance.
This appendix was written by Nina Lebedeva and the second author for  for a book about St. Petersburg mathematicians and their discoveries \cite{lebedeva-petrunin}.

Let us give a list of available texts on Alexandrov spaces with curvature bounded below: 
\begin{itemize}
\item The 2-dimensional theory is treated in the classical book of Alexandr Alexandrov \cite{alexandrov-1948}.
\item The first introduction to Alexandrov geometry of all dimensions is given in the original paper by Yuriy Burago, Michael Gromov, and Grigory Perelman \cite{burago-gromov-perelman} 
and its extension \cite{perelman1991} written by Perelman.
\item A brief and reader-friendly introduction was written by Katsuhiro Shiohama \cite[Sections 1--8]{shiohama}.
\item Another reader-friendly introduction, written by Dmitri Burago, Yuriy
Burago, and Sergei Ivanov \cite[Chapter 10]{burago-burago-ivanov}.
\item Survey by Conrad Plaut \cite{plaut:survey}.
\item Survey by the second author \cite{petrunin:survey}.
\end{itemize}

\parbf{Acknowledgments.}
Our notes were shaped in a number of lectures given by the authors
on different occasions at Penn State, including the MASS program,
at the Summer School ``Algebra and Geometry'' in Yaroslavl,
at SPbSU,
and the University of Toronto.
We want to thank these institutions for hospitality and support.

We were partially supported by the following grants:
Vitali Kapovitch ---   NSERC Discovery grants;
Anton Petrunin --- 
NSF grant DMS-2005279. %??? check!!!


