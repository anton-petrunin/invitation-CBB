\chapter*{Preface}

This is our second \textit{invitation} to Alexandrov geometry.
As in our previous book \cite{alexander-kapovitch-petrunin-2019},
we try to demonstrate reach interesting applications and theorems with a minimum of preparation.
This time we do it for spaces with curvature bounded below.
Namely,  we consider $\CBB$ spaces --- the metric spaces with curvature bounded below in the sense of Alexandrov.

This time we had to jump over couple of technical results,
namely generalized Picard's theorem and theorem about conic neighborhoods.




This subject is more technical,

These notes were shaped in a number of lectures given by the third author
at different occasions in Penn State, including the MASS program,
at the Summer School ``Algebra and Geometry'' in Yaroslavl,
and at SPbSU.

\medskip 

In Lecture~\ref{chap:prelim}, we discuss necessary preliminaries and fix notations.

Lecture~\ref{chap:defs} introduces the main spaces with curvature bounded below in the sense of Alexandrov.

In Lecture~\ref{chap:globalization} we formulate and prove the globalization theorem --- local Alexandrov condition implies global.
To simplify the presentation we consider only compact case,
but in the final remarks in this lecture we formulate the most general versions.

In Lecture~\ref{chap:derivative} we do beginning of calculus --- tangent space and space of directions, differential, and gradient.

Lecture~\ref{chap:GF} introduces gradient flow --- this is the main technical tool in the theory.

Lecture~\ref{chap:splitting} proves the line splitting theorem.

In Lecture~\ref{chap:dim} we introduce and discuss dimension of Alexandrov spaces.

Lecture~\ref{chap:lim} shows that lower curvature bound survives in the Gromov--Hausdorff limit and proves the Gromov's selection theorem --- the main source of applications of the theory.


Lecture~\ref{chap:stability} starts with Perelman's construction of concave functions.
Further, it applies it with Gromov's selection theorem to prove the homotopy finiteness theorem --- any class of Riemanian $m$-manifolds with lower sectional curvature bound and upper diameter bound contains at most finite number of homotopy types.

Lecture~\ref{chap:L/G} we show that quotient Alexandrov space by isometric group action is an Alexandrov space and give several applications of this statement.

In Lecture~\ref{chap:bry} we introduce boundary of finite-dimensioanal Alexandrov space and prove the doubling theorem.

In Lecture~\ref{chap:convex-body} we show that surface of a convex body in Euclidean space is an Alexandrov space. This is historically the first serious application of Alexandrov geometry.

Finally, in Appendix \ref{chap:alex-embedding}
 we sketch Alexandrov embedding theorem of convex polyhedra.

\medskip

Here is a list of some sources providing a good introduction to Alexandrov spaces with curvature bounded above, which we recommend for further information;
we will not assume familiarity with any of these sources.

\begin{itemize}
\item ???
\end{itemize}
