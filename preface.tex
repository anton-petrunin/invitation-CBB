\chapter*{Preface}

As in our previous book \cite{alexander-kapovitch-petrunin-2019},
we try to demonstrate reach interesting applications and theorems with a minimum of preparation.
This time we do it for spaces with curvature bounded below.
Namely,  we consider $\CBB$ spaces --- the metric spaces with curvature bounded below in the sense of Alexandrov.

This subject is more technical, so we could not make our presentation completely rigorous --- we jumped over proofs couple of technical results,
namely generalized Picard's theorem and theorem about conic neighborhoods.

\medskip 

In Lecture~\ref{chap:prelim}, we discuss necessary preliminaries and fix notations.

Lecture~\ref{chap:defs} introduces the main spaces with curvature bounded below in the sense of Alexandrov.

In Lecture~\ref{chap:globalization} we formulate and prove the globalization theorem --- local Alexandrov condition implies global.
To simplify the presentation we consider only compact case,
but in the final remarks in this lecture we formulate the most general versions.

In Lecture~\ref{chap:derivative} we do beginning of calculus --- tangent space and space of directions, differential, and gradient.

Lecture~\ref{chap:GF} introduces gradient flow --- this is the main technical tool in the theory.

Lecture~\ref{chap:splitting} proves the line splitting theorem.

In Lecture~\ref{chap:dim} we introduce and discuss dimension of Alexandrov spaces.

Lecture~\ref{chap:lim} shows that lower curvature bound survives in the Gromov--Hausdorff limit and proves the Gromov's selection theorem --- the main source of applications of the theory.


Lecture~\ref{chap:stability} starts with Perelman's construction of concave functions.
Further, it applies it with Gromov's selection theorem to prove the homotopy finiteness theorem --- any class of Riemanian $m$-manifolds with lower sectional curvature bound and upper diameter bound contains at most finite number of homotopy types.

Lecture~\ref{chap:L/G} we show that quotient Alexandrov space by isometric group action is an Alexandrov space and give several applications of this statement.

In Lecture~\ref{chap:bry} we introduce boundary of finite-dimensioanal Alexandrov space and prove the doubling theorem.

In Lecture~\ref{chap:convex-body} we show that surface of a convex body in Euclidean space is an Alexandrov space. This is historically the first serious application of Alexandrov geometry.

Finally, in Appendix \ref{chap:alex-embedding}
 we sketch Alexandrov embedding theorem of convex polyhedra.

\medskip

Here is a list of some sources providing a good introduction to Alexandrov spaces with curvature bounded below, which we recommend for further information;
we will not assume familiarity with any of these sources.

Let us give a list of available texts on Alexandrov spaces with curvature bounded below: 
\begin{itemize}
\item The first introduction to Alexandrov geometry is given in the original paper of Yuriy Burago, Michael Gromov, and Grigory Perelman \cite{burago-gromov-perelman} 
and its extension \cite{perelman1991} written by Perelman.
\item A brief and reader-friendly introduction was written by Katsuhiro Shiohama \cite[Sections 1--8]{shiohama}.
\item Another reader-friendly introduction, written by Dmiti Burago, Yuriy
Burago, and Sergei Ivanov \cite[Chapter 10]{burago-burago-ivanov}.
\item Survey by Conrad Plaut \cite{plaut:survey}.
\item Survey by the second author \cite{petrunin:survey}.
\end{itemize}

\parbf{Acknowledgments.}
Our notes were shaped in a number of lectures given by the authors
at different occasions in Penn State, including the MASS program,
at the Summer School ``Algebra and Geometry'' in Yaroslavl,
and at SPbSU + ???
We want to thank these institution for hospitality and support.


