%%!TEX root = the-hopf.tex
\chapter{Boundary}\label{chap:bry}

\section{Definition}

Let us give an inductive definition the boundary for finite-dimensional Alexandrov spaces.

Suppose $\spc{L}$ is a 1-dimensional Alexandrov space.
By Exercise \ref{ex:dim=1},
$\spc{L}$ is homeomorphic to a 1-dimensional manifold (possibly with nonempty boundary).
It allows us to define the boundary $\partial\spc{L}\subset \spc{L}$ as the boundary of a manifold.

Now assume that the notion of boundary is already defined in dimensions $1,\dots,m-1$.
Suppose  $\spc{L}$ is $m$-dimensional Alexandrov space.
We say that $p\in \spc{L}$ belongs to the boundary (briefly $p\in \partial \spc{L}$) if 
$\partial\Sigma_p\ne\emptyset$.
By \ref{thm:finite-space-of-directions} and \ref{ex:finite-space-of-directions-dim}, $\Sigma_p$ is $(m-1)$-dimensional Alexandrov space;
therefore its boundary is already defined.

\begin{thm}{Exercise}\label{ex:bry-convex}
Show that for closed convex set $K\subset \EE^m$ with nonempty interior, the topological boundary of $K$ as a subset of $\EE^m$ coincides with the boundary $K$ described above.
\end{thm}

\begin{thm}{Exercise}\label{ex:bry-product}
Let $\spc{L}$ be a finite-dimensional Alexandrov space.
Suppose $\spc{L}\iso\spc{L}_1\times\spc{L}_2$.
Show that 
\[\partial \spc{L}=(\partial\spc{L}_1\times\spc{L}_2)\,\cup\,(\spc{L}_1\times\partial\spc{L}_2).\]
\end{thm}


\section{Conic neighborhoods}

The following statement is close relative of Perelman's stability theorem \ref{thm:stability}, but its proof is slightly simpler \cite{perelman:morse}.
We are going to use this result without proof.

Recall that logarithm $\log_px\:\spc{L}\to \T_p$ is defined on page \pageref{page:log}.

\begin{thm}{Theorem}\label{thm:spherical-nbhd}
Any point $p$ in a finite-dimensional Alexandrov space $\spc{L}$
and all sufficiently small $\eps>0$
there is a homeomorphism $h_\eps\:\oBall(p,\eps)_{\spc{L}}\to \oBall(0,\eps)_{\T_p}$ such that $0=h_\eps(p)$.

Moreover, we may assume that 
\[
\sup_{x\in \oBall(p,\eps)}\{\,\tfrac1\eps\cdot\dist{\log_px}{h_\eps(x)}{\T_p}\,\}\to 0
\quad\text{as}\quad
\eps\to 0.\]
\end{thm}

This statement is often used together with \textit{uniqueness of conic neighborhood} stated below.

Suppose that a neighborhood $U$ of a point $x$ in a metric space $\spc{X}$
admits a homeomorphism to $\Cone\Sigma$ such that $x$ maps to the origin of the cone.
In this case, we say that $U$ has a \index{conic neighborhood}\emph{conic neighborhood} of~$x$.

\begin{thm}{Uniqueness of conic neighborhood}\label{lem:kwun}
Any two conic neighborhoods of a given point in a metric space are \index{pointed homeomorphic}\emph{pointed homeomorphic}; that is, there is a homeomorphism between neighborhoods that maps the origin of the cone to the origin.
\end{thm}

\begin{thm}{Advanced exercise}\label{ex:conic}
Prove \ref{lem:kwun} or read the original proof by Kyung Whan Kwun \cite{kwun1964}.
\end{thm}


\begin{thm}{Exercise}\label{ex:conic-tangent}
Suppose $x\mapsto x'$ is a homeomorphism between finite-dimensional Alexandrov spaces $\spc{L}$ and $\spc{L}'$. Show that 

\begin{subthm}{ex:conic-tangen:tangent}
$\T_x\cong \T_{x'}$,
\end{subthm}

\begin{subthm}{ex:conic-tangen:dir}
$\Susp\Sigma_x\cong \Susp\Sigma_{x'}$.
\end{subthm}

\begin{subthm}{ex:conic-tangen:example}
but in general $\Sigma_x\not\cong\Sigma_{x'}$.
\end{subthm}

\end{thm}



\section{Topology}

The following theorem states that boundary is a topological property
despite our definition used geometry.

\begin{thm}{Theorem}\label{thm:top-bry}
Let $\spc{L}$ and $\spc{L}'$ be homeomorphic finite-dimensional Alexandrov spaces.
Then $\dim \spc{L}=\dim\spc{L}'$ and
\[\partial\spc{L}\ne \emptyset
\quad\iff\quad
\partial\spc{L}'\ne \emptyset
\]
\end{thm}


Before diving into the proof, it is instructive to solve the following exercise.

\begin{thm}{Exercise}\label{ex:nonstability}
Construct two plane subsets $K_1$ and $K_2$ such that $K_1\ncong K_2$,
but $\RR\times K_1\cong \RR\times K_2$.
\end{thm}

Let $\spc{L}$ be an $m$-dimensional Alexandrov space and $m<\infty$.
Define \index{rank}\emph{rank} of $\spc{L}$ (briefly, \index{$\rank\spc{L}$ (rank)}$\rank\spc{L}$) as the minimal value $k$ such that $\spc{L}\iso\RR^{m-k}\times \spc{K}$,
where $\spc{K}$ is a $k$-dimensional Alexandrov space.

In the following proof we will apply induction on the rank of space.


\parit{Proof.}
The first statement follows from \ref{thm:dim=dim}.

Suppose we have a counterexample, say $\partial \spc{L}\ne \emptyset$, but $\partial \spc{L}'=\emptyset$.
Let $k\df\rank \spc{L}$ and $k'\df\rank \spc{L}'$.
We can assume that the pair $(k,k')$ is minimal in the lexicographic order;
in particular, $k$ is minimal.
Let $x\mapsto x'$ be a homeomorphism from $\spc{L}$ to $\spc{L}'$.

Choose $x\in \partial \spc{L}$.
Since $\partial \spc{L}'=\emptyset$, we have $x'\notin \partial \spc{L}'$.
Note that 
\[\rank \T_x\le k
\quad\text{and}\quad
\rank \T_{x'}\le k',
\]
By \ref{ex:conic-tangen:tangent}, $\T_x\cong\T_{x'}$.
Note that $\partial \T_x\ne\emptyset$ and $\partial \T_{x'}=\emptyset$.
Therefore, we may assume that $\spc{L}$ and $\spc{L}'$ are Euclidean cones
and the homeomorphism sends the origin to the origin.
The remaining part of the proof is divided in three cases.

\parit{Case 1.}
Suppose $k>1$.
Let $\spc{L}\iso \RR^{m-k}\times \spc{C}$, where $\spc{C}$ a $k$-dimensional $\Alex0$ cone.
Note that $\rank\T_y\le\rank\spc{L}$ for any $y\in\spc{L}$ and the equality holds only if $y$ projects to the origin of $\spc{C}$.

Since $k>1$ we can find $z\in\partial\spc{C}$ such that $z\ne 0$.
Choose $y$ that projects to $z$, so $\rank\T_y<\rank\spc{L}$.
By \ref{ex:conic-tangen:tangent}, $\T_y\cong\T_{y'}$,
$\partial  \T_y\ne\emptyset$ and $\partial \T_{y'}=\emptyset$.
The latter contradicts contradicts minimality of $k$.

\parit{Case 2.} Suppose $k\le1$ and $k'>1$.
Since $\partial \spc{L}\ne \emptyset$, we get that $k=1$ and $\spc{L}=\RR^{m-1}\times\RR_{\ge0}$.

Let $\spc{L}'\iso \RR^{m-k'}\times \spc{C}'$, where $\spc{C}'$ a $k'$-dimensional $\Alex0$ cone.
Since $\partial\spc{L}\cong\RR^{m-1}$,
the image of $\partial\spc{L}$ in $\spc{L}'$ does not lie in $\RR^{m-k'}\z\times\{0\}$.
In other words, we can choose $y\in \partial \spc{L}$ such that its image $y'\in \spc{L}'$ has nonzero projection in $\spc{C}'$.
Observe that $\T_y\cong\T_{y'}$,
\[
\rank\T_y\le k=1,
\quad
\rank\T_{y'}< k',
\quad
\partial \T_y=\emptyset,
\quad\text{and}\quad
\partial \T_{y'}\ne \emptyset\]
--- a contradiction.

\parit{Case 3.}
Suppose $k\le 1$ and $k'\le 1$.
Since $\partial \spc{L}\ne \emptyset$, $k=1$.
By \ref{ex:dim=1}, $\spc{L}\cong \RR_{\ge0}$.
Therefore, $\spc{L}'\cong\RR$, and $\spc{L}\not\cong\spc{L}'$ --- a contradiction.
\qeds

\begin{thm}{Exercise}\label{ex:bry2bry}
Let $x\mapsto x'$ be a homeomorphism $\Omega\to\Omega'$
between open subsets in finte-dimensional Alexandrov spaces $\spc{L}$ and $\spc{L}'$.
Show that $x\in \partial \spc{L}$ if and only if $x'\in \partial \spc{L}'$.

\end{thm}

\begin{thm}{Exercise}\label{ex:bry-closed}
Show that boundary of a finite-dimensional Alexandrov space is a closed subset.
\end{thm}

\section{Tangent space}

Let $X$ be a subset in a finite-dimensional Alexandrov space $\spc{L}$.
Choose $p\in \spc{L}$ and $\xi\in \Sigma_p$.
Suppose $\xi$ is a limit of directions $\dir{p}{x_n}$ for a sequence $x_1,x_2,\dots{}\in X$ that converges to $p$.
Then we say that $\xi$ is in the \index{space of directions!of a subset}\emph{space of directions} from $p$ to $X$;
briefly \index{$\Sigma_pX$ (space of directions)}$\xi\in\Sigma_pX$.

Further, the cone $\Cone(\Sigma_pX)$ will be called \index{tangent space!of a subset}\emph{tangent space} to $X$ at $p$;
it will be denoted by \index{$\T_pX$ (tangent space)}$\T_pX$.

\begin{thm}{Theorem}\label{thm:partial-Sigma}
For any finite-dimensional Alexandrov space $\spc{L}$, we have
\[\partial (\Sigma_p\spc{L})=\Sigma_p(\partial\spc{L})
\quad\text{and}\quad
\partial(\T_p\spc{L})=\T_p(\partial\spc{L}).\]
\end{thm}

\parit{Proof.}
Choose a sequence $x_n\in \partial \spc{L}$ such that $x_n\to p$ and $\dir p{x_n}\to\xi$.

Let $\eps_n=2\cdot \dist{p}{x_n}{}$,
and let $h_{\eps_n}\:\oBall(p,\eps_n)_{\spc{L}}\to \oBall(0,\eps_n)_{\T_p}$ be the homeomorphisms provided by \ref{thm:spherical-nbhd};
in particular, $\tfrac2{\eps_n}\cdot h_{\eps_n}(x_n)\to \xi$ as $n\to\infty$.
By \ref{ex:bry2bry}, $h_{\eps_n}(x_n)\in \partial \T_p$.
By \ref{ex:bry-closed}, $\xi\in \partial \T_p$.
Therefore,
\[\partial (\Sigma_p\spc{L})\supset\Sigma_p(\partial\spc{L})
\quad\text{and}\quad
\partial(\T_p\spc{L})\supset\T_p(\partial\spc{L}).\]

Similarly, choose $\xi\in\partial\Sigma_p$.
Let $h_{\eps_n}\:\oBall(p,\eps_n)_{\spc{L}}\to \oBall(0,\eps_n)_{\T_p}$ be the homeomorphisms provided by \ref{thm:spherical-nbhd} for a sequence $\eps_n\to 0$ as $n\to\infty$.
By \ref{ex:bry2bry}, $x_n=h_{\eps_n}^{-1}(\tfrac{\eps_n}2\cdot\xi)\in \partial \T_p$.
By \ref{thm:spherical-nbhd}, $\dir p{x_n}\to \xi$.
Hence
\[\partial (\Sigma_p\spc{L})\subset\Sigma_p(\partial\spc{L})
\quad\text{and}\quad\partial(\T_p\spc{L})\subset\T_p(\partial\spc{L}).\]
\qedsf

\section{Doubling}

Let $A$ be a closed subset in a metric space $\spc{X}$.
The \index{doubling}\emph{doubling} $\spc{W}$ of $\spc{X}$ across $A$ is two copies of $\spc{X}$ glued along $A$;
more precisely, the underlying set of $\spc{W}$ is the quotient $\spc{X}\times\{0,1\}/\sim$, where $(a,0)\sim (a,1)$ for any $a\in A$ and it has minimal metric such that both maps $\spc{X}\to \spc{W}$ defined by $x\mapsto (x,0)$ and $x\mapsto (x,1)$ are distance-preserving.

The metric on $\spc{W}$ can be also defined explicitly by
\[\dist{(x,i)}{(y,j)}{\spc{W}}=
\begin{cases}
\dist{x}{y}{\spc{X}}&\text{if}\quad i= j.
\\
\inf\set{\dist{x}{a}{\spc{X}}+\dist{y}{a}{\spc{X}}}{a\in A}&\text{if}\quad i\ne j.
\end{cases}
\]

The last part of the following statement is the so-called \index{doubling theorem}\emph{doubling theorem}.

\begin{thm}{Theorem}\label{thm:doubling}
Let $\spc{L}$ be a finite-dimensional Alexandrov space with nonempty boundary.
Suppose $f\z=\tfrac12\cdot\distfun_p^2$ for some $p\in \spc{L}$.
Then

\begin{subthm}{thm:doubling:concave}
If $\dim \spc{L}\ge 2$, then
$\distfun_{\partial \Sigma_x}(\xi)\le \tfrac\pi2$ for any $x\in\partial \spc{L}$ and $\xi\in \Sigma_x$.
Moreover, if $\distfun_{\partial \Sigma_x}(\xi)= \tfrac\pi2$, then $\mangle(\xi,\zeta)\le\tfrac\pi2$ for any $\zeta\in \Sigma_x$. 
\end{subthm}

\begin{subthm}{thm:partial-grad:grad}
$\nabla_xf\in \partial\T_x$ for any $x\in\partial \spc{L}$.
\end{subthm}

\begin{subthm}{thm:partial-grad:flow}
If $\alpha$ is an $f$-gradient curve that starts at $x\in \partial \spc{L}$, then $\alpha(t)\in \partial \spc{L}$ for any $t$.
Moreover, if $p\in \partial \spc{L}$, then $\gexp_p(v)\in \partial \spc{L}$ for any $v\in\partial\T_p$.
\end{subthm}

\begin{subthm}{thm:doubling:doubling}
The doubling $\spc{W}$ of $\spc{L}$ across $\partial \spc{L}$ is a Alexandrov space with the same curvature bound.
\end{subthm}

\end{thm}

\parit{Proof.}
We will denote by 
\ref{SHORT.thm:doubling:concave}$_m,\dots,$\ref{SHORT.thm:doubling:doubling}$_m$ the corresponding statement assuming $m=\dim\spc{L}$.

The proof goes by induction on $m$.
Note that \ref{SHORT.thm:doubling:doubling}$_1$ follows from \ref{ex:dim=1} --- this is the base of induction.
The step of induction is a combination of several implications listed below.

\parit{\ref{SHORT.thm:doubling:doubling}$_{m-1}\Rightarrow$\ref{SHORT.thm:doubling:concave}$_m$.}
If $m=2$, then $\dim\Sigma_x=1$; see \ref{ex:finite-space-of-directions-dim}.
By \ref{ex:dim=1}, $\Sigma_x$ isometric to a line segment $[0,\ell]$;
we need to show that $\ell\le\pi$.

Suppose $\ell>\pi$, then the tangent space $\T_x=\Cone\Sigma_x$ has several different lines thru the origin.
Recall that $\T_x$ is a Alexandrov space; see \ref{ex:finite-tan}.
By \ref{cor:splitting}, $\T_x$ is isometric to the Euclidean plane;
the latter contradicts that $\Sigma_x$ is a line segment.

Let $m=\dim \spc{L}>2$, so $\dim \Sigma_x>1$.
Suppose $\distfun_{\partial \Sigma_x}(\xi)> \tfrac\pi2$ for some $\xi$.
By \ref{SHORT.thm:doubling:doubling}$_{m-1}$, the doubling $\Xi$ of $\Sigma_x$ is $\Alex1$.
Denote by $\xi_0$ and $\xi_1$ the points in $\Xi$ that correspond to $\xi$.
Observe that $\dist{\xi_0}{\xi_1}{\Xi}>\pi$.
The latter contradicts \ref{ex:RisCBB(1)}.

Finally, if $\distfun_{\partial \Sigma_x}(\xi)= \tfrac\pi2$, then $\dist{\xi_0}{\xi_1}{\Xi}=\pi$.
Therefore $\Cone \Xi$ contains a line in directions of $\xi_0$ and $\xi_1$;
in other words, $\Xi$ is a spherical suspension with poles $\xi_0$ and $\xi_1$.
In particular, every point of $\Xi$ lies on distance at most $\tfrac\pi2$ from $\xi_0$ or $\xi_1$.
The natural projection $\Xi\to \Sigma_x$ does not increase distances and sends both  $\xi_0$ and $\xi_1$ to $\xi$.
Therefore, the second statement follows.

\parit{\ref{SHORT.thm:doubling:doubling}$_{m-1}+$\ref{SHORT.thm:doubling:concave}$_{m-1}+$\ref{SHORT.thm:doubling:concave}$_m\Rightarrow$\ref{SHORT.thm:partial-grad:grad}$_m$.}
We can assume that $s=\nabla_xf\ne 0$.
By \ref{prop:grad-exist}, $\nabla_xf\z=s\cdot \overline{\xi}$, where $s=\dd_xf(\overline{\xi})$ and $\overline{\xi}\in\Sigma_p$ is the direction that maximize $\dd_xf(\overline{\xi})$.

Let $\zeta\in \partial\Sigma_x$ be a direction that minimize the angle $\mangle(\overline{\xi},\zeta)$.
It is sufficient to show that $\zeta=\overline{\xi}$.

Assume $\zeta\ne \overline{\xi}$;
let $\eta=\dir\zeta{\overline{\xi}}$.
By \ref{SHORT.thm:doubling:concave}$_m$, $\mangle(\overline{\xi},\zeta)\le \tfrac\pi2$ and
\ref{SHORT.thm:doubling:concave}$_{m-1}$ implies that 
\[\mangle(\eta,\nu)\le \tfrac\pi2\eqlbl{eq:<pi/2}\]
for any $\nu\in \Sigma_\zeta\Sigma_x$; if $m=2$, then the last statement is evident. 

Consider function $\phi\:\Sigma_x\to\RR$ defined by $\phi(\xi)\df\dd_xf(\xi)$.
Applying \ref{ex:d(distfun):<} and \ref{eq:<pi/2}, we get that $\dd_\xi\phi(\eta)\le 0$.
Since $\dd_xf$ is convex, we have that $\phi''+\phi\le 0$.
If $\phi(\zeta)\le 0$, then it implies that $\phi(\overline{\xi})\le 0$ --- a contradiction.
If $\phi(\zeta)> 0$, then $\phi(\overline{\xi})<\phi(\zeta)$ --- a contradiction again.

\parit{\ref{SHORT.thm:partial-grad:grad}$_m\Rightarrow$\ref{SHORT.thm:partial-grad:flow}$_m$.}
Let $\alpha$ be an $f$-gradient curve and $\ell(t)=\distfun_{\partial L}\alpha(t)$.
Note that $\ell$ is a Lipschitz function.

Choose $t$;
let $x=\alpha(t)$ and $y\in \partial\spc{L}$ be a closest point to $x$.
By \ref{SHORT.thm:partial-grad:grad}$_m$, we have that $\nabla_y f\in\partial \T_y$.
Since the distance $\dist{x}{y}{}$ is minimal, 
we get $\langle \dir yx,v\rangle\le 0$ for any $v\in \partial \T_y$.
In particular,
\[\langle \dir yx,\nabla_y f\rangle\le 0\]
Applying Exercise~\ref{ex:monotonicity} to $x$ and $y$, 
we get
\[\ell'(t)\le \ell(t)\]
if the left-hand side is defined.
Since $\ell$ is Lipschitz, $\ell'$ is defined almost everywhere.
Integrating the inequality, we get 
\[\ell(t)\le e^t\cdot\ell(0)\]
for any $t\ge 0$.
In particular, if $\ell(0)=0$, then $\ell(t)=0$ for any $t\ge 0$.
Since $\partial\spc{L}$ is closed (\ref{ex:bry-closed}), the statement follows.

\parit{\ref{SHORT.thm:partial-grad:flow}$_{m}+$\ref{SHORT.thm:doubling:doubling}$_{m-1}\Rightarrow$\ref{SHORT.thm:doubling:doubling}$_m$.}
We will consider the case $\kappa=0$;
other cases can be done the same way, but formulas getting more complicated.

Denote by $\spc{L}_0$ and $\spc{L}_1$ the two copies of $\spc{L}$ in $\spc{W}$;
let us keep the notation $\partial \spc{L}$ for the common boundary of $\spc{L}_0$ and $\spc{L}_1$.
Next claim says that any geodesic in $\spc{W}$ either lies in $\partial \spc{L}$ or it crosses $\partial \spc{L}$ at most once.

\begin{clm}{}
Let $\gamma$ be a geodesic in $\spc{W}$.
Then either $\gamma$ has at most one interior point in $\partial \spc{L}$ or
$\gamma\subset \partial \spc{L}$.
\end{clm}

\begin{wrapfigure}{r}{45mm}
\vskip-2mm
\centering
\includegraphics{mppics/pic-1315}
\end{wrapfigure}

Assume $\gamma$ shares at least two points with $\partial \spc{L}$, say $x=\gamma(t_1)$ and $y=\gamma(t_2)$ and these are not endpoints of $\gamma$.
Remove from $\gamma$ the set $\gamma\cap \spc{L}_1$
and exchange it to its reflection across $\spc{L}_0$;
denote the obtained curve by $\hat\gamma$.

Note that any arc of $\hat\gamma$ with endpoint at $x$ or $y$ is a geodesic in $\spc{L}_0$.
Moreover, the arc of $\hat\gamma$ behind $y$ lies in the image of map $t\mapsto \GF^t_{f_x}(y)$, where $f_x=\tfrac12\cdot\distfun^2_x$.
By \ref{SHORT.thm:partial-grad:flow}, this arc lies in $\partial\spc{L}$.

Now choose a point $z$ on this arc, so $z\in \partial\spc{L}$.
Applying the same argument, we get that the arc of $\hat\gamma$ before $y$ lies in $\partial\spc{L}$.
Hence the claim follows.\claimqeds

Choose a point $p$ in $\spc{W}$;
let $f\df\tfrac12\cdot\distfun_p^2$.
It is sufficient to show that $(f\circ\gamma)''\le 1$ for any $t$.
If $p\in \partial\spc{L}$, then the statement follows from function comparison in $\spc{L}_0$ and $\spc{L}_1$.
So, we can assume that $p\in \spc{L}_0\setminus \partial\spc{L}$.

If $\gamma$ lies in $\partial \spc{L}$, then this inequality follows from the comparison in~$\spc{L}_0$.

\begin{wrapfigure}{r}{45mm}
\vskip-2mm
\centering
\includegraphics{mppics/pic-1325}
\end{wrapfigure}

Choose $y=\gamma(t_0)$; without loss of generality we can assume that $t_0=0$.

If $y\z\in \spc{L}_0\setminus\partial\spc{L}$, then $(f\circ\gamma)''(0)\le 1$ in the barrier sense;
it follows from the comparison in $\spc{L}_0$.

Assume $y\in \spc{L}_1\setminus\partial\spc{L}$.
Suppose $[py]$ crosses $\partial L$ at $x$.
Let $\Sigma_x$ be the space of directions of $\spc{L}$ at $x$
and let $\Xi$ be its doubling.
By \ref{SHORT.thm:doubling:doubling}$_{m-1}$, $\Xi$ is $\Alex1$.

Note that $\dir x{y}$ and $\dir xp$ lie in opposite sides of $\Xi$ and
\[\dist{\dir x{y}}{\dir xp}\Xi\ge \pi.\]
Otherwise, choose a direction $\xi\in\partial\Sigma$ such that
\[\dist{\dir x{y}}{\xi}\Xi+\dist{\xi}{\dir xp}\Xi<\pi,\]
Consider the radial curve $\alpha(t)=\gexp_x(t\cdot \xi)$.
By \ref{SHORT.thm:partial-grad:flow}$_m$, $\alpha$ lies in $\partial L$.
By \ref{prop:gexp}
\[\dist{p}{\alpha(s)}{\spc{L}_0}
+\dist{y}{\alpha(s)}{\spc{L}_1}
<\dist{p}{y}{\spc{W}}\]
for small values $s>0$
--- a contradition.

Note that $\Cone \Xi$ contains a line with directions $\dir x{y}$ and $\dir xp$.
By splitting theorem $\Cone \Xi$ split in these directions;
in particular, 
\[\dist{\dir x{y}}{\xi}{}+\dist{\xi}{\dir xp}{}=\pi.\]
for any $\xi\in\Xi$.
It follows that for any $\xi\in\Xi$ there is $\xi'\in\partial\Sigma_x$ such that 
$\xi$ and $\xi'$ lie on some geodesic $[\dir x{y} \dir xp]_\Xi$.

Choose such $\xi'(t)$ for $\xi(t)=\dir x{\gamma(t}$, where $t\approx 0$.
Consider the radial curve $\alpha_t(s)\df\gexp_x(s\cdot\xi'(t))$.
Applying the comparison and \ref{prop:gexp}, we get that 
\begin{align*}
\dist{p}{\gamma(t)}{\spc{W}}
&\le \dist{p}{\alpha(s)}{\spc{L}_0}+ \dist{\alpha(s)}{\gamma(t)}{\spc{L}_1}\le
\\
&\le \side\hinge yp{\gamma(t)}.
\end{align*}
for suitably chosen parameter $s$. 
Hence we get $(f\circ\gamma)''(0)\le 1$ in the barrier sense.

Finally if $\gamma(0)\in\partial\spc{L}$, then splitting argument shows that 
\[(f\circ\gamma)^+(0)+(f\circ\gamma)^-(0)\le 0.\]

Summarizing, we get that $(f\circ\gamma)''\le 1$ if $\gamma$ lies entirely in $\spc{L}_0$ or $\spc{L}_1$.
If $\gamma$ crosses $\partial \spc{L}$, then we know that it happens only once and at the crossing moment $t_0$ 
we have $f\circ\gamma^+(t_0)+f\circ\gamma^-(t_0)\le 0$.
All this implies that $(f\circ\gamma)''\le 1$.
\qeds

\begin{thm}{Exercise}\label{ex:bry-connected}
Let $\spc{L}$ be a finite-dimensional $\Alex1$ space with nonempty boundary $\partial\spc{L}$.
Show that $\partial\spc{L}$ is connected.
\end{thm}


\begin{thm}{Exercise}\label{ex:dist-to-bry}
Let $\spc{L}$ be a finite-dimensional $\Alex0$ space with nonempty boundary $\partial\spc{L}$.
Show that the distance function to the boundary
\[\distfun_{\partial\spc{L}}\:\spc{L}\to\RR\]
is concave.
\end{thm}

\begin{thm}{Exercise}\label{ex:liberman}
Let $\spc{L}$ be a finite-dimensional $\Alex0$ space with nonempty boundary $\partial\spc{L}$.
Suppose $\gamma$ is a geodesic in $\partial\spc{L}$ with the induced length metric.
Show that the function $t\mapsto \tfrac12\cdot\distfun_p^2\circ\gamma(t)$ is 1-concave for any point $p$. 
\end{thm}

\section{Remarks}

Theorem~\ref{thm:spherical-nbhd} can be used to prove the following. \footnote{\red V:  I would like to add some details here}

\begin{thm}{Topological stratification}\label{thm:top-stratification}
Any $m$-dimensional Alexandrov space with $m<\infty$ can be subdivided into subsets topological manifolds $S_0,\dots,S_m$ such that for every $i$ we have $\dim S_i=i$ or $S_i=\emptyset$.
Moreover,
\begin{subthm}{}
closure of $S_{m-1}$ is the boundary of the space, and
\end{subthm}

\begin{subthm}{}
$S_{m-2}=\emptyset$.
\end{subthm}

\end{thm}

Let us mention that this statement does not imply that (compact) finite-dimensional Alexandrov space is homeomorphic to CW-complex ???
\footnote{\red V: it does imply that the have homotopy type of CW complexes (this also follows from existence of covers by Perelman's convex sets). I don't think it's known whether or not every stratified manifold is homeomorphic to a CW complex}.

Note that doubling of a finite-dimensional Alexandrov space across its boundary is an Alexandrov space without boundary.
This observation can be used to deduce a statemet about general finite-dimensional Alexandrov space to an Alexandrov space without boundary;
so the following tools become available.

\begin{thm}{Fundamental-class lemma}\label{lem:fund-class}
Let $\spc{L}$ be a compact $m$-dimensional Alexandrov space, and $m<\infty$.
Then $\spc{L}$ has a fundamental class in Alexander--Spanier \footnote{\red V: for Alexandrov spaces  Alexander--Spanier  cohomology is the same as singular cohomology }cohomology with $\ZZ/2$ coefficients; 
that is, 
\[H^n(\spc{L},\ZZ/2)=\ZZ/2.\]

\end{thm}

This lemma was proved by Kartsen Grove and Peter Perersen \cite{grove-petersen};
It implies, for example, that gradient is an onto map  for functions of the following type $f=\tfrac12\distfun_p$ is surjective on finite-dimensional Alexandrov spaces.
It is also used in the proof of a generalized domain invariance theorem stated below. 

\begin{thm}{Domain invariance}\label{thm-inv-domain}
Let $\spc{L}_1$ and $\spc{L}_2$ be two $m$-dimensional Alexandrov spaces with empty boundary; $m$ is finite.
Suppose $\Omega_1$ is an open subset in $\spc{L}_1$ and $f\:\Omega_1\to \spc{L}_2$ is an injective continuous map.
Then $f(\Omega_1)$ is open in $\spc{L}_2$.
\end{thm}

Let $f$ be a semiconcave function.
A point $p\in \Dom f$ is called \index{critical point}\emph{critical} point of $f$ if $\dd_pf\le 0$; 
otherwise it is called \index{regular point}\emph{regular}.

The following statement plays technical role in the proof of stability theorem,
but this is also a useful technical tool.

\begin{thm}{Morse lemma}
Let $f$ be a semiconcave function on a finite-dimensional Alexandrov space without boundary.
Suppose $K$ is a compact set of regular points of $f$ in its level set $f=a$.
Then an open neighborhood $\Omega$ of $K$ admits homeomorphism $x\mapsto (h(x),f(x))$ to a product space $\Lambda\times (a-\eps,a+\eps)$.
\end{thm}

Subsets that satisfy condition in \ref{thm:partial-grad:flow} are called \index{extremal subset}\emph{extremal}.
More precisely, a subset $E$ in an Alexandrov space is called extremal if for any $x\in E$
and $f$-gradient curve that starts in $E$ remains in $E$;
here $f$ is arbitrary function of the form $\tfrac12\cdot \distfun_p^2$.

Extremal subsets were introduced by Grigori Perelman and the second author \cite{perelman-petrunin}.
They play an important role in Alexandrov geometry and its applications;
in particular they will pop up in the next lecture.

The following conjecture is one of the oldest questions in Alexandrov geometry that remains open.

\begin{thm}{Conjecture}
Let $S$ be a component of the boundary of a finite-dimensional Alexandrov space.
Then $S$ equipped with the induced length metric is an Alexandrov space with the same curvature bound.
\end{thm}

\section{Comments}

Invariance of domain theorem holds in more generality for stratified spaces \cite[Theorem 3.2]{kapovitch-zhu}.

Topological stratification theorem ~\ref{thm:top-stratification} can be sharpened as follows

\begin{thm}{Chjaracterization of Boundary}\label{thm:boundary-kd}
Let $\spc{L}$ an Alexandrov space of dimension $m$  and let $p \in \spc{L}$.
Then the following are equivalent
\begin{enumerate}
\item \label{item-boundary} $p\in \partial \spc{L}$;
\item\label{item-contractible} $\Sigma_p(\spc{L})$ is contractible;
\item \label{item-space-dir-homology} $\tilde H_{m-1}(\Sigma_p \spc{L},\ZZ_2)= 0$;
\item \label{item-local-homology} $H_m( \spc{L},\spc{L}\setminus \{p\},\ZZ_2)= 0$;
\item \label{item-closure-top-strata}  $p\in \bar S_{m-1}$;



\end{enumerate}
\end{thm}


This theorem can be used to show that gradient curves of semiconcave functions can be extended backwards at regular interior points.


