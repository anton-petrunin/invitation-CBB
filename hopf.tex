\chapter{Boundary}

\section{Definition}

Suppose $\spc{L}$ is a 1-dimensional complete geodesic $\CBB(\kappa)$ space.
Exercise \ref{ex:dim=1} allows us to define the boundary $\partial\spc{L}\subset \spc{L}$ as the boundary of a manifold.

Now let us use \ref{thm:finite-space-of-directions} and \ref{ex:finite-space-of-directions-dim} to give an iductive definition the boundary for finite-dimensional complete geodesic $\CBB(\kappa)$ space.
Assume that the notion of boundary is already defined in dimensions $1,\dots,m-1$.
Suppose  $\spc{L}$ is $m$-dimensional complete geodesic $\CBB(\kappa)$ space.
We say that $p\in \spc{L}$ belongs to the boundary (briefly $p\in \partial \spc{L}$) if 
$\partial\Sigma_p\ne\emptyset$.
By \ref{thm:finite-space-of-directions} and \ref{ex:finite-space-of-directions-dim}, $\Sigma_p$ is $(m-1)$-dimensional complete geodesic $\CBB(1)$ space;
therefore its boundary is already defined.

Let us emphasize, that the boundary is defined only for finitely-dimensional spaces.

\begin{thm}{Exercise}
Show that for closed convex set $K\subset \EE^m$ with nonempty interior, the topological boundary of $K$ as a subset of $\EE^m$ coincides with the boundary $K$ described above.
\end{thm}

\begin{thm}{Exercise}
Let $\spc{L}$ be a finite-dimensional complete geodesic $\CBB(0)$ space.
Suppose $\spc{L}\iso\spc{L}_1\times\spc{L}_2$.
Show that 
\[\partial \spc{L}=(\partial\spc{L}_1\times\spc{L}_2)\,\cup\,(\spc{L}_1\times\partial\spc{L}_2).\]
\end{thm}


\section{Conic neighborhoods}


The following statement is close relative of Perelman's stability theorem \ref{thm:stability};
its proof is nearly as hard, but slightly simpler.
We are going to use this result without proof.


\begin{thm}{Theorem}\label{thm:spherical-nbhd}
Any point $p$ in a finite-dimensional geodesic $\CBB(\kappa)$ space $\spc{L}$
admits a neighborhood that is homeomorphic to the tangent space $\T_p$ at this point.
\end{thm}

This statement is often used together with \emph{uniqueness of conic neighborhood} stated below.

Suppose that a neighborhood $U$ of a point $x$ in a metric space $\spc{X}$
admits a homeomorphism to cone $\Cone\Sigma$ such that $x$ maps to the tip of the cone.
In this case, we say that $U$ has a \emph{conic neighborhood} of $x$.

\begin{thm}{Uniqueness of conic neighborhood}\label{lem:kwun}
Any two conic neighborhoods of a given point in a metric space are homeomorphic.
\end{thm}

\begin{thm}{Advanced exercise}\label{ex:conic}
Prove the uniqueness of conic neighborhood or read the original proof by Kyung Whan Kwun \cite{kwun1964}.
\end{thm}

\begin{thm}{Exercise}\label{ex:conic-tangent}
Suppose $x\mapsto x'$ is a homeomorphism between finite-dimensional geodesic $\CBB(\kappa)$ spaces $\spc{L}$ and $\spc{L}'$.

\begin{subthm}{ex:conic-tangen:tangent}
Show that $\T_x$ is homeomorphic to $\T_{x'}$ for any $x$.
\end{subthm}

\begin{subthm}{ex:conic-tangen:dir}
Show that $\Susp\Sigma_x$ is homeomorphic to $\Susp\Sigma_{x'}$ for any $x$.
\end{subthm}

\begin{subthm}{ex:conic-tangen:example}
Give an example of $x\in\spc{L}$ and $x\in\spc{L}'$
such that $\Sigma_x$ is not homeomorphic to $\Sigma_{x'}$.
\end{subthm}


\end{thm}



\section{Topology}

\begin{thm}{Theorem}
Let $\spc{L}$ and $\spc{L}'$ be homeomorphic finite-dimensional geodesic $\CBB(\kappa)$ spaces.
Then $\dim \spc{L}=\dim\spc{L}'$ and
\[\partial\spc{L}\ne \emptyset
\quad\iff\quad
\partial\spc{L}'\ne \emptyset
\]
\end{thm}

Let $\spc{L}$ be an $m$-dimensional geodesic $\CBB(\kappa)$ space.
Define \emph{rank} of $\spc{L}$ (briefly, $\rank\spc{L}$) as the minimal value $k$ such that $\spc{L}\iso\RR^{m-k}\times \spc{K}$,
where $\spc{K}$ is a $k$-dimensional geodesic $\CBB(\kappa)$ space.

In the following proof we will apply induction on the rank of space.
Before reading this proof, it is instructive to solve the following exercise.

\begin{thm}{Exercise}
Construct two plane subsets $K_1$ and $K_2$ such that $K_1\ncong K_2$,
but $\RR\times K_1\cong \RR\times K_2$.
\end{thm}

\parit{Proof.}
The first statement follows from .

Suppose we have a counterexample, say $\partial \spc{L}\ne \emptyset$, but $\partial \spc{L}'=\emptyset$.
Let $k\df\rank \spc{L}$ and $k'\df\rank \spc{L}'$.

Note that $k>1$ or $k'>1$.
Otherwise, by \ref{ex:dim=1}, both $\spc{L}$ and $\spc{L}'$ are topological manifolds
and the statement follows since homeomorphism between topological manifolds respects their boundary.

We can assume that the pair of ranks $(k,k')$ is minimal in the lexicographic order.
Choose a homeomorphism $x\mapsto x'$ from $\spc{L}$ to $\spc{L}'$ 
and $x\in \partial \spc{L}$.
Since $\partial \spc{L}'=\emptyset$, we have $x'\notin \partial \spc{L}'$.
Note that 
\[\rank \T_x\le k
\quad\text{and}\quad
\rank \T_{x'}\le k',
\]
By \ref{ex:conic-tangen:tangent}, $\T_x$ is homeomorphic to $\T_{x'}$.
Therefore, we may assume that $\spc{L}$ and $\spc{L}'$ are Euclidean cones with tips at $x$ and $x'$ respectively.


Suppose $k>1$ and $\spc{L}\iso \RR^{m-k}\times \spc{C}$, where $\spc{C}$ a $k$-dimensional geodesic $\CBB(0)$ cone.
Note that $\rank\T_y\le\rank\spc{L}$ for any $y\in\spc{L}$ and the equality holds only if $y$ projects to the tip of $\spc{C}$.

Since $k>1$ we can find $z\in\partial\spc{C}$ such that $z\ne 0$.
Choose $y$ that projects to $z$, so $\rank\T_y<\rank\spc{L}$.
Applying \ref{ex:conic-tangen:tangent} again, we get that $\T_y$ is homeomorphic to $\T_{y'}$,
$\partial  \T_y\ne\emptyset$ and $\partial \T_{y'}=\emptyset$ which contradicts minimality of $k$.

Since $\partial \spc{L}\ne \emptyset$, we get that $k=1$ and $\spc{L}=\RR^{m-1}\times\RR_{\ge0}$.

Now assume $k'> 1$.
Let $\spc{L}'\iso \RR^{m-k'}\times \spc{C}'$, where $\spc{C}'$ a $k'$-dimensional geodesic $\CBB(0)$ cone.
Since $\partial\spc{L}$ is $(m-1)$-dimensional,
the projection of its image in $\spc{L}'$ to $\spc{C}'$
cannot be its tip.
In other words, we can choose $y\in \partial \spc{L}$ such that its image $y'\in \spc{L}'$ has nonzero projection in $\spc{C}'$.
Therfore, $\T_y$ is homeomorphic to $\T_{y'}$, 
\[\rank\T_y\le k=1,
\quad
\rank\T_{y'}< k',
\quad
0\in \partial \T_y,
\quad\text{and}\quad
\partial \T_{y'}\ne \emptyset\]
--- a contradiction.
\qeds

\begin{thm}{Exercise}\label{ex:bry2bry}
Let $x\mapsto x'$ be a homeomorphism $\Omega\to\Omega'$
between open subsets in finte-dimensional complete geodesic $\CBB(\kappa)$ spaces $\spc{L}$ and $\spc{L}'$.

Show that $x\in \partial \spc{L}$ if and only if $x'\in \partial \spc{L}'$.

\end{thm}

\begin{thm}{Exercise}\label{ex:bry-closed}
Show that boundary of a finite-dimensional complete geodesic $\CBB(\kappa)$ space is a closed subset.
\end{thm}

\begin{thm}{Exercise}
Let $\spc{L}_1$ and $\spc{L}_2$ be two $m$-dimensional geodesic $\CBB(\kappa)$ spaces; $m$ is finite.
Suppose $\Omega_1$ is an open subset in $\spc{L}_1$ and $f\:\Omega_1\to \spc{L}_2$ is an injective continuous map.
Show that $f$ is an embedding.
Moreover, if $\partial\spc{L}_2=\emptyset$, then $f(\Omega_1)$ is open in $\spc{L}_2$.  
\end{thm}

\section{Tangent space}

Let $X$ be a subset in a finite-dimensional complete geodesic $\CBB(\kappa)$ space $\spc{L}$.
Choose $p\in \spc{L}$ and $\xi\in \Sigma_p$.
Suppose $\xi$ is a limit of directions $\dir{p}{x_n}$ for a sequence $x_1,x_2,\dots{}\in X$ that converges to $p$.
Then we say that $\xi$ is in the \emph{space of directions} from $p$ to $X$; briefly $\xi\in\Sigma_pX$.

Further, the cone $\T_pX=\Cone(\Sigma_pX)$ will be called \emph{tangent space} to $X$ at $p$.

\begin{thm}{Theorem}\label{thm:partial-Sigma}
For any finite-dimensional complete geodesic $\CBB(\kappa)$ space $\spc{L}$, we have
\[\partial (\Sigma_p\spc{L})=\Sigma_p(\partial\spc{L})
\quad\text{and}\quad
\partial(\T_p\spc{L})=\T_p(\partial\spc{L}).\]
\end{thm}

\parit{Proof.}
Choose a sequence $x_n\to p$.
Suppose $\dir{p}{x_n}\to\xi$ as $n\to\infty$.

We can assume that $x_n\ne p$ for any $n$.
Let $\spc{L}_n\df \tfrac1{\dist{p}{x_n}{}}\cdot\spc{L}$.
Let us keep the same notation $x_n$ for the corresponding point in $\spc{L}_n$;
denote by $p_n\in \spc{L}_n$ the point that corresponds to $p$ in  $\spc{L}_n$.
Note that all spaces $\T_{p_n}\spc{L_n}$ can be identified with $\T_p\spc{L}$.

Given $y_n\in \spc{L}_n$ choose a geodesic path $\gamma$ from $p_n$ to $y_n$
and let $\log_{p_n}y_n=\gamma^+(0)$.
This way we defined a map $\log_{p_n}\:\spc{L}_n\to \T_p$.
Observe that $\log_{p_n}$ are a Hausdorff approximations for the Gromov--Hausdorff convergence 
$(\spc{L}_n,p_n)\GHto \T_p$.
Moreover $\log_{p_n}x_n\to \xi$ as $n\to \infty$.

The stability theorem implies that $\log_{p_n}$ can be approximated by homeomorphisms $h_n$.
More precisely, there is a sequence of maps $h_n\:\spc{L}_n\to \T_p$
that defined an open embedding of open ball $\oBall(p,r)$ for any fixed $r>0$ and all large $n$
and for any $r>0$ we have $\sup_{\dist{p_n}{y_n}{\spc{L}_n}<r}\{\dist{\log_{p_n}y_n}{h_n(y_n)}\to 0$ as $n\to\infty$.

Suppose $x_n\in \partial\spc{L}_n$.
By \ref{ex:bry2bry}, $h_n(x_n)\in\partial \T_p$.
Since in $\T_p$

Recall that $\lambda\cdot \spc{L}\GHto \T_p$; see ???.
Moreover, the map $f_\lambda\:x\mapsto \lambda\cdot\log_px$ defines Hausdorff approximations $\lambda\cdot \spc{L}\GHto \T_p$.

 

Choose a sequence $x_n\to p$;
let $\lambda_n=\tfrac1{\dist{p}{x_n}{}}$.
Suppose $\lambda_n\cdot \log_px\to v\in\T_p$.



Note that $\lambda_n\cdot [p x_n]\subset \lambda\spc{L}$ converge to a ge

\qeds

\section{Doubling}

Let $A$ be a closed subset $A$ in a metric space $\spc{X}$.
The doubling $\spc{W}$ of $\spc{X}$ across $A$ is two copies of $\spc{X}$ glued along $A$;
more precisely, the underlying set of $\spc{W}$ is the quotient $\spc{X}\times\{0,1\}/\sim$, where $(a,0)\sim (a,1)$ for any $a\in A$ and it has minimal metric such that both maps $\spc{X}\to \spc{W}$ defined by $x\mapsto (x,0)$ and $x\mapsto (x,1)$ are distance-preserving.

The metric on $\spc{W}$ can be defined explicitly by
\[\dist{(x,i)}{(y,j)}{\spc{W}}=
\begin{cases}
\dist{x}{y}{\spc{X}}&\text{if}\quad i= j.
\\
\inf\set{\dist{x}{a}{\spc{X}}+\dist{y}{a}{\spc{X}}}{a\in A}&\text{if}\quad i\ne j.
\end{cases}
\]

The last part of the following statement is the so-called \emph{doubling theorem}.

\begin{thm}{Theorem}\label{thm:doubling}
Let $\spc{L}$ be a finite-dimensional complete geodesic $\CBB(\kappa)$ space with nonempty boundary.
Suppose $f\z=\tfrac12\cdot\distfun_p^2$.
Then

\begin{subthm}{thm:doubling:concave}
Suppose $\dim \spc{L}\ge 2$, then
$\distfun_{\partial \Sigma_x}(\xi)\le \tfrac\pi2$ for any $x\in\partial \spc{L}$ and $\xi\in \Sigma_x$.
Moreover, if $\distfun_{\partial \Sigma_x}(\xi)= \tfrac\pi2$, then $\mangle(\xi,\zeta)\le\tfrac\pi2$ for any $\zeta\in \Sigma_x$. 
\end{subthm}

\begin{subthm}{thm:partial-grad:grad}
$\nabla_xf\in \partial\T_x$ for any $x\in\partial \spc{L}$.
\end{subthm}

\begin{subthm}{thm:partial-grad:flow}
If $\alpha$ is an $f$-gradient curve that starts at $x\in \partial \spc{L}$, then $\alpha(t)\in \partial \spc{L}$ for any $t$.
Moreover, $\gexp_p(v)\in \partial \spc{L}$ for any $v\in\partial\T_p$.
\end{subthm}

\begin{subthm}{thm:doubling:doubling}
the doubling $\hat{\spc{L}}$ of $\spc{L}$ across $\partial \spc{L}$ is a complete geodesic $\CBB(\kappa)$ space.
\end{subthm}

\end{thm}

The proof will be given by induction on the dimension of the space.


\parit{Proof.}
We will denote by 
\ref{SHORT.thm:doubling:concave}$_m,\dots$\ref{SHORT.thm:doubling:doubling}$_m$ the corresponding statement assuming $m=\dim\spc{L}$.
The proof goes by induction on $m$.

\parit{\ref{SHORT.thm:doubling:doubling}$_{m-1}\Rightarrow$\ref{SHORT.thm:doubling:concave}$_m$.}
If $m=2$, then $\Sigma_x$ is isometric to a line segment $[0,\ell]$;
we need to show that $\ell\le\pi$.

Suppose $\ell>\pi$, then $\T_x=\Cone\Sigma_x$ has several different lines.
It follows that $\T_x$ is isometric to the Euclidean plane;
the latter contradicts that $\Sigma_x$ is a line segment.

Suppose $m=\dim \spc{L}>2$, so $\dim \Sigma_x>1$.
Suppose $\distfun_{\partial \Sigma_x}(\xi)> \tfrac\pi2$ for some $\xi$.
By \ref{SHORT.thm:doubling:doubling}$_{m-1}$, the doubling $\spc{W}$ of $\Sigma_x$ is a geodesic $\CBB(\kappa)$ space.
Denote by $\xi_0$ and $\xi_1$ the points in $\spc{W}$ that correspond to $\xi$.
Observe that $\dist{\xi_0}{\xi_1}{\spc{W}}>\pi$.
The latter contradicts \ref{ex:RisCBB(1)}.

Finally, if $\distfun_{\partial \Sigma_x}(\xi)= \tfrac\pi2$, then $\dist{\xi_0}{\xi_1}{\spc{W}}=\pi$.
Therefore the tangent space $\Cone \spc{W}$ contains a line in directions of $\xi_0$ and $\xi_1$;
in other words, $\spc{W}$ is a spherical suspension with poles $\xi_0$ and $\xi_1$.
In particular, every point of $\spc{W}$ lies on distance at most $\tfrac\pi2$ from $\xi_0$ or $\xi_1$.
The natural projection $\spc{W}\to \Sigma_x$ does not increase distances and sends both  $\xi_0$ and $\xi_1$ to $\xi$.
Therefore, the second statement follows.

\parit{\ref{SHORT.thm:doubling:doubling}$_{m-1}+$\ref{SHORT.thm:doubling:concave}$_{m-1}+$\ref{SHORT.thm:doubling:concave}$_m\Rightarrow$\ref{SHORT.thm:partial-grad:grad}$_m$.}
We can assume that $s=\nabla_xf\ne 0$.
By \ref{prop:grad-exist}, $\nabla_xf\z=s\cdot \overline{\xi}$, where $s=\dd_xf(\overline{\xi})$ and $\overline{\xi}\in\Sigma_p$ is the direction that maximize $\dd_xf(\overline{\xi})$.

Let $\zeta\in \partial\Sigma_x$ be a direction that minimize the angle $\mangle(\overline{\xi},\zeta)$.
It is sufficient to show that $\zeta=\overline{\xi}$.

Assume $\zeta\ne \overline{\xi}$;
let $\eta=\dir\zeta{\overline{\xi}}$.
By \ref{SHORT.thm:doubling:concave}$_m$, $\mangle(\overline{\xi},\zeta)\le \tfrac\pi2$ and
\ref{SHORT.thm:doubling:concave}$_{m-1}$ implies that 
\[\mangle(\eta,\nu)\le \tfrac\pi2\eqlbl{eq:<pi/2}\]
for any $\nu\in \Sigma_\zeta\Sigma_x$.

Consider function $\phi\:\Sigma_x\to\RR$ defined by $\phi(\xi)\df\dd_xf(\xi)$.
Applying \ref{ex:d(distfun):<} and \ref{eq:<pi/2}, we get that $\dd_\xi\phi(\eta)\le 0$.
If $\phi(\zeta)\le 0$, then it implies that $\phi(\overline{\xi}\le 0$ --- a contradiction.
If $\phi(\zeta)> 0$, then $\phi(\overline{\xi}<\phi(\zeta)$ --- a contradiction again.

\parit{\ref{SHORT.thm:partial-grad:grad}$_m\Rightarrow$\ref{SHORT.thm:partial-grad:flow}$_m$.}
Let $\alpha$ be an $f$-gradient curve and $\ell(t)=\distfun_{\partial L}\alpha(t)$.
Note that $\ell$ is a Lipschitz function;
in particular it is differentiable almost everywhere.

Choose $t$;
let $x=\alpha(t)$ and $y\in \partial\spc{L}$ be a closest point to $x$.
By \ref{SHORT.thm:partial-grad:grad}$_m$, we have that $\nabla_y f\in\partial \T_y$.
Since the distance $\dist{x}{y}{}$ is minimal, 
we get $\langle \dir yx,v\rangle\le 0$ for any $v\in \partial \T_y$.
In particular,
\[\langle \dir yx,\nabla_y f\rangle\le 0\]
Applying Exercise~\ref{ex:monotonicity} to $x$ and $y$, 
we get
\[\ell^+(t)\le \ell(t)\]
if the left-hand side is defined.
Since $\ell$ is Lipschitz, $\ell^+$ is defined almost everywhere.
Integrating the inequality, we get 
\[\ell(t)\le e^t\cdot\ell(0)\]
for any $t\ge 0$.
In particular, if $\ell(0)=0$, then $\ell(t)=0$ for any $t\ge 0$.
By \ref{ex:bry-closed}, the statement follows.

\parit{\ref{SHORT.thm:partial-grad:flow}$_{m}\Rightarrow$\ref{SHORT.thm:doubling:doubling}$_m$.}
Now, let us show that doubling $\hat{\spc{L}}$ is $\CBB(0)$.
Denote by $\spc{L}_0$ and $\spc{L}_1$ the two copies of $\spc{L}$ in $\hat{\spc{L}}$;
let us keep the notation $\partial \spc{L}$ for the common boundary of $\spc{L}_0$ and $\spc{L}_1$.

\begin{wrapfigure}{r}{45mm}
\vskip-2mm
\centering
\includegraphics{mppics/pic-1315}
\end{wrapfigure}

Choose a geodesic $\gamma$ in $\hat{\spc{L}}$.
Suppose $\gamma$ shares at least two points with $\partial \spc{L}$, say $x=\gamma(t_1)$ and $y=\gamma(t_2)$.
The splitting argument as in \ref{SHORT.thm:doubling:concave} shows that the doubling of $\T_x\spc{L}$ splits in the direction $\gamma^\pm(t_1)$.
Similarly, the doubling of $\T_y\spc{L}$ splits in the direction $\gamma^\pm(t_y)$.
Note that the arc of $\gamma$ between $x$ and $y$ can be reflected across $\partial \spc{L}$ and the obtained curve is still a geodesic in $\hat{\spc{L}}$.
Using these observations together with part \ref{SHORT.thm:doubling:concave}, one can show that either $\gamma$ lies in $\partial \spc{L}$ or it crosses $\partial \spc{L}$ at most once.

Now choose a point $p$ in $\hat{\spc{L}}$;
let $f\df\tfrac12\cdot\distfun_p^2$.
Without loss of generality, we can assume that $p\in \spc{L}_0$.
It is sufficient to show that $(f\circ\gamma)''\le 1$ for any $t$.
If $\gamma$ lies in $\partial \spc{L}$, then this inequality follows from the comparison in $\spc{L}_0$.

\begin{wrapfigure}{r}{45mm}
\vskip-2mm
\centering
\includegraphics{mppics/pic-1325}
\end{wrapfigure}

In the remaining case, if $\gamma(t)\z\in \spc{L}_0\setminus\partial\spc{L}$, then $(f\circ\gamma)''(t)\le 1$ follows from the comparison in $\spc{L}_0$.
If $\gamma(t)\in \spc{L}_1\setminus\partial\spc{L}$, then the proof of inequality reminds the argument in part \ref{SHORT.thm:doubling:concave}, but it is a bit more tricky.
Finally if $\gamma(t)\in\partial\spc{L}$, then splitting argument shows that 
\[(f\circ\gamma)^+(t)+(f\circ\gamma)^-(t)\le 0.\]
These three statements imply that $(f\circ\gamma)''(t)\le 1$ for any $t$.
\qeds
