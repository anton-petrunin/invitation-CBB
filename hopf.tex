\chapter{Boundary}

\section{Definition}

Suppose $\spc{L}$ is a 1-dimensional complete geodesic $\CBB(\kappa)$ space.
Exercise \ref{ex:dim=1} allows us to define the boundary $\partial\spc{L}\subset \spc{L}$ as the boundary of a manifold.

Now let us use \ref{thm:finite-space-of-directions} and \ref{ex:finite-space-of-directions-dim} to give an iductive definition the boundary for finite-dimensional complete geodesic $\CBB(\kappa)$ space.
Assume that the notion of boundary is already defined in dimensions $1,\dots,m-1$.
Suppose  $\spc{L}$ is $m$-dimensional complete geodesic $\CBB(\kappa)$ space.
We say that $p\in \spc{L}$ belongs to the boundary (briefly $p\in \partial \spc{L}$) if 
$\partial\Sigma_p\ne\emptyset$.
By \ref{thm:finite-space-of-directions} and \ref{ex:finite-space-of-directions-dim}, $\Sigma_p$ is $(m-1)$-dimensional complete geodesic $\CBB(1)$ space;
therefore its boundary is already defined.

Let us emphasize, that the boundary is defined only for finitely-dimensional spaces.

\begin{thm}{Exercise}
Show that for closed convex set $K\subset \EE^m$ with nonempty interior, the topological boundary of $K$ as a subset of $\EE^m$ coincides with the boundary $K$ described above.
\end{thm}

\begin{thm}{Exercise}
Let $\spc{L}$ be a finite-dimensional complete geodesic $\CBB(0)$ space.
Suppose $\spc{L}\iso\spc{L}_1\times\spc{L}_2$.
Show that 
\[\partial \spc{L}=(\partial\spc{L}_1\times\spc{L}_2)\,\cup\,(\spc{L}_1\times\partial\spc{L}_2).\]
\end{thm}


\section{Conic neighborhoods}

We are going to jump over couple technical results.
The following statement was proved by Grigory Perelman \cite{perelman1991}, 
the proof was rewritten with more details by the second author \cite{kapovitch}.

\begin{thm}{Theorem}\label{thm:spherical-nbhd}
A small spherical neighborhood around point $p$ in a finite-dimensional geodesic $\CBB(\kappa)$ space $\spc{L}$
is homeomorphic to the tangent space $\T_p$ at this point.
\end{thm}

This statement is often used together with \emph{uniqueness of conic neighborhood} stated below.
It is proved by Kyung Whan Kwun \cite{kwun1964}.

Suppose that a neighborhood $U$ of a point $x$ in a metric space $\spc{X}$
admits a homeomorphism to cone $\Cone\Sigma$ such that $x$ maps to the tip of the cone.
In this case, we say that $U$ has a \emph{conic neighborhood} of $x$.

\begin{thm}{Uniqueness of conic neighborhood}\label{lem:kwun}
Any two conic neighborhoods of a given point in a metric space are homeomorphic.
\end{thm}

\begin{thm}{Exercise}\label{ex:conic}

\begin{subthm}{ex:conic:conic}
Prove the uniqueness of conic neighborhood
\end{subthm}

\begin{subthm}{ex:conic:tangent}
Suppose $x\mapsto x'$ is a homeomorphism between finite-dimensional geodesic $\CBB(\kappa)$ spaces $\spc{L}$ and $\spc{L}'$.
Show that $\T_x$ is homeomorphic to $\T_{x'}$ for any $x$.
\end{subthm}

\end{thm}


\section{Morse theory}

Let $f$ be a semiconcave function.
A point $p\in \Dom f$ is called \emph{critical} point of $f$ if $\dd_pf\le 0$; 
otherwise it is called \emph{regular}.

The proof of the following statement is quite technical, we omit its proof.

\begin{thm}{Theorem}
Let $f$ be a semiconcave function on a finite-dimensional geodesic $\CBB(\kappa)$ space.
Suppose $K$ is a compact set of regular points of $f$ in its level set $f=a$.
Then an open neighborhood $\Omega$ of $K$ admits homeomorphism $x\mapsto (h(x),f(x))$ to a product space $\Lambda\times (a-\eps,a+\eps)$.

\end{thm}

Note that distance function $\distfun_p$ has no critical points in a neighborhood of $p$ and the level set $\distfun_p=\eps$ is compact for small $\eps>0$.
Combining this observation with 

Applying the theorem we get the following.

\begin{thm}{Corollary}
A small spherical neighborhood of any point $p$ in a finite-dimensional geodesic $\CBB(\kappa)$ space $\spc{L}$ is homeomorphic to an open cone over small sphere around $p$.
\end{thm}


\section{Topology}

\begin{thm}{Theorem}
Suppose $x\mapsto x'$ is a homeomorphism $\spc{L}\to\spc{L}'$ between finite-dimensional geodesic $\CBB(\kappa)$ spaces.
Then $x\in \partial \spc{L}$ if and only if $x'\in \partial \spc{L}'$.

In particular, $\spc{L}$ has nonempty boundary if and only if so does $\spc{L}'$.
\end{thm}

Let $\spc{L}$ be an $m$-dimensional geodesic $\CBB(\kappa)$ space.
Define \emph{rank} of $\spc{L}$ (briefly, $\rank\spc{L}$) as the minimal value $k$ such that $\spc{L}\iso\RR^{m-k}\times \spc{N}$,
where $\spc{N}$ is a $k$-dimensional geodesic $\CBB(\kappa)$ space.

In the following proof we will apply induction on the rank of space.

\parit{Proof.}
Suppose we have a counterexample, $x\in \partial \spc{L}$, but $x'\notin \partial \spc{L}'$.

Note that $k\df\rank \spc{L}>1$ or $k'\df\rank \spc{L}'>1$.
Otherwise, by \ref{ex:dim=1}, both $\spc{L}$ and $\spc{L}'$ are topological manifolds
and the statement follows since homeomorphism between topological manifolds respects their boundary.

Assume we have a counterexample, say $x\in \partial \spc{L}$, but $x'\notin \partial \spc{L}'$.
Note that $\rank \T_x\ge \rank \spc{L}$ and $\rank \T_{x'}\ge \rank \spc{L}'$.
By \ref{ex:conic:tangent} $\T_x$ is homeomorphic to $\T_{x'}$.
So we may assume that $\spc{L}$ and $\spc{L}'$ are Euclidean cones with tips at $x$ and $x'$ respectively.

We can assume that the pair of ranks $(k,k')$ is minimal in the lexicographic order.
Suppose $k>1$ and $\spc{L}\iso \RR^{m-k}\times \spc{C}$, where $\spc{C}$ a $k$-dimensional geodesic $\CBB(0)$ cone.
Note that $\rank\T_y\le\rank\spc{L}$ for any $y\in\spc{L}$ and the equality holds only if $y$ projects to the tip of $\spc{C}$.
Since $k>1$ we can find $z\in\partial\spc{C}$ such that $z\ne 0$.
Choose $y$ that projects to $z$.
Applying \ref{ex:conic:tangent} again, we get a homeomorphism $\T_z\to \T_{z'}$.
Observe that $0\in\partial  \T_z$ and $\partial \T_{z'}=\emptyset$.
From above $\rank \T_z<k$ --- a contradiction.

Summarizing, if there is a counterexample, such that $x\in \partial \spc{L}$, but $x'\notin \partial \spc{L}'$,
then we can assume that $k=\rank \spc{L}\le 1$ and both spaces $\spc{L}$ and $\spc{L}'$ are Euclidean cones with tips at $x$ and $x'$ respectively.
In particular, $\spc{L}\iso \RR^{m-1}\times \RR_{\ge0}$;
therefore, 
$\partial\spc{L}= \RR^{m-1}\times \{0\}$

Now assume $k'\rank \spc{L}'> 1$.
Let $\spc{L}'\iso \RR^{m-k'}\times \spc{C}'$, where $\spc{C}'$ a $k'$-dimensional geodesic $\CBB(0)$ cone.
Since $\partial\spc{L}$ is $(m-1)$-dimensional,
the projection of its image in $\spc{L}'$ to $\spc{C}'$
cannot be its tip.
In other words, we can choose $y\in \partial \spc{L}$ such that its image $y'\in \spc{L}'$ has nonzero projection in $\spc{C}'$.
Therfore, $\T_y$ is homeomorphic to $\T_{y'}$, 
\[\rank\T_y\le k=1,
\quad
\rank\T_{y'}< k',
\quad
0\in \partial \T_y,
\quad\text{and}\quad
\partial \T_{y'}\ne \emptyset\]
--- a contradiction.
\qeds




By \ref{thm:spherical-nbhd}, a spherical neighborhood of $x$ is homeomorphic to the tangent cone $\T_x$.
Further, there is a neighborhood $U'\subset \spc{L}'$ of $x'$ hat is homeomorphic to the tangent cone $\T_x$.
On the other hand, a spherical neighborhood of $x'$ is homeomorphic to $\T_{x'}$.
Applying Kwun's lemma (\ref{lem:kwun}), we get that $\T_x$ is homeomorphic to $\T_{x'}$.


Note that $\rank \T_x\ge \rank \spc{L}$ and $\rank \T_{x'}\ge \rank \spc{L}'$.
We can assume that $\rank \spc{L}$ takes minimal possible value.
In this case $\rank \T_x= \rank \spc{L}$.
Moreover, the same argument as above shows that 
\[\rank \T_v\T_x=\rank \T_x\]
for any $v\in \partial\T_x$.

The latter is possible only if $k=1$, where $k\df \rank \T_x$;
in other words, $\T_x\iso \RR^{m-1}\times \RR_{\ge0}$.
Indeed, $\rank \T_x\ne 0$;
otherwise $\T_x$ is a Eucliedan space and therefore $\partial \T_x=\emptyset$.
Further, suppose $\rank \T_x=k\ge 2$, and $\T_x\iso \RR^{m-k}\times \spc{C}$.
Then we can choose $v\in \partial \T_x$ that projects to a nonzero vector in $\spc{C}$.
In this case $\rank \T_v\T_x<\rank \T_x$ --- a contradiction.

Summarizing, in the example above, we can assume that $\rank \spc{L}=1$.
We can also assume that $\rank \spc{L}'$ takes minimal possible value.
Suppose 







Let us show that $\rank \spc{L}=1$.
Now assume $k=\rank \spc{L}>1$ and $\spc{L}=\RR^{m-k}\times \spc{N}$.
The tangent space at $x$ splits as $\RR^{m-k}\times \T_{y}\spc{N}$, where $y$ is the projection of $x$ to $\spc{N}$.
Note that $\partial\Sigma_y\ne\emptyset$;
choose a direction $\xi\in\partial\Sigma_y$.



We may assume that $\spc{L}\iso\RR^m\times \spc{N}$ and $\spc{L}\iso\RR^n\times \spc{N}'$
for some integers $m$ and $n$ and $\dim \spc{N}+\dim \spc{N}'$ takes minimal possible value.
Let us 

Assume contrary; that is, $x\in \partial \spc{L}$, but $x'\notin \partial \spc{L}'$.
Let $m=\dim \spc{L}$; by??? $m=\dim\spc{L}'$.

\qeds




The following statements should agree with your intuition.

\begin{thm}{Theorem}\label{thm:bry-closed}
Boundary of a finite-dimensional complete geodesic $\CBB(\kappa)$ space is a closed subset.
\end{thm}

The proof is based on the following topologcal lemma.

\begin{thm}{Key lemma}\label{lem:bry-closed:key}
Let $\spc{V}$ and $\spc{W}$ be geodesic Euclidean cones.
Assume $\spc{V}$ and $\spc{W}$ are finite-dimesional $\CBB(0)$ spaces
and $\RR^{m}\times \spc{V}$ is homeomorphic to $\RR^{n}\times \spc{W}$ for some $m$ and $n$.
Then $\spc{V}$ has boundary if and only if $\spc{W}$ does.
\end{thm}

\parit{Proof of the key lemma \ref{lem:bry-closed:key}.}
Suppose $\spc{V}$ has boundary, but $\spc{W}$ does not.
We can assume that the pair $(\dim \spc{V},\dim \spc{W})$ is minimal in the lexicographic order.
Let us show that $\dim \spc{V}=\dim \spc{W}=1$.

First assume that $\dim \spc{V}>1$.
Choose $v\ne 0$ in $\partial\spc{V}$.
By \ref{thm:spherical-nbhd}, a spherical neighborhood of $v$ is homeomorphic to $\RR\times \spc{V}'$, where $\spc{V}'$ is a geodesic $\CBB(0)$ Eulidean cone and $\dim \spc{V}'=\dim \spc{V}-1$.
Therefore a neighborhood of $(0, v)\in \RR^n\times \spc{V}$ is homeomorphic to $\RR^{n+1}\times \spc{V}'$.
Let $(x,w)$ be the corresponding point in $\RR^m\times \spc{W}$;
it has a conic neighborhood homeomorphic to $\RR^{n+1}\times \spc{V}'$.

By Kwun's lemma, $(x,w)$ has conic neighborhood.
Let $F(x) \in \RR^{m_2}\times \{z\}$.
Suppose $z$ is the tip of $\Cone \Sigma_2$.
Then, by Kwun's lemma, $\RR^{m_1+1}\times \Cone\Sigma_\xi$ is homeomorphic to $\RR^{m_2}\times \Cone \Sigma_2$;
so $k_1$ is not minimal --- a contradition.
Now suppose $z$ is not the tip of $\Cone \Sigma_2$.
Then a conical neighborhood of $z$ is homeomorphic to $\RR^{m_2+1}\times \Cone\Sigma_\zeta\Sigma_2$.
Since $\Sigma_\zeta$ has no boundary, we arrive at a contradiction again.



Choose a homeomorphism $h\:\RR^{m_1}\times \Cone \Sigma_1\to \RR^{m_2}\times \Cone \Sigma_2$.
Suppose $\Sigma_1$ has boundary, but $\Sigma_2$ does not.

We can assume that $k_1$ takes minimal possible value; let us show that $k_1=1$.
If $k_1>1$, then we can choose $\xi\in \partial \Sigma_1$, so $\Sigma_\xi\Sigma_1$ has nonempty boundary.
Note that $\Cone \Sigma_1$ contains a point with conical neighborhood homeomorphic to $\RR\times \Cone\Sigma_\xi$.
By ??? $\dim\Sigma_\xi=\dim\Sigma_1-1$.

Therefore $\RR^{m_1}\times \Cone \Sigma_1$ contains a point, say $x$ with conical neighborhood homeomorphic to $\RR^{m_1+1}\times \Cone\Sigma_\xi$.
By Kwun's lemma, the corresponding point $F(x)$ in $\RR^{m_2}\times \Cone \Sigma_2$ has homeomorphic conic neighborhood.
Let $F(x) \in \RR^{m_2}\times \{z\}$.
Suppose $z$ is the tip of $\Cone \Sigma_2$.
Then, by Kwun's lemma, $\RR^{m_1+1}\times \Cone\Sigma_\xi$ is homeomorphic to $\RR^{m_2}\times \Cone \Sigma_2$;
so $k_1$ is not minimal --- a contradition.
Now suppose $z$ is not the tip of $\Cone \Sigma_2$.
Then a conical neighborhood of $z$ is homeomorphic to $\RR^{m_2+1}\times \Cone\Sigma_\zeta\Sigma_2$.
Since $\Sigma_\zeta$ has no boundary, we arrive at a contradiction again.


Let us show that  $k_1\ge k_2$;
moreover, $F(\RR^{m_1}\times \{0_1\})=\RR^{m_2}\times \{0_2\}$, where $0_i$ denotes the tip of $\Cone \Sigma_i$.

Indeed, if 

Assume $k_1 = 1$.
Then $\Sigma_1$ is a closed interval, say $\II$.
Let us apply induction on $k_2$.

If $k_2=1$, then $\Sigma_2$ is either circle or a closed interval.
The former is impossible since $\RR^{m_2+2}\cong\RR^{m_2}\times \Cone(\SSS^1)$ is not homeomorphic to the half-space $\RR^{m_1+1}\times \RR_\ge\cong\RR^{m_1}\times \Cone(\II)$.

Now suppose $k_2 > 1$ and we have already proved the statement for smaller values of $\dim \Sigma_2$.
Choose a homeomorphism $F\:\RR^{m_1}\times \Cone\Sigma_1\to\RR^{m_2}\times \Cone\Sigma_2$.
Suppose $\Sigma_2$ has no boundary.

Note that we necessarily have that $m_1 + k_1 = m_2 + k_2$.
Therefore $m_1>m_2$.
Hence 
\qeds


\parit{Proof of \ref{thm:bry-closed}.}
\qeds


Let $X$ be a subset in a finite-dimensional complete geodesic $\CBB(\kappa)$ space $\spc{L}$.
Choose $p\in \spc{L}$ and $\xi\in \Sigma_p$.
Suppose $\xi$ is a limit of directions $\dir{p}{x_n}$ for a sequence $x_1,x_2,\dots{}\in X$ that converges to $p$.
Then we say that $\xi$ is in the \emph{space of directions} from $p$ to $X$; briefly $\xi\in\Sigma_pX$.

Further, the cone $\T_pX=\Cone(\Sigma_pX)$ will be called \emph{tangent space} to $X$ at $p$.

\begin{thm}{Theorem}\label{thm:partial-Sigma}
For any finite-dimensional complete geodesic $\CBB(\kappa)$ space $\spc{L}$, we have
\[\partial (\Sigma_p\spc{L})=\Sigma(\partial\spc{L})
\quad\text{and}\quad
\partial(\T_p\spc{L})=\T_p(\partial\spc{L}).\]
\end{thm}

\begin{thm}{Exercise}
Construct two plane subsets $K_1$ and $K_2$ such that $K_1\ncong K_2$,
but $\RR\times K_1\cong \RR\times K_2$.
\end{thm}

\begin{thm}{Exercise}
Let $\spc{L}_1$ and $\spc{L}_2$ be two $m$-dimensional geodesic $\CBB(\kappa)$ spaces; $m$ is finite.
Suppose $\Omega_1$ is an open subset in $\spc{L}_1$ and $f\:\Omega_1\to \spc{L}_2$ is an injective continuous map.
Show that $f$ is an embedding.
Moreover, if $\partial\spc{L}_2=\emptyset$, then $f(\Omega_1)$ is open in $\spc{L}_2$.  
\end{thm}



\section{Geometry}

\begin{thm}{Theorem}\label{thm:partial-grad}
Let $\spc{L}$ be a finite-dimensional complete geodesic $\CBB(\kappa)$ space with nonempty boundary.
Suppose $p\in \spc{L}$ and $f\z=\tfrac12\cdot\distfun_p^2$.
Then

\begin{subthm}{thm:partial-grad:grad}
$\nabla_xf\in \partial\T_x$ for any $x\in\partial \spc{L}$.
\end{subthm}

\begin{subthm}{thm:partial-grad:flow}
If $\alpha$ is an $f$-gradient curve that starts at $x\in \partial \spc{L}$, then $\alpha(t)\in \partial \spc{L}$ for any $t$.
\end{subthm}

\end{thm}

\parit{Proof;} \ref{SHORT.thm:partial-grad:grad}.
We can assume that $s=\nabla_xf\ne 0$.
By ???, $\nabla_xf\ne 0=s\cdot \overline{\xi}$, where $s=\dd_xf(\overline{\xi})$ and $\overline{\xi}\in\Sigma_p$ is the direction that maximize $\dd_xf(\overline{\xi})$.

Let $\zeta\in \partial\Sigma_x$ be a direction that nimimize the angle $\mangle(\overline{\xi},\zeta)$.
It is sufficient to show that $\zeta=\overline{\xi}$.
 
Consider function $\phi\:\Sigma_x\to\RR$ defined by $\phi(\xi)\df\dd_xf(\xi)$.
By ??? $\mangle \hinge{\zeta}{\overline{\xi}}{\xi}_{\Sigma_x}\le \tfrac\pi 2$ for any $\xi\in\Sigma_x$.
Therefore, \ref{ex:d(distfun):=} implies that 
\[\dd_\zeta\phi(\dir\zeta{\overline{\xi}})\le 0.\]
It follows that $\phi(\overline{\xi})\le \phi(\zeta)$ and equality holds only if $\overline{\xi}=\zeta$,
hence the result.

\parit{\ref{SHORT.thm:partial-grad:flow}.}
Let $\alpha$ be an $f$-gradient curve and $\ell(t)=\distfun_{\partial L}\alpha(t)$.
Note that $\ell$ is a Lipschitz function;
in particular it is differentiable almost everywhere.

Choose $t$;
let $x=\alpha(t)$ and $y\in \partial\spc{L}$ be a closest point to $x$.
Applying Exercise~\ref{ex:monotonicity} to $x$ and $y$, 
and taking into account that $\nabla_yf\in \T\partial\spc{L}$, 
we get
\[\ell^+(t)\le \ell(t)\]
if the left-hand side is defined.
Since $\ell$ is Lipschitz, $\ell^+$ is defined everywhere.
Integrating the inequality, we get 
\[\ell(t)\le e^t\cdot\ell(0)\]
for any $t\ge 0$.
In particular, if $\ell(0)=0$, then $\ell(t)=0$ for any $t\ge 0$.
By \ref{thm:bry-closed}, the statement follows.
\qeds

The first statement in the theorem above is quite easy.
(Maybe I will add a proof later.)
The second part is proved as Picard theorem with the use of the first part.
Using the last statement for a sequence of points $x_n\to p$ one can get the following.

\begin{thm}{Theorem}\label{thm:gexp}
Let $\spc{L}$ be a finite-dimensional complete geodesic $\CBB(0)$ space.
For any $p\in \spc{L}$ there is a map $\gexp_p\:\T_p\to \spc{L}$  that meets the following conditions.

\begin{subthm}{}
If $\gamma_x$ is a geodesic path from $p$ to $x$, then $\gexp_p(\gamma^+(0))=x$, 
\end{subthm}

\begin{subthm}{}
The map $\gexp_p\:\T_p\to \spc{L}$ is short.
\end{subthm}

\begin{subthm}{}
If $v\in \partial\T_p$, then $\gexp_p(v)\in \partial  \spc{L}$.
\end{subthm}

\end{thm}

The map $\gexp_p\:\T_p\to \spc{L}$ described in the theorem is called \emph{gradient exponent} at $p$.
It provides an alternative for the exponential map for $\CBB$ spaces.
Gradient exponent is defined on the whole $\T_p$ while
the usual exponential map is defined on a relatively small set;
say its complement might be dense in $\T_p$.



\section{Doubling theorem}

\begin{thm}{Theorem}\label{thm:doubling}
Let $\spc{L}$ be a finite-dimensional complete geodesic $\CBB(0)$ space.
Suppose $\partial \spc{L}\ne \emptyset$.
Then 
\begin{subthm}{thm:doubling:concave}
$\distfun_{\partial \spc{L}}$ is a concave function, and
\end{subthm}

\begin{subthm}{}
the doubling $\hat{\spc{L}}$ of $\spc{L}$ across $\partial \spc{L}$ is a complete geodesic $\CBB(0)$ space.
\end{subthm}

\end{thm}


\parit{Sketch.} 
Let us apply induction on $m=\LinDim \spc{L}$.

\begin{wrapfigure}{r}{30mm}
\vskip-2mm
\centering
\includegraphics{mppics/pic-1305}
\end{wrapfigure}

Choose a geodesic $[pz]$; let $\gamma(0)=p$.
Suppose $p\notin\partial \spc{L}$.
Let $q\in \partial\spc{L}$ be a closest point to $p$ and $\alpha\df\mangle\hinge pzq$.

By the definition of boundary points, 
\[\partial \Sigma_q\z\ne\emptyset.\]
Let $\xi=\dir qp$.
Theorem~\ref{thm:partial-Sigma} implies that 
\[\dist{\xi}{\zeta}{\Sigma_q}\ge \tfrac\pi2
\eqlbl{eq:<>pi/2}\]
for any $\zeta\in\partial\Sigma_q$.

By \ref{thm:finite-space-of-directions}, $\Sigma_q$ is an $(m-1)$-dimensional geodesic $\CBB(1)$ space.
Applying the induction hypothesis, we get that the doubling $\hat\Sigma_q$ of $\Sigma_q$ across $\partial \Sigma_q$ is an $(m-1)$-dimensional geodesic $\CBB(1)$ space.
Denote by $\xi_1$ and $\xi_2$ the two directions in $\hat\Sigma_q$ that correspond to $\xi$.
Note that \ref{eq:<>pi/2} implies that $\dist{\xi_1}{\xi_2}{\hat\Sigma_q}\ge \pi$.
Applying the line splitting theorem (\ref{thm:splitting}), we can identify 
$\Cone\hat\Sigma_q$ with $\RR\times \partial\Sigma_q$.
It follows that 
\[\T_q=[0,\infty)\oplus \partial\T_q;\]
in particular, there is a natural projection $\proj\:\T_q\to \partial\T_q$.

Given $x\in [pz]$, choose a geodesic path $\gamma_x$ from $q$ to $x$.
Let 
\[y
\df
\gexp_q\circ\proj(\gamma^+_x(0)).\]
By \ref{thm:gexp}, $y\in \partial\spc{L}$ and 
\[\dist{x}{y}{}\le \dist{p}{q}{}+\dist{p}{x}{}\cdot \cos\alpha.\eqlbl{eq:|x-y|}\]
The latter inequality uses in addition the $\CBB$ comparison for $[pqx]$ and it requires some work.

Note that \ref{eq:|x-y|} implies that $f\circ\gamma$ is concave for any geodesic that lies in $\spc{L}\setminus \partial \spc{L}$.
If $\gamma(t)\in \partial \spc{L}$ for some $t$, then it is easy to see that $(f\circ\gamma)'(t)=0$.
These two statements imply that $f\circ\gamma$ is concave for any geodesic that lies in $\spc{L}$.

Now, let us show that doubling $\hat{\spc{L}}$ is $\CBB(0)$.
Denote by $\spc{L}_0$ and $\spc{L}_1$ the two copies of $\spc{L}$ in $\hat{\spc{L}}$;
let us keep the notation $\partial \spc{L}$ for the common boundary of $\spc{L}_0$ and $\spc{L}_1$.

\begin{wrapfigure}{r}{45mm}
\vskip-2mm
\centering
\includegraphics{mppics/pic-1315}
\end{wrapfigure}

Choose a geodesic $\gamma$ in $\hat{\spc{L}}$.
Suppose $\gamma$ shares at least two points with $\partial \spc{L}$, say $x=\gamma(t_1)$ and $y=\gamma(t_2)$.
The splitting argument as in \ref{SHORT.thm:doubling:concave} shows that the doubling of $\T_x\spc{L}$ splits in the direction $\gamma^\pm(t_1)$.
Similarly, the doubling of $\T_y\spc{L}$ splits in the direction $\gamma^\pm(t_y)$.
Note that the arc of $\gamma$ between $x$ and $y$ can be reflected across $\partial \spc{L}$ and the obtained curve is still a geodesic in $\hat{\spc{L}}$.
Using these observations together with part \ref{SHORT.thm:doubling:concave}, one can show that either $\gamma$ lies in $\partial \spc{L}$ or it crosses $\partial \spc{L}$ at most once.

Now choose a point $p$ in $\hat{\spc{L}}$;
let $f\df\tfrac12\cdot\distfun_p^2$.
Without loss of generality, we can assume that $p\in \spc{L}_0$.
It is sufficient to show that $(f\circ\gamma)''\le 1$ for any $t$.
If $\gamma$ lies in $\partial \spc{L}$, then this inequality follows from the comparison in $\spc{L}_0$.

\begin{wrapfigure}{r}{45mm}
\vskip-2mm
\centering
\includegraphics{mppics/pic-1325}
\end{wrapfigure}

In the remaining case, if $\gamma(t)\z\in \spc{L}_0\setminus\partial\spc{L}$, then $(f\circ\gamma)''(t)\le 1$ follows from the comparison in $\spc{L}_0$.
If $\gamma(t)\in \spc{L}_1\setminus\partial\spc{L}$, then the proof of inequality reminds the argument in part \ref{SHORT.thm:doubling:concave}, but it is a bit more tricky.
Finally if $\gamma(t)\in\partial\spc{L}$, then splitting argument shows that 
\[(f\circ\gamma)^+(t)+f\circ\gamma)^-(t)\le 0.\]
These three statements imply that $(f\circ\gamma)''(t)\le 1$ for any $t$.
\qeds


