The following example was constructed by Stephanie Halbeisen \cite{halbeisen}.
It is better to keep this example in your head in Section~\ref{sec:grad-def}.

\begin{thm}{Example}\label{Halbeisen's example}
There is a complete length space $\check{\spc{L}}$ with nonnegative curvature in the sense of Alexandrov
with a point $p\in\check{\spc{L}}$ such that the space of directions $\Sigma_p\check{\spc{L}}$ is not a $\pi$-length space, and therefore the tangent space $\T_p\check{\spc{L}}$ is not a length space. 
\end{thm}

If the dimension is finite, such examples do not exist; %?? see \ref{thm:tan4finite}; 
for proper spaces the question is open, see \ref{open:Halb-proper}.


\parit{Construction.}
Let $\HH$ be a Hilbert space formed by infinite sequences of real numbers $\bm{x}=(x_0,x_1,\dots)$ with the $\ell^2$-norm
$|\bm{x}|^2=\sum_i(x_i)^2$. 
Fix $\eps=0.001$ and consider two functions $f,\check f:\HH\to\RR$:
\[f(\bm{x})=|\bm{x}|,\]
\[\check f(\bm{x})
=
\max\left\{|\bm{x}|,\max_{n\ge1}\{(1+\eps)\cdot x_n-\tfrac{1}{n}\}\right\}.\] 
Both of these functions are convex and Lipschitz, therefore their graphs in $\HH\times \RR$ equipped with its length metric form infinite-dimensional spaces with nonnegative curvature, say $\spc{L}$  and $\check{\spc{L}}$ (this can be proved similarly to \ref{prop:conv-surf-CBB(0)}).

Let $p$ be the origin of $\HH\times \RR$.
Note that $\check{\spc{L}}\cap\spc{L}$ is a star-shaped subset of $\HH$ with center at $p$.
Further, $\check{\spc{L}}\setminus\spc{L}$ consists of a countable number of disjoint sets
\[\Omega_n=\set{(\bm{x},\check f(\bm{x}))\in\check{\spc{L}}}{(1+\eps)\cdot x_n-\tfrac{1}{n}>|\bm{x}|}.\]
Note that $\dist{\Omega_n}{p}{}>\tfrac{1}{n}$ for each $n$.
It follows that for any geodesic $[p q]$ in $\check{\spc{L}}$,
a small subinterval $[p \bar q]\subset [p q]$ 
is a straight line segment in $\HH\times\RR$, 
and also a geodesic in $\spc{L}$.
Thus we can treat $\Sigma_p\spc{L}$ and $\Sigma_p\check{\spc{L}}$ as one set, with two angle metrics $\mangle$ and $\check\mangle$.
Let us denote by $\mangle_{\HH\times \RR}$ the angle in $\HH\times\RR$.

The space $\spc{L}$  is isometric to the Euclidean cone
over $\Sigma_p\spc{L}$ with vertex at~$p$; 
$\Sigma_p\spc{L}$ is isometric to a sphere in Hilbert space with radius~$\frac{1}{\sqrt{2}}$.
In particular, $\mangle$ is the length metric of $\mangle_{\HH\times\RR}$ on $\Sigma_p{\spc{L}}$.

Therefore in order to show that $\check \mangle$ does not define a length metric on $\Sigma_p{\spc{L}}$,
it is sufficient to construct a pair of directions $(\xi_+,\xi_-)$ such that
\[\check \mangle(\xi_+,\xi_-)<\mangle(\xi_+,\xi_-).\] 
Set $\bm{e}_0=(1,0,0,\dots)$, $\bm{e}_1=(0,1,0,\dots),\dots\in \HH$. 
Consider the following two half-lines in $\HH\times \RR$:
\[\gamma_+(t)
=
\tfrac{t}{\sqrt{2}}\cdot(\bm{e}_0,1)
\quad  \text{and}\quad 
\gamma_-(t)
=
\tfrac{t}{\sqrt{2}}\cdot(-\bm{e}_0,1),
\quad t\in[0,+\infty).\] 
They form unit-speed geodesics in both $\spc{L}$ and $\check{\spc{L}}$.
Let $\xi_\pm$ be the directions of $\gamma_\pm$ at $p$.
Denote by $\sigma_n$ the half-planes in $\HH$ 
spanned by $\bm{e}_0$ and $\bm{e}_n$;
that is, $\sigma_n\z=\set{x\cdot\bm{e}_0+y\cdot\bm{e}_n}{y\ge 0}$.
Consider a sequence of $2$-dimensional sectors $Q_n=\check{\spc{L}}\cap (\sigma_n\times \RR)$. 
For each $n$, the sector $Q_n$ intersects $\Omega_n$ and is bounded by two geodesic half-lines $\gamma_\pm$.
Note that $Q_n\GHto Q$, where  $Q$ is a solid Euclidean angle
in $\EE^2$ with angle measure $\beta<\mangle(\xi_+,\xi_-)=\tfrac\pi{\sqrt{2}}$.
Indeed, $Q_n$ is path-isometric to the subset of $\EE^3$ described by
\begin{align*}
 y\ge0 \quad 
\text{and}\quad  
&
z=\max\left\{\sqrt{x^2+y^2},
(1+\eps)\cdot y-\tfrac{1}{n} \right\}
\intertext{with length metric.
Thus its limit $Q$ is path-isometric to the subset of $\EE^3$ described by}
y\ge0
\quad \text{and}\quad  
&
z=\max\left\{\sqrt{x^2+y^2},(1+\eps)\cdot y\right\}
\end{align*}
with length metric.
In particular, for any $t,\tau\ge0$, 
\begin{align*}
\dist{\gamma_+(t)}{\gamma_-(\tau)}{\check{\spc{L}}} 
&\le 
\lim_{n\to\infty}\dist{\gamma_+(t)}{\gamma_-(\tau)}{Q_n}
=
\\ 
&=\side \{\beta;t,\tau\},
\end{align*}
where $\side \{\beta;t,\tau\}\df\sqrt{t^2+\tau^2-2\cdot t\cdot \tau\cdot\cos \beta}$.
That is, $\check\mangle(\xi_+,\xi_-) \le \beta<\mangle(\xi_+,\xi_-)$.\qeds

{\sloppy 

\begin{thm}{Exercise}\label{ex:norays}
Construct a non-compact Alexandrov space that contains no half-lines.
\end{thm}

}
