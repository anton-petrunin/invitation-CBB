

\section{Existence}\label{sec:Alexandrov-existence}

\begin{thm}{Theorem}\label{thm:exist}
A polyhedral metric on the sphere is isometric to the surface of a convex polyhedron (possibly degenerate to a flat polygon) if and only if it has nonnegative curvature at each point.
\end{thm}

\begin{wrapfigure}{r}{30mm}
\vskip-5mm
\centering
\includegraphics{mppics/pic-1010}
\vskip-0mm
\end{wrapfigure}

By \ref{thm:alexandrov-uni'}, a convex polyhedron is completely defined by the intrinsic metric of its surface.
By \ref{thm:exist}, it follows that knowing the metric we could find the position of the edges.
However, in practice, it is not easy to do.

For example, the surface glued from a rectangle as shown on the diagram defines a tetrahedron.
Some of the glued lines appear inside facets of the tetrahedron and some edges (dashed lines) do not follow the sides of the rectangle.

\paragraph{Space of polyhedrons.}
Let us denote by $\mathbf{K}$ the space of all convex polyhedrons in the Euclidean space,
including polyhedrons that degenerate to a plane polygon.
Polyhedra in $\mathbf{K}$ will be considered up to a motion of the space,
and the whole space $\mathbf{K}$ will be considered with Hausdorff distance up to a motion of the space;
that is, the distance between $K$ and $K'$ is the exact lower bound on Hausdorff distance from $\iota(K)$ to $K'$, where $\iota$ is arbitrary motion of $\EE^3$.

Further, denote by $\mathbf{K}_n$ the polyhedrons in $\mathbf{K}$ with exactly $n$ vertices.
Since any polyhedron has at least 3 vertices, the space $\mathbf{K}$ admits a subdivision into a countable number of subsets $\mathbf{K}_3,\mathbf{K}_4,\dots$

\paragraph{Space of polyhedral metrics.}
The space of polyhedral metrics on the sphere with nonnegative curvature will be denoted by $\mathbf{P}$.
The metrics in $\mathbf{P}$ will be considered up to an isometry, and the whole space $\mathbf{P}$ will be equipped with the topology induced by the Gromov--Hausdorff metric.

The subset of $\mathbf{P}$ of all metrics with exactly $n$ essential vertices will be denoted by $\mathbf{P}_n$.
It is easy to see that any metric in $\mathbf{P}$ has at least 3 essential vertices.
Therefore $\mathbf{P}$ is subdivided into countably many subsets
 $\mathbf{P}_3,\mathbf{P}_4,\dots$

\paragraph{From a polyhedron to its surface.}

By \ref{prop:poly-CBB}, passing from a polyhedron to its surface defines a map
\[\iota\:\mathbf{K}\to \mathbf{P}.\]

By \ref{ex:vertex-essential-vertex}, the number of vertices of a polyhedron is equal to the number of essential vertices on its surface.
In other words, $\iota(\mathbf{K}_n)\subset \mathbf{P}_n$ for any $n\ge 3$.

Using the introduced notation, we can unite \ref{thm:alexandrov-uni'} and \ref{thm:exist} in the following more exact statement.

\begin{thm}{Reformulation}
For any integer $n\ge 3$,
the map $\iota$ induces a bijection between $\mathbf{K}_n$ and~$\mathbf{P}_n$.
\end{thm}

The proof is based on a construction of a one-parameter family of polyhedrons that starts at an arbitrary polyhedron
and ends at a polyhedron with its surface isometric to the given one.
This type of argument is called the \textit{continuity method}; it is often used in the theory of differential equations.


\parit{Sketch.}
By \ref{thm:alexandrov-uni'}, the map $\iota\:\mathbf{K}_n\to\mathbf{P}_n$ is injective.
Let us prove that it is surjective.

\begin{thm}{Lemma}
For any integer $n\ge 3$, the space $\mathbf{P}_n$ is connected.
\end{thm}

The proof of this lemma is not complicated, but it requires ingenuity;
it can be done by the direct construction of a one-parameter family of metrics in $\mathbf{P}_n$ that connects two given metrics.
Such a family can be obtained by а sequential application of the following construction and its inverse.

Let $P\in\mathbf{P}_n$.
Suppose $v$ and $w$ are essential vertices in $P$.
Let us cut $P$ along a geodesic from $v$ to $w$.
Note that the geodesic cannot pass thru an essential vertex of $P$.
Further, note that there is a three-parameter family of patches that can be used to patch the cut so that the obtained metric remains in $\mathbf{P}_n$;
in particular, the obtained metric has exactly $n$ essential vertices (after the patching, the vertices $v$ and $w$ may become inessential).


\begin{thm}{Lemma}
The map $\iota\:\mathbf{K}_n\to\mathbf{P}_n$ is open,
that is, it maps any open set in $\mathbf{K}_n$ to an open set in $\mathbf{P}_n$.

In particular, for any $n\ge 3$, the image $\iota(\mathbf{K}_n)$ is open in~$\mathbf{P}_n$.
\end{thm}

This statement is very close to the so-called \textit{invariance of domain theorem};
the latter states that a continuous injective map between manifolds of the same dimension is open.

Recall that $\iota$ is injective.
The proof of the invariance of domain theorem can be adapted to our case since both spaces $\mathbf{K}_n$ and $\mathbf{P}_n$ are $(3\cdot n-6)$-dimensional and both look like manifolds, altho, formally speaking, they are \textit{not} manifolds.
In a more technical language, $\mathbf{K}_n$ and $\mathbf{P}_n$ have the natural structure of $(3\cdot n-6)$-dimensional \textit{orbifolds},
and the map $\iota$ respects the \textit{orbifold structure}.

We will only show that both spaces $\mathbf{K}_n$ and $\mathbf{P}_n$ are $(3\cdot n-6)$-dimensional.

Choose  $K\in\mathbf{K}_n$.
Note that $K$ is uniquely determined by the $3\cdot n$ coordinates of its $n$ vertices.
We can assume that the first vertex is the origin, the second has two vanishing coordinates and the third has one vanishing coordinate; therefore, all polyhedrons in $\mathbf{K}_n$ that lie sufficiently close to $K$ can be described by $3\cdot n-6$ parameters.
If $K$ has no symmetries, then this description can be made one-to-one;
in this case, a neighborhood of $K$ in $\mathbf{K}_n$ is a $(3\cdot n-6)$-dimensional manifold.
If $K$ has a nontrivial symmetry group, then this description is not one-to-one but it does not have an impact on the dimension of~$\mathbf{K}_n$.

The case of polyhedral metrics is analogous.
We need to construct a subdivision of the sphere into plane triangles using only essential vertices.
By Euler's formula, there are exactly $3\cdot n-6$ edges in this subdivision.
Note that the lengths of edges completely describe the metric, and slight changes in these lengths produce a metric with the same property.
Again, if $P$ has no symmetries, then this description is one-to-one.

\begin{thm}{Lemma}
The map $\iota\:\mathbf{K}_n\to\mathbf{P}_n$ is closed;
that is, the image of a closed set in $\mathbf{K}_n$ is closed in $\mathbf{P}_n$.

In particular, for any $n\ge 3$, the set $\iota(\mathbf{K}_n)$ is closed in~$\mathbf{P}_n$.
\end{thm}

Choose a closed set $Z$ in $\mathbf{K}_n$.
Denote by $\bar Z$ the closure of $Z$ in $\mathbf{K}$; note that $Z=\mathbf{K}_n\cap \bar Z$.
Assume $K_1,K_2,\dots\in Z$ is a sequence of polyhedrons that converges to a polyhedron $K_\infty\in\bar Z$.
By \ref{lem:H>GH}, $\iota(K_n)$ converges to $\iota(K_\infty)$ in $\mathbf{P}$.
In particular, $\iota(\bar Z)$ is closed in $\mathbf{P}$.

Since $\iota(\mathbf{K}_n)\subset \mathbf{P}_n$ for any $n\ge 3$, we have $\iota (Z)=\iota(\bar Z)\cap \mathbf{P}_n$;
that is, $\iota (Z)$ is closed in $\mathbf{P}_n$.

\medskip

Summarizing, $\iota(\mathbf{K}_n)$ is a nonempty closed and open set in $\mathbf{P}_n$, and $\mathbf{P}_n$ is connected for any $n\ge 3$.
Therefore, $\iota(\mathbf{K}_n)=\mathbf{P}_n$; that is, $\iota\:\mathbf{K}_n\z\to\mathbf{P}_n$ is surjective.
\qeds

\section{Approximation}

By now, the embedding theorem is proved for polyhedral metrics on the sphere.
The general case is done by approximation, using the following statement.

\begin{thm}{Proposition}\label{prop:H>GH}
Let $K_1,K_2,\dots$ be a sequence of convex bodies that converge to $K_\infty$ in the sense of Hausdorff.
Then the surface of $K_n$ converges to the surface of $K_\infty$ in the sense of Gromov--Hausdorff.
\end{thm}

If $K_\infty$ is nondegenerate, then the statement follows from \ref{lem:H>GH}.
The degenerate case is left as an exercise.

Let $\spc{X}_\infty$ be an $\Alex0$ space that is homeomorphic to $\SSS^2$.
Suppose that $\spc{X}_\infty$ is a Gromov--Hausdorff limit of a sequence of spheres with polyhedral metrics $\spc{X}_1,\spc{X}_2,\dots$
By \ref{thm:exist}, there is a sequence of convex polyhedra $K_1,K_2,\dots$ with surfaces isometric to $\spc{X}_1,\spc{X}_2,\dots$, respectively.
Note that  $\diam K_n\le \diam \spc{X}_n$ for any $n$.
Therefore we can assume that all polyhedra $K_1,K_2,\dots$ lie in a closed ball of sufficiently large radius.

Applying Blaschke selection theorem, we can pass to a subsequence of $K_1,K_2,\dots$ that converges in the sense of Hausdorff; denote its limit by $K_\infty$.
By \ref{prop:H>GH} the surface of $K_\infty$ is isometric to $\spc{X}_\infty$.

Therefore it remains to prove the following lemma.

\begin{thm}{Lemma}\label{lem:GH-approximation}
Let $\spc{X}$ be an $\Alex0$ space that is homeomorphic to $\SSS^2$.
Then there is a sphere with polyhedral metrics $\spc{X}'$
that is arbitrarily close to $\spc{X}$ in the sense of Gromov--Hausdorff.
\end{thm}

\parit{Proof with two cheatings.}
Suppose we can triangulate $\spc{X}_\infty$ by small geodesic triangles;
that is, we can choose a finite set of points $p_1,\dots,p_n\z\in \spc{X}_\infty$ and some geodesics $[p_ip_j]$ that cut $\spc{X}_\infty$ into regions of small diameter bounded by geodesic triangles $[p_ip_jp_k]$.
(This is the first chating, the actual proof uses a triangulation with a weaker property.)

Observe that total angle around each $p_i$ cannot exceed $2\cdot \pi$.
That is, suppose $p_{j_1},\dots,p_{j_k}$ are points connected to $p_i$ by geodesics.
Assume that they are ordered in the natural cyclic order.
Then
\[\mangle\hinge{p_i}{p_{j_1}}{p_{j_2}}+\dots+\mangle\hinge{p_i}{p_{j_{k-1}}}{p_{j_k}}+\mangle\hinge{p_i}{p_{j_{k}}}{p_{j_1}}\le 2\cdot\pi.\]
By comparison, we get
\[\angk{p_i}{p_{j_1}}{p_{j_2}}+\dots+
\angk{p_i}{p_{j_{k-1}}}{p_{j_k}}+\angk{p_i}{p_{j_{k}}}{p_{j_1}}\le 2\cdot\pi.\eqlbl{eq:sum<=<2pi}\]

Now let us exchange each triangle by its model triangle.
That is, consider a model triangle for each region in the subdivision of $\spc{X}$ and glue them together by the same rule.
By \ref{eq:sum<=<2pi}, the obtained polyhedral surface $\spc{X}'$ has nonnegative curvature.
It remains to show that this way we can produce $\spc{X}'$ arbitrarily close to $\spc{X}$.

Denote by $p_i\to p_i'$ the natural map; it takes $p_i$ in $\spc{X}$ and returns the corresponding point in $\spc{X}'$.
Observe that
\[\dist{p_i'}{p_j'}{\spc{X}'}
\le
\dist{p_i}{p_j}{\spc{X}}.\eqlbl{eq:|pp|}\]
Indeed, choose a geodesic $\gamma$ from $p_i$ to $p_j$.
Let $p_i=x_0,x_1,\dots,x_n=p_j$ be the points of intersections of $\gamma$ with the edges of the triangulation listed as they appear on $\gamma$.
For each $i$, denote by $x_i'$ the corresponding point in $\spc{X}'$.
By comparison, we get
\[\dist{x_k'}{x_{k-1}'}{\spc{X}'}
\le
\dist{x_k}{x_{k-1}}{\spc{X}}.\]
for each $k$.
Therefore, \ref{eq:|pp|} follows.

Suppose $\eps>0$ is small, the points $p_1,\dots,p_n$ form an $\eps$-net in $\spc{X}$, all edges of the triangulation are smaller than $\eps$ and
\[\dist{p_i'}{p_j'}{\spc{X}'}
\ge
\dist{p_i}{p_j}{\spc{X}} -100\cdot \eps.\eqlbl{eq:|pp|>=}\]
Then, together with the inequality above it proves the lemma.

Now let us assume that the sides of model triangles in $\spc{X}'$ are geodesics.
(This is the second cheating; the sides of the model triangles are local geodesics in $\spc{X}'$,
but not necessarily geodesic; that is, they do not have to be length-minimizing.
The actual proof does not use this assumption.)

Choose a geodesic $\gamma'$ from $p_i'$ to $p_j'$ in $\spc{X}'$.
Note that $\gamma'$ visits each triangle in the triangulation of $\spc{X}'$ at most once.

Let $p_i'=x_0',x_1',\dots,x_n'\z=p_j'$ be the points of intersections of $\gamma'$ with the edges of the triangulation listed from $p_i'$ to $p_j'$.
For each $i$, denote by $x_i$ the corresponding point in $\spc{X}$.
Let $\Delta_k'$ be the triangle that contains arc $[x'_{k-1}x'_k]$ of $\gamma'$ and $\Delta_k$ the corresponding triangle in~$\spc{X}$.
Note that
\[\dist{x_k'}{x_{k-1}'}{\spc{X}'}
\ge
\dist{x_k}{x_{k-1}}{\spc{X}} -\eps\cdot K(\Delta_k),
\eqlbl{eq:|xx|<}\]
where $K(\Delta_k)$ denotes the access of $\Delta_k$;
that is, the sum of its internal angles minus $\pi$.

Euler's formula and \ref{eq:sum<=<2pi} imply that the sum of all accesses is at most $4\cdot\pi$.
Therefore, summing up \ref{eq:|xx|<}, we get
\[\dist{p_i'}{p_j'}{\spc{X}'}
\ge
\dist{p_i}{p_j}{\spc{X}}-4\cdot \pi\cdot \eps.\]
Whence \ref{eq:|pp|>=} follows.
\qeds

\section{Comments}

\parit{Existence theorem.}
This theorem was proved by Alexandr Alexandrov~\cite{alexandrov-1941}.
Our sketch is taken from \cite{lebedeva-petrunin};
a complete proof is nicely written in~\cite{alexandrov}.
In the original proof, the spaces $\mathbf{K}_n$ and $\mathbf{P}_n$ were modified so the they become $(3\cdot n-6)$-dimensional manifolds.
It was done by introducing extra structure (for $\mathbf{K}_n$ it is orientation + a marked vertex and an edge) that \textit{brakes symmetries} of the spaces.
After that one could apply the domain invariance theorem directly.
Alternatively, one may first remove from $\mathbf{K}_n$ and $\mathbf{P}_n$ elements (polyhedron or surface)with nontrivial symmetries (after that the spaces become manifolds) and show that any symmetric polyhedron (or surface) can be approximated by a non-symmetric polyhedron (or surface).

A very different proof was found by Yuri Volkov in his thesis \cite{volkov};
it uses a deformation of three-dimensional polyhedral space.

%P and \Sigma???
